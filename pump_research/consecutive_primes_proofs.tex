%\documentclass[letterpaper]{article}
\documentclass[a5paper]{article}

%% Language and font encodings
\usepackage[english]{babel}
\usepackage[utf8x]{inputenc}
\usepackage[T1]{fontenc}

\newcommand{\redtext}[1]{{\textcolor{red}{#1}}}
\newcommand{\greentext}[1]{{\textcolor{dgreen}{#1}}}
\newcommand{\bluetext}[1]{{\textcolor{dblue}{#1}}}
\newcommand{\blacktext}[1]{{\textcolor{black}{#1}}}
\newcommand{\remph}[1]{\textit{#1}}
\newcommand{\blemph}[1]{\textit{#1}}

%% Sets page size and margins
%\usepackage[letterpaper,top=1in,bottom=1in,left=1in,right=1in,marginparwidth=1.75cm]{geometry}
\usepackage[a5paper,top=1cm,bottom=1cm,left=1cm,right=1.5cm,marginparwidth=1.75cm]{geometry}

\usepackage{graphicx}
%\graphicspath{ {images/} }	  %%uncomment this if you want to put all your images for this document into their own folder

%% Useful packages
\usepackage{amssymb, amsmath, amsthm} 
%\usepackage{graphicx}  %%this is currently enabled in the default document, so it is commented out here. 
\usepackage{calrsfs}
\usepackage{braket}
\usepackage{mathtools}
\usepackage{lipsum}
\usepackage{tikz}
\usetikzlibrary{cd}
\usepackage{verbatim}
%\usepackage{ntheorem}% for theorem-like environments
\usepackage{mdframed}%can make highlighted boxes of text
%Use case: https://tex.stackexchange.com/questions/46828/how-to-highlight-important-parts-with-a-gray-background
\usepackage{wrapfig}
\usepackage{centernot}
\usepackage{subcaption}%\begin{subfigure}{0.5\textwidth}
\usepackage{pgfplots}
\pgfplotsset{compat=1.13}
\usepackage[colorinlistoftodos]{todonotes}
\usepackage[colorlinks=true, allcolors=blue]{hyperref}
\usepackage{xfrac}					%to make slanted fractions \sfrac{numerator}{denominator}
\usepackage{enumitem}            
    %syntax: \begin{enumerate}[label=(\alph*)]
    %possible arguments: f \alph*, \Alph*, \arabic*, \roman* and \Roman*
\usetikzlibrary{arrows,shapes.geometric,fit}

\DeclareMathAlphabet{\pazocal}{OMS}{zplm}{m}{n}
%% Use \pazocal{letter} to typeset a letter in the other kind 
%%  of math calligraphic font. 

%% This puts the QED block at the end of each proof, the way I like it. 
\renewenvironment{proof}{{\bfseries Proof}}{\qed}
\makeatletter
\renewenvironment{proof}[1][\bfseries \proofname]{\par
  \pushQED{\qed}%
  \normalfont \topsep6\p@\@plus6\p@\relax
  \trivlist
  %\itemindent\normalparindent
  \item[\hskip\labelsep
        \scshape
    #1\@addpunct{}]\ignorespaces
}{%
  \popQED\endtrivlist\@endpefalse
}
\makeatother

%% This adds a \rewnewtheorem command, which enables me to override the settings for theorems contained in this document.
\makeatletter
\def\renewtheorem#1{%
  \expandafter\let\csname#1\endcsname\relax
  \expandafter\let\csname c@#1\endcsname\relax
  \gdef\renewtheorem@envname{#1}
  \renewtheorem@secpar
}
\def\renewtheorem@secpar{\@ifnextchar[{\renewtheorem@numberedlike}{\renewtheorem@nonumberedlike}}
\def\renewtheorem@numberedlike[#1]#2{\newtheorem{\renewtheorem@envname}[#1]{#2}}
\def\renewtheorem@nonumberedlike#1{  
\def\renewtheorem@caption{#1}
\edef\renewtheorem@nowithin{\noexpand\newtheorem{\renewtheorem@envname}{\renewtheorem@caption}}
\renewtheorem@thirdpar
}
\def\renewtheorem@thirdpar{\@ifnextchar[{\renewtheorem@within}{\renewtheorem@nowithin}}
\def\renewtheorem@within[#1]{\renewtheorem@nowithin[#1]}
\makeatother

%% This makes theorems and definitions with names show up in bold, the way I like it. 
\makeatletter
\def\th@plain{%
  \thm@notefont{}% same as heading font
  \itshape % body font
}
\def\th@definition{%
  \thm@notefont{}% same as heading font
  \normalfont % body font
}
\makeatother

%===============================================
%==============Shortcut Commands================
%===============================================
\newcommand{\ds}{\displaystyle}
\newcommand{\B}{\mathcal{B}}
\newcommand{\C}{\mathbb{C}}
\newcommand{\F}{\mathbb{F}}
\newcommand{\N}{\mathbb{N}}
\newcommand{\R}{\mathbb{R}}
\newcommand{\Q}{\mathbb{Q}}
\newcommand{\T}{\mathcal{T}}
\newcommand{\Z}{\mathbb{Z}}
\renewcommand\qedsymbol{$\blacksquare$}
\newcommand{\qedwhite}{\hfill\ensuremath{\square}}
\newcommand*\conj[1]{\overline{#1}}
\newcommand*\closure[1]{\overline{#1}}
\newcommand*\mean[1]{\overline{#1}}
%\newcommand{\inner}[1]{\left< #1 \right>}
\newcommand{\inner}[2]{\left< #1, #2 \right>}
\newcommand{\powerset}[1]{\pazocal{P}(#1)}
%% Use \pazocal{letter} to typeset a letter in the other kind 
%%  of math calligraphic font. 
\newcommand{\cardinality}[1]{\left| #1 \right|}
\newcommand{\domain}[1]{\mathcal{D}(#1)}
\newcommand{\image}{\text{Im}}
\newcommand{\inv}[1]{#1^{-1}}
\newcommand{\preimage}[2]{#1^{-1}\left(#2\right)}
\newcommand{\script}[1]{\mathcal{#1}}


\newenvironment{highlight}{\begin{mdframed}[backgroundcolor=gray!20]}{\end{mdframed}}

\DeclarePairedDelimiter\ceil{\lceil}{\rceil}
\DeclarePairedDelimiter\floor{\lfloor}{\rfloor}

%===============================================
%===============My Tikz Commands================
%===============================================
\newcommand{\drawsquiggle}[1]{\draw[shift={(#1,0)}] (.005,.05) -- (-.005,.02) -- (.005,-.02) -- (-.005,-.05);}
\newcommand{\drawpoint}[2]{\draw[*-*] (#1,0.01) node[below, shift={(0,-.2)}] {#2};}
\newcommand{\drawopoint}[2]{\draw[o-o] (#1,0.01) node[below, shift={(0,-.2)}] {#2};}
\newcommand{\drawlpoint}[2]{\draw (#1,0.02) -- (#1,-0.02) node[below] {#2};}
\newcommand{\drawlbrack}[2]{\draw (#1+.01,0.02) --(#1,0.02) -- (#1,-0.02) -- (#1+.01,-0.02) node[below, shift={(-.01,0)}] {#2};}
\newcommand{\drawrbrack}[2]{\draw (#1-.01,0.02) --(#1,0.02) -- (#1,-0.02) -- (#1-.01,-0.02) node[below, shift={(+.01,0)}] {#2};}

%***********************************************
%**************Start of Document****************
%***********************************************

%===============================================
%===============Theorem Styles==================
%===============================================

%================Default Style==================
\theoremstyle{plain}% is the default. it sets the text in italic and adds extra space above and below the \newtheorems listed below it in the input. it is recommended for theorems, corollaries, lemmas, propositions, conjectures, criteria, and (possibly; depends on the subject area) algorithms.
\newtheorem{theorem}{Theorem}
\numberwithin{theorem}{section} %This sets the numbering system for theorems to number them down to the {argument} level. I have it set to number down to the {section} level right now.
\newtheorem*{theorem*}{Theorem} %Theorem with no numbering
\newtheorem{corollary}[theorem]{Corollary}
\newtheorem*{corollary*}{Corollary}
\newtheorem{conjecture}[theorem]{Conjecture}
\newtheorem{lemma}[theorem]{Lemma}
\newtheorem*{lemma*}{Lemma}
\newtheorem{proposition}[theorem]{Proposition}
\newtheorem*{proposition*}{Proposition}
\newtheorem{problemstatement}[theorem]{Problem Statement}


%==============Definition Style=================
\theoremstyle{definition}% adds extra space above and below, but sets the text in roman. it is recommended for definitions, conditions, problems, and examples; i've alse seen it used for exercises.
\newtheorem{definition}[theorem]{Definition}
\newtheorem*{definition*}{Definition}
\newtheorem{condition}[theorem]{Condition}
\newtheorem{problem}[theorem]{Problem}
\newtheorem{example}[theorem]{Example}
\newtheorem*{example*}{Example}
\newtheorem*{counterexample*}{Counterexample}
\newtheorem*{romantheorem*}{Theorem} %Theorem with no numbering
\newtheorem{exercise}{Exercise}
\numberwithin{exercise}{section}
\newtheorem{algorithm}[theorem]{Algorithm}

%================Remark Style===================
\theoremstyle{remark}% is set in roman, with no additional space above or below. it is recommended for remarks, notes, notation, claims, summaries, acknowledgments, cases, and conclusions.
\newtheorem{remark}[theorem]{Remark}
\newtheorem*{remark*}{Remark}
\newtheorem{notation}[theorem]{Notation}
\newtheorem*{notation*}{Notation}
%\newtheorem{claim}[theorem]{Claim}  %%use this if you ever want claims to be numbered
\newtheorem*{claim}{Claim}



%\renewcommand\qedsymbol{$\blacksquare$}
%\newcommand{\qedwhite}{\hfill\ensuremath{\square}}
 
\title{Some Problems About Consecutive Products of Primes}
\author{Trevor Klar \\ Eli Moore}
 
\begin{document}
\maketitle
\section{Introduction}
Suppose $p$ and $q$ are both prime numbers with $p < q$. Consider all integers of the form $p^\alpha q^\beta$ with $\alpha, \beta \in \N$ and let $\{a_n\}$ be the sequence of these integers in increasing order.
 
\begin{definition*}
If two integers $p^m, q^n$ are elements of $\{a_k\}$ such that $p^m=a_i$ and $q^n=a_{i+1}$, we say that $(p^m, q^n)$ is a \emph{critical pair}. Note that this notation means that $p^m,<q^n$. It is also possible that $(q^n, p^m)$ is a critical pair, so that $q^n < p^m$.
\end{definition*}
 
\begin{lemma}
    If $a_k = q^n$, then $a_{k+1} \neq q^{n+1}$.
\begin{proof} \textit{(By Contradiction)} 
    Assume that $a_k = q^n$, and suppose for contradiction that $a_{k+1} = q^{n+1}$. Since $1 < p < q$, then $q^n < pq^n < q^{n+1}$. However, this contradicts our assumption that $a_k = q^n$ and $a_{k+1} = q^{n+1}$, as $pq^n$ must be a term of the sequence which falls between $a_k$ and $a_{k+1}$.
\end{proof}
\end{lemma}

\begin{lemma}\label{finite_repetition_lemma}
    There exist at most finitely many $a_k = p^n$ such that $a_{k+1} = p^{n+1}$.
\begin{proof} 
    Since $p<q$, let $n$ be the largest $n \in \N$ such that $p^n < q$. Then it follows that $p^n < q < p^{n+1}$. This means that $\{a_n\}$ begins as $$\{a_n\} = \{1,p, p^2, ..., p^n, q, p^{n+1},...\}.$$
    %
    Claim: $\forall m \in \N, \quad p^{n+m} < p^mq < p^{n+m+1}$. Let $T(m)$ denote this statement. We now prove this claim by induction on $m$. We already know that $p^n < q < p^{n+1}$, so $p^{n+1} < pq < p^{n+2}$. Thus, $T(1)$ holds. We now assume $T(m)$ and show $T(m+1)$ holds:
    $$p^{n+m} < p^mq < p^{n+m+1} \implies p^{n+m+1} < p^{m+1}q < p^{n+m+2}$$
    %
    As such, every integer $p^{n+m} > p^n$ is followed by the term $p^mq$ before $p^{n+m+1}$ in the sequence $a_n$. Thus, $a_k = p^n$ and $a_{k+1} = p^{n+1}$ can only occur at the beginning of the sequence (finitely many times) as shown above.
\end{proof}
\end{lemma}

\begin{lemma}
    If $a_i = p^m$ and $a_{i+1} = q^n$, then $m$ and $n$ are relatively prime.
\begin{proof}\textit{(By Contradiction)} 
    Assume $a_i = p^m$ and $a_{i+1} = q^n$ and suppose for contradiction that gcd$(m,n) = d$. Then $m = m'd$ and $n = n'd$ for some $m', n' \in \N$. Since $p^m = p^{m'd} < q^{n'd} = q^n$, we have $p^{m'} < q^{n'}$. Consider the following inequality:
    \[
    \begin{array}{rcl}
        p^m & = & p^{m'd}\\
        & = & p^{m'd - m' + m'}\\
        & = & p^{m'(d-1)}p^{m'} \\
        & < & p^{m'(d-1)}q^{n'} \\
        & < & q^{n'(d-1)}q^{n'} \\
        & = & q^{n'd} \\
        & = & q^n.
    \end{array}
    \]
    Since $p^{m'(d-1)}q^{n'}$ must come between $p^m$ and $q^n$, $p^m$ and $q^n$ cannot be consecutive terms in ${a_k}$. Thus we have reached a contradiction. Notice, a similar argument holds for when $a_i = q^n$ and $a_{i+1} = p^m$.
    
\end{proof}
\end{lemma}

\begin{lemma}For any two consecutive $p^m$, $q^n \in \{a_k\}$, 
%$$ \frac{m}{n} \approx \frac{\ln(q)}{\ln(p)}.$$ And furthermore,  
$$\lim_{k \to \infty} \frac{m}{n} = \frac{\ln(q)}{\ln(p)}.$$
\begin{proof}
Since $p<q$, we know already that 
\[
    \begin{array}{rcl}
       m\ln{p}-n\ln{q} & < & \min(\ln{p},\ln{q})\\
                     & = & \ln{p},
    \end{array}
\]

So we can divide by $n\ln{p}$ to find that 

\[
    \begin{array}{rcl}
       \dfrac{m}{n}-\dfrac{\ln{q}}{\ln{p}} & < & \dfrac{\ln{p}}{n\ln{p}}\\
       & = & \sfrac{1}{n}
    \end{array}
\]
\end{proof}
\end{lemma}

\pagebreak

\section{The (flawed) Proof}

\begin{definition*}Let $p,q$ be distinct primes, and let $a,b \in \Z^+$. 

A \emph{pure power} of $p$ is an integer of the form $p^a$. 

This is as opposed to a \emph{mixed power} of $p$ and $q$, which is an integer of the form $p^a q^b$.
\end{definition*}

\begin{definition*}Let $p,q$ be distinct primes, and let $a,b,\alpha,\beta \in \Z^+$. 

We say that $p^\alpha q^\beta$ is an \emph{intermediate mixed power} of $p^a$ and $q^b$ if $p^\alpha q^\beta$ is between $p^a$ and $q^b$. 
\end{definition*}

\begin{definition*}Let $p,q$ be distinct primes, and let $a,b \in \Z^+$. 

A \emph{critical pair} of $p$ and $q$ is a pair of pure powers of $p$ and $q$ which do not have an intermediate mixed power. 
\end{definition*}

\begin{lemma}\label{thelemma}
If $p$, $q$ are distinct prime integers, then there exists at least one critical pair of $p$ and $q$. 

\begin{proof}
Without loss of generality, suppose that $p<q$. Then, let $n$ be the largest $n\in\N$ such that $p^n<q$. Now, $p^n<q^1$ is a critical pair and we are done.
\end{proof}
\end{lemma}

\begin{algorithm}\label{phase1} Let $p,q$ be distinct primes, and let $a,b \in \Z^+$. Suppose $q^a \approx p^b$ and, without loss of generality, suppose that $q^a>p^b$. That is, 
$$1 < \frac{q^a}{p^b} < 1+\epsilon, \quad \text{where } \epsilon <<.$$
If an intermediate mixed power exists, it is of the form 
\begin{equation}\label{thechink}
p^b < q^{a-k}p^{b+\ell} < q^a
\end{equation}
where $k,\ell \in \Z^+$. So, since $q^{a-k}p^{b+\ell} > p^b$, 
\[
\begin{array}{rcl}
1+\epsilon &>& \frac{q^a}{p^b}\\
&>&\frac{q^a}{q^{a-k}p^{b+\ell}}\\
&=&\frac{q^k}{p^{b+\ell}}.\\
\end{array}
\]
Now, let $k=\tilde{a}$, and let $b+\ell=\tilde{b}$. If an intermediate mixed power exists, apply Algorithm \ref{phase1} until one no longer exists. Note, since $a>k\geq 1$ and $b<b+\ell$, this process cannot continue indefinitely. 

Thus, we can always apply this algorithm to find a critical pair between any two pure powers of $p$ and $q$.
\end{algorithm}

\noindent\textbf{Claim: }There exist infinitely many critical pairs of any two distinct primes $p$ and $q$.

\begin{proof}\textbf{by Induction}. Let $p,q$ be distinct primes. Let $P(n)$ be the statement "There exist $n$ distinct critical pairs of $p$ and $q$." We will prove that there are infinitely many critical pairs of $p$ and $q$ by induction on $n$. \\
\\
By Lemma \ref{thelemma}, there must exist at least one critical pair $p^{b_0}<q^{a_0}$. Thus, $P(1)$ holds. \\
\\
Now, assume that $P(n)$ holds. Let $p^b=p^{b_n}$, and choose some $q^a$ such that 
$$1<\frac{q^a}{p^b}<1+\epsilon.$$
Apply Algorithm \ref{phase1} to obtain $p^{b_{n+1}},q^{a_{n+1}}$ such that $p^{b_n}<p^{b_{n+1}}$, and $p^{b_{n+1}}<q^{a_{n+1}}$ are a critical pair. Thus, we have a critical pair such that $p^{b_{n+1}} > p^{b_{n}} > \ldots > p^{b_{1}}$, so we have $n+1$ distinct critical pairs. Therefore, P(n+1) holds. 
\end{proof}

\noindent \textbf{Issue: }There is a critical problem with this proof. The statement given in Equation \ref{thechink} is false. It is actually true that if an intermediate power exists, it is of the form 
$$p^b < q^{a-k}p^{0+\ell} < q^a$$
$$\text{or}$$
$$p^b < q^{0+k}p^{b-\ell} < q^a.$$

\pagebreak
\section{Working proof}

\begin{definition*}Let $p,q$ be distinct primes, and let $a,b \in \Z^+$. 

A \emph{pure power} of $p$ is an integer of the form $p^a$. 

This is as opposed to a \emph{mixed power} of $p$ and $q$, which is an integer of the form $p^a q^b$.
\end{definition*}

\begin{definition*}Let $p,q$ be distinct primes, and let $a,b,\alpha,\beta \in \Z^+$. 

We say that $p^\alpha q^\beta$ is an \emph{intermediate mixed power} of $p^a$ and $q^b$ if $p^\alpha q^\beta$ is between $p^a$ and $q^b$. (That is, either $p^a < p^\alpha q^\beta < q^b$ or $q^b < p^\alpha q^\beta < p^a$)
\end{definition*}

\begin{definition*}Let $p,q$ be distinct primes, and let $a,b \in \Z^+$. 

A \emph{critical pair} of $p$ and $q$ is a pair of pure powers of $p$ and $q$ which do not have an intermediate mixed power. 
\end{definition*}

\begin{theorem}
Consider the pure powers $p^a, q^b$ with $p^a < q^b$ and $a,b \in \Z^+$. If, for all critical pairs $p^s,q^t$ with $s<a$ and $t<b$, 
$$1<\frac{q^b}{p^a}<\frac{q^t}{p^s}, \quad s,t \in \Z^+$$
then $p^a, q^b$ is a critical pair. 
\end{theorem}

\begin{proof} \textbf{by contradiction}
Assume that for all critical pairs $p^s,q^t$ with $s<a$ and $t<b$, 
$$1<\frac{q^b}{p^a}<\frac{q^t}{p^s},$$ and suppose for contradiction that $p^a, q^b$ is not a critical pair. Since $p^a, q^b$ is not a critical pair, then there exists an intermediate mixed power of the form
$$p^a < q^{b-\ell}p^{a-k} < q^b$$
where $1\leq k < a, 1\leq \ell < b$. So, since $q^{b-\ell}p^{a-k} > p^a$, 
\[\frac{q^b}{p^a}>\frac{q^b}{q^{b-\ell}p^{a-k}}=\frac{q^\ell}{p^{a-k}}.\]
Now, let $a - k=\tilde{a}$, and let $\ell=\tilde{b}$. If $p^{\tilde{a}}$ and $q^{\tilde{b}}$ are a critical pair, we have a contradiction. If they are not, then we can repeat the preceding process in this proof. Note, since $a>\tilde{a}\geq 1$ and $b>\tilde{b}\geq 1$, the process can be repeated at most $\min(a,b)$ times. At the end of this process, we are guaranteed to find at least one critical pair.
\end{proof}




%=============================================

\begin{highlight}
\textit{(this is basically Dirichlet's Lemma, we just need to connect the dots.)}
\begin{lemma}
Let $\alpha$ be an irrational number. Given any $\epsilon > 0$, there exists an $n \in \N$ such that $n\alpha-\floor{n\alpha}<\epsilon$.
\end{lemma}
\end{highlight}


%(this part will be reversed in the final proof?)
%\[
%\begin{array}{rcl}
%\dfrac{q^\Omega}{p^a} &=& 1\\
%q^\Omega &=& p^a\\
%\Omega\log_pq &=&a\\
%\Omega\log_pq &=&\floor{a}+\epsilon\\
%q^\Omega &=& p^{\floor{a}}p^\epsilon\\
%\dfrac{q^\Omega}{p^{\floor{a}}}&=&p^\epsilon\\
%&<&\min\left(\dfrac{q^\ell}{p^k}\right)\\
%\end{array}
%\]

\begin{theorem}
For any two distinct prime numbers $p$, and $q$, there exist infinitely many critical pairs. 
\begin{proof}
Let $p$ and $q$ be distinct primes. Suppose for contradiction that there exist finitely many critical pairs, and denote the set of these as $S=\{(p^{k_1}, q^{\ell_1}), \ldots, (p^{k_N}, q^{\ell_N})\}$. Of these critical pairs, consider the subset $C=\{(p^{k_i}, q^{\ell_i}) : \, p^{k_i} < q^{\ell_i}\}$, where $i\in\N$ such that $1\leq i \leq N$. This means that 
$$1 < \frac{q^{\ell_i}}{p^{k_i}}, \quad \forall (p^{k_i}, q^{\ell_i}) \in C.$$
Choose some $\epsilon\in\R$ such that 
$$1<p^\epsilon<\min\left(\frac{q^{\ell_i}}{p^{k_i}}\right).$$
Now, consider the irrational number $\log_pq$. By the Lemma, there exists some $\Omega\in\N$ such that 
$$\Omega\log_pq - \floor{\Omega\log_pq} < \epsilon.$$
To simplify the notation, let $a=\floor{\Omega\log_pq}$. Thus, with a little algebra, 
\[
\begin{array}{rcl}
\Omega\log_pq 	&<&a+\epsilon\\
q^\Omega 		&<& p^{a}p^\epsilon\\
\dfrac{q^\Omega}{p^a}&<&p^\epsilon\\
				&<&\min\left(\frac{q^{\ell_i}}{p^{k_i}}\right)\\
\end{array}
\]
and we find that for all $({q^{\ell_i}},{p^{k_i}})\in C$,
$$1<\dfrac{q^\Omega}{p^a}<\frac{q^{\ell_i}}{p^{k_i}}.$$
Therefore, by Proposition 3.2, $({q^\Omega},{p^a})$ is a critical pair with ${p^a}<{q^\Omega}$. But, since $\frac{q^\Omega}{p^a}<\frac{q^{\ell_i}}{p^{k_i}}$ for all $({q^{\ell_i}},{p^{k_i}})\in C$, then $({q^\Omega},{p^a})\not\in C$, which is a contradiction. 

Therefore, we have shown that $C$ cannot be finite, and since $C\subset S$, then $S$ cannot be finite either. 
\end{proof}
\end{theorem}


\end{document}
