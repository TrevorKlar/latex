\documentclass[12pt]{article}

\usepackage{geometry,calc}
\usepackage{amsmath,amssymb,hyperref}

\def\quiztitle{Practice Problems 1}
\def\quizsubtitle{Math 4B,\qquad Spring 2017,\qquad Dr. Paul}
\pagestyle{empty}

\geometry{body={6.5in, 9in}, left=1in, top=1in}
\input{../../AxesFunction.tex}

\everymath{\displaystyle}

\begin{document}

%\hfill Student Name: \rule{2in}{.1pt}\\

%\hfill Section Time (e.g. 8am): \rule{2in}{.1pt}\\

\begin{center}
{\Huge \quiztitle}
\vskip.1in
{\large \quizsubtitle}
\end{center}

Practice problems are for your own benefit. You won?t turn them in or have them graded, but I have the expectation that you have done these when I write my tests. You can check answers with a TA, in Math Lab, or with the professor.

\begin{enumerate}

 \item In this problem, you need to set up and solve a differential equation modeling the number of fish $F$ in Lake Cachuma as a function of time $t$ based on the following information:
 \begin{itemize}
  \item Authorities add fish to the lake at a \emph{constant rate}.
  \item Fishermen remove fish from the lake at a rate \emph{proportional} to the number of fish in the lake.
 \end{itemize}
 Write down your differential equation and one sentence explaining it.


\item Recall our Disease Spread model from class. Are there inflection points for the numbers of susceptible, infected, and resistant populations? Under what circumstances (i.e. for what values of $S,I,R$) do these occur?

\item Recall our Disease Spread model from class. What happens in the following variations of the situation:
\begin{enumerate}
 \item \textbf{Quarantine:} The government is considering issuing a quarantine, which would cut the transmission coefficient in half.
 \item \textbf{Vaccine:} Scientists have developed a vaccine for disease X, and they can administer the vaccine to approximately 2500 susceptible individuals per day.
 \item \textbf{Cure:} Scientists have developed a cure for disease X, which will immediately make an infected individual resistant. They can administer the cure to approximately 20\% of the infected individuals per day.
 \end{enumerate}
 
 \end{enumerate}

\end{document}