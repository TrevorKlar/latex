\documentclass[epsf]{article}
\usepackage{amsmath,amsthm,amsfonts,latexsym,amscd, framed}
\usepackage{graphicx}
%\title{Project \#2: The Accumulation Function}
%\date{Solutions due Friday, October 12th}

\textwidth=6.0truein\hoffset=-.5truein
\textheight=8.5truein\voffset=-.5truein

\begin{document}
%\maketitle
\newcommand{\R}{\mathbb{R}}
\newcommand{\noi}{\noindent}
\newcommand{\bs}{\bigskip}

%%%%

\begin{center}
{\Large Math 4B Fall 2014\\
\vskip 4mm
 Discussion Projects Evaluation Rubric}
\end{center}

This handout describes how discussion projects are evaluated.  Please pay careful attention to feedback on your DP solutions as the processing and reflection of this feedback is critical for learning and it will help you succeed on future DPs and on your exams. \\

Your discussion projects will be evaluated in the following two categories:

\begin{itemize}
\item {\bf Quality of Mathematical Ideas:} 
\begin{itemize}
\item Work that scores a 7 in this category address all questions and contains complete and correct mathematical reasoning.  In writing such solutions the authors brought to mind appropriate mathematical knowledge and used these ideas reasonably to form thorough solutions.  
\item Work that scores in the range 3-6 in this category mostly brings appropriate mathematical knowledge to mind and attempts to use these ideas to address the major questions.

\item Work that scores in the range 1-3 in this category has brought somewhat reasonable mathematical knowledge to mind and minimally addresses the major questions in the project.
\end{itemize}




\item {\bf Clarity of Communication:} 
\begin{itemize}
\item Work that scores a 3 in this category communicates the ideas with an appropriate mix of clearly worded complete sentences and correctly written mathematical statements.  It is readily apparent which question the authors are responding to throughout the solution.  The solution includes clear and thorough pictures and/or diagrams where appropriate and these images are appropriately referenced in the work.  All notation and mathematical vocabulary is consistent and is used correctly. 

\item Work that scores a 2 in this category communicates ideas with a somewhat coherent mix of clearly worded complete sentences and mathematical statements.  The solution includes pictures and/or diagrams where appropriate and attempts to use them to clarify important ideas.  Most of the notation and mathematical vocabulary is consistent and correct.

\item Work that scores a 1 in this category attempts to communicate important ideas.  A significant portion of the notation and mathematical vocabulary are used correctly.
\end{itemize}
\end{itemize}

\end{document}
You DP score will be the sum of your scores in each of the three categories above.  I know you are accustomed to receiving numerical feedback on a 100 pt scale, but I ask that you interpret your scores in these categories a bit differently using the following guidelines:
\begin{itemize}
\item {\bf Score of 3:} Very high quality work with no significant mistakes or shortcomings.  It will likely take some practice to get to the point where you can consistently produce work at this level in all three categories.

\item{\bf Score of 2:}  Work in this category is mostly complete, but with one or more significant flaws.  It will take work that is not insignificant to fix the problem(s) in this category. 

\item{\bf Score of 1:} Work in this category contains one or more major mistakes or several significant shortcomings.  It will take substantial work to fix the problem(s) in this category.
\end{itemize}


\end{document}

\item {\bf Structure of Mathematical Reasoning:} 
\begin{itemize}
\item Work that scores a 3 in this category is structured to reflect the flow of the mathematical reasoning. All key ideas are linked explicitly using appropriate transition words and phrases (e.g. ``because'', ``since'', ``therefor'', ``it follows that'').  Information is presented when it is most relevant to the mathematical reasoning.    
\item Work that scores a 2 in this category is structured in a way that indicates the flow of the mathematical reasoning.  A significant portion of the key ideas are appropriately linked with transition words or phrases.  Most information is presented when it is most relevant to the mathematical reasoning.
\item Work that scores a 1 in this category is structured in a way that detracts from the flow of mathematical reasoning.  Only a few key ideas are appropriately linked with transition words or phrases and some important information is presented either before or after it is most relevant to the mathematical reasoning.


\begin{center}
Scoring Rubric
\vskip 2mm
\begin{tabular}{| l | c |}
\hline
 & \\
quality of mathematical ideas &  \hspace{2 cm} \\
 & \\
\hline
 & \\
structure of mathematical reasoning & \hspace{2 cm} \\
 & \\
\hline
 & \\
clarity of communication & \\
 & \\
\hline
\end{tabular}
\end{center}


