\documentclass[epsf]{article}
\usepackage{amsmath,amsthm,amsfonts,latexsym,amscd, framed}
\usepackage{graphicx}
%\title{Project \#2: The Accumulation Function}
%\date{Solutions due Friday, October 12th}

\textwidth=6.0truein\hoffset=-.5truein
\textheight=8.5truein\voffset=-.5truein

\begin{document}
%\maketitle
\newcommand{\R}{\mathbb{R}}
\newcommand{\noi}{\noindent}
\newcommand{\bs}{\bigskip}

%%%%


\begin{center}
{\Large Project \#2: Solutions to 2nd Order Linear Homogeneous DEs.\\
\vskip 2mm
TA Guide Pt 1}
\end{center}

\vskip 4mm

\noi {\bf 1} Same as last time - encourage them to find a partner to work with and pair up any students who might need more encouragement.  Tip: Only hand out half as many copies of the pre-work as there are students, so there is some incentive to work with a partner.  Then set them loose on PW1 pt (a).  Many of them will have no idea what to ``guess''.  Encourage them by asking things like ``What functions do you know that look like their own derivatives?'' or just say ``Name a function, now check to see if it is a solution.''  After about 10 minutes, bring the class together.\\




\noi{\bf PW 1}  Consider the following DE
$$ y'' + y' - 6y = 0$$
\noi (a) Guess and check to find two (or more?) \textit{different} solutions to the DE.  Call the first solution $y_1(t)$ and the second solution $y_2(t)$.
\vskip 2mm

\noi {\bf 2} While your at the board, ask them to tell you their solutions and write down the ones they came up with.  Be sure to ask ``Did anyone get a different solution?'' etc. to draw out all the ideas.  Then ask ``Do you all agree that these are solutions''.   Check one with them if the room is in disagreement.  I'm betting that someone will be clever enough to try $e^{rt}$ and get to the characteristic polynomial.  If someone does, ask them to walk you through their solution for the class - i.e. they tell you what they did and you write it down to make sure the class can follow the idea.\\

\noi {\bf 3} Once they know some solutions and (hopefully) how to find all of the exponential solutions, label one $y_1$ and another $y_2$ for the class.  Then have them work on (b) and (c) for 5 or 10 minutes.  Circulate to make sure they all know what they are supposed to be doing for these parts of the problem.\\


\noi (b) Show that if you multiply your first solution $y_1$ by any scalar, say $\alpha$, that $\alpha y_1(t)$ will also be a solution to the DE.
\vskip 2mm
\noi (c) Show that the sum of your two solutions, $g(t) = y_1(t)+y_2(t)$, is also a solution to the DE.
\vskip 2mm

\noi {\bf 4} When about 1/3 of them are done with part (c), bring the class together and have someone new walk you through their solution of (b) while you write.  Then ask for someone new to walk you through (c).  Then have them work on (d) and (e) for 5-10 minutes.  Again, circulate to make sure they understand what they should be doing for these parts.\\

\noi (d) How many solutions can you find that satisfy the initial condition $y(0) =  1$?
\vskip 2mm
\noi (e) How many solutions can you find that satisfy the initial conditions $y(0) = 1$ AND $y'(0) = 5$?\\

\noi {\bf 5} When about 1/3 of the class has done something for part (e), bring them together again and ask for a new volunteer to walk you through (d).  Write down whatever they say (even if it is wrong, because a mistake is likely a misconception that the class needs to consider, then work through), then ask ``do you agree?'' ``Did anyone find more solutions?'' etc.  Then move on to (e), have a new student walk you through that one, write whatever they say (right or wrong) and have the class consider the solution.  At some point in the discussion of part (e) ask them if they think these initial conditions should guarantee a \textit{unique} solution.\\

\noi{\bf 6} Now work through PW2 the same way.  I doubt you'll have time to finish the whole thing, but hopefully you can at least get through part (a) to set them up for the rest.\\

\noi{\bf PW 2} Now consider the DE
$$ y'' + 4y = 0$$
\noi (a) Guess and check to find two (or more?) \textit{different} solutions to the DE.  Call the first solution $y_1(t)$ and the second solution $y_2(t)$.
\vskip 2mm
\noi (b) Show that if you multiply your first solution $y_1$ by any scalar, say $\alpha$, that $\alpha y_1(t)$ will also be a solution to the DE.
\vskip 2mm
\noi (c) Show that the sum of your two solutions, $g(t) = y_1(t)+y_2(t)$, is also a solution to the DE.
\vskip 2mm
\noi (d) How many solutions can you find that satisfy the initial condition $y(0) =  1$?
\vskip 2mm
\noi (e) How many solutions can you find that satisfy the initial conditions $y(0) = 1$ AND $y'(0) = 5$?\\

\end{document}