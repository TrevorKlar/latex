\documentclass[epsf]{article}
\usepackage{amsmath,amsthm,amsfonts,latexsym,amscd, framed}
\usepackage{graphicx}

\textwidth=6.0truein\hoffset=-.5truein
\textheight=8.5truein\voffset=-.5truein

\begin{document}
%\maketitle
\newcommand{\R}{\mathbb{R}}
\newcommand{\noi}{\noindent}
\newcommand{\bs}{\bigskip}

%%%%

\vspace{-0.5 in}
\noi NAME(S): \line(1,0){130} \hspace{.2in}TA (circle one): Elizabeth \ \ \ Christian 
\vskip .2 in
\hspace{.44 in} \line(1,0){130} \hspace{.2in}SECTION (circle one): 8AM \ 12PM\ 4PM \ 5PM 
\vskip 1mm
\hspace{3.9 in} 6PM\ \ 7PM





\begin{center}
{\Large Project \#2: 2nd Order Linear DEs\\
\vskip 2mm
Solutions Page}

\vskip 4mm
Feedback
\vskip 2mm
\begin{tabular}{| l | c |}
\hline
 & \\
quality of mathematical ideas (7 pts) &  \hspace{2 cm}   \\
 & \\
\hline
 & \\
clarity of communication (3 pts) & \\
 & \\
\hline
\end{tabular}
\end{center}

\vskip 2mm
Please write your group or individual solution on this page.  Staple any additional work for your solutions on the back of this page to turn in during section on Wednesday, October 29th.  If you cannot attend section, get your solutions to your TAs mailbox in SH 6623 by 4:00pm that day.\\



\noindent{\bf Problem 1} \footnote{This problem will be graded according to the DP Evaluation Rubric handout, available on GauchoSpace.}  Consider the DE $$y'' + 4y = 0.$$

\begin{itemize}
\item[(a)] Find a \textit{fundamental set} of solutions to the DE.  That is, find two linearly independent solutions to the DE, call them $y_1$ and $y_2$.
\vskip 1in
\item[(b)] \textit{Show} that any linear combination of your two solutions $y_1$ and $y_2$ from part (a) will also be a solution to the DE.

\newpage
\item[(c)] Explain how to find a solution (from your linear combination solutions in part (b)) that satisfies the general initial conditions $y(0)=a$ and $y'(0)=b$. 

\vskip 4in
\item[(d)] The DE $y'' + 4y = 0$ with initial conditions $y(0)=a$ and $y'(0)=b$ is an IVP.  Does this IVP \textit{always} have a solution?  Will there ever be more than one solution to the IVP that can be written as a linear combination of your functions $y_1$ and $y_2$?  Justify your claim by referring to your work for parts (b) and (c) above.
\end{itemize}






\end{document}

