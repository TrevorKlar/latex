\documentclass[epsf]{article}
\usepackage{amsmath,amsthm,amsfonts,latexsym,amscd, framed}
\usepackage{graphicx}

\textwidth=6.0truein\hoffset=-.5truein
\textheight=8.5truein\voffset=-.5truein

\begin{document}
%\maketitle
\newcommand{\R}{\mathbb{R}}
\newcommand{\noi}{\noindent}
\newcommand{\bs}{\bigskip}

%%%%

\vspace{-0.5 in}
\noi NAME(S): \line(1,0){130} \hspace{.2in}TA (circle one): Jared \ \ \ Chris \ \ \ ChangLiang \ \ \ Matt
\vskip .2 in
\hspace{.44 in} \line(1,0){130} \hspace{.2in}SECTION (circle one): 8AM \ 4PM \ 5PM \ 6PM \ 7PM




\begin{center}
{\Large Project \#1: Undoing the Chain Rule\\
\vskip 2mm
TA Guide Part II}

\vskip 4mm
Feedback
\vskip 2mm
\begin{tabular}{| l | c |}
\hline
 & \\
quality of mathematical ideas (7 pts) &  \hspace{2 cm}   \\
 & \\
\hline
 & \\
clarity of communication (3 pts) & \\
 & \\
\hline
\end{tabular}
\end{center}

\vskip 2mm
Please write your group or individual solution on this page.  Staple any additional work for your solutions on the back of this page to turn in during section WEEK 2.  If you cannot attend section, get your solutions to your TAs mailbox in SH 6617 by 4:00pm the day your section meets.\\

\noi {\bf 1}.  You'll want to get them organized.  Tell them we're going to discuss the solutions to DP\#1 and ask them to sit with their partner if they worked with someone in section.   Then pick one problem to model for them (say part (a)) and ask the class to walk you through the text-book ``how-to'' solution.  
\begin{itemize}
\item PROMPT: \textit{Someone tell me, where should I start?} They'll probably say things like ``get $y$ terms on same side'' or ``multiply by blah..''.  Make whatever move is suggested and ask the room ``Do you agree or disagree?"  I suspect they will walk you through the algebra just fine.  Write down the transition words and phrases they tell you - you want the solution on the board to be a model for what you expect them to do in Problem \#2.

\item PROMPT: \textit{ Am I integrating with respect to $x$ or $y$?}  This will be the tricky part.  They should be bothered by the idea of integrating with respect to one variable on one side and another on the other.  When you write the integration step, you should write it on a new line (as opposed to adding integral notation on top of the equation as written).  Write it as integration with respect to $x$ and then, on a new line, discuss why they works like integration with respect to $y$ -- i.e. write down some sentence that explains/indicates that we can do this because we are un-doing the chain rule.  You should probably think about how you want to explain this ahead of time.  Then you end up with the equation for a curve (or, rather a family of curves).

\item PROMPT: \textit{How is this (equation for a curve) also a solution to the DE?  Wasn't I looking for a function?}  Then (hopefully) they will explain to you how if you solve for $y$ you get an expression for $y(x)$.  Remind them what I mean by the instruction ``it's okay to leave your solution as an implicitly defined function''.  You can also show them how to solve for $y$ for the problem in (a) since they need to do that for one or two of the online HW problems.

\item PROMPT: \textit{Is there anything else we should add to this solution? i.e. any step that needs more clarification?}  See where this goes.  If it peters out and they're good with what's written on the board, give them 5-10 minutes to revise what they've written for Problem \#2.  You can go ahead and tell them that is the only problem we will grade.  
\end{itemize}

\noi Continued on back...

\newpage

\noi{\bf 2.} While they are revising, write on the board to turn in one solutions page for each group and to make sure all names are on the solutions page.  Tell there will be time to discuss HW questions, so ask them to think about what HW problems they would like to discuss if they are finished with their revisions.  Then, once papers are collected, you can respond to their HW questions.  \\  

\noindent{\bf Problem 1}  Use the ideas you developed in your pre-work to find some solutions to the following
differential equations. You may describe the solution as being an implicitly defined function, but be sure to note when you are ``undoing" the chain rule in your solution.
\begin{itemize}
\item[(a)] $y'=\frac{x+1}{y-1}$

\item[(b)] $y'=e^{x+y}$\\

\end{itemize}

\noindent{\bf Problem 2}\footnote{This problem will be graded according to the DP Evaluation Rubric handout, available on GauchoSpace.} Now solve the differential equation
$$ y' = \dfrac{3x^2 - 1}{2y}.$$
In the space below write up the solution as though it were a ``how to'' example in a text-book.  Make sure you explain the mathematics behind each step and be sure to note when you ``undo'' the chain rule.\\





\end{document}

