\documentclass[12pt]{article}
%%%%%%%%%%%%%%%%%%%%%%%%%%%%%%%%%%%%%%%%%
\usepackage{amscd}\usepackage{amsmath}\usepackage{amssymb}\usepackage{graphicx}
\usepackage{amsthm}\usepackage{latexsym}\usepackage{verbatim}\usepackage{multicol}
\setlength{\textheight}{8.5in} \setlength{\topmargin}{0.in}
\setlength{\headheight}{0.0in} \setlength{\headsep}{0.0in}
\setlength{\leftmargin}{0.5in}\setlength{\oddsidemargin}{0.0in}
%\setlength{\parindent}{1pc}\linespread{1.6}
\setlength{\textwidth}{6.7in}
\newcommand{\noi}{\noindent}
\newenvironment{loose_item}{
\begin{itemize}
  \setlength{\itemsep}{75pt}
  \setlength{\parskip}{40pt}
  \setlength{\parsep}{40pt}
}{\end{itemize}}
%%%%%%%%%%%%%%%%%%%%%%%%%%%%%%%%%%%%%%%%%
\begin{document}

\begin{center}
Math 4B Midterm Review Problems
\vskip 2mm
TA Guide
\vskip 2mm
October 30, 2014
  
\end{center}

\noindent These are practice problems to help you prepare for your midterm, you do not need to turn in solutions.  You should think of this as a starting point for organizing your study plan.   You should also review your DPs, old homework problems and problems from the lecture slides.
\vskip 2mm
\noi Your midterm will cover material from sections 1.1, 1.2, 2.1-2.5, 2.9, 3.1-3.4 and 3.7  \\


\noi{\bf Getting Started }You won't have time to work through all of these problems, so let the class vote on what they want to discuss most.  Write down the number and topic of each problem (e.g. \#4 - connecting difference/differential equations,  \#13 substitution, \#14 qualitative sketch of solutions, \#18 2nd order solution spaces/reduction of order and \#19 finding real solutions from complex).  Start with the most requested problem and work from there.  You probably want to work through all of these problems before section, some of them might be a bit tricky to do on the fly.\\

\noi{\bf For \#4.} For the savings account problem, have them work on (a), (b), and (c) for about 5 minutes or so.  When it seems like about 1/3 of them are done with a,b,c stop the class and come together to discuss the answers they got.  Then have them work on part (d) -- make sure they can make sense of the hint before you let go of their attention.  After about 1/3 of them have figured something out, bring the class together and talk about the answers they got.\\

\noi{\bf For \#13.}  For the substitution problems you could just have them walk you through the solutions in real time.  Tell them you want a different student to walk you through each step.  If they're clamming up, you could just solve the problems for them -- it will be faster, but it might discourage questions.  However, I find that they have lots of questions right before a midterm, so a shy class might be less of an issue today.\\

\noi{\bf For \#14. } The sketching solutions problem might feel like a new kind of problem for them, since WileyPLUS can't ask them to do this.  Start the picture for them and ask them something like ``What do we do to find equilibrium solutions?" - hopefully someone will say something coherent/helpful about the process that you can re-voice.  Then ask ``How can we tell if they are stable, semistable or unstable?'' - again, the hope is that someone will say something good that you can repeat.  Once they have that idea in their head, have them work through part (a) for 3-5 minutes.  Then bring the class together and add their new information to your picture.  Next have them work on (b) for 3-5 minutes, then come together to talk about the answers and sketch the graphs (maybe a student will do this for you and you can talk about the students' picture).\\

\noi{\bf For \#18. } Give them 3-5 minutes to think about and/or discuss parts (a) and (b) with classmates.  Then come together and talk about the answers to these parts.  The answer to question (b) really rests on the fact that we can solve the IVP with \textit{any} initial position and \textit{any} initial velocity...which means no restrictions on $a$ and $b$.   Then give them a few minutes to work on part (c) -- you may want to remind them of how it starts (we assume $y_2(t) = f(t)y_1(t)$, then plug this in to get a DE for the mystery function $f(t)$) before they get rolling.  There is a reduction of order formula that they might use, and that's fine.  This problem is also not too bad if you work from scratch.  Once it looks like roughly 1/3 are done, you can have them walk you through the solution.  Then you can solve the IVP.  I realize now that $b=0$ is lame because solving this IVP doesn't require using your new $y_2$.  Maybe asking them to solve IVP for $a=1$ and $b=5$ will be more interesting.\\

\noi{\bf For \#19. }  Give them 3 minutes or so to solve part (a), then discuss the solution as a class.  NOTE that your coefficients here are complex.  Next have them work on (b) and (c) for a few minutes and see if they get the same solution despite the coefficients being different.  Be sure to reiterate why this problem is interesting: it highlights that your fundamental set is not unique, and that only a complex linear combination of complex-valued solutions will give you a real solution.\\  


\begin{itemize}
\item[4.] A deposit into a savings account earns interest, which is just a fraction of your deposit added to the total at regular intervals.  
\begin{itemize}
\item[(a)] Suppose your account earns $8\%$ each year and that interest is compounded once a year, i.e. $8\%$ of the amount is added each year.  How much money will you have after 5 years with an initial deposit of \$$100$?   After $N$ years?
\item[(b)] Now suppose the interest is compounded monthly.  How much will you have in the account after 5 years?
\item[(c)] Write down a \textit{difference equation} that describes how the account value is changing.  Suppose the annual interest rate $r$ is compounded $n$ times per year.  Your difference equation should look something like
$$A_{k+1} - A_k = ??$$
\item[(d)] Now suppose the bank makes it's payments more and more often:  daily, hourly, every minute, every second... continuously.  What will your difference equation look like if interest is compounded continuously?  HINT: Let $A(t) = A_k$ and let $\Delta t = \frac{1}{n}$, then find an expression for $\Delta A = A(t+\Delta t) - A(t)$.  In the limit as $\Delta t \to 0$, you should get a \textit{differential equation}.
\item[(e)] Compare the return after 5 years on two accounts with $A_0 = \$100$ and $r = 8\%$ - one compounded monthly and one compounded continuously.  What kind of account do you want to invest in?
\end{itemize}

%%%%%%%%%%%%%%%




\item[13.] Make a substitution to solve the following DEs.  
\begin{itemize}
\item[(a)] $y' = \dfrac{3y^2-x^2}{2xy}$, let $v= \frac{y}{x}$.
\item[(b)] $y' + 1 = (y+x)^2$, let $v = x+y$.
\item[(c)] $2yy' = \cos(y^2)$, let $v = y^2$.
\end{itemize}


\item[14.] Consider the following autonomous DE:
$$ y' = y^2(4-y^2).$$
Use qualitative information to sketch solution curves to this equation:
\begin{itemize}
\item[(a)] Find the equilibrium solutions and classify them as stable, semistable or unstable.
\item[(b)] Find a formula for $y''$ and use this to determine the concavity of solutions for certain values of $y$.
\item[(c)] Sketch several graphs of solutions in the $ty$-plane.
\end{itemize}




\item[18.] Consider the IVP $t^2y'' - t(t+2)y' + (t+2)y = 0$, $y(1)=a$, $y'(1)=b$
\begin{itemize}
\item[(a)] Verify that $y_1(t) = t$ is a solution to the DE.  For which values of $a$ and $b$ is the solution to the IVP a scalar multiple of $y_1$?
\item[(b)] Explain how you know that the fundamental set for this DE will have at least one other solution.  Use Theorem 3.2.1 from your book in your argument.
\item[(c)] Use reduction of order to find the second solution for the fundamental set for this DE.
\item[(d)] Solve the IVP for $a = 1$ and $b=0$.
\end{itemize}



\item[19.] Consider the IVP $y''-2y'+2y = 0$, $y(0) = 2$ and $y'(0) = 0$.  
\begin{itemize}
\item[(a)] Find a fundamental set of \textit{complex} solutions to the DE.  Then find a solution to the IVP from among those complex solutions.
\item[(b)] Use Euler's Formula to find a fundamental set of \textit{real} solutions to the DE.
\item[(c)] Find a solution to the IVP from among your real solutions.  Then use Euler's formula to show that this solution is the same as the solution you found in part (a).
\end{itemize}



\end{itemize}


%%%%%%%%%%%%%%%%%%%%%%%%%%%%%%%%%%%%%%%%%
\end{document}
\item Consider a first order IVP of the form
$$y' = f(t,y), \ \ y(0) = y_0.$$
Which of the following statements correctly applies to Euler's Method for approximating solutions to differential equations.
Circle \textit{all} that apply.
\begin{itemize}
\item[(A)] The resulting difference equation $y_{n+1} = y_n + f(t_n,y_n)h$ follows from iterating tangent line approximations of the solution.
\item[(B)] Decreasing the size of the time step will require making more computations.
\item[(C)] The Euler's method approximation is always greater than the actual values of the solution.
\item[(D)] Each iteration of the Euler's Method only requires data from previous time steps.
\end{itemize}

%%%%%
