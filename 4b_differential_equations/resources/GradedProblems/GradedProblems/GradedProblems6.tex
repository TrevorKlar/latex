\documentclass[12pt]{article}

\usepackage{geometry,calc}
\usepackage{amsmath,amssymb,hyperref}

\def\quiztitle{Graded Problem 6}
\def\quizsubtitle{Math 4B,\qquad Spring 2017,\qquad Dr. Paul}
\pagestyle{empty}

\geometry{body={6.5in, 9in}, left=1in, top=1in}
\input{../../AxesFunction.tex}

\newcommand{\be}{\begin{enumerate}}
\newcommand{\ee}{\end{enumerate}}

\everymath{\displaystyle}

\begin{document}

%\hfill Student Name: \rule{2in}{.1pt}\\

%\hfill Section Time (e.g. 8am): \rule{2in}{.1pt}\\

\begin{center}
{\Huge \quiztitle}
\vskip.1in
{\large \quizsubtitle}
\end{center}

\begin{enumerate}

 \item Consider a 50 kg mass on a spring with spring constant 800 N/m.
 \be
  \item What damping constant would the shock absorber need to be critically damped (include units)?
  \item Suppose the system is damped using the shock absorber and damping constant from part (a). The mass starts at rest, but a driving force of $F(t)=70\sin(3t)$ is applied to mass. Find the position function for the mass.
  \item In your solution to part (b), what is the frequency and amplitude of the steady-state oscillation (i.e. of the oscillation that will persist as $t\rightarrow\infty$)?
 \ee
 
 \item Look back at the fox and mouse problem that we set up in class.
  \be
   \item Use Excel or some other program to get more specific information of what happens to the mouse-fox populations if there are initially $2000$ mice and $10$ foxes. Use an appropriate value for $\Delta t$.
   \item Give a visual representation of your numerical solutions.
   \item (Bonus) For an analytic approach, find the ``slope'' $\frac{dF}{dM}$ as an expression in $F$ and $M$ using the chain rule $\frac{dF}{dM}\frac{dM}{dt}=\frac{dF}{dt}$. Use separation of variables to find an implicit solution relating $F$ and $M$. (Do not try to solve for $F$.) Use \url{http://www.desmos.com} to graph the implicit solution with the initial values from part (a) and sketch the result. Write a sentence explaining how the graph you obtain here is different from the one obtained from the numerical approach, and why this happened.
  \ee
\end{enumerate}



\end{document}