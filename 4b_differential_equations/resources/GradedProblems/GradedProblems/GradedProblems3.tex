\documentclass[12pt]{article}

\usepackage{geometry,calc}
\usepackage{amsmath,amssymb,hyperref}

\def\quiztitle{Graded Problem 3}
\def\quizsubtitle{Math 4B,\qquad Spring 2017,\qquad Dr. Paul}
\pagestyle{empty}

\geometry{body={6.5in, 9in}, left=1in, top=1in}
\input{../../AxesFunction.tex}

\everymath{\displaystyle}

\begin{document}

%\hfill Student Name: \rule{2in}{.1pt}\\

%\hfill Section Time (e.g. 8am): \rule{2in}{.1pt}\\

\begin{center}
{\Huge \quiztitle}
\vskip.1in
{\large \quizsubtitle}
\end{center}

\begin{enumerate}

 \item The Lake Cachuma reservoir system is actually more complex than we let on earlier. Really, run-off water flows mostly into the Gibraltar Reservoir, water from Gibraltar Reservoir flows into Lake Cachuma, and we take water from Lake Cachuma for drinking water. Instead of tracking nitrates, we will track mercury in this problem.

 Suppose pure water flows into Gibraltar at a rate of $r_g=1$ trillion L/mo; water flows from Gibraltar into Cachuma at a rate of $r_c=1$ trillion L/mo; and we drain Cachuma at a rate of $r_d=1$ trillion L/mo. Each lake has a volume of 10 trillion L.
 
Currently neither lake has any mercury at all, but authorities want to plan for the possibility of a rockslide at the Sunbird Quicksilver Mine, which could deposit up to 10kg of mercury directly into Gibraltar. If this happens, what would be the maximum concentration of mercury that occurs in Cachuma?
  
  \item A 30-year-old woman accepts an engineering position with a starting salary of \$30,000 a year. Her salary $S(t)$ increases exponentially, with $S(t)=30e^{t/20}$ thousand dollars after $t$ years. Meanwhile, 12\% of her salary is deposited continuously in a retirement account, which accumulates interest at a continuous (nominal) annual rate of 6\%. How much money does she have in her retirement account when she is 70 (i.e., when $t=40$)?
  

  
\end{enumerate}



\end{document}