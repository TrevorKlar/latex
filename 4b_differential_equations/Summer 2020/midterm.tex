\documentclass[11pt]{exam}
\RequirePackage{amssymb, amsfonts, amsmath, latexsym, verbatim, xspace, setspace}
\RequirePackage{tikz, pgflibraryplotmarks}
\usepackage[margin=1in]{geometry}
\usepackage{pgfplots}
\usepackage{graphicx}

% Here's where you edit the Class, Exam, Date, etc.
\newcommand{\class}{Math 4B}
\newcommand{\term}{Summer Session B}
\newcommand{\quiznum}{Midterm}
\newcommand{\quizdate}{21 August 2020}

% For an exam, single spacing is most appropriate
\singlespacing
% \onehalfspacing
% \doublespacing

% For an exam, we generally want to turn off paragraph indentation
\parindent 0ex

\begin{document} 

% These commands set up the running header on the top of the exam pages
\pagestyle{head}
\firstpageheader{}{}{}
\runningheader{\class}{\quiznum\ - Page \thepage\ of 10}{\quizdate}
\runningheadrule

\begin{flushright}
\begin{tabular}{p{2.8in} r l}
\textbf{\class} & \textbf{Name:} & \makebox[2in]{\hrulefill}\\
\textbf{\term} &&\\
\textbf{\quiznum} &&\\
\textbf{\quizdate} &&\\
\end{tabular}\\
\end{flushright}
\rule[1ex]{\textwidth}{.1pt}


\begin{enumerate}
	
\item (1 point) Please write the following sentence: \\

``I, [your name], understand that if I get the answer and show no work, it will be assumed that I copied off someone else and may be reported for cheating."

\newpage
\item[2] (10 points each) Find the general solution of the given differential equation. Write your answer in explicit form. 

\begin{enumerate}
	\item $y'=t^2e^y$
\end{enumerate}

\pagebreak

\item[2.] (10 points each) Find the general solution of the given differential equation. Write your answer in explicit form. 

\begin{enumerate}
	\setcounter{enumii}{1}
	\item $2u''+4u=0$
\end{enumerate}

\pagebreak

\item[2.] (10 points each) Find the general solution of the given differential equation. Write your answer in explicit form. 

\begin{enumerate}
	\setcounter{enumii}{2}
	\item $y'=\dfrac{x^{-6}(x-1)}{5y^4}$
\end{enumerate}

\pagebreak

\setcounter{enumi}{2}
\item (10 points each) Solve the given initial value problem.

\begin{enumerate}
	\setcounter{enumii}{0}
	\item $-w''+10w'-25w=0$, \hspace{.3cm} $w(0)=1$, \hspace{.2cm} $w'(0)=-1$
\end{enumerate}

\pagebreak

\setcounter{enumi}{2}
\item (10 points each) Solve the given initial value problem.

\begin{enumerate}
	\setcounter{enumii}{1}
	\item $xy'+(x+1)y=x^2e^{-x}$, \hspace{.2cm} $x>0,  \hspace{.3cm} y(3)=0$\\
\end{enumerate}

\newpage

\item Given that $y_1(t)=t^{-3}$ is a solution of 

$$t^2y''+2ty'-6y=0, \hspace{.3cm} t>0,$$

\begin{enumerate}
	\item (15 points) Use the reduction of order method to find a second solution of the form $y_2(t)=t^k$.\\
	
\newpage
	
	\item (5 points) Do $y_1$ and $y_2$ from part (a) form a fundamental set of solutions? Why?
	
\newpage
	
	\item (5 points) Using your answers from parts (a) and (b), find the solution which satisfies the initial conditions $y(1)=1$ and $y'(1)=12$.\\
	
\end{enumerate}

\newpage

	\item (10 points) Suppose $y_1$ and $y_2$ form a fundamental set of solutions to some differential equation. Let $y_3=y_2+y_1$. Do you think that the pair $\{y_3,y_1\}$ would also form a fundamental set of solutions, yes or no? Explain your reasoning and thoughts. You may use theory from linear algebra and/or differential equations.\\

\end{enumerate}



















 \end{document}
 
 
 
 
 
 
 
 
 
 
 
 
 
 
 
 
 
 
 