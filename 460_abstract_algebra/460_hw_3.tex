\documentclass[letterpaper]{article}
%\documentclass[a5paper]{article}

%% Language and font encodings
\usepackage[english]{babel}
\usepackage[utf8x]{inputenc}
\usepackage[T1]{fontenc}


%% Sets page size and margins
\usepackage[letterpaper,top=.75in,bottom=1in,left=1in,right=1in,marginparwidth=1.75cm]{geometry}
%\usepackage[a5paper,top=1cm,bottom=1cm,left=1cm,right=1.5cm,marginparwidth=1.75cm]{geometry}

\usepackage{graphicx}
%\graphicspath{../images}	  %%where to look for images

%% Useful packages
\usepackage{amssymb, amsmath, amsthm} 
%\usepackage{graphicx}  %%this is currently enabled in the default document, so it is commented out here. 
\usepackage{calrsfs}
\usepackage{braket}
\usepackage{mathtools}
\usepackage{lipsum}
\usepackage{tikz}
\usetikzlibrary{cd}
\usepackage{verbatim}
%\usepackage{ntheorem}% for theorem-like environments
\usepackage{mdframed}%can make highlighted boxes of text
%Use case: https://tex.stackexchange.com/questions/46828/how-to-highlight-important-parts-with-a-gray-background
\usepackage{wrapfig}
\usepackage{centernot}
\usepackage{subcaption}%\begin{subfigure}{0.5\textwidth}
\usepackage{pgfplots}
\pgfplotsset{compat=1.13}
\usepackage[colorinlistoftodos]{todonotes}
\usepackage[colorlinks=true, allcolors=blue]{hyperref}
\usepackage{xfrac}					%to make slanted fractions \sfrac{numerator}{denominator}
\usepackage{enumitem}            
    %syntax: \begin{enumerate}[label=(\alph*)]
    %possible arguments: f \alph*, \Alph*, \arabic*, \roman* and \Roman*
\usetikzlibrary{arrows,shapes.geometric,fit}

\DeclareMathAlphabet{\pazocal}{OMS}{zplm}{m}{n}
%% Use \pazocal{letter} to typeset a letter in the other kind 
%%  of math calligraphic font. 

%% This puts the QED block at the end of each proof, the way I like it. 
\renewenvironment{proof}{{\bfseries Proof}}{\qed}
\makeatletter
\renewenvironment{proof}[1][\bfseries \proofname]{\par
  \pushQED{\qed}%
  \normalfont \topsep6\p@\@plus6\p@\relax
  \trivlist
  %\itemindent\normalparindent
  \item[\hskip\labelsep
        \scshape
    #1\@addpunct{}]\ignorespaces
}{%
  \popQED\endtrivlist\@endpefalse
}
\makeatother

%% This adds a \rewnewtheorem command, which enables me to override the settings for theorems contained in this document.
\makeatletter
\def\renewtheorem#1{%
  \expandafter\let\csname#1\endcsname\relax
  \expandafter\let\csname c@#1\endcsname\relax
  \gdef\renewtheorem@envname{#1}
  \renewtheorem@secpar
}
\def\renewtheorem@secpar{\@ifnextchar[{\renewtheorem@numberedlike}{\renewtheorem@nonumberedlike}}
\def\renewtheorem@numberedlike[#1]#2{\newtheorem{\renewtheorem@envname}[#1]{#2}}
\def\renewtheorem@nonumberedlike#1{  
\def\renewtheorem@caption{#1}
\edef\renewtheorem@nowithin{\noexpand\newtheorem{\renewtheorem@envname}{\renewtheorem@caption}}
\renewtheorem@thirdpar
}
\def\renewtheorem@thirdpar{\@ifnextchar[{\renewtheorem@within}{\renewtheorem@nowithin}}
\def\renewtheorem@within[#1]{\renewtheorem@nowithin[#1]}
\makeatother

%% This makes theorems and definitions with names show up in bold, the way I like it. 
\makeatletter
\def\th@plain{%
  \thm@notefont{}% same as heading font
  \itshape % body font
}
\def\th@definition{%
  \thm@notefont{}% same as heading font
  \normalfont % body font
}
\makeatother

%===============================================
%==============Shortcut Commands================
%===============================================
\newcommand{\ds}{\displaystyle}
\newcommand{\B}{\mathcal{B}}
\newcommand{\C}{\mathbb{C}}
\newcommand{\F}{\mathbb{F}}
\newcommand{\N}{\mathbb{N}}
\newcommand{\R}{\mathbb{R}}
\newcommand{\Q}{\mathbb{Q}}
\newcommand{\T}{\mathcal{T}}
\newcommand{\Z}{\mathbb{Z}}
\renewcommand\qedsymbol{$\blacksquare$}
\newcommand{\qedwhite}{\hfill\ensuremath{\square}}
\newcommand*\conj[1]{\overline{#1}}
\newcommand*\closure[1]{\overline{#1}}
\newcommand*\mean[1]{\overline{#1}}
%\newcommand{\inner}[1]{\left< #1 \right>}
\newcommand{\inner}[2]{\left< #1, #2 \right>}
\newcommand{\powerset}[1]{\pazocal{P}(#1)}
%% Use \pazocal{letter} to typeset a letter in the other kind 
%%  of math calligraphic font. 
\newcommand{\cardinality}[1]{\left| #1 \right|}
\newcommand{\domain}[1]{\mathcal{D}(#1)}
\newcommand{\image}{\text{Im}}
\newcommand{\inv}[1]{#1^{-1}}
\newcommand{\preimage}[2]{#1^{-1}\left(#2\right)}
\newcommand{\script}[1]{\mathcal{#1}}


\newenvironment{highlight}{\begin{mdframed}[backgroundcolor=gray!20]}{\end{mdframed}}

\DeclarePairedDelimiter\ceil{\lceil}{\rceil}
\DeclarePairedDelimiter\floor{\lfloor}{\rfloor}

%===============================================
%===============My Tikz Commands================
%===============================================
\newcommand{\drawsquiggle}[1]{\draw[shift={(#1,0)}] (.005,.05) -- (-.005,.02) -- (.005,-.02) -- (-.005,-.05);}
\newcommand{\drawpoint}[2]{\draw[*-*] (#1,0.01) node[below, shift={(0,-.2)}] {#2};}
\newcommand{\drawopoint}[2]{\draw[o-o] (#1,0.01) node[below, shift={(0,-.2)}] {#2};}
\newcommand{\drawlpoint}[2]{\draw (#1,0.02) -- (#1,-0.02) node[below] {#2};}
\newcommand{\drawlbrack}[2]{\draw (#1+.01,0.02) --(#1,0.02) -- (#1,-0.02) -- (#1+.01,-0.02) node[below, shift={(-.01,0)}] {#2};}
\newcommand{\drawrbrack}[2]{\draw (#1-.01,0.02) --(#1,0.02) -- (#1,-0.02) -- (#1-.01,-0.02) node[below, shift={(+.01,0)}] {#2};}

%***********************************************
%**************Start of Document****************
%***********************************************

%===============================================
%===============Theorem Styles==================
%===============================================

%================Default Style==================
\theoremstyle{plain}% is the default. it sets the text in italic and adds extra space above and below the \newtheorems listed below it in the input. it is recommended for theorems, corollaries, lemmas, propositions, conjectures, criteria, and (possibly; depends on the subject area) algorithms.
\newtheorem{theorem}{Theorem}
\numberwithin{theorem}{section} %This sets the numbering system for theorems to number them down to the {argument} level. I have it set to number down to the {section} level right now.
\newtheorem*{theorem*}{Theorem} %Theorem with no numbering
\newtheorem{corollary}[theorem]{Corollary}
\newtheorem*{corollary*}{Corollary}
\newtheorem{conjecture}[theorem]{Conjecture}
\newtheorem{lemma}[theorem]{Lemma}
\newtheorem*{lemma*}{Lemma}
\newtheorem{proposition}[theorem]{Proposition}
\newtheorem*{proposition*}{Proposition}
\newtheorem{problemstatement}[theorem]{Problem Statement}


%==============Definition Style=================
\theoremstyle{definition}% adds extra space above and below, but sets the text in roman. it is recommended for definitions, conditions, problems, and examples; i've alse seen it used for exercises.
\newtheorem{definition}[theorem]{Definition}
\newtheorem*{definition*}{Definition}
\newtheorem{condition}[theorem]{Condition}
\newtheorem{problem}[theorem]{Problem}
\newtheorem{example}[theorem]{Example}
\newtheorem*{example*}{Example}
\newtheorem*{counterexample*}{Counterexample}
\newtheorem*{romantheorem*}{Theorem} %Theorem with no numbering
\newtheorem{exercise}{Exercise}
\numberwithin{exercise}{section}
\newtheorem{algorithm}[theorem]{Algorithm}

%================Remark Style===================
\theoremstyle{remark}% is set in roman, with no additional space above or below. it is recommended for remarks, notes, notation, claims, summaries, acknowledgments, cases, and conclusions.
\newtheorem{remark}[theorem]{Remark}
\newtheorem*{remark*}{Remark}
\newtheorem{notation}[theorem]{Notation}
\newtheorem*{notation*}{Notation}
%\newtheorem{claim}[theorem]{Claim}  %%use this if you ever want claims to be numbered
\newtheorem*{claim}{Claim}



\pgfplotsset{compat=1.13}

%\newcommand{\T}{\mathcal{T}}
%\newcommand{\B}{\mathcal{B}}

%These commands are now in tskpreamble_nothms.tex, but are left as a comment here for reference.
%\newcommand{\arbcup}[1]{\bigcup\limits_{\alpha\in\Gamma}#1_\alpha}
%\newcommand{\arbcap}[1]{\bigcap\limits_{\alpha\in\Gamma}#1_\alpha}
%\newcommand{\arbcoll}[1]{\{#1_\alpha\}_{\alpha\in\Gamma}}
%\newcommand{\arbprod}[1]{\prod\limits_{\alpha\in\Gamma}#1_\alpha}
%\newcommand{\finitecoll}[1]{#1_1, \ldots, #1_n}
%\newcommand{\finitefuncts}[2]{#1(#2_1), \ldots, #1(#2_n)}
%\newcommand{\abs}[1]{\left|#1\right|}
%\newcommand{\norm}[1]{\left|\left|#1\right|\right|}

\title{Math 460 \linebreak
Homework 3}
\author{Trevor Klar}

\begin{document}

\maketitle

\begin{enumerate}
  \item Show that $\langle2-i\rangle$ (that is, the ideal generated by $2 - i$) is maximal in $\Z[i]$ by following these steps:
  	\begin{enumerate}[label=\alph*.]
  	\item Define a map $\phi:\Z\to\Z[i]/\langle2-i\rangle$ by $\phi(n)=n+\langle2-i\rangle$. Show $\phi$ is a ring homomorphism.
  	\begin{proof}\mbox{}
  		\begin{itemize}
  		\item Since $\phi(1)=1+\langle2-i\rangle$, then $\phi$ maps unity to unity.
  		\item Let $a, b \in \Z$. Then
  		$$\phi(a+b)=(a+b)+\langle2-i\rangle=\big(a+\langle2-i\rangle\big)+\big(b+\langle2-i\rangle\big)=\phi(a)+\phi(b).$$
  		\item Let $a, b \in \Z$. Then
  		$$\phi(ab)=(ab)+\langle2-i\rangle=\big(a+\langle2-i\rangle\big)\big(b+\langle2-i\rangle\big)=\phi(a)\phi(b).$$
  		\end{itemize}
  		Thus $\phi$ is a ring homomorphism.
  	\end{proof}

  	\item Now show $\phi$ is onto.
  	\begin{proof}
  		Let $(a+bi)+\langle2-i\rangle\in \Z[i]/\langle2-i\rangle$ be given. Choose $n=a+2b$. This means that
  		$$\phi(n)=\phi(a+2b)=(a+2b)+\langle2-i\rangle=(a+bi)+\langle2-i\rangle.$$
  		To see that this last equality holds, observe that
  		$$(a+2b)-(a+bi)=2b-bi=b(2-i)\in\langle2-i\rangle.$$
  		Therefore we can produce an integer which $\phi$ maps to any element of $\Z[i]/\langle2-i\rangle$, so $\phi$ is onto.
  	\end{proof}

  	\item Show $\ker\phi=5\Z$.
  	\begin{proof} \textbf{($\ker\phi\supseteq5\Z$)}
  %	We will show both directions at once.
  	Observe that $2+\langle2-i\rangle=i+\langle2-i\rangle$. Then for all $5n\in5\Z$, %and $k\in\ker\phi$,
  	\[\begin{array}{rcl}
  	\phi(5n)&=&5n+\langle2-i\rangle\\
  	&=&(2^2+1)n+\langle2-i\rangle\\
  	&=&(i^2+1)n+\langle2-i\rangle\\
  	&=&\langle2-i\rangle\\
  %	&=&\phi(k).
  	\end{array}\]
  	Thus, $5\Z\subseteq\ker\phi$.
  	\end{proof}
  	\begin{proof}\textbf{($\ker\phi\subseteq5\Z$)} Let $k\in\ker\phi$ be given. Then $k\in\langle2-i\rangle$, so there exists $a,b\in\Z$ such that
  	$$k=(a+bi)(2-i)=(2a+b)+(-a+2b)i,$$
  	which implies that $a=2b$, and $k=2a+b$. Thus
  	\[\begin{array}{rcl}
  	k&=&2a+b\\
  	&=&2(2b)+b\\
  	&=&5b.\\
  	\end{array}\]
  	Therefore, for all $k\in\ker\phi$, we can find $b\in\Z$ such that $k=5b$, so $\ker\phi\subseteq5\Z$.
  	\end{proof}

  	\item Now, use the FHT. \\
  	Therefore, since $\phi:\Z\to\Z[i]/\langle2-i\rangle$ is onto and $\ker\phi=5\Z$, then
  	$$\Z/5\Z=\Z_5\cong\Z[i]/\langle2-i\rangle$$
  	by the FHT. Since $\Z_5$ is a field, then so is $\Z[i]/\langle2-i\rangle$, which means that $\langle2-i\rangle$ is a maximal ideal in $\Z[i]$. \qed
  	\end{enumerate}

  \item Show that $\langle3-i\rangle$ (that is, the ideal generated by $3 - i$) is not prime in $\Z[i]$ by following these steps:
	\begin{enumerate}[label=\alph*.]
  	\item Define $\phi:\Z\to\Z[i]/\langle3-i\rangle$ by $\phi(n)=n+\langle3-i\rangle$. Show $\phi$ is an onto ring homomorphism.
  	\begin{proof}\mbox{}
    	\begin{itemize}
    		\item Since $\phi(1)=1+\langle3-i\rangle$, then $\phi$ maps unity to unity.
    		\item Let $a, b \in \Z$. Then
    		$$\phi(a+b)=(a+b)+\langle3-i\rangle=\big(a+\langle3-i\rangle\big)+\big(b+\langle3-i\rangle\big)=\phi(a)+\phi(b).$$
    		\item Let $a, b \in \Z$. Then
    		$$\phi(ab)=(ab)+\langle3-i\rangle=\big(a+\langle3-i\rangle\big)\big(b+\langle3-i\rangle\big)=\phi(a)\phi(b).$$
    		\item Let $(a+bi)+\langle3-i\rangle\in\Z[i]/\langle3-i\rangle$ be given. Choose $n=a+3b$. Then since $$(a+3b)-(a+bi)=b(3-i)\in\langle3-i\rangle,$$ then $$\phi(n)=(a+3b)+\langle3-i\rangle=(a+bi)+\langle3-i\rangle.$$
    	\end{itemize}
    	Therefore, $\phi$ is an onto homomorphism.
  	\end{proof}
    \item Show $\ker\phi=10\Z$.
    \begin{proof}\textbf{($\ker\phi\supseteq10\Z$)} 
    	Let $10n\in10\Z$ be given. Observe that $3+\langle3-i\rangle=i+\langle3-i\rangle$. Then 
    	\[\begin{array}{rcl}
    	\phi(10n)&=&10n+\langle3-i\rangle\\
    	&=&(3^2+1)n+\langle3-i\rangle\\
    	&=&(i^2+1)n+\langle3-i\rangle\\
    	&=&\langle3-i\rangle\\
    	\end{array}\]
			Thus $10\Z\subseteq\ker\phi$.
    \end{proof}
    \begin{proof}\textbf{($\ker\phi\subseteq10\Z$)} 
    	Let $k\in\ker\phi$ be given. Then $k\in\langle3-i\rangle$, which means that there exists $a,b\in\Z$ such that 
    	$$k=(a+bi)(3-i)=(3a+b)+(3b-a)i,$$
    	which means that $a=3b$ and $k=3a+b$. Thus
    	\[\begin{array}{rcl}
   		k&=&3a+b\\
   		&=&3(3b)+b\\
   		&=&10b\\
    	\end{array}\]
			Therefore, $\ker\phi\subseteq10\Z$. 
    \end{proof}
  	\item By FHT, $$\Z/10\Z=\Z_{10}\cong\Z[i]/\langle3-i\rangle,$$ so $\langle3-i\rangle$ is not prime.
  	\begin{proof}
  		If $\langle3-i\rangle$ were prime, then $\Z[i]/\langle3-i\rangle$ would be an integral domain. However, $\Z_{10}$ has zero divisors ($2\times5=0$), so neither ring is an integral domain, so $\langle3-i\rangle$ is not prime.
  	\end{proof}
  \end{enumerate}
	\pagebreak  
  \item Define $N:\Z[\sqrt{6}]\to\Z$ by $N\big(a+b\sqrt{6}\big) = a^2-6b^2.$
  \begin{enumerate}[label=\alph*.]
  	\item Show that $N$ is multiplicative, i.e. $N(xy)=N(x)N(y)$ for all $x,y\in\Z[\sqrt{6}].$
  	\begin{proof}
  		Let $x,y\in\Z[\sqrt{6}]$ be given. Then we can write $x=a+b\sqrt{6}$ and $y=c+d\sqrt{6}$. Then 
  		\[\begin{array}{rcl}
  			N(xy)&=&N\big((a+b\sqrt{6})(c+d\sqrt{6})\big)\\
  			&=&N\big((ac+6bd)+(ad+bc)\sqrt{6}\big)\\
  			&=&a^2c^2+12abcd+36b^2d^2 - 6a^2d^2-12abcd-6b^2c^2\\
  			&=&a^2c^2-6a^2d^2-6b^2c^2+36b^2d^2\\
  			&=&(a^2-6b^2)(c^2-6d^2)\\
  			&=&N(a+b\sqrt{6})N(c+d\sqrt{6})\\
  			&=&N(x)N(y)
  		\end{array}\]
  		and we are done.
  	\end{proof}
  	\item Use $N$ to explain why the only invertible elements in $\Z[\sqrt{6}]$ have $N$ of 1.
  	\begin{proof}\textbf{($\implies$)}
  		Let $x\in\Z[\sqrt{6}]$ be invertible. Then there exists $y\in\Z[\sqrt{6}]$ such that $xy=1$. Taking $N$ of both sides, we find that
  		$$N(x)N(y)=N(xy)=N(1)=1.$$
  		Since $N(x),N(y)\in\Z$, then $N(x)=N(y)=\pm1$. This means that if $x=a+b\sqrt{6}$, then 
  		$$a^2-6b^2=\pm1.$$
  		This second-order Diophantine equation has no solutions for the negative case, but infinitely many for the positive, i.e. $N(1)=N(-1)=N(5+2\sqrt6)=N(5-2\sqrt6)=1$. Thus, any invertible element $x\in\Z[\sqrt{6}]$ is such that $N(x)=1$. 
  	\end{proof}
  	\begin{proof}\textbf{($\impliedby$)}
  		Let $x\in\Z[\sqrt{6}]$ with $N(x)=1$. Then if we write $x=(a+b\sqrt6)$, choose $y=(a-b\sqrt6)$. Then 
  		$$xy=(a+b\sqrt6)(a-b\sqrt6)=a^2-6b^2=N(x)=1,$$
  		so $xy=1$ and $x$ is invertible. 
  	\end{proof}
  	\item Show that $\sqrt6$ is not an irreducible element in $\Z[\sqrt6]$ by writing it as a product of two non-invertible elements in $\Z[\sqrt6]$. 
  	\begin{proof}
  		Since $N(\sqrt{6})=-6$, we have that $\sqrt6$ is nonzero and not a unit. However, 
  		$$(2+\sqrt6)(3-\sqrt6)=\sqrt6,$$ so $\sqrt6$ is reducible. To confirm this, we check that $(2+\sqrt6)$ and $(3-\sqrt6)$ are non-units. $N(2+\sqrt6)=-2$ and $N(3-\sqrt6)=3$, and we know that $N(x)=1$ for all $x\in U\big(\Z[\sqrt{6}]\big)$, so we are done. 
  	\end{proof}
  	\item Prove that $(1+\sqrt6)$ is irreducible in $\Z[\sqrt6]$. 
  	\begin{proof}
  		Suppose $x, y\in \Z[\sqrt6]$ such that $xy=(1+\sqrt6)$. Then 
  		$$N(x)N(y)=N(1+\sqrt6)=-5.$$
  		Since $N(x),N(y)\in\Z$, Then $N(x),N(y)$ are 1,-5 or -1,5. We know already that there are no elements of $\Z[\sqrt6]$ with $N$ of -1, so that means that either $N(x)=1$ or $N(y)=1$, and thus one of them is a unit. 
  	\end{proof}
  	\item Prove that $(1+\sqrt6)$ is prime in $\Z[\sqrt6]$. 
  	\begin{proof}
	  	Suppose $(1+\sqrt6)=ab$ for some $a,b\in\Z[\sqrt6]$. Since $(1+\sqrt6)$ is irreducible, then either $a$ or $b$ is a unit. If $a$ is invertible, then $a^{-1}(1+\sqrt6)=a^{-1}ab=b$, so $(1+\sqrt6)|b$. Otherwise if $b$ is invertible, then $(1+\sqrt6)b^{-1}=abb^{-1}=a$, so $(1+\sqrt6)|a$.
  	\end{proof}
  \end{enumerate}
	\pagebreak  
  \item Show that the domains $\Z[\sqrt{-6}]$ and $\Z[\sqrt{-7}]$ are not UFDs. Just look at how we did $\Z[\sqrt{-3}]$ in class.
  \textbf{Lemma} The only units of $\Z[\sqrt{-n}]$ where $1<n\in\Z$ are $\pm1$. 
  \begin{proof}
  	Consider $\eta:\Z[\sqrt{-n}]\to\N$ defined by $\eta(a+b\sqrt{-n})=a^2+nb^2$. $\eta$ is multiplicative for the same reasons as in (3a), so for any units $x,y$ we have that $\eta(x)\eta(y)=\eta(1)=1$. If we write $x=(a+b\sqrt{-n})$, then 
  	$$\eta(x)=a^2+nb^2=1,$$
  	which can only hold for $b=0$, $a=\pm1$ (since $a,b\in\Z$). 
  \end{proof}
 	Now we prove that $\Z[\sqrt{-6}]$ and $\Z[\sqrt{-7}]$ are not UFDs. 
 	\begin{enumerate}[label=\alph*.]
 		\item \textbf{Claim.} $\Z[\sqrt{-6}]$ is not a UFD, because 
 		$(2+\sqrt{-6})(2-\sqrt{-6})=10=(5)(2)$
 		and all of $2, 5, (2\pm\sqrt{-6})$ are irreducible. 
 		\begin{proof}
 			To see that $(2+\sqrt{-6})$ is irreducible, observe that $(2+\sqrt{-6})\neq\pm1$ and thus is not a unit and nonzero. Suppose $xy=(2+\sqrt{-6})$ where $x,y\in\Z[\sqrt{-6}]$ are not units. Then $\eta(x)\eta(y)=\eta(2+\sqrt{-6})=10$, so 
 			$$\eta(x)=(a^2+6b^2)=5,2.$$
 			This has no solutions, since $(a^2+6b^2)>5$ for $b\neq0$ and $5,2$ are not square numbers. By the same argument, $(2-\sqrt{-6})$ is irreducible as well. \\
 			A similar argument shows that $5$ is irreducible. Observe that $5\neq\pm1$ and thus is not a unit and nonzero. Suppose $xy=$ where $x,y\in\Z[\sqrt{-6}]$ are not units. Since $\eta(5)=\eta(5+6\sqrt{-6})=25$, then 
 			$$\eta(x)=(a^2+6b^2)=5, $$
 			and we have already seen that this has no solutions. The same argument shows that $2$ is also irreducible.
 		\end{proof}
 		\item \textbf{Claim.} $\Z[\sqrt{-7}]$ is not a UFD, because $(1+\sqrt{-7})(1-\sqrt{-7})=8=2^3$ and all of $2, (1\pm\sqrt{-7})$ are irreducible. 
		\begin{proof}
	 		We use a similar proof as in (4a), and omit some notation. To see that $(1+\sqrt{-7})$ is irreducible, observe that $(1+\sqrt{-7})\neq\pm1$ and $\eta(x)\eta(y)=8$, so 
 			$$\eta(x)=(a^2+7b^2)=2,4. $$
 			Though $(2+0\sqrt{-7})$ is a solution to $\eta(x)=4$, there are no solutions to $\eta(x)=2$, so there is no such $x\in\Z[\sqrt{-7}]$ such that $2x=(1+\sqrt{-7})$. Thus we conclude that $(1+\sqrt{-7})$ is irreducible, and so is $(1-\sqrt{-7})$, since it has the same $\eta$. Also $2$ is irreducible by the same reasoning as in the previous problem. 
 		\end{proof}		 			
 	\end{enumerate}
\end{enumerate}
\end{document}
