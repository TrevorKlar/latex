\documentclass[letterpaper]{article}
%\documentclass[a5paper]{article}

%% Language and font encodings
\usepackage[english]{babel}
\usepackage[utf8x]{inputenc}
\usepackage[T1]{fontenc}


%% Sets page size and margins
\usepackage[letterpaper,top=.75in,bottom=1in,left=1in,right=1in,marginparwidth=1.75cm]{geometry}
%\usepackage[a5paper,top=1cm,bottom=1cm,left=1cm,right=1.5cm,marginparwidth=1.75cm]{geometry}

\usepackage{graphicx}
%\graphicspath{../images}	  %%where to look for images

%% Useful packages
\usepackage{amssymb, amsmath, amsthm} 
%\usepackage{graphicx}  %%this is currently enabled in the default document, so it is commented out here. 
\usepackage{calrsfs}
\usepackage{braket}
\usepackage{mathtools}
\usepackage{lipsum}
\usepackage{tikz}
\usetikzlibrary{cd}
\usepackage{verbatim}
%\usepackage{ntheorem}% for theorem-like environments
\usepackage{mdframed}%can make highlighted boxes of text
%Use case: https://tex.stackexchange.com/questions/46828/how-to-highlight-important-parts-with-a-gray-background
\usepackage{wrapfig}
\usepackage{centernot}
\usepackage{subcaption}%\begin{subfigure}{0.5\textwidth}
\usepackage{pgfplots}
\pgfplotsset{compat=1.13}
\usepackage[colorinlistoftodos]{todonotes}
\usepackage[colorlinks=true, allcolors=blue]{hyperref}
\usepackage{xfrac}					%to make slanted fractions \sfrac{numerator}{denominator}
\usepackage{enumitem}            
    %syntax: \begin{enumerate}[label=(\alph*)]
    %possible arguments: f \alph*, \Alph*, \arabic*, \roman* and \Roman*
\usetikzlibrary{arrows,shapes.geometric,fit}

\DeclareMathAlphabet{\pazocal}{OMS}{zplm}{m}{n}
%% Use \pazocal{letter} to typeset a letter in the other kind 
%%  of math calligraphic font. 

%% This puts the QED block at the end of each proof, the way I like it. 
\renewenvironment{proof}{{\bfseries Proof}}{\qed}
\makeatletter
\renewenvironment{proof}[1][\bfseries \proofname]{\par
  \pushQED{\qed}%
  \normalfont \topsep6\p@\@plus6\p@\relax
  \trivlist
  %\itemindent\normalparindent
  \item[\hskip\labelsep
        \scshape
    #1\@addpunct{}]\ignorespaces
}{%
  \popQED\endtrivlist\@endpefalse
}
\makeatother

%% This adds a \rewnewtheorem command, which enables me to override the settings for theorems contained in this document.
\makeatletter
\def\renewtheorem#1{%
  \expandafter\let\csname#1\endcsname\relax
  \expandafter\let\csname c@#1\endcsname\relax
  \gdef\renewtheorem@envname{#1}
  \renewtheorem@secpar
}
\def\renewtheorem@secpar{\@ifnextchar[{\renewtheorem@numberedlike}{\renewtheorem@nonumberedlike}}
\def\renewtheorem@numberedlike[#1]#2{\newtheorem{\renewtheorem@envname}[#1]{#2}}
\def\renewtheorem@nonumberedlike#1{  
\def\renewtheorem@caption{#1}
\edef\renewtheorem@nowithin{\noexpand\newtheorem{\renewtheorem@envname}{\renewtheorem@caption}}
\renewtheorem@thirdpar
}
\def\renewtheorem@thirdpar{\@ifnextchar[{\renewtheorem@within}{\renewtheorem@nowithin}}
\def\renewtheorem@within[#1]{\renewtheorem@nowithin[#1]}
\makeatother

%% This makes theorems and definitions with names show up in bold, the way I like it. 
\makeatletter
\def\th@plain{%
  \thm@notefont{}% same as heading font
  \itshape % body font
}
\def\th@definition{%
  \thm@notefont{}% same as heading font
  \normalfont % body font
}
\makeatother

%===============================================
%==============Shortcut Commands================
%===============================================
\newcommand{\ds}{\displaystyle}
\newcommand{\B}{\mathcal{B}}
\newcommand{\C}{\mathbb{C}}
\newcommand{\F}{\mathbb{F}}
\newcommand{\N}{\mathbb{N}}
\newcommand{\R}{\mathbb{R}}
\newcommand{\Q}{\mathbb{Q}}
\newcommand{\T}{\mathcal{T}}
\newcommand{\Z}{\mathbb{Z}}
\renewcommand\qedsymbol{$\blacksquare$}
\newcommand{\qedwhite}{\hfill\ensuremath{\square}}
\newcommand*\conj[1]{\overline{#1}}
\newcommand*\closure[1]{\overline{#1}}
\newcommand*\mean[1]{\overline{#1}}
%\newcommand{\inner}[1]{\left< #1 \right>}
\newcommand{\inner}[2]{\left< #1, #2 \right>}
\newcommand{\powerset}[1]{\pazocal{P}(#1)}
%% Use \pazocal{letter} to typeset a letter in the other kind 
%%  of math calligraphic font. 
\newcommand{\cardinality}[1]{\left| #1 \right|}
\newcommand{\domain}[1]{\mathcal{D}(#1)}
\newcommand{\image}{\text{Im}}
\newcommand{\inv}[1]{#1^{-1}}
\newcommand{\preimage}[2]{#1^{-1}\left(#2\right)}
\newcommand{\script}[1]{\mathcal{#1}}


\newenvironment{highlight}{\begin{mdframed}[backgroundcolor=gray!20]}{\end{mdframed}}

\DeclarePairedDelimiter\ceil{\lceil}{\rceil}
\DeclarePairedDelimiter\floor{\lfloor}{\rfloor}

%===============================================
%===============My Tikz Commands================
%===============================================
\newcommand{\drawsquiggle}[1]{\draw[shift={(#1,0)}] (.005,.05) -- (-.005,.02) -- (.005,-.02) -- (-.005,-.05);}
\newcommand{\drawpoint}[2]{\draw[*-*] (#1,0.01) node[below, shift={(0,-.2)}] {#2};}
\newcommand{\drawopoint}[2]{\draw[o-o] (#1,0.01) node[below, shift={(0,-.2)}] {#2};}
\newcommand{\drawlpoint}[2]{\draw (#1,0.02) -- (#1,-0.02) node[below] {#2};}
\newcommand{\drawlbrack}[2]{\draw (#1+.01,0.02) --(#1,0.02) -- (#1,-0.02) -- (#1+.01,-0.02) node[below, shift={(-.01,0)}] {#2};}
\newcommand{\drawrbrack}[2]{\draw (#1-.01,0.02) --(#1,0.02) -- (#1,-0.02) -- (#1-.01,-0.02) node[below, shift={(+.01,0)}] {#2};}

%***********************************************
%**************Start of Document****************
%***********************************************

%===============================================
%===============Theorem Styles==================
%===============================================

%================Default Style==================
\theoremstyle{plain}% is the default. it sets the text in italic and adds extra space above and below the \newtheorems listed below it in the input. it is recommended for theorems, corollaries, lemmas, propositions, conjectures, criteria, and (possibly; depends on the subject area) algorithms.
\newtheorem{theorem}{Theorem}
\numberwithin{theorem}{section} %This sets the numbering system for theorems to number them down to the {argument} level. I have it set to number down to the {section} level right now.
\newtheorem*{theorem*}{Theorem} %Theorem with no numbering
\newtheorem{corollary}[theorem]{Corollary}
\newtheorem*{corollary*}{Corollary}
\newtheorem{conjecture}[theorem]{Conjecture}
\newtheorem{lemma}[theorem]{Lemma}
\newtheorem*{lemma*}{Lemma}
\newtheorem{proposition}[theorem]{Proposition}
\newtheorem*{proposition*}{Proposition}
\newtheorem{problemstatement}[theorem]{Problem Statement}


%==============Definition Style=================
\theoremstyle{definition}% adds extra space above and below, but sets the text in roman. it is recommended for definitions, conditions, problems, and examples; i've alse seen it used for exercises.
\newtheorem{definition}[theorem]{Definition}
\newtheorem*{definition*}{Definition}
\newtheorem{condition}[theorem]{Condition}
\newtheorem{problem}[theorem]{Problem}
\newtheorem{example}[theorem]{Example}
\newtheorem*{example*}{Example}
\newtheorem*{counterexample*}{Counterexample}
\newtheorem*{romantheorem*}{Theorem} %Theorem with no numbering
\newtheorem{exercise}{Exercise}
\numberwithin{exercise}{section}
\newtheorem{algorithm}[theorem]{Algorithm}

%================Remark Style===================
\theoremstyle{remark}% is set in roman, with no additional space above or below. it is recommended for remarks, notes, notation, claims, summaries, acknowledgments, cases, and conclusions.
\newtheorem{remark}[theorem]{Remark}
\newtheorem*{remark*}{Remark}
\newtheorem{notation}[theorem]{Notation}
\newtheorem*{notation*}{Notation}
%\newtheorem{claim}[theorem]{Claim}  %%use this if you ever want claims to be numbered
\newtheorem*{claim}{Claim}



\pgfplotsset{compat=1.13}

%\newcommand{\T}{\mathcal{T}}
%\newcommand{\B}{\mathcal{B}}

%These commands are now in tskpreamble_nothms.tex, but are left as a comment here for reference.
%\newcommand{\arbcup}[1]{\bigcup\limits_{\alpha\in\Gamma}#1_\alpha}
%\newcommand{\arbcap}[1]{\bigcap\limits_{\alpha\in\Gamma}#1_\alpha}
%\newcommand{\arbcoll}[1]{\{#1_\alpha\}_{\alpha\in\Gamma}}
%\newcommand{\arbprod}[1]{\prod\limits_{\alpha\in\Gamma}#1_\alpha}
%\newcommand{\finitecoll}[1]{#1_1, \ldots, #1_n}
%\newcommand{\finitefuncts}[2]{#1(#2_1), \ldots, #1(#2_n)}
%\newcommand{\abs}[1]{\left|#1\right|}
%\newcommand{\norm}[1]{\left|\left|#1\right|\right|}

\title{Math 460 \linebreak
Homework 4}
\author{Trevor Klar}

\begin{document}

\maketitle

\begin{enumerate}
	\item Show	$f(x) = x^4 + x^3 + 1$ is irreducible over $\Q$, then answer the following:
	\begin{proof}
		$f(x)$ is irreducible by the Rational Root Theorem, since the only possible roots in $\Q$ are $\pm1$, and neither is a root of $f(x)$. 
	\end{proof}
		\begin{enumerate}[label=(\roman*)]
			\item If $u$ is a root of $f(x)$ in an extension field of $\Q$, determine $[\Q(u):\Q]$ and give a $\Q$-basis for $\Q(u)$. \\
			\textbf{Answer} Since $\deg f(x)=4$, then $[\Q(u):\Q]=4$ and $1, u, u^{2}, u^{3}$ is a $\Q$-basis for $\Q(u)$. 
			\item Express each of the following elements in terms of a basis (You should not have to solve for the scalars for these): $\inv{u}, \inv{(u^2)}, \inv{(u^3)}$. \\
			\textbf{Answer} $\inv{u}=-u^{3}-u^{2}, \quad \inv{(u^2)}=-u^{2}-u, \quad \inv{(u^3)}=-u-1$. 
			\item Express $\inv{(1-u)}$ as a linear combination of the basis elements (You will have to solve for the scalars for this).\\
			\textbf{Answer} Let's solve. 
				\[\begin{array}{rcl}
					1&=&(1-u)(a+bu+cu^2+du^3)\\
					&=&a+(b-a)u+(c-b)u^2+(d-c)u^3-du^4\\
					&=&a+(b-a)u+(c-b)u^2+(d-c)u^3+d(u^3+1)\\
					&=&(a+d)+(b-a)u+(c-b)u^2+(2d-c)u^3,\\
				\end{array}\]
				so $a=b=c$, thus $a+d=1$ and $2d=a$, which gives $a=b=c=\sfrac{2}{3}, d=\sfrac{1}{3}$. Thus, $\inv{(1-u)}=\frac{2}{3}+\frac{2}{3}u+\frac{2}{3}u^2+\frac{1}{3}u^3$.
		\end{enumerate}
	\item Determine whether the following polynomials are irreducible over the indicated fields. If irreducible, give a reason. If reducible, factor it into irreducible factors.
		\begin{enumerate}[label=(\roman*)]
			\item $x^{10}+2x+6, \Q$ \\
				\textbf{Answer} Irreducible by Eisenstein's Criterion, since $2$ divides 2 and 6 but not 1, and $2^2$ does not divide 6. 
			\item $x^4+2, \Z_3$ \\
				\textbf{Answer} Reducible. Since 1 is a root and $\Z_3$ is a field, then by the Division Algorithm $x^{4}+2=(x-1)q(x)+r(x)$, so $r(1)=0$ and $x-1=x+2$ is a factor. Synthetic division $\mod 3$ will find the following factorization: $x^{4}+2=(x+2)(x+1)(x^2+1)$
			\item $x^{4}+3x^{2}+1, \Q$ \\ 
				\textbf{Answer} Irreducible since it is positive for all real $x$, so it has no roots. Further, the Rational Root test gives $\pm 1$ as the only possible rational roots, and computation eliminates $(x\pm1)$ as factors.
			\item $x^{5}+5x^{3}+4$ \\ 
			 \textbf{Answer} Using the linear substitution $\phi\big(p(x)\big)=p(x+1)$, we find that $$\phi\big(x^{5}+5x^{3}+4\big)=x^{5}+5x^{4}+15x^{3}+25x^{2}+20x+10,$$
			 and since 5 divides all but the leading coefficient and 25 does not divide the constant term, we have that $x^{5}+5x^{3}+4$ is irreducible by Eisenstein's Criterion. 
			\item $x^{4}+x^{2}+1$ \\
				\textbf{Answer} Irreducible. It has no roots, so it has no linear or 3rd degree factors. It is quadratic in form, and the quadratic formula shows that it factors as $$\left (x^2+\frac{1}{3}+\frac{\sqrt{3}}{2}i\right )\left (x^2+\frac{1}{3}-\frac{\sqrt{3}}{2}i\right )$$ which can be further factored in the complex numbers, but the non-real complex numbers are closed under square roots. Thus, $x^{4}+x^{2}+1$ can be factored as $(x-z_1)(x-z_2)(x-z_3)(x-z_4)$, where all $z_i$ are non-real complex numbers. 
			\item $x^{4}+2x^2+3, \Z_5$ \\
				\textbf{Answer} Computation checks that 1 and 2 are not roots, and since the function is even, neither are 3 and 4. Though the polynomial is quadratic in form, it does not factor as $(ax^2+b)(cx^2+d)$, since no two elements of $\Z_5$ add to 2 and multiply to 3. If this polynomial does reduce, its factors as the product of two general quadratics, but I don't know how to calculate that other than a brutal system of equations which yielded nothing but pain, and after a bit of work, the attempt was abandoned. 
				
				One thing worth pointing out is that if we apply the linear substitution $x=x+1$, we find that the polynomial becomes $x^{4}+4x^3+8x^2+8x+6$, which is irreducible over $\Q$ by Eisenstein's Criterion. I think that this implies that it is irreducible over $\Z_5$ as well, but I'm not sure. 
 		\end{enumerate}
\end{enumerate}
\end{document}
