 \documentclass[12pt,letterpaper]{article}

\usepackage{fancyhdr,fancybox,tensor}

%% Useful packages
\usepackage{amssymb, amsmath, amsthm} 
%\usepackage{graphicx}  %%this is currently enabled in the default document, so it is commented out here. 
\usepackage{calrsfs}
\usepackage{braket}
\usepackage{mathtools}
\usepackage{lipsum}
\usepackage{tikz}
\usetikzlibrary{cd}
\usepackage{verbatim}
%\usepackage{ntheorem}% for theorem-like environments
\usepackage{mdframed}%can make highlighted boxes of text
%Use case: https://tex.stackexchange.com/questions/46828/how-to-highlight-important-parts-with-a-gray-background
\usepackage{wrapfig}
\usepackage{centernot}
\usepackage{subcaption}%\begin{subfigure}{0.5\textwidth}
\usepackage{pgfplots}
\pgfplotsset{compat=1.13}
\usepackage[colorinlistoftodos]{todonotes}
\usepackage[colorlinks=true, allcolors=blue]{hyperref}
\usepackage{xfrac}					%to make slanted fractions \sfrac{numerator}{denominator}
\usepackage{enumitem}            
    %syntax: \begin{enumerate}[label=(\alph*)]
    %possible arguments: f \alph*, \Alph*, \arabic*, \roman* and \Roman*
\usetikzlibrary{arrows,shapes.geometric,fit}

\DeclareMathAlphabet{\pazocal}{OMS}{zplm}{m}{n}
%% Use \pazocal{letter} to typeset a letter in the other kind 
%%  of math calligraphic font. 

%% This puts the QED block at the end of each proof, the way I like it. 
\renewenvironment{proof}{{\bfseries Proof}}{\qed}
\makeatletter
\renewenvironment{proof}[1][\bfseries \proofname]{\par
  \pushQED{\qed}%
  \normalfont \topsep6\p@\@plus6\p@\relax
  \trivlist
  %\itemindent\normalparindent
  \item[\hskip\labelsep
        \scshape
    #1\@addpunct{}]\ignorespaces
}{%
  \popQED\endtrivlist\@endpefalse
}
\makeatother

%% This adds a \rewnewtheorem command, which enables me to override the settings for theorems contained in this document.
\makeatletter
\def\renewtheorem#1{%
  \expandafter\let\csname#1\endcsname\relax
  \expandafter\let\csname c@#1\endcsname\relax
  \gdef\renewtheorem@envname{#1}
  \renewtheorem@secpar
}
\def\renewtheorem@secpar{\@ifnextchar[{\renewtheorem@numberedlike}{\renewtheorem@nonumberedlike}}
\def\renewtheorem@numberedlike[#1]#2{\newtheorem{\renewtheorem@envname}[#1]{#2}}
\def\renewtheorem@nonumberedlike#1{  
\def\renewtheorem@caption{#1}
\edef\renewtheorem@nowithin{\noexpand\newtheorem{\renewtheorem@envname}{\renewtheorem@caption}}
\renewtheorem@thirdpar
}
\def\renewtheorem@thirdpar{\@ifnextchar[{\renewtheorem@within}{\renewtheorem@nowithin}}
\def\renewtheorem@within[#1]{\renewtheorem@nowithin[#1]}
\makeatother

%% This makes theorems and definitions with names show up in bold, the way I like it. 
\makeatletter
\def\th@plain{%
  \thm@notefont{}% same as heading font
  \itshape % body font
}
\def\th@definition{%
  \thm@notefont{}% same as heading font
  \normalfont % body font
}
\makeatother

%===============================================
%==============Shortcut Commands================
%===============================================
\newcommand{\ds}{\displaystyle}
\newcommand{\B}{\mathcal{B}}
\newcommand{\C}{\mathbb{C}}
\newcommand{\F}{\mathbb{F}}
\newcommand{\N}{\mathbb{N}}
\newcommand{\R}{\mathbb{R}}
\newcommand{\Q}{\mathbb{Q}}
\newcommand{\T}{\mathcal{T}}
\newcommand{\Z}{\mathbb{Z}}
\renewcommand\qedsymbol{$\blacksquare$}
\newcommand{\qedwhite}{\hfill\ensuremath{\square}}
\newcommand*\conj[1]{\overline{#1}}
\newcommand*\closure[1]{\overline{#1}}
\newcommand*\mean[1]{\overline{#1}}
%\newcommand{\inner}[1]{\left< #1 \right>}
\newcommand{\inner}[2]{\left< #1, #2 \right>}
\newcommand{\powerset}[1]{\pazocal{P}(#1)}
%% Use \pazocal{letter} to typeset a letter in the other kind 
%%  of math calligraphic font. 
\newcommand{\cardinality}[1]{\left| #1 \right|}
\newcommand{\domain}[1]{\mathcal{D}(#1)}
\newcommand{\image}{\text{Im}}
\newcommand{\inv}[1]{#1^{-1}}
\newcommand{\preimage}[2]{#1^{-1}\left(#2\right)}
\newcommand{\script}[1]{\mathcal{#1}}


\newenvironment{highlight}{\begin{mdframed}[backgroundcolor=gray!20]}{\end{mdframed}}

\DeclarePairedDelimiter\ceil{\lceil}{\rceil}
\DeclarePairedDelimiter\floor{\lfloor}{\rfloor}

%===============================================
%===============My Tikz Commands================
%===============================================
\newcommand{\drawsquiggle}[1]{\draw[shift={(#1,0)}] (.005,.05) -- (-.005,.02) -- (.005,-.02) -- (-.005,-.05);}
\newcommand{\drawpoint}[2]{\draw[*-*] (#1,0.01) node[below, shift={(0,-.2)}] {#2};}
\newcommand{\drawopoint}[2]{\draw[o-o] (#1,0.01) node[below, shift={(0,-.2)}] {#2};}
\newcommand{\drawlpoint}[2]{\draw (#1,0.02) -- (#1,-0.02) node[below] {#2};}
\newcommand{\drawlbrack}[2]{\draw (#1+.01,0.02) --(#1,0.02) -- (#1,-0.02) -- (#1+.01,-0.02) node[below, shift={(-.01,0)}] {#2};}
\newcommand{\drawrbrack}[2]{\draw (#1-.01,0.02) --(#1,0.02) -- (#1,-0.02) -- (#1-.01,-0.02) node[below, shift={(+.01,0)}] {#2};}

%***********************************************
%**************Start of Document****************
%***********************************************
 %find me at /home/trevor/texmf/tex/latex/tskpreamble_nothms.tex
%===============================================
%===============Theorem Styles==================
%===============================================

%================Default Style==================
\theoremstyle{plain}% is the default. it sets the text in italic and adds extra space above and below the \newtheorems listed below it in the input. it is recommended for theorems, corollaries, lemmas, propositions, conjectures, criteria, and (possibly; depends on the subject area) algorithms.
\newtheorem{theorem}{Theorem}
\numberwithin{theorem}{section} %This sets the numbering system for theorems to number them down to the {argument} level. I have it set to number down to the {section} level right now.
\newtheorem*{theorem*}{Theorem} %Theorem with no numbering
\newtheorem{corollary}[theorem]{Corollary}
\newtheorem*{corollary*}{Corollary}
\newtheorem{conjecture}[theorem]{Conjecture}
\newtheorem{lemma}[theorem]{Lemma}
\newtheorem*{lemma*}{Lemma}
\newtheorem{proposition}[theorem]{Proposition}
\newtheorem*{proposition*}{Proposition}
\newtheorem{problemstatement}[theorem]{Problem Statement}


%==============Definition Style=================
\theoremstyle{definition}% adds extra space above and below, but sets the text in roman. it is recommended for definitions, conditions, problems, and examples; i've alse seen it used for exercises.
\newtheorem{definition}[theorem]{Definition}
\newtheorem*{definition*}{Definition}
\newtheorem{condition}[theorem]{Condition}
\newtheorem{problem}[theorem]{Problem}
\newtheorem{example}[theorem]{Example}
\newtheorem*{example*}{Example}
\newtheorem*{counterexample*}{Counterexample}
\newtheorem*{romantheorem*}{Theorem} %Theorem with no numbering
\newtheorem{exercise}{Exercise}
\numberwithin{exercise}{section}
\newtheorem{algorithm}[theorem]{Algorithm}

%================Remark Style===================
\theoremstyle{remark}% is set in roman, with no additional space above or below. it is recommended for remarks, notes, notation, claims, summaries, acknowledgments, cases, and conclusions.
\newtheorem{remark}[theorem]{Remark}
\newtheorem*{remark*}{Remark}
\newtheorem{notation}[theorem]{Notation}
\newtheorem*{notation*}{Notation}
%\newtheorem{claim}[theorem]{Claim}  %%use this if you ever want claims to be numbered
\newtheorem*{claim}{Claim}


%%
%% Page set-up:
%%
\pagestyle{empty}
\lhead{\textsc{201c - Functional Analysis} \\Quarter of COVID-19} 
\rhead{\textsc{Labutin, Spring 2020} \\ Trevor Klar}
%\chead{\Large\textbf{A New Integration Technique \\ }}
\renewcommand{\headrulewidth}{0pt}
%
\renewcommand{\footrulewidth}{0pt}
%\lfoot{
%Office: \quad \quad \, M 2-3 \, \, SH 6431x \\
%Math Lab: \, W 12-2 \, SH 1607
%}
%\rfoot{trevorklar@math.ucsb.edu}

\setlength{\parindent}{0in}
\setlength{\textwidth}{7in}
\setlength{\evensidemargin}{-0.25in}
\setlength{\oddsidemargin}{-0.25in}
\setlength{\parskip}{.5\baselineskip}
\setlength{\topmargin}{-0.5in}
\setlength{\textheight}{9in}

\setlist[enumerate,1]{label=\textbf{\arabic*.}}

\let\oldphi\phi
\renewcommand{\phi}{\varphi}
\renewcommand{\epsilon}{\varepsilon}
\let\wideclosure\closure
\renewcommand{\closure}{\bar}

\begin{document}
\pagestyle{fancy}
\begin{center}
{\Large Homework 2}%=================UPDATE THIS=================%
\end{center}

\begin{enumerate}
\item Prove that both weak and weak* topologies are Hausdorff. 

\begin{proof}
Let $x\neq y\in X$. Define $f(y-x)=1$ and extend linearly on the subspace spanned by $y-x$. By Hahn-Banach, we can extend $f$ to $\phi:X\to\R$ and find that $W_\frac{1}{2}(x;\phi)$ and $W_\frac{1}{2}(y;\phi)$ are disjoint weak neighborhoods which separate $x$ and $y$. Thus the weak topology is Hausdorff. \qedwhite

Let $\phi\neq\psi\in X^*$. Since they are not equal as functions, there is some $x\in X$ with $\phi(x)\neq\psi(x)$. Denoting $r=\phi(x)-\psi(x)$, then $W_\frac{1}{2}(\phi;x)$ and $W_\frac{1}{2}(\psi;x)$ are disjoint weak* neighborhoods which separate $\psi$ and $\phi$. Thus the weak* topology is Hausdorff.
\end{proof}

\renewcommand{\B}{\closure{B}(X)}
\item Let $X$ be Banach, and let $S$ be the unit sphere in $X$. Find the weak closure $\closure{S}^w$ of $S$. 

\answer $\closure{S}^w=\B$, the closed unit ball in $X$.

%\renewcommand{\B}{\closure{B}^X}
\renewcommand{\S}{\closure{S}^w}
\begin{proof}
Let $x\in \B$. Since any weak neighborhood $W$ of $x$ contains an infinite-dimensional hyperplane in $X$, then $W$ also contains a point $y$ of any magnitude greater than that of $x$, in particular there exists $y\in W$ with $\norm{y}=1$, so $y\in S$. Thus $x\in \S$. 

Next, let $x\not \in \B$. By Hahn-Banach there exist a functional $\phi$ which separates\footnote{That is, maps $x$ into $(1,\infty)$ and maps $\B$ into $(0,1)$. } the convex compact set $\{x\}$ from the convex closed set $\B$, so letting $r=\phi(x)-1$, we have $W_r(x;\phi)$ contains $x$ and is disjoint with $\B$, so $x\not\in \S$. 
\end{proof}

\item (i) Show that the set of all weak* neighborhoods
$$W(\phi;x_1, \dots, x_n)$$
forms a basis for a topology on $X^*$. 

(ii) Show that convergence of a sequence $(\phi_n)_{n=1}^\infty$ in this topology is equivalent to weak* convergence. 

\begin{proof}
(i) Denote the set of all weak* neighborhoods by $\script{B}$. Note that $\script{B}$ covers $X^*$ since $W(\phi;0)$ is the whole space for any $\phi\in X^*$. 

Let $W(\phi_1,x), W(\phi_2,y) \in \script{B}$. 

\jpg{width=0.3\textwidth}{hw2-p3-2}

For any $\psi$ in the intersection of these two weak* neighborhoods, 
\begin{align*}
\abs{\angles{\psi,x}-\angles{\phi_1,x}}&<1 \text{ and} \\
\abs{\angles{\psi,y}-\angles{\phi_2,y}}&<1,
\end{align*}
so if we denote $r$ as the smaller of the two quantities above, then 
$$\psi \in W_r(\psi;x,y) \subset W(\phi_1;x) \cap W(\phi_2;y).$$
Since this holds for two arbitrary weak* neighborhoods, then it holds for finitely many. Therefore $\script{B}$ is a basis for a topology. \qedwhite 


%\pagebreak
(ii) Denote convergence in the topology by $\phi_n\xto{T}\phi$. 

Suppose $\phi_n\xto{T}\phi$. Then by definition, for every $\epsilon>0$ and $x\in X$, there exists $N>0$ such that for all $n>N$, 
$$\phi_n\in W_\epsilon(\phi;x),$$
which is to say 
$$\abs{\angles{\phi_n,x}-\angles{\phi,x}}<\epsilon,$$
which statement is exactly the definition of 
$$\phi_n\xto{w^*}\phi.$$ 
This proof also works in reverse, so we are done.
\end{proof}

\item Let $(x_n)$ be a sequence in $\ell^1$ such that $x_n\xto{w}y$ and $\norm{x_n}_{\ell^1}\to\norm{y}_{\ell^1}$. Prove that $x_n\xto{\ell^1} y$. 


\begin{proof}
Let $\epsilon>0$. Since $\norm{x_n}\to\norm{y}$, there exists $N_1$ such that for all $n>N_1$, 
$$\left|\sum_{j=1}^\infty\abs{x_{nj}}-\sum_{j=1}^\infty\abs{y_j}\right|<\epsilon.$$
Since $\norm{y}<\infty$, then there exists $J$ such that 
$$\sum_{j=J}^\infty\abs{y_j}<\epsilon,$$ 
which means that for all $n>N_1$ we have $\sum_{j=J}^\infty\abs{x_{nj}}<2\epsilon$, so 
\begin{align}
\sum_{j=J}^\infty \abs{x_{nj}-y_j} &\leq \sum_{j=J}^\infty \abs{x_{nj}}+\abs{y_j} \nonumber \\
&< 3\epsilon.
\end{align}

Now observe that since $x_n\xto{w}y$, then in particular $\angles{x_n,e_j}\xto{n}\angles{y,e_j}$ where $e_j$ is the functional which simply returns the $j$-th coordinate. This means that	for all $j$ we have $ x_{nj}\xto{n}y_j$, so there exists some $M_j$ such that if $n>M_j$, we have $\abs{x_{nj}-y_j}<\epsilon$. Let $N_2=\max_{j\leq J} M_j$, then for all $n>N_2$,
\begin{equation}
\sum_{j=1}^J \abs{x_{nj}-y_j}<J\epsilon.
\end{equation}
Combining (1) and (2) yields 
$\sum_{j=1}^\infty \abs{x_{nj}-y_j}<(J+3)\epsilon,$
and after rescaling, we're done.
\end{proof}


\renewcommand{\B}{\closure{B}}
\item Prove that the closed unit ball $\B(X)$ in a Banach space $X$ is weakly closed. Prove that $\B(X^*)$ is weak* closed. 

\begin{proof}
%Use p2 and B-A.
We showed in problem 2 that the weak closure of $S(X)$ is $\B(X)$, so it is weakly closed. 

\renewcommand{\B}{\closure{B}(X^*)}
Now we show that $\B$ is weak* closed. Observe:
	\begin{itemize}
	\item $\B$ is weak* compact by Banach-Alaoglu.
	\item $\B$ is weak* Hausdorff. \\
	\textit{Proof} Let $\phi\neq\psi\in\B$. Since $\phi\neq\psi$ as functions on $X$, there exists $x\in X$ with $\angles{x,\phi}\neq\angles{x,\psi}$. Letting $a=\text{avg}(\angles{x,\phi},\angles{x,\psi})$ we have $\hat{x}\in X\subset X^{**}$ a linear functional on $X^*$ such that $\angles{x,\phi}<a$ and $\angles{x,\psi}>a$. Since $\hat{x}$ is weak* continuous by definition of the weak* topology, then $\preimage{\hat{x}}{-\infty,a}$ and $\preimage{\hat{x}}{a,\infty}$ are open sets which separate $\phi$ and $\psi$, so $\B$ is weak* Hausdorff. \qedwhite
	\item $\B$ is weak* closed. \\
	\textit{Proof} In this proof, all topological terms refer to the weak* topology. We will show that $\B^\complement=X^*\setminus\B$ is open. Let $\psi\not\in\B$. Since $\B$ is Hausdorff, for every $\phi_\alpha\in\B$, there exist open sets $U_\alpha, V_\alpha$ which separate $\phi_\alpha$ and $\psi$, respectively. Since $\{U_\alpha\}$ is an open cover of $\B$, it has a finite subcover $\{U_i\}$ with a corresponding finite collection of sets $\{V_i\}$. Since $\bigcap_i V_i\subset V_i$ for all $i$, then the intersection is disjoint with $\bigcup_i U_i$ which covers $\B$. Thus $\bigcap_{i=1}^N V_i\subset V_i$ is an open subset of $\B^\complement$ containing $\psi$, so we're done. \qedhere
	\end{itemize}
\end{proof}


\renewcommand{\B}{\closure{B}(X^*)}
\renewcommand{\phi}{\oldphi}
\newcommand{\dball}{\tensor[_d]{B}{_{r}}(0)}
\item Prove the statement from the lectures: Let $X$ be a separable Banach space with a dense set $U=\{u_n\}_{n\in\N}$.
%, and let $\B$ be the closed unit ball in $X^*$
Then the weak* topology restricted to $\B$ denoted $\sigma(\B,X)$, 
coincides with the topology of the metric
$$d(\phi,\psi)=\sum_{n=1}^\infty 2^{-n} \frac{\abs{(\phi-\psi)(u_n)}}{1+\abs{(\phi-\psi)(u_n)}}.$$

\begin{proof}
Since we are working in $\B$, then every functional has norm at most 1. Since the two topologies are both translation invariant, it suffices to show that $W(0;p)$ is open in the $d$-topology and that the $d$-ball $\tensor[_d]{B}{_{r}}(0)$ is open in the weak* topology. 

\pagebreak

Let $W(0,p)$ be an arbitrary subbasic weak* neighborhood in $\sigma(\B,X)$, centered at 0. Let 
$$\phi\in W(0,p).$$

We will produce a $d$-ball $\dball \subset W(0,p)$. 
Since $U$ is dense in $X$, then there exists some $u_N$ such that 
$$\norm{u_N-p}<\tfrac{1}{3}.$$
Let
$$r=\frac{1}{(3)(2^{N+1})}_.$$
Then if $\psi \in \dball$, then
$$\sum_{n=1}^\infty 2^{-n} \frac{\abs{(\psi)(u_n)}}{1+\abs{(\psi)(u_n)}} < r,$$
and since the whole sum is bounded by $r$, then in particular so is each term since they are all positive. Thus
\begin{alignat*}{3}
&&2^{-N} \frac{\abs{\psi(u_N)}}{1+\abs{\psi(u_N)}} & < \frac{1}{(3)(2^{N+1})} \\
&\implies &2{\abs{\psi(u_N)}} &< \frac{1+\abs{\psi(u_N)}}{3} \\
&\implies &{\abs{\psi(u_N)}} &< \frac{1}{5}.
\end{alignat*}
Now we observe that $\psi\in W(0,p)$:
\begin{align*}
|\psi(p)|&\leq |\psi(p-u_N)| + |\psi(u_N)| \\
&< \norm{\psi}\norm{u_N-p} + \frac{1}{5} \\
&< \frac{1}{3} + \frac{1}{5}\\
&< 1
\end{align*}
Thus the weak* topology is a subset of the $d$-topology. \qedwhite

Let $\dball$ be an arbitrary $d$-ball centered at 0. Then fix $\phi\in \dball$, and observe that
$$\norm{\phi}_d=\sum_{n=1}^\infty 2^{-n}\frac{\abs{\phi(u_n)}}{1+\abs{\phi(u_n)}}<r$$
and since this sum converges, there exists $N>0$ such that $\sum_{n=1}^\infty 2^{-n}\frac{\abs{\phi(u_n)}}{1+\abs{\phi(u_n)}}<\epsilon$ for all $\epsilon>0$. Let 
$$\delta=\min_{n\leq N}\left(\frac{r}{(N)2^{-n+1}-r}\right),$$
so that for all $n<N$, 
$$2^{-n}\frac{\delta}{1+\delta}<\frac{r}{2N}.$$
Consider $W_\delta(0;u_1, u_2, \dots, u_N).$ For any $\phi$ in this weak neighborhood, $\abs{\phi(u_n)}<\delta$ for all $n<N$, so 
\begin{align*}
\sum_{n=1}^\infty 2^{-n}\frac{\abs{\phi(u_n)}}{1+\abs{\phi(u_n)}}
&=\sum_{n=1}^N 2^{-n}\frac{\abs{\phi(u_n)}}{1+\abs{\phi(u_n)}}
  + \sum_{n=N}^\infty 2^{-n}\frac{\abs{\phi(u_n)}}{1+\abs{\phi(u_n)}}\\
&\leq \sum_{n=1}^N \frac{r}{2N}
  + \epsilon\\
&=\frac{r}{2}+\epsilon\\
&<r.
\end{align*}
Thus the $d$-topology is a subset of the weak* topology.
\end{proof}










\end{enumerate}
\end{document}
