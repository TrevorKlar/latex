\documentclass[12pt,letterpaper]{article}

\usepackage{fancyhdr,fancybox}

%% Useful packages
\usepackage{amssymb, amsmath, amsthm} 
%\usepackage{graphicx}  %%this is currently enabled in the default document, so it is commented out here. 
\usepackage{calrsfs}
\usepackage{braket}
\usepackage{mathtools}
\usepackage{lipsum}
\usepackage{tikz}
\usetikzlibrary{cd}
\usepackage{verbatim}
%\usepackage{ntheorem}% for theorem-like environments
\usepackage{mdframed}%can make highlighted boxes of text
%Use case: https://tex.stackexchange.com/questions/46828/how-to-highlight-important-parts-with-a-gray-background
\usepackage{wrapfig}
\usepackage{centernot}
\usepackage{subcaption}%\begin{subfigure}{0.5\textwidth}
\usepackage{pgfplots}
\pgfplotsset{compat=1.13}
\usepackage[colorinlistoftodos]{todonotes}
\usepackage[colorlinks=true, allcolors=blue]{hyperref}
\usepackage{xfrac}					%to make slanted fractions \sfrac{numerator}{denominator}
\usepackage{enumitem}            
    %syntax: \begin{enumerate}[label=(\alph*)]
    %possible arguments: f \alph*, \Alph*, \arabic*, \roman* and \Roman*
\usetikzlibrary{arrows,shapes.geometric,fit}

\DeclareMathAlphabet{\pazocal}{OMS}{zplm}{m}{n}
%% Use \pazocal{letter} to typeset a letter in the other kind 
%%  of math calligraphic font. 

%% This puts the QED block at the end of each proof, the way I like it. 
\renewenvironment{proof}{{\bfseries Proof}}{\qed}
\makeatletter
\renewenvironment{proof}[1][\bfseries \proofname]{\par
  \pushQED{\qed}%
  \normalfont \topsep6\p@\@plus6\p@\relax
  \trivlist
  %\itemindent\normalparindent
  \item[\hskip\labelsep
        \scshape
    #1\@addpunct{}]\ignorespaces
}{%
  \popQED\endtrivlist\@endpefalse
}
\makeatother

%% This adds a \rewnewtheorem command, which enables me to override the settings for theorems contained in this document.
\makeatletter
\def\renewtheorem#1{%
  \expandafter\let\csname#1\endcsname\relax
  \expandafter\let\csname c@#1\endcsname\relax
  \gdef\renewtheorem@envname{#1}
  \renewtheorem@secpar
}
\def\renewtheorem@secpar{\@ifnextchar[{\renewtheorem@numberedlike}{\renewtheorem@nonumberedlike}}
\def\renewtheorem@numberedlike[#1]#2{\newtheorem{\renewtheorem@envname}[#1]{#2}}
\def\renewtheorem@nonumberedlike#1{  
\def\renewtheorem@caption{#1}
\edef\renewtheorem@nowithin{\noexpand\newtheorem{\renewtheorem@envname}{\renewtheorem@caption}}
\renewtheorem@thirdpar
}
\def\renewtheorem@thirdpar{\@ifnextchar[{\renewtheorem@within}{\renewtheorem@nowithin}}
\def\renewtheorem@within[#1]{\renewtheorem@nowithin[#1]}
\makeatother

%% This makes theorems and definitions with names show up in bold, the way I like it. 
\makeatletter
\def\th@plain{%
  \thm@notefont{}% same as heading font
  \itshape % body font
}
\def\th@definition{%
  \thm@notefont{}% same as heading font
  \normalfont % body font
}
\makeatother

%===============================================
%==============Shortcut Commands================
%===============================================
\newcommand{\ds}{\displaystyle}
\newcommand{\B}{\mathcal{B}}
\newcommand{\C}{\mathbb{C}}
\newcommand{\F}{\mathbb{F}}
\newcommand{\N}{\mathbb{N}}
\newcommand{\R}{\mathbb{R}}
\newcommand{\Q}{\mathbb{Q}}
\newcommand{\T}{\mathcal{T}}
\newcommand{\Z}{\mathbb{Z}}
\renewcommand\qedsymbol{$\blacksquare$}
\newcommand{\qedwhite}{\hfill\ensuremath{\square}}
\newcommand*\conj[1]{\overline{#1}}
\newcommand*\closure[1]{\overline{#1}}
\newcommand*\mean[1]{\overline{#1}}
%\newcommand{\inner}[1]{\left< #1 \right>}
\newcommand{\inner}[2]{\left< #1, #2 \right>}
\newcommand{\powerset}[1]{\pazocal{P}(#1)}
%% Use \pazocal{letter} to typeset a letter in the other kind 
%%  of math calligraphic font. 
\newcommand{\cardinality}[1]{\left| #1 \right|}
\newcommand{\domain}[1]{\mathcal{D}(#1)}
\newcommand{\image}{\text{Im}}
\newcommand{\inv}[1]{#1^{-1}}
\newcommand{\preimage}[2]{#1^{-1}\left(#2\right)}
\newcommand{\script}[1]{\mathcal{#1}}


\newenvironment{highlight}{\begin{mdframed}[backgroundcolor=gray!20]}{\end{mdframed}}

\DeclarePairedDelimiter\ceil{\lceil}{\rceil}
\DeclarePairedDelimiter\floor{\lfloor}{\rfloor}

%===============================================
%===============My Tikz Commands================
%===============================================
\newcommand{\drawsquiggle}[1]{\draw[shift={(#1,0)}] (.005,.05) -- (-.005,.02) -- (.005,-.02) -- (-.005,-.05);}
\newcommand{\drawpoint}[2]{\draw[*-*] (#1,0.01) node[below, shift={(0,-.2)}] {#2};}
\newcommand{\drawopoint}[2]{\draw[o-o] (#1,0.01) node[below, shift={(0,-.2)}] {#2};}
\newcommand{\drawlpoint}[2]{\draw (#1,0.02) -- (#1,-0.02) node[below] {#2};}
\newcommand{\drawlbrack}[2]{\draw (#1+.01,0.02) --(#1,0.02) -- (#1,-0.02) -- (#1+.01,-0.02) node[below, shift={(-.01,0)}] {#2};}
\newcommand{\drawrbrack}[2]{\draw (#1-.01,0.02) --(#1,0.02) -- (#1,-0.02) -- (#1-.01,-0.02) node[below, shift={(+.01,0)}] {#2};}

%***********************************************
%**************Start of Document****************
%***********************************************
 %find me at /home/trevor/texmf/tex/latex/tskpreamble_nothms.tex
%===============================================
%===============Theorem Styles==================
%===============================================

%================Default Style==================
\theoremstyle{plain}% is the default. it sets the text in italic and adds extra space above and below the \newtheorems listed below it in the input. it is recommended for theorems, corollaries, lemmas, propositions, conjectures, criteria, and (possibly; depends on the subject area) algorithms.
\newtheorem{theorem}{Theorem}
\numberwithin{theorem}{section} %This sets the numbering system for theorems to number them down to the {argument} level. I have it set to number down to the {section} level right now.
\newtheorem*{theorem*}{Theorem} %Theorem with no numbering
\newtheorem{corollary}[theorem]{Corollary}
\newtheorem*{corollary*}{Corollary}
\newtheorem{conjecture}[theorem]{Conjecture}
\newtheorem{lemma}[theorem]{Lemma}
\newtheorem*{lemma*}{Lemma}
\newtheorem{proposition}[theorem]{Proposition}
\newtheorem*{proposition*}{Proposition}
\newtheorem{problemstatement}[theorem]{Problem Statement}


%==============Definition Style=================
\theoremstyle{definition}% adds extra space above and below, but sets the text in roman. it is recommended for definitions, conditions, problems, and examples; i've alse seen it used for exercises.
\newtheorem{definition}[theorem]{Definition}
\newtheorem*{definition*}{Definition}
\newtheorem{condition}[theorem]{Condition}
\newtheorem{problem}[theorem]{Problem}
\newtheorem{example}[theorem]{Example}
\newtheorem*{example*}{Example}
\newtheorem*{counterexample*}{Counterexample}
\newtheorem*{romantheorem*}{Theorem} %Theorem with no numbering
\newtheorem{exercise}{Exercise}
\numberwithin{exercise}{section}
\newtheorem{algorithm}[theorem]{Algorithm}

%================Remark Style===================
\theoremstyle{remark}% is set in roman, with no additional space above or below. it is recommended for remarks, notes, notation, claims, summaries, acknowledgments, cases, and conclusions.
\newtheorem{remark}[theorem]{Remark}
\newtheorem*{remark*}{Remark}
\newtheorem{notation}[theorem]{Notation}
\newtheorem*{notation*}{Notation}
%\newtheorem{claim}[theorem]{Claim}  %%use this if you ever want claims to be numbered
\newtheorem*{claim}{Claim}


%%
%% Page set-up:
%%
\pagestyle{empty}
\lhead{\textsc{201 - Real Analysis \\}} 
\rhead{\textsc{Harutyunyan, Fall 2019} \\ Trevor Klar}
%\chead{\Large\textbf{A New Integration Technique \\ }}
\renewcommand{\headrulewidth}{0pt}
%
\renewcommand{\footrulewidth}{0pt}
%\lfoot{
%Office: \quad \quad \, M 2-3 \, \, SH 6431x \\
%Math Lab: \, W 12-2 \, SH 1607
%}
%\rfoot{trevorklar@math.ucsb.edu}


\setlength{\parindent}{0in}
\setlength{\textwidth}{7in}
\setlength{\evensidemargin}{-0.25in}
\setlength{\oddsidemargin}{-0.25in}
\setlength{\parskip}{.5\baselineskip}
\setlength{\topmargin}{-0.5in}
\setlength{\textheight}{9in}

\setlist[enumerate,1]{label=\textbf{\arabic*.}}

\begin{document}
\pagestyle{fancy}
\begin{center}
{\Large Homework 5}%=================UPDATE THIS=================%
\end{center}

\begin{enumerate}
\item Let $X$ be a nonempty set and let $\mu$ be a measure on $X$. Prove that any nonnegative \mumeasurable{} function $f : X \to [0, \infty]$ is \muintegrable{} on $X,$ i.e., the lower integral equals the upper integral:
$${\int_*}_X f \der \mu = \int^*_X f \der \mu .$$
\begin{proof}
Let $f$ be nonnegative and \mumeasurable{}, and let $A_1\subset A_2\subset\dots$ be any sequence of measurable sets in $X$ such that $0<\measure{A_i}<\infty$ for every $i$, and $\bigcup_{m=1}^\infty A_m=X$\footnotemark. Now fix $M\in\N$, let $f^M$ be 
\[
f^M(x) = \begin{cases}
	\min\big(f(x),M\big), & \text{if } x\in A_M\\
	0 & \text{otherwise.}
\end{cases}
\]
\footnotetext{We can assume \Wlog{} that it is possible to produce this sequence of sets since if we cannot, then for every increasing sequence of sets whose union is $X$, $\measure{A_M}=\infty$ for some $M$, which means every function which is strictly positive $\mu$-a.e. has infinite upper and lower integral, which also gives us what we want.}
Thus $f^M$ is supported on $A_M$ and bounded above by $M$. Now for each $n\in\N$, define simple functions $\underline{g}^M_n$ and $\overline{g}^M_n$ by dividing the codomain $\R^+$ into intervals of length $\frac{1}{n}$. So for each $i= 1, 2, \dots$ we have 
\begin{align*}
\underline{g}^M_n&=\sum_{i=1}^\infty \left(\tfrac{i-1}{n}\right) \Chi_{\preimage{(f^M)}{\left[\tfrac{i-1}{n},\tfrac{i}{n}\right]}} \\
\overline{g}^M_n&=\sum_{i=1}^\infty \left(\tfrac{i}{n}\right) \Chi_{\preimage{(f^M)}{\left[\tfrac{i-1}{n},\tfrac{i}{n}\right]}}
\end{align*}
\jpg{width=0.7\textwidth}{201hw5p1}
Now we can observe that for every $n$, $\underline{g}^M_n<f<\overline{g}^M_n \, \mu-$a.e. and $$\int\underline{g}^M_n \der\mu - \int\overline{g}^M_n\der\mu=\tfrac{1}{n}\mu(A_M),$$
so we can choose $n$ large enough that $\tfrac{1}{n}\mu(A_M)<\varepsilon$ for any $\varepsilon$. Thus 
$${\int_*}_X f^M \der \mu = \int^*_X f^M \der \mu .$$
To finish the proof, we let $m$ vary over $\N$ and note that every $f^m$ is \muintegrable{}, and $\{f^m\}_{m=1}^\infty$ is an increasing sequence of functions which converges to $f$, so $f$ is integrable by MCT and 
$$\int_X f\der \mu = \lim_{m \to\infty} \int_X f^m \der\mu.$$
\end{proof}

\item Let $X$ be a nonempty set and let $\mu$ be a measure on $X$. Prove that if \mumeasurable{} functions $f, g : X \to [−\infty,\infty]$ are such that $f$ is $\mu$-summable on $X$, and $g$ is bounded on $X$ ($|g(x)| \leq M$ for all $x \in X$), then the product $fg$ is $\mu$-summable and
$$\int_X |fg|\der\mu \leq M\int_X |f|\der\mu.$$
\begin{proof}
By problem 1, we know that $|f|$ and $|g|$ are integrable. So 
\begin{align*}
	\int_X |fg|\der\mu &= \int_X |f||g|\der\mu\\
	&\leq \int_X (|f|M)\der\mu\\
	&= \int_X |Mf|\der\mu\\
\end{align*}
and, since for any $\mu$-summable simple function $\varphi$ we know that 
	\begin{align*}
	\int M\varphi \der\mu&= \int \left(M\sum_{i=1}^\infty (a_i) \Chi_{A_i} \right)\der\mu \\
	&= \sum_{i=1}^\infty M(a_i) \measure{A_i}\\
	&= M\sum_{i=1}^\infty (a_i) \measure{A_i}\\
	&= M\int \varphi \der\mu,\\
	\end{align*}
then ${\int_*}_X |Mf| \der \mu = \int^*_X |Mf| \der \mu = M{\int_*}_X |f| \der \mu = M\int^*_X |f| \der \mu $ so 
$$\int_X |Mf|\der\mu = M\int_X |f|\der\mu < \infty.$$
\end{proof}

\pagebreak
\item Let $\mu$ be a Radon measure and let $f:\R^n\to\R$ be $\mu$-summable. Prove that for any $\varepsilon>0$, there exists $\delta>0$ such that for every \mumeasurable{} set $A\subset\R^n$ with $\measure{A}<\delta$ one has 
$$ \int_A |f|\der\mu < \varepsilon.$$
\begin{proof}
Let $f_b=|f|\Chi_{\{|f|>b\}}$. Since $f$ is \musummable{}, then $|f|<\infty$ \muae{}, so $f_b\to0$ \muae{}\,. Then the sequence $f_b$ is dominated by $|f|$, so by the Dominated Convergence Theorem, 
$$\lim_{b\to\infty} \int_{\R^n} f_b \der\mu = \int_{\R^n} \lim_{b\to\infty}f_b \der\mu = 0.$$
So for any $\varepsilon>0$, there exists some $b\in\N$ such that $\frac{\varepsilon}{2}>\int_{\R^n} f_b \der\mu = \int_{\Chi_{\{|f|>b\}}} |f| \der\mu $. Now let $\delta=\frac{\varepsilon}{2b}$ and let $A\subset\R^n$ with $\measure{A}<\delta$. Then 
\begin{align*}
	\int_A |f| \der\mu &= \int_{A\cap{\{|f|>b\}}} |f| \der\mu + \int_{A\cap{\{|f|\leq b\}}} |f| \der\mu\\
%&	\hspace{-2.25in} \text{and since } b\geq|f|, \quad {A\cap{\{|f|>b\}}}\subset{\{|f|>b\}}, \quad {A\cap{\{|f|>b\}}}\subset A,\\
	&\leq \int_{{\{|f|>b\}}} |f| \der\mu + \int_{A} b \der\mu\\
	&= \int_{\R^n} f_b \der\mu + b\measure{A}\\
	&\leq \tfrac{\varepsilon}{2	} + \tfrac{\varepsilon}{2}\\
\end{align*}
and we are done.
\end{proof}

\pagebreak
\item Let $X$ be a nonempty set and let $\mu$ be a measure on $X$. Assume \musummable{} functions $f,f_n:X\to[-\infty,\infty]$ are such that 
$$f_n\to f \quad \text{\muae{} in } X,$$
and 
$$\int_X |f_n| \der\mu \to \int_X |f| \der\mu .$$
Prove that 
$$\int_X |f_n-f|\der\mu \to 0.$$
\begin{proof}
Since 
\begin{itemize}
	\item $f,f_n$ are \mumeasurable{} and $|f|,|f_n|$ are \musummable{},  
	\item $f_n\to f$ \muae{},
	\item $|f_n|\leq |f_n|$,
	\item $|f_n|\to |f|$ \muae{},
	\item $\int_X |f_n| \der\mu \to \int_X |f| \der\mu$, 	
\end{itemize}
Then all the conditions of the Variant of Dominated Convergence Theorem from the text are satisfied, and we are done. 
\end{proof}

\pagebreak
\item Compute the limit 
$$\lim_{n\to\infty} \int_0^n \left(1-\frac{x}{n}\right)^n \ln\left(2+\cos\left(\frac{x}{n}\right)\right)\dx . $$
\answer	$\ln3$. 
\begin{proof}
Let 
\begin{align*}
f_n&=\Chi_{[0,n]} \left(1-\frac{x}{n}\right)^n, \quad\text{and} \\
g_n&=\Chi_{[0,n]} \ln\left(2+\cos\left(\frac{x}{n}\right)\right), \quad\text{so that} \\
F_n&=f_ng_n=\Chi_{[0,n]} \left(1-\frac{x}{n}\right)^n \ln\left(2+\cos\left(\frac{x}{n}\right)\right).
\end{align*}
Now the desired limit is $ \lim_{n\to\infty} \int_\R F_n \der\mu.$ Taking derivatives, we find that 
\begin{align*}
\frac{\der}{\der n}f_n &= \left(1-\frac{x}{n}\right)^n\left(\frac{x}{\left(1-\frac{x}{n}\right)n}+\ln\left(1-\frac{x}{n}\right)\right)\\
\frac{\der}{\der n}g_n &= \frac{x\sin\left(\frac{x}{n}\right)}{\left(\cos\left(\frac{x}{n}\right)+2\right)n^2}
\end{align*}
%and since we are only concerned with $x,n$ values such that $0<x<n$\footnote{For any $n$, the set where $x\in\{0,n\}$ has measure zero, so doesn't affect the integral; and if $x>n$, then $F_n(x)=0$ and $F_{n+1}(x)\geq0$.} then $0<\frac{x}{n}<1$ and so both derivatives above are positive. Thus we can conclude that $F_n$ is an increasing sequence of functions
and since we are only concerned with $x,n$ values such that $0<x<n$\footnote{For any $n$, the set where $x\in\{0,n\}$ has measure zero, so doesn't affect the integral; and if $x>n$, then $F_n(x)=0$ and $F_{n+1}(x)\geq0$.} then $0<\frac{x}{n}<1$ and so all the quantities above are positive, except the $\ln$ term. Thus we can conclude that $F_n$ is an increasing sequence of functions if we can show that $h_n=\frac{x}{\left(1-\frac{x}{n}\right)n}+\ln\left(1-\frac{x}{n}\right)>0$. For any fixed $n$ and $0\leq x<n$, $h_n(0)=0$, and $h_n(x)$ is continuous and increasing, since $h_n'= \frac{x}{(x-n)^2}$, which is positive. Thus $h_n$ is positive, and therefore $F_n$ is an increasing sequence of measurable nonnegative functions. 

By the Monotone Convergence Theorem, 
$$\lim_{n\to\infty}\int_\R F_n \dx = \int_\R \lim_{n\to\infty} F_n \dx = \int_0^\infty e^{-x}\ln\left(2+\cos(0)\right)\dx = \ln3 $$
%\begin{align*}
%I can't figure out how to show this.
%\end{align*}

\end{proof}

\end{enumerate}

\end{document}
