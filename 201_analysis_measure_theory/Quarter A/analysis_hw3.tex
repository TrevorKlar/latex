\documentclass[12pt,letterpaper]{article}

\usepackage{fancyhdr,fancybox}

%% Useful packages
\usepackage{amssymb, amsmath, amsthm} 
%\usepackage{graphicx}  %%this is currently enabled in the default document, so it is commented out here. 
\usepackage{calrsfs}
\usepackage{braket}
\usepackage{mathtools}
\usepackage{lipsum}
\usepackage{tikz}
\usetikzlibrary{cd}
\usepackage{verbatim}
%\usepackage{ntheorem}% for theorem-like environments
\usepackage{mdframed}%can make highlighted boxes of text
%Use case: https://tex.stackexchange.com/questions/46828/how-to-highlight-important-parts-with-a-gray-background
\usepackage{wrapfig}
\usepackage{centernot}
\usepackage{subcaption}%\begin{subfigure}{0.5\textwidth}
\usepackage{pgfplots}
\pgfplotsset{compat=1.13}
\usepackage[colorinlistoftodos]{todonotes}
\usepackage[colorlinks=true, allcolors=blue]{hyperref}
\usepackage{xfrac}					%to make slanted fractions \sfrac{numerator}{denominator}
\usepackage{enumitem}            
    %syntax: \begin{enumerate}[label=(\alph*)]
    %possible arguments: f \alph*, \Alph*, \arabic*, \roman* and \Roman*
\usetikzlibrary{arrows,shapes.geometric,fit}

\DeclareMathAlphabet{\pazocal}{OMS}{zplm}{m}{n}
%% Use \pazocal{letter} to typeset a letter in the other kind 
%%  of math calligraphic font. 

%% This puts the QED block at the end of each proof, the way I like it. 
\renewenvironment{proof}{{\bfseries Proof}}{\qed}
\makeatletter
\renewenvironment{proof}[1][\bfseries \proofname]{\par
  \pushQED{\qed}%
  \normalfont \topsep6\p@\@plus6\p@\relax
  \trivlist
  %\itemindent\normalparindent
  \item[\hskip\labelsep
        \scshape
    #1\@addpunct{}]\ignorespaces
}{%
  \popQED\endtrivlist\@endpefalse
}
\makeatother

%% This adds a \rewnewtheorem command, which enables me to override the settings for theorems contained in this document.
\makeatletter
\def\renewtheorem#1{%
  \expandafter\let\csname#1\endcsname\relax
  \expandafter\let\csname c@#1\endcsname\relax
  \gdef\renewtheorem@envname{#1}
  \renewtheorem@secpar
}
\def\renewtheorem@secpar{\@ifnextchar[{\renewtheorem@numberedlike}{\renewtheorem@nonumberedlike}}
\def\renewtheorem@numberedlike[#1]#2{\newtheorem{\renewtheorem@envname}[#1]{#2}}
\def\renewtheorem@nonumberedlike#1{  
\def\renewtheorem@caption{#1}
\edef\renewtheorem@nowithin{\noexpand\newtheorem{\renewtheorem@envname}{\renewtheorem@caption}}
\renewtheorem@thirdpar
}
\def\renewtheorem@thirdpar{\@ifnextchar[{\renewtheorem@within}{\renewtheorem@nowithin}}
\def\renewtheorem@within[#1]{\renewtheorem@nowithin[#1]}
\makeatother

%% This makes theorems and definitions with names show up in bold, the way I like it. 
\makeatletter
\def\th@plain{%
  \thm@notefont{}% same as heading font
  \itshape % body font
}
\def\th@definition{%
  \thm@notefont{}% same as heading font
  \normalfont % body font
}
\makeatother

%===============================================
%==============Shortcut Commands================
%===============================================
\newcommand{\ds}{\displaystyle}
\newcommand{\B}{\mathcal{B}}
\newcommand{\C}{\mathbb{C}}
\newcommand{\F}{\mathbb{F}}
\newcommand{\N}{\mathbb{N}}
\newcommand{\R}{\mathbb{R}}
\newcommand{\Q}{\mathbb{Q}}
\newcommand{\T}{\mathcal{T}}
\newcommand{\Z}{\mathbb{Z}}
\renewcommand\qedsymbol{$\blacksquare$}
\newcommand{\qedwhite}{\hfill\ensuremath{\square}}
\newcommand*\conj[1]{\overline{#1}}
\newcommand*\closure[1]{\overline{#1}}
\newcommand*\mean[1]{\overline{#1}}
%\newcommand{\inner}[1]{\left< #1 \right>}
\newcommand{\inner}[2]{\left< #1, #2 \right>}
\newcommand{\powerset}[1]{\pazocal{P}(#1)}
%% Use \pazocal{letter} to typeset a letter in the other kind 
%%  of math calligraphic font. 
\newcommand{\cardinality}[1]{\left| #1 \right|}
\newcommand{\domain}[1]{\mathcal{D}(#1)}
\newcommand{\image}{\text{Im}}
\newcommand{\inv}[1]{#1^{-1}}
\newcommand{\preimage}[2]{#1^{-1}\left(#2\right)}
\newcommand{\script}[1]{\mathcal{#1}}


\newenvironment{highlight}{\begin{mdframed}[backgroundcolor=gray!20]}{\end{mdframed}}

\DeclarePairedDelimiter\ceil{\lceil}{\rceil}
\DeclarePairedDelimiter\floor{\lfloor}{\rfloor}

%===============================================
%===============My Tikz Commands================
%===============================================
\newcommand{\drawsquiggle}[1]{\draw[shift={(#1,0)}] (.005,.05) -- (-.005,.02) -- (.005,-.02) -- (-.005,-.05);}
\newcommand{\drawpoint}[2]{\draw[*-*] (#1,0.01) node[below, shift={(0,-.2)}] {#2};}
\newcommand{\drawopoint}[2]{\draw[o-o] (#1,0.01) node[below, shift={(0,-.2)}] {#2};}
\newcommand{\drawlpoint}[2]{\draw (#1,0.02) -- (#1,-0.02) node[below] {#2};}
\newcommand{\drawlbrack}[2]{\draw (#1+.01,0.02) --(#1,0.02) -- (#1,-0.02) -- (#1+.01,-0.02) node[below, shift={(-.01,0)}] {#2};}
\newcommand{\drawrbrack}[2]{\draw (#1-.01,0.02) --(#1,0.02) -- (#1,-0.02) -- (#1-.01,-0.02) node[below, shift={(+.01,0)}] {#2};}

%***********************************************
%**************Start of Document****************
%***********************************************
 %find me at /home/trevor/texmf/tex/latex/tskpreamble_nothms.tex
%===============================================
%===============Theorem Styles==================
%===============================================

%================Default Style==================
\theoremstyle{plain}% is the default. it sets the text in italic and adds extra space above and below the \newtheorems listed below it in the input. it is recommended for theorems, corollaries, lemmas, propositions, conjectures, criteria, and (possibly; depends on the subject area) algorithms.
\newtheorem{theorem}{Theorem}
\numberwithin{theorem}{section} %This sets the numbering system for theorems to number them down to the {argument} level. I have it set to number down to the {section} level right now.
\newtheorem*{theorem*}{Theorem} %Theorem with no numbering
\newtheorem{corollary}[theorem]{Corollary}
\newtheorem*{corollary*}{Corollary}
\newtheorem{conjecture}[theorem]{Conjecture}
\newtheorem{lemma}[theorem]{Lemma}
\newtheorem*{lemma*}{Lemma}
\newtheorem{proposition}[theorem]{Proposition}
\newtheorem*{proposition*}{Proposition}
\newtheorem{problemstatement}[theorem]{Problem Statement}


%==============Definition Style=================
\theoremstyle{definition}% adds extra space above and below, but sets the text in roman. it is recommended for definitions, conditions, problems, and examples; i've alse seen it used for exercises.
\newtheorem{definition}[theorem]{Definition}
\newtheorem*{definition*}{Definition}
\newtheorem{condition}[theorem]{Condition}
\newtheorem{problem}[theorem]{Problem}
\newtheorem{example}[theorem]{Example}
\newtheorem*{example*}{Example}
\newtheorem*{counterexample*}{Counterexample}
\newtheorem*{romantheorem*}{Theorem} %Theorem with no numbering
\newtheorem{exercise}{Exercise}
\numberwithin{exercise}{section}
\newtheorem{algorithm}[theorem]{Algorithm}

%================Remark Style===================
\theoremstyle{remark}% is set in roman, with no additional space above or below. it is recommended for remarks, notes, notation, claims, summaries, acknowledgments, cases, and conclusions.
\newtheorem{remark}[theorem]{Remark}
\newtheorem*{remark*}{Remark}
\newtheorem{notation}[theorem]{Notation}
\newtheorem*{notation*}{Notation}
%\newtheorem{claim}[theorem]{Claim}  %%use this if you ever want claims to be numbered
\newtheorem*{claim}{Claim}


\DeclareMathOperator{\diameter}{diam}
\newcommand{\diam}[1]{\diameter\left(#1\right)}

%%
%% Page set-up:
%%
\pagestyle{empty}
\lhead{\textsc{201 - Real Analysis \\}} 
\rhead{\textsc{Harutyunyan, Fall 2019} \\ Trevor Klar}
%\chead{\Large\textbf{A New Integration Technique \\ }}
\renewcommand{\headrulewidth}{0pt}
%
\renewcommand{\footrulewidth}{0pt}
%\lfoot{
%Office: \quad \quad \, M 2-3 \, \, SH 6431x \\
%Math Lab: \, W 12-2 \, SH 1607
%}
%\rfoot{trevorklar@math.ucsb.edu}


\setlength{\parindent}{0in}
\setlength{\textwidth}{7in}
\setlength{\evensidemargin}{-0.25in}
\setlength{\oddsidemargin}{-0.25in}
\setlength{\parskip}{.5\baselineskip}
\setlength{\topmargin}{-0.5in}
\setlength{\textheight}{9in}

\setlist[enumerate,1]{label=\textbf{\arabic*.}}

\begin{document}
\pagestyle{fancy}
\begin{center}
{\Large Homework 3}%=================UPDATE THIS=================%
\end{center}

\begin{enumerate}
\item Let $\mu$ be a Lebesgue measure and let $\{A_n\}_{n=1}^\infty$ be a sequence of Lebesgue measurable subsets of $[0, 1]$. Assume the set $B$ consists of those points $x \in [0, 1]$ that belong
to infinitely many of the $A_n$.
	\begin{enumerate}[label=(\roman*)]
	\item Prove that $B$ is Lebesgue-measurable.
		\begin{proof}
		Let $x\in B$. Then $x$ is in infinitely many of the $A_n$; so for every $k \geq 1$, $x\in A_n$ for some $n\geq k$. That is, $x\in\bigcap_{k=1}^\infty\bigcup_{n=k}^\infty A_n$ and in fact, $B=\bigcap_{k=1}^\infty\bigcup_{n=k}^\infty A_n$. This is a Borel set, so it is Lebesgue-measurable. 
		\end{proof}
	\item Prove that if $\measure{A_n} > \delta > 0$ for every $n \in N$, then $\mu(B) \geq \delta$.
		\begin{proof}
		Let $B_k=\bigcup_{n=k}^\infty A_n$. Since $A_k\subseteq B_k$, then $\delta<\measure{A_k}\leq\measure{B_k}$, for all $k$. Now consider $\bigcap_{k=1}^M B_k$. Since 
		\begin{align*}
		B_j\cap B_k&= \left(\bigcup_{n\geq k}A_n\right)\cap\left(\bigcup_{n\geq j}A_n\right)\\
		&= \bigcup_{n\geq \max(j,k)}A_n\\
		&= B_{\max(j,k)},
		\end{align*}		 
		then $\bigcap_{k=1}^M B_k=B_M$. Then $B_k \searrow B$ and $\measure{B_1}<1$, so $\lim_{n\to\infty} \measure{B_n} = \measure{B} \geq \delta$. 
		\end{proof}
	\item Prove that if $\sum_{n=1}^\infty \mu(A_n) < \infty$, then $\mu(B) = 0$.
		\begin{proof}
		Let $B_k=\bigcup_{n=k}^\infty A_n$. Now
		$$\measure{B_k}=\measure{\bigcup_{n=k}^\infty A_n} \leq \sum_{n=k}^\infty \mu(A_n).$$
		Since $\sum_{n=1}^\infty \mu(A_n) < \infty$, then $\mu(A_n)\to0$, which means the tail of the sum also goes to $0$ as $k\to\infty$. Thus $\measure{B_k}\to0$, and $B_k \searrow B$ and $\measure{B_1}>1$, so 
		$$\lim_{n\to\infty} \measure{B_n} = \measure{B} =0.$$ 
		\end{proof}
	\item Give an example where $\sum_{n=1}^\infty \mu(A_n) = \infty$, but $\mu(B) = 0$. 
		\answer Let $A_n=[0,1/n]$. Then $\sum_{n=1}^\infty \mu(A_n) = \sum_{n=1}^\infty 1/n = \infty$, but $B=\{0\}$ and $\mu(B) = 0$. 
	\end{enumerate}
	
\pagebreak
\item Prove that if $A\subset\R$ is Lebesgue-measurable with $\measure{A}>0$, then there is a subset of $A$ that is not Lebesgue-measurable. 
\begin{lemma*}
	If $A\subset\R$ is Lebesgue-measurable with $\measure{A}>0$, then there exists a subset $\tilde{A}\subset A$ with $\tilde{A}$ bounded and $\measure{\tilde{A}}>0$. \textbf{Proof:} Suppose not. Then for every $\tilde{A}\subset A$, either $\tilde{A}$ unbounded or $\measure{\tilde{A}}=0$. If we consider the sets 
	$$A_n=\big\lbrace[n,n+1)\cap A\big\rbrace_{n\in\Z},$$
	then each $A_n$ is bounded, thus it has measure zero. Since each $A-n$ is measurable and $A=\coprod\limits_{n\in Z}A_n$, then $0=\sum\limits_{n\in\Z}\measure{A_n}=\measure{A}>0$, contradiction. \qedwhite
\end{lemma*}
\begin{proof}
By the lemma, \Wlog{} we can assume that $A$ is bounded, so let $[-a,a]\supset A$. Define an equivalence relation on $A$ as follows. For all $x,y\in A$, 
\begin{align*}
x\sim y \text{ if } \exists\, q\in(\Q\cap[-a,a]) \text{ such that } x-y=q
\end{align*}
A little thought will show that $\sim$ is reflexive, symmetric, and transitive. Thus the collection of all equivalence classes $\{[x]|x\in A\}$ is a partition of $A$. Define $V$ by choosing exactly one representative of each equivalence class. Then for each $x\in A$, there exists a unique $y\in V$ such that $x\sim y$, and $V\subset A$. Now all the remains is to show that $V$ is not Lebesgue-measurable. 

Suppose for contradiction that $V$ is measurable, and consider 
$$\{V+q\,|\,q\in(\Q\cap[-a,a])\}.$$ 
(From now on in this proof, we assume $q\in(\Q\cap[-a,a])$.) Since every $a\in A$ has a $y\in V$ such that $x\sim y$, then $A\subseteq\bigcup_q (V+q)$. And since $A\subseteq [-a,a]$ and every $q\in[-a,a]$, then $\bigcup_q (V+q)\subseteq [-2a,2a]$. Thus by monotonicity, 
$$0<\measure{A}\leq\measure{\bigcup_{q}(V+q)}\leq4a<\infty.$$
Since $V+q_1$ and $V+q_2$ are disjoint and measurable for all $q_1\neq q_2$, then $\measure{\coprod_{q}(V+q)}=\sum_q\measure{V+q}=\sum_q\measure{V}$ since Lebesgue measure is translation-invariant. Now on one hand, if $\measure{V}>0$ then $\sum_q\measure{V}=\infty$, but $\sum_q\measure{V}<\infty$. On the other hand, if $\measure{V}=0$ then $\sum_q\measure{V}=0$, but $\sum_q\measure{V}>0$. Thus $0<\measure{V}=0$, contradiction. Therefore $V$ cannot be measurable. 
\end{proof}

\pagebreak
\item Let $\mu$ be the Lebesgue measure on $\R$. Construct a Borel set $A\subset \R$ such that $\measure{A}>0$ and $\measure{A\cap I}<\measure{I}$ for every non-degenerate interval $I\subset\R$. 
\begin{proof}
Let $r_k$ be an enumeration of the rationals, and let 
$$A=(-100,-100)\setminus\bigcup_{k=1}^\infty B(r_k,1/2^k).$$
Let $I$ be any non-degenerate interval, let $a=\inf I$, and let $b=\sup I$. Then $(a,b)\subseteq I$, where $a<b$. In the case that $a=-\infty$ or $b=\infty$, then $\measure{I}=\infty$, and $\measure{A\cap I}\leq\measure{A}\leq200$, so we're done. So consider the case where $a,b\in\R$. Since $\measure{I}=b-a$ and $(a,b)\subseteq I$, we will show that $\measure{(a,b)\cap A}<b-a$. Choose some $r_k\in(a,b)$. Then $B(r_k,1/2^k)\cap (a,b)$ is open, so there exists some $\epsilon>0$ such that $B(r_k,\epsilon)\subset B(r_k,1/2^k)\cap (a,b)$. Now, since $B(r_k,\epsilon)\subset B(r_k,1/2^k)\subset A^\complement$ but $B(r_k,\epsilon)\subset(a,b)$, then 
$$I\cap A \subseteq I\setminus B(r_k,\epsilon) \subset I,$$ 
so since all these sets are measurable, 
$$\measure{I\cap A}\leq \measure{I}-\measure{B(r_k,\epsilon)} < \measure{I}.$$
\end{proof}

\item Let $A\subset\R$ be a Lebesgue-measurable set. Prove that the set 
$$B=\bigcup\nolimits_{x\in A}[x-1, x+1]$$
is Lebesgue-measurable. 
\begin{proof}
Observe that $B=\bigcup_{x\in A} B_1(x) \cup A-1 \cup A+1$. The union of balls is Borel, and translation invariance of Lebesgue measure tells us that the other two sets are measurable as well. Thus $B$ is a union of 3 measurable sets, and thus measurable. 
\end{proof}




\end{enumerate}

\end{document}
