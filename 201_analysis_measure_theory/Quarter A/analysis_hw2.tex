\documentclass[12pt,letterpaper]{article}

\usepackage{fancyhdr,fancybox}

%% Useful packages
\usepackage{amssymb, amsmath, amsthm} 
%\usepackage{graphicx}  %%this is currently enabled in the default document, so it is commented out here. 
\usepackage{calrsfs}
\usepackage{braket}
\usepackage{mathtools}
\usepackage{lipsum}
\usepackage{tikz}
\usetikzlibrary{cd}
\usepackage{verbatim}
%\usepackage{ntheorem}% for theorem-like environments
\usepackage{mdframed}%can make highlighted boxes of text
%Use case: https://tex.stackexchange.com/questions/46828/how-to-highlight-important-parts-with-a-gray-background
\usepackage{wrapfig}
\usepackage{centernot}
\usepackage{subcaption}%\begin{subfigure}{0.5\textwidth}
\usepackage{pgfplots}
\pgfplotsset{compat=1.13}
\usepackage[colorinlistoftodos]{todonotes}
\usepackage[colorlinks=true, allcolors=blue]{hyperref}
\usepackage{xfrac}					%to make slanted fractions \sfrac{numerator}{denominator}
\usepackage{enumitem}            
    %syntax: \begin{enumerate}[label=(\alph*)]
    %possible arguments: f \alph*, \Alph*, \arabic*, \roman* and \Roman*
\usetikzlibrary{arrows,shapes.geometric,fit}

\DeclareMathAlphabet{\pazocal}{OMS}{zplm}{m}{n}
%% Use \pazocal{letter} to typeset a letter in the other kind 
%%  of math calligraphic font. 

%% This puts the QED block at the end of each proof, the way I like it. 
\renewenvironment{proof}{{\bfseries Proof}}{\qed}
\makeatletter
\renewenvironment{proof}[1][\bfseries \proofname]{\par
  \pushQED{\qed}%
  \normalfont \topsep6\p@\@plus6\p@\relax
  \trivlist
  %\itemindent\normalparindent
  \item[\hskip\labelsep
        \scshape
    #1\@addpunct{}]\ignorespaces
}{%
  \popQED\endtrivlist\@endpefalse
}
\makeatother

%% This adds a \rewnewtheorem command, which enables me to override the settings for theorems contained in this document.
\makeatletter
\def\renewtheorem#1{%
  \expandafter\let\csname#1\endcsname\relax
  \expandafter\let\csname c@#1\endcsname\relax
  \gdef\renewtheorem@envname{#1}
  \renewtheorem@secpar
}
\def\renewtheorem@secpar{\@ifnextchar[{\renewtheorem@numberedlike}{\renewtheorem@nonumberedlike}}
\def\renewtheorem@numberedlike[#1]#2{\newtheorem{\renewtheorem@envname}[#1]{#2}}
\def\renewtheorem@nonumberedlike#1{  
\def\renewtheorem@caption{#1}
\edef\renewtheorem@nowithin{\noexpand\newtheorem{\renewtheorem@envname}{\renewtheorem@caption}}
\renewtheorem@thirdpar
}
\def\renewtheorem@thirdpar{\@ifnextchar[{\renewtheorem@within}{\renewtheorem@nowithin}}
\def\renewtheorem@within[#1]{\renewtheorem@nowithin[#1]}
\makeatother

%% This makes theorems and definitions with names show up in bold, the way I like it. 
\makeatletter
\def\th@plain{%
  \thm@notefont{}% same as heading font
  \itshape % body font
}
\def\th@definition{%
  \thm@notefont{}% same as heading font
  \normalfont % body font
}
\makeatother

%===============================================
%==============Shortcut Commands================
%===============================================
\newcommand{\ds}{\displaystyle}
\newcommand{\B}{\mathcal{B}}
\newcommand{\C}{\mathbb{C}}
\newcommand{\F}{\mathbb{F}}
\newcommand{\N}{\mathbb{N}}
\newcommand{\R}{\mathbb{R}}
\newcommand{\Q}{\mathbb{Q}}
\newcommand{\T}{\mathcal{T}}
\newcommand{\Z}{\mathbb{Z}}
\renewcommand\qedsymbol{$\blacksquare$}
\newcommand{\qedwhite}{\hfill\ensuremath{\square}}
\newcommand*\conj[1]{\overline{#1}}
\newcommand*\closure[1]{\overline{#1}}
\newcommand*\mean[1]{\overline{#1}}
%\newcommand{\inner}[1]{\left< #1 \right>}
\newcommand{\inner}[2]{\left< #1, #2 \right>}
\newcommand{\powerset}[1]{\pazocal{P}(#1)}
%% Use \pazocal{letter} to typeset a letter in the other kind 
%%  of math calligraphic font. 
\newcommand{\cardinality}[1]{\left| #1 \right|}
\newcommand{\domain}[1]{\mathcal{D}(#1)}
\newcommand{\image}{\text{Im}}
\newcommand{\inv}[1]{#1^{-1}}
\newcommand{\preimage}[2]{#1^{-1}\left(#2\right)}
\newcommand{\script}[1]{\mathcal{#1}}


\newenvironment{highlight}{\begin{mdframed}[backgroundcolor=gray!20]}{\end{mdframed}}

\DeclarePairedDelimiter\ceil{\lceil}{\rceil}
\DeclarePairedDelimiter\floor{\lfloor}{\rfloor}

%===============================================
%===============My Tikz Commands================
%===============================================
\newcommand{\drawsquiggle}[1]{\draw[shift={(#1,0)}] (.005,.05) -- (-.005,.02) -- (.005,-.02) -- (-.005,-.05);}
\newcommand{\drawpoint}[2]{\draw[*-*] (#1,0.01) node[below, shift={(0,-.2)}] {#2};}
\newcommand{\drawopoint}[2]{\draw[o-o] (#1,0.01) node[below, shift={(0,-.2)}] {#2};}
\newcommand{\drawlpoint}[2]{\draw (#1,0.02) -- (#1,-0.02) node[below] {#2};}
\newcommand{\drawlbrack}[2]{\draw (#1+.01,0.02) --(#1,0.02) -- (#1,-0.02) -- (#1+.01,-0.02) node[below, shift={(-.01,0)}] {#2};}
\newcommand{\drawrbrack}[2]{\draw (#1-.01,0.02) --(#1,0.02) -- (#1,-0.02) -- (#1-.01,-0.02) node[below, shift={(+.01,0)}] {#2};}

%***********************************************
%**************Start of Document****************
%***********************************************
 %find me at /home/trevor/texmf/tex/latex/tskpreamble_nothms.tex
%===============================================
%===============Theorem Styles==================
%===============================================

%================Default Style==================
\theoremstyle{plain}% is the default. it sets the text in italic and adds extra space above and below the \newtheorems listed below it in the input. it is recommended for theorems, corollaries, lemmas, propositions, conjectures, criteria, and (possibly; depends on the subject area) algorithms.
\newtheorem{theorem}{Theorem}
\numberwithin{theorem}{section} %This sets the numbering system for theorems to number them down to the {argument} level. I have it set to number down to the {section} level right now.
\newtheorem*{theorem*}{Theorem} %Theorem with no numbering
\newtheorem{corollary}[theorem]{Corollary}
\newtheorem*{corollary*}{Corollary}
\newtheorem{conjecture}[theorem]{Conjecture}
\newtheorem{lemma}[theorem]{Lemma}
\newtheorem*{lemma*}{Lemma}
\newtheorem{proposition}[theorem]{Proposition}
\newtheorem*{proposition*}{Proposition}
\newtheorem{problemstatement}[theorem]{Problem Statement}


%==============Definition Style=================
\theoremstyle{definition}% adds extra space above and below, but sets the text in roman. it is recommended for definitions, conditions, problems, and examples; i've alse seen it used for exercises.
\newtheorem{definition}[theorem]{Definition}
\newtheorem*{definition*}{Definition}
\newtheorem{condition}[theorem]{Condition}
\newtheorem{problem}[theorem]{Problem}
\newtheorem{example}[theorem]{Example}
\newtheorem*{example*}{Example}
\newtheorem*{counterexample*}{Counterexample}
\newtheorem*{romantheorem*}{Theorem} %Theorem with no numbering
\newtheorem{exercise}{Exercise}
\numberwithin{exercise}{section}
\newtheorem{algorithm}[theorem]{Algorithm}

%================Remark Style===================
\theoremstyle{remark}% is set in roman, with no additional space above or below. it is recommended for remarks, notes, notation, claims, summaries, acknowledgments, cases, and conclusions.
\newtheorem{remark}[theorem]{Remark}
\newtheorem*{remark*}{Remark}
\newtheorem{notation}[theorem]{Notation}
\newtheorem*{notation*}{Notation}
%\newtheorem{claim}[theorem]{Claim}  %%use this if you ever want claims to be numbered
\newtheorem*{claim}{Claim}


\DeclareMathOperator{\diameter}{diam}
\newcommand{\diam}[1]{\diameter\left(#1\right)}

%%
%% Page set-up:
%%
\pagestyle{empty}
\lhead{\textsc{201 - Real Analysis}} 
\rhead{\textsc{Harutyunyan, Fall 2019}}
%\chead{\Large\textbf{A New Integration Technique \\ }}
\renewcommand{\headrulewidth}{0pt}
%
\renewcommand{\footrulewidth}{0pt}
%\lfoot{
%Office: \quad \quad \, M 2-3 \, \, SH 6431x \\
%Math Lab: \, W 12-2 \, SH 1607
%}
%\rfoot{trevorklar@math.ucsb.edu}


\setlength{\parindent}{0in}
\setlength{\textwidth}{7in}
\setlength{\evensidemargin}{-0.25in}
\setlength{\oddsidemargin}{-0.25in}
\setlength{\parskip}{.5\baselineskip}
\setlength{\topmargin}{-0.5in}
\setlength{\textheight}{9in}

\setlist[enumerate,1]{label=\textbf{\arabic*.}}

\begin{document}
\pagestyle{fancy}
\begin{center}
{\Large Homework 2}%=================UPDATE THIS=================%
\end{center}

\begin{enumerate}
\item Let $x\in\R^n$ and let $K\subset\R^n$ be compact. Denote $U=\R^n-K$ and define for each fixed $s\in K$ the function 
$$u_s(x)=\max\left(2-\frac{\abs{x-s}}{\text{dist}(x,K)},0\right), \quad x\in U.$$
Let $s_i$ be a countable dense subset of $K$ and define 
$$\sigma(x)=\sum_{i=1}^\infty2^{-i}u_{s_i}(x), \quad x\in U.$$
It is not difficult to prove that then $0<\sigma(x)\leq1$ for all $x\in U$, thus we can define 
$$v_i(x)=\frac{2^{-i}u_{s_i}(x)}{\sigma(x)}, \quad x\in U.$$
Assume next $f:K\to\R$ is continuous and define 
$$\bar{f}(x)=\sum_{i=1}^\infty	v_i(x)f(s_i), \quad x\in U.$$
Prove that $\bar{f}(x)$ is continuous in $U$. 
\begin{proof}
We will show that $u_s$ is continuous and 
\begin{center}
$u_s$ continuous $\implies \sigma$ continuous $\implies v_i$ continuous $\implies \bar{f}$ continuous. 
\end{center}
\begin{itemize}
\item ($u_s$) We already know that max and euclidean distance functions are continuous, so if dist$(x,K)$ is continuous, then $u_s$ is comprised of compositions, sums, and products of continuous functions, so is continuous. So all that remains is to show that $\dist{x}{K}$ is continuous. Let $x\in U = K^\complement$, $\epsilon>0$ and $y\in\R^n$ such that $|x-y|<\frac{\epsilon}{2}$. Then for any $k\in K$, 
$$|x-k|-\frac{\epsilon}{2}\leq|y-k|\leq|x-k|+\frac{\epsilon}{2}$$
by triangle inequality, so taking infs and using $\epsilon$ instead of $\frac{\epsilon}{2}$ to obtain strict inequalities, we find that 
$$\dist{x}{K}-\epsilon<\dist{y}{K}<\dist{x}{K}+\epsilon$$
so dist$(x,K)$ is continuous. 
\item ($\sigma$) First, observe that for all $s\in K, x\in U$, $\frac{|x-s|}{\dist{x}{K}}$ is always $\geq 1$ and approaches 1 as $x$ gets very far from $K$. This tells us that $0 \leq u_{s_i} \leq 1$ for every $s_i$. Then we can use the Weierstrauss M-test. For $x\in U$, 
$$\sigma(x)=\sum_{i=1}^\infty2^{-i}u_{s_i}(x)=\sum_{i=1}^\infty|2^{-i}u_{s_i}(x)|\leq \sum_{i=1}^\infty2^{-i}=1,$$
so since $2^{-i}u_{s_i}$ are continuous functions, then so is $\sigma$. 
\item ($v_i$) $v_i$ is a product of continuous functions, so it is continuous whenever $\sigma(x)\neq0$, so let's check that $\sigma$ is always positive. Suppose for contradiction that there exists $x\in U$ such that $\sigma	(x)=0$\footnote{$\sigma$ is certainly never negative because it is a sum of nonnegative numbers.}. Each term of $\sigma$ is the product of a nonzero number with $u_{s_i}$, so $\sigma(x)=0$ iff all $u_{s_i}(x)=0$. This means that $|x-s_i|\geq2\,\dist{x}{K}$ for all $s_i$, which is impossible since $\{s_i\}$ is dense in $K$. To see the contradiction, observe that for any $k\in K$, there is a sequence $\{s_i\}_{i\in I\subset \N}$ which converges to $k$, so 
$$\inf_{i\in N}|x-s_i|=\inf_{k\in K}|x-k|=\dist{x}{K},$$ 
thus there exists some $s_i$ such that $|x-s_i|<2\,\dist{x}{K}$. Therefore $\sigma$ never vanishes, and $v_i$ is continuous. 
\item ($\bar{f}$) Since $f$ is a continuous function on a compact domain, then it is bounded. Denote the bound $B\geq f(x)$ for all $x\in K$. Then 
\begin{align*}
\bar{f}(x)&=\sum_{i=1}^\infty v_i(x)f(s_i)\\
&=\sum_{i=1}^\infty \frac{2^{-i}u_{s_i}(x)}{\sigma(x)}f(s_i)\\
&=\frac{1}{\sigma(x)}\sum_{i=1}^\infty 2^{-i}u_{s_i}(x)f(s_i)\\
&\leq\frac{1}{\sigma(x)}\sum_{i=1}^\infty 2^{-i}(1)(B)\\
&=\leq\frac{1}{\sigma(x)},
\end{align*}
So since the functions used above are continuous, then by the Weierstrauss M-test, $\bar{f}$ is continuous. \qedhere
\end{itemize}
\end{proof}
\end{enumerate}

\pagebreak
\begin{definition*}
A function $f:\R^n\to\R$ is called \textbf{lower semi-continuous at the point} $x\in\R^n$ if, for any sequence $x_k\in\R^n$ with $x_k\to x$ one has 
$$\liminf_{k\to\infty	}f(x_k)\geq f(x).$$
\end{definition*}

\begin{enumerate}[resume]
\item Prove that any lower semi-continuous function is Borel measurable. 
\begin{proof}
Consider ${f}^{-1}(-\infty,a]$. If ${f}^{-1}(-\infty,a]$ is closed, then $f$ is Borel measurable. Let $x_n$ be any convergent sequence in ${f}^{-1}(-\infty,a]$, and say that $x_n\to\gamma$, then $\gamma$ is an arbitrary limit point of ${f}^{-1}(-\infty,a]$. Since $f$ is lower semi-continuous, then 
$$\liminf_{n\to\infty} f(x_n) \geq f(\gamma).$$
Since $a\geq f(x_n)$ for all $n$, then 
$$a\geq\liminf_{n\to\infty} f(x_n) \geq f(\gamma),$$
so ${f}^{-1}(-\infty,a]$ contains all its limit points and thus is closed. 
\end{proof}

\item Prove the following statements:
	\begin{enumerate}[label=(\roman*)]
	\item Let $a<b$ and $a_k<b_k$ for $k\in\N$. If
	$$[a,b)\subseteq\bigcup_{k=1}^\infty[a_k,b_k),$$
	then 
	$$b-a\leq\sum_{k=1}^\infty(b_k-a_k).$$
	\begin{proof}
	\WLOG{} suppose that there are no extraneous intervals, that is,  for all $i, j$ we have $[a,b)\cap[{a}_i,{b}_i)\neq\emptyset$ and $[a_i,b_i)\not\subseteq[a_j,b_j)$. Let $\epsilon>0$. Then 	$[a,b-\epsilon]\subseteq\bigcup_{k=1}^\infty[a_k,b_k),$ and $[a,b-\epsilon]$ is compact, so there exists a finite subcover\footnote{It's very late. I just realized that this doesn't work, because this isn't an open cover. I think that it can be fixed by using $(a_k-\frac{\epsilon}{2^k},b_k)$, but I can't fix it tonight.}
	$$[a,b-\epsilon]\subseteq\bigcup_{i=1}^n[a_{k_i},b_{k_i}).$$
	
	For any $i,j$ such that $[a_i,b_i)\cap[a_j,b_j)\neq\emptyset$, we can write 
	$$[a_i,b_i)\cup[a_j,b_j)=[a_i,a_j)\cup[a_j,b_i)\cup[b_i,b_j),$$
	and note that 
	\begin{align*}
	(b_i-a_i)+(b_j-a_j)&=(a_j-a_i)+2(b_i-a_j)+(b_j-b_i)\\
	&>(a_j-a_i)+(b_i-a_j)+(b_j-b_i).
	\end{align*}
	So any finite nondisjoint union of intervals $[a_i,b_i)$ can be rewritten as a finite disjoint union with smaller length. Thus we can renumber and write 
	$$[a,b-\epsilon]\subseteq\coprod_{i=1}^n[\hat{a}_i,\hat{b}_i)=\bigcup_{i=1}^n[a_{k_i},b_{k_i}).$$
	Since there are no extraneous intervals, then $\hat{a}_1\leq a$, and $b-\epsilon < \hat{b}_n$, and $\hat{b}_i=\hat{a}_{i+1} $ for all $i$. Thus 
	$$(b-a)-\epsilon\leq (\hat{b}_n-\hat{a}_1) = \sum_{i=1}^n(\hat{b}_i-\hat{a}_i) < \sum_{i=1}^n(b_{k_i}-a_{k_i}) < \sum_{k=1}^\infty(b_{k}-a_{k}),$$
	Since this holds for all $\epsilon>0$, we can let $\epsilon\to0$ and find that 
		$$b-a\leq\sum_{k=1}^\infty(b_k-a_k),$$
		as desired. 
	\end{proof}
	\item Let $[a_k,b_k)$ be disjoint intervals and $c_k<d_k$ for all $k$. If 
	$$\bigcup_{k=1}^\infty[a_k,b_k)\subseteq\bigcup_{k=1}^\infty[c_k,d_k),$$
	then 
	$$\sum_{k=1}^\infty(b_k-a_k)\leq\sum_{k=1}^\infty(d_k-c_k).$$
	\begin{proof}
	For every $k,i\in\N$, if $[a_k,b_k)\cap[c_i,d_i)\neq\emptyset$ and   $[a_{k+1},b_{k+1})\cap[c_i,d_i)\neq\emptyset$, then split $[c_i,d_i)$ at $\frac{b_k+a_{k+1}}{2}$, that is, remove $[c_i,d_i)$ from the collection and replace it with $[c_i,\frac{b_k+a_{k+1}}{2})$ and $[\frac{b_k+a_{k+1}}{2},d_i)$. Then after renumbering, we have that 
	$$\bigcup_{k=1}^\infty[a_k,b_k)\subseteq\bigcup_{1\leq i,k < \infty}[\hat{c}_{k_i},\hat{d}_{k_i})=\bigcup_{k=1}^\infty[c_k,d_k),$$
	where $[a_k,b_k)\subseteq\bigcup_{i=1}^\infty[\hat{c}_{k_i},\hat{d}_{k_i})$ for all $k$. We know from the previous problem that 
	$$(b_k-a_k)\leq\sum_{i=1}^\infty(\hat{d}_{k_i}- \hat{c}_{k_i})$$
	for all $k$, so
	\begin{align*}
	\sum_{k=1}^\infty(b_k-a_k)&\leq\sum_{k=1}^\infty\left(\sum_{i=1}^\infty(\hat{d}_{k_i}- \hat{c}_{k_i})\right)\\	
	&=\sum_{1\leq i,k < \infty}(\hat{d}_{k_i}- \hat{c}_{k_i})\\
	&=\sum_{k=1}^\infty(d_k-c_k),
	\end{align*}
	and we're done.
	\end{proof}
	\end{enumerate}

\item Prove that if a Lebesgue measurable set $A\subset\R$ has positive Lebesgue measure, then the set 
$$A-A=\{a-b:a,b\in A\}$$
contains a neighborhood of the origin. Is the statement true if one only assumes $\measure{A} > 0$ (i.e., $A$ is not Lebesgue measurable)?
\begin{proof}
Since $A$ is Lebesgue measurable, then we can approximate $A$ with a compact set $K\subseteq A$ and an open set $U\supseteq A$ such that $\measure{U}-\measure{K}<\epsilon$, for any $\epsilon>0$. Since $K$ compact and $U^\complement$ closed with $K, U^\complement$ disjoint, then $\dist{K}{U^\complement}>0$. If we let $0<\delta<\dist{K}{U^\complement}$, then 
$$K+(-\delta,\delta)\subset U$$
because $\dist{k}{U^\complement}>\delta$ for all $k\in K$. Now we will show that for any $r$ with $\abs{r}<\delta$, that $K\cap K+r\neq\emptyset$ and $B_\delta(0)\subset K-K \subset A-A$. Suppose for contradiction that $\abs{r}<\delta$ and $K\cap K+r=\emptyset$. Since $K,K+r$ are measurable and disjoint, and Lebesgue measure is translation invariant, then 
$$\measure{K\cup(K+r)}=2\measure{K}.$$
Since $K\cup(K+r)\subseteq U$, then 
$$\measure{K\cup(K+r)}\leq\measure{K}+\epsilon,$$
But for $\epsilon<\measure{K}$, this is a contradiction. 
\end{proof}
\answer If one does not assume that $A$ is measurable, the result does not hold. For example, let $A=\script{V}$, a Vitali set in $[0,1]$ constructed in the usual way. Then $\measure{\script{V}}=1>0$,  but $\script{V}-\script{V}$ contains no rational numbers except 0 by the construction of $\script{V}$. 
\end{enumerate}

\end{document}
