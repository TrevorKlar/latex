\documentclass[12pt,letterpaper]{article}

\usepackage{fancyhdr,fancybox}

%% Useful packages
\usepackage{amssymb, amsmath, amsthm} 
%\usepackage{graphicx}  %%this is currently enabled in the default document, so it is commented out here. 
\usepackage{calrsfs}
\usepackage{braket}
\usepackage{mathtools}
\usepackage{lipsum}
\usepackage{tikz}
\usetikzlibrary{cd}
\usepackage{verbatim}
%\usepackage{ntheorem}% for theorem-like environments
\usepackage{mdframed}%can make highlighted boxes of text
%Use case: https://tex.stackexchange.com/questions/46828/how-to-highlight-important-parts-with-a-gray-background
\usepackage{wrapfig}
\usepackage{centernot}
\usepackage{subcaption}%\begin{subfigure}{0.5\textwidth}
\usepackage{pgfplots}
\pgfplotsset{compat=1.13}
\usepackage[colorinlistoftodos]{todonotes}
\usepackage[colorlinks=true, allcolors=blue]{hyperref}
\usepackage{xfrac}					%to make slanted fractions \sfrac{numerator}{denominator}
\usepackage{enumitem}            
    %syntax: \begin{enumerate}[label=(\alph*)]
    %possible arguments: f \alph*, \Alph*, \arabic*, \roman* and \Roman*
\usetikzlibrary{arrows,shapes.geometric,fit}

\DeclareMathAlphabet{\pazocal}{OMS}{zplm}{m}{n}
%% Use \pazocal{letter} to typeset a letter in the other kind 
%%  of math calligraphic font. 

%% This puts the QED block at the end of each proof, the way I like it. 
\renewenvironment{proof}{{\bfseries Proof}}{\qed}
\makeatletter
\renewenvironment{proof}[1][\bfseries \proofname]{\par
  \pushQED{\qed}%
  \normalfont \topsep6\p@\@plus6\p@\relax
  \trivlist
  %\itemindent\normalparindent
  \item[\hskip\labelsep
        \scshape
    #1\@addpunct{}]\ignorespaces
}{%
  \popQED\endtrivlist\@endpefalse
}
\makeatother

%% This adds a \rewnewtheorem command, which enables me to override the settings for theorems contained in this document.
\makeatletter
\def\renewtheorem#1{%
  \expandafter\let\csname#1\endcsname\relax
  \expandafter\let\csname c@#1\endcsname\relax
  \gdef\renewtheorem@envname{#1}
  \renewtheorem@secpar
}
\def\renewtheorem@secpar{\@ifnextchar[{\renewtheorem@numberedlike}{\renewtheorem@nonumberedlike}}
\def\renewtheorem@numberedlike[#1]#2{\newtheorem{\renewtheorem@envname}[#1]{#2}}
\def\renewtheorem@nonumberedlike#1{  
\def\renewtheorem@caption{#1}
\edef\renewtheorem@nowithin{\noexpand\newtheorem{\renewtheorem@envname}{\renewtheorem@caption}}
\renewtheorem@thirdpar
}
\def\renewtheorem@thirdpar{\@ifnextchar[{\renewtheorem@within}{\renewtheorem@nowithin}}
\def\renewtheorem@within[#1]{\renewtheorem@nowithin[#1]}
\makeatother

%% This makes theorems and definitions with names show up in bold, the way I like it. 
\makeatletter
\def\th@plain{%
  \thm@notefont{}% same as heading font
  \itshape % body font
}
\def\th@definition{%
  \thm@notefont{}% same as heading font
  \normalfont % body font
}
\makeatother

%===============================================
%==============Shortcut Commands================
%===============================================
\newcommand{\ds}{\displaystyle}
\newcommand{\B}{\mathcal{B}}
\newcommand{\C}{\mathbb{C}}
\newcommand{\F}{\mathbb{F}}
\newcommand{\N}{\mathbb{N}}
\newcommand{\R}{\mathbb{R}}
\newcommand{\Q}{\mathbb{Q}}
\newcommand{\T}{\mathcal{T}}
\newcommand{\Z}{\mathbb{Z}}
\renewcommand\qedsymbol{$\blacksquare$}
\newcommand{\qedwhite}{\hfill\ensuremath{\square}}
\newcommand*\conj[1]{\overline{#1}}
\newcommand*\closure[1]{\overline{#1}}
\newcommand*\mean[1]{\overline{#1}}
%\newcommand{\inner}[1]{\left< #1 \right>}
\newcommand{\inner}[2]{\left< #1, #2 \right>}
\newcommand{\powerset}[1]{\pazocal{P}(#1)}
%% Use \pazocal{letter} to typeset a letter in the other kind 
%%  of math calligraphic font. 
\newcommand{\cardinality}[1]{\left| #1 \right|}
\newcommand{\domain}[1]{\mathcal{D}(#1)}
\newcommand{\image}{\text{Im}}
\newcommand{\inv}[1]{#1^{-1}}
\newcommand{\preimage}[2]{#1^{-1}\left(#2\right)}
\newcommand{\script}[1]{\mathcal{#1}}


\newenvironment{highlight}{\begin{mdframed}[backgroundcolor=gray!20]}{\end{mdframed}}

\DeclarePairedDelimiter\ceil{\lceil}{\rceil}
\DeclarePairedDelimiter\floor{\lfloor}{\rfloor}

%===============================================
%===============My Tikz Commands================
%===============================================
\newcommand{\drawsquiggle}[1]{\draw[shift={(#1,0)}] (.005,.05) -- (-.005,.02) -- (.005,-.02) -- (-.005,-.05);}
\newcommand{\drawpoint}[2]{\draw[*-*] (#1,0.01) node[below, shift={(0,-.2)}] {#2};}
\newcommand{\drawopoint}[2]{\draw[o-o] (#1,0.01) node[below, shift={(0,-.2)}] {#2};}
\newcommand{\drawlpoint}[2]{\draw (#1,0.02) -- (#1,-0.02) node[below] {#2};}
\newcommand{\drawlbrack}[2]{\draw (#1+.01,0.02) --(#1,0.02) -- (#1,-0.02) -- (#1+.01,-0.02) node[below, shift={(-.01,0)}] {#2};}
\newcommand{\drawrbrack}[2]{\draw (#1-.01,0.02) --(#1,0.02) -- (#1,-0.02) -- (#1-.01,-0.02) node[below, shift={(+.01,0)}] {#2};}

%***********************************************
%**************Start of Document****************
%***********************************************
 %find me at /home/trevor/texmf/tex/latex/tskpreamble_nothms.tex
%===============================================
%===============Theorem Styles==================
%===============================================

%================Default Style==================
\theoremstyle{plain}% is the default. it sets the text in italic and adds extra space above and below the \newtheorems listed below it in the input. it is recommended for theorems, corollaries, lemmas, propositions, conjectures, criteria, and (possibly; depends on the subject area) algorithms.
\newtheorem{theorem}{Theorem}
\numberwithin{theorem}{section} %This sets the numbering system for theorems to number them down to the {argument} level. I have it set to number down to the {section} level right now.
\newtheorem*{theorem*}{Theorem} %Theorem with no numbering
\newtheorem{corollary}[theorem]{Corollary}
\newtheorem*{corollary*}{Corollary}
\newtheorem{conjecture}[theorem]{Conjecture}
\newtheorem{lemma}[theorem]{Lemma}
\newtheorem*{lemma*}{Lemma}
\newtheorem{proposition}[theorem]{Proposition}
\newtheorem*{proposition*}{Proposition}
\newtheorem{problemstatement}[theorem]{Problem Statement}


%==============Definition Style=================
\theoremstyle{definition}% adds extra space above and below, but sets the text in roman. it is recommended for definitions, conditions, problems, and examples; i've alse seen it used for exercises.
\newtheorem{definition}[theorem]{Definition}
\newtheorem*{definition*}{Definition}
\newtheorem{condition}[theorem]{Condition}
\newtheorem{problem}[theorem]{Problem}
\newtheorem{example}[theorem]{Example}
\newtheorem*{example*}{Example}
\newtheorem*{counterexample*}{Counterexample}
\newtheorem*{romantheorem*}{Theorem} %Theorem with no numbering
\newtheorem{exercise}{Exercise}
\numberwithin{exercise}{section}
\newtheorem{algorithm}[theorem]{Algorithm}

%================Remark Style===================
\theoremstyle{remark}% is set in roman, with no additional space above or below. it is recommended for remarks, notes, notation, claims, summaries, acknowledgments, cases, and conclusions.
\newtheorem{remark}[theorem]{Remark}
\newtheorem*{remark*}{Remark}
\newtheorem{notation}[theorem]{Notation}
\newtheorem*{notation*}{Notation}
%\newtheorem{claim}[theorem]{Claim}  %%use this if you ever want claims to be numbered
\newtheorem*{claim}{Claim}


\DeclareMathOperator{\diameter}{diam}
\newcommand{\diam}[1]{\diameter\left(#1\right)}

%%
%% Page set-up:
%%
\pagestyle{empty}
\lhead{\textsc{201 - Real Analysis \\}} 
\rhead{\textsc{Harutyunyan, Fall 2019} \\ Trevor Klar}
%\chead{\Large\textbf{A New Integration Technique \\ }}
\renewcommand{\headrulewidth}{0pt}
%
\renewcommand{\footrulewidth}{0pt}
%\lfoot{
%Office: \quad \quad \, M 2-3 \, \, SH 6431x \\
%Math Lab: \, W 12-2 \, SH 1607
%}
%\rfoot{trevorklar@math.ucsb.edu}


\setlength{\parindent}{0in}
\setlength{\textwidth}{7in}
\setlength{\evensidemargin}{-0.25in}
\setlength{\oddsidemargin}{-0.25in}
\setlength{\parskip}{.5\baselineskip}
\setlength{\topmargin}{-0.5in}
\setlength{\textheight}{9in}

\setlist[enumerate,1]{label=\textbf{\arabic*.}}

\begin{document}
\pagestyle{fancy}
\begin{center}
{\Large Homework 4}%=================UPDATE THIS=================%
\end{center}

\begin{enumerate}
\item Let $X$ be a nonempty topological space and let $\mu$ be a measure on $X$. Prove that if the functions $f_n:X\to [-\infty,+\infty]$ are \mumeasurable{} for $n\in \N$, then the set 
$$A=\{x\in X : \lim_{n\to\infty}f_n(x) \text{ exists}\}$$
is \mumeasurable{}. 
\begin{proof}
To simplify notation, denote $f^*(x)=\limsup\limits_{n\to\infty} f_n(x)$ and $f_*(x)=\liminf\limits_{n\to\infty} f_n(x)$. 
Let $F:\R\to\R$ be 
$$F(x)=f^*(x)-f_*(x)
$$
%\begin{cases}
%, & \text{when } f_*(x) < \infty\\
%0, & 
%\end{cases}$$
Actually, $F$ is only defined on a subset of $\R$, that when the limsup and the liminf are not both infinite. 
%We will handle this case separately. 
Observe that $F$ is \mumeasurable{}, since it is a sum of two \mumeasurable{} functions, and 
$$\preimage{F}{\{0\}}\cup\{x\in \R:F(x) \text{ is undefined}\}=A,$$
since $F(x)$ is undefined exactly when $\lim_{n\to\infty} f_n(x)=\pm\infty$ and $F(x)=0$ exactly when $\lim_{n\to\infty} f_n(x)$ exists and is finite. Thus, 
$$A={F}^{-1}{\{0\}}
\cup \big(f_*^{-1}{\{+\infty\}} \cap {{f^*}^{-1}}{\{+\infty\}}\big)
\cup \big({f_*^{-1}}{\{-\infty\}} \cap {{f^*}^{-1}}{\{-\infty\}}\big).
$$
Now we show that each of the above sets is \mumeasurable{}, which means that $A$ consists of unions and intersections of \mumeasurable{} sets, and thus $A$ is \mumeasurable{}. 
\begin{itemize}
	\item $F$ is always positive since $f^*\geq f_*$ everywhere. So $F^{-1}\{0\}=F^{-1}[-\infty, 0]$ and thus is \mumeasurable{}. 
	\item For any \mumeasurable{} $f$ (including $f^*$ and $f_*$), we have that 
	\begin{align*}
	{f}^{-1}{\{\infty\}}&=\left(\bigcup_{n=1}^\infty	f^{-1}[-\infty,n)\right)^\complement\\
	{f}^{-1}{\{-\infty\}}&=\left(\bigcup_{n=1}^\infty	f^{-1}[n,\infty]\right)^\complement\\
	\end{align*}		
	and thus is measurable.
\end{itemize}
\end{proof}



\pagebreak
\item Prove that any Lebesgue-measurable function $f:\R\to\R$ that satisfies the relation 
$$f(x+y)=f(x)+f(y)\quad \text{ for all } \quad x,y\in\R$$
must be linear. 
\begin{proof}
We need to show that for all $\lambda\in \R$, 
\begin{equation} \tag{$\dagger$}
f(\lambda x)=\lambda f(x).
\end{equation}
First, observe that for any $n\in \N$,
$$f(nx)=f(\,\overbrace{x+\dots+x}^n\,)=\overbrace{f(x)+\dots+f(x)}^n=nf(x),$$
so $(\dagger)$ holds for $\lambda\in \N$. Next, observe that 
$$f(x)=f\bigg(\,\overbrace{\frac{x}{n}+\dots+\frac{x}{n}}^n\,\bigg)=\overbrace{f\left(\frac{x}{n}\right)+\dots+f\left(\frac{x}{n}\right)}^n=nf\left(\frac{x}{n}\right),$$
so $\frac{1}{n}f\left(x\right)=f\left(\frac{x}{n}\right)$, which together with the previous result means that for every $\frac{p}{q}\in\Q$, 
$$f\left(\frac{p}{q}x\right)=pf\left(\frac{x}{q}\right)=\frac{p}{q}f\left(x\right),$$
so $(\dagger)$ holds for $\lambda\in \Q$. To prove the final result, we will need the following lemma.
	\begin{itemize}
	\item[\,]\vspace*{-18pt}
	%\begin{lemma}
	%$f$ is differentiable, and thus continuous. 
	%\end{lemma}
	%\textbf{Proof of Lemma}
	%For all $x\in \R$,
	%$$\lim_{h\to0} \frac{f(x+h)-f(x)}{h}= 
	%\lim_{h\to0} \frac{f(x)+f(h)-f(x)}{h}=
	%\lim_{h\to0} \frac{f(h)}{h}=
	%%\lim_{h\to0} \frac{1}{h}f(h)=
	%\lim_{h\to0} f(1)=
	%f(1)
	%$$
	\begin{lemma*}
	$f$ is continuous at $x=0$. 
	\end{lemma*}
	Let $\varepsilon>0$. Since $\Q$ is dense in $\R$ (thought of as the codomain of $f$), then $\bigcup_{i=1}^\infty B_\frac{\varepsilon}{2}(q_i)=\R$, where $\infcoll{q}$ is an enumeration of the rationals. Since $f$ is measurable, then every $\preimage{f}{B_\frac{\varepsilon}{2}(q_i)}$ is measurable. Since $\preimage{f}{\R}=\R$ and 
	$$\preimage{f}{\R}=\preimage{f}{\bigcup_{i=1}^\infty B_\frac{\varepsilon}{2}(q_i)}=\bigcup_{i=1}^\infty\preimage{f}{B_\frac{\varepsilon}{2}(q_i)},$$
	then $\bigcup_{i=1}^\infty\preimage{f}{B_\frac{\varepsilon}{2}(q_i)}$ covers $\R$ so at least one of them has positive measure, by subadditivity. Say $\preimage{f}{B_\frac{\varepsilon}{2}(q_k)}$ does and call it $A$. Since $\measure{A}>0$ then $A-A$ contains a neighborhood of zero, call it $B_\delta(0)$. Any element $x$ of $B_\delta(0)\subset A-A$ can be written in the form $x=a_1-a_2$, so $f(x)=f(a_1-a_2)=f(a_1)-f(a_2)$. That is, 
	$$f(B_\delta(0))\subseteq f(A)-f(A)\subset B_\frac{\varepsilon}{2}(q_k)-B_\frac{\varepsilon}{2}(q_k)\subset B_\varepsilon(0),$$
	so $f$ is continuous at 0. 
	%\indent \textbf{Proof of Lemma}
	%Suppose for contradiction that $f$ is not continuous at 0. Then there exists some $\varepsilon>0$ such that for every $\delta>0$, there exists some $x$ such that $0<\abs{x-0}=<\delta$ and $\abs{f(x)-f(0)}\geq\varepsilon$. [the rest of this is wrong]
	%
	%We can simplify the previous statement since for every $x\in\R$,
	%$$f(0)+f(x)=f(x+0)=f(x) \text{, then }f(0)=0.$$ 
	%
	%Also, observe that assuming that $\abs{x}<\delta$ for any $\delta>0$ places no constraint on $x$\footnote[2]{If this is not apparent, suppose that for every $\delta>0$, if $\abs{x}<\delta$, some proposition $P(x)$ is true. Observe that $P(x)$ is true for all $x\in\R$: Let $x$ be arbitrary, and let $\delta=\abs{x}+1$. Then $\abs{x}<\delta$ for some $\delta$, so $P(x)$ is true.}, so we can say
	%$$\text{there exists some }\varepsilon>0\text{ such that }\abs{f(x)}\geq \varepsilon\text{ for all }x.$$
	%
	%Let $x$ be any real number, and suppose \Wlog{} that $\abs{f(x)}=f(x)$. Then $f(x)\geq \varepsilon$, but there exists some $n\in\N$ such that $\frac{1}{n}f(x)<\varepsilon$, which means that $f\left(\frac{x}{n}\right)<\varepsilon$, contradiction. 
	\qedwhite
	\end{itemize}
	
We can use this to show that $f$ is continuous everywhere. For any $x\in\R$, 
\begin{align*}
\lim_{h\to0}\left[f(x+h)\right]&=\lim_{h\to0}\left[f(x)+f(h)\right]\\
&=\lim_{h\to0}f(x)+\lim_{h\to0}f(h)\\
&=f(x)+f(0)\\
&=f(x)+0
\end{align*}

%Now we can show that $(\dagger)$ holds over all of $\R$. 
Let $\lambda\in\R$, and let $\{q_n\}_{n=1}^\infty$ be a sequence of rational numbers that converges to $\lambda$. Then,
\begin{align*}
f(\lambda x)&=f\left(\lim_{n\to\infty}q_n x\right) \quad\quad \text{and since $f$ is continuous,}\\
&=\lim_{n\to\infty}f\left(q_n x\right)\\
&=\lim_{n\to\infty}q_nf\left( x\right)\\
&=\lambda f(x)
\end{align*}
so $(\dagger)$ holds for $\lambda\in \R$, and we are done. 
\end{proof}

\pagebreak
\item Let $f:(0,1)\to\R$ be such that for every $x\in (0,1)$ there exists $\delta_x>0$ and a Borel-measurable function $g_x:\R\to\R$ (both dependent on $x$), such that $f(y)=g_x(y)$ for all $y\in D_x$, where $D_x=B_{\delta_x}(x)\cap (0,1)$. Prove that $f$ is Borel-measurable. 
\begin{proof}
Observe that $\{D_x\}_{x\in(0,1)}$ is an open cover of $(0,1)$. Since we know that any open set is a countable union of open balls with rational radii and centers, we can produce a countable subcover. Let $\{B_i\}_{i=1}^\infty$ be an enumeration of the rational balls in $\R$, and let 
$$\Gamma=\{i\in\N \mid B_i\subset D_x \text{ for some } x\in(0,1)\}.$$
Then $\{B_i\}_{i\in\Gamma}$ covers $(0,1)$ and for each $i\in\Gamma$, we can choose a corresponding $x_i$ such that $B_i\subset D_{x_i}$. Then 
$\{D_{x_i}\}_{i\in\Gamma}$
is a countable subcover of $\{D_x\}_{x\in(0,1)}$. 

Let $I\in\R$ be any open interval. Then 
\begin{align*}
\preimage{f}{I}&=\bigcup_{i\in\Gamma}\left(f\big|_{D_{x_i}}\right)^{-1}(I)\\
&=\bigcup_{i\in\Gamma}\left(g_{x_i}\big|_{D_{x_i}}\right)^{-1}(I)\\
&=\bigcup_{i\in\Gamma}\left\lbrace y\in \big(B_{\delta_{x_i}}({x_i})\cap (0,1)\big) \mid g_{x_i}(y)\in I\right\rbrace\\
&=\bigcup_{i\in\Gamma} B_{\delta_{x_i}}({x_i})\cap (0,1) \cap g^{-1}_{x_i}(I)\\
\end{align*}
which is a Borel set since $g_x$ is Borel-measurable for all $x$, so it is a countable union of Borel sets. Thus $f$ is Borel-measurable.
\end{proof}

\item Give an example of a collection of Lebesgue-measurable functions $\{f_\alpha\}_{\alpha\in A}$ where each \linebreak $f_\alpha:\R\to\R$ and the function
$$g(x)=\sup_{\alpha\in A} f_\alpha(x), \quad \quad x\in\R$$
is finite for all $x\in\R$ but $g$ is not Lebesgue-measurable. Here $A$ is a nonempty index set. 
\answer Let $V\subset[0,1]$ be a Vitali set, and for each $\alpha\in V$, let $f_\alpha=\Chi_{\{\alpha\}}$. Then 
$$g(x)=\sup_{\alpha\in V} f_\alpha(x)=\Chi_V$$
which is not Lebesgue-measurable, as we know. 


\end{enumerate}
\end{document}
