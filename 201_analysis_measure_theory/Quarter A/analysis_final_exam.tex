\documentclass[12pt,letterpaper]{article}

\usepackage{fancyhdr,fancybox}

%% Useful packages
\usepackage{amssymb, amsmath, amsthm} 
%\usepackage{graphicx}  %%this is currently enabled in the default document, so it is commented out here. 
\usepackage{calrsfs}
\usepackage{braket}
\usepackage{mathtools}
\usepackage{lipsum}
\usepackage{tikz}
\usetikzlibrary{cd}
\usepackage{verbatim}
%\usepackage{ntheorem}% for theorem-like environments
\usepackage{mdframed}%can make highlighted boxes of text
%Use case: https://tex.stackexchange.com/questions/46828/how-to-highlight-important-parts-with-a-gray-background
\usepackage{wrapfig}
\usepackage{centernot}
\usepackage{subcaption}%\begin{subfigure}{0.5\textwidth}
\usepackage{pgfplots}
\pgfplotsset{compat=1.13}
\usepackage[colorinlistoftodos]{todonotes}
\usepackage[colorlinks=true, allcolors=blue]{hyperref}
\usepackage{xfrac}					%to make slanted fractions \sfrac{numerator}{denominator}
\usepackage{enumitem}            
    %syntax: \begin{enumerate}[label=(\alph*)]
    %possible arguments: f \alph*, \Alph*, \arabic*, \roman* and \Roman*
\usetikzlibrary{arrows,shapes.geometric,fit}

\DeclareMathAlphabet{\pazocal}{OMS}{zplm}{m}{n}
%% Use \pazocal{letter} to typeset a letter in the other kind 
%%  of math calligraphic font. 

%% This puts the QED block at the end of each proof, the way I like it. 
\renewenvironment{proof}{{\bfseries Proof}}{\qed}
\makeatletter
\renewenvironment{proof}[1][\bfseries \proofname]{\par
  \pushQED{\qed}%
  \normalfont \topsep6\p@\@plus6\p@\relax
  \trivlist
  %\itemindent\normalparindent
  \item[\hskip\labelsep
        \scshape
    #1\@addpunct{}]\ignorespaces
}{%
  \popQED\endtrivlist\@endpefalse
}
\makeatother

%% This adds a \rewnewtheorem command, which enables me to override the settings for theorems contained in this document.
\makeatletter
\def\renewtheorem#1{%
  \expandafter\let\csname#1\endcsname\relax
  \expandafter\let\csname c@#1\endcsname\relax
  \gdef\renewtheorem@envname{#1}
  \renewtheorem@secpar
}
\def\renewtheorem@secpar{\@ifnextchar[{\renewtheorem@numberedlike}{\renewtheorem@nonumberedlike}}
\def\renewtheorem@numberedlike[#1]#2{\newtheorem{\renewtheorem@envname}[#1]{#2}}
\def\renewtheorem@nonumberedlike#1{  
\def\renewtheorem@caption{#1}
\edef\renewtheorem@nowithin{\noexpand\newtheorem{\renewtheorem@envname}{\renewtheorem@caption}}
\renewtheorem@thirdpar
}
\def\renewtheorem@thirdpar{\@ifnextchar[{\renewtheorem@within}{\renewtheorem@nowithin}}
\def\renewtheorem@within[#1]{\renewtheorem@nowithin[#1]}
\makeatother

%% This makes theorems and definitions with names show up in bold, the way I like it. 
\makeatletter
\def\th@plain{%
  \thm@notefont{}% same as heading font
  \itshape % body font
}
\def\th@definition{%
  \thm@notefont{}% same as heading font
  \normalfont % body font
}
\makeatother

%===============================================
%==============Shortcut Commands================
%===============================================
\newcommand{\ds}{\displaystyle}
\newcommand{\B}{\mathcal{B}}
\newcommand{\C}{\mathbb{C}}
\newcommand{\F}{\mathbb{F}}
\newcommand{\N}{\mathbb{N}}
\newcommand{\R}{\mathbb{R}}
\newcommand{\Q}{\mathbb{Q}}
\newcommand{\T}{\mathcal{T}}
\newcommand{\Z}{\mathbb{Z}}
\renewcommand\qedsymbol{$\blacksquare$}
\newcommand{\qedwhite}{\hfill\ensuremath{\square}}
\newcommand*\conj[1]{\overline{#1}}
\newcommand*\closure[1]{\overline{#1}}
\newcommand*\mean[1]{\overline{#1}}
%\newcommand{\inner}[1]{\left< #1 \right>}
\newcommand{\inner}[2]{\left< #1, #2 \right>}
\newcommand{\powerset}[1]{\pazocal{P}(#1)}
%% Use \pazocal{letter} to typeset a letter in the other kind 
%%  of math calligraphic font. 
\newcommand{\cardinality}[1]{\left| #1 \right|}
\newcommand{\domain}[1]{\mathcal{D}(#1)}
\newcommand{\image}{\text{Im}}
\newcommand{\inv}[1]{#1^{-1}}
\newcommand{\preimage}[2]{#1^{-1}\left(#2\right)}
\newcommand{\script}[1]{\mathcal{#1}}


\newenvironment{highlight}{\begin{mdframed}[backgroundcolor=gray!20]}{\end{mdframed}}

\DeclarePairedDelimiter\ceil{\lceil}{\rceil}
\DeclarePairedDelimiter\floor{\lfloor}{\rfloor}

%===============================================
%===============My Tikz Commands================
%===============================================
\newcommand{\drawsquiggle}[1]{\draw[shift={(#1,0)}] (.005,.05) -- (-.005,.02) -- (.005,-.02) -- (-.005,-.05);}
\newcommand{\drawpoint}[2]{\draw[*-*] (#1,0.01) node[below, shift={(0,-.2)}] {#2};}
\newcommand{\drawopoint}[2]{\draw[o-o] (#1,0.01) node[below, shift={(0,-.2)}] {#2};}
\newcommand{\drawlpoint}[2]{\draw (#1,0.02) -- (#1,-0.02) node[below] {#2};}
\newcommand{\drawlbrack}[2]{\draw (#1+.01,0.02) --(#1,0.02) -- (#1,-0.02) -- (#1+.01,-0.02) node[below, shift={(-.01,0)}] {#2};}
\newcommand{\drawrbrack}[2]{\draw (#1-.01,0.02) --(#1,0.02) -- (#1,-0.02) -- (#1-.01,-0.02) node[below, shift={(+.01,0)}] {#2};}

%***********************************************
%**************Start of Document****************
%***********************************************
 %find me at /home/trevor/texmf/tex/latex/tskpreamble_nothms.tex
%===============================================
%===============Theorem Styles==================
%===============================================

%================Default Style==================
\theoremstyle{plain}% is the default. it sets the text in italic and adds extra space above and below the \newtheorems listed below it in the input. it is recommended for theorems, corollaries, lemmas, propositions, conjectures, criteria, and (possibly; depends on the subject area) algorithms.
\newtheorem{theorem}{Theorem}
\numberwithin{theorem}{section} %This sets the numbering system for theorems to number them down to the {argument} level. I have it set to number down to the {section} level right now.
\newtheorem*{theorem*}{Theorem} %Theorem with no numbering
\newtheorem{corollary}[theorem]{Corollary}
\newtheorem*{corollary*}{Corollary}
\newtheorem{conjecture}[theorem]{Conjecture}
\newtheorem{lemma}[theorem]{Lemma}
\newtheorem*{lemma*}{Lemma}
\newtheorem{proposition}[theorem]{Proposition}
\newtheorem*{proposition*}{Proposition}
\newtheorem{problemstatement}[theorem]{Problem Statement}


%==============Definition Style=================
\theoremstyle{definition}% adds extra space above and below, but sets the text in roman. it is recommended for definitions, conditions, problems, and examples; i've alse seen it used for exercises.
\newtheorem{definition}[theorem]{Definition}
\newtheorem*{definition*}{Definition}
\newtheorem{condition}[theorem]{Condition}
\newtheorem{problem}[theorem]{Problem}
\newtheorem{example}[theorem]{Example}
\newtheorem*{example*}{Example}
\newtheorem*{counterexample*}{Counterexample}
\newtheorem*{romantheorem*}{Theorem} %Theorem with no numbering
\newtheorem{exercise}{Exercise}
\numberwithin{exercise}{section}
\newtheorem{algorithm}[theorem]{Algorithm}

%================Remark Style===================
\theoremstyle{remark}% is set in roman, with no additional space above or below. it is recommended for remarks, notes, notation, claims, summaries, acknowledgments, cases, and conclusions.
\newtheorem{remark}[theorem]{Remark}
\newtheorem*{remark*}{Remark}
\newtheorem{notation}[theorem]{Notation}
\newtheorem*{notation*}{Notation}
%\newtheorem{claim}[theorem]{Claim}  %%use this if you ever want claims to be numbered
\newtheorem*{claim}{Claim}


%%
%% Page set-up:
%%
\pagestyle{empty}
\lhead{\textsc{201 - Real Analysis \\}} 
\rhead{\textsc{Harutyunyan, Fall 2019} \\ Trevor Klar}
%\chead{\Large\textbf{A New Integration Technique \\ }}
\renewcommand{\headrulewidth}{0pt}
%
\renewcommand{\footrulewidth}{0pt}
%\lfoot{
%Office: \quad \quad \, M 2-3 \, \, SH 6431x \\
%Math Lab: \, W 12-2 \, SH 1607
%}
%\rfoot{trevorklar@math.ucsb.edu}


\setlength{\parindent}{0in}
\setlength{\textwidth}{7in}
\setlength{\evensidemargin}{-0.25in}
\setlength{\oddsidemargin}{-0.25in}
\setlength{\parskip}{.5\baselineskip}
\setlength{\topmargin}{-0.5in}
\setlength{\textheight}{9in}

\setlist[enumerate,1]{label=\textbf{\arabic*.}}

\begin{document}
\pagestyle{fancy}
\begin{center}
{{\LARGE Final Exam}}%=================UPDATE THIS=================%
\end{center}

\pagebreak
\begin{enumerate}
\item Let $X$ be a nonempty topological space and let $\{\mu_n\}_{n=1}^\infty$ be a sequence of Borel regular measures on $X$. Assume for any $A \subset X$ the sequence $\mu_n(A)$ decreases and define
$\mu(A) = \lim_{n\to\infty} \mu_n(A)$. Prove that if $\mu_1(X) < \infty$, then $\mu$ is a measure on $X$. 

\textbf{Lemma (MCT$\mathbf{\searrow}$)} The Monotone Convergence Theorem holds for nonnegative \mumeasurable{} functions $f_n\decreasesto f$\footnote{$f_n\decreasesto f$ means that $f_n\geq f_{n+1}$ for all $n$ and $\lim_{n\to\infty}f_n=f$.}, if $f_1$ is \musummable{}. \textbf{Proof}  $\{(f_1-f_n)\}_{n=1}^\infty$ is a nonnegative sequence of functions with $(f_1-f_n)\increasesto(f_1-f)$, so by the ordinary MCT
$$\lim_{n\to\infty}\int(f_1-f_n)=\int(f_1-f)$$
and so 
$$\lim_{n\to\infty}\int f_1-\lim_{n\to\infty}\int f_n=\int f_1-\int f$$ 
Thus $\lim_{n\to\infty}\int f_n = \int f$ and MCT$\searrow$ is proved. \qedwhite

\begin{proof}
We are given that 
	\begin{itemize}
	\item each $\mu_n$ is Borel regular,
	\item $\mu_1(X)<\infty$, and %which means that $\mu_n(A)<\infty$ for any $A\subset X$, $n\in\N$.
	\item $\mu_n(A)\decreasesto\mu(A)$ for any $A\subset X$.
	\end{itemize}

First, observe that $\mu_n(\emptyset)=0$ for all $n$, so $\mu(\emptyset)=0$. Now let $A\subset\bigcup_{i=1}^\infty A_i$, with $A,A_i\in X$ for all $i\in\N$. We need to show that 
$$\measure{A}\leq\sum_{i=1}^\infty\measure{A_i}.$$

Since each $\mu_n$ is a measure,
\begin{equation}
\mu_n(A)\leq\sum_{i=1}^\infty\mu_n(A_i)
\end{equation}
for all $n$. Now since for any $A$, $\mu_n(A)$ is a decreasing real sequence bounded below by 0, then it always converges, so taking limits in both sides of (1), 
\begin{equation}
\mu(A)\leq\lim_{n\to\infty}\sum_{i=1}^\infty \mu_n(A_i).
\end{equation}
%for all $n$. Now since $\mu_n(A)$ is a decreasing real sequence bounded below by 0, then it converges and  
%$$\mu(A)=\lim_{n\to\infty}\mu_n(A)=\liminf_{n\to\infty}\mu_n(A).$$
%So taking $\liminf$ of both sides of (1), 
%\begin{equation}
%\mu(A)\leq\liminf_{n\to\infty}\sum_{i=1}^\infty \mu_n(A_i).
%\end{equation}
Now we will view this sum as an integral. Let $f_n(x)=
\begin{cases}
\mu_n(A_i) \text{ where } i=\floor{x}, &\text{if } x\geq1\\
0 &\text{if } x<1
\end{cases}$.
Then each $f_n$ is simple and nonnegative, and the Lebesgue integral of $f_n$ is
$$\int_\R f_n(x)=\sum_{i=1}^\infty f_n(i)=\sum_{i=1}^\infty \mu_n(A_i),$$
and we can substitute into (2) to find 
\begin{equation}
\mu(A)\leq \lim_{n\to\infty}\int_\R f_n(x).
%\mu(A)\leq \liminf_{n\to\infty}\int_\R f_n(x).
\end{equation}

Observe that $\mu_1(A)\leq\mu_1(X)<\infty$ and $\mu_n\decreasesto\mu$, so $\mu(A)<\infty$ always. Considering the $A_i$ sets, for any $n\in\N$ either $\sum_{i=1}^\infty\mu_n(A_i)$ is finite or it is infinite. 

Case I: Suppose there exists some $K$ such that  $\sum_{i=1}^\infty\mu_K(A_i)$ is finite. 

Following are a few facts about the functions $f_n(x)=\mu_n(A_i)$:
	\begin{enumerate}[label=(\roman*)]
	\item Since $\mu_n\decreasesto\mu$, then $\sum_{i=1}^\infty\mu_k(A_i)<\infty$ for all $k>K$.
	\item Each $f_n$ is a nonnegative simple function, and thus measurable.  
	\item $f_k$ is $\mu$-summable for every $k>K$, since $\int_\R f_k(x)=\sum_{i=1}^\infty\mu_k(A_i)<\infty$. 
	\item $f_n\decreasesto f$, since $\mu_n\decreasesto\mu$. 
	\item $f_1$ is bounded above by $\mu_1(X)$, since every $A_i\subset X$ and $\mu_1$ has the monotonicity property. 
	\item $f$ is measurable by (i) and (iii) above. 
	\item We can assume $f$ is \musummable{}, since if not then $\sum_{i=1}^\infty\mu(A_i)=\int_\R f=\infty>\mu(A)$ and we're done. 
	\end{enumerate}
Let $g_n=f_{n+K}$. Now we can check that the hypotheses of MCT$\searrow$ are satisfied:
	\begin{itemize}
	\item $g_n$ are \mumeasurable{}
	\item $g_1$ is \musummable{}. 
	\item $g_n\decreasesto f$
	\end{itemize}
So we can apply MCT$\searrow$ and conclude that 
$$\lim_{n\to\infty}\int_\R f_n = \lim_{n\to\infty}\int_\R g_n = \int_\R f$$ 
so substituting into equation (3), we find that 
$$\mu(A)\leq\int_\R f(x)= \sum_{i=1}^\infty\mu(A_i)$$
and we are done.

%Case II: Suppose $\sum_{i=1}^\infty\mu_n(A_i)$ is infinite for every $n$, and let $B\in\R$ be any large number. Then for each $n$, there exists a smallest $I_n\in\N$ such that 
%$$\sum_{i=1}^{I_n}\mu_n(A_i)>B.$$ 
%Since $\mu_n(A_i)\to\mu(A_i)$, then for any $\varepsilon>0$, there exists $N$ such that $|\mu_n(A_i)-\mu(A_i)|<\varepsilon$ for every $n>N$. 

Case II: Suppose $\sum_{i=1}^\infty\mu_n(A_i)$ is infinite for every $n$.

Since each $\mu_n$ is a Borel regular measure and $\mu_1$ is in particular, for each $A_i$ there exists a respective Borel set $B_i$ such that $B_i\subset A_i$\footnote{The textbook gives the set containment the other way, but if we find Borel set $\tilde{B}_i$ with $A^\complement_i\subset\tilde{B}_i$, then $\tilde{B}_i^\complement$ is our desired $B_i$.} and $\mu_1(A_i)=\mu_1(B_i)$, so
$$\mu_1(A)\leq\sum_{i=1}^\infty\mu_1(B_i)=\sum_{i=1}^\infty\mu_1(A_i)$$
Let $D_1=B_1$, and $D_n=B_n\setminus\bigcup_{i=1}^{n-1} B_i$. Then $\{D_i\}_{i=1}^\infty$ is a disjoint collection of Borel (and thus measurable) sets with $\bigcup_{i=1}^\infty D_i = \bigcup_{i=1}^\infty B_i$. Observe that 
$$\sum_{i=1}^\infty \mu_1(D_i) = \mu_1 \left(\bigcup_{i=1}^\infty D_i\right)<\infty,$$
and for any $n$, since $\mu_n$ is Borel, 
$$\sum_{i=1}^\infty \mu_n(D_i) = \sum_{i=1}^\infty \left(\mu_n(B_i) - \mu_n\left(\bigcap\nolimits_{j=1}^i B_j\right)\right)\leq \sum_{i=1}^\infty \mu_n(B_i),$$
so we can apply Case I to $A\subset\bigcup_{i=1}^\infty  D_i$ to conclude that $\mu(A)\leq\sum_{i=1}^\infty \mu(D_i)$, and since $\mu$ has the monotonicity property\footnote{This is because $\mu_n$ is a measure, so $\mu_n(A)\leq\mu_n(B)$ for all $n$, and taking limits, $\mu(A)\leq\mu(B)$.} and $D_i\subset B_i \subset A_i$, 
$$\mu(A)\leq\sum_{i=1}^\infty \mu(D_i)\leq\sum_{i=1}^\infty \mu(B_i)\leq\sum_{i=1}^\infty \mu(A_i)$$
and we are done. 
\end{proof}

\pagebreak
\item Let $f:\R\to\R$ be Lebesgue-measurable. Prove that there exists a Borel-measurable function $g:\R\to\R$ such that $f(x)=g(x)$ \muae{} in $\R$. 
\begin{proof}
We will show that (i) every Lebesgue-measurable simple function has the desired property, and show that (ii) this implies nonnegative Lebesgue-measurable functions have the property, and thus (iii) all Lebesgue-measurable functions have the property. 
	\begin{enumerate}[label=(\roman*)]
	\item Let $\sigma	= \sum_{i=1}^\infty a_i \Chi_{A_i}$ be a nonnegative Lebesgue-measurable simple function with all $A_i$ sets pairwise disjoint and of finite measure\footnote{Such a disjoint collection of sets partitions $\R$, and if there are any with infinite measure, we can refine the partition by dividing the sets at every integer, i.e. if $A_i=(10,\infty)$, replace $A_i$ with $A_{i_1}=(10, 11], A_{i_2}=(12, 12],$ etc.}. 
%Then for any open set $U$, 
%$$\preimage{\sigma}{U}=\bigcup_{i\in\Gamma} A_i \text{, \quad where } \Gamma = \{i\in\N : a_i\in U\}$$
We know that for every Lebesgue-measurable set $L$ with finite measure, there exists a compact (and thus Borel) set $K$ such that $K\subset L$ and $\measure{L\setminus K}<\varepsilon$ for every $\varepsilon$. So for each $A_i$, we find a collection of compact sets $\{K_i^n\}_{n=1}^\infty$ such that $K_i^n\subset A_i$ and $\measure{A_i\setminus K_i^n}<\frac{1}{k}$. Then call $K_i=\bigcup_{n=1}^\infty K_i^n$, and $K_i\subset A_i$, $K_i$ is Borel, and $\measure{K_i}=\measure{A_i}$. 
%(In fact this is how the measure is defined\footnote{It's times like this that I \emph{really} wish we used the term \emph{outer measure.}}), so for each $A_i$, we find a Borel set $\tilde{B}_i$ such that $A_i^\complement\subset \tilde{B}_i$, and call $B_i=\tilde{B}^\complement_i$. Then

Thus we can define $\beta=\sum_{i=1}^\infty a_i \Chi_{K_i}$, and note that $\beta=\sigma$ \muae{}, and if $\beta(x)\neq\sigma(x)$, then $\beta(x)=0$\footnote{In case you were concerned, $\preimage{\beta}{\{0\}}$ is Borel even if no $a_i=0$, since it is $\left(\bigcup_{i-1}^\infty K_i\right)^\complement$ in that case.}. 

	\item Next, let $f$ be any nonnegative Lebesgue-measurable function, and let $\sigma_n$ be a sequence of nonnegative Lebesgue-measurable simple functions with $\sigma_n\,\increasesto f$. By (i), produce Borel measurable functions $\beta_n$ with $\beta_n=\sigma_n$ \muae{}. Since $\sigma_n\to f$ and $\beta_n=0$ whenever $\beta_n\neq\sigma_n$, then $\beta_n$ converges to a function we can call $g=\lim_{n\to\infty} \beta_n$. To see that $g$ is Borel measurable, we show that $\liminf \beta_n$ and $\limsup \beta_n$ are Borel measurable.
	\begin{align*}
	\preimage{(\limsup_{n\to\infty} \beta_n)}{-\infty,b} &= \{x\in\R: \limsup_{n\to\infty} \beta_n(x)<b\}\\
	&= \{x\in\R: \forall k>0, \exists n>k \text{ s.t. } \beta_n(x)<b\}\\
	&= \bigcap_{k=1}^\infty\bigcup_{n>k}\preimage{\beta_n}{\infty,b}
	\end{align*}
which is Borel. A similar argument shows that $\liminf \beta_n$ is Borel measurable, so $g=\liminf \beta_n=\limsup \beta_n$ is as well. 
%	$$\preimage{g}{U}=\preimage{\left(\lim_{n\to\infty}\beta_n\right)}{U}=\left\lbrace x\in\R : \lim_{n\to\infty}\beta_n(x) \in U\right\rbrace$$
%	and let $x\in\preimage{g}{U}$. Since $U$ is open, there exists $\varepsilon>0$ such that the ball $B_\varepsilon(g(x))\subset U$, so there are infinitely many $n$ such that $\beta_n(x)\in U$, which means $x\in\limsup_{n\to\infty} \preimage{\beta_n}{U}$, which is a Borel set. Conversely if $x\in\limsup_{n\to\infty} \preimage{\beta_n}{U}$, then 
	
	\item Finally, we observe that if $f$ is any Lebesgue-measurable function, it can be written as $f=f^+-f^-$ where 
	\begin{align*}
	f^+(x)&=\begin{cases}
	f(x), & f(x)\geq0\\
	0, & \text{otherwise}\\
	\end{cases} &
	f^-(x)&=\begin{cases}
	f(x), & -f(x)\leq0\\
	0, & \text{otherwise}\\
	\end{cases} \\
	\end{align*}
	and we can use (ii) to produce Borel measurable functions $g^+$ and $g^-$ such that $g^+=f^+$ \muae{} and $g^-=f^-$ \muae{}, so letting $g=g^+-g^-$, we find that $g=f$ \muae{}, and all that remains is to show that $g$ is Borel measurable: 
	
	\hspace{.55in}$\preimage{g}{-\infty, b} = \{x\in\R : g^+(x)-g^-(x)<b\}
	= \begin{cases}
	\preimage{(g^-)}{b,\infty}, & \text{if }b\leq0\\
	%\preimage{(g^-)}{0}\cap\preimage{(g^+)}{0}, & \text{if }b=0\\
	\preimage{(g^+)}{0,b}, & \text{if }b>0\\
	\end{cases}$
	%
	%\begin{align*}
	%\preimage{g}{-\infty, b} &= \{x\in\R : g^+(x)-g^-(x)<b\}\\
	%&= \begin{cases}
	%\preimage{(g^-)}{b,\infty}, & \text{if }b\leq0\\
	%%\preimage{(g^-)}{0}\cap\preimage{(g^+)}{0}, & \text{if }b=0\\
	%\preimage{(g^+)}{0,b}, & \text{if }b>0\\
	%\end{cases}
	%\end{align*}
	
	which is a Borel set in either case. \qedhere
	\end{enumerate}
\end{proof}






\pagebreak
\item Let $X$ be nonempty and let $\mu$ be a measure on $X$. Assume $A_n \subset X$ are \mumeasurable{} for $n = 1, 2, \dots$ and assume the sequence $\Chi_{A_n}$ converges in measure to some function $f : X \to \R$. Prove that there exists a \mumeasurable{} set $A \subset X$ such that $f = \Chi_A$ \muae{} in $X$.
\begin{proof}
Since $\Chi_{A_n} \xrightarrow{\mu} f$, then there exists a subsequence $\Chi_{A_{n_k}} \to f$ \muae{}. Thus we can let 
$$A=\{x\in X : \lim_{k\to\infty} \Chi_{A_{n_k}}(x) = 1\}$$
That is, $A^\complement$ contains all $x\in X$ where $\lim_{k\to\infty} \Chi_{A_{n_k}}(x) = 0$ or DNE. Now observe that 
$$\Chi_A=f \text{ \muae{},}$$
Since $\Chi_A=\lim_{k\to\infty} \Chi_{A_{n_k}}$ except when the limit DNE, and the limit certainly does not agree with $f$ when it DNE, so 
$$\mu(\{x\in X : \lim_{k\to\infty} \Chi_{A_{n_k}}(x) \text{ DNE}\})=0.$$
Thus $\Chi_A = \lim_{k\to\infty} \Chi_{A_{n_k}}$ \muae{} and $\lim_{k\to\infty} \Chi_{A_{n_k}} = f$ \muae{}, so $\Chi_A = f$ \muae{}

To see that $A$ is measurable, observe that $\Chi_A = \lim_{k\to\infty} \Chi_{A_{n_k}}$ \muae	{} and each $\Chi_{A_{n_k}}$ is a measurable function, so their limit is measurable. Thus 
$$\{x\in X : \tfrac{1}{2} < \Chi_A(x) < \tfrac{3}{2} \}=A$$
is measurable, and we're done. 
\end{proof}

\pagebreak
\item Let $X$ be nonempty and let $\mu$ be a measure on $X$. Assume $f_n, f : X \to \R$ are \mumeasurable{}	functions such that for each $\varepsilon > 0$ one has
$$\sum_{n=1}^\infty \mu\big(\{x: |f_n(x)- f(x)|>\varepsilon\}\big)<\infty.$$
Prove that $f_n\to f$ \muae{} in $X$. 
\begin{proof}
Let $\varepsilon>0$.	Since the sum is finite, then the tail of the sum goes to zero, so the terms go to zero. That is, since 
\begin{align*}
\sum_{n=1}^\infty \mu\big(\{x: |f_n(x)- f(x)|>\varepsilon\}\big)&<\infty, \quad \text{then}\\
\lim_{k\to\infty} \sum_{n=k}^\infty \mu\big(\{x: |f_n(x)- f(x)|>\varepsilon\}\big)&=0, \quad \text{so}\\
\lim_{n\to\infty} \mu\big(\{x: |f_n(x)- f(x)|>\varepsilon\}\big)&=0, \quad \text{so}\\
\end{align*}
\vspace{-32pt}
$$f_n\xrightarrow{\mu}f. $$
Since $f_n\xrightarrow{\mu}f$, then there exists a subsequence $f_{n_k}\to f$ \muae{} 

To see that the more general case of $f_n\to f$ \muae{} holds, suppose not. Denote
$$A_\delta = \{x\in X : \lim_{k\to\infty} f_{n_k}(x) \neq f(x)\}.$$
We know that 
\begin{align*}
\sum_{n=1}^\infty \mu (B^\varepsilon_n )&<\infty \text{, where}\\
B^\varepsilon_n &= \{x: |f_n(x)- f(x)|>\varepsilon\}\\
\end{align*}
For any $x\in X$, if $\lim_{n\to\infty}|f_n(x)-f(x)|$ exists and nonzero, then $x\in A_\delta$. So we can observe the following about "the bad set" of $f_n$:
\begin{align*}
\measure{A} &> 0 \text{, where }  \\
A &= \left\lbrace x\in X : \lim_{n\to\infty}|f_n(x)-f(x)| \text{ DNE}\right\rbrace.
\end{align*}
Let $x\in A\setminus A_\delta$, so $f_{n_k}(x)\to f(x)$, but $f_{n}(x)\not\to f(x)$. Then there exists a subsequence $f_{n_j}(x)$ such that $f_{n_j}(x)\to L\neq f(x)$, where $L\in[-\infty,\infty]$. Then 
$$\lim_{j\to\infty}|f_{n_j}(x)-f(x)|=|L-f(x)|$$
so for small $\varepsilon$, there exists $J\in\N$ such that $|f_{n_j}(x)-f(x)|>\varepsilon$ for every $j>J$. This means $x$ is in infinitely many $B^\varepsilon_n$, so $(A\setminus A_\delta)\subset \limsup_{n\to\infty} B^\varepsilon_n$ and by the Borel-Cantelli Lemma, they both have measure zero. This contradicts that $A$ has positive measure, 
%since $A_\delta$ has measure zero thus is measurable, 
since $A\subset (A\setminus A_\delta) \cup A_\delta$. \qedhere
%Let $\varepsilon=\frac{1}{k}$. Let 
%\begin{align*}
%A_k&=\{x: |f_k(x)- f(x)|>\tfrac{1}{k}\}\\
%B_m&=\bigcup_{k=m}^\infty A_k\\
%\end{align*}


\end{proof}



\end{enumerate}

\end{document}
