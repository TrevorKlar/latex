 \documentclass[12pt,letterpaper]{article}

\usepackage{fancyhdr,fancybox,tensor}

%% Useful packages
\usepackage{amssymb, amsmath, amsthm} 
%\usepackage{graphicx}  %%this is currently enabled in the default document, so it is commented out here. 
\usepackage{calrsfs}
\usepackage{braket}
\usepackage{mathtools}
\usepackage{lipsum}
\usepackage{tikz}
\usetikzlibrary{cd}
\usepackage{verbatim}
%\usepackage{ntheorem}% for theorem-like environments
\usepackage{mdframed}%can make highlighted boxes of text
%Use case: https://tex.stackexchange.com/questions/46828/how-to-highlight-important-parts-with-a-gray-background
\usepackage{wrapfig}
\usepackage{centernot}
\usepackage{subcaption}%\begin{subfigure}{0.5\textwidth}
\usepackage{pgfplots}
\pgfplotsset{compat=1.13}
\usepackage[colorinlistoftodos]{todonotes}
\usepackage[colorlinks=true, allcolors=blue]{hyperref}
\usepackage{xfrac}					%to make slanted fractions \sfrac{numerator}{denominator}
\usepackage{enumitem}            
    %syntax: \begin{enumerate}[label=(\alph*)]
    %possible arguments: f \alph*, \Alph*, \arabic*, \roman* and \Roman*
\usetikzlibrary{arrows,shapes.geometric,fit}

\DeclareMathAlphabet{\pazocal}{OMS}{zplm}{m}{n}
%% Use \pazocal{letter} to typeset a letter in the other kind 
%%  of math calligraphic font. 

%% This puts the QED block at the end of each proof, the way I like it. 
\renewenvironment{proof}{{\bfseries Proof}}{\qed}
\makeatletter
\renewenvironment{proof}[1][\bfseries \proofname]{\par
  \pushQED{\qed}%
  \normalfont \topsep6\p@\@plus6\p@\relax
  \trivlist
  %\itemindent\normalparindent
  \item[\hskip\labelsep
        \scshape
    #1\@addpunct{}]\ignorespaces
}{%
  \popQED\endtrivlist\@endpefalse
}
\makeatother

%% This adds a \rewnewtheorem command, which enables me to override the settings for theorems contained in this document.
\makeatletter
\def\renewtheorem#1{%
  \expandafter\let\csname#1\endcsname\relax
  \expandafter\let\csname c@#1\endcsname\relax
  \gdef\renewtheorem@envname{#1}
  \renewtheorem@secpar
}
\def\renewtheorem@secpar{\@ifnextchar[{\renewtheorem@numberedlike}{\renewtheorem@nonumberedlike}}
\def\renewtheorem@numberedlike[#1]#2{\newtheorem{\renewtheorem@envname}[#1]{#2}}
\def\renewtheorem@nonumberedlike#1{  
\def\renewtheorem@caption{#1}
\edef\renewtheorem@nowithin{\noexpand\newtheorem{\renewtheorem@envname}{\renewtheorem@caption}}
\renewtheorem@thirdpar
}
\def\renewtheorem@thirdpar{\@ifnextchar[{\renewtheorem@within}{\renewtheorem@nowithin}}
\def\renewtheorem@within[#1]{\renewtheorem@nowithin[#1]}
\makeatother

%% This makes theorems and definitions with names show up in bold, the way I like it. 
\makeatletter
\def\th@plain{%
  \thm@notefont{}% same as heading font
  \itshape % body font
}
\def\th@definition{%
  \thm@notefont{}% same as heading font
  \normalfont % body font
}
\makeatother

%===============================================
%==============Shortcut Commands================
%===============================================
\newcommand{\ds}{\displaystyle}
\newcommand{\B}{\mathcal{B}}
\newcommand{\C}{\mathbb{C}}
\newcommand{\F}{\mathbb{F}}
\newcommand{\N}{\mathbb{N}}
\newcommand{\R}{\mathbb{R}}
\newcommand{\Q}{\mathbb{Q}}
\newcommand{\T}{\mathcal{T}}
\newcommand{\Z}{\mathbb{Z}}
\renewcommand\qedsymbol{$\blacksquare$}
\newcommand{\qedwhite}{\hfill\ensuremath{\square}}
\newcommand*\conj[1]{\overline{#1}}
\newcommand*\closure[1]{\overline{#1}}
\newcommand*\mean[1]{\overline{#1}}
%\newcommand{\inner}[1]{\left< #1 \right>}
\newcommand{\inner}[2]{\left< #1, #2 \right>}
\newcommand{\powerset}[1]{\pazocal{P}(#1)}
%% Use \pazocal{letter} to typeset a letter in the other kind 
%%  of math calligraphic font. 
\newcommand{\cardinality}[1]{\left| #1 \right|}
\newcommand{\domain}[1]{\mathcal{D}(#1)}
\newcommand{\image}{\text{Im}}
\newcommand{\inv}[1]{#1^{-1}}
\newcommand{\preimage}[2]{#1^{-1}\left(#2\right)}
\newcommand{\script}[1]{\mathcal{#1}}


\newenvironment{highlight}{\begin{mdframed}[backgroundcolor=gray!20]}{\end{mdframed}}

\DeclarePairedDelimiter\ceil{\lceil}{\rceil}
\DeclarePairedDelimiter\floor{\lfloor}{\rfloor}

%===============================================
%===============My Tikz Commands================
%===============================================
\newcommand{\drawsquiggle}[1]{\draw[shift={(#1,0)}] (.005,.05) -- (-.005,.02) -- (.005,-.02) -- (-.005,-.05);}
\newcommand{\drawpoint}[2]{\draw[*-*] (#1,0.01) node[below, shift={(0,-.2)}] {#2};}
\newcommand{\drawopoint}[2]{\draw[o-o] (#1,0.01) node[below, shift={(0,-.2)}] {#2};}
\newcommand{\drawlpoint}[2]{\draw (#1,0.02) -- (#1,-0.02) node[below] {#2};}
\newcommand{\drawlbrack}[2]{\draw (#1+.01,0.02) --(#1,0.02) -- (#1,-0.02) -- (#1+.01,-0.02) node[below, shift={(-.01,0)}] {#2};}
\newcommand{\drawrbrack}[2]{\draw (#1-.01,0.02) --(#1,0.02) -- (#1,-0.02) -- (#1-.01,-0.02) node[below, shift={(+.01,0)}] {#2};}

%***********************************************
%**************Start of Document****************
%***********************************************
 %find me at /home/trevor/texmf/tex/latex/tskpreamble_nothms.tex
%===============================================
%===============Theorem Styles==================
%===============================================

%================Default Style==================
\theoremstyle{plain}% is the default. it sets the text in italic and adds extra space above and below the \newtheorems listed below it in the input. it is recommended for theorems, corollaries, lemmas, propositions, conjectures, criteria, and (possibly; depends on the subject area) algorithms.
\newtheorem{theorem}{Theorem}
\numberwithin{theorem}{section} %This sets the numbering system for theorems to number them down to the {argument} level. I have it set to number down to the {section} level right now.
\newtheorem*{theorem*}{Theorem} %Theorem with no numbering
\newtheorem{corollary}[theorem]{Corollary}
\newtheorem*{corollary*}{Corollary}
\newtheorem{conjecture}[theorem]{Conjecture}
\newtheorem{lemma}[theorem]{Lemma}
\newtheorem*{lemma*}{Lemma}
\newtheorem{proposition}[theorem]{Proposition}
\newtheorem*{proposition*}{Proposition}
\newtheorem{problemstatement}[theorem]{Problem Statement}


%==============Definition Style=================
\theoremstyle{definition}% adds extra space above and below, but sets the text in roman. it is recommended for definitions, conditions, problems, and examples; i've alse seen it used for exercises.
\newtheorem{definition}[theorem]{Definition}
\newtheorem*{definition*}{Definition}
\newtheorem{condition}[theorem]{Condition}
\newtheorem{problem}[theorem]{Problem}
\newtheorem{example}[theorem]{Example}
\newtheorem*{example*}{Example}
\newtheorem*{counterexample*}{Counterexample}
\newtheorem*{romantheorem*}{Theorem} %Theorem with no numbering
\newtheorem{exercise}{Exercise}
\numberwithin{exercise}{section}
\newtheorem{algorithm}[theorem]{Algorithm}

%================Remark Style===================
\theoremstyle{remark}% is set in roman, with no additional space above or below. it is recommended for remarks, notes, notation, claims, summaries, acknowledgments, cases, and conclusions.
\newtheorem{remark}[theorem]{Remark}
\newtheorem*{remark*}{Remark}
\newtheorem{notation}[theorem]{Notation}
\newtheorem*{notation*}{Notation}
%\newtheorem{claim}[theorem]{Claim}  %%use this if you ever want claims to be numbered
\newtheorem*{claim}{Claim}


%%
%% Page set-up:
%%
\pagestyle{empty}
\lhead{\textsc{201c - Functional Analysis} \\Quarter of COVID-19} 
\rhead{\textsc{Labutin, Spring 2020} \\ Trevor Klar}
%\chead{\Large\textbf{A New Integration Technique \\ }}
\renewcommand{\headrulewidth}{0pt}
%
\renewcommand{\footrulewidth}{0pt}
%\lfoot{
%Office: \quad \quad \, M 2-3 \, \, SH 6431x \\
%Math Lab: \, W 12-2 \, SH 1607
%}
%\rfoot{trevorklar@math.ucsb.edu}

\setlength{\parindent}{0in}
\setlength{\textwidth}{7in}
\setlength{\evensidemargin}{-0.25in}
\setlength{\oddsidemargin}{-0.25in}
\setlength{\parskip}{.5\baselineskip}
\setlength{\topmargin}{-0.5in}
\setlength{\textheight}{9in}

\setlist[enumerate,1]{label=\textbf{\arabic*.}}

\let\oldphi\phi
\renewcommand{\phi}{\varphi}
\renewcommand{\epsilon}{\varepsilon}

\begin{document}
\pagestyle{fancy}
\begin{center}
{\Large Homework 5}%=================UPDATE THIS=================%
\end{center}

\renewcommand{\B}{\bar{B}(\ell^\infty)}
\textbf{Chapter 1}

\begin{enumerate}

%\setcounter{enumi}{1}
\item[] \vspace*{-24pt} 
	\begin{theorem*}[Bathtub Principle]
	Let $(\Omega, \Sigma, \mu)$ be a measure space and let $f:\Omega\to\R$ be measurable with $\mu\{f<t\}$ finite for all $t\in\R$. Fix $G>0$, and define a class of measurable functions on $\Omega$ by 
	$$\script{C}=\left\lbrace g \; \middle| \; 0\leq g\leq 1 \quad\text{and}\quad \int_\Omega g \der\mu = G\right\rbrace.$$
	Then the minimization problem 
	\setcounter{equation}{0}
	\begin{equation}
	I=\inf_{g\in\script{C}} \int_\Omega fg\der\mu 
	\end{equation}
	is solved by 
	\setcounter{equation}{1}
	\begin{equation}
	g=\Chi_{\{f<s\}} + c\Chi_{\{f=s\}}
	\end{equation}
	and
	\setcounter{equation}{2}
	\begin{equation}
	I=\int_{\{f<s\}} f\der\mu + cs\mu\{f=s\}
	\end{equation}
	where $s$ is the supremum of all $t$ such that 
	\setcounter{equation}{3}
	\begin{equation}
	\mu\{f<t\}\leq G,
	\end{equation}
	and $c$ is a scalar such that 
	\setcounter{equation}{4}
	\begin{equation}
	\mu\{f<s\} + c\mu\{f=s\} = G.
	\end{equation}
	\end{theorem*}
	\begin{proof}
	We know that $\mu\braces{f<t}$ is finite for all $t$, and since $\braces{f<a}\subseteq\braces{f<b}$ for $a<b$, then $\mu\braces{f<t}$ increases as $t$ increases. We would like to bound this measure, so let $s$ be the supremum of all $t$ such that 
	\setcounter{equation}{3}
	\begin{equation}
	\mu\{f<t\}\leq G.
	\end{equation}
	\textsc{Case} $(s=\infty)$ We assume that since $g$ is thought to be a density, then $\mu(\Omega)\geq G$. This means that if $s=\infty$, then since $\{f<\infty\}=\preimage{f}{\R}=\Omega$, we have that $\mu(\Omega)\leq G$. Thus in (5) $c=0$, and so equation (2) is given by 
	$$g=\Chi_{\braces{f<\infty}}=\Chi_\Omega=1.$$
	Now $g$ has integral $G$ and it is the \emph{only} function in $\script{C}$, since any other function in $\script{C}$ is equal almost everywhere or has strictly smaller integral. Thus (2) trivially solves (1), and equations (1) and (3) are both $I=\int_\Omega f$. 
	
	\textsc{Case} ($s<\infty$) Suppose $s$ is finite. Then either $\mu\{f=s\}=0$ or $\mu\{f=s\}>0$.
	\begin{itemize}
	\item[] 
	\textsc{Claim} Either $\mu\{f<s\}=G$, or $\mu\{f<s\}+\mu\{f=s\}>G$. 
	
	\textsc{Proof of Claim} Suppose that $\mu\{f<s\}\neq G$.
	Then clearly $\mu\{f<s\}\not> G$, so $\mu\{f<s\}< G$. Since $s$ is the least upper bound of the set, then $\mu\braces{f<s+\epsilon}>G$ for all $\epsilon$. Thus 
	\begin{align*}
	\mu\{f<s\}+\mu\{f=s\}&>G.
	\end{align*}
	\end{itemize}
	Justified by the claim above, let $0\leq c <1$ so that 
	\setcounter{equation}{4}
	\begin{equation}
	\mu\{f<s\} + c\mu\{f=s\} = G.
	\end{equation}
	
	Now define 	
	\setcounter{equation}{1}
	\begin{equation}
	g=\Chi_{\{f<s\}} + c\Chi_{\{f=s\}},
	\end{equation}
	and let's compute the integral in (3).
	\begin{align*}
	\int_\Omega fg\der\mu &= \int_\Omega f \left( \Chi_{\{f<s\}} + c\Chi_{\{f=s\}}\right) \der\mu \\
	&=\int_{\{f<s\}} f\der\mu + \int_{\{f=s\}} c\der\mu \\
	&=\int_{\{f<s\}} f\der\mu + cs\mu\{f=s\}.
	\end{align*}
	Now all that remains is to show that (2) solves the minimization problem
	\setcounter{equation}{0}
	\begin{equation}
	I=\inf_{g\in\script{C}} \int_\Omega fg\der\mu. 
	\end{equation}
	
	For now, suppose that $\mu\{f=s\}=0$. Let $h$ be any element of $\script{C}$ which is distinct from $g$, so they differ on a set of positive measure. If $h=g$ on $\{f<s\}$ then they are the equal almost everywhere since $\int_{\{f<s\}} g\der\mu=G$, so call 
	$$A=\{x\in\{f<s\} : h(x)\neq g(x)\}$$
	and note that $\mu(A)>0$ and in fact $h<g$ on $A$. Since this means $\int_{\{f<s\}} h \der\mu<G$, then $h$ and $g$ also differ on $\{f>s\}$ on a set of positive measure. So call this set
	$$A'=\{x\in\{f>s\} : h(x)\neq g(x)\}$$
	and note that $\mu(A')=\mu(A)$ and $h>g$ on $A'$.
	\jpg{width=0.7\textwidth}{hw5-bathtub-principle-1}
	Now we we show that $\int_\Omega fg\der\mu \leq \int_\Omega fh\der\mu$. First, since $\int_A (g-h)=\int_{A'}h$, then
	\begin{equation}
	\int_A f(g-h) \leq \int_A s  (g-h) = \int_{A'} sh \leq \int_{A'}f(g-h).\tag{$\dagger$}
	\end{equation}
	Thus 
	\begin{align*}
	\int_\Omega fg &= \int_{\{f<s\}} fg + \int_{\{f\geq s\}} fg \\
	&=\left( \int_{\{f<s\}} fh + \int_{A} f(g-h) \right) + \left( \int_{\{f\geq s\}} fh - \int_{A'} fh \right)\\ 
	&\leq \int_{\{f<s\}} fh + \int_{A'} fh + \int_{\{f\geq s\}} fh - \int_{A'} fh & \text{by }(\dagger)\\ 
	&= \int_{\{f<s\}} fh + \int_{\{f\geq s\}} fh + \int_{A'} fh - \int_{A'} fh \\
	&= \int_\Omega fh
	\end{align*}
	So since $\int_\Omega fg \der\mu$ is an element of the set $\braces{\int_\Omega fh \middle| h\in\script{C}}$ in equation (1) and it is a lower bound of that set, then it is the infimum. 
	
	Earlier we supposed that $\mu\{f=s\}=0$. If instead $\mu\{f=s\}>0$, then $g=c$ (between 1 and 0) on $\{f=s\}$, so $h$ can have three behaviors there: $h=g$ on $\{f=s\}$, there exists $B\subset \{f=s\}$ such that $h<g$ on $B$, or there exists $B\subset \{f=s\}$ such that $h>g$ on $B$.
	
	\textsc{Case I} If $h=g$ on $\{f=s\}$, we can apply ($\dagger$) and use the same proof as when we assumed $\mu\{f=s\}=0$. 
	
	\textsc{Case II} Suppose there exists $B\subset \{f=s\}$ such that $h<g$ on $B$. Then since $g=1$ on $\{f<s\}$, there must exist $B'\subset \{f>s\}$ such that $\int_B (g-h)=\int_{B'} h$, so 
	\begin{equation}
	\int_B f(g-h) = \int_B s  (g-h) = \int_{B'} sh < \int_{B'}f(g-h).\tag{$\ddagger$}
	\end{equation}
	\jpg{width=0.4\textwidth}{hw5-bathtub-principle-2}
	Following the same strategy of proof as when we assumed $\mu\{f=s\}=0$, we can find that 
	$$\int_{\{f\geq s\}}fg < \int_{\{f\geq s\}}fh.$$
	%If in addition $h<g$ on $\{f< s\}$, we can apply ($\dagger$) and we're done. 
	
	\textsc{Case III} Suppose there exists $B\subset \{f=s\}$ such that $h>g$ on $B$. Since $g=0$ on $\{f>s\}$, there must exist $B'\subset \{f<s\}$ such that $\int_{B'} (g-h)=\int_{B} (h-g)$, so 
	\begin{equation}
	\int_{B'} f(g-h) < \int_{B'} s  (g-h) = \int_{B} s(h-g) = \int_{B} f(h-g).\tag{$\dagger\dagger$}
	\end{equation}
	
	\jpg{width=0.7\textwidth}{hw5-bathtub-principle-3}
	Thus 
	\begin{align*}
	\int_\Omega fg &= \int_{\{f<s\}} fg + \int_{\{f= s\}} fg + \int_{\{f> s\}} fg\\
	&= \left( \int_{\{f<s\}} fh + \int_{B'} f(g-h) \right) + \left( \int_{\{f=s\}} fh - \int_{B} f(h-g) \right) + \int_{\{f>s\}} fg\\ 
	&< \int_{\{f<s\}} fh + \int_{B} f(h-g) + \int_{\{f=s\}} fh - \int_{B} f(h-g)  + \int_{\{f>s\}} fg & \text{by }(\dagger\dagger)\\ 
	&= \int_{\{f<s\}} fh  + \int_{\{f=s\}} fh  + \int_{\{f>s\}} fg \\
	&= \int_{\{f<s\}} fh  + \int_{\{f=s\}} fh  + \int_{\{f>s\}} fh &\hspace*{-.75in} \text{Since }g=0\text{ on } \{f>s\}\\
	&= \int_\Omega fh
	\end{align*}
	Therefore in any case, 
	\setcounter{equation}{0}
	\begin{equation}
	\inf_{h\in\script{C}} \int_\Omega fh\der\mu=\int_\Omega fg\der\mu
	\end{equation}	
	and we're done. 
	\end{proof}
	
	

\end{enumerate}

\pagebreak
\textbf{Chapter 4}

\textbf{4.3}
The \textbf{weak $L^p$-space}, denoted $L^p_w(\R^n)$, is defined as the set of all measurable functions such that
\setcounter{equation}{2}
\begin{equation}
\angles{f}_{p,w}=\sup_{\alpha>0}\;\alpha\,\big(\mu\braces{|f|>\alpha}\big)^{1/p}<\infty
\end{equation}
%why even write q here if the value of q has no bearing on the finiteness of LHS?

The expression given by (3) does not define a norm. For $p > 1$ there is an
alternative expression, equivalent\footnote{Equivalent in the sense that convergence in $\angles{f}$ is equivalent to convergence in $\norm{f}$.} to (3), that is indeed a norm. It is given
by
\setcounter{equation}{4}
\begin{equation}
\norm{f}_{p,w}=\sup_A |A|^{-1/p'} \int_A |f| \dx,
\end{equation}
where $A$ is any set of finite measure. Using Theorem 1.14 (bathtub principle) it is not hard to see that
(3) and (5) are equivalent.

%\begin{definition*}
%The weak $L^q$-norm
%\end{definition*}

\begin{enumerate}
%\setcounter{enumi}{1}
\item Prove that (5) above actually defines a norm--the weak $L^p$-norm.
\begin{proof}
\begin{enumerate}
\item $\norm{\bigcdot}_{p,w}$ is absolutely homogeneous:
	\begin{align*}
	\norm{\lambda f}_{p,w}
	&=\sup_A |A|^{-1/p'} \int_A |\lambda f| \dx \\
	&=|\lambda|\sup_A |A|^{-1/p'} \int_A |f| \dx \\
	&=|\lambda|\norm{ f}_{p,w}
	\end{align*}
	
\item $\norm{\bigcdot}_{p,w}$ is positive definite:

If $\norm{f}_{p,w}=0$, then $\int_A|f|\dx=0$ for all $A$, which means $f=0$ almost everywhere. 

\item $\norm{\bigcdot}_{p,w}$ has the triangle inequality:
	\begin{align*}
	\norm{f+g}_{p,w}
	&=\sup_A |A|^{-1/p'} \int_A | f + g| \dx \\
	&\leq \sup_A |A|^{-1/p'} \left(\int_A | f | \dx + \int_A |  g| \dx \right)\\
	&\leq \sup_A \left(|A|^{-1/p'} \int_A |f| \dx + |A|^{-1/p'} \int_A |g| \dx\right) \\
	&\leq \left(\sup_A |A|^{-1/p'} \int_A |f| \dx\right) + \left(\sup_A |A|^{-1/p'} \int_A |g| \dx\right) \\
	&\norm{f}_{p,w}+\norm{g}_{p,w}\qedhere
	\end{align*}
\end{enumerate}
\end{proof}

\item Prove the equivalence of the two definitions of weak $L^p$ given in Sect. 4.3. That is, %if $\angles{f}_{p,w}$ denotes the LHS of 4.3(3), then 
prove that 
$$C_1\angles{f}_{p,w}\leq \norm{f}_{p,w}\leq C_2\angles{f}_{p,w},$$
where $C_1$ and $C_2$ are universal constants independent of $f$. Find explicit values for these constants.

\begin{proof}
First we will show that $\norm{f}_{p,w}\geq C_1\angles{f}_{p,w}$. 

For any $\alpha>0$, let $A_\alpha=\{|f|>\alpha\}$. Then 
\begin{align*}
|A_\alpha|^{-1/p'}\int_{A_\alpha} |f|\dx & \geq |A_\alpha|^{-1/p'} \int_{A_\alpha}\alpha\dx \\
&= |A_\alpha|^{-1/p'} \alpha |A_\alpha| \\
&= \alpha|A_\alpha|^{-1/p} \\
&= \alpha\,\big(\mu\braces{|f|>\alpha}\big)^{1/p}.
\end{align*}

Thus taking supremum of both sides, 
\begin{align*}
\norm{f}_{p,w}&=\sup_A |A|^{-1/p'} \int_A |f| \dx\\
&\geq \sup_{\alpha>0} |A_\alpha|^{-1/p'} \int_{A_\alpha} |f| \dx\\
&\geq \sup_{\alpha>0} \alpha\,\big(\mu\braces{|f|>\alpha}\big)^{1/p}\\
&= \angles{f}_{p,w}
\end{align*}
so $C_1=1$. \qedwhite

Next we show that $\norm{f}_{p,w}\leq C_2\angles{f}_{p,w}$. Before we start, observe that equation (3) gives us that 
\begin{equation*}
\angles{f}^p=\sup_{t>0}t^p\abs{\braces{|f|>t}},
\end{equation*}
where we suppress the notation and write $\angles{f}^p$ to mean $(\angles{f}_{p,w})^p$. Thus for any particular $t>0$, we have 
\begin{equation}
\frac{\angles{f}^p}{t^p}\geq\abs{\braces{|f|>t}}. \tag{$\dagger$}
\end{equation}

Now we begin the proof. Equation (5) gives us that 
\setcounter{equation}{4}
\begin{equation}
\norm{f}_{p,w}=\sup_A |A|^{-1/p'} \int_A |f| \dx,
\end{equation}
And to bound the integral in (5) we rewrite it and split the integral at a level $T$ (to be determined later):
\begin{align*}
\int_A |f| \dx
&=\int_0^\infty \abs{\braces{|f|>t}\cap A} \dt \\
&=\int_0^T \abs{\braces{|f|>t}\cap A} \dt + \int_T^\infty \abs{\braces{|f|>t}\cap A} \dt\\
&\leq T|A| + \int_T^\infty \abs{\braces{|f|>t}\cap A} \dt\\
&\leq T|A| + \int_T^\infty \abs{\braces{|f|>t}} \dt\\
&\leq T|A| + \int_T^\infty \frac{\angles{f}^p}{t^p} \dt & \text{by }(\dagger)\\
&= T|A| + \frac{\angles{f}^p}{(p-1)\left(T^{p-1}\right)} \\
\end{align*}
Next, we will find a value of $T$ to minimize the right hand side above, when everything else is held constant. Write $T|A| + \frac{\angles{f}^p}{(p-1)\left(T^{p-1}\right)}$ as a function of $T$ with $\beta=p-1$ and constants $B_1, B_2$:
\begin{align*}
\phi(T) %&= T|A| + \frac{\angles{f}^p}{(q-1)\left(T^{q-1}\right)} \\
&= TB_1 + \frac{B_2}{T^{\beta}}
\end{align*}
Since $\phi'(T)=B_1-\beta B_2 T^{-\beta-1}$ and $-\beta B_2 T^{-\beta-1}$ is an increasing function with limit 0 as $T\to\infty$, then as long as $B_1>0$ (it is), then $\phi$ has exactly one minimum. Solving for $T$ in $\phi'=0$ will show that we should fix 
\begin{align*}
T&=\left(\frac{\beta B_2}{B_1}\right)^{\frac{1}{\beta+1}}\\
&=\left(\frac{(p-1) \angles{f}^p}{|A|(p-1)}\right)^{\frac{1}{p}}\\
&=\left(\frac{ \angles{f}^p}{|A|}\right)^{\frac{1}{p}}\\
&={|A|^{-1/p}}{ \angles{f}}\\
\end{align*}
Thus 
\begin{align*}
\int_A |f| \dx
&\leq T|A| + \frac{\angles{f}^p}{(p-1)\left(T^{p-1}\right)} \\
&=|A|^{1/p'}\angles{f} + \frac{|A|^{1/p'}\angles{f}}{(p-1)} \\
&=|A|^{1/p'}\angles{f} \left(1 + \frac{1}{(p-1)} \right), \\
&=|A|^{1/p'}\angles{f} (p'), 
\end{align*}

and finally we can conclude that 
\begin{align*}
\norm{f}_{p,w}
&=\sup_A |A|^{-1/p'} \int_A |f| \dx \\
&\leq \angles{f} (p'),
\end{align*}
so $C_2=p'$, and we're done. 
% C_1 should be easy. For C_2, see Week 6 May 6 office hours 52:18.
\end{proof}

\setcounter{enumi}{3}
\item Gaussian integrals appear frequently and it is important to know how to
compute them.
	\begin{enumerate}[label=(\alph*)]
	\item Show that 
	$$\int_{-\infty}^{\infty}\exp{(-\lambda x^2)}\dx = \sqrt{\pi/\lambda}$$
	by evaluating the square of the integral by means of polar coordinates.
	\begin{proof}
	We will show that $\left(\int_{-\infty}^{\infty}\exp{(-\lambda x^2)}\dx\right)^{2} = \pi/\lambda$. First, observe that 
	\begin{align*}
	\left(\int_{-\infty}^{\infty}\exp{(-\lambda x^2)}\dx\right)^{2}
	&= \left(\int_{-\infty}^{\infty}\exp{(-\lambda x^2)}\dx\right)\left(\int_{-\infty}^{\infty}\exp{(-\lambda y^2)}\dy\right) \\
	&= \int_{-\infty}^{\infty} \int_{-\infty}^{\infty} \exp{(-\lambda x^2)} \exp{(-\lambda y^2)}\dy\dx
	\end{align*}
	and by changing to polar coordinates, 
	\begin{align*}
	\phantom{\left(\int_{-\infty}^{\infty}\exp{(-\lambda x^2)}\dx\right)^{2}}
	&= \int_0^\infty \int_0^{2\pi} \exp{(-\lambda r^2)} r\der r\der\theta \\
	&= \left(\int_0^\infty \exp{(-\lambda r^2)} r\der r \right)\left( \int_0^{2\pi} \der\theta\right)\\
	&= \bigg[ -\frac{1}{2\lambda} e^{-\lambda r^2}\bigg]_{r=0}^\infty \bigg( 2\pi \bigg) \\
	&= 	\left(0+\frac{1}{2\lambda}\right)(2\pi)\\
	&= \frac{\pi}{\lambda}.
	\end{align*}
	and we're done, since taking square roots yields the desired integral. 
	\end{proof}		
	
	\item For $A$ a symmetric $n\times	n$ matrix whose real part is positive definite, show that
	$$\int_{\R^n} \exp{(-x^\top Ax)}\dx = {\pi^{n/2}}/{\sqrt{\det A}}.$$
	In the real, symmetric case this can be done by a simple change of variables.
	
	\begin{proof}
	Since $A$ is positive definite, then $A$ is unitarily diagonalizable with positive determinant. So we can write $A=UDU^*$, and make a change of variables $x\mapsto Ux$. Then the Jacobian is $\det U=1$, so 
	\begin{align*}
	\int_{\R^n} \exp{(-x^\top Ax)}\dx 
	&= \int_{\R^n} \exp{(-(Ux)^\top (UDU^*)(Ux))}\dx \\
	&= \int_{\R^n} \exp{(-x^\top(U^*U)D(U^*U)x)}\dx \\
		&= \int_{\R^n} \exp{(-x^\top Dx)}\dx \\
		&= \int_{\R^n} \exp{\left(\sum_{i=1}^n \lambda_i x_i^2\right)}\dx &\text{where }\lambda_i\text{ are eigenvalues}\\
		&= \prod_{i=1}^n \int_{-\infty}^\infty \exp{(\lambda_i x_i^2)}\der x_i \\
		&= \prod_{i=1}^n \sqrt{\pi/\lambda_i} \\
		&= \sqrt{\frac{\pi^n}{\det D}} \\
		&= \sqrt{\frac{\pi^n}{\det A}} &&\qedhere
	\end{align*}
	\end{proof}		
	
	\item For a vector $v$ in $\mathbb{C}^n$ show, by "completing the square", that 
	$$\int_{\R^n} \exp(-\angles{x,Ax}+2\angles{v,x})\dx=\left(\pi^{n/2}/\sqrt{\det A}\right) \exp\left(\angles{v,A^{-1}v}\right).$$
	
	\begin{proof}
	The expression $-\angles{x,Ax}+2\angles{v,x}$ sort of looks like $(x+v)^2$, in an inner-producty sort of way. If we play with the numbers, we find that 
	$$\angles{-x+vA^{-1},\; Ax-v} \;=\; -\angles{x,Ax} + 2\angles{v,x} - \angles{vA^{-1},v},$$
	so since $\exp(- \angles{vA^{-1},v})$ is constant with respect to $x$, we find that 
	\begin{align*}
	&\int_{\R^n} \exp\left(-\angles{x,Ax}+2\angles{v,x}\right)\dx \\
	&\qquad = \exp\left(\angles{vA^{-1},v}\right) \exp\left(- \angles{vA^{-1},v}\right) \int_{\R^n} \exp\left(-\angles{x,Ax}+2\angles{v,x}\right)\dx \\
	&\qquad = \exp\left(\angles{vA^{-1},v}\right) \int_{\R^n} \exp\left(-\angles{x,Ax}+2\angles{v,x} - \angles{vA^{-1},v}\right)\dx \\
	&\qquad = \exp\left(\angles{vA^{-1},v}\right) \int_{\R^n} \exp\left(\angles{-x+vA^{-1},\; Ax-v}\right)\dx \\
	\end{align*}
	\begin{align*}		
	&\qquad = \exp\left(\angles{vA^{-1},v}\right) \int_{\R^n} \exp\left(\angles{-(x-vA^{-1}),\; A(x-vA^{-1})}\right)\dx &\text{(i)}\\
	&\qquad = \exp\left(\angles{vA^{-1},v}\right) \int_{\R^n} \exp\left(\angles{-x,\; Ax}\right)\dx &\text{(ii)}\\
	&\qquad =  \exp\left(\angles{v,A^{-1}v}\right) \left(\pi^{n/2}/\sqrt{\det A}\right).
	\end{align*}
	Step (i) is justified by the fact that since $A$ is symmetric, then $vA^{-1}=A^{-1}v$ (letting any vector be a row or column vector as is convenient). In step (ii), we are making a change of variables, which comes for free since adding the constant $-vA^{-1}$ gives the same Jacobian as adding zero. 
	\end{proof}
	
	\end{enumerate}

\end{enumerate}



\end{document}
