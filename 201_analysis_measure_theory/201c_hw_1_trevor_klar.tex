 \documentclass[12pt,letterpaper]{article}

\usepackage{fancyhdr,fancybox}

%% Useful packages
\usepackage{amssymb, amsmath, amsthm} 
%\usepackage{graphicx}  %%this is currently enabled in the default document, so it is commented out here. 
\usepackage{calrsfs}
\usepackage{braket}
\usepackage{mathtools}
\usepackage{lipsum}
\usepackage{tikz}
\usetikzlibrary{cd}
\usepackage{verbatim}
%\usepackage{ntheorem}% for theorem-like environments
\usepackage{mdframed}%can make highlighted boxes of text
%Use case: https://tex.stackexchange.com/questions/46828/how-to-highlight-important-parts-with-a-gray-background
\usepackage{wrapfig}
\usepackage{centernot}
\usepackage{subcaption}%\begin{subfigure}{0.5\textwidth}
\usepackage{pgfplots}
\pgfplotsset{compat=1.13}
\usepackage[colorinlistoftodos]{todonotes}
\usepackage[colorlinks=true, allcolors=blue]{hyperref}
\usepackage{xfrac}					%to make slanted fractions \sfrac{numerator}{denominator}
\usepackage{enumitem}            
    %syntax: \begin{enumerate}[label=(\alph*)]
    %possible arguments: f \alph*, \Alph*, \arabic*, \roman* and \Roman*
\usetikzlibrary{arrows,shapes.geometric,fit}

\DeclareMathAlphabet{\pazocal}{OMS}{zplm}{m}{n}
%% Use \pazocal{letter} to typeset a letter in the other kind 
%%  of math calligraphic font. 

%% This puts the QED block at the end of each proof, the way I like it. 
\renewenvironment{proof}{{\bfseries Proof}}{\qed}
\makeatletter
\renewenvironment{proof}[1][\bfseries \proofname]{\par
  \pushQED{\qed}%
  \normalfont \topsep6\p@\@plus6\p@\relax
  \trivlist
  %\itemindent\normalparindent
  \item[\hskip\labelsep
        \scshape
    #1\@addpunct{}]\ignorespaces
}{%
  \popQED\endtrivlist\@endpefalse
}
\makeatother

%% This adds a \rewnewtheorem command, which enables me to override the settings for theorems contained in this document.
\makeatletter
\def\renewtheorem#1{%
  \expandafter\let\csname#1\endcsname\relax
  \expandafter\let\csname c@#1\endcsname\relax
  \gdef\renewtheorem@envname{#1}
  \renewtheorem@secpar
}
\def\renewtheorem@secpar{\@ifnextchar[{\renewtheorem@numberedlike}{\renewtheorem@nonumberedlike}}
\def\renewtheorem@numberedlike[#1]#2{\newtheorem{\renewtheorem@envname}[#1]{#2}}
\def\renewtheorem@nonumberedlike#1{  
\def\renewtheorem@caption{#1}
\edef\renewtheorem@nowithin{\noexpand\newtheorem{\renewtheorem@envname}{\renewtheorem@caption}}
\renewtheorem@thirdpar
}
\def\renewtheorem@thirdpar{\@ifnextchar[{\renewtheorem@within}{\renewtheorem@nowithin}}
\def\renewtheorem@within[#1]{\renewtheorem@nowithin[#1]}
\makeatother

%% This makes theorems and definitions with names show up in bold, the way I like it. 
\makeatletter
\def\th@plain{%
  \thm@notefont{}% same as heading font
  \itshape % body font
}
\def\th@definition{%
  \thm@notefont{}% same as heading font
  \normalfont % body font
}
\makeatother

%===============================================
%==============Shortcut Commands================
%===============================================
\newcommand{\ds}{\displaystyle}
\newcommand{\B}{\mathcal{B}}
\newcommand{\C}{\mathbb{C}}
\newcommand{\F}{\mathbb{F}}
\newcommand{\N}{\mathbb{N}}
\newcommand{\R}{\mathbb{R}}
\newcommand{\Q}{\mathbb{Q}}
\newcommand{\T}{\mathcal{T}}
\newcommand{\Z}{\mathbb{Z}}
\renewcommand\qedsymbol{$\blacksquare$}
\newcommand{\qedwhite}{\hfill\ensuremath{\square}}
\newcommand*\conj[1]{\overline{#1}}
\newcommand*\closure[1]{\overline{#1}}
\newcommand*\mean[1]{\overline{#1}}
%\newcommand{\inner}[1]{\left< #1 \right>}
\newcommand{\inner}[2]{\left< #1, #2 \right>}
\newcommand{\powerset}[1]{\pazocal{P}(#1)}
%% Use \pazocal{letter} to typeset a letter in the other kind 
%%  of math calligraphic font. 
\newcommand{\cardinality}[1]{\left| #1 \right|}
\newcommand{\domain}[1]{\mathcal{D}(#1)}
\newcommand{\image}{\text{Im}}
\newcommand{\inv}[1]{#1^{-1}}
\newcommand{\preimage}[2]{#1^{-1}\left(#2\right)}
\newcommand{\script}[1]{\mathcal{#1}}


\newenvironment{highlight}{\begin{mdframed}[backgroundcolor=gray!20]}{\end{mdframed}}

\DeclarePairedDelimiter\ceil{\lceil}{\rceil}
\DeclarePairedDelimiter\floor{\lfloor}{\rfloor}

%===============================================
%===============My Tikz Commands================
%===============================================
\newcommand{\drawsquiggle}[1]{\draw[shift={(#1,0)}] (.005,.05) -- (-.005,.02) -- (.005,-.02) -- (-.005,-.05);}
\newcommand{\drawpoint}[2]{\draw[*-*] (#1,0.01) node[below, shift={(0,-.2)}] {#2};}
\newcommand{\drawopoint}[2]{\draw[o-o] (#1,0.01) node[below, shift={(0,-.2)}] {#2};}
\newcommand{\drawlpoint}[2]{\draw (#1,0.02) -- (#1,-0.02) node[below] {#2};}
\newcommand{\drawlbrack}[2]{\draw (#1+.01,0.02) --(#1,0.02) -- (#1,-0.02) -- (#1+.01,-0.02) node[below, shift={(-.01,0)}] {#2};}
\newcommand{\drawrbrack}[2]{\draw (#1-.01,0.02) --(#1,0.02) -- (#1,-0.02) -- (#1-.01,-0.02) node[below, shift={(+.01,0)}] {#2};}

%***********************************************
%**************Start of Document****************
%***********************************************
 %find me at /home/trevor/texmf/tex/latex/tskpreamble_nothms.tex
%===============================================
%===============Theorem Styles==================
%===============================================

%================Default Style==================
\theoremstyle{plain}% is the default. it sets the text in italic and adds extra space above and below the \newtheorems listed below it in the input. it is recommended for theorems, corollaries, lemmas, propositions, conjectures, criteria, and (possibly; depends on the subject area) algorithms.
\newtheorem{theorem}{Theorem}
\numberwithin{theorem}{section} %This sets the numbering system for theorems to number them down to the {argument} level. I have it set to number down to the {section} level right now.
\newtheorem*{theorem*}{Theorem} %Theorem with no numbering
\newtheorem{corollary}[theorem]{Corollary}
\newtheorem*{corollary*}{Corollary}
\newtheorem{conjecture}[theorem]{Conjecture}
\newtheorem{lemma}[theorem]{Lemma}
\newtheorem*{lemma*}{Lemma}
\newtheorem{proposition}[theorem]{Proposition}
\newtheorem*{proposition*}{Proposition}
\newtheorem{problemstatement}[theorem]{Problem Statement}


%==============Definition Style=================
\theoremstyle{definition}% adds extra space above and below, but sets the text in roman. it is recommended for definitions, conditions, problems, and examples; i've alse seen it used for exercises.
\newtheorem{definition}[theorem]{Definition}
\newtheorem*{definition*}{Definition}
\newtheorem{condition}[theorem]{Condition}
\newtheorem{problem}[theorem]{Problem}
\newtheorem{example}[theorem]{Example}
\newtheorem*{example*}{Example}
\newtheorem*{counterexample*}{Counterexample}
\newtheorem*{romantheorem*}{Theorem} %Theorem with no numbering
\newtheorem{exercise}{Exercise}
\numberwithin{exercise}{section}
\newtheorem{algorithm}[theorem]{Algorithm}

%================Remark Style===================
\theoremstyle{remark}% is set in roman, with no additional space above or below. it is recommended for remarks, notes, notation, claims, summaries, acknowledgments, cases, and conclusions.
\newtheorem{remark}[theorem]{Remark}
\newtheorem*{remark*}{Remark}
\newtheorem{notation}[theorem]{Notation}
\newtheorem*{notation*}{Notation}
%\newtheorem{claim}[theorem]{Claim}  %%use this if you ever want claims to be numbered
\newtheorem*{claim}{Claim}


%%
%% Page set-up:
%%
\pagestyle{empty}
\lhead{\textsc{201c - Functional Analysis} \\Quarter of COVID-19} 
\rhead{\textsc{Labutin, Spring 2020} \\ Trevor Klar}
%\chead{\Large\textbf{A New Integration Technique \\ }}
\renewcommand{\headrulewidth}{0pt}
%
\renewcommand{\footrulewidth}{0pt}
%\lfoot{
%Office: \quad \quad \, M 2-3 \, \, SH 6431x \\
%Math Lab: \, W 12-2 \, SH 1607
%}
%\rfoot{trevorklar@math.ucsb.edu}

\setlength{\parindent}{0in}
\setlength{\textwidth}{7in}
\setlength{\evensidemargin}{-0.25in}
\setlength{\oddsidemargin}{-0.25in}
\setlength{\parskip}{.5\baselineskip}
\setlength{\topmargin}{-0.5in}
\setlength{\textheight}{9in}

\setlist[enumerate,1]{label=\textbf{\arabic*.}}

\let\oldphi\phi
\renewcommand{\phi}{\varphi}
\renewcommand{\epsilon}{\varepsilon}

\begin{document}
\pagestyle{fancy}
\begin{center}
{\Large Homework 1}%=================UPDATE THIS=================%
\end{center}

\jpg{width=\linewidth}{201c_hw1_p1}

\begin{enumerate}
\item[]

\begin{proof}
	\begin{enumerate}[label=(\alph*)]
	\item Let $f^{1}, f^2, \ldots \in c_0\subset\ell^\infty$, with $f^i\xto{i}f$ in the $\ell^\infty$ norm. Let $\epsilon>0$. Then $\exists M$ such that if $i>M$, then $\sup_{n}\abs{f^i_n-f_n}<\epsilon$. 
	
	Now fix $i>M$. Since $f^i\in c_0$, then $f^i_n \xto{n} 0$, so $\exists N$ such that if $n>N$, then $\abs{f^i_n}<\epsilon$. Thus $\forall n > N$, 
		\begin{align*}
		\abs{f_n} &\leq \abs{f_n-f^i_n} + \abs{f^i_n}\\
		&< \epsilon + \epsilon \\
		&=2\epsilon,
		\end{align*}
	and we conclude that $f_n\xto{n}0$, and $f\in c_0$. \qedwhite
	
	\item 
		\begin{enumerate}[label=(\roman*)]
		\item Note that $F_f$ is obviously linear. For all $f\in \ell^1$, $F_f$ is bounded since $\norm{F_f}_*=\sup\limits_{\norm{x}\leq 1}\abs{F_f(x)}$ and for all $x$ with $\norm{x}_{c_0}\leq 1$, 
		\begin{alignat*}{2}
		\abs{\sum_n x_nf_n} 
		&\leq \sum_n \abs{x_nf_n} \\
		&\leq \sum_n \abs{f_n} & \quad \text{since }\sup |x_n| =1
		\end{alignat*}
		thus $\norm{F_f}_*\leq\norm{f}_{\ell^1}$ which is finite since $f\in \ell^1$. 
		
		Thus $F_f$ is a bounded linear functional $c_0\to \R$, so $F_f\in c_0^*$. \qedwhite
		
		\item We have shown already that $\norm{F_f}_*\leq\norm{f}_{\ell^1}$, so to prove that $\norm{F_f}_*=\norm{f}_{\ell^1}$, it remains to prove the other direction.
		
		Let $f\in \ell^1$. For each $f_n\in \C$, let 
		$u_n=z_ne_n,$
		where $e_n=(\overbrace{0, 0, \dots, 0, 1,}^n 0, \dots )$ and $z_n$ is the complex number such that $f_nz_n=\abs{f_n}$ (note that $\abs{z_n}=1$). Then $v_j=\sum_{n=1}^j u_n$ has norm 1 for all $j$, and $F_f(v_j)\increasesto\norm{f}$ as $j\to \infty$. Thus $\norm{F_f}$ cannot be less than $\norm{f}$, so $\norm{F_f}_*\geq\norm{f}_{\ell^1}$. \qedwhite
		\pagebreak
		\item Let $\phi\in c_0^*$. Let $f$ be the sequence defined by $f_n=\phi(e_n)$ for all $n$.
		
		\textsc{Claim:} $\sum_n\abs{f_n}<\infty$. To see this, suppose for contradiction that $\sum_n\abs{f_n}=\infty$. We know that $\norm{\sum\limits_{n=1}^N e_n}_{c_0}=1$ for every $N\in \N$, so 
		\begin{equation}\tag{$*$}
		\left\lbrace \sum\limits_{n=1}^N e_n : N\in \N\right\rbrace \subset \left\lbrace	x\in c_0 : \norm{x}\leq 1\right\rbrace.
		\end{equation}
		Thus 
		\begin{align*}
		\infty &= \sum_{n=1}^\infty\abs{f_n}\\
		&= \sup_N \sum_{n=1}^N\abs{f_n\cdot 1} \\
		&= \sup_N \sum_{n=1}^N\abs{\phi(e_n)} \\
		&\leq \sup_N \abs{\phi\left(\sum_{n=1}^N e_n\right)} & \text{by }\Delta\text{ ineq. and linearity}\\
		&\leq \sup_{\norm{x}=1}\abs{\phi(x)} &\qquad \text{by }(*)\\\
		&= \norm{\phi}_* \\
		&< \infty.
		\end{align*}
		Thus we have shown by contradiction that $\sum_n\abs{f_n}<\infty$. \qedhere
		\end{enumerate}
	\end{enumerate}
\end{proof}

\setcounter{enumi}{1}
\item Let $X$ be a Banach space with $E\subset X^*$. Suppose for every $x\in X$ the set $\{\phi(x)\}_{\phi\in E}\subset \R$ is bounded. Prove that $E$ is strongly bounded in $X^*$. Explain why your proof collapses if $X$ is not complete. 

\begin{proof}
This follows immediately from the Uniform Boundedness Principle below, which requires $X$ to be Banach. 
\end{proof}

\begin{theorem*}\textbf{(Uniform Boundedness Principle)}
Let $X$ be a Banach space and let $Y$ be a normed linear space. Let $\script{F}$ be a collection of bounded linear operators from $X$ to $Y$. If for every $x\in X$ we have that $\sup\limits_{T\in \script{F}}\norm{T(x)}_Y<\infty$ then $\sup\limits_{T\in\script{F}}\norm{T}_*<\infty$. 
\end{theorem*}

\pagebreak
\item (a) Let $X$ be a Banach space and $(\phi_j)$ be a sequence in $X^*$. Suppose that $\angles{\phi_j , x}$ converges for any $x \in X$. Prove that there exists $\phi \in X^*$ such that
$\phi_j \xto{w*} \phi$. (In fancy terminology "$X^*$ is always $w*$ sequentially complete".)

\begin{proof}
We can of course define a functional $\phi(x)=\lim_j\angles{\phi_j,x}$ and note that it is linear, but we need to show that this $\phi$ is bounded. Since the sequence $\angles{\phi_j,x}$ is convergent and thus bounded, then by problem 2 the set $\{\phi_j\}_j$ is bounded in $X^*$, call this bound $M$. Thus for all $x$ with $\norm{x}\leq1$,
\begin{align*}
|\angles{\phi, x}| &= |\lim_j\langle\phi_j, x\rangle| \\
&= \lim_j|\langle\phi_j, x\rangle| \\
\leq M
\end{align*}
and we're done. 
\end{proof}

(b) Formulate the analogous statement for the $w$-convergence for a sequence $(x_n)\in X$. Try to extend your proof to this situation. when does the proof collapse?

\textbf{Question} Let $X$ be a Banach space and $(x_n)$ be a sequence in $X$. Suppose that $\angles{\phi , x_n}$ converges for any $\phi \in X^*$. Does there exist $x \in X$ such that $x_n \xto{w} x$?

\answer We can follow the strategy from (a) and define a \emph{functional} $\hat{x}\in X^{**}$ so that $\angles{\phi , x_n}\to \angles{\phi , \hat{x}}$, but we are only guaranteed that a corresponding $x\in X$ exists exactly when $X$ is reflexive. \qed

%\pagebreak
\item Let $X$ be Banach. Prove that a sequence $(\phi_j)$ in $X^*$ converges $w*$ if and only if it is strongly bounded and there exists a dense set $E$ with $\closure{E}=X$, such that the number sequence $\angles{\phi_j,u}$ converges for all $u\in E$. 

\begin{proof}
The forward direction is straightforward; the sequence is strongly bounded by problem 2, and $X$ is of course dense in itself and has the desired property. 

For the converse direction, suppose $\sup_j\norm{\phi_j}=C$ and $E$ exists as above. It suffices to show that $\angles{\phi_j,x}$ also converges if $x\in E^\complement$, since problem 3 completes the proof. Let $x\in X$. Since $x\in \closure{E}$ there exists a sequence $(u_n)\in E$ which converges to $x$. 
\begin{align*}
\lim_j\angles{\phi_j,x} &=\lim_j\angles{\phi_j,\lim_n u_n} \\
&=\lim_j\lim_n\angles{\phi_j, u_n} \\
&=\lim_n\lim_j\angles{\phi_j, u_n} \\
\end{align*}

\pagebreak
\textsc{Claim:} The sequence $\left(\lim_j\angles{\phi_j, u_n})\right)_{n=1}^\infty$ is a real Cauchy sequence, and thus converges. 

\textsc{Proof of claim:} Let $\epsilon >0$. Since $u_n\to x$, then it is also a Cauchy sequence, so there exists $N>0$ such that $\forall n,m>N$, 
$$\abs{u_n-u_m}<\epsilon$$
$$\implies \forall j \; \abs{\phi_j(u_n)-\phi_j(u_m)}=\abs{\phi_j(u_n-u_m)}\leq \norm{\phi_j}\epsilon$$
$$\implies \abs{\lim_j\phi_j(u_n)-\lim_j\phi_j(u_m)}\leq C\epsilon$$
so $\lim_n\lim_j\angles{\phi_j, u_n}=\lim_j\angles{\phi_j,x}$ converges for all $x\in X$, and we can define $\phi(x)=\lim_j\angles{\phi_j,x}$. Problem 3 assures us that $\phi\in X^*$, and we are done. 
\end{proof}

\renewcommand{\phi}{\oldphi}
\item Let $I=[0,1]$. Let $C^1(I)$ denote the space of continuously differentiable functions,\footnote{That is, $g,g'\in C(I)$. For example, a polynomial.} and \linebreak let $\der \phi_n=\cos(\pi nx)\der\lambda^1(x)$. 

(a) Prove that 
$$\int_Ig\der\phi_n\xto{n}0 \quad\quad\forall g\in C^1(I).$$
(b) Prove that $\der\phi_n\xto{w*}0$ as measures in $C(I)^*$. 

\begin{proof}(a) 
Using integration by parts, we find that 
\begin{align*}
\int_I g(x)\cos(\pi nx) \dx &= g(x)\frac{1}{\pi n}\sin(\pi nx) - \int_I g'(x)\frac{1}{\pi n}\sin(\pi nx)\dx \\
&= \frac{1}{\pi n}\left[g(x)\sin(\pi nx) - \int_I g'(x)\sin(\pi nx)\dx\right],
\end{align*}
and in the limit as $n\to \infty$, everything goes to 0. \qedwhite

(b) For arbitrary $f\in C(I)$, $f'$ may not so exist, so we can't use integration by parts. However, by the Weierstrauss Approximation Theorem, for every $\epsilon>0$, there exists a polynomial $g$ such that $\sup_I\abs{f-g}\leq\epsilon$. Thus, 
\begin{align*}
\int f \der\phi_n &= \int f\der\phi_n - \int g\der\phi_n + \int g\der\phi_n \\
&= \int (f-g)\der\phi_n + \int g\der\phi_n,
\end{align*}
and this integral is bounded above and below by 
$$\int (\pm \epsilon +g)\der\phi_n$$
respectively, which integrands are themselves polynomials, so they vanish in the limit. Therefore $\lim_{n\to\infty}\int f\der\phi_n=0$ by the squeeze theorem. 
\end{proof}

\end{enumerate}
\end{document}
