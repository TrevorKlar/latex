 \documentclass[12pt,letterpaper]{article}

\usepackage{fancyhdr,fancybox,tensor}

%% Useful packages
\usepackage{amssymb, amsmath, amsthm} 
%\usepackage{graphicx}  %%this is currently enabled in the default document, so it is commented out here. 
\usepackage{calrsfs}
\usepackage{braket}
\usepackage{mathtools}
\usepackage{lipsum}
\usepackage{tikz}
\usetikzlibrary{cd}
\usepackage{verbatim}
%\usepackage{ntheorem}% for theorem-like environments
\usepackage{mdframed}%can make highlighted boxes of text
%Use case: https://tex.stackexchange.com/questions/46828/how-to-highlight-important-parts-with-a-gray-background
\usepackage{wrapfig}
\usepackage{centernot}
\usepackage{subcaption}%\begin{subfigure}{0.5\textwidth}
\usepackage{pgfplots}
\pgfplotsset{compat=1.13}
\usepackage[colorinlistoftodos]{todonotes}
\usepackage[colorlinks=true, allcolors=blue]{hyperref}
\usepackage{xfrac}					%to make slanted fractions \sfrac{numerator}{denominator}
\usepackage{enumitem}            
    %syntax: \begin{enumerate}[label=(\alph*)]
    %possible arguments: f \alph*, \Alph*, \arabic*, \roman* and \Roman*
\usetikzlibrary{arrows,shapes.geometric,fit}

\DeclareMathAlphabet{\pazocal}{OMS}{zplm}{m}{n}
%% Use \pazocal{letter} to typeset a letter in the other kind 
%%  of math calligraphic font. 

%% This puts the QED block at the end of each proof, the way I like it. 
\renewenvironment{proof}{{\bfseries Proof}}{\qed}
\makeatletter
\renewenvironment{proof}[1][\bfseries \proofname]{\par
  \pushQED{\qed}%
  \normalfont \topsep6\p@\@plus6\p@\relax
  \trivlist
  %\itemindent\normalparindent
  \item[\hskip\labelsep
        \scshape
    #1\@addpunct{}]\ignorespaces
}{%
  \popQED\endtrivlist\@endpefalse
}
\makeatother

%% This adds a \rewnewtheorem command, which enables me to override the settings for theorems contained in this document.
\makeatletter
\def\renewtheorem#1{%
  \expandafter\let\csname#1\endcsname\relax
  \expandafter\let\csname c@#1\endcsname\relax
  \gdef\renewtheorem@envname{#1}
  \renewtheorem@secpar
}
\def\renewtheorem@secpar{\@ifnextchar[{\renewtheorem@numberedlike}{\renewtheorem@nonumberedlike}}
\def\renewtheorem@numberedlike[#1]#2{\newtheorem{\renewtheorem@envname}[#1]{#2}}
\def\renewtheorem@nonumberedlike#1{  
\def\renewtheorem@caption{#1}
\edef\renewtheorem@nowithin{\noexpand\newtheorem{\renewtheorem@envname}{\renewtheorem@caption}}
\renewtheorem@thirdpar
}
\def\renewtheorem@thirdpar{\@ifnextchar[{\renewtheorem@within}{\renewtheorem@nowithin}}
\def\renewtheorem@within[#1]{\renewtheorem@nowithin[#1]}
\makeatother

%% This makes theorems and definitions with names show up in bold, the way I like it. 
\makeatletter
\def\th@plain{%
  \thm@notefont{}% same as heading font
  \itshape % body font
}
\def\th@definition{%
  \thm@notefont{}% same as heading font
  \normalfont % body font
}
\makeatother

%===============================================
%==============Shortcut Commands================
%===============================================
\newcommand{\ds}{\displaystyle}
\newcommand{\B}{\mathcal{B}}
\newcommand{\C}{\mathbb{C}}
\newcommand{\F}{\mathbb{F}}
\newcommand{\N}{\mathbb{N}}
\newcommand{\R}{\mathbb{R}}
\newcommand{\Q}{\mathbb{Q}}
\newcommand{\T}{\mathcal{T}}
\newcommand{\Z}{\mathbb{Z}}
\renewcommand\qedsymbol{$\blacksquare$}
\newcommand{\qedwhite}{\hfill\ensuremath{\square}}
\newcommand*\conj[1]{\overline{#1}}
\newcommand*\closure[1]{\overline{#1}}
\newcommand*\mean[1]{\overline{#1}}
%\newcommand{\inner}[1]{\left< #1 \right>}
\newcommand{\inner}[2]{\left< #1, #2 \right>}
\newcommand{\powerset}[1]{\pazocal{P}(#1)}
%% Use \pazocal{letter} to typeset a letter in the other kind 
%%  of math calligraphic font. 
\newcommand{\cardinality}[1]{\left| #1 \right|}
\newcommand{\domain}[1]{\mathcal{D}(#1)}
\newcommand{\image}{\text{Im}}
\newcommand{\inv}[1]{#1^{-1}}
\newcommand{\preimage}[2]{#1^{-1}\left(#2\right)}
\newcommand{\script}[1]{\mathcal{#1}}


\newenvironment{highlight}{\begin{mdframed}[backgroundcolor=gray!20]}{\end{mdframed}}

\DeclarePairedDelimiter\ceil{\lceil}{\rceil}
\DeclarePairedDelimiter\floor{\lfloor}{\rfloor}

%===============================================
%===============My Tikz Commands================
%===============================================
\newcommand{\drawsquiggle}[1]{\draw[shift={(#1,0)}] (.005,.05) -- (-.005,.02) -- (.005,-.02) -- (-.005,-.05);}
\newcommand{\drawpoint}[2]{\draw[*-*] (#1,0.01) node[below, shift={(0,-.2)}] {#2};}
\newcommand{\drawopoint}[2]{\draw[o-o] (#1,0.01) node[below, shift={(0,-.2)}] {#2};}
\newcommand{\drawlpoint}[2]{\draw (#1,0.02) -- (#1,-0.02) node[below] {#2};}
\newcommand{\drawlbrack}[2]{\draw (#1+.01,0.02) --(#1,0.02) -- (#1,-0.02) -- (#1+.01,-0.02) node[below, shift={(-.01,0)}] {#2};}
\newcommand{\drawrbrack}[2]{\draw (#1-.01,0.02) --(#1,0.02) -- (#1,-0.02) -- (#1-.01,-0.02) node[below, shift={(+.01,0)}] {#2};}

%***********************************************
%**************Start of Document****************
%***********************************************
 %find me at /home/trevor/texmf/tex/latex/tskpreamble_nothms.tex
%===============================================
%===============Theorem Styles==================
%===============================================

%================Default Style==================
\theoremstyle{plain}% is the default. it sets the text in italic and adds extra space above and below the \newtheorems listed below it in the input. it is recommended for theorems, corollaries, lemmas, propositions, conjectures, criteria, and (possibly; depends on the subject area) algorithms.
\newtheorem{theorem}{Theorem}
\numberwithin{theorem}{section} %This sets the numbering system for theorems to number them down to the {argument} level. I have it set to number down to the {section} level right now.
\newtheorem*{theorem*}{Theorem} %Theorem with no numbering
\newtheorem{corollary}[theorem]{Corollary}
\newtheorem*{corollary*}{Corollary}
\newtheorem{conjecture}[theorem]{Conjecture}
\newtheorem{lemma}[theorem]{Lemma}
\newtheorem*{lemma*}{Lemma}
\newtheorem{proposition}[theorem]{Proposition}
\newtheorem*{proposition*}{Proposition}
\newtheorem{problemstatement}[theorem]{Problem Statement}


%==============Definition Style=================
\theoremstyle{definition}% adds extra space above and below, but sets the text in roman. it is recommended for definitions, conditions, problems, and examples; i've alse seen it used for exercises.
\newtheorem{definition}[theorem]{Definition}
\newtheorem*{definition*}{Definition}
\newtheorem{condition}[theorem]{Condition}
\newtheorem{problem}[theorem]{Problem}
\newtheorem{example}[theorem]{Example}
\newtheorem*{example*}{Example}
\newtheorem*{counterexample*}{Counterexample}
\newtheorem*{romantheorem*}{Theorem} %Theorem with no numbering
\newtheorem{exercise}{Exercise}
\numberwithin{exercise}{section}
\newtheorem{algorithm}[theorem]{Algorithm}

%================Remark Style===================
\theoremstyle{remark}% is set in roman, with no additional space above or below. it is recommended for remarks, notes, notation, claims, summaries, acknowledgments, cases, and conclusions.
\newtheorem{remark}[theorem]{Remark}
\newtheorem*{remark*}{Remark}
\newtheorem{notation}[theorem]{Notation}
\newtheorem*{notation*}{Notation}
%\newtheorem{claim}[theorem]{Claim}  %%use this if you ever want claims to be numbered
\newtheorem*{claim}{Claim}


%%
%% Page set-up:
%%
\pagestyle{empty}
\lhead{\textsc{201c - Functional Analysis} \\Quarter of COVID-19} 
\rhead{\textsc{Labutin, Spring 2020} \\ Trevor Klar}
%\chead{\Large\textbf{A New Integration Technique \\ }}
\renewcommand{\headrulewidth}{0pt}
%
\renewcommand{\footrulewidth}{0pt}
%\lfoot{
%Office: \quad \quad \, M 2-3 \, \, SH 6431x \\
%Math Lab: \, W 12-2 \, SH 1607
%}
%\rfoot{trevorklar@math.ucsb.edu}

\setlength{\parindent}{0in}
\setlength{\textwidth}{7in}
\setlength{\evensidemargin}{-0.25in}
\setlength{\oddsidemargin}{-0.25in}
\setlength{\parskip}{.5\baselineskip}
\setlength{\topmargin}{-0.5in}
\setlength{\textheight}{9in}

\setlist[enumerate,1]{label=\textbf{\arabic*.}}

\let\oldphi\phi
\renewcommand{\phi}{\varphi}
\renewcommand{\epsilon}{\varepsilon}

\begin{document}
\pagestyle{fancy}
\begin{center}
{\Large Homework 8}%=================UPDATE THIS=================%
\end{center}

\renewcommand{\B}{\bar{B}(\ell^\infty)}
\newcommand{\Rn}{\R^n}
\renewcommand{\phi}{\oldphi}
\renewcommand{\S}{\script{S}}

\begin{enumerate}
\item Use theorems on the Fourier transform from the textbook and the lectures to execute the following steps:
	\begin{enumerate}[label=(\alph*)]
	\item For a Schwartz function $\phi\in\script{S}(\Rn)$ with $n\geq 2$, define $\frac{\del_1}{|\nabla|}\phi$ using the Fourier transform. Prove that the operator $\frac{\del_1}{|\nabla|}:\script{S}\to X$ for, say, $X=\bar{C}_0(\Rn)$ the space of continuous functions limiting to 0 at infinity, is
defined.
	\begin{definition*}
	Define $\frac{\del_1}{|\nabla|}:\script{S}\to \bar{C}_0(\Rn)$ by 
	$$\frac{\del_1}{|\nabla|}(\phi)=\left(\frac{k_1}{|k|}\hat{\phi}(k)\right)^\vee.$$
	\end{definition*}
	\begin{proof}
	We know that $\S$ is closed under Fourier so $k_1 \hat{\phi}(k)\in \S$. So it suffices to show that we can take Fourier transforms of $\frac{1}{|k|}\phi$ for $\phi\in\S$, since $\left(\frac{1}{|k|}\phi\right)^\vee$ is a particular one. We can simply integrate to compute Fourier for any $L^1$ function, so it suffices to show that $\frac{1}{|k|}\phi\in L^1$. Note that 
	$$\int_{\Rn}\frac{1}{|k|}\phi = \int_{B_1}\frac{1}{|k|}\phi + \int_{B_1^\complement}\frac{1}{|k|}\phi.$$
	Now $\int_{B_1}\frac{1}{|k|^\alpha}\phi$ is finite since $1=\alpha<n\leq 2$, and $\int_{B_1^\complement}\frac{1}{|k|}\phi$ is finite since 
	$$\int_{B_1^\complement}\frac{1}{|k|}\phi\leq \int_{B_1^\complement}\phi<\infty$$
	because $\phi\in\S$. So $\frac{1}{|k|}\phi\in L^1$ and we're done.
	\end{proof}
	
%	\pagebreak
	\item Prove that if, for all $\phi\in\S$, we have that 
	$$\norm{\tfrac{\del_1}{|\nabla|}\phi}_q\leq C\norm{\phi}_p$$
	for some $p,q\in\N$, then $p=q$.
	\begin{proof}
	Fix $\phi\in\S$. Let $\phi_\lambda=\phi(\frac{x}{\lambda})$ which is in $\S$ for all $\lambda>0$. Then by using scaling, we find that 
	\jpg{width=0.33\linewidth}{hw8-p1-b-lhs}
	\jpg{width=0.33\linewidth}{hw8-p1-b-rhs}
	And thus we have, for all $\lambda>0$, 
	\jpg{width=0.4\linewidth}{hw8-p1-b-3}
	and so $p$ must be equal to $q$. 
	\end{proof}
	
	\item A fundamental and hard theorem of real variable theory states that
the converse is true. However, for one particular $p$ it is easy to prove that $\norm{\frac{\del_1}{|\nabla|}\phi}_p\leq C\norm{\phi}_p$ \quad $\forall \phi\in\script{S}$. Find that $p$ and prove the estimate. What is your $C$?
	\begin{proof}
	Since Hausdorff-Young gives us that $\norm{\hat{\phi}}_2\leq\norm{\phi}_2$ for all $\phi\in\S,$ then 
	\begin{align*}
	\norm{\left(\frac{k_1}{|k|}\hat{\phi}\right)^\vee}_2 
	& \leq \norm{\left(\frac{k_1}{|k|}\hat{\phi}\right)}_2 \\
	& \leq \norm{\hat{\phi}}_2 \quad \text{because } \frac{k_1}{|k|}\leq1\\
	&<\norm{\phi}_2
	\end{align*}
	So $C=1$. 
	\end{proof}
	\end{enumerate}

\pagebreak	
\item Assume $f\in C^1_\text{loc}(\R\setminus\braces{a})$	 and $\int_{a-1}^{a+1}|f'|\dx<\infty$. Show that (a) one sided limits $f(a\pm0)$ are finite and (b) use them to compute the \textit{distributional derivative} $f'$ on $\R$.
\begin{proof}
To see that the one-sided limits$\lim_{t\to a^\pm}f(t)$ are finite, observe that the Fundamental Theorem of Calculus gives us, for any $t\in B_1(a)$, 
$$f(t)= f(a+1)- \int_t^{a+1} f'(x) \dx,$$
and taking limits on both sides we find that 
$$\lim_{t\to a^+}f(t) = f(a+1)- \int_a^{a+1} f'(x) \dx < \infty$$
And similarly we have 
$$\lim_{t\to a^-}f(t) = f(a-1) + \int_{a-1}^{a} f'(x) \dx < \infty$$
and (a) is proved. For use in part (b), denote $f(a^+):= \lim_{t\to a^+}f(t)$ and $f(a^-):= \lim_{t\to a^-}f(t)$. \qedwhitehere
\end{proof}
\begin{proof}
For any $\phi$, the distributional derivative $Df(\phi)$ is 
\begin{align*}
(Df)(\phi) &=- \int_{-\infty}^\infty f\phi' \\
&= -\int_{-\infty}^{a^-} f\phi'  -\int_{a^+}^\infty f\phi'\\
&\overtext{=}{IBP}-\left(\big[f\phi\big]_{-\infty}^{a^-} - \int_{-\infty}^{a^-} f'\phi\right) - \left(\big[f\phi\big]_{a^+}^\infty - \int_{a^+}^\infty f'\phi\right) \\
&= -\big[f(a^-)\phi(a)-f(a^+)\phi(a)\big]+ \left(\int_{-\infty}^{a^-} f'\phi + \int_{a^+}^\infty f'\phi\right) \\
&= \int_{-\infty}^{a^-} f'\phi + \big[f(a^+)-f(a^-)\big]\phi(a) + \int_{a^+}^\infty f'\phi
\end{align*}
where writing $a^-$ or $a^+$ in the bounds denotes an improper integral, that is, $\int_{-\infty}^{a^-}:=\lim_{\substack{\alpha\decreasesto-\infty \\ \beta\increasesto a}}\int_\alpha^\beta$.
\end{proof}
	
\end{enumerate}


\pagebreak
\textbf{Chapter 6}

\renewcommand{\N}{\script{N}}
\newcommand{\D}{\script{D}}
\renewcommand{\L}{L}

\begin{enumerate}
\setcounter{enumi}{5}
\item Prove that the distributional derivative of a monotone nondecreasing
function on $\R$ is a Borel measure. [Hint: Use 6.13 and 6.22]

%\renewcommand{\cdot}{\bigcdot}
\begin{proof}
Let $f$ be a monotonic function, so that $T_f$ is a distribution in $\D'(\R)$, and %denote $F\in\D'(\R)$ as $F=\int_\R f(y)\phi(y)\dy$. 
let $j\in C^\infty_c(\R)$ with $\int j=1$, and for each $n\in\N$, let $j_n=2^nj(2^{-n}x)$. Using Theorem 6.13, then %there exists a function $f_n\in C^\infty(\R)$ such that 
the distributions $j_n*T_f\xto{n}T_f$
and each $j_n*T_f=T_{j_n*f}$. Note that each $j_n*f$ is itself a monotone function, since if $a<b$ then 
$$j_n*f(a) = \int_\R j_n(\cdot)f(a-\cdot) < \int_\R j_n(\cdot)f(b-\cdot) = j_n*f(b)$$
because $f$ is monotone. Now observe that since $f_n$ has a classical derivative, then the distributional derivative $DT_{f_n}={T_{f_n'}}$ and $f_n'$ is nonnegative because $f_n$ is monotonic. Now for all $\phi\geq0$, 
$${DT_{f_n}(\phi)}={T_{f_n'}(\phi)}=\int_\R f'(x)\phi(x)\dx\geq 0$$
so every $DT_{f_n}$ is a positive distribution, and since 
$$DT_{f_n}(\phi)\xto{n} DT_{f}(\phi),$$
for all $\phi$, then it is a positive distribution as well. Thus by Theorem 6.22, we can conclude that $DT_{f}$ is a Borel measure. 
\end{proof}

\item Let $\N_T$ be the null-space of a distribution, $T$. Show that there is a function $\phi_0\in \D$ so that every element $\phi\in\D$ can be written as $\phi = \lambda\phi_0+\psi$ with $\psi\in\N_T$ and $\lambda\in\R$. One says that the null-space $\N_T$ has `codimension one'.

[Hint: Recall the proof that the kernel of any linear
functional in any vector space has the codimension 1]

\begin{proof}
Let $\tilde{\phi}\in \D$ so that $T\tilde{\phi}\neq0$ (if this doesn't exist then $T=0$ and $\N_T=\D$ so we're done). Denote $\phi_0=\frac{\tilde{\phi}}{{T{\tilde{\phi}}}}$, so that $$T\phi_0=1.$$ Then for any $\phi\in \D$, we can denote $$\lambda=T\phi,$$ and observe that $T\phi=T(\lambda\phi_0)$, so $\phi-\lambda\phi_0\in\N_T$. Denoting $\psi=\phi-\lambda\phi_0$, we find that
$$\phi = \lambda\phi_0+\psi$$
and we're done. 
\end{proof}

\pagebreak
\item Show that a function $f$ is in $W^{1,\infty}(\Omega)$ if and only if $f=g$ a.e. where $g$ is a function that is bounded and Lipschitz continuous on $\Omega$, i.e., there exists a constant $C$ such that
$$|g(x)-g(y)|\leq C|x-y| \quad \text{for all } x,y\in\Omega.$$
%\begin{lemma}
%If a sequence of functions $g_n$ converges to a function $g \textbf{ as distributions}$, then $g_n$ converges uniformly (almost everywhere) to $g$ as functions. 
%\end{lemma}
%\textsc{Proof of Lemma}
%Suppose $g_n\to g$ as distributions and there exists a set of nonzero measure $A$ such that $ $
%\qedwhite

\begin{proof}$(\implies)$
Let $g\in W^{1,\infty}$. By Theorem 6.13, construct $g_n$ to be a sequence of $C^\infty$ functions converging to $g$ as distributions. Since they converge as distributions to $g$, then they converge uniformly almost everywhere to $g$\footnote{This was proved in lecture. I know how to write the proof of this fact, but I'm omitting it since we know this already.}. So there exists $N>0$ such that for all $n>N$ and all $x$, 
$$\abs{g_n(x)-g(x)}<\epsilon.$$
Then for any particular $n>N$, 
\begin{align*}
g(x+h)-g(x)&\leq g_n(x+h)-g_n(x)+2\epsilon \\
&= \int_x^{x+h} g'_n \dt + 2\epsilon & \text{ by FTC} \\
&\leq |h| \; \norm{g'_n}_\infty +2\epsilon 
\end{align*}\qedwhitehere
\end{proof}

\begin{proof}$(\impliedby)$
\jpg{width=0.99\linewidth}{hw8-p8-b-1}
\jpg{width=0.99\linewidth}{hw8-p8-b-2}
\end{proof}





\setcounter{enumi}{10}
\item Functions in $W^{1,p}(\Rn)$ can be very rough for $n\geq 2$ and $p\leq n$. 
	\begin{enumerate}[label=(\alph*)]
	\item Construct a spherically symmetric function in $W^{1,p}(\Rn)$ that diverges
to infinity as $x\to 0$. 
	\answer Let 
	$$f(x)=\ln\left(|x|^{-a}\right)\Chi_{B_1}$$
	Where $B_1$ denotes the unit ball, and $a>0$ is to be determined. Note that $f$ is written in terms of $|x|$, so it is spherically symmetric, and as $x\to0$, $f(x)$ clearly approaches $\infty$. \\
	\textbf{Claim:} With the right choice of $a$, then $f\in W^{1,p}(\R^n)$. \
	
		\begin{itemize}
		\item $f\in\L^p$. On $B_1$, we have 
		$$\ln\left(|x|^{-a}\right)=\abs{\ln\left(|x|^{-a}\right)}\leq|x|^{-a},$$
		so if we choose $0<a<\frac{n}{p}$, then 
		$$\abs{\ln\left(|x|^{-a}\right)}^p\leq|x|^{-ap},$$
		which is integrable over $B_1$. 
		
		\item $\del_i f\in L^p$. Observe that for any $i=1\dots n$, 
		$$|\del_i f|^p=\left(\frac{ax_i}{|x|^2}\right)^p\leq \frac{a^p}{|x|^p}$$
		which is integrable over $B_1$ since $p<n$, so the claim is proved. \qed
		\end{itemize}	  

	\item Use this to construct a function in $W^{1,p}(\Rn)$ that diverges to infinity at every rational point in the unit cube. 
	
	[Hint. Write the function in (b) as a sum over the rationals. How do you prove that the sum converges to a $W^{1,p}(\Rn)$ function?]
	
	\answer Let $r_j$ be an enumeration of the rationals. Note that the following function clearly diverges to infinity since it's value is at least that of $f(0)$ using the function $f$ from part (b).
	\jpg{width=0.99\linewidth}{hw8-p11-b-1}
	\jpg{width=0.99\linewidth}{hw8-p11-b-2}
	\end{enumerate}
\end{enumerate}



\end{document}
