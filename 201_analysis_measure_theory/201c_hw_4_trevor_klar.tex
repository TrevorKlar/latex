\documentclass[12pt,letterpaper]{article}

\usepackage{fancyhdr,fancybox,tensor}

%% Useful packages
\usepackage{amssymb, amsmath, amsthm} 
%\usepackage{graphicx}  %%this is currently enabled in the default document, so it is commented out here. 
\usepackage{calrsfs}
\usepackage{braket}
\usepackage{mathtools}
\usepackage{lipsum}
\usepackage{tikz}
\usetikzlibrary{cd}
\usepackage{verbatim}
%\usepackage{ntheorem}% for theorem-like environments
\usepackage{mdframed}%can make highlighted boxes of text
%Use case: https://tex.stackexchange.com/questions/46828/how-to-highlight-important-parts-with-a-gray-background
\usepackage{wrapfig}
\usepackage{centernot}
\usepackage{subcaption}%\begin{subfigure}{0.5\textwidth}
\usepackage{pgfplots}
\pgfplotsset{compat=1.13}
\usepackage[colorinlistoftodos]{todonotes}
\usepackage[colorlinks=true, allcolors=blue]{hyperref}
\usepackage{xfrac}					%to make slanted fractions \sfrac{numerator}{denominator}
\usepackage{enumitem}            
    %syntax: \begin{enumerate}[label=(\alph*)]
    %possible arguments: f \alph*, \Alph*, \arabic*, \roman* and \Roman*
\usetikzlibrary{arrows,shapes.geometric,fit}

\DeclareMathAlphabet{\pazocal}{OMS}{zplm}{m}{n}
%% Use \pazocal{letter} to typeset a letter in the other kind 
%%  of math calligraphic font. 

%% This puts the QED block at the end of each proof, the way I like it. 
\renewenvironment{proof}{{\bfseries Proof}}{\qed}
\makeatletter
\renewenvironment{proof}[1][\bfseries \proofname]{\par
  \pushQED{\qed}%
  \normalfont \topsep6\p@\@plus6\p@\relax
  \trivlist
  %\itemindent\normalparindent
  \item[\hskip\labelsep
        \scshape
    #1\@addpunct{}]\ignorespaces
}{%
  \popQED\endtrivlist\@endpefalse
}
\makeatother

%% This adds a \rewnewtheorem command, which enables me to override the settings for theorems contained in this document.
\makeatletter
\def\renewtheorem#1{%
  \expandafter\let\csname#1\endcsname\relax
  \expandafter\let\csname c@#1\endcsname\relax
  \gdef\renewtheorem@envname{#1}
  \renewtheorem@secpar
}
\def\renewtheorem@secpar{\@ifnextchar[{\renewtheorem@numberedlike}{\renewtheorem@nonumberedlike}}
\def\renewtheorem@numberedlike[#1]#2{\newtheorem{\renewtheorem@envname}[#1]{#2}}
\def\renewtheorem@nonumberedlike#1{  
\def\renewtheorem@caption{#1}
\edef\renewtheorem@nowithin{\noexpand\newtheorem{\renewtheorem@envname}{\renewtheorem@caption}}
\renewtheorem@thirdpar
}
\def\renewtheorem@thirdpar{\@ifnextchar[{\renewtheorem@within}{\renewtheorem@nowithin}}
\def\renewtheorem@within[#1]{\renewtheorem@nowithin[#1]}
\makeatother

%% This makes theorems and definitions with names show up in bold, the way I like it. 
\makeatletter
\def\th@plain{%
  \thm@notefont{}% same as heading font
  \itshape % body font
}
\def\th@definition{%
  \thm@notefont{}% same as heading font
  \normalfont % body font
}
\makeatother

%===============================================
%==============Shortcut Commands================
%===============================================
\newcommand{\ds}{\displaystyle}
\newcommand{\B}{\mathcal{B}}
\newcommand{\C}{\mathbb{C}}
\newcommand{\F}{\mathbb{F}}
\newcommand{\N}{\mathbb{N}}
\newcommand{\R}{\mathbb{R}}
\newcommand{\Q}{\mathbb{Q}}
\newcommand{\T}{\mathcal{T}}
\newcommand{\Z}{\mathbb{Z}}
\renewcommand\qedsymbol{$\blacksquare$}
\newcommand{\qedwhite}{\hfill\ensuremath{\square}}
\newcommand*\conj[1]{\overline{#1}}
\newcommand*\closure[1]{\overline{#1}}
\newcommand*\mean[1]{\overline{#1}}
%\newcommand{\inner}[1]{\left< #1 \right>}
\newcommand{\inner}[2]{\left< #1, #2 \right>}
\newcommand{\powerset}[1]{\pazocal{P}(#1)}
%% Use \pazocal{letter} to typeset a letter in the other kind 
%%  of math calligraphic font. 
\newcommand{\cardinality}[1]{\left| #1 \right|}
\newcommand{\domain}[1]{\mathcal{D}(#1)}
\newcommand{\image}{\text{Im}}
\newcommand{\inv}[1]{#1^{-1}}
\newcommand{\preimage}[2]{#1^{-1}\left(#2\right)}
\newcommand{\script}[1]{\mathcal{#1}}


\newenvironment{highlight}{\begin{mdframed}[backgroundcolor=gray!20]}{\end{mdframed}}

\DeclarePairedDelimiter\ceil{\lceil}{\rceil}
\DeclarePairedDelimiter\floor{\lfloor}{\rfloor}

%===============================================
%===============My Tikz Commands================
%===============================================
\newcommand{\drawsquiggle}[1]{\draw[shift={(#1,0)}] (.005,.05) -- (-.005,.02) -- (.005,-.02) -- (-.005,-.05);}
\newcommand{\drawpoint}[2]{\draw[*-*] (#1,0.01) node[below, shift={(0,-.2)}] {#2};}
\newcommand{\drawopoint}[2]{\draw[o-o] (#1,0.01) node[below, shift={(0,-.2)}] {#2};}
\newcommand{\drawlpoint}[2]{\draw (#1,0.02) -- (#1,-0.02) node[below] {#2};}
\newcommand{\drawlbrack}[2]{\draw (#1+.01,0.02) --(#1,0.02) -- (#1,-0.02) -- (#1+.01,-0.02) node[below, shift={(-.01,0)}] {#2};}
\newcommand{\drawrbrack}[2]{\draw (#1-.01,0.02) --(#1,0.02) -- (#1,-0.02) -- (#1-.01,-0.02) node[below, shift={(+.01,0)}] {#2};}

%***********************************************
%**************Start of Document****************
%***********************************************
 %find me at /home/trevor/texmf/tex/latex/tskpreamble_nothms.tex
%===============================================
%===============Theorem Styles==================
%===============================================

%================Default Style==================
\theoremstyle{plain}% is the default. it sets the text in italic and adds extra space above and below the \newtheorems listed below it in the input. it is recommended for theorems, corollaries, lemmas, propositions, conjectures, criteria, and (possibly; depends on the subject area) algorithms.
\newtheorem{theorem}{Theorem}
\numberwithin{theorem}{section} %This sets the numbering system for theorems to number them down to the {argument} level. I have it set to number down to the {section} level right now.
\newtheorem*{theorem*}{Theorem} %Theorem with no numbering
\newtheorem{corollary}[theorem]{Corollary}
\newtheorem*{corollary*}{Corollary}
\newtheorem{conjecture}[theorem]{Conjecture}
\newtheorem{lemma}[theorem]{Lemma}
\newtheorem*{lemma*}{Lemma}
\newtheorem{proposition}[theorem]{Proposition}
\newtheorem*{proposition*}{Proposition}
\newtheorem{problemstatement}[theorem]{Problem Statement}


%==============Definition Style=================
\theoremstyle{definition}% adds extra space above and below, but sets the text in roman. it is recommended for definitions, conditions, problems, and examples; i've alse seen it used for exercises.
\newtheorem{definition}[theorem]{Definition}
\newtheorem*{definition*}{Definition}
\newtheorem{condition}[theorem]{Condition}
\newtheorem{problem}[theorem]{Problem}
\newtheorem{example}[theorem]{Example}
\newtheorem*{example*}{Example}
\newtheorem*{counterexample*}{Counterexample}
\newtheorem*{romantheorem*}{Theorem} %Theorem with no numbering
\newtheorem{exercise}{Exercise}
\numberwithin{exercise}{section}
\newtheorem{algorithm}[theorem]{Algorithm}

%================Remark Style===================
\theoremstyle{remark}% is set in roman, with no additional space above or below. it is recommended for remarks, notes, notation, claims, summaries, acknowledgments, cases, and conclusions.
\newtheorem{remark}[theorem]{Remark}
\newtheorem*{remark*}{Remark}
\newtheorem{notation}[theorem]{Notation}
\newtheorem*{notation*}{Notation}
%\newtheorem{claim}[theorem]{Claim}  %%use this if you ever want claims to be numbered
\newtheorem*{claim}{Claim}


%%
%% Page set-up:
%%
\pagestyle{empty}
\lhead{\textsc{201c - Functional Analysis} \\Quarter of COVID-19} 
\rhead{\textsc{Labutin, Spring 2020} \\ Trevor Klar}
%\chead{\Large\textbf{A New Integration Technique \\ }}
\renewcommand{\headrulewidth}{0pt}
%
\renewcommand{\footrulewidth}{0pt}
%\lfoot{
%Office: \quad \quad \, M 2-3 \, \, SH 6431x \\
%Math Lab: \, W 12-2 \, SH 1607
%}
%\rfoot{trevorklar@math.ucsb.edu}

\setlength{\parindent}{0in}
\setlength{\textwidth}{7in}
\setlength{\evensidemargin}{-0.25in}
\setlength{\oddsidemargin}{-0.25in}
\setlength{\parskip}{.5\baselineskip}
\setlength{\topmargin}{-0.5in}
\setlength{\textheight}{9in}

\setlist[enumerate,1]{label=\textbf{\arabic*.}}

\let\oldphi\phi
\renewcommand{\phi}{\varphi}
\renewcommand{\epsilon}{\varepsilon}

\begin{document}
\pagestyle{fancy}
\begin{center}
{\Large Homework 4}%=================UPDATE THIS=================%
\end{center}

\renewcommand{\B}{\bar{B}(\ell^\infty)}
\textbf{Chapter 2}

\begin{enumerate}
%1
\setcounter{enumi}{1}
\item
	\begin{enumerate}[label=(\alph*)]
	\item Prove 2.1(6): When $f\in L^\infty(\Omega)\cap L^q(\Omega)$ for some $q$, then $f\in L^p(\Omega)$ for all $p>q$ and 
$$\norm{f}_\infty = \lim_{p\to\infty}\norm{f}_p.$$
	
	\item Prove that when $\infty\geq r\geq q\geq 1$, 
	$$f\in L^r(\Omega)\cap L^q(\Omega) \implies f \in L^p(\Omega)$$ 
	for all $r\geq p\geq q$.
	\end{enumerate} 
	
\begin{proof}(a)
Let $f\in L^q$, and $p>q$ with $\norm{f}_\infty<\infty$. We need to show that $\abs{f}^p$ is summable, that is, $\int |f|^p <\infty$.\footnote{To suppress notation, we will omit the region of integration to denote that the region is all of $\Omega$. We will also omit the measure unless it is needed.}
\begin{align*}
\int |f|^p &= \int \abs{f}^q\abs{f}^{p-q} \\
&\leq \int \abs{f}^q \cdot\norm{f}_\infty^{p-q} \\
&=\norm{f}_\infty^{p-q} \int \abs{f}^q \\
&<\infty.
\end{align*}
Thus $f\in L^p(\Omega)$. 

Now observe that for all $p>q$, $\norm{f}_p\leq\norm{f}_\infty$. 
\begin{align*}
\norm{f}_p&=\left(\int |f|^p \right)^{\frac{1}{p}} \\
&\leq\left(\int \norm{f}_\infty^p \right)^{\frac{1}{p}} \\
&=\big(\norm{f}_\infty^p \mu(\Omega) \big)^{\frac{1}{p}} \\
&=\norm{f}_\infty \cdot \big(\mu(\Omega) \big)^{\frac{1}{p}} \\
\end{align*}
and since $\mu(\Omega)$ is finite, then 
$$\lim_{p\to\infty}\left(\norm{f}_\infty \cdot \big(\mu(\Omega) \big)^{\frac{1}{p}}\right) = \norm{f}_\infty.$$
Thus part (a) is proved. \qedwhite

\pagebreak
(b) Let $p$ such that $\infty\geq r\geq p \geq q \geq 1$, and $f\in L^r(\Omega)\cap L^q(\Omega)$. Denote 
\begin{align*}
\{|f|\leq1\} &= \{x\in \Omega : |f(x)|\leq 1\},\text{ and }\\
\{|f|>1\} &= \{x\in \Omega : |f(x)|> 1\}.
\end{align*}
then observe that $|f|^p$ is summable:
\begin{align*}
\int |f|^p &= \int_{\{|f|\leq1\}} |f|^p + \int_{\{|f|>1\}} |f|^p \\
&= \int_{\{|f|\leq1\}} |f|^q + \int_{\{|f|>1\}} |f|^r \\
&= \int |f|^q + \int |f|^r \\
&<\infty. \qedhere
\end{align*}
\end{proof}	

%\pagebreak	
\setcounter{enumi}{8}
\item In Sect. 2.9 three ways are shown for which an $L^p(\R^n)$ sequence $f^k$ can
converge weakly to zero but $f^k$ does not converge to anything strongly.
Verify this for the three examples given in 2.9 (page 56):
	\begin{enumerate}
	\item $f_k$ `oscillates to death': An example is $f_k(x) = \sin kx$ for $0 \leq x \leq 1$ and zero otherwise.
	\begin{proof} As we did in HW1 Problem 5, we use integration by parts and find that $\forall\,g\in C(\R)$, 
\begin{align*}
\int_0^1 g(x)\sin(kx) \dx &= -g(x)\frac{1}{k}\cos(kx) - \int_0^1 g'(x)\frac{1}{k}\cos(kx)\dx \\
&= \frac{1}{k}\left[-g(x)\cos(kx) - \int_0^1 g'(x)\cos(kx)\dx\right],
\end{align*}
and in the limit as $n\to \infty$, everything goes to 0. 

For arbitrary $g\in L^2(\R)$, $g'$ may not so exist, so we use Weierstrauss Approximation Theorem, for every $\epsilon>0$, there exists a polynomial $h$ such that $\sup_I\abs{g-h}\leq\epsilon$. Thus, 
\begin{align*}
\int g(x) \sin(kx)\dx &= \int g(x)\sin(kx)\dx - \int h(x)\sin(kx)\dx + \int h(x)\sin(kx)\dx \\
&= \int (g(x)-h(x))\sin(kx)\dx + \int g(x)\sin(kx)\dx,
\end{align*}
and this integral is bounded above and below by 
$$\int (\pm \epsilon +h(x))\sin(kx)\dx$$
respectively, which integrands are themselves polynomials, so they vanish in the limit. Therefore $\lim_{n\to\infty}\int g(x)\sin(kx)\dx=0$ by the squeeze theorem, and we conclude that $f_k\xto{w}0$. 

Finally, note that if we fix $x$ and let $k\to\infty$, then $\sin(x)$ takes values all over $[0,1]$, so it doesn't converge pointwise. Therefore, $f_k$ doesn't converge strongly to anything. \qedwhitehere
\end{proof}
	\item $f_k$ `goes up the spout': An example is $f_k(x) = k^{\sfrac{1}{p}}	g(kx)$ where $g$ is any fixed function in $L^p(\R^1)$. This sequence becomes very large near $x = 0$.
	\begin{proof}
	$f_k\xto{w}0$ because for any $h\in L^{p'}(\R)$, 
	\begin{align*}
	\int f_k(x)h(x)\dx&=k^{\sfrac{1}{p}}\int g(kx)h(x)\dx\\	
	&=k^{\sfrac{1}{p}}\int g(kx)h(x)\dx\\	
	&\leq k^{\sfrac{1}{p}} \left(\int |g(kx)|^p\right)^{\frac{1}{p}} \norm{h}_{p'} \\
	&\leq \frac{k^{\sfrac{1}{p}}}{k^p} \norm{g}_p \norm{h}_{p'} &\text{by change in variables}\\
	\end{align*}
	and as $k\to\infty$, this all goes to zero. 
	
	Observe that $f_k$ converges pointwise to zero, so if it does converge strongly, it must converge to 0: Indeed, since $\Chi_{B_1(x_0)}$ is an $L^{p'}(\R)$ function for any fixed $x_0$, and since $f_k\xto{w}0$, then 
	$$\int f_k\cdot \Chi_{B_1(x_0)}\to 0,$$
	so $f_k(x)\xto{k}0$. 	However, $\norm{f_k}_p$ is constant and nonzero:
	\begin{align*}
	\norm{f_k}_p &= \left(\int \abs{k^{\sfrac{1}{p}}g(kx)}^p\right)^{\frac{1}{p}} \\
	&= k^{\sfrac{1}{p}}\left(\int \abs{g(kx)}^p\right)^{\frac{1}{p}} \\
	&= k^{\sfrac{1}{p}}\left(\frac{1}{k}\int \abs{g(t)}^p\right)^{\frac{1}{p}}&\text{by change in variables} \\
	&= \frac{k^{\sfrac{1}{p}}}{k^{\sfrac{1}{p}}}\left(\int \abs{g(t)}^p\right)^{\frac{1}{p}}\\
	&= \norm{g}_p
	\end{align*}
	Thus $f_k$ cannot converge strongly to zero, and it cannot converge strongly to anything else. \qedwhitehere
	\end{proof}
	
	\pagebreak
	\item $f_k$ `wanders off to infinity': An example is $f_k(x) = g(x + k)$ for some fixed function $g$ in $L^p(\R^1)$.
	\begin{proof}
	Assuming we can prove that $f_k\xto{w}0$, then $f_k$ converges pointwise to zero as well for the same reasons as in (ii). Thus if the sequence converges strongly, then it converges to zero. However $f_k$ clearly cannot converge strongly to zero, since Lebesgue measure is translation-invariant, so $\norm{f_k}_p=\norm{g}_p$ for all $k$. 
	
	As regards weak convergence, I think we can approximate with compact supported functions and use Dominated Convergence to yield the result. Intuitively, $g(x+k)$ aught to get small as $k\to\infty$, and multiplying by the fixed number $h(x)$ won't stop the small-enizing. As to the details, I ran out of time. This homework was too long. 
	\end{proof}
	\end{enumerate}

\setcounter{enumi}{10}
\item With the usual $j_\epsilon \in C^\infty_c$, show that if $f$ is continuous on $\R$, then $j_\epsilon*f(x)$ converges to $f(x)$ for all $x$, and it does so uniformly on each compact subset of $\R^n$. 
\begin{proof}
Let 
$$j_\epsilon = \epsilon\Chi_{[0,1/\epsilon]}.$$
Then 
\begin{align*}
j_\epsilon*f(x)&=\int f(x-t)\epsilon\Chi_{[0,1/\epsilon]}(t)\dt \\
&=\epsilon\int_0^{1/\epsilon} f(x-t)\dt \\
&=\text{average value of }f(t)\text{ on }[x-1/\epsilon,x]
\end{align*}
and since $f$ is continuous, taking $\lim_{\epsilon\to\infty}$\footnote{I'm \emph{very} uncomfortable with sending a variable called $\epsilon$ to $\infty$. Maybe what I'm doing is fine, but it's late, my brain hurts, and this homework was way too long.}, we obtain $\lim_{t\to x^-}f(x)=f(x)$.  
\end{proof}

\setcounter{enumi}{17} 
\item Prove that every convex function $f$ has a support plane at every $x$ in the interior of its domain $D\subset\R$, as claimed in Sect. 2.1. See also Exercise 3.1.

\begin{proof}
We know that every convex function is continuous on an open set (such as $\text{int} D$), but and if $f$ is also differentiable, then clearly it has a tangent plane which is in fact a support plane. 

In this spirit, we can find a support plane for a general convex function by using right derivatives, which always exist. %Let $G_f:\R^n\to\R^{n+1}$ denote the graph: $G_f(x)=(x,f(x))$. Then 
The support plane is 
$$\text{span}(1,f'_+(x)),$$
%the span of the right directional derivative of $G_f$ in the direction of each basis vector $e_1, \dots, e_n$. 

To see this, observe that since $f$ is convex, then 
$$\frac{f(x+t)-f(x)}{t}$$
decreases as $t\to0^+$, so since
$$f'_+(x)=\lim_{t\to0^+}\frac{f(x+t)-f(x)}{t},$$
%where $f_i$ is the $i$-th coordinate function of $f$ and $x_i$ is the $i$-th coordinate. 
then $f(x)+tf'_+(x)\leq f(x+t)$ for all nonnegative $t$. 

Now considering negative values of $t$; since $f$ is convex, then 
$$f'_-\leq f'_+, $$
and $f'_-$ increases as $t\to 0^-$, so $f(x)+tf'_+(x)\leq f(x)+tf'_-(x)\leq f(x+t)$ for $t<0$, so 
$$f(x)+tf'_+(x)\leq f(x+t)$$
always, and we're done.
\end{proof}

\setcounter{enumi}{22}
\item Find a sequence of functions with the property that 
	\begin{enumerate}
	\item $f_n\xto{w}0$ in $L^2(\Omega)$, and
	\item $f_n\to0$ strongly in $L^\frac{3}{2}(\Omega)$, but 
	\item $f_n\not\to0$ strongly in $L^2(\Omega)$.
	\end{enumerate}
	
	\answer	Let $f_n$ be the following sequence in $g\in L^2(\R)$ which "goes up the spout": 
	$$f_n=\sqrt{n} \;\Chi_{\left[0,{1/n}\right]}$$
	\begin{proof}
	(i) For any $g\in L^2(\R)$, 
	\begin{align*}
	\abs{\int f_ng \der\mu} %& \leq \left(\int_0^{1/n} \sqrt{n}^2\right)^\frac{1}{2} \norm{g}_2 \\
	&=\abs{\int_0^{1/n} \sqrt{n}g } \\
	&\leq \sqrt{n}\int_0^{1/n}\abs{g } & \text{by Holder}\\
	&\leq \sqrt{n}\int_0^{1/n}\abs{g }^2 &\text{since }x>x^2\text{ in }[0,1/n]\\
	\end{align*}
	And since $\int_\R\abs{g}^2$ is finite, then $\lim_{n\to\infty} \int_0^{1/n} \abs{g}^2=0$. 	Thus $f_n\xto{w}0$. \qedwhite

	(ii) Observe, 
	\begin{align*}
	\norm{f_n}_{\frac{3}{2}} &= \abs{\int_0^{1/n} \sqrt{n}^\frac{3}{2} }^{\frac{2}{3}}\\
	&=n^{{-5}/{6}} \\
	&\xto{n}0.
	\end{align*}
	Thus $f_n\to0$ strongly in $L^\frac{3}{2}(\R)$. \qedwhite
	
	\pagebreak
	(iii) However, $\norm{f_n}_2$ is constantly 1, since 
	\begin{align*}
	\norm{f_n}_2 &= \left(\int_0^n \sqrt{n}^2\right)^\frac{1}{2} \\
	&=\sqrt{1} \\
	&=1,
	\end{align*}
	so $f_n\not\to0$ in $L^2(\R)$. 
	\end{proof}

\end{enumerate}
\end{document}
