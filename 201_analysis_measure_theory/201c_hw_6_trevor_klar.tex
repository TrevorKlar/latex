 \documentclass[12pt,letterpaper]{article}

\usepackage{fancyhdr,fancybox,tensor}

%% Useful packages
\usepackage{amssymb, amsmath, amsthm} 
%\usepackage{graphicx}  %%this is currently enabled in the default document, so it is commented out here. 
\usepackage{calrsfs}
\usepackage{braket}
\usepackage{mathtools}
\usepackage{lipsum}
\usepackage{tikz}
\usetikzlibrary{cd}
\usepackage{verbatim}
%\usepackage{ntheorem}% for theorem-like environments
\usepackage{mdframed}%can make highlighted boxes of text
%Use case: https://tex.stackexchange.com/questions/46828/how-to-highlight-important-parts-with-a-gray-background
\usepackage{wrapfig}
\usepackage{centernot}
\usepackage{subcaption}%\begin{subfigure}{0.5\textwidth}
\usepackage{pgfplots}
\pgfplotsset{compat=1.13}
\usepackage[colorinlistoftodos]{todonotes}
\usepackage[colorlinks=true, allcolors=blue]{hyperref}
\usepackage{xfrac}					%to make slanted fractions \sfrac{numerator}{denominator}
\usepackage{enumitem}            
    %syntax: \begin{enumerate}[label=(\alph*)]
    %possible arguments: f \alph*, \Alph*, \arabic*, \roman* and \Roman*
\usetikzlibrary{arrows,shapes.geometric,fit}

\DeclareMathAlphabet{\pazocal}{OMS}{zplm}{m}{n}
%% Use \pazocal{letter} to typeset a letter in the other kind 
%%  of math calligraphic font. 

%% This puts the QED block at the end of each proof, the way I like it. 
\renewenvironment{proof}{{\bfseries Proof}}{\qed}
\makeatletter
\renewenvironment{proof}[1][\bfseries \proofname]{\par
  \pushQED{\qed}%
  \normalfont \topsep6\p@\@plus6\p@\relax
  \trivlist
  %\itemindent\normalparindent
  \item[\hskip\labelsep
        \scshape
    #1\@addpunct{}]\ignorespaces
}{%
  \popQED\endtrivlist\@endpefalse
}
\makeatother

%% This adds a \rewnewtheorem command, which enables me to override the settings for theorems contained in this document.
\makeatletter
\def\renewtheorem#1{%
  \expandafter\let\csname#1\endcsname\relax
  \expandafter\let\csname c@#1\endcsname\relax
  \gdef\renewtheorem@envname{#1}
  \renewtheorem@secpar
}
\def\renewtheorem@secpar{\@ifnextchar[{\renewtheorem@numberedlike}{\renewtheorem@nonumberedlike}}
\def\renewtheorem@numberedlike[#1]#2{\newtheorem{\renewtheorem@envname}[#1]{#2}}
\def\renewtheorem@nonumberedlike#1{  
\def\renewtheorem@caption{#1}
\edef\renewtheorem@nowithin{\noexpand\newtheorem{\renewtheorem@envname}{\renewtheorem@caption}}
\renewtheorem@thirdpar
}
\def\renewtheorem@thirdpar{\@ifnextchar[{\renewtheorem@within}{\renewtheorem@nowithin}}
\def\renewtheorem@within[#1]{\renewtheorem@nowithin[#1]}
\makeatother

%% This makes theorems and definitions with names show up in bold, the way I like it. 
\makeatletter
\def\th@plain{%
  \thm@notefont{}% same as heading font
  \itshape % body font
}
\def\th@definition{%
  \thm@notefont{}% same as heading font
  \normalfont % body font
}
\makeatother

%===============================================
%==============Shortcut Commands================
%===============================================
\newcommand{\ds}{\displaystyle}
\newcommand{\B}{\mathcal{B}}
\newcommand{\C}{\mathbb{C}}
\newcommand{\F}{\mathbb{F}}
\newcommand{\N}{\mathbb{N}}
\newcommand{\R}{\mathbb{R}}
\newcommand{\Q}{\mathbb{Q}}
\newcommand{\T}{\mathcal{T}}
\newcommand{\Z}{\mathbb{Z}}
\renewcommand\qedsymbol{$\blacksquare$}
\newcommand{\qedwhite}{\hfill\ensuremath{\square}}
\newcommand*\conj[1]{\overline{#1}}
\newcommand*\closure[1]{\overline{#1}}
\newcommand*\mean[1]{\overline{#1}}
%\newcommand{\inner}[1]{\left< #1 \right>}
\newcommand{\inner}[2]{\left< #1, #2 \right>}
\newcommand{\powerset}[1]{\pazocal{P}(#1)}
%% Use \pazocal{letter} to typeset a letter in the other kind 
%%  of math calligraphic font. 
\newcommand{\cardinality}[1]{\left| #1 \right|}
\newcommand{\domain}[1]{\mathcal{D}(#1)}
\newcommand{\image}{\text{Im}}
\newcommand{\inv}[1]{#1^{-1}}
\newcommand{\preimage}[2]{#1^{-1}\left(#2\right)}
\newcommand{\script}[1]{\mathcal{#1}}


\newenvironment{highlight}{\begin{mdframed}[backgroundcolor=gray!20]}{\end{mdframed}}

\DeclarePairedDelimiter\ceil{\lceil}{\rceil}
\DeclarePairedDelimiter\floor{\lfloor}{\rfloor}

%===============================================
%===============My Tikz Commands================
%===============================================
\newcommand{\drawsquiggle}[1]{\draw[shift={(#1,0)}] (.005,.05) -- (-.005,.02) -- (.005,-.02) -- (-.005,-.05);}
\newcommand{\drawpoint}[2]{\draw[*-*] (#1,0.01) node[below, shift={(0,-.2)}] {#2};}
\newcommand{\drawopoint}[2]{\draw[o-o] (#1,0.01) node[below, shift={(0,-.2)}] {#2};}
\newcommand{\drawlpoint}[2]{\draw (#1,0.02) -- (#1,-0.02) node[below] {#2};}
\newcommand{\drawlbrack}[2]{\draw (#1+.01,0.02) --(#1,0.02) -- (#1,-0.02) -- (#1+.01,-0.02) node[below, shift={(-.01,0)}] {#2};}
\newcommand{\drawrbrack}[2]{\draw (#1-.01,0.02) --(#1,0.02) -- (#1,-0.02) -- (#1-.01,-0.02) node[below, shift={(+.01,0)}] {#2};}

%***********************************************
%**************Start of Document****************
%***********************************************
 %find me at /home/trevor/texmf/tex/latex/tskpreamble_nothms.tex
%===============================================
%===============Theorem Styles==================
%===============================================

%================Default Style==================
\theoremstyle{plain}% is the default. it sets the text in italic and adds extra space above and below the \newtheorems listed below it in the input. it is recommended for theorems, corollaries, lemmas, propositions, conjectures, criteria, and (possibly; depends on the subject area) algorithms.
\newtheorem{theorem}{Theorem}
\numberwithin{theorem}{section} %This sets the numbering system for theorems to number them down to the {argument} level. I have it set to number down to the {section} level right now.
\newtheorem*{theorem*}{Theorem} %Theorem with no numbering
\newtheorem{corollary}[theorem]{Corollary}
\newtheorem*{corollary*}{Corollary}
\newtheorem{conjecture}[theorem]{Conjecture}
\newtheorem{lemma}[theorem]{Lemma}
\newtheorem*{lemma*}{Lemma}
\newtheorem{proposition}[theorem]{Proposition}
\newtheorem*{proposition*}{Proposition}
\newtheorem{problemstatement}[theorem]{Problem Statement}


%==============Definition Style=================
\theoremstyle{definition}% adds extra space above and below, but sets the text in roman. it is recommended for definitions, conditions, problems, and examples; i've alse seen it used for exercises.
\newtheorem{definition}[theorem]{Definition}
\newtheorem*{definition*}{Definition}
\newtheorem{condition}[theorem]{Condition}
\newtheorem{problem}[theorem]{Problem}
\newtheorem{example}[theorem]{Example}
\newtheorem*{example*}{Example}
\newtheorem*{counterexample*}{Counterexample}
\newtheorem*{romantheorem*}{Theorem} %Theorem with no numbering
\newtheorem{exercise}{Exercise}
\numberwithin{exercise}{section}
\newtheorem{algorithm}[theorem]{Algorithm}

%================Remark Style===================
\theoremstyle{remark}% is set in roman, with no additional space above or below. it is recommended for remarks, notes, notation, claims, summaries, acknowledgments, cases, and conclusions.
\newtheorem{remark}[theorem]{Remark}
\newtheorem*{remark*}{Remark}
\newtheorem{notation}[theorem]{Notation}
\newtheorem*{notation*}{Notation}
%\newtheorem{claim}[theorem]{Claim}  %%use this if you ever want claims to be numbered
\newtheorem*{claim}{Claim}


%%
%% Page set-up:
%%
\pagestyle{empty}
\lhead{\textsc{201c - Functional Analysis} \\Quarter of COVID-19} 
\rhead{\textsc{Labutin, Spring 2020} \\ Trevor Klar}
%\chead{\Large\textbf{A New Integration Technique \\ }}
\renewcommand{\headrulewidth}{0pt}
%
\renewcommand{\footrulewidth}{0pt}
%\lfoot{
%Office: \quad \quad \, M 2-3 \, \, SH 6431x \\
%Math Lab: \, W 12-2 \, SH 1607
%}
%\rfoot{trevorklar@math.ucsb.edu}

\setlength{\parindent}{0in}
\setlength{\textwidth}{7in}
\setlength{\evensidemargin}{-0.25in}
\setlength{\oddsidemargin}{-0.25in}
\setlength{\parskip}{.5\baselineskip}
\setlength{\topmargin}{-0.5in}
\setlength{\textheight}{9in}

\setlist[enumerate,1]{label=\textbf{\arabic*.}}

\let\oldphi\phi
\renewcommand{\phi}{\varphi}
\renewcommand{\epsilon}{\varepsilon}

\begin{document}
\pagestyle{fancy}
\begin{center}
{\Large Homework 6}%=================UPDATE THIS=================%
\end{center}

\renewcommand{\B}{\bar{B}(\ell^\infty)}
\textbf{Chapter 5}

\begin{enumerate}
\setcounter{enumi}{1}
\item Prove the Reimann-Lebesgue lemma mentioned in Section 5.1., i.e., for $f\in L^1(\R^n)$, 
$$\hat{f}(k)\to 0\text{ as }k\to \infty.$$
[\textit{Hint}. 5.3(1) is useful. 
\jpg{width=0.9\textwidth}{plancharel-thm-5-3-1}
\begin{proof}
I didn't have time to do all of these problems, since this week we have a midterm in this class and another class, in addition to the regular work which I can barely keep up with. 
\end{proof}

\setcounter{enumi}{4}
\item Complete the proof of Theorem 5.8, i.e., work out the approximation
argument mentioned at the end of Sect. 5.8:
\jpg{width=.9\textwidth}{theorem-5-8}

\begin{proof}
Let $h,u\in C^\infty_0(\R^n)$. In particular, $h,u\in L^1(\R^n)$, so $\widehat{h*u}=\hat h \hat u$ by (1). Now consider the operators 
\begin{align*}
T:L^p(\R^n)\times L^q(\R^n) &\to L^\infty(\R^n) \text{ given by}\\
 (f,g)&\mapsto f*g\\
 \\
S:L^p(\R^n)\times L^q(\R^n) &\to L^{r'}(\R^n) \text{ given by}\\
 (f,g)&\mapsto \hat f \hat g\\
 \\
\hat \bigcdot : L^{r'}&\to C_0 \text{, the Fourier Transform,}
\end{align*}
all of which are clearly continuous operators. Observe that $\hat \bigcdot \circ T = S$ on $C^\infty_0\times C^\infty_0$, and thus we have two continuous maps which agree on a dense subset of their domains, so they agree everywhere. 
\end{proof}

\item %For $f\in C^\infty_c(\R^n)$ show that its Fourier transform $\hat{f}$ is also in $C^\infty$. Show also that 
%$$g_a(k):=$$
\mbox{}\vspace*{-28pt}
\jpg{width=0.99\linewidth}{hw6-ch5-p6}
\begin{proof}
I didn't have time to do all of these problems, since this week we have a midterm in this class and another class, in addition to the regular work which I can barely keep up with. 
\end{proof}

\setcounter{enumi}{8}
\item Verify that 5.6(1) cannot hold when $\rho > 2$ by considering Gaussian 
functions, as in 5.2(1), with $\lambda=a+ib$ and with $a>0$. 

\jpg{width=0.99\linewidth}{5-6(1)-title}
\jpg{width=0.99\linewidth}{5-6(1)}
\jpg{width=0.99\linewidth}{theorem-5-2(1)}
\begin{proof}
I didn't have time to do all of these problems, since this week we have a midterm in this class and another class, in addition to the regular work which I can barely keep up with. 
\end{proof}


\end{enumerate}

%\pagebreak
\textbf{Problems from the PDF}

\textbf{Assignment 6}

\begin{enumerate}
\setcounter{enumi}{1}
\item This problem is based on the scaling. 
	\begin{enumerate}[label=(\alph*)]
	\item Let for some $1\leq p\leq \infty$ the inequality (Sobolev inequality)
	$$\norm{u}_p\leq C(n,p)\norm{\nabla u}_1
	$$
	hold for all $u\in C_0^\infty(\R^n), n\geq2$, with the constant $C$ independent of $u$. What are the possible values (the \textit{necessary} conditions) of $p$?
	\begin{proof}
	For all $\lambda> 0$, let $u_\lambda:= u\left(\frac{x}{\lambda}\right)$. Now $u_\lambda\in C^\infty_0$, so the Sobolev inequality holds for $u_\lambda$. Consider the right hand side, 
	\begin{align*}
	\norm{\nabla u}_1 
	&= \int \abs{\nabla\left[u\left(\tfrac{x}{\lambda}\right)\right]}\dx \\
	&= \int \frac{1}{\lambda}\abs{\left(\nabla u\right)\left(\tfrac{x}{\lambda}\right)}\dx &&\text{ by the chain rule}\\
	&= \lambda^{n-1}\int \abs{\left(\nabla u\right)\left(x\right)}\dx &&\text{ by change of variables}\\
	&= \lambda^{n-1}\norm{\nabla u}_1.
	\end{align*}
	Now consider the left hand side,
	\begin{align*}
	\norm{u_\lambda}_p &= \left(\int_{R^n}\abs{u\left(\tfrac{x}{\lambda}\right)}^p\dx\right)^{1/p} \\
	&= \left(\lambda^{n}\int_{R^n}\abs{u\left(x\right)}^p\dx\right)^{1/p} &&\text{ by change of variables}\\
	&= \lambda^{\sfrac{n}{p}}\left(\int_{R^n}\abs{u\left(x\right)}^p\dx\right)^{1/p}\\
	&= \lambda^{\sfrac{n}{p}} \norm{u}_p.
	\end{align*}
	Substituting and collecting $\lambda$ on the left side gives 
	$$\lambda^{\frac{n}{p}-n+1}\norm{u}_p \leq C\norm{\nabla u}_1,
	$$
	and this is true for all $\lambda>0$, even while the right hand side is finite and constant with respect to $\lambda$. Thus 
	$$\tfrac{n}{p}-n+1\leq 0,	$$
	otherwise $\lambda^{\frac{n}{p}-n+1}\norm{u}_p\to\infty$ as $\lambda\to\infty$. However we also have that 
	$$\tfrac{n}{p}-n+1\geq 0,	$$
	otherwise $\lambda^{\frac{n}{p}-n+1}\norm{u}_p\to\infty$ as $\lambda\to0$.
	Thus $\tfrac{n}{p}-n+1= 0,	$ so $p=\frac{n}{n-1}=n'$. \qedwhitehere
	\end{proof}
	
%	\pagebreak
	\item \mbox{}\vspace*{-22pt}
	\jpg{width=0.99\linewidth}{hw6-pdf2-b}
	\begin{proof}
	Suppose that $u\in C^1(B_r)$ for $r>0$, and let $\rho:\R^n\to\R^n$ be the function which scales by $r$, that is $r(x)=rx$. Notice that $u\circ \rho \in C^1(B_1)$. Thus 
	\begin{align*}
	\sup_{x,y\in B_{r/2}}\abs{u(x)-u(y)}
	&=\sup_{x,y\in B_{1/2}}\abs{u(rx)-u(ry)} \\
	&=\sup_{x,y\in B_{1/2}}\abs{(u\circ\rho)(x)-(u\circ \rho)(y)} \\
	&\leq C(n,p)\left(\int_{B_1}\big|\nabla (u\circ\rho)(x)\big|^p \dx \right)^{1/p} &&\text{by Morrey's inequality} \\
	&\leq C(n,p)\left(\int_{B_1}r^p\big|(\nabla u)(rx)\big|^p \dx \right)^{1/p} &&\text{by chain rule} \\
	&= C(n,p)r\left(\int_{B_1}\big|(\nabla u)(rx)\big|^p \dx \right)^{1/p}\\
	&= C(n,p)r^{1-n}\left(\int_{B_1}\abs{\nabla u(x)}^p \dx \right)^{1/p} && \text{by change of variables} 
	\end{align*}
	Thus $C(p,n,r)=C(n,p)r^{1-n}$. 
	\end{proof}
	\end{enumerate}

\end{enumerate}



\end{document}
