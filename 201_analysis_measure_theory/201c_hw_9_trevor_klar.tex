\documentclass[12pt,letterpaper]{article}

\usepackage{fancyhdr,fancybox,tensor}

%% Useful packages
\usepackage{amssymb, amsmath, amsthm} 
%\usepackage{graphicx}  %%this is currently enabled in the default document, so it is commented out here. 
\usepackage{calrsfs}
\usepackage{braket}
\usepackage{mathtools}
\usepackage{lipsum}
\usepackage{tikz}
\usetikzlibrary{cd}
\usepackage{verbatim}
%\usepackage{ntheorem}% for theorem-like environments
\usepackage{mdframed}%can make highlighted boxes of text
%Use case: https://tex.stackexchange.com/questions/46828/how-to-highlight-important-parts-with-a-gray-background
\usepackage{wrapfig}
\usepackage{centernot}
\usepackage{subcaption}%\begin{subfigure}{0.5\textwidth}
\usepackage{pgfplots}
\pgfplotsset{compat=1.13}
\usepackage[colorinlistoftodos]{todonotes}
\usepackage[colorlinks=true, allcolors=blue]{hyperref}
\usepackage{xfrac}					%to make slanted fractions \sfrac{numerator}{denominator}
\usepackage{enumitem}            
    %syntax: \begin{enumerate}[label=(\alph*)]
    %possible arguments: f \alph*, \Alph*, \arabic*, \roman* and \Roman*
\usetikzlibrary{arrows,shapes.geometric,fit}

\DeclareMathAlphabet{\pazocal}{OMS}{zplm}{m}{n}
%% Use \pazocal{letter} to typeset a letter in the other kind 
%%  of math calligraphic font. 

%% This puts the QED block at the end of each proof, the way I like it. 
\renewenvironment{proof}{{\bfseries Proof}}{\qed}
\makeatletter
\renewenvironment{proof}[1][\bfseries \proofname]{\par
  \pushQED{\qed}%
  \normalfont \topsep6\p@\@plus6\p@\relax
  \trivlist
  %\itemindent\normalparindent
  \item[\hskip\labelsep
        \scshape
    #1\@addpunct{}]\ignorespaces
}{%
  \popQED\endtrivlist\@endpefalse
}
\makeatother

%% This adds a \rewnewtheorem command, which enables me to override the settings for theorems contained in this document.
\makeatletter
\def\renewtheorem#1{%
  \expandafter\let\csname#1\endcsname\relax
  \expandafter\let\csname c@#1\endcsname\relax
  \gdef\renewtheorem@envname{#1}
  \renewtheorem@secpar
}
\def\renewtheorem@secpar{\@ifnextchar[{\renewtheorem@numberedlike}{\renewtheorem@nonumberedlike}}
\def\renewtheorem@numberedlike[#1]#2{\newtheorem{\renewtheorem@envname}[#1]{#2}}
\def\renewtheorem@nonumberedlike#1{  
\def\renewtheorem@caption{#1}
\edef\renewtheorem@nowithin{\noexpand\newtheorem{\renewtheorem@envname}{\renewtheorem@caption}}
\renewtheorem@thirdpar
}
\def\renewtheorem@thirdpar{\@ifnextchar[{\renewtheorem@within}{\renewtheorem@nowithin}}
\def\renewtheorem@within[#1]{\renewtheorem@nowithin[#1]}
\makeatother

%% This makes theorems and definitions with names show up in bold, the way I like it. 
\makeatletter
\def\th@plain{%
  \thm@notefont{}% same as heading font
  \itshape % body font
}
\def\th@definition{%
  \thm@notefont{}% same as heading font
  \normalfont % body font
}
\makeatother

%===============================================
%==============Shortcut Commands================
%===============================================
\newcommand{\ds}{\displaystyle}
\newcommand{\B}{\mathcal{B}}
\newcommand{\C}{\mathbb{C}}
\newcommand{\F}{\mathbb{F}}
\newcommand{\N}{\mathbb{N}}
\newcommand{\R}{\mathbb{R}}
\newcommand{\Q}{\mathbb{Q}}
\newcommand{\T}{\mathcal{T}}
\newcommand{\Z}{\mathbb{Z}}
\renewcommand\qedsymbol{$\blacksquare$}
\newcommand{\qedwhite}{\hfill\ensuremath{\square}}
\newcommand*\conj[1]{\overline{#1}}
\newcommand*\closure[1]{\overline{#1}}
\newcommand*\mean[1]{\overline{#1}}
%\newcommand{\inner}[1]{\left< #1 \right>}
\newcommand{\inner}[2]{\left< #1, #2 \right>}
\newcommand{\powerset}[1]{\pazocal{P}(#1)}
%% Use \pazocal{letter} to typeset a letter in the other kind 
%%  of math calligraphic font. 
\newcommand{\cardinality}[1]{\left| #1 \right|}
\newcommand{\domain}[1]{\mathcal{D}(#1)}
\newcommand{\image}{\text{Im}}
\newcommand{\inv}[1]{#1^{-1}}
\newcommand{\preimage}[2]{#1^{-1}\left(#2\right)}
\newcommand{\script}[1]{\mathcal{#1}}


\newenvironment{highlight}{\begin{mdframed}[backgroundcolor=gray!20]}{\end{mdframed}}

\DeclarePairedDelimiter\ceil{\lceil}{\rceil}
\DeclarePairedDelimiter\floor{\lfloor}{\rfloor}

%===============================================
%===============My Tikz Commands================
%===============================================
\newcommand{\drawsquiggle}[1]{\draw[shift={(#1,0)}] (.005,.05) -- (-.005,.02) -- (.005,-.02) -- (-.005,-.05);}
\newcommand{\drawpoint}[2]{\draw[*-*] (#1,0.01) node[below, shift={(0,-.2)}] {#2};}
\newcommand{\drawopoint}[2]{\draw[o-o] (#1,0.01) node[below, shift={(0,-.2)}] {#2};}
\newcommand{\drawlpoint}[2]{\draw (#1,0.02) -- (#1,-0.02) node[below] {#2};}
\newcommand{\drawlbrack}[2]{\draw (#1+.01,0.02) --(#1,0.02) -- (#1,-0.02) -- (#1+.01,-0.02) node[below, shift={(-.01,0)}] {#2};}
\newcommand{\drawrbrack}[2]{\draw (#1-.01,0.02) --(#1,0.02) -- (#1,-0.02) -- (#1-.01,-0.02) node[below, shift={(+.01,0)}] {#2};}

%***********************************************
%**************Start of Document****************
%***********************************************
 %find me at /home/trevor/texmf/tex/latex/tskpreamble_nothms.tex
%===============================================
%===============Theorem Styles==================
%===============================================

%================Default Style==================
\theoremstyle{plain}% is the default. it sets the text in italic and adds extra space above and below the \newtheorems listed below it in the input. it is recommended for theorems, corollaries, lemmas, propositions, conjectures, criteria, and (possibly; depends on the subject area) algorithms.
\newtheorem{theorem}{Theorem}
\numberwithin{theorem}{section} %This sets the numbering system for theorems to number them down to the {argument} level. I have it set to number down to the {section} level right now.
\newtheorem*{theorem*}{Theorem} %Theorem with no numbering
\newtheorem{corollary}[theorem]{Corollary}
\newtheorem*{corollary*}{Corollary}
\newtheorem{conjecture}[theorem]{Conjecture}
\newtheorem{lemma}[theorem]{Lemma}
\newtheorem*{lemma*}{Lemma}
\newtheorem{proposition}[theorem]{Proposition}
\newtheorem*{proposition*}{Proposition}
\newtheorem{problemstatement}[theorem]{Problem Statement}


%==============Definition Style=================
\theoremstyle{definition}% adds extra space above and below, but sets the text in roman. it is recommended for definitions, conditions, problems, and examples; i've alse seen it used for exercises.
\newtheorem{definition}[theorem]{Definition}
\newtheorem*{definition*}{Definition}
\newtheorem{condition}[theorem]{Condition}
\newtheorem{problem}[theorem]{Problem}
\newtheorem{example}[theorem]{Example}
\newtheorem*{example*}{Example}
\newtheorem*{counterexample*}{Counterexample}
\newtheorem*{romantheorem*}{Theorem} %Theorem with no numbering
\newtheorem{exercise}{Exercise}
\numberwithin{exercise}{section}
\newtheorem{algorithm}[theorem]{Algorithm}

%================Remark Style===================
\theoremstyle{remark}% is set in roman, with no additional space above or below. it is recommended for remarks, notes, notation, claims, summaries, acknowledgments, cases, and conclusions.
\newtheorem{remark}[theorem]{Remark}
\newtheorem*{remark*}{Remark}
\newtheorem{notation}[theorem]{Notation}
\newtheorem*{notation*}{Notation}
%\newtheorem{claim}[theorem]{Claim}  %%use this if you ever want claims to be numbered
\newtheorem*{claim}{Claim}


%%
%% Page set-up:
%%
\pagestyle{empty}
\lhead{\textsc{201c - Functional Analysis} \\Quarter of COVID-19} 
\rhead{\textsc{Labutin, Spring 2020} \\ Trevor Klar}
%\chead{\Large\textbf{A New Integration Technique \\ }}
\renewcommand{\headrulewidth}{0pt}
%
\renewcommand{\footrulewidth}{0pt}
%\lfoot{
%Office: \quad \quad \, M 2-3 \, \, SH 6431x \\
%Math Lab: \, W 12-2 \, SH 1607
%}
%\rfoot{trevorklar@math.ucsb.edu}

\setlength{\parindent}{0in}
\setlength{\textwidth}{7in}
\setlength{\evensidemargin}{-0.25in}
\setlength{\oddsidemargin}{-0.25in}
\setlength{\parskip}{.5\baselineskip}
\setlength{\topmargin}{-0.5in}
\setlength{\textheight}{9in}

\setlist[enumerate,1]{label=\textbf{\arabic*.}}

\let\oldphi\phi
\renewcommand{\phi}{\varphi}
\renewcommand{\epsilon}{\varepsilon}

\begin{document}
\pagestyle{fancy}
\begin{center}
{\Large Homework 9}%=================UPDATE THIS=================%
\end{center}

\renewcommand{\B}{\bar{B}(\ell^\infty)}
\newcommand{\Rn}{\R^n}
\renewcommand{\phi}{\oldphi}
\renewcommand{\S}{\script{S}}
\newcommand{\D}{\script{D}}

\begin{enumerate}
\item Find \textit{all} distributional solutions $y$ to the following equations:
	\begin{enumerate}[label=(\alph*)]
	\item $xy=0$ in $\D'(\R)$ 
	
	[Hint: Represent any $\phi\in\D$ as $\phi(x)\equiv \phi(0)\eta(x)+x\psi(x)$ with $\eta$ independent of $\phi$. 
	
	\begin{proof}
	We can write 
	$$\phi=\phi_0\eta + x\psi,$$
	where $\eta\in\D$ is any test function with $\eta(0)=1$ and $\psi:=\frac{1}{x}(\phi-\phi_0\eta)$. Suppose $T$ is a solution, so $xT$ is the zero distribution. Then 
	\begin{align*}
	T\phi&=T(\phi_0\eta+x\psi) \\
	&= \phi_0T\eta + T(x\psi) & \text{by linearity} \\
	&= \phi_0T\eta + xT(\psi) & \text{by def. of mult. by $C^\infty$ f'ns} \\
	&= \phi_0T\eta & \text{by assumption} \\
	&= (T\eta)\delta_0(\phi)
	\end{align*}
	and since $T\eta$ is an arbitrary constant with respect to $\phi$, then any multiple of the Dirac delta $\delta_0$ is a solution to (a). 
	\end{proof}
	\item $y'=0$ in $\D'((a,b))$
	
	\begin{proof}
	Suppose $T$ is a solution, so the distributional derivative $DT=0$. Since 
	$$DT=0\in C^0(a,b),$$ 
	then by Theorem 6.10 
	$$T=f\in C^1(a,b)$$ 
	for some $f$, and $0=DT$ is the classical derivative $f'$. Since $f'=0$, then it is a constant, so $T=C$. Thus the solutions to (b) are constant distributions. 
	\end{proof}
	\end{enumerate}
	
\item Consider the function $f(x)=|x|^{-1}$ on $\R$. Although this function is not in $L^1_{\text{loc}}$, it is defined as a distribution for test functions on $\R$ that vanish at the origin, by 
$$T_f(\phi) = \int_\R |x|^{-1}\phi(x)\dx.$$
	\begin{enumerate}[label=(\alph*)]
	\item Show that there is a distribution $T\in\D'(\R)$ that agrees with $T_f$ for functions that vanish at the origin. Give an explicit formula for one such $T$. 
	\begin{proof}
	Note that for all $\phi\in\D$, $(\phi-\phi_0)\in\D$ and vanishes at zero, so $T_f(\phi-\phi_0)$ is a distribution. If $\phi$ vanishes at 0, then 
	$$T_f(\phi-\phi_0) = T_f(\phi-0) = T_f(\phi).$$
	\end{proof}
	\item Characterize all such $T$'s. Theorem 6.14 may be helpful here. 
	\end{enumerate}

\item Compute the following limit in $\D'(\R)$. 
$$\lim_{\epsilon\to 0}\frac{\epsilon}{\pi x^2}\sin^2 \left(\frac{x}{\epsilon}\right)$$

\answer
The answer is the Dirac delta $\delta_0$. To see this, observe that $\int_\R\frac{1}{\pi x^2}\sin^2 (x) \dx = 1$, so
\begin{align*}
&\lim_{\epsilon\to 0}\left(\int_\R\frac{\epsilon}{\pi x^2}\sin^2 \left(\tfrac{x}{\epsilon}\right)\phi(x)\dx     -    \phi(0)\right) \\
= &\lim_{\epsilon\to 0}\left(\int_\R\frac{1}{\pi u^2}\sin^2 (u)\phi(\epsilon u)\der u      -      \phi(0)\right) \\
= &\lim_{\epsilon\to 0}\left(\int_\R\frac{1}{\pi u^2}\sin^2 (u)\phi(\epsilon u)\der u      -      \phi(0)\int_\R\frac{1}{\pi u^2}\sin^2 (u)\der u\right) \\
= &\lim_{\epsilon\to 0}\int_\R\frac{1}{\pi u^2}\sin^2 (u) \bigg(\phi(\epsilon u) - \phi(0)\bigg)\der u  \\
\end{align*}
and note that $\phi\in\D$ and in particular, $\phi$ is continuous with compact support, so $\norm{\phi(\epsilon u) - \phi(0)}_\infty\leq C< \infty$, so the sequence of integrands above is dominated by 
$\frac{C}{\pi u^2}\sin^2 (u),$
so by the dominated convergence theorem we can move the limit inside the integral and find that 
\begin{align*}
&\lim_{\epsilon\to 0}\int_\R\frac{1}{\pi u^2}\sin^2 (u) \bigg(\phi(\epsilon u) - \phi(0)\bigg)\der u  \\
=&\int_\R\frac{1}{\pi u^2}\sin^2 (u) \bigg(\phi(0) - \phi(0)\bigg)\der u  \\
=&0.
\end{align*}
Therefore the action of the limit is exactly that of $\delta_0$. 
\qed

\end{enumerate}



\end{document}
