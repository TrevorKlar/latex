 \documentclass[12pt,letterpaper]{article}

\usepackage{fancyhdr,fancybox,tensor}

%% Useful packages
\usepackage{amssymb, amsmath, amsthm} 
%\usepackage{graphicx}  %%this is currently enabled in the default document, so it is commented out here. 
\usepackage{calrsfs}
\usepackage{braket}
\usepackage{mathtools}
\usepackage{lipsum}
\usepackage{tikz}
\usetikzlibrary{cd}
\usepackage{verbatim}
%\usepackage{ntheorem}% for theorem-like environments
\usepackage{mdframed}%can make highlighted boxes of text
%Use case: https://tex.stackexchange.com/questions/46828/how-to-highlight-important-parts-with-a-gray-background
\usepackage{wrapfig}
\usepackage{centernot}
\usepackage{subcaption}%\begin{subfigure}{0.5\textwidth}
\usepackage{pgfplots}
\pgfplotsset{compat=1.13}
\usepackage[colorinlistoftodos]{todonotes}
\usepackage[colorlinks=true, allcolors=blue]{hyperref}
\usepackage{xfrac}					%to make slanted fractions \sfrac{numerator}{denominator}
\usepackage{enumitem}            
    %syntax: \begin{enumerate}[label=(\alph*)]
    %possible arguments: f \alph*, \Alph*, \arabic*, \roman* and \Roman*
\usetikzlibrary{arrows,shapes.geometric,fit}

\DeclareMathAlphabet{\pazocal}{OMS}{zplm}{m}{n}
%% Use \pazocal{letter} to typeset a letter in the other kind 
%%  of math calligraphic font. 

%% This puts the QED block at the end of each proof, the way I like it. 
\renewenvironment{proof}{{\bfseries Proof}}{\qed}
\makeatletter
\renewenvironment{proof}[1][\bfseries \proofname]{\par
  \pushQED{\qed}%
  \normalfont \topsep6\p@\@plus6\p@\relax
  \trivlist
  %\itemindent\normalparindent
  \item[\hskip\labelsep
        \scshape
    #1\@addpunct{}]\ignorespaces
}{%
  \popQED\endtrivlist\@endpefalse
}
\makeatother

%% This adds a \rewnewtheorem command, which enables me to override the settings for theorems contained in this document.
\makeatletter
\def\renewtheorem#1{%
  \expandafter\let\csname#1\endcsname\relax
  \expandafter\let\csname c@#1\endcsname\relax
  \gdef\renewtheorem@envname{#1}
  \renewtheorem@secpar
}
\def\renewtheorem@secpar{\@ifnextchar[{\renewtheorem@numberedlike}{\renewtheorem@nonumberedlike}}
\def\renewtheorem@numberedlike[#1]#2{\newtheorem{\renewtheorem@envname}[#1]{#2}}
\def\renewtheorem@nonumberedlike#1{  
\def\renewtheorem@caption{#1}
\edef\renewtheorem@nowithin{\noexpand\newtheorem{\renewtheorem@envname}{\renewtheorem@caption}}
\renewtheorem@thirdpar
}
\def\renewtheorem@thirdpar{\@ifnextchar[{\renewtheorem@within}{\renewtheorem@nowithin}}
\def\renewtheorem@within[#1]{\renewtheorem@nowithin[#1]}
\makeatother

%% This makes theorems and definitions with names show up in bold, the way I like it. 
\makeatletter
\def\th@plain{%
  \thm@notefont{}% same as heading font
  \itshape % body font
}
\def\th@definition{%
  \thm@notefont{}% same as heading font
  \normalfont % body font
}
\makeatother

%===============================================
%==============Shortcut Commands================
%===============================================
\newcommand{\ds}{\displaystyle}
\newcommand{\B}{\mathcal{B}}
\newcommand{\C}{\mathbb{C}}
\newcommand{\F}{\mathbb{F}}
\newcommand{\N}{\mathbb{N}}
\newcommand{\R}{\mathbb{R}}
\newcommand{\Q}{\mathbb{Q}}
\newcommand{\T}{\mathcal{T}}
\newcommand{\Z}{\mathbb{Z}}
\renewcommand\qedsymbol{$\blacksquare$}
\newcommand{\qedwhite}{\hfill\ensuremath{\square}}
\newcommand*\conj[1]{\overline{#1}}
\newcommand*\closure[1]{\overline{#1}}
\newcommand*\mean[1]{\overline{#1}}
%\newcommand{\inner}[1]{\left< #1 \right>}
\newcommand{\inner}[2]{\left< #1, #2 \right>}
\newcommand{\powerset}[1]{\pazocal{P}(#1)}
%% Use \pazocal{letter} to typeset a letter in the other kind 
%%  of math calligraphic font. 
\newcommand{\cardinality}[1]{\left| #1 \right|}
\newcommand{\domain}[1]{\mathcal{D}(#1)}
\newcommand{\image}{\text{Im}}
\newcommand{\inv}[1]{#1^{-1}}
\newcommand{\preimage}[2]{#1^{-1}\left(#2\right)}
\newcommand{\script}[1]{\mathcal{#1}}


\newenvironment{highlight}{\begin{mdframed}[backgroundcolor=gray!20]}{\end{mdframed}}

\DeclarePairedDelimiter\ceil{\lceil}{\rceil}
\DeclarePairedDelimiter\floor{\lfloor}{\rfloor}

%===============================================
%===============My Tikz Commands================
%===============================================
\newcommand{\drawsquiggle}[1]{\draw[shift={(#1,0)}] (.005,.05) -- (-.005,.02) -- (.005,-.02) -- (-.005,-.05);}
\newcommand{\drawpoint}[2]{\draw[*-*] (#1,0.01) node[below, shift={(0,-.2)}] {#2};}
\newcommand{\drawopoint}[2]{\draw[o-o] (#1,0.01) node[below, shift={(0,-.2)}] {#2};}
\newcommand{\drawlpoint}[2]{\draw (#1,0.02) -- (#1,-0.02) node[below] {#2};}
\newcommand{\drawlbrack}[2]{\draw (#1+.01,0.02) --(#1,0.02) -- (#1,-0.02) -- (#1+.01,-0.02) node[below, shift={(-.01,0)}] {#2};}
\newcommand{\drawrbrack}[2]{\draw (#1-.01,0.02) --(#1,0.02) -- (#1,-0.02) -- (#1-.01,-0.02) node[below, shift={(+.01,0)}] {#2};}

%***********************************************
%**************Start of Document****************
%***********************************************
 %find me at /home/trevor/texmf/tex/latex/tskpreamble_nothms.tex
%===============================================
%===============Theorem Styles==================
%===============================================

%================Default Style==================
\theoremstyle{plain}% is the default. it sets the text in italic and adds extra space above and below the \newtheorems listed below it in the input. it is recommended for theorems, corollaries, lemmas, propositions, conjectures, criteria, and (possibly; depends on the subject area) algorithms.
\newtheorem{theorem}{Theorem}
\numberwithin{theorem}{section} %This sets the numbering system for theorems to number them down to the {argument} level. I have it set to number down to the {section} level right now.
\newtheorem*{theorem*}{Theorem} %Theorem with no numbering
\newtheorem{corollary}[theorem]{Corollary}
\newtheorem*{corollary*}{Corollary}
\newtheorem{conjecture}[theorem]{Conjecture}
\newtheorem{lemma}[theorem]{Lemma}
\newtheorem*{lemma*}{Lemma}
\newtheorem{proposition}[theorem]{Proposition}
\newtheorem*{proposition*}{Proposition}
\newtheorem{problemstatement}[theorem]{Problem Statement}


%==============Definition Style=================
\theoremstyle{definition}% adds extra space above and below, but sets the text in roman. it is recommended for definitions, conditions, problems, and examples; i've alse seen it used for exercises.
\newtheorem{definition}[theorem]{Definition}
\newtheorem*{definition*}{Definition}
\newtheorem{condition}[theorem]{Condition}
\newtheorem{problem}[theorem]{Problem}
\newtheorem{example}[theorem]{Example}
\newtheorem*{example*}{Example}
\newtheorem*{counterexample*}{Counterexample}
\newtheorem*{romantheorem*}{Theorem} %Theorem with no numbering
\newtheorem{exercise}{Exercise}
\numberwithin{exercise}{section}
\newtheorem{algorithm}[theorem]{Algorithm}

%================Remark Style===================
\theoremstyle{remark}% is set in roman, with no additional space above or below. it is recommended for remarks, notes, notation, claims, summaries, acknowledgments, cases, and conclusions.
\newtheorem{remark}[theorem]{Remark}
\newtheorem*{remark*}{Remark}
\newtheorem{notation}[theorem]{Notation}
\newtheorem*{notation*}{Notation}
%\newtheorem{claim}[theorem]{Claim}  %%use this if you ever want claims to be numbered
\newtheorem*{claim}{Claim}


%%
%% Page set-up:
%%
\pagestyle{empty}
\lhead{\textsc{201c - Functional Analysis} \\Quarter of COVID-19} 
\rhead{\textsc{Labutin, Spring 2020} \\ Trevor Klar}
%\chead{\Large\textbf{A New Integration Technique \\ }}
\renewcommand{\headrulewidth}{0pt}
%
\renewcommand{\footrulewidth}{0pt}
%\lfoot{
%Office: \quad \quad \, M 2-3 \, \, SH 6431x \\
%Math Lab: \, W 12-2 \, SH 1607
%}
%\rfoot{trevorklar@math.ucsb.edu}

\setlength{\parindent}{0in}
\setlength{\textwidth}{7in}
\setlength{\evensidemargin}{-0.25in}
\setlength{\oddsidemargin}{-0.25in}
\setlength{\parskip}{.5\baselineskip}
\setlength{\topmargin}{-0.5in}
\setlength{\textheight}{9in}

\setlist[enumerate,1]{label=\textbf{\arabic*.}}

\let\oldphi\phi
\renewcommand{\phi}{\varphi}
\renewcommand{\epsilon}{\varepsilon}

\begin{document}
\pagestyle{fancy}
\begin{center}
{\Large Homework 3}%=================UPDATE THIS=================%
\end{center}

\renewcommand{\B}{\bar{B}(\ell^\infty)}
\begin{enumerate}
%1
\item %show that $x_n\xto{d}0 \iff \x_n\xto{w^*}0$. 
Think of $\ell^1$ as a linear space and $\ell^{\infty}$ as its dual. Let $\B$ be the closed unit ball with respect to the metric $\norm{f-g}_{\ell^\infty}$. For every $f,g\in \B$ define another metric 
$$d(f,g)=\sum_{n=1}^\infty 2^{-n}\abs{f_n-g_n}.$$
Prove that $\sigma(\B,\ell^1)$\footnote{the weak* topology of $\ell^\infty$ restricted to $\B$} coincides with the topology of the metric $d$. 

\newcommand{\dballzero}[1]{\tensor[_d]{B}{_{#1}}(0)}
\newcommand{\dballf}[1]{\tensor[_d]{B}{_{#1}}(f)}
\begin{proof}
Since the two topologies are both translation invariant, it suffices to show that $W(0;p)$ is open in the $d$-topology and that the $d$-ball $\dballzero{r}$ is open in $\sigma(\B,\ell^1)$. 

\textsc{Part I}: We show that $\sigma(\B,\ell^1)\subset\script{T}_d(\B)$. 
	Let $W(0,p)$ be an arbitrary subbasic weak* neighborhood in $\sigma(\B,\ell^1)$ centered at 0. 
	Fix $f\in W(0,p)$. This means that 
		\begin{align*}
		\sup_n |f_n|&\leq 1 \\
		\abs{\sum_n f_np_n} &<1
		\end{align*}				
	Since $p\in\ell^1$, then $\sum_n|p_n|$ converges, so there exists $N$ such that 
	\begin{equation}
	\sum_{n=N}^\infty |p_n| < \frac{1}{4}.
	\end{equation}
	
	Let $r>0$ such that for all $n\leq N$,
	\begin{equation}
	|p_n|\leq\frac{2^{-n}}{4r},
	\end{equation}
	and consider $\dballf{r}$. 
		
		\textsc{Claim:} $\dballf{r}\subset W(0,p)$. %since $\forall\, g\in \dballf{r},$		 \quad 		$g\in W(0,p)\cap \B$.
		
		\textsc{Proof} Let $g\in \dballf{r}$. Then 
		\begin{align}
		\sum_n 2^{-n}|f_n-g_n|&< r, \text{ and} \\
		\sup_n|f_n-g_n|&=2.
		\end{align}		
		So 
		\begin{align*}
		\abs{\angles{(f-g),p}} &= \abs{\sum_{n=1}^\infty (f_n-g_n)(p_n)} \\
		&\leq \sum_{n=1}^\infty |f_n-g_n||p_n| \\
		&= \sum_{1}^N |f_n-g_n||p_n| + \sum_{N}^\infty |f_n-g_n||p_n|\\
		&\leq \sum_{1}^N \frac{2^{-n}}{4r}|f_n-g_n| + \sum_{N}^\infty 2|p_n| &\text{applying (2) and (4)}\\
		&< \frac{1}{2}+\frac{1}{2} &\text{applying (3) and (1)} \\
		&=1
		\end{align*}
Thus every $f\in W(0,p)$ has a $d$-ball containing $f$ which is a subset of $W(0,p)$, so Part I is proved. \qedwhite

\textsc{Part II:} First note that 
$$\norm{f}_d\leq\norm{f}_{\ell^\infty}$$
since, if $f_n$ is an absolutely decreasing sequence, then
$$\sum_n 2^{-n}|f_n|\leq\sup_n|f_n| \quad\text{ (with equality if $f_n$ is constant)},$$
and swapping any coordinates of $f_n$ will cause $\norm{f}_d$ to decrease while $\norm{f}_{\ell^\infty}$ remains constant.

\newcommand{\dball}{\tensor[_d]{B}{_{r}}(f)}
\newcommand{\lball}{\tensor[_{\ell^\infty}]{B}{_{r}}(f)}
Thus the balls $\lball\subset\dball$ whenever they have the same radius and center. This means that for any ball $\dball$ with $g\in\dball$, there is of course some 
$$\tensor[_d]{B}{_{r'}}(g)\subset\dball,$$ 
and 
$$\tensor[_{\ell^\infty}]{B}{_{r'}}(g)\subset \tensor[_d]{B}{_{r'}}(g)$$
so we're done. 
\end{proof}

\pagebreak
\renewcommand{\phi}{\oldphi}
%2
\item Let $u_n\xto{w}u$ in a Banach space $X$, and let $\phi_n\xto{w*}\phi$ in $X^*$. Give an example in $X=\ell^2$ to show that $\angles{\phi_n,u_n}$ need not be convergent. 

Prove that if either  $u_n\xto{}u$ or $\phi_n\xto{}\phi$ strongly, then $\angles{\phi_n,u_n}\to\angles{\phi,u}$.

\begin{example*}
Recall that $\ell^{2*}=\ell^2$. Let $u_j=(-1)^j e_j$ and let $\phi_n=e_n$. Then $u_j\xto{w}0$ and $\phi_n\xto{w*}0$, but $\angles{\phi_k,u_k}$ alternates between 1 and $-1$, and doesn't converge.
\qedwhite

\begin{remark*}
Wait, why doesn't it work that $\angles{\phi_n,u_j}\xto{j}\angles{\phi_n,u}\xto{n}\angles{\phi,u}$; using weak convergence followed by weak* convergence?

It does. However $\angles{\phi_k,u_k}\xto{k}\angles{\phi,u}$ requires that we can get $\angles{\phi_k,u_k}$ arbitrarily close to $\angles{\phi,u}$ without taking \textit{either one }of the limits.
\end{remark*}
\end{example*}
\begin{proof} \textsc{Case I:}
Suppose $\norm{x_n}\to\norm{x}$. Then there exists $N>0$ such that $n>N \implies \norm{x_n-x}<\epsilon$ for all $\epsilon>0$. Since $\phi_j\xto{w*}\phi$, then by the Uniform Boundedness Principle $\norm{\phi_j}\leq C$. Thus
\begin{align*}
\abs{(\phi_n-\phi)(x_n-x)}&\leq \phi_n(x_n-x) + \phi(x_n-x)\\
&\leq C\epsilon + C\epsilon
=2C\epsilon,
\end{align*}
and after rescaling, we're done. \qedwhite

\textsc{Case II:}
If on the other hand $\norm{\phi_n\to\phi}$, then $\hat{x}_n\xto{w**}\hat{x}$, and we can use the same proof as above. 
\end{proof}

\item Let $\Omega\subset\R^n$ such that $\abs{\Omega}<\infty$, and let $(f_n)$ be a sequence in $L^p(\Omega)$. Suppose $f_n\to 0$ \muae{} in $\Omega$, and $f_n\xto{w}0$.\footnote{$\abs{\omega}$ denotes the Lebesgue measure of $\Omega$, and $p < 1$.}

Prove that for $p=2$ and $q=1$\footnote{This actually holds for any $1\leq q <p$.} one has $\norm{f_n}_{L^q(\Omega)}\to0$.

\begin{proof}
%Hint: use Egorov’s theorem about the uniform convergence away of
%a small set. To handle the small set use Holder inequality.
Let $\epsilon>0$. Since $f_n\xto{w}0$, then by the Uniform Boundedness Principle $\norm{f_n}_2<B$. By Egoroff's Theorem, $\exists\, N>0$ such that $\forall\, n>N\; \int_U\abs{f_n}<\epsilon$ where $\mu(U^\complement)<\epsilon$. So $\forall\, n>N$, 
\begin{align*}
\int_\Omega |f_n|&=\int_U|f_n| + \int_{U^\complement}|f_n|\\
&\leq \epsilon + \int_{U^\complement}|f_n (1)|\\
&\leq \epsilon + \norm{f_n}_2\left(\mu\big(U^\complement\big)\right)^{\sfrac{1}{2}}\\
&\leq \epsilon + B\sqrt{\epsilon}
\end{align*}
and after rescaling, we're done. 
\end{proof}

%\item %don't do number 4
\pagebreak
\renewcommand{\phi}{\varphi}
\setcounter{enumi}{4}
\item Let $X$ be reflexive. Prove that any closed subspace $W$ of $X$ is also reflexive. 

\renewcommand{\B}{\bar{B}}
\begin{proof}
We know that $X$ is reflexive iff $\B(X)$ is weakly compact, and in the subspace topology $\B(W)$ is exactly $W\cap \B(X)$, so it suffices to show that $W\cap \B(X)$ is weakly compact. %Let $(x_n)$ be a sequence in $W\cap \B(X)$. 
Since $W$ is closed and convex, then by Hahn-Banach there exists a functional $\phi$ separating $p\in W^\complement$ from $W$, so $W(p,\phi)\subset W^\complement$ and $W$ is weakly closed. Thus $\B(W)$ is a weak closed subset of the weak compact set $\B(X)$, so $\B(W)$ is weak compact. Therefore $W$ is reflexive. 
\end{proof}


\end{enumerate}
\end{document}
