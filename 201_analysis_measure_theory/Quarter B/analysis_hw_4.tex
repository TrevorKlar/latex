\documentclass[12pt,letterpaper]{article}

\usepackage{fancyhdr,fancybox}

%% Useful packages
\usepackage{amssymb, amsmath, amsthm} 
%\usepackage{graphicx}  %%this is currently enabled in the default document, so it is commented out here. 
\usepackage{calrsfs}
\usepackage{braket}
\usepackage{mathtools}
\usepackage{lipsum}
\usepackage{tikz}
\usetikzlibrary{cd}
\usepackage{verbatim}
%\usepackage{ntheorem}% for theorem-like environments
\usepackage{mdframed}%can make highlighted boxes of text
%Use case: https://tex.stackexchange.com/questions/46828/how-to-highlight-important-parts-with-a-gray-background
\usepackage{wrapfig}
\usepackage{centernot}
\usepackage{subcaption}%\begin{subfigure}{0.5\textwidth}
\usepackage{pgfplots}
\pgfplotsset{compat=1.13}
\usepackage[colorinlistoftodos]{todonotes}
\usepackage[colorlinks=true, allcolors=blue]{hyperref}
\usepackage{xfrac}					%to make slanted fractions \sfrac{numerator}{denominator}
\usepackage{enumitem}            
    %syntax: \begin{enumerate}[label=(\alph*)]
    %possible arguments: f \alph*, \Alph*, \arabic*, \roman* and \Roman*
\usetikzlibrary{arrows,shapes.geometric,fit}

\DeclareMathAlphabet{\pazocal}{OMS}{zplm}{m}{n}
%% Use \pazocal{letter} to typeset a letter in the other kind 
%%  of math calligraphic font. 

%% This puts the QED block at the end of each proof, the way I like it. 
\renewenvironment{proof}{{\bfseries Proof}}{\qed}
\makeatletter
\renewenvironment{proof}[1][\bfseries \proofname]{\par
  \pushQED{\qed}%
  \normalfont \topsep6\p@\@plus6\p@\relax
  \trivlist
  %\itemindent\normalparindent
  \item[\hskip\labelsep
        \scshape
    #1\@addpunct{}]\ignorespaces
}{%
  \popQED\endtrivlist\@endpefalse
}
\makeatother

%% This adds a \rewnewtheorem command, which enables me to override the settings for theorems contained in this document.
\makeatletter
\def\renewtheorem#1{%
  \expandafter\let\csname#1\endcsname\relax
  \expandafter\let\csname c@#1\endcsname\relax
  \gdef\renewtheorem@envname{#1}
  \renewtheorem@secpar
}
\def\renewtheorem@secpar{\@ifnextchar[{\renewtheorem@numberedlike}{\renewtheorem@nonumberedlike}}
\def\renewtheorem@numberedlike[#1]#2{\newtheorem{\renewtheorem@envname}[#1]{#2}}
\def\renewtheorem@nonumberedlike#1{  
\def\renewtheorem@caption{#1}
\edef\renewtheorem@nowithin{\noexpand\newtheorem{\renewtheorem@envname}{\renewtheorem@caption}}
\renewtheorem@thirdpar
}
\def\renewtheorem@thirdpar{\@ifnextchar[{\renewtheorem@within}{\renewtheorem@nowithin}}
\def\renewtheorem@within[#1]{\renewtheorem@nowithin[#1]}
\makeatother

%% This makes theorems and definitions with names show up in bold, the way I like it. 
\makeatletter
\def\th@plain{%
  \thm@notefont{}% same as heading font
  \itshape % body font
}
\def\th@definition{%
  \thm@notefont{}% same as heading font
  \normalfont % body font
}
\makeatother

%===============================================
%==============Shortcut Commands================
%===============================================
\newcommand{\ds}{\displaystyle}
\newcommand{\B}{\mathcal{B}}
\newcommand{\C}{\mathbb{C}}
\newcommand{\F}{\mathbb{F}}
\newcommand{\N}{\mathbb{N}}
\newcommand{\R}{\mathbb{R}}
\newcommand{\Q}{\mathbb{Q}}
\newcommand{\T}{\mathcal{T}}
\newcommand{\Z}{\mathbb{Z}}
\renewcommand\qedsymbol{$\blacksquare$}
\newcommand{\qedwhite}{\hfill\ensuremath{\square}}
\newcommand*\conj[1]{\overline{#1}}
\newcommand*\closure[1]{\overline{#1}}
\newcommand*\mean[1]{\overline{#1}}
%\newcommand{\inner}[1]{\left< #1 \right>}
\newcommand{\inner}[2]{\left< #1, #2 \right>}
\newcommand{\powerset}[1]{\pazocal{P}(#1)}
%% Use \pazocal{letter} to typeset a letter in the other kind 
%%  of math calligraphic font. 
\newcommand{\cardinality}[1]{\left| #1 \right|}
\newcommand{\domain}[1]{\mathcal{D}(#1)}
\newcommand{\image}{\text{Im}}
\newcommand{\inv}[1]{#1^{-1}}
\newcommand{\preimage}[2]{#1^{-1}\left(#2\right)}
\newcommand{\script}[1]{\mathcal{#1}}


\newenvironment{highlight}{\begin{mdframed}[backgroundcolor=gray!20]}{\end{mdframed}}

\DeclarePairedDelimiter\ceil{\lceil}{\rceil}
\DeclarePairedDelimiter\floor{\lfloor}{\rfloor}

%===============================================
%===============My Tikz Commands================
%===============================================
\newcommand{\drawsquiggle}[1]{\draw[shift={(#1,0)}] (.005,.05) -- (-.005,.02) -- (.005,-.02) -- (-.005,-.05);}
\newcommand{\drawpoint}[2]{\draw[*-*] (#1,0.01) node[below, shift={(0,-.2)}] {#2};}
\newcommand{\drawopoint}[2]{\draw[o-o] (#1,0.01) node[below, shift={(0,-.2)}] {#2};}
\newcommand{\drawlpoint}[2]{\draw (#1,0.02) -- (#1,-0.02) node[below] {#2};}
\newcommand{\drawlbrack}[2]{\draw (#1+.01,0.02) --(#1,0.02) -- (#1,-0.02) -- (#1+.01,-0.02) node[below, shift={(-.01,0)}] {#2};}
\newcommand{\drawrbrack}[2]{\draw (#1-.01,0.02) --(#1,0.02) -- (#1,-0.02) -- (#1-.01,-0.02) node[below, shift={(+.01,0)}] {#2};}

%***********************************************
%**************Start of Document****************
%***********************************************
 %find me at /home/trevor/texmf/tex/latex/tskpreamble_nothms.tex
%===============================================
%===============Theorem Styles==================
%===============================================

%================Default Style==================
\theoremstyle{plain}% is the default. it sets the text in italic and adds extra space above and below the \newtheorems listed below it in the input. it is recommended for theorems, corollaries, lemmas, propositions, conjectures, criteria, and (possibly; depends on the subject area) algorithms.
\newtheorem{theorem}{Theorem}
\numberwithin{theorem}{section} %This sets the numbering system for theorems to number them down to the {argument} level. I have it set to number down to the {section} level right now.
\newtheorem*{theorem*}{Theorem} %Theorem with no numbering
\newtheorem{corollary}[theorem]{Corollary}
\newtheorem*{corollary*}{Corollary}
\newtheorem{conjecture}[theorem]{Conjecture}
\newtheorem{lemma}[theorem]{Lemma}
\newtheorem*{lemma*}{Lemma}
\newtheorem{proposition}[theorem]{Proposition}
\newtheorem*{proposition*}{Proposition}
\newtheorem{problemstatement}[theorem]{Problem Statement}


%==============Definition Style=================
\theoremstyle{definition}% adds extra space above and below, but sets the text in roman. it is recommended for definitions, conditions, problems, and examples; i've alse seen it used for exercises.
\newtheorem{definition}[theorem]{Definition}
\newtheorem*{definition*}{Definition}
\newtheorem{condition}[theorem]{Condition}
\newtheorem{problem}[theorem]{Problem}
\newtheorem{example}[theorem]{Example}
\newtheorem*{example*}{Example}
\newtheorem*{counterexample*}{Counterexample}
\newtheorem*{romantheorem*}{Theorem} %Theorem with no numbering
\newtheorem{exercise}{Exercise}
\numberwithin{exercise}{section}
\newtheorem{algorithm}[theorem]{Algorithm}

%================Remark Style===================
\theoremstyle{remark}% is set in roman, with no additional space above or below. it is recommended for remarks, notes, notation, claims, summaries, acknowledgments, cases, and conclusions.
\newtheorem{remark}[theorem]{Remark}
\newtheorem*{remark*}{Remark}
\newtheorem{notation}[theorem]{Notation}
\newtheorem*{notation*}{Notation}
%\newtheorem{claim}[theorem]{Claim}  %%use this if you ever want claims to be numbered
\newtheorem*{claim}{Claim}


%%
%% Page set-up:
%%
\pagestyle{empty}
\lhead{\textsc{201 - Real Analysis \\}} 
\rhead{\textsc{Harutyunyan, Winter 2020} \\ Trevor Klar}
%\chead{\Large\textbf{A New Integration Technique \\ }}
\renewcommand{\headrulewidth}{0pt}
%
\renewcommand{\footrulewidth}{0pt}
%\lfoot{
%Office: \quad \quad \, M 2-3 \, \, SH 6431x \\
%Math Lab: \, W 12-2 \, SH 1607
%}
%\rfoot{trevorklar@math.ucsb.edu}

\setlength{\parindent}{0in}
\setlength{\textwidth}{7in}
\setlength{\evensidemargin}{-0.25in}
\setlength{\oddsidemargin}{-0.25in}
\setlength{\parskip}{.5\baselineskip}
\setlength{\topmargin}{-0.5in}
\setlength{\textheight}{9in}

\setlist[enumerate,1]{label=\textbf{\arabic*.}}

\renewcommand{\phi}{\varphi}
\renewcommand{\epsilon}{\varepsilon}

\begin{document}
\pagestyle{fancy}
\begin{center}
{\Large Homework 4}%=================UPDATE THIS=================%
\end{center}

\jpg{width=\linewidth}{hw4-p1-1}

\answer Both are true. 

\begin{proof}
If $\{B_n\}$ is an increasing sequence, then $B_1\subset \cap_{n=1}^{\infty} B_n$, so parts 1 and 2 are shown. 

Otherwise, suppose $\{B_n\}$ is a decreasing sequence. Then for any fixed $N$, whenever $y\in B_N$ we have that $y\in B_n$ for all $n\leq N$. Let $A$ denote the set of all limit points of the set $\{x_n\}_{n=1}^\infty$. Then for all $y\in A$, there exists infinitely many $x_n$ such that $\norm{x_n - y}<r$, which means $y\in B_n$. Thus $A\subset \cap_{n=1}^{\infty} B_n$. 

\textbf{Claim.} $A\neq\emptyset$. 

\textsc{Proof of claim.} Let $m>n$. Since $B_m\subseteq B_n$, then $r_m\leq r_n$ and $x_m\in B_n$. Thus 
\[\norm{x_n-x_m}\leq r_n-r_m.\]
Letting $n\to\infty$, $\{r_n\}$ is a decreasing sequence bounded below by $r$, so it converges, and thus it is Cauchy. So for any $\epsilon>0$, there exists $M>0$ such that if $n,m>M$, then $\norm{x_n-x_m}\leq r_n-r_m<\epsilon$, so $\{x_n\}$ is Cauchy as well. Therefore $\{x_n\}$ converges with $x_n \to x$, and $x\in A$. 

\textbf{Claim.} $\closure{B}_r(x)\subset \cap_{n=1}^\infty B_n$. 

\textsc{Proof of claim.} We know that $\{r_n\}$ converges, so \Wlog{} suppose $r_n\to r$. We will show that for all $y\in \closure{B}_r(x)$, there exists $M\in \N$ such that for all $n>M$, we have that $y\in B_n$. This proves the claim, since we will have shown that $\closure{B}_r(x)\subset B_m $ for all $m>M,$ and $\closure{B}_r(x)\subset B_m$ implies $\closure{B}_r(x)\subset B_n$ for all $n<m$. Observe:
\begin{align*}
y\in \closure{B}_r(x) &\implies \norm{x-y}\leq r , \text{ so}\\
\\
\norm{x_n-y}&\leq \norm{x_n-x} + \norm{x-y} \\
&\leq r_n-r+r \\
&=r_n
\end{align*}
\end{proof}

\pagebreak
\jpg{width=\linewidth}{hw4-p2-1}

\begin{definition*}
A set $A$ is \emph{Baire first category} if it is a countable union of nowhere dense sets.
\end{definition*}

\answer For $k\geq 2$, let $C_k$ denote a fat Cantor set where at the $n$th stage (counting from 0) of the construction of $C_k$, intervals of length $2^{-k(n+1)}$ are removed. For example, 
\begin{center}
\begin{tabular}{lcccl}
$C_2$ removes intervals of length &$\frac{1}{4}$, &$\frac{1}{8},$ &$\frac{1}{16}$ &$\ldots$ \\
$C_3$ removes intervals of length &$\frac{1}{8},$ &$\frac{1}{16}$, & $\frac{1}{32}$ &$\ldots$ \\
$C_4$ removes intervals of length &$\frac{1}{16}$, & $\frac{1}{32}$, &$\frac{1}{64}$ &$\ldots$ \\
etc.
\end{tabular}
\end{center}
Thus the total length of removed intervals in the construction of $C_k$ (we will denote this $S_k$) is 
\[\sum_{n=0}^\infty \frac{2^{n}}{2^{2n+k}}
=2^{-k+1}.
\]
Since $\lim_{k\to\infty} S_k = 0$, then $\lim_{k\to\infty}\mu(C_k)=1$. Let $A=\bigcup_{k=1}^\infty C_k$. Now $\mu(A)=1$, and $A$ is Borel by construction, so it is Lebesgue measurable. Thus it remains to be shown that each $C_k$ is nowhere dense, so $A$ is Baire first category. 

Denote the $n$th stage of construction of $C_k$ by $C_{k,n}$. If we consider $C_{k,-1}$ to be the interval $[0,1]$ with nothing removed, then for any open interval $I$, we know that $I\cap C_{k,-1}$ is a non-degenerate interval $U$ whenever it is nonempty.

During the rest of the stages, a central interval in $[0,1]$ will be removed, and central intervals from the two remaining intervals will be removed, and so on. At stage $n$, $[0,1]$ will be divided into $2^{n+1}$ intervals, and as $n\to \infty$, each of the intervals in $C_{k,n}$ has length less than $2^{-(n+1)}$ which goes to $0$. Thus since $U$ has nonzero length, some part of $U$ must be removed at some stage. Since we are removing intervals, the part removed from $U$ must also be an interval, and so must have positive length, which means $\closure{U}$ does not contain it. Thus $U=I\cap A$ is not dense in $I$. Therefore each $C_k$ is nowhere dense, and we are done.
 \qed
\end{document}
