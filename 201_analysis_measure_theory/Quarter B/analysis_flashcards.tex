\documentclass{article}
\usepackage[paperwidth=.5\paperwidth,paperheight=.25\paperheight]{geometry}
\usepackage{pgfpages}
\pagestyle{empty}
\thispagestyle{empty}
\pgfpagesuselayout{8 on 1}[a4paper]
\makeatletter
\@tempcnta=1\relax
\loop\ifnum\@tempcnta<9\relax
\pgf@pset{\the\@tempcnta}{bordercode}{\pgfusepath{stroke}}
\advance\@tempcnta by 1\relax
\repeat
\makeatother

%% Useful packages
\usepackage{amssymb, amsmath, amsthm} 
%\usepackage{graphicx}  %%this is currently enabled in the default document, so it is commented out here. 
\usepackage{calrsfs}
\usepackage{braket}
\usepackage{mathtools}
\usepackage{lipsum}
\usepackage{tikz}
\usetikzlibrary{cd}
\usepackage{verbatim}
%\usepackage{ntheorem}% for theorem-like environments
\usepackage{mdframed}%can make highlighted boxes of text
%Use case: https://tex.stackexchange.com/questions/46828/how-to-highlight-important-parts-with-a-gray-background
\usepackage{wrapfig}
\usepackage{centernot}
\usepackage{subcaption}%\begin{subfigure}{0.5\textwidth}
\usepackage{pgfplots}
\pgfplotsset{compat=1.13}
\usepackage[colorinlistoftodos]{todonotes}
\usepackage[colorlinks=true, allcolors=blue]{hyperref}
\usepackage{xfrac}					%to make slanted fractions \sfrac{numerator}{denominator}
\usepackage{enumitem}            
    %syntax: \begin{enumerate}[label=(\alph*)]
    %possible arguments: f \alph*, \Alph*, \arabic*, \roman* and \Roman*
\usetikzlibrary{arrows,shapes.geometric,fit}

\DeclareMathAlphabet{\pazocal}{OMS}{zplm}{m}{n}
%% Use \pazocal{letter} to typeset a letter in the other kind 
%%  of math calligraphic font. 

%% This puts the QED block at the end of each proof, the way I like it. 
\renewenvironment{proof}{{\bfseries Proof}}{\qed}
\makeatletter
\renewenvironment{proof}[1][\bfseries \proofname]{\par
  \pushQED{\qed}%
  \normalfont \topsep6\p@\@plus6\p@\relax
  \trivlist
  %\itemindent\normalparindent
  \item[\hskip\labelsep
        \scshape
    #1\@addpunct{}]\ignorespaces
}{%
  \popQED\endtrivlist\@endpefalse
}
\makeatother

%% This adds a \rewnewtheorem command, which enables me to override the settings for theorems contained in this document.
\makeatletter
\def\renewtheorem#1{%
  \expandafter\let\csname#1\endcsname\relax
  \expandafter\let\csname c@#1\endcsname\relax
  \gdef\renewtheorem@envname{#1}
  \renewtheorem@secpar
}
\def\renewtheorem@secpar{\@ifnextchar[{\renewtheorem@numberedlike}{\renewtheorem@nonumberedlike}}
\def\renewtheorem@numberedlike[#1]#2{\newtheorem{\renewtheorem@envname}[#1]{#2}}
\def\renewtheorem@nonumberedlike#1{  
\def\renewtheorem@caption{#1}
\edef\renewtheorem@nowithin{\noexpand\newtheorem{\renewtheorem@envname}{\renewtheorem@caption}}
\renewtheorem@thirdpar
}
\def\renewtheorem@thirdpar{\@ifnextchar[{\renewtheorem@within}{\renewtheorem@nowithin}}
\def\renewtheorem@within[#1]{\renewtheorem@nowithin[#1]}
\makeatother

%% This makes theorems and definitions with names show up in bold, the way I like it. 
\makeatletter
\def\th@plain{%
  \thm@notefont{}% same as heading font
  \itshape % body font
}
\def\th@definition{%
  \thm@notefont{}% same as heading font
  \normalfont % body font
}
\makeatother

%===============================================
%==============Shortcut Commands================
%===============================================
\newcommand{\ds}{\displaystyle}
\newcommand{\B}{\mathcal{B}}
\newcommand{\C}{\mathbb{C}}
\newcommand{\F}{\mathbb{F}}
\newcommand{\N}{\mathbb{N}}
\newcommand{\R}{\mathbb{R}}
\newcommand{\Q}{\mathbb{Q}}
\newcommand{\T}{\mathcal{T}}
\newcommand{\Z}{\mathbb{Z}}
\renewcommand\qedsymbol{$\blacksquare$}
\newcommand{\qedwhite}{\hfill\ensuremath{\square}}
\newcommand*\conj[1]{\overline{#1}}
\newcommand*\closure[1]{\overline{#1}}
\newcommand*\mean[1]{\overline{#1}}
%\newcommand{\inner}[1]{\left< #1 \right>}
\newcommand{\inner}[2]{\left< #1, #2 \right>}
\newcommand{\powerset}[1]{\pazocal{P}(#1)}
%% Use \pazocal{letter} to typeset a letter in the other kind 
%%  of math calligraphic font. 
\newcommand{\cardinality}[1]{\left| #1 \right|}
\newcommand{\domain}[1]{\mathcal{D}(#1)}
\newcommand{\image}{\text{Im}}
\newcommand{\inv}[1]{#1^{-1}}
\newcommand{\preimage}[2]{#1^{-1}\left(#2\right)}
\newcommand{\script}[1]{\mathcal{#1}}


\newenvironment{highlight}{\begin{mdframed}[backgroundcolor=gray!20]}{\end{mdframed}}

\DeclarePairedDelimiter\ceil{\lceil}{\rceil}
\DeclarePairedDelimiter\floor{\lfloor}{\rfloor}

%===============================================
%===============My Tikz Commands================
%===============================================
\newcommand{\drawsquiggle}[1]{\draw[shift={(#1,0)}] (.005,.05) -- (-.005,.02) -- (.005,-.02) -- (-.005,-.05);}
\newcommand{\drawpoint}[2]{\draw[*-*] (#1,0.01) node[below, shift={(0,-.2)}] {#2};}
\newcommand{\drawopoint}[2]{\draw[o-o] (#1,0.01) node[below, shift={(0,-.2)}] {#2};}
\newcommand{\drawlpoint}[2]{\draw (#1,0.02) -- (#1,-0.02) node[below] {#2};}
\newcommand{\drawlbrack}[2]{\draw (#1+.01,0.02) --(#1,0.02) -- (#1,-0.02) -- (#1+.01,-0.02) node[below, shift={(-.01,0)}] {#2};}
\newcommand{\drawrbrack}[2]{\draw (#1-.01,0.02) --(#1,0.02) -- (#1,-0.02) -- (#1-.01,-0.02) node[below, shift={(+.01,0)}] {#2};}

%***********************************************
%**************Start of Document****************
%***********************************************
 %find me at /home/trevor/texmf/tex/latex/tskpreamble_nothms.tex

\newenvironment{flashcard}[2][]{%
\noindent  \textsc{#1}

\vfill 
\centerline{{\Large{#2}}}
\vfill
\newpage \vspace*{\stretch{1}} \noindent
}
{\vspace*{\stretch{1}}\newpage}

\usepackage[latin1]{inputenc}
\usepackage{amsfonts}
\usepackage{amsmath}

\begin{document}

\begin{flashcard}[Theorem]{Fatou's Lemma}
If $f_n$ are all \mumeasurable{}, then 
$$\int\liminf_n f_n \der\mu \leq \liminf_n \int f_n \der\mu.$$
\end{flashcard}

\begin{flashcard}[Definition]{A measure}
\begin{itemize}
\item $\mu(\emptyset)=0$, positive
\item monotonicity
\item subadditivity
\end{itemize}
\end{flashcard}

\begin{flashcard}[Definition]{Borel measure}
Borel sets are measurable. (The measure splits)
\end{flashcard}

\begin{flashcard}[Definition]{Regular measure}
"Every set is $\mu$-almost a measurable set."
\end{flashcard}

\begin{flashcard}[Definition]{Borel-regular measure}
"Every set is $\mu$-almost a Borel set, and $\mu$ is Borel."
\end{flashcard}

\begin{flashcard}[Definition]{Radon measure}
"Compact sets have finite $\mu$-measure, and $\mu$ is Borel-regular."
\end{flashcard}

\begin{flashcard}[Theorem]{Radon-Nikodym Theorem}
Let $\mu, \nu$ be Radon measures on $\R^n$, with $\nu<<\mu$. Then 
$$\nu(A)=\int_A D_\mu \nu \der\mu$$
for all \mumeasurable{} $A\subseteq\R^n$. 
\end{flashcard}

\begin{flashcard}[Theorem]{Young's Inequality}
If $a,b\geq0$ and $p,q>1$ such that $\frac{1}{p}+\frac{1}{q}=1$, then 
$$ab\leq \frac{a^p}{p}+\frac{b^q}{q}.$$
\end{flashcard}

\begin{flashcard}[Theorem]{Holder Inequality}
If $f\in L^p(\Omega)$ and $g\in L^q(\Omega)$ where $p,q$ conjugate exponents, then 
\begin{align*}
\norm{fg}_{L^1}&\leq \norm{f}_{L^p} \norm{g}_{L^q}, \quad \text{ that is, } \\
\int_\Omega \abs{f(x)g(x)}\dx &\leq \left(\int_\Omega \abs{f(x)}^p\dx\right)^{\frac{1}{p}} \left(\int_\Omega \abs{g(x)}^q\dx\right)^{\frac{1}{q}}
\end{align*}
This can be thought of a "sort of" a Cauchy-Schwarz for Banach Spaces.
\end{flashcard}

\begin{flashcard}[Definition]{Lipschitz Continuity}
A function $f:X\to Y$ is \emph{Lipschitz continuous} if there exists $M>0$ such that for all $x_1, x_2\in X$, 
$$\norm{f(x_1)-f(x_2)}_Y \leq M\norm{x_1-x_2}_X.$$
If $f:X\to X$ and $M<1$, then $f$ is in particular a \emph{contraction}. 
\end{flashcard}

\begin{flashcard}[Definition]{Baire First Category}
A set $A$ is \_\_\_\_\_\_\_\_\_\_\_ if it is a countable union of nowhere dense sets.
\end{flashcard}

\begin{flashcard}[Definition]{Nowhere dense set}
A set $A$ is \_\_\_\_\_\_\_\_\_\_\_ if $\closure{A}$ has empty interior. 
\end{flashcard}

\begin{flashcard}[Definition]{Sequentially compact space}
A topological space $X$ is \_\_\_\_\_\_\_\_\_\_\_ if every sequence in $X$ has a convergent subsequence (to a point in $X$). 
\end{flashcard}

\end{document}