\documentclass[12pt,letterpaper]{article}

\usepackage{fancyhdr,fancybox}

%% Useful packages
\usepackage{amssymb, amsmath, amsthm} 
%\usepackage{graphicx}  %%this is currently enabled in the default document, so it is commented out here. 
\usepackage{calrsfs}
\usepackage{braket}
\usepackage{mathtools}
\usepackage{lipsum}
\usepackage{tikz}
\usetikzlibrary{cd}
\usepackage{verbatim}
%\usepackage{ntheorem}% for theorem-like environments
\usepackage{mdframed}%can make highlighted boxes of text
%Use case: https://tex.stackexchange.com/questions/46828/how-to-highlight-important-parts-with-a-gray-background
\usepackage{wrapfig}
\usepackage{centernot}
\usepackage{subcaption}%\begin{subfigure}{0.5\textwidth}
\usepackage{pgfplots}
\pgfplotsset{compat=1.13}
\usepackage[colorinlistoftodos]{todonotes}
\usepackage[colorlinks=true, allcolors=blue]{hyperref}
\usepackage{xfrac}					%to make slanted fractions \sfrac{numerator}{denominator}
\usepackage{enumitem}            
    %syntax: \begin{enumerate}[label=(\alph*)]
    %possible arguments: f \alph*, \Alph*, \arabic*, \roman* and \Roman*
\usetikzlibrary{arrows,shapes.geometric,fit}

\DeclareMathAlphabet{\pazocal}{OMS}{zplm}{m}{n}
%% Use \pazocal{letter} to typeset a letter in the other kind 
%%  of math calligraphic font. 

%% This puts the QED block at the end of each proof, the way I like it. 
\renewenvironment{proof}{{\bfseries Proof}}{\qed}
\makeatletter
\renewenvironment{proof}[1][\bfseries \proofname]{\par
  \pushQED{\qed}%
  \normalfont \topsep6\p@\@plus6\p@\relax
  \trivlist
  %\itemindent\normalparindent
  \item[\hskip\labelsep
        \scshape
    #1\@addpunct{}]\ignorespaces
}{%
  \popQED\endtrivlist\@endpefalse
}
\makeatother

%% This adds a \rewnewtheorem command, which enables me to override the settings for theorems contained in this document.
\makeatletter
\def\renewtheorem#1{%
  \expandafter\let\csname#1\endcsname\relax
  \expandafter\let\csname c@#1\endcsname\relax
  \gdef\renewtheorem@envname{#1}
  \renewtheorem@secpar
}
\def\renewtheorem@secpar{\@ifnextchar[{\renewtheorem@numberedlike}{\renewtheorem@nonumberedlike}}
\def\renewtheorem@numberedlike[#1]#2{\newtheorem{\renewtheorem@envname}[#1]{#2}}
\def\renewtheorem@nonumberedlike#1{  
\def\renewtheorem@caption{#1}
\edef\renewtheorem@nowithin{\noexpand\newtheorem{\renewtheorem@envname}{\renewtheorem@caption}}
\renewtheorem@thirdpar
}
\def\renewtheorem@thirdpar{\@ifnextchar[{\renewtheorem@within}{\renewtheorem@nowithin}}
\def\renewtheorem@within[#1]{\renewtheorem@nowithin[#1]}
\makeatother

%% This makes theorems and definitions with names show up in bold, the way I like it. 
\makeatletter
\def\th@plain{%
  \thm@notefont{}% same as heading font
  \itshape % body font
}
\def\th@definition{%
  \thm@notefont{}% same as heading font
  \normalfont % body font
}
\makeatother

%===============================================
%==============Shortcut Commands================
%===============================================
\newcommand{\ds}{\displaystyle}
\newcommand{\B}{\mathcal{B}}
\newcommand{\C}{\mathbb{C}}
\newcommand{\F}{\mathbb{F}}
\newcommand{\N}{\mathbb{N}}
\newcommand{\R}{\mathbb{R}}
\newcommand{\Q}{\mathbb{Q}}
\newcommand{\T}{\mathcal{T}}
\newcommand{\Z}{\mathbb{Z}}
\renewcommand\qedsymbol{$\blacksquare$}
\newcommand{\qedwhite}{\hfill\ensuremath{\square}}
\newcommand*\conj[1]{\overline{#1}}
\newcommand*\closure[1]{\overline{#1}}
\newcommand*\mean[1]{\overline{#1}}
%\newcommand{\inner}[1]{\left< #1 \right>}
\newcommand{\inner}[2]{\left< #1, #2 \right>}
\newcommand{\powerset}[1]{\pazocal{P}(#1)}
%% Use \pazocal{letter} to typeset a letter in the other kind 
%%  of math calligraphic font. 
\newcommand{\cardinality}[1]{\left| #1 \right|}
\newcommand{\domain}[1]{\mathcal{D}(#1)}
\newcommand{\image}{\text{Im}}
\newcommand{\inv}[1]{#1^{-1}}
\newcommand{\preimage}[2]{#1^{-1}\left(#2\right)}
\newcommand{\script}[1]{\mathcal{#1}}


\newenvironment{highlight}{\begin{mdframed}[backgroundcolor=gray!20]}{\end{mdframed}}

\DeclarePairedDelimiter\ceil{\lceil}{\rceil}
\DeclarePairedDelimiter\floor{\lfloor}{\rfloor}

%===============================================
%===============My Tikz Commands================
%===============================================
\newcommand{\drawsquiggle}[1]{\draw[shift={(#1,0)}] (.005,.05) -- (-.005,.02) -- (.005,-.02) -- (-.005,-.05);}
\newcommand{\drawpoint}[2]{\draw[*-*] (#1,0.01) node[below, shift={(0,-.2)}] {#2};}
\newcommand{\drawopoint}[2]{\draw[o-o] (#1,0.01) node[below, shift={(0,-.2)}] {#2};}
\newcommand{\drawlpoint}[2]{\draw (#1,0.02) -- (#1,-0.02) node[below] {#2};}
\newcommand{\drawlbrack}[2]{\draw (#1+.01,0.02) --(#1,0.02) -- (#1,-0.02) -- (#1+.01,-0.02) node[below, shift={(-.01,0)}] {#2};}
\newcommand{\drawrbrack}[2]{\draw (#1-.01,0.02) --(#1,0.02) -- (#1,-0.02) -- (#1-.01,-0.02) node[below, shift={(+.01,0)}] {#2};}

%***********************************************
%**************Start of Document****************
%***********************************************
 %find me at /home/trevor/texmf/tex/latex/tskpreamble_nothms.tex
%===============================================
%===============Theorem Styles==================
%===============================================

%================Default Style==================
\theoremstyle{plain}% is the default. it sets the text in italic and adds extra space above and below the \newtheorems listed below it in the input. it is recommended for theorems, corollaries, lemmas, propositions, conjectures, criteria, and (possibly; depends on the subject area) algorithms.
\newtheorem{theorem}{Theorem}
\numberwithin{theorem}{section} %This sets the numbering system for theorems to number them down to the {argument} level. I have it set to number down to the {section} level right now.
\newtheorem*{theorem*}{Theorem} %Theorem with no numbering
\newtheorem{corollary}[theorem]{Corollary}
\newtheorem*{corollary*}{Corollary}
\newtheorem{conjecture}[theorem]{Conjecture}
\newtheorem{lemma}[theorem]{Lemma}
\newtheorem*{lemma*}{Lemma}
\newtheorem{proposition}[theorem]{Proposition}
\newtheorem*{proposition*}{Proposition}
\newtheorem{problemstatement}[theorem]{Problem Statement}


%==============Definition Style=================
\theoremstyle{definition}% adds extra space above and below, but sets the text in roman. it is recommended for definitions, conditions, problems, and examples; i've alse seen it used for exercises.
\newtheorem{definition}[theorem]{Definition}
\newtheorem*{definition*}{Definition}
\newtheorem{condition}[theorem]{Condition}
\newtheorem{problem}[theorem]{Problem}
\newtheorem{example}[theorem]{Example}
\newtheorem*{example*}{Example}
\newtheorem*{counterexample*}{Counterexample}
\newtheorem*{romantheorem*}{Theorem} %Theorem with no numbering
\newtheorem{exercise}{Exercise}
\numberwithin{exercise}{section}
\newtheorem{algorithm}[theorem]{Algorithm}

%================Remark Style===================
\theoremstyle{remark}% is set in roman, with no additional space above or below. it is recommended for remarks, notes, notation, claims, summaries, acknowledgments, cases, and conclusions.
\newtheorem{remark}[theorem]{Remark}
\newtheorem*{remark*}{Remark}
\newtheorem{notation}[theorem]{Notation}
\newtheorem*{notation*}{Notation}
%\newtheorem{claim}[theorem]{Claim}  %%use this if you ever want claims to be numbered
\newtheorem*{claim}{Claim}


%%
%% Page set-up:
%%
\pagestyle{empty}
\lhead{\textsc{201 - Real Analysis \\}} 
\rhead{\textsc{Harutyunyan, Winter 2020} \\ Trevor Klar}
%\chead{\Large\textbf{A New Integration Technique \\ }}
\renewcommand{\headrulewidth}{0pt}
%
\renewcommand{\footrulewidth}{0pt}
%\lfoot{
%Office: \quad \quad \, M 2-3 \, \, SH 6431x \\
%Math Lab: \, W 12-2 \, SH 1607
%}
%\rfoot{trevorklar@math.ucsb.edu}

\setlength{\parindent}{0in}
\setlength{\textwidth}{7in}
\setlength{\evensidemargin}{-0.25in}
\setlength{\oddsidemargin}{-0.25in}
\setlength{\parskip}{.5\baselineskip}
\setlength{\topmargin}{-0.5in}
\setlength{\textheight}{9in}

\setlist[enumerate,1]{label=\textbf{\arabic*.}}

\begin{document}
\pagestyle{fancy}
\begin{center}
{\Large Homework 2}%=================UPDATE THIS=================%
\end{center}

\jpg{width=\linewidth}{hw2-p1-1}
\begin{proof}\mbox{}
\begin{itemize}
\item (Monotonicity) Let $A \subseteq B$, and denote $\script{O}_A$ the collection of all open sets containing $A$, and $\script{O}_B$ the collection of all open sets containing $B$.
\begin{align*}
\nu(B) &= \inf\{\nu(U) : U\in\script{O}_B\} \\
&\geq \inf\{\nu(U) : U\in\script{O}_A\} \quad\quad \text{since }\script{O}_B\subseteq \script{O}_A,\\
&=\nu(A).
\end{align*}

\item $\nu(\emptyset)=0$\footnote{Note, we have to require that our definition of $\nu$ on open sets only applies to nonempty open sets, since otherwise $\nu(\emptyset)=\nu(a,a)=f(a-)-f(a+)\neq0$ if $a$ is a point of discontinuity.}, since 
	\begin{align*}
	\lim_{n\to\infty}\nu\left (1,1+\tfrac{1}{n}\right ) &= \lim_{n\to\infty}\left (\lim_-f\left (1+\frac{1}{n}\right )-\lim_+f(1)\right ) \\
	&= \lim_+f(1)-\lim_+f(1) \\
	&=0
	\end{align*}
Thus for any $\epsilon>0,$ there exists $n$ such that $\nu\left (1,1+\tfrac{1}{n}\right )<\epsilon$, and by monotonicity, \\
$\nu(\emptyset)\leq\nu\left (1,1+\tfrac{1}{n}\right )<\epsilon$.

\vfill	
\pagebreak
\item (Subadditivity for intervals)%
\footnote{In order for our definitions of $\nu(U)$ and $\nu(A)$ to agree when $U=A$ where $U$ is an interval containing a point of jump discontinuity, we need to restrict our definition of $\nu(U)$ to nonempty open sets \emph{on which $f$ is continuous}. We can only have jump discontinuities, and only countably many, since $f$ is increasing.} 
Let $\{I_i\}$ be any countable collection of nonempty open intervals. \WLOG{} we can assume that $f$ is continuous on each $I_i$, since we could always partition an interval by splitting at (at most countable many) discontinuities. Then 
\begin{align*}
\sum_{i=1}^\infty \nu(I_i) &= \sum_{i=1}^\infty f(b_i)-f(a_i)  \\
\end{align*}
and to consider the union $\bigcup_{i=1}^\infty I_i$, we just consider the list of all endpoints $\Gamma=\{(a_j,b_j)\}_{j=1}^\infty$, where we omit any endpoint which is contained in another interval. i.e. if $(a_1,b_1)\cap(a_2,b_2)\neq\emptyset$, then we omit $b_1$ and $a_2$ to obtain $\{(a_1,b_2)\}$. Thus 
\begin{align*}
 \sum_{i=1}^\infty f(b_i)-f(a_i) 
&\geq \sum_{(a_j,b_j)\in\Gamma} f(b_j)-f(a_j) \\
&=\sum_{(a_j,b_j)\in\Gamma} \nu(a_j,b_j) \\
&=\nu\left (\bigcup_{i=1}^\infty I_i \right )
\end{align*}
\item (Subadditivity for open sets) Every open set $U_i$ is a union of disjoint open intervals, so 
\begin{align*}
\nu\bigcup_{i\in\N} U_i &= \nu \bigcup_{i\in\N} \left (\bigcup_{j\in\N} I_{ij}\right ) \\
&= \nu\bigcup_{i,j\in\N} I_{ij} \\
&\leq \sum_{i,j\in\N} \nu(I_{ij}) \\
&= \sum_{i\in\N} \left (\sum_{j\in\N} \nu(I_{ij})\right ) \\
&= \sum_{i\in\N} \nu(U_i)
\end{align*}

\vfill
\pagebreak
\item (Subadditivity for general sets) Let $\{A_i\}$ be a countable collection of sets in $\R^n$, and for each $i$, denote $\script{O}_{A_i}$ the collection of all open sets containing $A_i$. 
\begin{align*}
\sum_{i=1}^\infty \nu(A_i) &= \sum_{i=1}^\infty \inf\{\nu(U) : U\in \script{O}_{A_i}\}\\
&= \inf\left\lbrace\sum_{i=1}^\infty\nu(U_i) : U_i\in \script{O}_{A_i} \quad \forall i\in \N\right\rbrace & (\footnotemark) \\
&\geq \inf\left\lbrace\nu\left (\bigcup_{i=1}^\infty U_i\right ) : U_i\in \script{O}_{A_i} \quad \forall i\in \N\right\rbrace \\
&=\nu\left (\bigcup_{i=1}^\infty A_i\right )
\end{align*}
\footnotetext{In case this equality is not obvious, observe that for any $\epsilon$ we can choose a particular collection of $U_i$ such that $\inf\sum \leq \sum \leq \sum(\inf+\frac{\epsilon}{2^i})$, and conversely we can choose a collection of $U_i$ such that $\sum\inf \leq \sum \leq (\inf\sum)+\epsilon$.}
\end{itemize} 
Thus we have shown that $\nu(\emptyset)=0$, and that $\nu$ has the monotonicity and subadditivity properties, therefore it is a measure.
\end{proof}

\jpg{width=\linewidth}{hw2-p2-1}
\begin{proof} Let $\{q_i\}$ be an enumeration of the rationals, and let
\begin{align*}
A^\complement&=\bigcup_{i=1}^{\infty}B_{2^{-i}}(q_i), \text{ with } \\
f(x)&= \begin{cases}
0, & x\in A\\ 
\infty, & x\in A^\complement
\end{cases}
\end{align*}
Observe that $m(A)=\infty$, and $m(A^\complement)\leq2$, but in fact we could scale the radii of the balls comprising $A$ to make the sum arbitrarily small (thus if we prove (1), then (2) follows).
\vfill
\pagebreak
Now let $x\not\in A$, and observe that $f(x)=0$. Also $x$ is irrational, but any ball containing $x$ contains rationals, and thus has nonempty intersection with $B_{2^{-i}}(q_i)$ for some $q_i$. Thus for any $r>0$, 
\begin{align*}
\frac{1}{m(B_r(x)}\int_{B_r(x)} f\der m 
&= \frac{1}{m(B_r(x)}\left(\int_{B_r(x)\setminus B_{2^{-1}}(q_{i})} \hspace*{-4em}f\der m\hspace*{2em} + \int_{B_r(x)\cap B_{2^{-1}}(q_{i})}\hspace*{-4em} f\der m \hspace*{2em}\right) \\
&= \frac{1}{m(B_r(x)}\left(\int_{B_r(x)\setminus B_{2^{-1}}(q_{i})} \hspace*{-4em}f\der m\hspace*{2em} + \infty \right)\\
&=\infty
\end{align*}
and we are done.
\end{proof}

\jpg{width=\linewidth}{hw2-p3-1}
 
\textbf{Remark.} I think it's worth pointing out that this problem is also a counterexample to the Lebesgue-Besicovitch Differentiation Theorem, as was number 2. In 2, our function wasn't $L^{1}_{\text{loc}}$, so the function did not equal the limit anywhere in $A$. In this problem, all the conditions are satisfied since $$\lim_{r\to0}\frac{m(E\cap[-r,r])}{2r}=\lim_{r\to0}\mint{-}_{B_{r}(0)}\Chi_{E}\der m,$$ but critically, the theorem only holds \muae{}, and we are only considering the limit at the single point $\{0\}$. 

\begin{proof}
We will construct $E$ by starting with 
\begin{align*}
E_1&=\bigcup_{i=1}^{\infty} A_i, \\
\text{where each }A_i&=\left[-\frac{2}{2^i},-\frac{1}{2^i}\right]\cup\left[\frac{1}{2^i},\frac{2}{2^i}\right]
\end{align*}
\jpg{width=0.6\textwidth}{hw2-p3-2}
To construct $B_i$, from each component of $A_i$ remove the central interval whose length is $\frac{\alpha}{2^{-i}}$. Then let $E_2=\bigcup_{i-1}^\infty	 B_i$. 

\jpg{width=0.6\textwidth}{hw2-p3-3}

Observe that whenever $r=2^{-n}$ for some $n\in\N$, we have $\mint{-}_{B_{r}(0)}\Chi_{E_2}\der m=\alpha$, since the ratio of the removed amount for each component is exactly $\alpha$ by construction. However, 
\begin{itemize}
\item As $r$ decreases from $2^{-n-1}$ to 
$2^{-n-1}-\left(\frac{\alpha}{2}\right)2^{-n-1}$, the average value decreases,
\item As $r$ decreases from $2^{-n-1}-\left(\frac{\alpha}{2}\right)2^{-n-1}$ to $2^{-n}+\left(\frac{\alpha}{2}\right)2^{-n-1}$, the average value increases,
\item If $r=2^{-n},2^{-n-1},\text{avg}(2^{-n},2^{-n-1})$, then the average value is $\alpha$. 
\end{itemize}
Furthermore, the maximum and minimum values of $\mint{-}_{B_{r}(0)}\Chi_{E_2}\der m=\alpha$ do not get any smaller as $r\to0$, so the limit does not exist.

To fix this, for each $B_i$ define $C_i$ by modifying $B_i$ so that the removed amount is not one central interval, but $i$ equally spaced intervals whose total length is $\alpha2^{-i}$. Let $E=\bigcup_{i=1}^\infty C_i$. 

\jpg{width=0.6\textwidth}{hw2-p3-4}

This causes the maximum and minimum values of $\mint{-}_{B_{r}(0)}\Chi_{E}\der m$ to be closer to $\alpha$ by a factor of $n$, for $r\in(2^{-n},2^{-n-1})$. 
\end{proof}

\jpg{width=\linewidth}{hw2-p4-1}
We will follow that standard proof that any continuously differentiable function with nonnegative derivative is increasing. We will show that an analogue of Rolle's Theorem implies an analogue of the Mean Value Theorem, which implies the result. 
\begin{lemma}(Rolle's Theorem analogue)
Let $f:[a,b]\to\R$ be continuous, with $f(a)=f(b)$. Then there exists some $c\in(a,b)$ such that 
\begin{align*}
D^+f(c)&=\limsup_{h\to0^+}\frac{f(c+h)-f(c)}{h} \\
&\leq 0.
\end{align*}
\end{lemma}
\begin{proof}
Since $f$ is continuous on a compact set, it attains its max and min on $[a,b]$. If $\max f=\min f=f(a)=f(b)$, then $f$ is constant, so $D^+f=0$ everywhere on $[a,b]$ and we're done. Otherwise, suppose $f(c)$ is a maximum. Since $f(a)=f(b)$, then $c<b$. Then $f(c+h)\leq f(c)$ for all $h>0$, so the difference quotient is negative for all $h$-values. Thus
\begin{align*}
D^+f(c)&=\limsup_{h\to0^+}\frac{f(c+h)-f(c)}{h} \\
%
&\leq 0.
\end{align*}
\end{proof}
\begin{lemma}(Mean Value Theorem analogue)
Let $f:[a,b]\to\R$ be continuous. For all $\alpha, \beta\in [a,b]$, there exists $c\in[a,b]$ such that $D^+f(c)\leq\frac{f(\beta)-f(\alpha)}{\beta-\alpha}$. 
\end{lemma}
\begin{proof}
Write $f$ as $f(x)=g(x)+r(x-\alpha)$, where $r=\frac{f(\beta)-f(\alpha)}{\beta-\alpha}$. Then $g(\alpha)=g(\beta)$, so by Lemma 1, there exists $c\in[\alpha,\beta]$ such that $D^+g(c)\leq0$, so 
\begin{align*}
D^+f(c)&=D^+g(c)+D^+(rx)(c)\\
&=D^+g(c)+r\\
&\leq r\\
&=\frac{f(\beta)-f(\alpha)}{\beta-\alpha}
\end{align*}
and we're done.
\end{proof}
\begin{proof}(Main result)
Suppose $f:[a,b]\to\R$ is continuous, and $D^+f(x)\geq0$ for each $x\in[a,b)$. By Lemma 2, there exists some $c$ in $(a,b)$ such that $D^+f(c)\leq\frac{f(b)-f(a)}{b-a}$. Now $D^+f(x)\geq0$ everywhere, so the difference quotient is positive. This means that since $b-a>0$, then $f(b)-f(a)>0$, and we're done. 
\end{proof}
\vfill
\pagebreak
\jpg{width=\linewidth}{hw2-p5-1}
%I can't find a counterexample, since the Cantor function is not differentiable everywhere, just almost everywhere. I think $x^n\sin(\sfrac{1}{x^m})$ (where $f(0)=0$) could be a good candidate for appropriate choice of $m,n$, but whenever I chose values so the function was differentiable, I couldn't see how to show that the function wasn't absolutely continuous. In fact, It's possible (though I don't know how to prove it) that they're all absolutely continuous for any non-stupid values of $m,n$. 
Yes. 
\begin{proof}
Let $f:[a,b]\to\R$ be differentiable. Then $f'$ is continuous on $[a,b]$, so $f'$ attains its max and min. Let $\epsilon>0$, we will show that $f$ is absolutely continuous. Let $\alpha=\max(\abs{\max(f')}, \abs{\min(f')})$. For any $x,y\in[a,b]$, the difference quotient is bounded by $\alpha$:
$$\abs{\frac{f(y)-f(x)}{y-x}}\leq\alpha$$
since by MVT we can find some $c$ with $f'(c)=\frac{f(y)-f(x)}{y-x}$. Thus 
$$\abs{f(y)-f(x)}\leq\alpha\abs{y-x}.$$ 
This means we can let $\delta=\frac{\epsilon}{\alpha}$ and find that if $\sum_{i=1}^n\abs{y_i-x_i}<\delta,$ then 
\begin{align*}
\sum_{i=1}^n\abs{f(y_i)-f(x_i)}&\leq\sum_{i=1}^n\alpha\abs{y_i-x_i} \\
&< \alpha\delta \\
&= \epsilon
\end{align*}
and we're done.
\end{proof}



\end{document}
