\documentclass[12pt,letterpaper]{article}

\usepackage{fancyhdr,fancybox}

%% Useful packages
\usepackage{amssymb, amsmath, amsthm} 
%\usepackage{graphicx}  %%this is currently enabled in the default document, so it is commented out here. 
\usepackage{calrsfs}
\usepackage{braket}
\usepackage{mathtools}
\usepackage{lipsum}
\usepackage{tikz}
\usetikzlibrary{cd}
\usepackage{verbatim}
%\usepackage{ntheorem}% for theorem-like environments
\usepackage{mdframed}%can make highlighted boxes of text
%Use case: https://tex.stackexchange.com/questions/46828/how-to-highlight-important-parts-with-a-gray-background
\usepackage{wrapfig}
\usepackage{centernot}
\usepackage{subcaption}%\begin{subfigure}{0.5\textwidth}
\usepackage{pgfplots}
\pgfplotsset{compat=1.13}
\usepackage[colorinlistoftodos]{todonotes}
\usepackage[colorlinks=true, allcolors=blue]{hyperref}
\usepackage{xfrac}					%to make slanted fractions \sfrac{numerator}{denominator}
\usepackage{enumitem}            
    %syntax: \begin{enumerate}[label=(\alph*)]
    %possible arguments: f \alph*, \Alph*, \arabic*, \roman* and \Roman*
\usetikzlibrary{arrows,shapes.geometric,fit}

\DeclareMathAlphabet{\pazocal}{OMS}{zplm}{m}{n}
%% Use \pazocal{letter} to typeset a letter in the other kind 
%%  of math calligraphic font. 

%% This puts the QED block at the end of each proof, the way I like it. 
\renewenvironment{proof}{{\bfseries Proof}}{\qed}
\makeatletter
\renewenvironment{proof}[1][\bfseries \proofname]{\par
  \pushQED{\qed}%
  \normalfont \topsep6\p@\@plus6\p@\relax
  \trivlist
  %\itemindent\normalparindent
  \item[\hskip\labelsep
        \scshape
    #1\@addpunct{}]\ignorespaces
}{%
  \popQED\endtrivlist\@endpefalse
}
\makeatother

%% This adds a \rewnewtheorem command, which enables me to override the settings for theorems contained in this document.
\makeatletter
\def\renewtheorem#1{%
  \expandafter\let\csname#1\endcsname\relax
  \expandafter\let\csname c@#1\endcsname\relax
  \gdef\renewtheorem@envname{#1}
  \renewtheorem@secpar
}
\def\renewtheorem@secpar{\@ifnextchar[{\renewtheorem@numberedlike}{\renewtheorem@nonumberedlike}}
\def\renewtheorem@numberedlike[#1]#2{\newtheorem{\renewtheorem@envname}[#1]{#2}}
\def\renewtheorem@nonumberedlike#1{  
\def\renewtheorem@caption{#1}
\edef\renewtheorem@nowithin{\noexpand\newtheorem{\renewtheorem@envname}{\renewtheorem@caption}}
\renewtheorem@thirdpar
}
\def\renewtheorem@thirdpar{\@ifnextchar[{\renewtheorem@within}{\renewtheorem@nowithin}}
\def\renewtheorem@within[#1]{\renewtheorem@nowithin[#1]}
\makeatother

%% This makes theorems and definitions with names show up in bold, the way I like it. 
\makeatletter
\def\th@plain{%
  \thm@notefont{}% same as heading font
  \itshape % body font
}
\def\th@definition{%
  \thm@notefont{}% same as heading font
  \normalfont % body font
}
\makeatother

%===============================================
%==============Shortcut Commands================
%===============================================
\newcommand{\ds}{\displaystyle}
\newcommand{\B}{\mathcal{B}}
\newcommand{\C}{\mathbb{C}}
\newcommand{\F}{\mathbb{F}}
\newcommand{\N}{\mathbb{N}}
\newcommand{\R}{\mathbb{R}}
\newcommand{\Q}{\mathbb{Q}}
\newcommand{\T}{\mathcal{T}}
\newcommand{\Z}{\mathbb{Z}}
\renewcommand\qedsymbol{$\blacksquare$}
\newcommand{\qedwhite}{\hfill\ensuremath{\square}}
\newcommand*\conj[1]{\overline{#1}}
\newcommand*\closure[1]{\overline{#1}}
\newcommand*\mean[1]{\overline{#1}}
%\newcommand{\inner}[1]{\left< #1 \right>}
\newcommand{\inner}[2]{\left< #1, #2 \right>}
\newcommand{\powerset}[1]{\pazocal{P}(#1)}
%% Use \pazocal{letter} to typeset a letter in the other kind 
%%  of math calligraphic font. 
\newcommand{\cardinality}[1]{\left| #1 \right|}
\newcommand{\domain}[1]{\mathcal{D}(#1)}
\newcommand{\image}{\text{Im}}
\newcommand{\inv}[1]{#1^{-1}}
\newcommand{\preimage}[2]{#1^{-1}\left(#2\right)}
\newcommand{\script}[1]{\mathcal{#1}}


\newenvironment{highlight}{\begin{mdframed}[backgroundcolor=gray!20]}{\end{mdframed}}

\DeclarePairedDelimiter\ceil{\lceil}{\rceil}
\DeclarePairedDelimiter\floor{\lfloor}{\rfloor}

%===============================================
%===============My Tikz Commands================
%===============================================
\newcommand{\drawsquiggle}[1]{\draw[shift={(#1,0)}] (.005,.05) -- (-.005,.02) -- (.005,-.02) -- (-.005,-.05);}
\newcommand{\drawpoint}[2]{\draw[*-*] (#1,0.01) node[below, shift={(0,-.2)}] {#2};}
\newcommand{\drawopoint}[2]{\draw[o-o] (#1,0.01) node[below, shift={(0,-.2)}] {#2};}
\newcommand{\drawlpoint}[2]{\draw (#1,0.02) -- (#1,-0.02) node[below] {#2};}
\newcommand{\drawlbrack}[2]{\draw (#1+.01,0.02) --(#1,0.02) -- (#1,-0.02) -- (#1+.01,-0.02) node[below, shift={(-.01,0)}] {#2};}
\newcommand{\drawrbrack}[2]{\draw (#1-.01,0.02) --(#1,0.02) -- (#1,-0.02) -- (#1-.01,-0.02) node[below, shift={(+.01,0)}] {#2};}

%***********************************************
%**************Start of Document****************
%***********************************************
 %find me at /home/trevor/texmf/tex/latex/tskpreamble_nothms.tex
%===============================================
%===============Theorem Styles==================
%===============================================

%================Default Style==================
\theoremstyle{plain}% is the default. it sets the text in italic and adds extra space above and below the \newtheorems listed below it in the input. it is recommended for theorems, corollaries, lemmas, propositions, conjectures, criteria, and (possibly; depends on the subject area) algorithms.
\newtheorem{theorem}{Theorem}
\numberwithin{theorem}{section} %This sets the numbering system for theorems to number them down to the {argument} level. I have it set to number down to the {section} level right now.
\newtheorem*{theorem*}{Theorem} %Theorem with no numbering
\newtheorem{corollary}[theorem]{Corollary}
\newtheorem*{corollary*}{Corollary}
\newtheorem{conjecture}[theorem]{Conjecture}
\newtheorem{lemma}[theorem]{Lemma}
\newtheorem*{lemma*}{Lemma}
\newtheorem{proposition}[theorem]{Proposition}
\newtheorem*{proposition*}{Proposition}
\newtheorem{problemstatement}[theorem]{Problem Statement}


%==============Definition Style=================
\theoremstyle{definition}% adds extra space above and below, but sets the text in roman. it is recommended for definitions, conditions, problems, and examples; i've alse seen it used for exercises.
\newtheorem{definition}[theorem]{Definition}
\newtheorem*{definition*}{Definition}
\newtheorem{condition}[theorem]{Condition}
\newtheorem{problem}[theorem]{Problem}
\newtheorem{example}[theorem]{Example}
\newtheorem*{example*}{Example}
\newtheorem*{counterexample*}{Counterexample}
\newtheorem*{romantheorem*}{Theorem} %Theorem with no numbering
\newtheorem{exercise}{Exercise}
\numberwithin{exercise}{section}
\newtheorem{algorithm}[theorem]{Algorithm}

%================Remark Style===================
\theoremstyle{remark}% is set in roman, with no additional space above or below. it is recommended for remarks, notes, notation, claims, summaries, acknowledgments, cases, and conclusions.
\newtheorem{remark}[theorem]{Remark}
\newtheorem*{remark*}{Remark}
\newtheorem{notation}[theorem]{Notation}
\newtheorem*{notation*}{Notation}
%\newtheorem{claim}[theorem]{Claim}  %%use this if you ever want claims to be numbered
\newtheorem*{claim}{Claim}


%%
%% Page set-up:
%%
\pagestyle{empty}
\lhead{\textsc{201 - Real Analysis \\}} 
\rhead{\textsc{Harutyunyan, Winter 2020} \\ Trevor Klar}
%\chead{\Large\textbf{A New Integration Technique \\ }}
\renewcommand{\headrulewidth}{0pt}
%
\renewcommand{\footrulewidth}{0pt}
%\lfoot{
%Office: \quad \quad \, M 2-3 \, \, SH 6431x \\
%Math Lab: \, W 12-2 \, SH 1607
%}
%\rfoot{trevorklar@math.ucsb.edu}

\setlength{\parindent}{0in}
\setlength{\textwidth}{7in}
\setlength{\evensidemargin}{-0.25in}
\setlength{\oddsidemargin}{-0.25in}
\setlength{\parskip}{.5\baselineskip}
\setlength{\topmargin}{-0.5in}
\setlength{\textheight}{9in}

\setlist[enumerate,1]{label=\textbf{\arabic*.}}

\begin{document}
\pagestyle{fancy}
\begin{center}
{\Large Homework 1}%=================UPDATE THIS=================%
\end{center}

\jpg{width=\linewidth}{hw1-p1}
\begin{proof}$(\implies)$
Suppose $\{f_n\}$ is uniformly integrable, and let $\varepsilon>0$ be given. Then there exists $M$ such that the uniform integrability property holds for $f_n$ for all $n$. Let $\delta=\frac{\varepsilon}{M}$. Then for any \mumeasurable{} set $A$ with $\measure{A}<\delta$, 
\begin{align*}
\abs{\int_A f_n \der\mu}
&= \abs{{\int_{A\cap\{|f_n|>M\}}}\hspace*{-3em}f_n\hspace*{1em}\der\mu 
 +      {\int_{A\cap\{|f_n|\leq M\}}}\hspace*{-3em}f_n\hspace*{1em}\der\mu} \\
&\leq \abs{{\int_{A\cap\{|f_n|>M\}}}\hspace*{-3em}f_n\hspace*{1em}\der\mu} 
 +      \abs{{\int_{A\cap\{|f_n|\leq M\}}}\hspace*{-3em}f_n\hspace*{1em}\der\mu} \\
&\leq {\int_{A\cap\{|f_n|>M\}}}\hspace*{-3em}|f_n|\hspace*{1em}\der\mu 
 +      {\int_{A\cap\{|f_n|\leq M\}}}\hspace*{-3em}|f_n|\hspace*{1em}\der\mu \\
&< \varepsilon + M\mu(A)\\
&< \varepsilon + M\delta \\
&= 2\varepsilon
\end{align*}
and after rescaling and observing that $n$ was arbitrary, we've shown that $\{f_n\}$ is uniformly absolutely continuous. Now to see that $\ds\sup_n\int_X |f_n| \der\mu < \infty$, choose any $\varepsilon>0$ and let $M>0$ so that the uniform integrability property hold for all $f_n$. Then for all $n$,
\begin{align*}
\int_X |f_n|\der\mu 
&= {\int_{\{|f_n|>M\}}}\hspace*{-2.5em}|f_n|\hspace*{0em}\der\mu 
 +      {\int_{\{|f_n|\leq M\}}}\hspace*{-2.5em}|f_n|\hspace*{0em}\der\mu 
< \varepsilon + M\mu(X),\\
\end{align*}
which is finite and constant with respect to $n$, so the supremum over $n$ is finite as well. 
\end{proof}

\begin{proof}$(\impliedby)$
Let $\varepsilon>0$. Since $\{f_n\}$ is uniformly absolutely continuous, then there exists $\delta>0$ such that for all $A\subset X$ with $\mu(A)<\delta$, we have 
$$ \abs{\int_{A}f_n \der \mu}<\varepsilon \quad \forall n. $$
Let $M=\dfrac{1}{\delta}\sup\limits_n \ds\int_X |f_n|\der\mu$. Since 
$$ \mu\{|f_n|>M\} 
\leq \frac{1}{M}\int_{\{|f_n|>M\}} \hspace*{-2em}|f_n| \der\mu 
\leq \frac{1}{M}\sup_n\int_{X} |f_n| \der\mu 
= \delta,$$
then $ \abs{\ds\int_{\{|f_n|>M\}} \hspace*{-2em} f_n \der \mu}<\varepsilon $, so 
\begin{align*}
\int_{\{|f_n|>M\}} \hspace*{-2em} |f_n| \der \mu 
&= \int_{\{f_n>M\}} \hspace*{-2em} f_n^+ \der \mu + \int_{\{f_n<-M\}} \hspace*{-2.5em} f_n^- \der \mu \\
&= \abs{\int_{\{f_n>M\}} \hspace*{-2em} f_n \der \mu} + \abs{\int_{\{f_n<-M\}} \hspace*{-2.5em} f_n \der \mu} \\
&= \abs{\int_{\{|f_n|>M\}} \hspace*{-2.3em} f_n \der \mu} + \abs{\int_{\{|f_n|>M\}} \hspace*{-2.5em} f_n \der \mu} \\
&= 2\varepsilon
\end{align*}
\end{proof}

\pagebreak
\jpg{width=\linewidth}{hw1-p2}
%
\begin{proof}$(\implies)$
%Let $\varepsilon>0$. Since $f_n\to f$ \muae{}, then for all $\epsilon>0$, there exists $N\in\N$ such that for all $n>N$, $\abs{f_n-f}<\epsilon$. In particular, we can find $N$ such that $\abs{f_n-f}<\frac{\varepsilon}{\mu(X)}$ \muae{} in $X$ whenever $n>N$. Thus for all $n>N$, 
%\begin{align*}
%\int_X \abs{f_n-f}\der\mu &< \int_X \frac{\varepsilon}{\mu(X)}\der\mu \\
%&= \frac{\varepsilon}{\mu(X)} \mu(X) \\ 
%&= \varepsilon,
%\end{align*}
%so the limit as $n\to\infty$ is 0. 
%\end{proof}
Let $\varepsilon>0$. First we establish a few bounds.
\begin{enumerate}[label=(\roman*)]
\item Since $f$ is \musummable{}, there exists $\delta_1>0$ such that if $\mu(A)<\delta_1$, then $\abs{ \int_A f\der\mu}<\varepsilon_1$, where $\varepsilon_1=\frac{\varepsilon}{6}$.
\item Since $\{f_n\}$ is a uniformly integrable sequence, then it is uniformly absolutely continuous, so there exists $\delta_2>0$ such that if $\mu(A)<\delta_2$, then $\abs{\int_A f_n \der\mu}<\epsilon_1$ for all $n$. 
\item Since $\mu(X)<\infty$, then by Egoroff's Theorem there exists a set $A\subset X$ such that $f_n \to f$ uniformly on $A$, and $\mu(A^\complement)<\min(\delta_1, \delta_2)$. 
\item Since $f_n\to f$ uniformly on $A$, then there exists $N>0$ such that for all $n>N$, we have $\abs{f_n-f}<\varepsilon_2$, where $\varepsilon_2=\frac{\varepsilon}{3\mu(A)}$. 
\end{enumerate}
Now we apply the results above. For all $n>N$, 
\begin{align*}
\int_X|f_n-f|\der\mu 
&= \int_A|f_n-f|\der\mu + \int_{A^\complement}|f_n-f|\der\mu \\
&\leq \varepsilon_2\mu(A) + \int_{A^\complement}|f_n-f|\der\mu & \text{by (iv)}\\
&\leq \varepsilon_2\mu(A) + \int_{A^\complement}|f_n|\der\mu + \int_{A^\complement}|f|\der\mu & \Delta \text{ ineq.} \\
&= \varepsilon_2\mu(A) 
  + \left( \abs{\int_{A^\complement\cap\{f_n >   0\} }\hspace*{-2.5em}f_n\der\mu} +  \abs{\int_{A^\complement\cap\{f_n\leq 0\} }\hspace*{-2.5em}f_n\der\mu} \right) 
  + \left( \abs{\int_{A^\complement\cap\{f >   0\} }\hspace*{-2.5em}f\der\mu} +  \abs{\int_{A^\complement\cap\{f\leq 0\} }\hspace*{-2.5em}f\der\mu} \right) \\
%
&\leq \varepsilon_2\mu(A) 
  + \left( \abs{\int_{A^\complement\cap\{f_n >   0\} }\hspace*{-2.5em}f_n\der\mu} +  \abs{\int_{A^\complement\cap\{f_n\leq 0\} }\hspace*{-2.5em}f_n\der\mu} \right) 
  + 2\varepsilon_1
  & \text{by (i)} \\
&\leq \varepsilon_2\mu(A) + 2\varepsilon_1 + 2\varepsilon_1 & \text{by (ii)} \\
&= \frac{\varepsilon}{3} + \frac{\varepsilon}{3} + \frac{\varepsilon}{3} \\
&= \varepsilon
\end{align*}  
Thus $\int_X|f_n-f|\der\mu  \to 0$ as $n\to\infty$.
\end{proof}

\pagebreak
\begin{proof}$(\impliedby)$
We will show that (i) $\{f_n\}$ is uniformly absolutely continuous and (ii) ${\sup_n\int_X |f_n| \der\mu <\infty}$. 
\begin{enumerate}[label=(\roman*)]
\item Let $\varepsilon>0$. Since $\lim_{n\to\infty}\int_X \abs{f_n-f}\der\mu = 0$, then there exists $N>0$ such that if $n>N$, then ${\int_X\abs{f_n-f}\der\mu < \varepsilon}$. 

\textsc{Case I} ($n>N$): Suppose $n>N$. Since $f$ is \musummable{}, then there exists $\delta_0 >0$ such that for any $A\subset X$ with $\mu(A)<\delta_0 $, then $\int_A|f|\der\mu<\varepsilon$. Thus we find that 
$$
\varepsilon
>
\int_X\abs{f_n-f}\der\mu 
\geq
\int_A\abs{f_n-f}\der\mu 
\geq
\int_A|f_n|\der\mu-\int_A|f|\der\mu 
$$
and considering the left and right hand sides from above, we see
$$\abs{\int_A f_n \der \mu} \leq \int_A|f_n|\der\mu < \int_A|f|\der\mu + \varepsilon = 2\varepsilon.$$

\textsc{Case II} ($n\leq N$): For each $n = 1,\, \dots,\, N$ we know that $f_n$ is \musummable{}, so there exists $\delta_n$ such that if $\mu(A)<\delta_n$, then 
$$\abs{\int_A f_n \der\mu}<\varepsilon.$$ 

Thus we set $\delta = \min(\delta_0, \dots, \delta_N)$ and find that for all $n$, if $\mu(A)<\delta$, then $\abs{\int_A f_n\der\mu }<\varepsilon$. \qedwhite

\item Let $\varepsilon>0$. Since $\lim_{n\to\infty}\int_X\abs{f_n-f}\der\mu=0$, then there exists $N>0$ such that if $n>N$, then 
$$\varepsilon>\int_X\abs{f_n-f}\der\mu\geq \int_X\abs{f_n}\der\mu - \int_X\abs{f}\der\mu,$$ 
so $\int_X\abs{f_n}\der\mu < \int_X\abs{f}\der\mu + \epsilon$. Thus 
$$\sup_n\int_X|f_n|\der\mu \leq \max\left(\int_X\abs{f}\der\mu + \epsilon, \int_X|f_1|\der\mu, \dots, \int_X|f_N|\der\mu \right)<\infty.$$
\qedhere
\end{enumerate}
\end{proof}

\pagebreak
\jpg{width=\linewidth}{hw1-p3}





\end{document}
