\documentclass[12pt,letterpaper]{article}

\usepackage{fancyhdr,fancybox}
\usepackage{mathtools}

%% Useful packages
\usepackage{amssymb, amsmath, amsthm} 
%\usepackage{graphicx}  %%this is currently enabled in the default document, so it is commented out here. 
\usepackage{calrsfs}
\usepackage{braket}
\usepackage{mathtools}
\usepackage{lipsum}
\usepackage{tikz}
\usetikzlibrary{cd}
\usepackage{verbatim}
%\usepackage{ntheorem}% for theorem-like environments
\usepackage{mdframed}%can make highlighted boxes of text
%Use case: https://tex.stackexchange.com/questions/46828/how-to-highlight-important-parts-with-a-gray-background
\usepackage{wrapfig}
\usepackage{centernot}
\usepackage{subcaption}%\begin{subfigure}{0.5\textwidth}
\usepackage{pgfplots}
\pgfplotsset{compat=1.13}
\usepackage[colorinlistoftodos]{todonotes}
\usepackage[colorlinks=true, allcolors=blue]{hyperref}
\usepackage{xfrac}					%to make slanted fractions \sfrac{numerator}{denominator}
\usepackage{enumitem}            
    %syntax: \begin{enumerate}[label=(\alph*)]
    %possible arguments: f \alph*, \Alph*, \arabic*, \roman* and \Roman*
\usetikzlibrary{arrows,shapes.geometric,fit}

\DeclareMathAlphabet{\pazocal}{OMS}{zplm}{m}{n}
%% Use \pazocal{letter} to typeset a letter in the other kind 
%%  of math calligraphic font. 

%% This puts the QED block at the end of each proof, the way I like it. 
\renewenvironment{proof}{{\bfseries Proof}}{\qed}
\makeatletter
\renewenvironment{proof}[1][\bfseries \proofname]{\par
  \pushQED{\qed}%
  \normalfont \topsep6\p@\@plus6\p@\relax
  \trivlist
  %\itemindent\normalparindent
  \item[\hskip\labelsep
        \scshape
    #1\@addpunct{}]\ignorespaces
}{%
  \popQED\endtrivlist\@endpefalse
}
\makeatother

%% This adds a \rewnewtheorem command, which enables me to override the settings for theorems contained in this document.
\makeatletter
\def\renewtheorem#1{%
  \expandafter\let\csname#1\endcsname\relax
  \expandafter\let\csname c@#1\endcsname\relax
  \gdef\renewtheorem@envname{#1}
  \renewtheorem@secpar
}
\def\renewtheorem@secpar{\@ifnextchar[{\renewtheorem@numberedlike}{\renewtheorem@nonumberedlike}}
\def\renewtheorem@numberedlike[#1]#2{\newtheorem{\renewtheorem@envname}[#1]{#2}}
\def\renewtheorem@nonumberedlike#1{  
\def\renewtheorem@caption{#1}
\edef\renewtheorem@nowithin{\noexpand\newtheorem{\renewtheorem@envname}{\renewtheorem@caption}}
\renewtheorem@thirdpar
}
\def\renewtheorem@thirdpar{\@ifnextchar[{\renewtheorem@within}{\renewtheorem@nowithin}}
\def\renewtheorem@within[#1]{\renewtheorem@nowithin[#1]}
\makeatother

%% This makes theorems and definitions with names show up in bold, the way I like it. 
\makeatletter
\def\th@plain{%
  \thm@notefont{}% same as heading font
  \itshape % body font
}
\def\th@definition{%
  \thm@notefont{}% same as heading font
  \normalfont % body font
}
\makeatother

%===============================================
%==============Shortcut Commands================
%===============================================
\newcommand{\ds}{\displaystyle}
\newcommand{\B}{\mathcal{B}}
\newcommand{\C}{\mathbb{C}}
\newcommand{\F}{\mathbb{F}}
\newcommand{\N}{\mathbb{N}}
\newcommand{\R}{\mathbb{R}}
\newcommand{\Q}{\mathbb{Q}}
\newcommand{\T}{\mathcal{T}}
\newcommand{\Z}{\mathbb{Z}}
\renewcommand\qedsymbol{$\blacksquare$}
\newcommand{\qedwhite}{\hfill\ensuremath{\square}}
\newcommand*\conj[1]{\overline{#1}}
\newcommand*\closure[1]{\overline{#1}}
\newcommand*\mean[1]{\overline{#1}}
%\newcommand{\inner}[1]{\left< #1 \right>}
\newcommand{\inner}[2]{\left< #1, #2 \right>}
\newcommand{\powerset}[1]{\pazocal{P}(#1)}
%% Use \pazocal{letter} to typeset a letter in the other kind 
%%  of math calligraphic font. 
\newcommand{\cardinality}[1]{\left| #1 \right|}
\newcommand{\domain}[1]{\mathcal{D}(#1)}
\newcommand{\image}{\text{Im}}
\newcommand{\inv}[1]{#1^{-1}}
\newcommand{\preimage}[2]{#1^{-1}\left(#2\right)}
\newcommand{\script}[1]{\mathcal{#1}}


\newenvironment{highlight}{\begin{mdframed}[backgroundcolor=gray!20]}{\end{mdframed}}

\DeclarePairedDelimiter\ceil{\lceil}{\rceil}
\DeclarePairedDelimiter\floor{\lfloor}{\rfloor}

%===============================================
%===============My Tikz Commands================
%===============================================
\newcommand{\drawsquiggle}[1]{\draw[shift={(#1,0)}] (.005,.05) -- (-.005,.02) -- (.005,-.02) -- (-.005,-.05);}
\newcommand{\drawpoint}[2]{\draw[*-*] (#1,0.01) node[below, shift={(0,-.2)}] {#2};}
\newcommand{\drawopoint}[2]{\draw[o-o] (#1,0.01) node[below, shift={(0,-.2)}] {#2};}
\newcommand{\drawlpoint}[2]{\draw (#1,0.02) -- (#1,-0.02) node[below] {#2};}
\newcommand{\drawlbrack}[2]{\draw (#1+.01,0.02) --(#1,0.02) -- (#1,-0.02) -- (#1+.01,-0.02) node[below, shift={(-.01,0)}] {#2};}
\newcommand{\drawrbrack}[2]{\draw (#1-.01,0.02) --(#1,0.02) -- (#1,-0.02) -- (#1-.01,-0.02) node[below, shift={(+.01,0)}] {#2};}

%***********************************************
%**************Start of Document****************
%***********************************************
 %find me at /home/trevor/texmf/tex/latex/tskpreamble_nothms.tex
%===============================================
%===============Theorem Styles==================
%===============================================

%================Default Style==================
\theoremstyle{plain}% is the default. it sets the text in italic and adds extra space above and below the \newtheorems listed below it in the input. it is recommended for theorems, corollaries, lemmas, propositions, conjectures, criteria, and (possibly; depends on the subject area) algorithms.
\newtheorem{theorem}{Theorem}
\numberwithin{theorem}{section} %This sets the numbering system for theorems to number them down to the {argument} level. I have it set to number down to the {section} level right now.
\newtheorem*{theorem*}{Theorem} %Theorem with no numbering
\newtheorem{corollary}[theorem]{Corollary}
\newtheorem*{corollary*}{Corollary}
\newtheorem{conjecture}[theorem]{Conjecture}
\newtheorem{lemma}[theorem]{Lemma}
\newtheorem*{lemma*}{Lemma}
\newtheorem{proposition}[theorem]{Proposition}
\newtheorem*{proposition*}{Proposition}
\newtheorem{problemstatement}[theorem]{Problem Statement}


%==============Definition Style=================
\theoremstyle{definition}% adds extra space above and below, but sets the text in roman. it is recommended for definitions, conditions, problems, and examples; i've alse seen it used for exercises.
\newtheorem{definition}[theorem]{Definition}
\newtheorem*{definition*}{Definition}
\newtheorem{condition}[theorem]{Condition}
\newtheorem{problem}[theorem]{Problem}
\newtheorem{example}[theorem]{Example}
\newtheorem*{example*}{Example}
\newtheorem*{counterexample*}{Counterexample}
\newtheorem*{romantheorem*}{Theorem} %Theorem with no numbering
\newtheorem{exercise}{Exercise}
\numberwithin{exercise}{section}
\newtheorem{algorithm}[theorem]{Algorithm}

%================Remark Style===================
\theoremstyle{remark}% is set in roman, with no additional space above or below. it is recommended for remarks, notes, notation, claims, summaries, acknowledgments, cases, and conclusions.
\newtheorem{remark}[theorem]{Remark}
\newtheorem*{remark*}{Remark}
\newtheorem{notation}[theorem]{Notation}
\newtheorem*{notation*}{Notation}
%\newtheorem{claim}[theorem]{Claim}  %%use this if you ever want claims to be numbered
\newtheorem*{claim}{Claim}


%%
%% Page set-up:
%%
\pagestyle{empty}
\lhead{\textsc{201B - Real Analysis \\}} 
\rhead{\textsc{Harutyunyan, Winter 2020} \\ Trevor Klar}
%\chead{\Large\textbf{A New Integration Technique \\ }}
\renewcommand{\headrulewidth}{0pt}
%
\renewcommand{\footrulewidth}{0pt}
%\lfoot{
%Office: \quad \quad \, M 2-3 \, \, SH 6431x \\
%Math Lab: \, W 12-2 \, SH 1607
%}
%\rfoot{trevorklar@math.ucsb.edu}

\setlength{\parindent}{0in}
\setlength{\textwidth}{7in}
\setlength{\evensidemargin}{-0.25in}
\setlength{\oddsidemargin}{-0.25in}
\setlength{\parskip}{.5\baselineskip}
\setlength{\topmargin}{-0.5in}
\setlength{\textheight}{9in}

\setlist[enumerate,1]{label=\textbf{\arabic*.}}

\renewcommand{\phi}{\varphi}
\renewcommand{\epsilon}{\varepsilon}

\begin{document}
\pagestyle{fancy}
\begin{center}
{\Large Final Exam}%=================UPDATE THIS=================%
\end{center}

\begin{enumerate}

\item Let the sequences $\{a_n\}, \{r_n\}\subset \R$ be such that 
$$\sum_{i=1}^\infty |a_n|<\infty.$$
Prove that the series 
$$\sum_{i=1}^\infty \frac{a_n}{\sqrt{\abs{x-r_n}}}$$
converges absolutely for almost every $x\in\R$. 

\begin{proof}
Let $\{a_n\}, \{r_n\}$ be given as above, and let $g_n(x)=\frac{1}{\sqrt{\abs{x-r_n}}}$. We will show that the integral 
$$\int_{\alpha}^{\alpha+2} \sum_{n=1}^\infty {a_n}\,g_n(x)\dx$$
is finite over any region of length 2,\footnote{That is, for any $\alpha\in\R$.} which means the series is infinite on a set of measure 0.

First note that for any fixed $n$, 
$$\int_{\alpha}^{\alpha+2} g_n\dx = \int_{\alpha}^{\alpha+2} (x-r_n)^{-\sfrac{1}{2}} \dx= \frac{2(\alpha+2-r_n)}{\sqrt{\abs{\alpha+2-r_n}}} - \frac{2(\alpha-r)}{\sqrt{\abs{\alpha-r}}},$$
and by differentiating with respect to $\alpha$ we find that this value is greatest when 
$$0=\frac{1}{\sqrt{\abs{\alpha-r_n-2}}} - \frac{1}{\sqrt{\abs{\alpha-r_n}}},$$
which is to say that $\alpha-r_n=1$. Thus we conclude that 
$$\int_{\alpha}^{\alpha+2} g_n\dx\leq \int_{r_n-1}^{r_n+1} g_n\dx=\int_{-1}^1\frac{1}{\sqrt{t}}\dt =4 $$ for all $\alpha\in\R$ and all $ n\in\N$. 

To finish the proof, observe that 
$$\int_{\alpha}^{\alpha+2} \sum_{n=1}^\infty \frac{|a_n|}{\sqrt{\abs{x-r_n}}}\dx 
= \sum_{n=1}^\infty \int_{\alpha}^{\alpha+2} {|a_n|}\,g_n(x)\dx
= \sum_{n=1}^\infty {|a_n|} \int_{\alpha}^{\alpha+2}  g_n(x)\dx
%\leq \left(\sum_{n=1}^\infty {|a_n|}\right)(4)
\leq 4\sum_{n=1}^\infty {|a_n|}
$$
which is finite. Since the series is infinite only on a zero-measure subset of an arbitrary interval of length 2, then the union of all such subsets also has measure zero, and we're done. 
\end{proof}

\pagebreak
\item Suppose $f\in L^1[0,1]^2$ with respect to two dimensional Lebesgue measure $\mu$. Prove that if 
$$\int_{[0,a],[0,b]} f \der\mu = 0$$
for all $(a,b)\in [0,1]^2$, then $f=0$ \muae{} in $[0,1]^2$. 

\begin{proof}Since we can decompose $f$ as $f=f^+-f^-$, then 
$$f(x)=0\iff f^+(x)=f^-(x)=0$$ 
so \Wlog{} suppose $f$ is nonnegative.

($\int_R=0$) Observe that the integral over any rectangle
\begin{align*}
\int_{[a,b]\times[c,d]}  f\der\mu
&= \int_{[0,b]\times[0,d]}  f\der\mu 
	- \int_{[0,a]\times[0,d]}  f\der\mu 
	- \int_{[0,b]\times[0,c]}  f\der\mu 
	+ \int_{[0,a]\times[0,c]}  f\der\mu \\
&= 0-0-0+0.
\end{align*}
%and note that any two rectangles intersect in a rectangle, 

($\int_U=0$) Let $U$ be open. For each $n\in\N$, define a cover of $[0,1]^2$ by $2^{2n}$ squares of side length $2^{-n}$ and denote it $\{Q_n^i\}_{i=1}^\infty$. For any given $n$, there are finitely many $Q_n^i\subseteq U$, and so the union of all such is a countable union of cubes. To see that it covers $U$, let $(x,y)\in U$. Since $U$ is open, some neighborhood of $(x,y)$ is a subset of $U$, and certainly some sufficiently large $n$ gives a cube fully contained in that neighborhood which contains $n$\footnote{This can be made more rigorous, but this proof is getting absurdly long.}. Then 
$$\int_U f\der\mu = \sum_{i,n:Q_n^i\subseteq U} \int_{Q_n^i} f\der\mu = 0.$$

($\int_G=0$) Let $G$ be a $G_\delta$ set, so $G=\bigcap_{i=1}^\infty U_i$, where each $U_i$ is open. Then 
$$\int_G f\der\mu = 0$$
[I can't figure out how to prove this part.]

($\int_B=0$) Let $B$ be any Borel set in $[0,1]^2$. Since $\mu$ is Radon, then for each $n\in\N$ there exists an open set $U_n$ such that $B\subseteq U_n$ and $\mu(U_n\setminus B)<\frac{1}{n}$. Since $U_n$ is a decreasing sequence\footnote{Although $U_{n+1}$ is not necessarily a subset of $U_n$ by default, we know that they both contain $B$, and intersecting them yields $\tilde{U}_{n+1}=U_n \cap U_{n+1}$ so that $B\subset \tilde{U}_{n+1} \subset U_n$ and $\mu(\tilde{U}_{n+1}\setminus B)\leq \mu({U}_{n+1}\setminus B) < \frac{1}{n}$. Starting with $\tilde{U}_1=U_1$ and constructing the rest inductively yields the desired $\{\tilde{U}_n\}_{n=1}^\infty$. We drop the $\sim$ notation above.}
of sets with $\mu(U_1)\leq\mu[0,1]^2=1<\infty,$  then $\mu(U)=\lim\limits_n(\mu(U-n))$ where $U=\bigcap\limits_n U_n$. Thus 
\begin{align*}
B&\subset U, \\
\text{and } \mu(U\setminus B)&< \frac{1}{n} \; \forall n, \\
\text{so } \mu(U\setminus B)&=0.
\end{align*}
Therefore 
\begin{align*}
\int_B f\der\mu &= \int_U f\der\mu - \int_{U\setminus B} f\der\mu \\
&= \int_U f\der\mu \\
&=0 \text{ since }U\text{ is }G_\delta.
\end{align*}

($\int_A = 0$) Let $A=\{(x,y)\in[0,1]^2 : f(x,y)>0\}$, and suppose for contradiction that $\mu(A)>0$. This gives us that 
$$\int_A f \der\mu >0.$$
Since $\mu$ is Borel-regular, there exists a Borel set $B\supset A$ such that $\mu(B)=\mu(A)$. This means that $\mu(B\setminus A)=0$. Therefore 
\begin{align*}
\int_A f\der\mu &= \int_B f\der\mu - \int_{B\setminus A} f\der\mu \\
&= \int_B f\der\mu \\
&=0 \quad\quad\quad\quad\text{since $B$ is Borel.}
\end{align*}
This contradicts that $\int_A f\der\mu>0$, and so we conclude that $\mu(A)=0$. 
\end{proof}

\pagebreak
\item Let $A \subset X$ be a closed subspace of of a Banach space $X$ and let $k\in X$ be fixed. Is the distance 
$$\dist{k}{A}=\inf\{\norm{k-a} \;:\; a\in A\}$$
attained? 

\answer No. As a counterexample, let $X=\ell_\infty$, the space of all bounded sequences of real numbers, with norm $\norm{x}=\sup\limits_n \abs{x_n}$. Let 
\[\begin{array}{rcllll}
k&= &2, &2, &2, &2, \dots \\
a_1&= &0.9, &2, &2, &2, \dots \\ 
a_2&= &2, &0.99, &2, &2, \dots \\ 
a_3&= &2, &2, &0.999, &2, \dots \\ 
&\vdots
\end{array}\]
Observe that $A=\{a_n : n\in \N\}$ is closed, since it has no accumulation points: For any $n,m\in\N$, 
\begin{align*}
\norm{a_n-a_m}&=\norm{\big(0,0,\dots,0, (1+10^{-n}), 0, \dots, 0, (1+10^{-m}), 0, 0, \dots\big)} \\
&= \max\{(1+10^{-n}), (1+10^{-m})\} \\
&\geq1.
\end{align*}
Next note that for every $a_n$ the distance $\norm{k-a_n}=1+10^{-n}$, but
\begin{align*}
\dist{k}{A}
&=\inf\limits_n\norm{k-a_n} \\
&=\inf\limits_n\{1+10^{-n}\} \\
&=1,
\end{align*}
so $\dist{k}{A}$ is never attained. \qed
\end{enumerate}

\end{document}
