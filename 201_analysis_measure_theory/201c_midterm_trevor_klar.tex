 \documentclass[12pt,letterpaper]{article}

\usepackage{fancyhdr,fancybox,tensor}

%% Useful packages
\usepackage{amssymb, amsmath, amsthm} 
%\usepackage{graphicx}  %%this is currently enabled in the default document, so it is commented out here. 
\usepackage{calrsfs}
\usepackage{braket}
\usepackage{mathtools}
\usepackage{lipsum}
\usepackage{tikz}
\usetikzlibrary{cd}
\usepackage{verbatim}
%\usepackage{ntheorem}% for theorem-like environments
\usepackage{mdframed}%can make highlighted boxes of text
%Use case: https://tex.stackexchange.com/questions/46828/how-to-highlight-important-parts-with-a-gray-background
\usepackage{wrapfig}
\usepackage{centernot}
\usepackage{subcaption}%\begin{subfigure}{0.5\textwidth}
\usepackage{pgfplots}
\pgfplotsset{compat=1.13}
\usepackage[colorinlistoftodos]{todonotes}
\usepackage[colorlinks=true, allcolors=blue]{hyperref}
\usepackage{xfrac}					%to make slanted fractions \sfrac{numerator}{denominator}
\usepackage{enumitem}            
    %syntax: \begin{enumerate}[label=(\alph*)]
    %possible arguments: f \alph*, \Alph*, \arabic*, \roman* and \Roman*
\usetikzlibrary{arrows,shapes.geometric,fit}

\DeclareMathAlphabet{\pazocal}{OMS}{zplm}{m}{n}
%% Use \pazocal{letter} to typeset a letter in the other kind 
%%  of math calligraphic font. 

%% This puts the QED block at the end of each proof, the way I like it. 
\renewenvironment{proof}{{\bfseries Proof}}{\qed}
\makeatletter
\renewenvironment{proof}[1][\bfseries \proofname]{\par
  \pushQED{\qed}%
  \normalfont \topsep6\p@\@plus6\p@\relax
  \trivlist
  %\itemindent\normalparindent
  \item[\hskip\labelsep
        \scshape
    #1\@addpunct{}]\ignorespaces
}{%
  \popQED\endtrivlist\@endpefalse
}
\makeatother

%% This adds a \rewnewtheorem command, which enables me to override the settings for theorems contained in this document.
\makeatletter
\def\renewtheorem#1{%
  \expandafter\let\csname#1\endcsname\relax
  \expandafter\let\csname c@#1\endcsname\relax
  \gdef\renewtheorem@envname{#1}
  \renewtheorem@secpar
}
\def\renewtheorem@secpar{\@ifnextchar[{\renewtheorem@numberedlike}{\renewtheorem@nonumberedlike}}
\def\renewtheorem@numberedlike[#1]#2{\newtheorem{\renewtheorem@envname}[#1]{#2}}
\def\renewtheorem@nonumberedlike#1{  
\def\renewtheorem@caption{#1}
\edef\renewtheorem@nowithin{\noexpand\newtheorem{\renewtheorem@envname}{\renewtheorem@caption}}
\renewtheorem@thirdpar
}
\def\renewtheorem@thirdpar{\@ifnextchar[{\renewtheorem@within}{\renewtheorem@nowithin}}
\def\renewtheorem@within[#1]{\renewtheorem@nowithin[#1]}
\makeatother

%% This makes theorems and definitions with names show up in bold, the way I like it. 
\makeatletter
\def\th@plain{%
  \thm@notefont{}% same as heading font
  \itshape % body font
}
\def\th@definition{%
  \thm@notefont{}% same as heading font
  \normalfont % body font
}
\makeatother

%===============================================
%==============Shortcut Commands================
%===============================================
\newcommand{\ds}{\displaystyle}
\newcommand{\B}{\mathcal{B}}
\newcommand{\C}{\mathbb{C}}
\newcommand{\F}{\mathbb{F}}
\newcommand{\N}{\mathbb{N}}
\newcommand{\R}{\mathbb{R}}
\newcommand{\Q}{\mathbb{Q}}
\newcommand{\T}{\mathcal{T}}
\newcommand{\Z}{\mathbb{Z}}
\renewcommand\qedsymbol{$\blacksquare$}
\newcommand{\qedwhite}{\hfill\ensuremath{\square}}
\newcommand*\conj[1]{\overline{#1}}
\newcommand*\closure[1]{\overline{#1}}
\newcommand*\mean[1]{\overline{#1}}
%\newcommand{\inner}[1]{\left< #1 \right>}
\newcommand{\inner}[2]{\left< #1, #2 \right>}
\newcommand{\powerset}[1]{\pazocal{P}(#1)}
%% Use \pazocal{letter} to typeset a letter in the other kind 
%%  of math calligraphic font. 
\newcommand{\cardinality}[1]{\left| #1 \right|}
\newcommand{\domain}[1]{\mathcal{D}(#1)}
\newcommand{\image}{\text{Im}}
\newcommand{\inv}[1]{#1^{-1}}
\newcommand{\preimage}[2]{#1^{-1}\left(#2\right)}
\newcommand{\script}[1]{\mathcal{#1}}


\newenvironment{highlight}{\begin{mdframed}[backgroundcolor=gray!20]}{\end{mdframed}}

\DeclarePairedDelimiter\ceil{\lceil}{\rceil}
\DeclarePairedDelimiter\floor{\lfloor}{\rfloor}

%===============================================
%===============My Tikz Commands================
%===============================================
\newcommand{\drawsquiggle}[1]{\draw[shift={(#1,0)}] (.005,.05) -- (-.005,.02) -- (.005,-.02) -- (-.005,-.05);}
\newcommand{\drawpoint}[2]{\draw[*-*] (#1,0.01) node[below, shift={(0,-.2)}] {#2};}
\newcommand{\drawopoint}[2]{\draw[o-o] (#1,0.01) node[below, shift={(0,-.2)}] {#2};}
\newcommand{\drawlpoint}[2]{\draw (#1,0.02) -- (#1,-0.02) node[below] {#2};}
\newcommand{\drawlbrack}[2]{\draw (#1+.01,0.02) --(#1,0.02) -- (#1,-0.02) -- (#1+.01,-0.02) node[below, shift={(-.01,0)}] {#2};}
\newcommand{\drawrbrack}[2]{\draw (#1-.01,0.02) --(#1,0.02) -- (#1,-0.02) -- (#1-.01,-0.02) node[below, shift={(+.01,0)}] {#2};}

%***********************************************
%**************Start of Document****************
%***********************************************
 %find me at /home/trevor/texmf/tex/latex/tskpreamble_nothms.tex
%===============================================
%===============Theorem Styles==================
%===============================================

%================Default Style==================
\theoremstyle{plain}% is the default. it sets the text in italic and adds extra space above and below the \newtheorems listed below it in the input. it is recommended for theorems, corollaries, lemmas, propositions, conjectures, criteria, and (possibly; depends on the subject area) algorithms.
\newtheorem{theorem}{Theorem}
\numberwithin{theorem}{section} %This sets the numbering system for theorems to number them down to the {argument} level. I have it set to number down to the {section} level right now.
\newtheorem*{theorem*}{Theorem} %Theorem with no numbering
\newtheorem{corollary}[theorem]{Corollary}
\newtheorem*{corollary*}{Corollary}
\newtheorem{conjecture}[theorem]{Conjecture}
\newtheorem{lemma}[theorem]{Lemma}
\newtheorem*{lemma*}{Lemma}
\newtheorem{proposition}[theorem]{Proposition}
\newtheorem*{proposition*}{Proposition}
\newtheorem{problemstatement}[theorem]{Problem Statement}


%==============Definition Style=================
\theoremstyle{definition}% adds extra space above and below, but sets the text in roman. it is recommended for definitions, conditions, problems, and examples; i've alse seen it used for exercises.
\newtheorem{definition}[theorem]{Definition}
\newtheorem*{definition*}{Definition}
\newtheorem{condition}[theorem]{Condition}
\newtheorem{problem}[theorem]{Problem}
\newtheorem{example}[theorem]{Example}
\newtheorem*{example*}{Example}
\newtheorem*{counterexample*}{Counterexample}
\newtheorem*{romantheorem*}{Theorem} %Theorem with no numbering
\newtheorem{exercise}{Exercise}
\numberwithin{exercise}{section}
\newtheorem{algorithm}[theorem]{Algorithm}

%================Remark Style===================
\theoremstyle{remark}% is set in roman, with no additional space above or below. it is recommended for remarks, notes, notation, claims, summaries, acknowledgments, cases, and conclusions.
\newtheorem{remark}[theorem]{Remark}
\newtheorem*{remark*}{Remark}
\newtheorem{notation}[theorem]{Notation}
\newtheorem*{notation*}{Notation}
%\newtheorem{claim}[theorem]{Claim}  %%use this if you ever want claims to be numbered
\newtheorem*{claim}{Claim}


%%
%% Page set-up:
%%
\pagestyle{empty}
\lhead{\textsc{201c - Functional Analysis} \\Quarter of COVID-19} 
\rhead{\textsc{Labutin, Spring 2020} \\ Trevor Klar}
%\chead{\Large\textbf{A New Integration Technique \\ }}
\renewcommand{\headrulewidth}{0pt}
%
\renewcommand{\footrulewidth}{0pt}
%\lfoot{
%Office: \quad \quad \, M 2-3 \, \, SH 6431x \\
%Math Lab: \, W 12-2 \, SH 1607
%}
%\rfoot{trevorklar@math.ucsb.edu}

\setlength{\parindent}{0in}
\setlength{\textwidth}{7in}
\setlength{\evensidemargin}{-0.25in}
\setlength{\oddsidemargin}{-0.25in}
\setlength{\parskip}{.5\baselineskip}
\setlength{\topmargin}{-0.5in}
\setlength{\textheight}{9in}

\setlist[enumerate,1]{label=\textbf{\arabic*.}}

\let\oldphi\phi
\renewcommand{\phi}{\varphi}
\renewcommand{\epsilon}{\varepsilon}

\begin{document}
\pagestyle{fancy}
\begin{center}
{\Large Midterm Exam}%=================UPDATE THIS=================%
\end{center}

\renewcommand{\B}{\bar{B}(\ell^\infty)}

\begin{enumerate}

\item Let $\epsilon>0$. Prove that there exists a sequence $(x_n)$ of real numbers such that 
$$\sum_{n=1}^\infty x_n^2 < \infty \text{ but } n^\epsilon x_n \not \to 0.$$

\begin{proof}
Consider the sequence $\braces{n^\epsilon}$. For all $k\in\N$, there exists some $n_k\in\N$ such that 
$$n_k^\epsilon>k^2$$
so let $\braces{n_k^\epsilon}$ be that subsequence. For the same indeces $n_k$, Let 
$$x_{n_k}=k^{-1}$$
and for all other $n$, let 
$$x_n=n^{-1}.$$
Then 
\begin{align*}
\sum_{n=1}^\infty x_n^2 &= \sum_{k=1}^\infty x_{n_k}^2 + \sum_{n\not\in\{n_k\}} x_n^2 \\
&<\sum_{k=1}^\infty k^{-2} + \sum_{n\in\N} n^{-2} \\
&<\infty,
\end{align*}
but $n^\epsilon x_n \not\to0$, since the subsequence indexed by $n_k$ is 
\begin{align*}
n_k^\epsilon x_{n_k} &= n_k^\epsilon k^{-1}\\
&> k^2 k^{-1} \\
&=k
\end{align*}
and this goes to infinity as $k\to\infty$. 
\end{proof}

\pagebreak
\item 
	\begin{enumerate}
	\item Let $X$ be a normed. Assume $X^*$ is separable. Prove that $X$ is separable. 
	
	\begin{proof}
	Since $X^*$ is separable, then so is 
	$$S(X^*)=\{\phi\in X^* \;:\; \norm{\phi}=1\},$$
	so let $\braces{\phi_n}$ be a countable dense subset of $S(X^*)$. Since for each $n$ we have that 
	$$\sup_{\norm{x}=1} \abs{\angles{\phi_n,x}}=\norm{\phi_n}=1,$$
	then for each $\phi_n$ we can choose some $x_n\in X$ with $\norm{x_n}=1$ such that 
	\begin{equation}
	\abs{\angles{\phi_n,x_n}}>\frac{1}{2}.\tag{$\dagger$}
	\end{equation}
	Let 
	\begin{align*}
	D&=\text{span}\{x_n\}\\
	&=\braces{\sum_{j=1}^n r_jx_j \;:\; r_j\in\R, n\in\N}\phantom{.}
	\end{align*}
	and denote $A$ as the set of all finite linear combinations of $\{x_n\}$ with rational coefficients, that is, 
	$$A=\braces{\sum_{j=1}^n q_jx_j \;:\; q_j\in\Q, n\in\N}.$$
	Then $A$ is countable since $\Q\times\N$ is countable, and (as we will show presently) it is dense in $D$. Let $\sum_{j=1}^n r_jx_j\in D$, and let $\epsilon>0$. For each $j$, we can find some $q_j\in\Q$ such that $\abs{r_j-q_j}<\frac{\epsilon}{n\norm{x_j}}$, 
	$$\norm{r_jx_j-q_jx_j}=\abs{r_j-q_j}\norm{x_j}<\frac{\epsilon}{n},$$
	so by triangle inequality, 
	$$\norm{\sum_j^nr_jx_j-\sum_j^nq_jx_j}=\norm{\sum_j^n(r_jx_j-q_jx_j)}\leq \sum_j^n\norm{r_jx_j-q_jx_j} < \epsilon.$$	
%	 since we can find 
%	\begin{alignat*}{2}
%	q_jx&\xto{j}rx  &&\forall rx\in\text{span}\braces{x_n}, \text{ and } \\
%	\sum_{j=1}^n r_jx&\xto{n}\sum_{j=1}^\infty r_jx \quad &&\forall \sum_{j=1}^\infty r_jx \in \text{span}\braces{x_n}.
%	\end{alignat*}
	
	
	
	
	
	Now we will show that $D$ is dense in $X$. 
	
	Suppose for contradiction that 	$\closure{D}\neq X$. Since $\closure{D}$ is the span of vectors in $X$, then it is a linear subspace of $X$, and so by Hahn-Banach we can find some $\psi\in X^*$ such that $\psi|_{\closure{D}}=0$. Since $\braces{\phi_n}$ is dense in $S(X^*)$, then we can find a particular $\phi_n$ such that 
	$$\norm{\psi-\phi_n}_*<\frac{1}{4}.$$
	Now since every $x_n\in\closure{D}$ with $\norm{x_n}=1$, we have that $\angles{\psi,x_n}=0$, so by applying ($\dagger$) and the equation above, 
	$$\frac{1}{2}< \abs{\angles{\phi_n,x_n}} = \abs{\angles{\phi_n,x_n}-\angles{\psi,x_n}}\leq\norm{\phi_n-\psi}_*\norm{x_n} < \frac{1}{4}$$
	which is a contradiction. Thus $A$ is a countable set, and $A$ is dense in $D$ which is dense in $X$, so $A$ is dense in $X$, and we're done. 
	\qedwhitehere
	\end{proof}
	
	\item Give an example of a separable $X_0$ such that $X_0^*$ is not separable. Prove the separability and non-separability of your
example.

	\answer $\ell_1$ is separable, and $\ell_1^*=\ell_\infty$ is not separable. 
	\begin{proof}[Proof ($\ell_1$ is separable)]
	let $A$ be the set of all finite rational sequences, that is, 
	$$A=\braces{\sum_{j=1}^n q_je_j \;:\; q_j\in\Q, n\in\N}.$$
	Then $A$ is countable since $\Q\times\N$ is countable, and it is dense in $\ell_1$ since we can use a similar strategy as in the previous problem to find, for any $\epsilon>0$, $q_j\in \Q$ such that 
%	\begin{alignat*}{2}
	$$\norm{q_je_j-re_j}<\epsilon2^{-j}   $$
	for all elements of the form $ re_j$ in $\ell_1$, and so
	$$\norm{\sum_{j=1}^\infty q_je_j-\sum_{j=1}^\infty r_je_j}<\epsilon.$$
	 by triangle inequality for all $\sum_{j=1}^\infty r_je_j \in \ell_1$.
%	\end{alignat*}
	\qedwhitehere
	\end{proof}
	\begin{proof}[Proof ($\ell_\infty$ is not separable)]
	Thinking of elements of $\ell_\infty$ as functions $\N\to\R$, consider the power set $\script{P}(\N)$. We can define a subset $\script{G}\subset\ell_\infty$ given by 
	$$\script{G}=\braces{\Chi_G(n) : G\in \script{P}(\N)},$$
	and we see that this set is uncountable, and no two sequences in $\script{G}$ are closer than 1, since they all differ by 1 at some $n$. Any countable subset of $\ell_\infty$ can only be within distance $1/2$ of at most countable many elements of $\script{G}$, so it cannot be dense. 
	\end{proof}
	\end{enumerate}
	
	
\item Let $X,Y$ be normed spaces, $A:X\to Y$ be (algebraically) linear. Assume that for any sequence $(x_n)$ in $X$ such that $x_n\to0$ weakly the corresponding sequence $Ax_n\to 0$ weakly in $Y$. Prove that $A$ is a bounded operator.
\begin{proof}
Let $x_n\to 0$ strongly. Then $x_n\xto{w}0$, so $Ax_n\xto{w}0$. We can write the sequence $(x_n)$ as $(\lambda_n u_n)$, where the scalars $|\lambda_n|\to0$ and every vector $\norm{u_n}=1$. Then since $Ax_n\xto{w}0$, then for all $\psi\in Y^*$ we have $\psi(Ax_n)\to 0$, so 
\begin{align*}
\psi(Ax_n)
&= \psi \circ	A(\lambda_nu_n) \\
&= \lambda_n \psi(Au_n) &\text{by linearity}\\
&\xto{n}0.
\end{align*}
Suppose for contradiction that there is some $(u_n)$ such that $|\psi(Au_n)|\to\infty$. Then $\frac{1}{|\psi(Au_n)|}\to0$, and we know that $\lambda_n \psi(Au_n) \xto{n}0$ for every sequence $(\lambda_n)\to0$. But $\frac{|\psi(Au_n)|}{|\psi(Au_n)|}\equiv1$, which is a contradiction.

Thus ${|\psi(Au_n)|}\to B < \infty$ for any $(u_n)$ with every $\norm{u_n}=1$, so we're done. To see this, note that the sequence ${|\psi(Au_n)|}$ is bounded for any $\psi\in Y^*$ and any sequence in the unit ball $(u_n)\subset \closure{B}(X)$, which means $Au_n$ is also bounded. Therefore $\sup\limits_{\norm{u}=1}\norm{Au}<\infty$. 
\end{proof}

\pagebreak
\item Prove that $X^*$ "separates points" of $X$ (a Banach space). That is, prove that for all $x,y\in X$ such that $x\neq y$, there exists $\phi\in X^*$ such that $\phi(x)\neq\phi(y)$. 
\begin{proof}
 Fix $x\neq y\in X$, and on the linear subspace 
$$\text{span}(y-x),$$
define a linear functional $\phi$ by 
$$\phi(y-x)=1 \text{ and extending linearly.}$$
Then by Hahn-Banach, we can extend $\phi$ to a functional on all of $X$. Then $\phi(y)=\phi(x)+1$ and we're done.
\end{proof}


\end{enumerate}



\end{document}
