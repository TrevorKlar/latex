 




\documentclass[12pt]{article}

\usepackage{fullpage}
\usepackage{graphics}
\usepackage{tikz}
\usepackage[parfill]{parskip}
\usepackage[utf8]{inputenc}
\usepackage{etoolbox}

\newbool{alt}
\booltrue{alt}
%\boolfalse{alt}

% Standard things to include for math
\usepackage{amsmath,amssymb,amsfonts,amsthm}

\usepackage{wasysym}

\usepackage{enumitem}



\usepackage{geometry}
 \geometry{
 a4paper,
 top=2cm,
 bottom=2cm,
 left=2cm,
 right=2cm,
 }




% Some of Ebrahim's definitions
\newcommand{\done}{\\\hspace*{0pt}\hfill$\blacksquare$}
\def\N{\mathbb{N}}
\def\R{\mathbb{R}}
\def\Q{\mathbb{Q}}
\def\Z{\mathbb{Z}}
\def\e{\epsilon}
\newcommand{\seq}[1]{\left(#1\right)_{n\in\N}}
\newcommand{\Euc}[1]{\mathbb{R}^{#1}}
\newcommand{\pathvarNaked}{p}
\newcommand{\pathvar}{\vec{\pathvarNaked}}
\newcommand{\pathvarAlt}{\vec{q}}
\newcommand{\exercisesList}[2]{\textcolor{ForestGreen}{Exercises from WeBWorK HW#1: #2.}}
\newcommand{\exerciseText}[1]{\textcolor{ForestGreen}{Exercise: #1}}
\def\vx{\vec{x}}
\def\vu{\vec{u}}
\def\vv{\vec{v}}
\def\vw{\vec{w}}
\newcommand{\der}[2]{\frac{\textrm{d}#1}{\textrm{d}#2}}
\newcommand{\derOp}[1]{\der{\phantom{#1}}{#1}}
\newcommand{\pder}[2]{\frac{\partial #1}{\partial #2}}
\newcommand{\pderOp}[1]{\pder{\phantom{#1}}{#1}}
\newcommand{\colvectwo}[2]{\left[\begin{array}{c}#1\\#2\end{array}\right]}
\newcommand{\colvectwoXYVEC}[2]{\left[#1\right]\,\cbv{x} + \left[#2\right]\,\cbv{y}}
\newcommand{\colvecthree}[3]{\left[\begin{array}{c}#1\\#2\\#3\end{array}\right]}
\newcommand{\colvecfour}[4]{\left[\begin{array}{c}#1\\#2\\#3\\#4\end{array}\right]}
\newcommand{\colvecfourL}[4]{\left[\begin{array}{l}#1\\#2\\#3\\#4\end{array}\right]}
\newcommand{\norm}[1]{\left|\hspace{-1.5pt}\left|#1\right|\hspace{-1.5pt}\right|}
\def\grad{\vec{\nabla}}
\newcommand{\Dir}[3]{\operatorname{Dir}(#1,#2,#3)}
\newcommand{\D}[1]{\textrm{D}#1}
\def\vn{\vec{n}}
\newcommand{\cbv}[1]{\partial_{#1}}
\newcommand{\arrayBrackets}[2]{\left[ \begin{array}{#1} #2 \end{array} \right] }

\makeatletter
\newcommand*\dotp{\mathpalette\bigcdot@{.5}}
\newcommand*\bigcdot@[2]{\mathbin{\vcenter{\hbox{\scalebox{#2}{$\m@th#1\bullet$}}}}}
\makeatother



\newcommand{\ansbox}[2]{\raisebox{-.5\height}{\framebox(#1,#2){}}}

\def\endans{\hspace{1em}\ansbox{40}{40}}


\newcommand{\NEcheckbox}{ % Put check box on northeast corner of page
\begin{tikzpicture}[remember picture,overlay]
\path (current page.north east) ++(-1,-1) node[below left] {
{\small graded?} {\Large\Square}
};
\end{tikzpicture}
}

\newcommand{\LEFTcheckbox}{ % Put check box in the left margin
\\\begin{tikzpicture}[remember picture,overlay]
\path ++(-2,0) node[below left] {
 {\Large\Square}
};
\end{tikzpicture}
}

\newcommand{\LEFTcheckboxOwnLine}{ % Put check box in the left margin, use this one if on own line
\begin{tikzpicture}[remember picture,overlay]
\path ++(-2,0) node[below left] {
 {\Large\Square}
};
\end{tikzpicture}
}






\pagenumbering{gobble}
\begin{document}




% --- Score table ---

%\def\gap{\hspace*{2.5em}}
%
%\begin{tikzpicture}[overlay, remember picture]
%\path (current page.north east) ++(-1,-1) node[below left] {
%\begin{tabular}{c|c|c|c|c|c|c|c}
% 1 &  2 & 3 & 4 & 5 & 6 & 7 & $\Sigma$\\ \hline
% \gap & \gap & \gap & \gap & \gap & \gap & \gap & \gap \\
% \gap & \gap & \gap & \gap & \gap & \gap & \gap & \gap
%\end{tabular}
%};
%\end{tikzpicture}

% ---


\def\namepos{(current page.north west) ++(5,-0.8)}
\begin{tikzpicture}[overlay, remember picture]
\draw [black] \namepos rectangle ++(10,-1);
\draw [black] \namepos ++(0,-1.2) rectangle ++(10,-1);
%\draw [black] \namepos ++(0,-1.2) ++(0,-1.2) rectangle ++(10,-1);
%\draw [black] \namepos ++(0,-1.2) ++(0,-1.2) ++(0,-1.2) rectangle ++(10,-1);
\path \namepos ++(0,-0.4) node [left] {\textbf{Name:}};
\path \namepos ++(0,-0.4) ++ (0,-1.2) node [left] {\textbf{Perm:}};
%\path \namepos ++(0,-0.4) ++ (0,-1.2) ++ (0,-1.2) node [left] {\textbf{Seat Number:}};
%\path \namepos ++(0,-0.4) ++ (0,-1.2) ++ (0,-1.2) ++ (0,-1.2) node [left] {\textbf{ID Checker Name:}};
\end{tikzpicture}

\def\versionRowText{
\textit{version \ifbool{alt}{1}{2},row}
\raisebox{-3pt}{\framebox(15,15){A}}
}
\begin{tikzpicture}[overlay, remember picture]
\path (current page.north east) ++ (-2,-1.5) node [left]{}; %{\versionRowText};
\end{tikzpicture}



\vspace{5em}

\begin{center}\textbf{Math 34A Midterm 2}

({\it 100 pts total})
\end{center}
\begin{enumerate}


%\newpage
%%%%%%%%%%% (1)
\newpage
\item ({\it 6 pts}) Use the log table provided with this exam to find

%
%\begin{tikzpicture}
% %grid
%  \draw[step=1cm,black!70,very thin] (0,0) grid (10,10);
%  %axes
%  \draw[very thick,->] (0,0) -- (10,0) node[anchor= west] {\bf{$x$}};
%  \draw[very thick,->] (0,0) -- (0,10) node[anchor=south] {\bf{$y$}};
%  \foreach \x in {.2,.4,.6,.8,1}
%    \draw (10*\x cm,1pt) -- (10*\x cm,-1pt) node[anchor=north] {$\mathbf{\x}$};
%  \foreach \y in {2,4,6,8,10}
%    \draw (1pt,\y cm) -- (-1pt,\y cm) node[anchor=east] {$\mathbf{\y}$};
%   %Function Name
%   \node[black,fill=white] at (2,8.2)
%   {{\large $y=10^x$}};
%   %function
%   \draw[scale=1.0,domain=0:10,smooth,variable=\x,blue,very thick] plot (\x,{10^(.1*\x)})
%   %Code from previous exam below:
%   %\draw[thick,black] plot[domain=0:10] (\x,{10^(.1*\x)})
%   ;
% \end{tikzpicture}
\begin{enumerate}

\item  ({\it 4 pts}) $10^{1.820}$
\vfill

\hfill$ \ $ \ansbox{200}{70}

\item  ({\it 8 pts}) $\log(9.973/8.980)$
\vfill

\hfill$ \ $ \ansbox{200}{70}

\item  ({\it 12 pts}) $\log(\sqrt[5]{537})$
\vfill

\hfill$ \ $ \ansbox{200}{70}
\end{enumerate}


\pagebreak
%%%%%%%%%%% (2)
\item ({\it 8 pts}) Use properties of logs to solve for $x$. Use the log table to evaluate any logs in your answer and simplify completely.
$$10^{2x-3} = 2^{10} $$
\vfill

 \hfill$ \ x= \ $ \ansbox{200}{70}

%%%%%%%%%%% (3)
\item
\begin{enumerate}
\item ({\it 4 pts}) Find the equation of the line that passes through the points $(-4,3)$ and $(-2,5)$. Give the answer in the form $y=mx + b$.
\vfill

 \hfill$ \ y= \ $ \ansbox{200}{70}

\item ({\it 4 pts}) Find the equation of the line that has slope $\frac{3}{5}$ and goes through the point $(4,10)$. Give the answer in the form $y=mx + b$.
\vfill

 \hfill$ \ y= \ $ \ansbox{200}{70}

\end{enumerate}
\end{enumerate}

\pagebreak

\begin{enumerate} \setcounter{enumi}{2}
\item
\begin{enumerate} \setcounter{enumii}{2}

\item ({\it 4 pts}) What are the coordinates of the point where the lines $y=x-8$ and $y=1-\frac 1 2 x$ intersect?
\vfill

 \hfill$ \ (x,y)= \ $ \ansbox{200}{70}
\end{enumerate}


%%%%%%%%%%% (4)
\item ({\it 8 pts}) Initially, can A contains 8 liters of red paint and can B contains 16 liters of blue paint. I pour half of the red paint into can B. After mixing the paint in can B, I pour half of the paint in can B into can A.




How many liters of blue paint are now in can A?
\vfill

 \hfill \ansbox{200}{70} L


\newpage
%%%%%%%%%%% (5)

\item ({\it 12 pts})
The distance traveled by a rising helium balloon $t$ seconds after leaving a child's hand is modeled by $f(t) =t^2$ meters.
\begin{enumerate}
\item ({\it 4 pts})
Find the average speed of the balloon over the time period from 1 second to $1.2$ seconds.
\vfill

\hfill\ansbox{120}{70} \ $m/s$


\item ({\it 4 pts})
Find the average speed of the balloon over the time period from 1 second to $1+h$ seconds.
\vfill


\hfill\ansbox{200}{70} \ $m/s$


\item ({\it 4 pts})
In your own words, what would we do to find the instantaneous speed of the balloon exactly 1 second after it was dropped?



\hfill\ansbox{420}{70}


\end{enumerate}

\pagebreak
%%%%%%%%%%% (6)
\item  ({\it 8 pts}) An enterprising 34A student observes that many students forgot to bring 3x5 note cards with them to Midterm 1. She has plenty of packages of note cards lying around, so she decides to sell them to students all over campus. If she charges 10 cents a card, she will sell 500 cards. But for each cent she increases the price, the number of cards she sells will decrease by 10.
\begin{enumerate}
\item ({\it 4 pts}) If she picks a price of $(10 + h)\cent$ per card, find the number of cards she will sell (in terms of $h$).
\vfill

 \hfill  \ansbox{150}{70} cards
\item ({\it 4 pts}) Use your answer from part (a) to find the total amount of money (in cents) she will receive from selling cards if her price is $(10 + h)\cent$ each. Please simplify your answer (it should be in terms of $h$).


\vfill

\hfill
\ansbox{200}{70} \ {\Large \textcent}
\end{enumerate}



%%%%%%%%%%% (7)
\newpage

\item ({\it 8 pts}) Compute the following sum (your answer should be a number).
$$\sum_{n=4}^6 \frac{n(n+1)}{2}$$


 \hfill\ansbox{140}{70}
\vfill

%%%%%%%%%%% (8)
%\newpage
\item ({\it 8 pts}) Find {\large $\underset{h \rightarrow 0}{\lim}$} \ \ {\large $\frac{48 + 24h + 3h^2 - 48}{h}$}
\vfill

\hfill\ansbox{140}{70}




%\end{enumerate}

\newpage
%%%%%%%%%%% (9)
\item ({\it 12 pts}) A square garden consists of a semicircular pond and the rest is a lawn. The length of each side of the square garden is $\ell$. If the area of the square is 900 $m^2$, then find the area of the lawn.



{\includegraphics[width=4in]{Midterm_2_Pond_Screenshot.png} }
\vfill

\hfill lawn area$ \ = \ $ \ansbox{200}{70}$\ m^2$
















\end{enumerate}



\end{document}
