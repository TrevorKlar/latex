\documentclass{article}
\usepackage{graphicx}
\usepackage{amsmath}
\usepackage{amssymb}
\usepackage{tabto}
\usepackage{amsfonts}
%\usepackage{MnSymbol}
\usepackage{wasysym}
\usepackage{amsthm}
\usepackage{indentfirst}
\usepackage[utf8x]{inputenc}
\usepackage{caption}
\usepackage{subcaption}
\usepackage{adjustbox}
\usepackage{verbatim}
\usepackage{tikz, pgfplots}
\usepackage{tkz-euclide}
\usepgfplotslibrary{fillbetween}
\usepackage{multicol}
\usepackage{enumitem}
\usepackage{relsize}
\usepackage{xfrac}
\usepackage{array}
\usepackage{enumitem}
\usepackage{graphicx}
\usepackage[english]{babel}
\usepackage{fancyhdr}

\newtheorem{theorem}{Theorem}[section]
\newtheorem{corollary}{Corollary}[theorem]
\newtheorem{lemma}[theorem]{Lemma}

\graphicspath{ {images/} }


\renewcommand{\v}{\vec{v}}
\renewcommand{\u}{\vec{u}}
\renewcommand{\w}{\vec{w}}


\setlength\parindent{0pt}
\addtolength{\oddsidemargin}{-.875in}
\addtolength{\evensidemargin}{-.875in}
\addtolength{\textwidth}{1.75in}
\DeclareMathOperator{\im}{im}
\DeclareMathOperator{\Aut}{Aut} 
\DeclareMathOperator{\Span}{span}
\DeclareMathOperator{\End}{End}
\DeclareMathOperator{\lcm}{lcm}
\DeclareMathOperator{\Int}{int}
\DeclareMathOperator{\If}{if}
\DeclareMathOperator{\Or}{or}
\DeclareMathOperator{\Nd}{and}
\DeclareMathOperator{\st}{such\ that}
\DeclareMathOperator{\writhe}{writhe}
\DeclareMathOperator{\R}{\mathbb{R}}
\DeclareMathOperator{\PP}{\mathbb{P}}
\font\msbmx=msbm10 at 10pt
\textfont15=\msbmx
\mathchardef\subsetneqq="3F24
\mathchardef\subsetneq="3F28
\mathchardef\supsetneq="3F29

\newcommand\inv[1]{#1\raisebox{1.15ex}{$\scriptscriptstyle-\!1$}}
\newcommand{\pn}[1]{\left( #1 \right)}
\newcommand{\bk}[1]{\left[ #1 \right]}
\newcommand{\set}[1]{\left\{ #1 \right\}}
\newcommand{\vc}[1]{\left\langle #1 \right\rangle}
\newcommand{\abs}[1]{\left\lvert #1 \right\rvert}
\newcommand{\norm}[1]{\left\lVert #1 \right\rVert}
\newcommand{\mat}[1]{\ensuremath{ \begin{bmatrix} #1 \end{bmatrix} }}

\newcommand{\ansbox}[2]{\raisebox{-.5\height}{\framebox(#1,#2){}}}




\addtolength{\topmargin}{-.875in}
\addtolength{\textheight}{1.75in}
\pgfplotsset{soldot/.style={color=black,only marks,mark=*},
	holdot/.style={color=black,fill=white,only marks,mark=*},
	compat=1.12}
\pagestyle{fancy}
\lhead{Math 34A, UCSB}
\rhead{ Spring 2022}
\chead{Schley }

\begin{document}
\newtheorem*{theorem*}{Theorem}
	
	\centerline{\Large{ Limits}}\vspace{12 pt}
	\begin{enumerate} 
	\item $\underset{x \rightarrow 0}{\lim} \ 2+x=\fbox{2}$
	\item $\underset{x \rightarrow 0}{\lim} \ 2+3x=\fbox{2}$
	\item $\underset{x \rightarrow 0}{\lim} \ \frac{2x}{3x}=\fbox{2/3}$
\item $\underset{x \rightarrow \infty}{\lim} \ \frac{1}{x}=\fbox{0}$
	\item $\underset{x \rightarrow \infty}{\lim} \ 2 + \frac{3}{x}=\fbox{2}$ 
	\item $\underset{x \rightarrow 0}{\lim} \ \frac{2x + x^2}{3x - 6x^2}=\fbox{2/3}$
	\item $\underset{x \rightarrow 1}{\lim} \ \frac{x-1}{(x-1)(x+1)}=\fbox{1/2}$ 
	\\ \\
	\centerline{\Large{ Logs \ \ \ \ \ \ \ \ \ \ \ }}\vspace{12 pt}
	\item $\log_2(4) = \fbox{2}$
	\item $\log_2(8) = \fbox{3}$
	\item $\log_2(16) = \fbox{4}$
	\item $\log_2(2) = \fbox{1}$
	\item $\log_2(\frac{1}{2}) = \fbox{-1}$
	\item $\log_3(9) = \fbox{2}$
	\item $\log_3(81) = \fbox{4}$
	\item $\log_3(\frac{1}{27}) = \fbox{-3}$
	\item $\log_4(16) = \fbox{2}$
	\item $\log_{4}(64) = \fbox{3}$
	\item $\log_5(25) = \fbox{2}$
	\item $\log_5(125) = \fbox{3}$
	\item $\log_{10}(100) = \fbox{2}$
	\item $\log_{10}(\frac{1}{10}) = \fbox{-1}$
	\item $\log_{10}(.1) = \fbox{-1}$
	\item $\log_{10}(.001) = \fbox{-3}$
	\item $\log_{10}(1,000,000) = \fbox{6}$
	
	\end{enumerate}
	
	\vfill
\centerline{\arge{\it Algebra/Arithmetic Review}}

\newpage
	\centerline{\Large{ Fractional/Negative Exponents}}\vspace{12 pt}
\begin{enumerate}
\item $ 9^{\frac{1}{2}} = $ \\ \\
\item $ 9^{-1} = $ \\ \\
\item $ 9^{-\frac{1}{2}} = $ \\ \\
\item $ 8^{\frac{1}{3}} =$ \\ \\
\item $ 8^{-\frac{1}{3}} =$ \\ \\
\item $ 8^{\frac{4}{3}} =$ \\ \\
\item $ 8^{-\frac{2}{3}} =$ \\ \\
\item $ 64^{\frac{1}{2}} =$ \\ \\
\item $ 64^{\frac{1}{3}} =$ \\ \\
\item $ 64^{\frac{2}{3}} =$ \\ \\
\end{enumerate}

	\centerline{\Large{ Fractions and Reciprocals}}\vspace{12 pt}
\begin{enumerate}
\item $ \left( 10^{-1}+15^{-1}+6^{-1} \right)^{-1} =$ \vspace{60pt} \\
\item $ \left( 12^{-1} - 24^{-1} + 36^{-1} \right)^{-1} =$ \vspace{60pt} \\

\end{enumerate}

	\centerline{\Large{ Decimals}}\vspace{12 pt}
Write each the following as a decimal, then write its equivalent value as a percent. 
\begin{enumerate}
\item $ \frac{1}{10} + \frac{2}{100} + \frac{3}{1000} = $ \\ \\
\item $ \frac{1}{5} + \frac{1}{20} + \frac{3}{500} = $ \\ \\
\item $ \frac{3}{4} - \frac{1}{5} + \frac{1}{200} = $ \\ \\
\item $ \frac{4}{100} + \frac{5}{10,000} + \frac{6}{1,000,000} = $ \\ \\
\item $ \frac{4}{10} - \frac{5}{100} + \frac{6}{1,000} = $ \\ \\
\end{enumerate}


	\centerline{\Large{ Distributing}}\vspace{12 pt}
Multiply out the following
\begin{enumerate}
\item $(a+b)(c+d)(e+f)=$ \\ \\ \\
\item $(a-b)(c-d)(e-f)=$ \\ \\ \\
\item $(x+2)(x-5)=$ \\ \\ \\
\item $(a+b+c)(d+e+f)=$ \\ \\ \\
\item $(k+c)(k^4-k^3c+k^2c^2-kc^3)=$ \\ \\ \\
\item $(4xy^2k^{-2} + 3x^{-1})(xk^3-yk)=$ \\ \\ \\
\end{enumerate}



\newpage
	\centerline{\Large{ Factoring}}\vspace{12 pt}
Factor the following polynomials
\begin{enumerate}
\item $x^3-1=$ \\ \\
\item $x^3+y^3=$ \\ \\
\item $x^2 + 15x + 50=$ \\ \\
\item $x^2 - 15x + 50=$ \\ \\
\item $x^2 + 5x - 50=$ \\ \\
\item $x^2 - 5x - 50=$ \\ \\
\item $12x^2 - 7x + 1=$ \\ \\
\end{enumerate}

	\centerline{\Large{ Canceling Linear Factors}}\vspace{12 pt}
	Simplify the following rational functions. They should all end up as polynomials. 
\begin{enumerate}
\item $ \frac{x^2-1}{x-1}=$ \\ \\
\item $ \frac{x^3-8}{x-2}=$ \\ \\
\item $ \frac{3x^2 + 6x + 3}{x+2}=$ \\ \\
\end{enumerate}

	
	\centerline{\Large{ Fractional/Negative Exponents (Algebra)}}\vspace{12 pt}
For each expression, simplify and write the result as a fraction using only positive exponents. \#3 is a challenge. 
\begin{enumerate}
\item $ \frac{12a^{-2}b^3c^{-4}}{16a^{-3}b^{-1}c^3} =$ \\ \\ 
\item $ \frac{10kxy - 4k^{-1}x^2y}{6k^{-2}x^{-2}y^{-2} + 12kxy} =$ \\ \\ 
\item $\left((a^{12} b^6)^{-\frac{1}{2}} \right)^{-\frac{1}{3}} =$ \\ \\ 
\end{enumerate}



    

	
\end{document}
	