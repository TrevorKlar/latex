\documentclass[12pt]{article}

\usepackage{fullpage}
\usepackage{graphics}
\usepackage{tikz}
\usepackage[parfill]{parskip}
\usepackage[utf8]{inputenc}
\usepackage{etoolbox}

\newbool{alt}
\booltrue{alt}
%\boolfalse{alt}
 
% Standard things to include for math   
\usepackage{amsmath,amssymb,amsfonts,amsthm}

\usepackage{wasysym}

\usepackage{enumitem}



\usepackage{geometry}
 \geometry{
 a4paper,
 top=2cm,
 bottom=2cm,
 left=2cm,
 right=2cm,
 }




% Some of Ebrahim's definitions
\newcommand{\done}{\\\hspace*{0pt}\hfill$\blacksquare$}
\def\N{\mathbb{N}}
\def\R{\mathbb{R}}
\def\Q{\mathbb{Q}}
\def\Z{\mathbb{Z}}
\def\e{\epsilon}
\newcommand{\seq}[1]{\left(#1\right)_{n\in\N}}
\newcommand{\Euc}[1]{\mathbb{R}^{#1}}
\newcommand{\pathvarNaked}{p}
\newcommand{\pathvar}{\vec{\pathvarNaked}}
\newcommand{\pathvarAlt}{\vec{q}}
\newcommand{\exercisesList}[2]{\textcolor{ForestGreen}{Exercises from WeBWorK HW#1: #2.}}
\newcommand{\exerciseText}[1]{\textcolor{ForestGreen}{Exercise: #1}}
\def\vx{\vec{x}}
\def\vu{\vec{u}}
\def\vv{\vec{v}}
\def\vw{\vec{w}}
\newcommand{\der}[2]{\frac{\textrm{d}#1}{\textrm{d}#2}}
\newcommand{\derOp}[1]{\der{\phantom{#1}}{#1}}
\newcommand{\pder}[2]{\frac{\partial #1}{\partial #2}}
\newcommand{\pderOp}[1]{\pder{\phantom{#1}}{#1}}
\newcommand{\colvectwo}[2]{\left[\begin{array}{c}#1\\#2\end{array}\right]}
\newcommand{\colvectwoXYVEC}[2]{\left[#1\right]\,\cbv{x} + \left[#2\right]\,\cbv{y}}
\newcommand{\colvecthree}[3]{\left[\begin{array}{c}#1\\#2\\#3\end{array}\right]}
\newcommand{\colvecfour}[4]{\left[\begin{array}{c}#1\\#2\\#3\\#4\end{array}\right]}
\newcommand{\colvecfourL}[4]{\left[\begin{array}{l}#1\\#2\\#3\\#4\end{array}\right]}
\newcommand{\norm}[1]{\left|\hspace{-1.5pt}\left|#1\right|\hspace{-1.5pt}\right|}
\def\grad{\vec{\nabla}}
\newcommand{\Dir}[3]{\operatorname{Dir}(#1,#2,#3)}
\newcommand{\D}[1]{\textrm{D}#1}
\def\vn{\vec{n}}
\newcommand{\cbv}[1]{\partial_{#1}}
\newcommand{\arrayBrackets}[2]{\left[ \begin{array}{#1} #2 \end{array} \right] }

\makeatletter
\newcommand*\dotp{\mathpalette\bigcdot@{.5}}
\newcommand*\bigcdot@[2]{\mathbin{\vcenter{\hbox{\scalebox{#2}{$\m@th#1\bullet$}}}}}
\makeatother



\newcommand{\ansbox}[2]{\raisebox{-.5\height}{\framebox(#1,#2){}}}

\def\endans{\hspace{1em}\ansbox{40}{40}}


\newcommand{\NEcheckbox}{ % Put check box on northeast corner of page
\begin{tikzpicture}[remember picture,overlay] 
\path (current page.north east) ++(-1,-1) node[below left] {
{\small graded?} {\Large\Square}
};
\end{tikzpicture}
}

\newcommand{\LEFTcheckbox}{ % Put check box in the left margin
\\\begin{tikzpicture}[remember picture,overlay] 
\path ++(-2,0) node[below left] {
 {\Large\Square}
};
\end{tikzpicture}
}

\newcommand{\LEFTcheckboxOwnLine}{ % Put check box in the left margin, use this one if on own line
\begin{tikzpicture}[remember picture,overlay] 
\path ++(-2,0) node[below left] {
 {\Large\Square}
};
\end{tikzpicture}
}






\pagenumbering{gobble}
\begin{document}




% --- Score table ---

%\def\gap{\hspace*{2.5em}}
%
%\begin{tikzpicture}[overlay, remember picture]
%\path (current page.north east) ++(-1,-1) node[below left] {
%\begin{tabular}{c|c|c|c|c|c|c|c}
% 1 &  2 & 3 & 4 & 5 & 6 & 7 & $\Sigma$\\ \hline
% \gap & \gap & \gap & \gap & \gap & \gap & \gap & \gap \\
% \gap & \gap & \gap & \gap & \gap & \gap & \gap & \gap
%\end{tabular}
%};
%\end{tikzpicture}

% ---


\def\namepos{(current page.north west) ++(5,-0.8)}
\begin{tikzpicture}[overlay, remember picture]
\draw [black] \namepos rectangle ++(10,-1);
\draw [black] \namepos ++(0,-1.2) rectangle ++(10,-1);
%\draw [black] \namepos ++(0,-1.2) ++(0,-1.2) rectangle ++(10,-1);
%\draw [black] \namepos ++(0,-1.2) ++(0,-1.2) ++(0,-1.2) rectangle ++(10,-1);
\path \namepos ++(0,-0.4) node [left] {\textbf{Name:}};
\path \namepos ++(0,-0.4) ++ (0,-1.2) node [left] {\textbf{Perm:}};
%\path \namepos ++(0,-0.4) ++ (0,-1.2) ++ (0,-1.2) node [left] {\textbf{Seat Number:}};
%\path \namepos ++(0,-0.4) ++ (0,-1.2) ++ (0,-1.2) ++ (0,-1.2) node [left] {\textbf{ID Checker Name:}};
\end{tikzpicture}

\def\versionRowText{
\textit{version \ifbool{alt}{1}{2},row}
\raisebox{-3pt}{\framebox(15,15){A}}
}
\begin{tikzpicture}[overlay, remember picture]
\path (current page.north east) ++ (-2,-1.5) node [left] %{\versionRowText};
\end{tikzpicture}



\vspace{5em}

\begin{center}\textbf{Math 34A Practice Midterm 1, Spring 2022}\end{center}
\begin{enumerate}

%%%%%%%%%%% (1)
\newpage
\item ({\it 2pts}) Solve for $x$ in the following equation. 
$$\frac{4}{5(x-n)} - \frac{5}{4(x+n)} = 0 $$
\vspace{230pt} \\ \phantom{.} \hfill$x = \ $ \ansbox{100}{60} \\ \\


%%%%%%%%%%% (2)
\item ({\it 2pts}) Multiply out and simplify.
$$(2x-3y^{-1})(3x^{-1} + xy) $$
\vspace{230pt} \\ \phantom{.} \hfill$ \ $ \ansbox{200}{70} 

%%%%%%%%%%% (3)
\newpage
\item ({\it 2pts}) Substitute $x=ab + c$ into the expression below and simplify
$$2x^2 + 3cx - 1. $$
\vfill \\ \phantom{.} \hfill$ \ $ \ansbox{300}{80} \\ \\


%%%%%%%%%%% (4)
\item ({\it 2pts}) Find the point of intersection of two lines. The first line passes through the points (0,0) and (2,3) while the second line passes through the points (0,5) and (4,0). Finding $m_1, b_1$ and $m_2, b_2$ for the equations below may help you find the answer.  
\begin{align*} 
y &= m_1x + b_1 \\ 
y &= m_2x + b_2
\end{align*}
\vfill \\ 
\phantom{.} \hfill$x= \ $ \ansbox{200}{50} \\
\phantom{.} \hfill$y= \ $ \ansbox{200}{50} 



%%%%%%%%%%% (5)
\newpage
\item ({\it 4pts}) Jason started driving from Phoenix towards Isla Vista at noon at a speed of 75 mph. At 2pm Marie started driving from Isla Vista towards Phoenix at 100 mph. Jason and Marie met at 4pm. Meanwhile a bad guy follows Jason. He leaves Phoenix at 1pm and drives along the same route at 90 mph. \\ \\ 
How many miles from Isla Vista was the bad guy when Jason and Marie met? 
\vfill \\ \phantom{.} \hfill$ \ $ \ansbox{200}{60} \ miles \\ \\
%%%%%%%%%%% (6)
\newpage

\item ({\it 4 points})   
\begin{enumerate} 
\item A cylinder with radius $r$ and height $h$ has a volume of $36\pi m^3$. Express the surface area in terms of the height $h$. \\
(The formula for the volume of a cylinder is $\pi r^2 h$, and the surface area formula is $2\pi r^2 + 2\pi rh$.)
\vfill \\ 
\phantom{.} \hfill Surface Area $ = \ $ \ansbox{240}{75} \\
\vspace{10pt} \\ 
\item The side of a roof is going to be built. Its shape is an isosceles triangle with base $b$ and height $h$ (figue below). The area of this triangle is 10$m^2$. Express the perimeter in terms of the height. 
\vfill \\ 
\phantom{.} \hfill Perimeter $ = \ $ \ansbox{240}{75} 

\end{enumerate} 


%%%%%%%%%%% (7)
\newpage
\item ({\it 3 points}) What are the following limits? 
\begin{enumerate} 
\item {\large $\underset{h \rightarrow 0}{\lim}$} \ \ {\large $10-7h$} \\ \\ \phantom{.} \hfill\ansbox{140}{70} \\ \\  
\item {\large $\underset{x \rightarrow 2}{\lim} \ \ $} {\Large $\frac{x^2 - 4x + 4}{2x-4}$} \\ \\ \phantom{.} \hfill\ansbox{140}{70} \\ \\  
\item {\large $\underset{n \rightarrow \infty}{\lim}$} \ \ {\Large$\frac{3n - 2}{n + 3}$} \\ \\ \phantom{.} \hfill\ansbox{140}{70} \end{enumerate}


%%%%%%%%%%% (8)
\item ({\it 3 points}) Compute the logarithms below. 
\begin{enumerate} 
\item $\log_3(27)$ \\ \phantom{.} \hfill\ansbox{140}{70} \\ \\  
\item $\log_{10}(1,000,000)$ \\ \phantom{.} \hfill\ansbox{140}{70} \\ \\  
\item $\log_{2}(\frac{1}{4})$ \\ \phantom{.} \hfill\ansbox{140}{70} \\ \\  
\end{enumerate}


%%%%%%%%%%% (9)
















\end{enumerate}



\end{document}

