\documentclass[12pt]{article}

\usepackage{fullpage}
\usepackage{graphics}
\usepackage{tikz}
\usepackage[parfill]{parskip}
\usepackage[utf8]{inputenc}
\usepackage{etoolbox}

\newbool{alt}
\booltrue{alt}
%\boolfalse{alt}
 
% Standard things to include for math   
\usepackage{amsmath,amssymb,amsfonts,amsthm}

\usepackage{wasysym}

\usepackage{enumitem}



\usepackage{geometry}
 \geometry{
 a4paper,
 top=2cm,
 bottom=2cm,
 left=2cm,
 right=2cm,
 }




% Some of Ebrahim's definitions
\newcommand{\done}{\\\hspace*{0pt}\hfill$\blacksquare$}
\def\N{\mathbb{N}}
\def\R{\mathbb{R}}
\def\Q{\mathbb{Q}}
\def\Z{\mathbb{Z}}
\def\e{\epsilon}
\newcommand{\seq}[1]{\left(#1\right)_{n\in\N}}
\newcommand{\Euc}[1]{\mathbb{R}^{#1}}
\newcommand{\pathvarNaked}{p}
\newcommand{\pathvar}{\vec{\pathvarNaked}}
\newcommand{\pathvarAlt}{\vec{q}}
\newcommand{\exercisesList}[2]{\textcolor{ForestGreen}{Exercises from WeBWorK HW#1: #2.}}
\newcommand{\exerciseText}[1]{\textcolor{ForestGreen}{Exercise: #1}}
\def\vx{\vec{x}}
\def\vu{\vec{u}}
\def\vv{\vec{v}}
\def\vw{\vec{w}}
\newcommand{\der}[2]{\frac{\textrm{d}#1}{\textrm{d}#2}}
\newcommand{\derOp}[1]{\der{\phantom{#1}}{#1}}
\newcommand{\pder}[2]{\frac{\partial #1}{\partial #2}}
\newcommand{\pderOp}[1]{\pder{\phantom{#1}}{#1}}
\newcommand{\colvectwo}[2]{\left[\begin{array}{c}#1\\#2\end{array}\right]}
\newcommand{\colvectwoXYVEC}[2]{\left[#1\right]\,\cbv{x} + \left[#2\right]\,\cbv{y}}
\newcommand{\colvecthree}[3]{\left[\begin{array}{c}#1\\#2\\#3\end{array}\right]}
\newcommand{\colvecfour}[4]{\left[\begin{array}{c}#1\\#2\\#3\\#4\end{array}\right]}
\newcommand{\colvecfourL}[4]{\left[\begin{array}{l}#1\\#2\\#3\\#4\end{array}\right]}
\newcommand{\norm}[1]{\left|\hspace{-1.5pt}\left|#1\right|\hspace{-1.5pt}\right|}
\def\grad{\vec{\nabla}}
\newcommand{\Dir}[3]{\operatorname{Dir}(#1,#2,#3)}
\newcommand{\D}[1]{\textrm{D}#1}
\def\vn{\vec{n}}
\newcommand{\cbv}[1]{\partial_{#1}}
\newcommand{\arrayBrackets}[2]{\left[ \begin{array}{#1} #2 \end{array} \right] }

\makeatletter
\newcommand*\dotp{\mathpalette\bigcdot@{.5}}
\newcommand*\bigcdot@[2]{\mathbin{\vcenter{\hbox{\scalebox{#2}{$\m@th#1\bullet$}}}}}
\makeatother



\newcommand{\ansbox}[2]{\raisebox{-.5\height}{\framebox(#1,#2){}}}

\def\endans{\hspace{1em}\ansbox{40}{40}}


\newcommand{\NEcheckbox}{ % Put check box on northeast corner of page
\begin{tikzpicture}[remember picture,overlay] 
\path (current page.north east) ++(-1,-1) node[below left] {
{\small graded?} {\Large\Square}
};
\end{tikzpicture}
}

\newcommand{\LEFTcheckbox}{ % Put check box in the left margin
\\\begin{tikzpicture}[remember picture,overlay] 
\path ++(-2,0) node[below left] {
 {\Large\Square}
};
\end{tikzpicture}
}

\newcommand{\LEFTcheckboxOwnLine}{ % Put check box in the left margin, use this one if on own line
\begin{tikzpicture}[remember picture,overlay] 
\path ++(-2,0) node[below left] {
 {\Large\Square}
};
\end{tikzpicture}
}






\pagenumbering{gobble}
\begin{document}




% --- Score table ---

%\def\gap{\hspace*{2.5em}}
%
%\begin{tikzpicture}[overlay, remember picture]
%\path (current page.north east) ++(-1,-1) node[below left] {
%\begin{tabular}{c|c|c|c|c|c|c|c}
% 1 &  2 & 3 & 4 & 5 & 6 & 7 & $\Sigma$\\ \hline
% \gap & \gap & \gap & \gap & \gap & \gap & \gap & \gap \\
% \gap & \gap & \gap & \gap & \gap & \gap & \gap & \gap
%\end{tabular}
%};
%\end{tikzpicture}

% ---


\def\namepos{(current page.north west) ++(5,-0.8)}
\begin{tikzpicture}[overlay, remember picture]
\draw [black] \namepos rectangle ++(10,-1);
\draw [black] \namepos ++(0,-1.2) rectangle ++(10,-1);
%\draw [black] \namepos ++(0,-1.2) ++(0,-1.2) rectangle ++(10,-1);
%\draw [black] \namepos ++(0,-1.2) ++(0,-1.2) ++(0,-1.2) rectangle ++(10,-1);
\path \namepos ++(0,-0.4) node [left] {\textbf{Name:}};
\path \namepos ++(0,-0.4) ++ (0,-1.2) node [left] {\textbf{Perm:}};
%\path \namepos ++(0,-0.4) ++ (0,-1.2) ++ (0,-1.2) node [left] {\textbf{Seat Number:}};
%\path \namepos ++(0,-0.4) ++ (0,-1.2) ++ (0,-1.2) ++ (0,-1.2) node [left] {\textbf{ID Checker Name:}};
\end{tikzpicture}

\def\versionRowText{
\textit{version \ifbool{alt}{1}{2},row}
\raisebox{-3pt}{\framebox(15,15){A}}
}
\begin{tikzpicture}[overlay, remember picture]
\path (current page.north east) ++ (-2,-1.5) node [left] {};%{\versionRowText};
\end{tikzpicture}



\vspace{5em}

\begin{center}\textbf{Math 34A Midterm 1, Summer 2022}\end{center}


\pagebreak
\phantom{.}
%%%%%%%%%%% (1)
\pagebreak
\begin{enumerate}
\item ({\it 2pts}) Solve the system of equations below for $x$ and $y$. Your answers should be in terms of $a$ and $b$. 
\begin{align*}
2x +3y &= a \\
x + y &= b 
\end{align*}
\vfill 
\phantom{.} \hfill$x = \ $ \ansbox{100}{60}
\phantom{.} \hfill$y = \ $ \ansbox{100}{60} 


%%%%%%%%%%% (2)
\item ({\it 2pts}) Multiply out and simplify. Your answer should have no negative exponents in it. 
$$\left( \frac{a^{-1}b}{xy} \right)^{-2} \cdot \frac{a^{-1}b}{\sqrt[4]{b^{-4}x^{4}y^{-8}}}$$
\vfill

 \hfill$ \ $ \ansbox{150}{70} 

%%%%%%%%%%% (3)
\newpage
\item ({\it 2pts}) Substitute $x=a+b$ into the expression below and simplify completely. There should be no parentheses in your answer. 
$$x(a^4 - a^3b + a^2b^2 - ab^3 + b^4)(x-2b)$$
\vfill

 \hfill$ \ $ \ansbox{300}{70}


\newpage
%%%%%%%%%%% (4)
\item ({\it 4pts}) Your classmate Eve has been studying for the 34A midterm so hard that they forgot to eat dinner. You want to make them a pizza as quickly as possible, and you set the oven to preheat to 450$^\circ$. You notice that at exactly 8:13PM, the oven's temperature is 90$^\circ$. You check back at exactly 8:17PM and the oven's temperature is now 330$^\circ$. Using linear extrapolation, at what time do you estimate will the oven will be preheated to 450$^\circ$?

\vfill 

 \hfill \ansbox{200}{50} 



%%%%%%%%%%% (5)
\newpage
\item ({\it 4pts}) A city planner wants to build a park with a playground surrounded by a field, and to keep the kids safe she wants to build a fence around it. The field is to be four times as wide as it is long. Fencing purchases are \$350 for shipping plus \$33 per foot of fencing. Express the cost of fencing for the perimeter in terms of the length of the field. Simplify your answer. 
\vfill 

\hfill Fencing Cost$ \ = \ \$$ \ansbox{250}{70}  \\ \\
%%%%%%%%%%% (6)
\newpage

\item ({\it 5 points})   
You are considering the purchase of a 55in TV (TV sizes are measured by the diagonal, not the length or width). You know that the aspect ratio of a screen is the ratio of the width to the height. However, the manufacturer will only disclose the height of the TV, not the width. Express the \textbf{aspect ratio} in terms of the height $h$ of the TV. 
\vfill 

 \hfill Aspect Ratio $ = \ $ \ansbox{240}{140} \\
\vspace{10pt} \\ 


%%%%%%%%%%% (7)
\newpage
\item ({\it 3 points}) What are the following limits? 
\begin{enumerate} 
\item {\large $\lim\limits_{x\to 6} 10x-5$} \\ 

\hfill\ansbox{140}{70} \\ \\  
\item {\large $\lim\limits_{x\to \infty} \frac{14x+4}{16x+3}$} \\ \\ \phantom{.} \hfill\ansbox{140}{70} \\ \\  
\item {\large $\lim\limits_{x\to \infty} \frac{10x^2+x}{-7x}$} \\ 


\hfill\ansbox{140}{70} \end{enumerate}


%%%%%%%%%%% (8)
\item ({\it 3 points}) Compute the logarithms below. 
\begin{enumerate} 
\item $\log_2(8)$ 


\hfill\ansbox{140}{70} \\ \\  
\item $\log_{10}(.01)$ 


\hfill\ansbox{140}{70} \\ \\  
\item $\log_{5}(125)$ 


\hfill\ansbox{140}{70}
\end{enumerate}


%%%%%%%%%%% (9)
















\end{enumerate}



\end{document}

