\documentclass[11pt,letterpaper]{article}

\usepackage{graphicx}
\usepackage{amsmath}
\usepackage{multicol}
% \usepackage{hyperref}
\usepackage{url}
\usepackage{fancyhdr,fancybox}
\usepackage{enumitem}
\usepackage{booktabs}

\usepackage[bookmarks,colorlinks,linkcolor=blue,citecolor=blue,pdfstartview=FitH,urlcolor=blue]{hyperref}%
% \usepackage[scaled=0.86]{berasans}
% \def\UrlFont{\fontfamily{fvs}\selectfont}
% \urlstyle{same} % use the same fonts for urls as for the text.

\usepackage{titlesec}
\titlespacing*\section{0pt}{14pt plus 4pt minus 2pt}{0pt plus 2pt minus 2pt}


% \setlength{\parindent}{0in}
\setlength{\textwidth}{7in}
\setlength{\evensidemargin}{-0.25in}
\setlength{\oddsidemargin}{-0.25in}
\setlength{\parskip}{.25\baselineskip}
\setlength{\topmargin}{-0.5in}
\setlength{\textheight}{9.1in}


%               Problem and Part
\newcounter{problemnumber}
\newcounter{partnumber}
\newcommand{\Problem}{\stepcounter{problemnumber}\setcounter{partnumber}{0}\item[\fbox{\parbox{.18in}{\hfil\theproblemnumber\hfil}}]}
\newcommand{\InProblem}[1]{\stepcounter{problemnumber}\setcounter{partnumber}{0}\fbox{\parbox{.18in}{\hfil\theproblemnumber\hfil}}\ \ \parbox[t]{1.6in}{#1}}
\newcommand{\InSmallProblem}[1]{\stepcounter{problemnumber}\setcounter{partnumber}{0}\fbox{\parbox{.18in}{\hfil\theproblemnumber\hfil}}\ \ \parbox[t]{1.05in}{#1}}
\newcommand{\Part}{\stepcounter{partnumber}\item[(\alph{partnumber})]}
\newcommand{\InPart}[1]{\stepcounter{partnumber}(\alph{partnumber})\ \ \parbox[t]{1.75in}{#1}}
\newcommand{\InSmallPart}[1]{\stepcounter{partnumber}(\alph{partnumber})\ \ \parbox[t]{1.05in}{#1}}

\pagestyle{empty}
%\lhead{\large\textsc{Math 1a}}
\lhead{}
\rhead{}
\chead{\LARGE\textbf{Math 34A: Calculus For Social Sciences}}
\cfoot{}
\renewcommand{\headrulewidth}{0pt}

\begin{document}
\thispagestyle{fancy}
\begin{center}
{\Large Summer 2022 Course Information and Syllabus} 
\end{center}

\section*{}
\textbf{Instructor:} Trevor Klar, South Hall 6431X, \href{mailto:trevorklar@math.ucsb.edu}{trevorklar@math.ucsb.edu} 
\begin{itemize}[nosep]
\item \textbf{Lecture:} MTWR, 12:30-1:35 Ellison Hall 2626 
\item \textbf{Office Hours:} MTWR after class 2:00-3:00, and by appointment (Really! Just send me an email and I'm happy to make an appointment). I'm also available for appointments by Zoom. 
\item \textbf{Office Location:} South Hall 6431X (Grad Tower, 6th floor, blue side, first door on the right)
\item \textbf{Zoom Link:} \href{https://ucsb.zoom.us/my/trevorklar}{https://ucsb.zoom.us/my/trevorklar} or join with Meeting ID \texttt{trevorklar}
\end{itemize}
\noindent \textbf{Teaching Assistant:} Alfredo Ramirez, \href{mailto:a_ramirez730@math.ucsb.edu}{a{\_}ramirez730@math.ucsb.edu}
\begin{itemize}[nosep]
\item \textbf{Sections:} MW, 2:30 South Hall 1609 and 3:30 HSSB 2251
\item \textbf{Office Hours:} Tues 2-3 PM and Thurs 11:15am-12:15pm
\item \textbf{Office Location:} South Hall 6431V
\item \textbf{Math Lab Hours:} 
\end{itemize}

\section*{About Me}
I've been teaching and tutoring math for 8 years. Before I came to grad school at UCSB I was a high school math teacher for 5 years in Lancaster, California. I earned my Master's in Mathematics at UCSB in 2021, I have 2 cute kids, and I like 90s video games. 

Every individual student comes to my classroom with a different set of circumstances, beliefs, abilities, cultures, challenges, talents, and other factors that will affect their learning and success. I believe that everybody can learn math, and I do my best to tailor my teaching practices to meet the needs of my diverse student population. Mistakes are a part of learning, and everyone's opinions, questions, and way of being is accepted. My classroom is a safe place to explore and learn. 

\section*{About the Course}
Welcome to Math 34A! In this course we will study the foundations of single-variable calculus, the mathematics of quantities that depend on only one factor ("functions of one variable") and of the ways in which these quantities change as this single factor is varied. We will look at the process of differentiation, which tells us the rate at which a function is changing. This requires the study of limits, so we can understand an instantaneous change rather than an average change. We'll then apply the differentiation process to understand how to solve problems from a variety of disciplines. 

Since this is a summer course, we will be moving at approximately double the pace of a normal class. \textbf{You should think of this as an 8-unit class, because we will be doing 4 units of work in half the time.} Plan to spend a little time every day on homework (after lecture while it's fresh would be a great time) every class day. 

% \paragraph*{Prerequisites:}
% Math 1b or equivalent, or a 5 on the BC Advanced Placement Examination
% in Mathematics. If you have specific questions about your readiness
% for this course, please see me in my office. 

% \newpage

\paragraph*{Textbooks:}
The required textbook for this course is \emph{Calculus and
  Mathematical Reasoning for Social and Life Sciences}\ by Daryl
Cooper (ISBN-10: 0-7872-8698-2), and a free pdf of the book is posted on Gauchospace.  We will cover the first 8 chapters, but other chapters are useful as well (check out chapter 16 on how to study math
and science).  You should \textbf{read this book}\ to get a fuller
understanding of the material.

\pagebreak

\section*{Student Learning Outcomes}
By the end of this course, students will be able to:
\begin{itemize}[nosep, label={--}]
\item Translate a word problem into a figure and an equation, solve the equation, and interpret the solution to answer the word problem. 
\item Relate logs and exponential equations as inverses of each other. 
\item Evaluate logarithms using the graph of an exponential function. 
\item Compute limits and derivatives. 
\item Explain how the definition of the derivative relates to the slope of a secant line.
\item Interpret a derivative as the slope of a tangent, and as a rate of change. 
\item Interpret second derivatives as concavity, and as acceleration. 
\item Use the Power Rule and Exponential Rule.
\item Maximize and Minimize a function.
\item Synthesize all of the above skills to solve real-world word problems from a variety of disciplines. 
\end{itemize}

\section*{Grading}
%\vspace*{-0.1in}
\parbox[b]{5.25in}{%
  Your grade's raw score will be the weighted average of the four main
  graded components in the course. Each assignment has the same weight as other assignments in the same category (regardless of point totals) so to compute e.g. your Quiz grade for the course, use the percent scores of each of your quizzes and average them. Then use that average in the formula below to compute your course grade:
  \begin{center}
  {\bf
      20\% Homework %$^{*}$
    + 10\% Quizzes
    + 40\% Midterms
    + 30\% Final Exam 

    %{\small$*$ The homework score includes a section (quiz) score and i$>$clicker score.}
  }
  \end{center}
  This raw numeric score converted to a letter
  grade according to the scheme at right.  %(These cutoffs \emph{may}
  %be adjusted downwards at the end of the course, to make your letter
  %grades higher.)  
  }\hfill 
\parbox{1.4in}{%
  \vspace*{-2.1in}
  \begin{tabular}{cc}
    \toprule
    \textbf{Score} & \textbf{Grade} \\ \toprule
    $\ge 93$ & A \\
    $\ge 90$ & A--  \\ \midrule
    $\ge 87$ & B+ \\
    $\ge 83$ & B \\
    $\ge 80$ & B-- \\ \midrule
    $\ge 75$ & C+ \\
    $\ge 70$ & C \\
    $\ge 65$ & C-- \\ \midrule
    $\ge 60$ & D+  \\
    $\ge 55$ & D  \\
    \bottomrule
  \end{tabular}
}

%% \vspace*{-0.35in}
%\section*{Great Effort}
%%
%If you make a \emph{great effort}\ in this class 
%(which will be judged based on in-class participation, in-section
%participation, homework participation, and effort on exams), any grade
%from C to B$+$ will be automatically bumped to a B.

\section*{Lectures}
Since this is a summer course, we will be moving at approximately double the pace of a normal class. The book is free, and very helpful! I strongly recommend consulting the book when you're not sure how to do a problem. I will say (and post on GauchoSpace) which sections of the book are being covered in each lecture. 

\section*{Homework}
A homework assignment will be posted after every class meeting on Webwork, which is accessed through the link on our GauchoSpace page. Homework will be due the following day at 11:59 PM, so you will have 2 evenings to do it, but you should be doing homework every class day. A little bit of consistent daily practice has been shown to be much more effective than cramming in a lot of learning all in one sitting.  

\paragraph*{Collaboration:}\ \ You are encouraged to form study groups
and to work with each other on your homework.  However, \emph{you
  should understand any solutions you claim to know an answer for.}  It
is unacceptable to copy a solution or answer from the internet, a
solution manual, fellow student, or any other source.

%\paragraph*{A Comment about Homework:}
%Try to think of homework as a way to gain a deeper understanding of
%the material, rather than as a set of isolated problems.  As soon as
%homework is posted, read through it, even if it deals with topics that
%haven't yet been covered in class.  As you listen to lectures, make
%connections with the homework assigned; ask yourself how concepts and
%techniques from different homework sets relate to each other.  When
%your homework is marked, look over any mistakes you've made and be
%sure you can correct them.If you still have questions after you have
%reflected on your homework carefully, be sure to follow up on them
%with someone on the teaching staff!

\section*{Quizzes}
There will be a quiz in every section meeting (10 in total) based on the material learned so far (including review topics). The quizzes are designed to be a short checkup of how you're doing so far; they'll give you early feedback about how effective your studying methods are so far. If you're struggling, \textit{get help! }

While doing well on the quizzes is important, what is more important to me is that you learn the material so that you can do well on the Midterm Exams and the Final. To that end, you can optionally do \textbf{Quiz Corrections} after your quiz is graded. If, for each incorrect answer, you describe all of the following:
\begin{itemize}[nosep]
\item What the problem was asking you to do
\item The mistake(s) in your work
\item How to fix your answer in a way that would give you full credit (including the correctly and completely reworking the problem)
\item What you learned from this mistake
\end{itemize}
then \textbf{for each problem you correct, you will earn 50\% of the missing points back on the corresponding Quiz. }This means a 50\% can be corrected to a 75\%, a 90\% to a 95\%, etc. 

\section*{Exams}
There will be 3 midterms and a final, all in-person. \textbf{Bring your UCSB Access Card, your exam will not be accepted without ID.} This course is naturally cumulative; because we will be building on previous skills every week to advance our ability to solve problems, exams will naturally be cumulative as well. Exams in this class are “brains only”, meaning no calculators may be used (you should also not use a calculator on your homework, so you can practice doing computations on paper). You may use a handwritten 3"x5" notecard. 
%The preliminary dates
%for the exams are:
%\begin{center}
%  \hspace*{1in}
%  \begin{tabular}{lll}
%    \toprule
%    \textbf{Exam} & \textbf{Date} & \textbf{Time} \\ 
%    \toprule
%    Midterm 1
%    & {Wednesday, April 19} 
%    & In class
%    \\ % \midrule
%    Midterm 2
%    & {Wednesday, May 10} 
%    & In class
%    \\ % \midrule
%    Midterm 3
%    & {Wednesday, May 31} 
%    & In class
%    \\ \midrule
%    Final Exam 
%    & Wednesday, June 14
%    & {4:00 to 7:00\textsc{pm}} 
%    \\ \toprule
%  \end{tabular}
%  \hspace*{1in}\ 
%\end{center}

\section*{Makeup Policy}
Attendance at lecture and section is critical for success in this course. I am morally opposed to online exams and makeup exams, but I understand that sometimes life happens, and circumstances beyond your control can cause you to miss class (I’m looking at you, COVID). In the interest of equity and compassion, 
\begin{itemize}[nosep]
\item \textbf{Your lowest two quiz scores will be dropped.}
\item \textbf{Your lowest midterm score will be dropped.}
\item \textbf{There will be no makeups for the Final Exam except in verified cases of medical emergency.}
\end{itemize}
Note that if something unavoidable comes up and you have to miss class, you don’t have to ask me, and I don’t have to make a judgement call to excuse you or not. You just get an automatic free pass. My only caution would be: save your dropped assignments for when you need them. If you do get sick or have some unforeseen emergency, you’ll be really glad you didn’t already go to the beach instead of class. 

Homework will be assigned after every class, and you should try to do it the same day. It will be due before 11:59 PM the next day, but \textbf{you can also submit late homework for 75\% credit. The late deadline for homework is every Sunday at 11:59 PM.} Here are some of my reasons for this policy:
\begin{itemize}[nosep, label={--}]
\item Ultimately I want you to learn the material, and late is better than never. 
\item You don’t get full credit for late work because I want to reward those who do it on time. 
\item I don’t want you to get stuck in a vicious cycle of always trying to catch up on late work, so we start fresh every week on Monday. 
\end{itemize}

\pagebreak

\section*{Regrade Requests and Grading Policies}
\begin{itemize}[nosep]
\item If you don’t already have an account, you will be receiving an email to log in to your account on Gradescope. We use this software to make grading much less time consuming, and to facilitate communication about how things were graded. 
\item Gradescope has a feature called a Regrade Request, where you can ask the instructor or TA about the grading on a particular question. Regrade Requests are great for questions like
  \begin{itemize}[nosep, label={$\circ$}]
  \item “It says here that the answer is $2x+4$, and I wrote $2(x+2)$. I think those are both equivalent, so could you take a second look?”
  \item “You marked this question as ‘no response’, but I actually did write down the answer, I just forgot to put it in the box. Could that count as my answer?”
  \item “I see that my answer is different than what the rubric says is the answer, but I’m not sure what I did wrong on this problem. I’m not asking for points back, but could you clarify my mistake so I can learn for next time?”
  \end{itemize}
\item \textbf{Please don’t email me or your TA without doing a regrade request first.} If you have questions about grading or want to ask for points back, a regrade request is the way to do that. This ensures that the person who graded the problems sees your question and has your work in front of them.
\item \textbf{Please don’t ask for partial credit “just to see what happens”.} I put a lot of thought into grading fairly and providing partial credit whenever I can, so if there isn’t a rubric option for partial credit, it was an intentional choice. 
\item On exams and quizzes, decide what you think is the correct answer, and write only that in the box. \textbf{If you write two answers, I will grade the one that is wrong.} This isn’t to be punitive, it’s because I want you to practice the skill of thinking hard and choosing the best answer. Also if you write two different answers, the grader is going to think that you weren’t sure which answer was correct and so you wrote both. 
\item \textbf{Check your scores on Gradescope right away.} Regrades will only be available for one week, and after that your score is final. It's important for you to have a good sense of how you're doing in the class, and as more time passes it gets harder and harder for graders to go back and review old material. 
\end{itemize}

\section*{Getting Help}
\begin{itemize}[nosep]
\item The \textbf{Math Lab} is a room staffed by graduate students where you can get \textbf{free homework help}. Math Lab is open from 12-5 PM every weekday in South Hall 1607. Their website is \\
\href{https://www.math.ucsb.edu/undergrad/math-lab}{https://www.math.ucsb.edu/undergrad/math-lab}
\item \textbf{Campus Learning Assistance Services (CLAS)} is a hybrid (online and in-person) program you can sign up for to get help on your homework. Details at \href{https://clas.sa.ucsb.edu}{https://clas.sa.ucsb.edu}.
\item \textbf{Office hours} are a great way to get help straight from the source! My office hours are right after class this summer, so feel free to walk with me back to my office for questions about homework, lecture, your grade, or whatever!
\item \textbf{Make a friend in class.} It can be a big help if you miss a class and need the notes, you get stuck on a homework problem at 10:00 at night and you know they're working late too, and just to have a peer to collaborate with. Just make sure you ask your friend \textit{why} and not only \textit{what}. That's the best way to build understanding!
\end{itemize}

\pagebreak

\section*{Disclaimer}
\begin{itemize}[nosep]
\item I try to answer emails promptly, usually within 1 business day. I don't mind if you email me late at night or on weekends, as long as you don't mind if I answer it when I'm back in my office. If you don't hear from me in 3 business days, go ahead and send me a follow-up email to make sure that I got it. 
\item I reserve the right to make changes to this syllabus, but I will give notice in advance if I do. 
\end{itemize}

\section*{University Policies}

\paragraph*{Academic Integrity} The UCSB Student Conduct Code exists to support the highest standards of social and academic behavior and ensure an environment conducive to student learning. It is expected that students attending the University of California, Santa Barbara understand and subscribe to the ideal of academic integrity, and are willing to bear individual responsibility for their work. Any submission that fulfills an academic requirement must represent a student’s original work. Any act of academic dishonesty will subject a person to University disciplinary action. See \href{https://studentconduct.sa.ucsb.edu/academic-integrity}{https://studentconduct.sa.ucsb.edu/academic-integrity} for details. 

\paragraph*{Copyright and Course Materials} Materials in this course—unless otherwise indicated—are protected by United States copyright law [Title 17, U.S. Code].  Materials are presented in an educational context for personal use and study and should not be shared, distributed, or sold in print—or digitally—outside the course without permission. 

\paragraph*{Disabled Students Program (DSP)} Students registered with DSP are required to submit requests for accommodations through the online system. It's a good idea to submit these requests at the beginning of the term. I'd be happy to meet with you individually to discuss your needs and make arrangements for exams if necessary. Contact the DSP office for more information on how to apply, the website is \href{https://dsp.sa.ucsb.edu/accommodations}{https://dsp.sa.ucsb.edu/accommodations}.

\vspace*{0.2in}

\noindent
\textbf{\emph{Enjoy the summer session.  Good luck!}}


\end{document}

%%% Local Variables:
%%% mode: latex
%%% TeX-master: t
%%% End:
