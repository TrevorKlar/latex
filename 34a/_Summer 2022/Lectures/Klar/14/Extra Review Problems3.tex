\documentclass{article}
\usepackage{graphicx}
\usepackage{amsmath}
\usepackage{amssymb}
\usepackage{tabto}
\usepackage{amsfonts}
%\usepackage{MnSymbol}
\usepackage{wasysym}
\usepackage{amsthm}
\usepackage{indentfirst}
\usepackage[utf8x]{inputenc}
\usepackage{caption}
\usepackage{subcaption}
\usepackage{adjustbox}
\usepackage{verbatim}
\usepackage{tikz, pgfplots}
\usepackage{tkz-euclide}
\usepgfplotslibrary{fillbetween}
\usepackage{multicol}
\usepackage{enumitem}
\usepackage{relsize}
\usepackage{xfrac}
\usepackage{array}
\usepackage{enumitem}
\usepackage{graphicx}
\usepackage[english]{babel}
\usepackage{fancyhdr}

\newtheorem{theorem}{Theorem}[section]
\newtheorem{corollary}{Corollary}[theorem]
\newtheorem{lemma}[theorem]{Lemma}

\graphicspath{ {images/} }


\renewcommand{\v}{\vec{v}}
\renewcommand{\u}{\vec{u}}
\renewcommand{\w}{\vec{w}}


\setlength\parindent{0pt}
\addtolength{\oddsidemargin}{-.875in}
\addtolength{\evensidemargin}{-.875in}
\addtolength{\textwidth}{1.75in}
\DeclareMathOperator{\im}{im}
\DeclareMathOperator{\Aut}{Aut} 
\DeclareMathOperator{\Span}{span}
\DeclareMathOperator{\End}{End}
\DeclareMathOperator{\lcm}{lcm}
\DeclareMathOperator{\Int}{int}
\DeclareMathOperator{\If}{if}
\DeclareMathOperator{\Or}{or}
\DeclareMathOperator{\Nd}{and}
\DeclareMathOperator{\st}{such\ that}
\DeclareMathOperator{\writhe}{writhe}
\DeclareMathOperator{\R}{\mathbb{R}}
\DeclareMathOperator{\PP}{\mathbb{P}}
\font\msbmx=msbm10 at 10pt
\textfont15=\msbmx
\mathchardef\subsetneqq="3F24
\mathchardef\subsetneq="3F28
\mathchardef\supsetneq="3F29

\newcommand\inv[1]{#1\raisebox{1.15ex}{$\scriptscriptstyle-\!1$}}
\newcommand{\pn}[1]{\left( #1 \right)}
\newcommand{\bk}[1]{\left[ #1 \right]}
\newcommand{\set}[1]{\left\{ #1 \right\}}
\newcommand{\vc}[1]{\left\langle #1 \right\rangle}
\newcommand{\abs}[1]{\left\lvert #1 \right\rvert}
\newcommand{\norm}[1]{\left\lVert #1 \right\rVert}
\newcommand{\mat}[1]{\ensuremath{ \begin{bmatrix} #1 \end{bmatrix} }}

\newcommand{\ansbox}[2]{\raisebox{-.5\height}{\framebox(#1,#2){}}}




\addtolength{\topmargin}{-.875in}
\addtolength{\textheight}{1.75in}
\pgfplotsset{soldot/.style={color=black,only marks,mark=*},
	holdot/.style={color=black,fill=white,only marks,mark=*},
	compat=1.12}
\pagestyle{fancy}
\lhead{Math 34A, UCSB}
\rhead{Winter 2020}

\pagenumbering{gobble}
\begin{document}
\newtheorem*{theorem*}{Theorem}
	
	
	\centerline{\Large{ Peaks and Valleys}}\vspace{12 pt}
	{\bf \large Key Idea:} What must the slope of the graph be at the highest and lowest points? \\ \\
	\begin{enumerate} 
	\item Find the coordinates of the vertex (lowest point) of the parabola 
	$$y=x^2 + 4x - 72.$$
	\phantom{.} \hfill $x=$\ \ansbox{90}{50} \\ 
	\phantom{.} \hfill $y=$\ \ansbox{90}{50} \\ 
    
    
    
    \item You have 100m of fencing to make a pen on a farm, and one of sides of your pen is provided by the wall of your barn. So you only need fencing for three sides. If $\ell$ and $w$ are the dimensions (length and width) of your pen, the total fencing equation below relates the length and width 
    $$2\ell + w = 100.$$
    Here there are two lengths and only one width because the barn wall serves as the second width. \\ 
    \begin{enumerate}
    \item You know the area of the pen in terms of $\ell$ and $w$. Express the area of the pen in terms of \underline{$\ell$ only}. \vspace{.8in}\\
    \phantom{.}\hfill$A(\ell) =$\ \ansbox{190}{50}
    \item Find the length that results in the largest area $A(\ell)$ for your pen. \vspace{.9in}\\
    \phantom{.}\hfill$\ell =$\ \ansbox{190}{50}
    \item Use your answer in part (b) to find the maximum area for your pen. \vspace{.8in}\\
    \phantom{.}\hfill $A_{\text{max}} =$\ \ansbox{190}{50}
    \end{enumerate}
	
	
	
	\end{enumerate}
    

	
\end{document}
	