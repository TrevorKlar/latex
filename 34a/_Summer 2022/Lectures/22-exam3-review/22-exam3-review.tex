\documentclass[handout]{beamer}
% \documentclass{beamer}

%%
%%
%%
% From http://tex.stackexchange.com/questions/2072/beamer-navigation-circles-without-subsections
% Solution #2 or 3:
% \usepackage{etoolbox}
% \makeatletter
% % replace the subsection number test with a test that always returns true
% \patchcmd{\slideentry}{\ifnum#2>0}{\ifnum2>0}{}{\@error{unable to patch}}%
% \makeatother
% Solution #1:
\usepackage{remreset}% tiny package containing just the \@removefromreset command
\makeatletter
%\@removefromreset{subsection}{section}
%\makeatother
%\setcounter{subsection}{1}




\usepackage{etex}
\usepackage{pgf}
\usepackage{tikz}
\usepackage{url}
\usepackage{amsmath}
\usepackage{color}
% \definecolor{red}{rgb}{1,0,0}
\usepackage{ulem}
% \usepackage{booktabs}
\usepackage{colortbl,booktabs}
\renewcommand*{\thefootnote}{\fnsymbol{footnote}}
\usepackage{fancybox}
\usepackage[framemethod=TikZ]{mdframed}
\mdfdefinestyle{FactStyle}{%
  outerlinewidth=0.5,
  roundcorner=1pt,
  leftmargin=1cm,
  linecolor=blue,
  outerlinecolor=blue!70!black,
  backgroundcolor=yellow!40
}
\usepackage{cancel}

  \newcommand\Warning{%
    \makebox[2.4em][c]{%
      \makebox[0pt][c]{\raisebox{.2em}{\Large!}}%
      \makebox[0pt][c]{\color{red}\Huge$\bigtriangleup$}}}%

\usepackage{stackengine}
\usepackage{scalerel}
\usepackage{xcolor}
  \newcommand\dangersign[1][2ex]{%
    \renewcommand\stacktype{L}%
    \scaleto{\stackon[1.3pt]{\color{red}$\triangle$}{\tiny !}}{#1}%
  }



\usepackage{dcolumn}
\newcolumntype{d}[1]{D{.}{.}{#1}}

% From
% http://tex.stackexchange.com/questions/109900/how-can-i-box-multiple-aligned-equations
\usepackage{empheq}
\usepackage{tcolorbox}  \newtcbox{\othermathbox}[1][]{%
  nobeforeafter, tcbox raise base, 
  colback=black!10, colframe=red!30, 
  left=1em, top=0.5em, right=1em, bottom=0.5em}

\newcommand\blue{\color{blue}}
\newcommand\red{\color{red}}
\newcommand\green{\color{green!75!black}}
\newcommand\purple{\color{purple}}
\newcommand\bluegreen{\color{blue!75!green}}
\newcommand\orange{\color{orange}}
\newcommand\redgreen{\color{red!50!green}}
\newcommand\grey{\color{black}}
\newcommand\gap{\vspace{.1in}}
\newcommand\nb{${\red\bullet}\ $}
\newcommand\halfgap{\vspace{.05in}}
\newcommand\divideline{\line(1,0){352}}
\usepackage{marvosym} % for \Smiley

\newcommand{\bluealert}[1]{{\blue\textbf{#1}}}

% \usepackage{beamerthemesplit} %Key package for beamer
\usetheme{Singapore}
% \usetheme{Szeged}
% \usetheme{Garfield}
% \usetheme{CambridgeUS}
% \usenavigationsymbolstemplate{} %Gets rid of slide navigation symbols


\setbeamercolor{separation line}{use=structure,bg=structure.fg!50!bg}
% \begin{beamercolorbox}[colsep=0.5pt]
%   {upper separation line foot}
% \end{beamercolorbox}



\makeatletter
\setbeamertemplate{footline}
{
  \leavevmode%
  \hbox{%
% \begin{beamercolorbox}[colsep=0.5pt]
%   {upper separation line foot}
% \end{beamercolorbox}


  \begin{beamercolorbox}[wd=.5\paperwidth,ht=2.25ex,dp=2ex,colsep=0.5pt]%
    {upper separation line foot}
    \usebeamerfont{author in head/foot}%
    \hspace*{2ex}\insertshortdate:\ \insertshorttitle
  \end{beamercolorbox}%
  \begin{beamercolorbox}[wd=.5\paperwidth,ht=2.25ex,dp=2ex,right]{title in head/foot}%
    \usebeamerfont{title in head/foot}
    {\insertshortauthor}\hspace*{2ex}
  \end{beamercolorbox}}%
  % \begin{beamercolorbox}[wd=.333333\paperwidth,ht=2.25ex,dp=2ex,right]{date in head/foot}%
  %   \usebeamerfont{date in head/foot}\insertshortdate{}\hspace*{2em}
  %   \insertframenumber{} / \inserttotalframenumber\hspace*{2ex} 
  % \end{beamercolorbox}%
  \vskip0pt%
}
\makeatother

\usetikzlibrary{decorations.markings}
\usetikzlibrary{arrows}


\title{Final Exam Review}
\author{Schley, UCSB Mathematics}
\date{March 15, 2017}
%\institute{}


\useinnertheme{default}

\usefonttheme{serif}
% \usecolortheme{rose}
% \usecolortheme{whale}
% \usecolortheme{orchid}
\usecolortheme{crane}
% \usecolortheme{dolphin}


%TEMPLATE
\setbeamertemplate{navigation symbols}{}

\setbeamertemplate{note page}[compress]

\setbeamertemplate{frametitle}{
  \vspace{0.5em}
  % \begin{centering}
  {\huge\blue\textbf{\textmd{\insertframetitle}}}
  \par
  % \end{centering}
}

% From http://tex.stackexchange.com/questions/7032/good-way-to-make-textcircled-numbers:
\newcommand*\circled[1]{\tikz[baseline=(char.base)]{\node[shape=circle,draw,fill=orange,inner sep=1pt] (char) {#1};}} 
% \renewcommand{\labelenumi}{\circled{\textbf{\arabic{enumi}}}}

\let\olddescription\description
\let\oldenddescription\enddescription
\usepackage{enumitem}
\let\description\olddescription
\let\enddescription\oldenddescription

% \usepackage[loadonly]{enumitem}
\setlist[enumerate,1]{label=\colorbox{orange}{\arabic*.},font=\bfseries}
%\setlist[enumerate,2]{label=\colorbox{blue!25}{(\alph*)},font=\bfseries}
% \setlist[enumerate,1]{label=\arabic*.,font=\bfseries}
\setlist[itemize,1]{label=\red$\bullet$}
\setlist[itemize,2]{label=\blue$\bullet$}

\newcommand\answer[1]{\fbox{#1}}
% \renewcommand\answer[1]{}

\newcommand{\antilog}{\operatorname{antilog}}

\newcommand{\instructor}{Nathan Schley ({\it Sh}+{\it lye})}
\newcommand{\officehours}{T R 11-11:50, T 3:45-4:35 Details on Gauchospace.}
\newcommand{\email}{schley@math.ucsb.edu}
\newcommand{\officeloc}{South Hall 6701}
\newcommand{\copyrightinfo}{2022\ Daryl Cooper, Peter M.\ Garfield, Ebrahim Ebrahim \& Nathan Schley}
    













\title{}
\title{Midterm 3 Review}
\date{May 26, 2017}


\begin{document}
\small

\section*{Administration}

\frame{
  \frametitle{Office Hours!}
  % \ \vspace*{0.25in}

  {\Large{}Instructor:}\\
  \ \hspace*{0.2in} Peter M.\ Garfield, \url{garfield@math.ucsb.edu}\\[0.25em]

  {\Large{}Office Hours:}\\
  \ \hspace*{0.2in} \sout{Mondays 2--3\textsc{pm}}\ {\red{}Not Monday!}\\
  \ \hspace*{0.2in} Tuesdays 10:30--11:30\textsc{am}, {\red{}Next Tuesday: 2:00--3:30\textsc{pm}, too!}\\
  \ \hspace*{0.2in} Thursdays 1--2\textsc{pm}\\
  \ \hspace*{0.2in} or by appointment \\[0.25em]

  {\Large{}Office:}\\
  \ \hspace*{0.2in} South Hall 6510\\[0.5em]

  \copyright\ 2017\ Daryl Cooper, Peter M.\ Garfield

  % \vspace*{2in}
}


\section{Exam Review}

\frame{
  \frametitle{Winter '17 \#1}

  Use the graph given to find 
  \begin{enumerate}
    \item[\colorbox{blue!50}{(a)}]
      {
        $\log(6.3 \times 3.2)$
      }%
      \bigskip

    \item[\colorbox{blue!50}{(b)}]
      {
        Solve $10^{x} = 10/73$
      }%
      \bigskip

    \item[\colorbox{blue!50}{(c)}] 
      {%
        Find a value $c$ so that the  average rate of change of $10^x$ 
        between $x=0.3$ and $x=c$ is $5$
      }

  \end{enumerate}


}

\frame{
  \frametitle{Winter '17 \#2}

  Compute the following derivatives.
  \begin{enumerate}
    \item[\colorbox{blue!50}{(a)}]
      {
        $\dfrac{d}{dx} \left( 5x^4 - 4x + 2 \right) = $
      }%
      \bigskip

    \item[\colorbox{blue!50}{(b)}]
      {
        $\dfrac{d^2}{dx^2} \left( 2e^{5x} - 3x^2 \right) = $
      }%
      \bigskip

    \item[\colorbox{blue!50}{(c)}]
      {
        $\dfrac{d}{dx} \left( x^{e} + e^{x} + e^{k} \right) = $\\[0.5em]
        {[\ $k$ is a constant\ ]}%
      }%

    \end{enumerate}

}

\frame{
  \frametitle{Winter '17 \#3}

    The depth of a certain lake decreases with time as runoff brings
    silt in to fill the lake.  Suppose $f(t)$ gives the depth, in
    meters, of the lake $t$ years after the year 2010.  Suppose $f(7)
    = 100$ and $f'(7) = -3$.  Use the tangent line approximation to
    estimate\ldots
    \bigskip

    \begin{enumerate}
    \item[\colorbox{blue!50}{(a)}]
      {
        The expected depth of the lake in the year 2020. 
      }% 
      \bigskip

    \item[\colorbox{blue!50}{(b)}]
      {
        When (what year) will the depth of the lake be $70$\ meters? 
      }%
    \end{enumerate}

}

\frame{
  \frametitle{Winter '17 \#4}

  This question is about the function
  \begin{equation*}
    f(x) = x^3 + 3x^2 + 4x + 3
  \end{equation*}

  \begin{enumerate}
    \item[\colorbox{blue!50}{(a)}]
      {
        What is the slope of the graph $y=f(x)$ at $x=-2$?
      }
      \bigskip

    \item[\colorbox{blue!50}{(b)}]
      {
        What is the equation of the tangent line to the graph at
        $x=-2$? \\[-0.5em]
        {\tiny[\ give answer in the form $y=mx+b$\ ]}
      }
      \bigskip

    \item[\colorbox{blue!50}{(c)}]
      {
        On what interval is the graph of $y=f(x)$ concave up?
      }
      \bigskip

    \item[\colorbox{blue!50}{(d)}]
      {
        For what value(s) of $x$ does the graph have slope $4$?
      }

    \end{enumerate}

}

\frame{
  \frametitle{Winter '17 \#5}

  The height of a rocket above the ground in meters after $t$ seconds
  is $h(t) = 400 + 20t - 5t^2$.
  \begin{enumerate}
    \item[\colorbox{blue!50}{(a)}]
      {
        What was the velocity of the rocket after $t$ seconds?%
      }%
      %
      \bigskip

    \item[\colorbox{blue!50}{(b)}]
      {
        What was the acceleration of the rocket after $t$ seconds?%
      }%
      \bigskip

    \item[\colorbox{blue!50}{(c)}]
      {
        What was the initial speed of the rocket?
      }%
      \bigskip

    \item[\colorbox{blue!50}{(d)}]
      {
        After how many seconds was the velocity $15\ \text{m}/\text{s}$?
      }%
      \bigskip

    \item[\colorbox{blue!50}{(e)}]
      {
        What was the average speed of the rocket between $t=0$ and $t=2$
        seconds? 
      }
    \end{enumerate}

}
\end{document}