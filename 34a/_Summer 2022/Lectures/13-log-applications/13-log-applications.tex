% \documentclass[handout]{beamer}
\documentclass{beamer}

%%
%%
%%
% From http://tex.stackexchange.com/questions/2072/beamer-navigation-circles-without-subsections
% Solution #2 or 3:
% \usepackage{etoolbox}
% \makeatletter
% % replace the subsection number test with a test that always returns true
% \patchcmd{\slideentry}{\ifnum#2>0}{\ifnum2>0}{}{\@error{unable to patch}}%
% \makeatother
% Solution #1:
\usepackage{remreset}% tiny package containing just the \@removefromreset command
\makeatletter
\@removefromreset{subsection}{section}
\makeatother
\setcounter{subsection}{1}


\usepackage{etex}
\usepackage{pgf}
\usepackage{tikz}
\usepackage{url}
\usepackage{amsmath}
\usepackage{color}
% \definecolor{red}{rgb}{1,0,0}
\usepackage{ulem}
% \usepackage{booktabs}
\usepackage{colortbl,booktabs}
\renewcommand*{\thefootnote}{\fnsymbol{footnote}}
\usepackage{fancybox}
\usepackage[framemethod=TikZ]{mdframed}
\mdfdefinestyle{FactStyle}{%
  outerlinewidth=0.5,
  roundcorner=1pt,
  leftmargin=1cm,
  linecolor=blue,
  outerlinecolor=blue!70!black,
  backgroundcolor=yellow!40
}
\usepackage{cancel}

  \newcommand\Warning{%
    \makebox[2.4em][c]{%
      \makebox[0pt][c]{\raisebox{.2em}{\Large!}}%
      \makebox[0pt][c]{\color{red}\Huge$\bigtriangleup$}}}%

\usepackage{stackengine}
\usepackage{scalerel}
\usepackage{xcolor}
  \newcommand\dangersign[1][2ex]{%
    \renewcommand\stacktype{L}%
    \scaleto{\stackon[1.3pt]{\color{red}$\triangle$}{\tiny !}}{#1}%
  }



\usepackage{dcolumn}
\newcolumntype{d}[1]{D{.}{.}{#1}}

% From
% http://tex.stackexchange.com/questions/109900/how-can-i-box-multiple-aligned-equations
\usepackage{empheq}
\usepackage{tcolorbox}  \newtcbox{\othermathbox}[1][]{%
  nobeforeafter, tcbox raise base, 
  colback=black!10, colframe=red!30, 
  left=1em, top=0.5em, right=1em, bottom=0.5em}

\newcommand\blue{\color{blue}}
\newcommand\red{\color{red}}
\newcommand\green{\color{green!75!black}}
\newcommand\purple{\color{purple}}
\newcommand\bluegreen{\color{blue!75!green}}
\newcommand\orange{\color{orange}}
\newcommand\redgreen{\color{red!50!green}}
\newcommand\grey{\color{black}}
\newcommand\gap{\vspace{.1in}}
\newcommand\nb{${\red\bullet}\ $}
\newcommand\halfgap{\vspace{.05in}}
\newcommand\divideline{\line(1,0){352}}
\usepackage{marvosym} % for \Smiley

\newcommand{\bluealert}[1]{{\blue\textbf{#1}}}

% \usepackage{beamerthemesplit} %Key package for beamer
\usetheme{Singapore}
% \usetheme{Szeged}
% \usetheme{Garfield}
% \usetheme{CambridgeUS}
% \usenavigationsymbolstemplate{} %Gets rid of slide navigation symbols


\setbeamercolor{separation line}{use=structure,bg=structure.fg!50!bg}
% \begin{beamercolorbox}[colsep=0.5pt]
%   {upper separation line foot}
% \end{beamercolorbox}



\makeatletter
\setbeamertemplate{footline}
{
  \leavevmode%
  \hbox{%
% \begin{beamercolorbox}[colsep=0.5pt]
%   {upper separation line foot}
% \end{beamercolorbox}


  \begin{beamercolorbox}[wd=.5\paperwidth,ht=2.25ex,dp=2ex,colsep=0.5pt]%
    {upper separation line foot}
    \usebeamerfont{author in head/foot}%
    \hspace*{2ex}\insertshortdate:\ \insertshorttitle
  \end{beamercolorbox}%
  \begin{beamercolorbox}[wd=.5\paperwidth,ht=2.25ex,dp=2ex,right]{title in head/foot}%
    \usebeamerfont{title in head/foot}
    {\insertshortauthor}\hspace*{2ex}
  \end{beamercolorbox}}%
  % \begin{beamercolorbox}[wd=.333333\paperwidth,ht=2.25ex,dp=2ex,right]{date in head/foot}%
  %   \usebeamerfont{date in head/foot}\insertshortdate{}\hspace*{2em}
  %   \insertframenumber{} / \inserttotalframenumber\hspace*{2ex} 
  % \end{beamercolorbox}%
  \vskip0pt%
}
\makeatother

\usetikzlibrary{decorations.markings}
\usetikzlibrary{arrows}


\title{Final Exam Review}
\author{Peter Garfield, UCSB Mathematics}
\date{March 15, 2017}
%\institute{}


\useinnertheme{default}

\usefonttheme{serif}
% \usecolortheme{rose}
% \usecolortheme{whale}
% \usecolortheme{orchid}
\usecolortheme{crane}
% \usecolortheme{dolphin}


%TEMPLATE
\setbeamertemplate{navigation symbols}{}

\setbeamertemplate{note page}[compress]

\setbeamertemplate{frametitle}{
  \vspace{0.5em}
  % \begin{centering}
  {\huge\blue\textbf{\textmd{\insertframetitle}}}
  \par
  % \end{centering}
}

% From http://tex.stackexchange.com/questions/7032/good-way-to-make-textcircled-numbers:
\newcommand*\circled[1]{\tikz[baseline=(char.base)]{\node[shape=circle,draw,fill=orange,inner sep=1pt] (char) {#1};}} 
% \renewcommand{\labelenumi}{\circled{\textbf{\arabic{enumi}}}}

\let\olddescription\description
\let\oldenddescription\enddescription
\usepackage{enumitem}
\let\description\olddescription
\let\enddescription\oldenddescription

% \usepackage[loadonly]{enumitem}
\setlist[enumerate,1]{label=\colorbox{orange}{\arabic*.},font=\bfseries}
%\setlist[enumerate,2]{label=\colorbox{blue!25}{(\alph*)},font=\bfseries}
% \setlist[enumerate,1]{label=\arabic*.,font=\bfseries}
\setlist[itemize,1]{label=\red$\bullet$}
\setlist[itemize,2]{label=\blue$\bullet$}

\newcommand\answer[1]{\fbox{#1}}
% \renewcommand\answer[1]{}

\newcommand{\antilog}{\operatorname{antilog}}







\title{}
\title{Applications of Logs}
\date{May 3, 2017}


\begin{document}
\small

\section*{Administration}

\frame{
  \frametitle{Office Hours!}
  % \ \vspace*{0.25in}

  {\Large{}Instructor:}\\
  \ \hspace*{0.2in} Peter M.\ Garfield, \url{garfield@math.ucsb.edu}\\[0.25em]

  {\Large{}Office Hours:}\\
  \ \hspace*{0.2in} Mondays 2--3\textsc{pm}\\
  \ \hspace*{0.2in} Tuesdays 10:30--11:30\textsc{am}\\
  \ \hspace*{0.2in} Thursdays 1--2\textsc{pm}\\
  \ \hspace*{0.2in} or by appointment \\[0.25em]

  {\Large{}Office:}\\
  \ \hspace*{0.2in} South Hall 6510\\[0.5em]

  \copyright\ 2017\ Daryl Cooper, Peter M.\ Garfield

  % \vspace*{2in}
}


\section*{Population}

\frame{
  \frametitle{\S7.9: Population Growth}
  
  Assume each generation of bunnies has $3$ times as many bunnies as
  previous one.  Initially have $100$ bunnies. How many bunnies after
  $n$ generations?
  \begin{center}
    A$= 100\times 3n$
    \quad 
    B$ = 100+3n$
    \quad 
    C$ = 100(1+3n)$
    \\
    D$= 100^{3n}$
    \quad 
    E$= 100\times 3^n$
  \end{center}
  \pause
  \alert{Answer:}\ \answer{E}
  \pause
  \halfgap
 
  Start with $100$\\
  After 1 generation have $100\times 3$ bunnies\\ \pause
  After 2 generations have $100\times 3\times 3$ bunnies \\ \pause
  After 3 generations have $100\times 3\times 3\times 3$ bunnies \\ \pause
  So$\ldots$after $n$ generations have 
  \begin{equation*}
    100\times \underbrace{{\blue3\times3\times\cdots\times 3}}_{\text{$n$\ times}} 
    = 100\times 3^n\ \text{bunnies}.
  \end{equation*}

}

\frame{
  \frametitle{More Bunnies}

  We saw that:
  \begin{itemize}
  \item if we start with $100$\ bunnies, and
  \item the bunny population triples every generation,
  \end{itemize}
  then we have $100 \times 3^n$ bunnies after $n$ generations.
  \bigskip

  \begin{enumerate}
    \setcounter{enumi}{0}
  \item How many generations until there are $10^7 = 10,000,000$ bunnies?
    \begin{center}
      A$ = \log(5/3)$
      \qquad 
      B$ = 5-\log(3)$
      \qquad 
      C$= 5/\log(3)$ 
      \\
      D$ = 5/3$
      \qquad 
      E$ = 10^5/3$
      \pause
      % \quad
      % \fbox{C}
    \end{center}
    \begin{center}
      A$\approx 0.22$
      \qquad 
      B$\approx 4.52$
      \qquad 
      C$\approx 10.48$
      \\ 
      \ 
      \qquad
      D$\approx 1.67$
      \qquad 
      E$\approx 3,333$
      \pause
      \qquad
      \answer{C}
    \end{center}

  \end{enumerate}
}


\frame{
  \frametitle{Flu Outbreak}

  \begin{enumerate}
    \setcounter{enumi}{1}
  \item At the start of an outbreak of H1N1 flu in a large herd of
    cattle, there were $5$ infected individuals. The numbers doubles
    every $3$ days. How many days until there are $80$ infected cows?
    \medskip
  \end{enumerate}
  
  \begin{center}
    A$ = \log(16)/\log(2)$
    \quad 
    B$=\log(16/2)$
    \quad 
    C$=16/\log(2)$
    \\[1em]
    D$=3\log(16)/\log(2)$
    \quad 
    E$ = \log(48/2)$
  \end{center}
  \pause
  \vspace*{0.2in}

  \begin{enumerate}
    \setcounter{enumi}{1}
  \item At the start of an outbreak of H1N1 flu in a large class of
    students, there were $5$ infected individuals. The numbers doubles
    every $3$ days. How many days until there are $80$ infected
    students?  
    \smallskip
  \end{enumerate}

  \begin{center}
    A$ = \log(16)/\log(2)$
    \quad 
    B$=\log(16/2)$
    \quad 
    C$=16/\log(2)$
    \\[1em]
    D$=3\log(16)/\log(2)$
    \quad 
    E$ = \log(48/2)$
    \pause
    \qquad
    \fbox{D}
  \end{center}
  \vspace*{1in}
}

\section*{Doubling Time}

\frame{
  \frametitle{Doubling Time Formula}

  Suppose something doubles every $\red{}K$\
  {\redgreen{}minutes}\footnote{%
    Any time unit will work, not just minutes. Just be consistent!}.  If
  there is a mass of $\blue{}A$ at time $t=0$, how much is there at
  time $t$\ {\redgreen{}minutes}?
  \pause

  \begin{empheq}[box=\othermathbox]{align*}
    \text{mass after t  {\redgreen minutes}} = {\blue{A}}\times 2^{(t/{\red K})}
  \end{empheq}
  Idea: $t/{\red K}$ is {\blue number of doubling periods} in $t$  {\redgreen{}minutes}.
  \pause
  \gap

  \begin{enumerate}
    \setcounter{enumi}{2}
  \item A disease spreads through a community. On March 1 there were
    100 infected people. The number of people doubles in a $3$
    days. How many infected people are there $t$ days after March 1?
    \begin{center}
      A$ = 2^t$
      \quad 
      B$= 3\times 2^{t/100}$
      \quad 
      C$ = 100\times 2^t$
      \quad 
      D$= 100\times 2^{t/3}$
      \pause
      \fbox{D}
    \end{center}
  \end{enumerate}

  \gap   

  How many days until there are $1,000$ infected people?
  \begin{center}
    A$ = \log(10)/\log(2)$
    \quad  
    B$=  3\log(10)/\log(2)$
    \quad 
    C$=3\log(5)$
    \\
    D$=3 (\log(10)-\log(2))$
    \quad 
    E$=3\log(20)$
    \pause
    \quad 
    \fbox{B}
  \end{center}
  % \vspace*{2in}

}

\frame{
  \frametitle{A More Complicated Example}

  \begin{empheq}[box=\othermathbox]{align*}
    \text{mass after t  {\redgreen minutes}} = {\blue{A}}\times 2^{(t/{\red K})}
  \end{empheq}
  where
  \begin{itemize}
  \item $K$ is the \colorbox{yellow}{doubling time}, and
  \item $t/{\red K}$ is the \colorbox{yellow}{number of doubling periods} in $t$ {\redgreen{}minutes}.
  \end{itemize}
  % \gap

  \begin{enumerate}
    \setcounter{enumi}{3}
  \item A colony of mold is growing on a cheeseburger in the back of a
    dorm refrigerator. When discovered it has a mass of
    $10\ \text{mg}$. One week later it was found to have a mass of
    $30\ \text{mg}$.  What is the {\red doubling time} measured in
    {\redgreen days}?
    \begin{center}
      A$=\log(2)/\log(3)$
      \quad        
      B$= 7\log(2)/\log(3)$
      \quad 
      C$=7\log(2/3)$
      \quad 
      D$= 7\log(3/2)$
    \end{center}
  \end{enumerate}

  \alert{Hint:}\ We know $A$ and the mass $t$\ {\redgreen{}days}\
  after discovery (for some $t$). 
  \pause
  \halfgap

  Solving $30 = 10\times 2^{7/{\red K}}$ gives \fbox{B}

}

\section*{Half-Life}

\frame{
  \frametitle{\S7.11: Half-Life, Doubling Time}
  % \gap

  The \underline{\blue half-life}\ of a radioactive isotope is the
  time it takes for {\blue half} of the isotope to decay.
  \halfgap 

  \alert{Example:} Isotope W has a {\blue half-life} of {\blue $10$}
  years. How much remains after {\red 20} years? 
  \pause
  {\red None?}
  \pause
  \begin{equation*}
    \frac{1}{2}\times\frac{1}{2}\times\text{(amount you start with)}
  \end{equation*}
  \pause
  \vspace*{-1em}

  \alert{Idea:}\ In half-life problems, convert time into {\blue half-lives}.
  \halfgap
  \pause 
  
  In this problem, the half-life is {\blue 10 years}. Therefore, {\red
    20 years}\ is {\blue two half-lives}. 
  \pause
  \halfgap

  \alert{In general:}\ After $\red n$ half-lives, 
  \begin{equation*}
    \text{\blue remaining amount} 
    = \left(\frac{1}{2}\right)^{\red n}\times 
    \text{({\blue amount started with})}
  \end{equation*}
  \pause
  \vspace*{-1.5em}
  
  \begin{enumerate}
    \setcounter{enumi}{4}
  \item Start with ${\red 120}$ grams of an isotope with
    a half-life of {\blue 12} years. How many grams remains after {\redgreen
      $36$} years? 
    \begin{center}
      A$=0$
      \quad 
      B$ = 10$
      \quad 
      C$ = 15$
      \quad 
      D$ = 20$
      \quad 
      E$ = 40$
      \pause
      \quad
      \fbox{C}
    \end{center}

  \end{enumerate}

}

\frame{
  \frametitle{Another Example}

  \alert{In general:}\ After $\red n$ half-lives, 
  \begin{equation*}
    \text{\blue remaining amount} 
    = \left(\frac{1}{2}\right)^{\red n}\times 
    \text{({\blue amount started with})}
  \end{equation*}
  \pause
  \vspace*{-0.5em}

  \begin{enumerate}
    \setcounter{enumi}{5}
  \item An isotope has a half-life of {\red$5$\ years}.

    \colorbox{blue!25}{(a)}\ If we start with {\blue$70$\ grams}, how
    many grams will be left after $t$ years?
    \begin{center}
      A$\displaystyle={\blue70}\left(\frac{1}{2}\right)^t$
      \quad 
      B$\displaystyle = {\red5}\left(\frac{1}{2}\right)^{{\blue70}t}$
      \quad
      C$\displaystyle = {\blue70}\left(\frac{1}{2}\right)^{{\red5}t}$
      \\
      D$\displaystyle={\blue70}\left(\frac{1}{2}\right)^{t/{\red5}}$
      \quad 
      E$=0$
      \pause
      \quad
      \fbox{D}
    \end{center}
    % \gap

    \colorbox{blue!25}{(b)}\ How many years until ${\purple 10}$ grams
    remain? 
    \begin{center}
      A$=5(\log(7)-\log(2))$
      \quad 
      B$=\log(7)/\log(2)$
      \quad 
      C$= 5\log(7/2)$\\[0.5em]

      D$=5\log(7)/\log(2)$
      \quad 
      E$=\log(7)/(5\log(2))$
      \pause
      \quad
      \fbox{D}
    \end{center}
  \end{enumerate}

}


\frame{
  \frametitle{Half-Life Formula}

  Suppose something has a half-life of $\red{}K$\
  {\redgreen{}years}\footnote{%
    Any time unit will work, not just years. Just be consistent!}.  If
  there is a mass of $\blue{}A$ at time $t=0$, how much is there at
  time $t$\ {\redgreen{}years}?
  \pause

  \begin{empheq}[box=\othermathbox]{align*}
    \text{mass after $t$  {\redgreen years}} = {\blue{A}}\times \left(\tfrac{1}{2}\right)^{(t/{\red K})}
  \end{empheq}
  Idea: $t/{\red K}$ is {\blue number of half-lives} in $t$ {\redgreen{}years}.
  \pause
  \gap

  \begin{enumerate}
    \setcounter{enumi}{6}
  \item (\alert{Radiocarbon Dating})\ A bone is found with
    ${\purple 2\%}$ of the usual amount of {\blue carbon-14} in
    it. The half-life of carbon-14 is {\red $5730$\ years}.  How old
    (in years) is the bone?
    \begin{center}
      A$={\red 5730} \log(.01)/\log(2)$
      \quad 
      B$={\red 5730}\log(50)/\log(2)$\\[0.5em]
      \hspace*{1in}
      C$= {\red 5730}\times 50$
      \quad 
      D$ = \text{wicked old}$
      \pause
    \end{center}
  \end{enumerate}
  \alert{Answer:}\ \answer{B} $\approx 32,000$ years


}

\end{document}

\frame{
  \frametitle{Summary of Logs}

  $\log(y)$ is {\blue how many tens you multiply together to get $y$}.
  \bigskip

  \begin{minipage}{0.4\linewidth}
    \begin{empheq}[box=\othermathbox]{align*}
      10^{\log(y)} = y
    \end{empheq}
  \end{minipage}
  \hspace*{0.25in}
  \begin{minipage}{0.4\linewidth}
    \begin{empheq}[box=\othermathbox]{align*}
      \log\left( 10^a \right) = a
    \end{empheq}
  \end{minipage}
  \bigskip

  \begin{minipage}{0.4\linewidth}
    \begin{empheq}[box=\othermathbox]{align*}
      10^{a} {\red\times} 10^b = 10^{a{\red+}b}
    \end{empheq}
  \end{minipage}
  \hspace*{0.25in}
  \begin{minipage}{0.4\linewidth}
    \begin{empheq}[box=\othermathbox]{align*}
      \log(x{\red\times}y) = \log(x) {\red+} \log(y)
    \end{empheq}
  \end{minipage}
  \bigskip

  \begin{minipage}[b]{0.4\linewidth}
    \begin{empheq}[box=\othermathbox]{align*}
      \left( 10^{a} \right)^{\red{}p} = 10^{a{\red{}p}}
    \end{empheq}
  \end{minipage}
  \hspace*{0.25in}
  \begin{minipage}[b]{0.4\linewidth}
    \begin{empheq}[box=\othermathbox]{align*}
      \log(a^{\,\red{}p}) = {\red{}p}\log(a)
    \end{empheq}
  \end{minipage}
  \bigskip

  \pause
  {\red{}Each of these pairs of equalities says one thing!}


}

\frame{
  \frametitle{\S7.13: Logs in Other Bases}

  $\log(y)$ is {\blue how many tens you multiply together to get $y$}.
  \bigskip

  \uncover<2->{%
    $\log_2(y)$ is {\blue how many {\red{}twos}\ you multiply together to get $y$}.
    \bigskip
  }

  \uncover<3->{%
    So ${\orange2}^{\red 3}={\blue 8} \qquad\text{means the same thing as}\qquad\log_{{\orange2}}({\blue 8})={\red 3}$
    \gap
    \pause
  }

  \uncover<4->{%
    \alert{Examples:}
  }
  \begin{align*}
    \uncover<4->{\log_{{\redgreen2}}({\blue 16})
    & = }
      \uncover<5->{%
      {\red 4}
    && \text{because}\ {\redgreen2}^{\red 4}={\blue 16} \\
    \log_{{\redgreen2}}({\blue 32})
    & = }
      \uncover<6->{%
      {\red 5}
    &&\text{because}\ {\redgreen2}^{\red 5}={\blue 32}\\
    \log_{{\redgreen2}}({\blue 1/8})
    & = }
      \uncover<7->{%
      {\red -3}
    && \text{because}\ {\redgreen2}^{\red -3}={\blue 1/8}
       }
  \end{align*}

  \uncover<8->{%
    The five laws of logs work {\blue for any base} {\red b}\
    exactly the same way except\ldots
  }

  \uncover<9->{%
    \begin{minipage}{0.4\linewidth}
      \begin{empheq}[box=\othermathbox]{align*}
        {\red{}b}^{\log(y)} = y
      \end{empheq}
    \end{minipage}
    \hspace*{0.25in}
    \begin{minipage}{0.4\linewidth}
      \begin{empheq}[box=\othermathbox]{align*}
        \log_{\red{}b}\left( {\red{}b}^a \right) = a
      \end{empheq}
    \end{minipage}
  }

}

\frame{
  \frametitle{Summary \& Examples}

  \alert{Important bases:}
  \begin{itemize}
  \item $\log_{\red2}$ is used extensively in computer science
  \item $\ln = \log_{\red{}e}$ is used everywhere (the natural log)
    ($e\approx2.718$) \\
    \pause
    $\log_{\red e}({\blue y})={\redgreen x}$ means ${\red e}^{\redgreen x}={\blue y}$
    \\ \pause
    $\log_{\red e}({\blue y})$ is how many ${\red e}$'s you multiply to get ${\blue y}$.
    \\ \pause
    Read as: `` log base ${\red e}$ of ${\blue y}$ equals ${\redgreen x}$.''
  \end{itemize}
  \pause
  \gap 

  \alert{Examples:}
  \smallskip

  $\log_{\red 3}({\blue 81})=$
  \vspace*{-1.85em}

  \begin{center}
    A$=0$
    \quad 
    B$=1$
    \quad 
    C$=2$
    \quad 
    D$=3$
    \quad 
    E$=4$
    \pause
    \quad
    \fbox{E} 
  \end{center}
  \gap

  $\log_{\green 5}({\blue 25})=$
  \vspace*{-1.85em}

  \begin{center}
    A$=0$
    \quad 
    B$=1$
    \quad 
    C$=2$
    \quad 
    D$=3$
    \quad 
    E$=4$
    \pause
    \quad
    \fbox{C}
  \end{center}
  \halfgap

  Simplify $\ln\left(\left(e^{3x}\times e^{y}\right)^2\right)$
  \begin{center}
    A$=6x+ y$
    \quad 
    B$ = 2x+2y$
    \quad 
    C$ = 3x+2y$
    \quad 
    D$ = 6x+2y$
    \quad 
    E$=6xy$
    \quad 
    \pause
    \fbox{D}
  \end{center}

}


\end{document}

