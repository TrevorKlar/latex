\documentclass[handout]{beamer}
% \documentclass{beamer}

%%
%%
%%
% From http://tex.stackexchange.com/questions/2072/beamer-navigation-circles-without-subsections
% Solution #2 or 3:
% \usepackage{etoolbox}
% \makeatletter
% % replace the subsection number test with a test that always returns true
% \patchcmd{\slideentry}{\ifnum#2>0}{\ifnum2>0}{}{\@error{unable to patch}}%
% \makeatother
% Solution #1:
\usepackage{remreset}% tiny package containing just the \@removefromreset command
\makeatletter
%\@removefromreset{subsection}{section}
%\makeatother
%\setcounter{subsection}{1}




\usepackage{etex}
\usepackage{pgf}
\usepackage{tikz}
\usepackage{url}
\usepackage{amsmath}
\usepackage{color}
% \definecolor{red}{rgb}{1,0,0}
\usepackage{ulem}
% \usepackage{booktabs}
\usepackage{colortbl,booktabs}
\renewcommand*{\thefootnote}{\fnsymbol{footnote}}
\usepackage{fancybox}
\usepackage[framemethod=TikZ]{mdframed}
\mdfdefinestyle{FactStyle}{%
  outerlinewidth=0.5,
  roundcorner=1pt,
  leftmargin=1cm,
  linecolor=blue,
  outerlinecolor=blue!70!black,
  backgroundcolor=yellow!40
}
\usepackage{cancel}

  \newcommand\Warning{%
    \makebox[2.4em][c]{%
      \makebox[0pt][c]{\raisebox{.2em}{\Large!}}%
      \makebox[0pt][c]{\color{red}\Huge$\bigtriangleup$}}}%

\usepackage{stackengine}
\usepackage{scalerel}
\usepackage{xcolor}
  \newcommand\dangersign[1][2ex]{%
    \renewcommand\stacktype{L}%
    \scaleto{\stackon[1.3pt]{\color{red}$\triangle$}{\tiny !}}{#1}%
  }



\usepackage{dcolumn}
\newcolumntype{d}[1]{D{.}{.}{#1}}

% From
% http://tex.stackexchange.com/questions/109900/how-can-i-box-multiple-aligned-equations
\usepackage{empheq}
\usepackage{tcolorbox}  \newtcbox{\othermathbox}[1][]{%
  nobeforeafter, tcbox raise base, 
  colback=black!10, colframe=red!30, 
  left=1em, top=0.5em, right=1em, bottom=0.5em}

\newcommand\blue{\color{blue}}
\newcommand\red{\color{red}}
\newcommand\green{\color{green!75!black}}
\newcommand\purple{\color{purple}}
\newcommand\bluegreen{\color{blue!75!green}}
\newcommand\orange{\color{orange}}
\newcommand\redgreen{\color{red!50!green}}
\newcommand\grey{\color{black}}
\newcommand\gap{\vspace{.1in}}
\newcommand\nb{${\red\bullet}\ $}
\newcommand\halfgap{\vspace{.05in}}
\newcommand\divideline{\line(1,0){352}}
\usepackage{marvosym} % for \Smiley

\newcommand{\bluealert}[1]{{\blue\textbf{#1}}}

% \usepackage{beamerthemesplit} %Key package for beamer
\usetheme{Singapore}
% \usetheme{Szeged}
% \usetheme{Garfield}
% \usetheme{CambridgeUS}
% \usenavigationsymbolstemplate{} %Gets rid of slide navigation symbols


\setbeamercolor{separation line}{use=structure,bg=structure.fg!50!bg}
% \begin{beamercolorbox}[colsep=0.5pt]
%   {upper separation line foot}
% \end{beamercolorbox}



\makeatletter
\setbeamertemplate{footline}
{
  \leavevmode%
  \hbox{%
% \begin{beamercolorbox}[colsep=0.5pt]
%   {upper separation line foot}
% \end{beamercolorbox}


  \begin{beamercolorbox}[wd=.5\paperwidth,ht=2.25ex,dp=2ex,colsep=0.5pt]%
    {upper separation line foot}
    \usebeamerfont{author in head/foot}%
    \hspace*{2ex}\insertshortdate:\ \insertshorttitle
  \end{beamercolorbox}%
  \begin{beamercolorbox}[wd=.5\paperwidth,ht=2.25ex,dp=2ex,right]{title in head/foot}%
    \usebeamerfont{title in head/foot}
    {\insertshortauthor}\hspace*{2ex}
  \end{beamercolorbox}}%
  % \begin{beamercolorbox}[wd=.333333\paperwidth,ht=2.25ex,dp=2ex,right]{date in head/foot}%
  %   \usebeamerfont{date in head/foot}\insertshortdate{}\hspace*{2em}
  %   \insertframenumber{} / \inserttotalframenumber\hspace*{2ex} 
  % \end{beamercolorbox}%
  \vskip0pt%
}
\makeatother

\usetikzlibrary{decorations.markings}
\usetikzlibrary{arrows}


\title{Final Exam Review}
\author{Schley, UCSB Mathematics}
\date{March 15, 2017}
%\institute{}


\useinnertheme{default}

\usefonttheme{serif}
% \usecolortheme{rose}
% \usecolortheme{whale}
% \usecolortheme{orchid}
\usecolortheme{crane}
% \usecolortheme{dolphin}


%TEMPLATE
\setbeamertemplate{navigation symbols}{}

\setbeamertemplate{note page}[compress]

\setbeamertemplate{frametitle}{
  \vspace{0.5em}
  % \begin{centering}
  {\huge\blue\textbf{\textmd{\insertframetitle}}}
  \par
  % \end{centering}
}

% From http://tex.stackexchange.com/questions/7032/good-way-to-make-textcircled-numbers:
\newcommand*\circled[1]{\tikz[baseline=(char.base)]{\node[shape=circle,draw,fill=orange,inner sep=1pt] (char) {#1};}} 
% \renewcommand{\labelenumi}{\circled{\textbf{\arabic{enumi}}}}

\let\olddescription\description
\let\oldenddescription\enddescription
\usepackage{enumitem}
\let\description\olddescription
\let\enddescription\oldenddescription

% \usepackage[loadonly]{enumitem}
\setlist[enumerate,1]{label=\colorbox{orange}{\arabic*.},font=\bfseries}
%\setlist[enumerate,2]{label=\colorbox{blue!25}{(\alph*)},font=\bfseries}
% \setlist[enumerate,1]{label=\arabic*.,font=\bfseries}
\setlist[itemize,1]{label=\red$\bullet$}
\setlist[itemize,2]{label=\blue$\bullet$}

\newcommand\answer[1]{\fbox{#1}}
% \renewcommand\answer[1]{}

\newcommand{\antilog}{\operatorname{antilog}}

\newcommand{\instructor}{Nathan Schley ({\it Sh}+{\it lye})}
\newcommand{\officehours}{T R 11-11:50, T 3:45-4:35 Details on Gauchospace.}
\newcommand{\email}{schley@math.ucsb.edu}
\newcommand{\officeloc}{South Hall 6701}
\newcommand{\copyrightinfo}{2022\ Daryl Cooper, Peter M.\ Garfield, Ebrahim Ebrahim \& Nathan Schley}
    













\title{}
\title{Ch.\ 7: Arithmetic Using Logs}
\date{April 28, 2017}


\begin{document}
\small

\section*{Administration}

\frame{
  \frametitle{Office Hours!}
  % \ \vspace*{0.25in}

  {\Large{}Instructor:}\\
  \ \hspace*{0.2in} Trevor Klar, \url{trevorklar@math.ucsb.edu}\\[0.25em]

  {\Large{}Office Hours:}\\
  \ \hspace*{0.2in} Mondays 2--3\textsc{pm}\\
  \ \hspace*{0.2in} Tuesdays 10:30--11:30\textsc{am}\\
  \ \hspace*{0.2in} Thursdays 1--2\textsc{pm}\\
  \ \hspace*{0.2in} or by appointment \\[0.25em]

  {\Large{}Office:}\\
  \ \hspace*{0.2in} South Hall 6431X (Grad Tower, 6th floor, blue side, first door on the right)\\[0.5em]

  \copyright\ 2017\ Daryl Cooper, Trevor Klar

  % \vspace*{2in}
}


% \frame{
%   \frametitle{Homework Survey:}

%   {\large\alert{For the size of the universe question on Homework}}
%   \begin{itemize}
%   \item[A] I did it on my own
%     \smallskip

%   \item[B] I got help but now I understand
%     \smallskip

%   \item[C] I got help but I don't understand
%     \smallskip

%   \item[D] I did not get the right answers 
%     \bigskip
%   \end{itemize}

% }


\if0 % commenting out!
\frame{
  \frametitle{HW 10 \#8 (Cooper 4.1.8)}

  {\blue
    Light travels $186,282$ miles in one second. A light year is the
    distance light travels in one year. It is about $4.3$ light years to
    the nearest star, Proxima Centauri, (other than our sun!).
    \smallskip

    (a) How many miles away is Proxima Centauri? 
  }
  \medskip

  \begin{align*}
    \uncover<2->{
      1\ \text{light-year}
      & = \big(\ \text{speed of light}\ \big) \times \big(\ 1\
        \text{year}\ \big) \\
    }
    \uncover<3->{%
      & = \left( 186,282\ \frac{\text{miles}}{\text{second}} \right)
        \cdot \frac{60\ \text{sec}}{1\ \text{min}}
        \cdot \frac{60\ \text{min}}{1\ \text{hr}}
        \cdot \frac{24\ \text{hr}}{1\ \text{day}}
        \cdot \frac{365\ \text{days}}{1\ \text{yr}}
        \cdot \big(\ 1\ \text{yr}\ \big) \\
    }
    \uncover<4->{%
      & = 5,874,589,152,000\ \text{miles}
        }
    \end{align*}
    \uncover<5->{%
      So the distance to Proxima Centauri is
    }
    \begin{align*}
      \uncover<5->{%
      4.3\ \text{light-years}
      & = \big(\ 4.3\ \text{light-years}\
        \big)\times\frac{5,874,589,152,000\ \text{miles}}{1\
        \text{light-year}}
        \\
      }
    \uncover<6->{%
      & \approx 2.53 \times 10^{13}\ \text{miles}
        }
    \end{align*}
    \vspace*{2in}

}
\frame{
  \frametitle{HW 10 \#8 (Cooper 4.1.8)}

  {\blue
    Light travels $186,282$ miles in one second. A light year is the
    distance light travels in one year. It is about $4.3$ light years to
    the nearest star, Proxima Centauri, (other than our sun!).
    \smallskip

    (b) The distance from Los Angeles to New York is about $2,000$
    miles. If this distance is represented by the width of a grain of
    sand (say $1/100$ of an inch), how many miles away would Proxima
    Centauri be?
  }
  \medskip

  \uncover<2->{%
    How many grains of sand are there from LA to NY?
  }

  \uncover<3->{%
    \begin{align*}
      \big(\ 2000\ \text{miles}\ \big)
      \cdot \frac{5280\ \text{ft}}{1\ \text{mile}}
      \cdot \frac{12\ \text{in}}{1\ \text{ft}}
      \cdot \frac{100\ \text{grains}}{1\ \text{in}}
      \approx 1.27\times 10^{10}\ \text{grains}.
    \end{align*}
  }
  \uncover<4->{%
    So we're scaling by a factor of $1.27\times 10^{10}$.  That is,
    \begin{equation*}
      1\ \text{``fake mile''} = 1.27 \times 10^{10}\ \text{real miles}.
    \end{equation*}
  }
  \uncover<5->{%
    Thus the distance to Proxima Centauri is 
    \begin{align*}
      4.3\ \text{light-years}
      \approx 2.53 \times 10^{13}\ \text{real miles}
        \cdot \frac{1\ \text{fake mile}}{1.27 \times 10^{10}\
        \text{miles}} 
      \approx 1993\ \text{``fake miles''}.
    \end{align*}
  }
  \vspace*{2in}
}
\fi % end commenting out


\section*{Review}

\frame{
  \frametitle{Finding the log of any number}

  {\red(1)}\ Write the number as ${\redgreen 10^n}\times$({\red number between 1 and 10})\\[0.5em]

  {\red(2)}\ Find the log of the {\red number between 1 and 10}\ using
  table or graph\\[0.5em]

  {\red(3)}\ Log is ${\redgreen{}n} + \log(\text{\red number between 1
    and 10})$
  \gap\ 
  \gap\ 
  \pause

  \alert{\large{}Example:}\ Find $\log(573)$
  \gap

  {\red(1)}\ $\log({\blue 573}) = \log({\redgreen 100}\times{\red 5.73})
  = \log({\redgreen 100}) + \log({\red 5.73}) = {\redgreen 2}+\log({\red 5.73})$
  \smallskip

  {\red(2)}\ \fbox{$\log({\red 5.73}) \approx 0.7582$}
  \smallskip

  {\red(3)}\ $\log({\blue573}) \approx 2 + 0.7582 = 2.7582$
  \gap

  Find $\log({\blue 57.3})$
  \begin{center}
    A$\approx7.582$
    \quad 
    B$\approx10+0.7582$
    \quad 
    C$\approx 1+0.7582$
    \quad 
    D\ Other
    \quad
    \pause
    \fbox{C}
  \end{center}
  \gap

  Find $\log(0.573)$
  \begin{center}
    A$\approx-1.7582$
    \quad 
    B$\approx-1+0.7582$
    \quad 
    C$\approx-0.7582$
    \quad 
    D\ Other
    \quad
    \pause
    \fbox{B}
  \end{center}

}


\frame{
  \frametitle{Finding the antilog of any number}

  \alert{Example:} ${\blue2}.306$ is not on $x$-axis of graph $y=10^x$
  or in middle of log table.  So how do you use table or graph to find
  {${\red\mbox{antilog}}({\blue 2}.306)$}?
  \gap 
  
  {\redgreen\large{}Think about it:}
  \vspace*{-2em}
  \begin{align*}
    {\red\mbox{antilog}}({\blue 2}.306)  
    & =  10^{{\blue2}.306}  \\
    \uncover<2->{%
    & = \underbrace{10^{\blue 2}}_{\text{\red{}duh!}} \times
      \underbrace{ 10^{0.306}}_{\text{\blue{}look it up!}} 
      \\ 
    }
    \uncover<3->{%
    & \approx {\blue100} \times 2.02 \\
    & = 202 
      }
  \end{align*}

  \uncover<3->{This is like the {\blue moving decimal point trick}\ for logs.}

  \uncover<4->{%
    \divideline

    From log table: $10^{0.{\green 86}}\approx7.25$. \qquad
    Use this to find ${\red\mbox{antilog}}(3.{\green 86})$ 
  }
  \uncover<5->{%
    \fbox{\blue$\approx 7250$}\\
    \begin{center}
      A$=\text{I got it right}$
      \qquad 
      B$=\text{I was close}$
      \qquad
      C$=\text{I was wrong}$
    \end{center}
  }
}

\section*{Multiplication}

\frame{
  \frametitle{\S7.5: Using logs to multiply}

  First rule of logs:\  $\log(a{\blue \times }b)= \log(a){\blue +}\log(b)$
  \gap

  Example: Find $2.7\times 1.6$ using logs

  {\red{}Hint:}\ $\log(2.7)\approx 0.43$ and $\log(1.6)\approx 0.20$
  \gap


  {\blue Method}\\
  {\red(i)}\ Look up $\log(2.7)$ and $\log(1.6)$\\
  {\red(ii)}\ Add these\\
  {\red(iii)}\ Take the ${\red\mbox{antilog}}$ of result from (ii)\\
  {\red(iv)}\ Think: Is the answer {\red reasonable} or did I goof up?\\
  \begin{center}
    A$= \text{done}$
    \qquad
    B$=\text{confused}$
  \end{center}

  \vspace*{2in}

}


\frame{
  \frametitle{\S7.5: Using logs to multiply}

  First rule of logs:\  $\log(a{\blue \times }b)= \log(a){\blue +}\log(b)$
  \gap

  Example: Find $2.7\times 1.6$ using logs

  {\red{}Hint:}\ $\log(2.7)\approx 0.43$ and $\log(1.6)\approx 0.20$
  \gap

  {\blue Look how I write the answer.}
  \smallskip

  \begin{itemize}
  \item $\log(2.7{\red \times}1.6) {\blue =} \log(2.7){\red
      +}\log(1.6)$
    \pause

  \item Look up $\log(2.7){\blue \approx} 0.43$ and $\log(1.6){\blue \approx} 0.20$,
    so $\log(2.7{\red \times}1.6){\blue \approx} 0.43{\red +}0.20 =
    {\green 0.63}$
    \pause

  \item {\purple Is this the answer ?}\pause\  {\large{ \red Heck
        No!}} It is the {\red log} of the answer\\ 
    \pause

  \item $2.7{\red \times}1.6 \approx {\red\mbox{antilog}}({\green 0.63})=10^{\green 0.63}$

  \item Look up $10^{\green 0.63} \approx 4.3$\\

  \item Is my answer \fbox{4.3} reasonable? Yes, about $2\times 2=4$.
  \end{itemize}
  \vspace*{1in}

}


\frame{
  \frametitle{A Really Bad Answer}

  \begin{center}
    \tcbox[colback=red!5!white,colframe=red!75!black]{%
    \begin{minipage}{3in}
      $2.7\times 1.6 
      \quad 
      \log(2.7 \times1.6)
      \qquad 
      \log(2.7)+ \log(1.6)$

      \hspace{1.8in}$=0.43$ \quad $=0.20$
      \bigskip

      $0.43+0.20 = 0.63\ \longleftarrow$ my answer!!
    \end{minipage}
    }
  \end{center}

  {\blue Common mistake:}\  Writing math {\red so badly} it is {\blue
    not even wrong}\footnote{%
    ``Not even wrong'' is due to the physicist Wolfgang Pauli.
  }.
  \pause
  \gap

  {\red Stare at what is written} does it make {\blue any sense}?
  \gap
  \pause

  The answer can't be right: how can $2.7\times 1.6$ be $0.43$?
  \smallskip

  Where is the mistake?  It is so badly written that there is no
  mistake to find because it is \alert{nonsense}.
  \pause 


}

\frame{
  \frametitle{Summary of Good Advice}

  \begin{itemize}
  \item \colorbox{yellow}{Write your work properly.}\pause
  \item {\blue Eat your fruit \& vegetables}
  \item {\blue Exercise regularly}
  \item {\blue Get enough sleep}
    \\
    \pause
    % 
    $[\text{This advice brought to you by your mother}]$
  \end{itemize}

}

\frame{
  \frametitle{Examples:}

  \alert{Example:}\ Find $352\times 17.7$ using logs and tables.
  \smallskip

  {\red{}Hint:}\ $\log(3.52)\approx 0.5465$ and $\log(1.77)\approx 0.2480$
  \begin{center}
    A$=\text{done}$
    \qquad
    B$=\text{I'm working!}$
    \qquad
    C$=\text{confused}$
  \end{center}
  \pause
  \gap

  {\large\blue{}My Steps:}
  \begin{itemize}
  \item[\red(1)] $\log(352)=2+\log(3.52)\approx 2.5465$\quad (move the
    decimal point)

  \item[\red(2)] $\log(17.7)=1+\log(1.77)\approx 1.2480$\quad (move
    the decimal point)
    \pause
    
  \item[\red(3)] Add: $\log(352{\red\times} 17.7) = \log(352){\red +}\log(17.7)\approx 2.5465+1.2480= 3.7945$
    \pause

  \item[\red(4)] So: \qquad $352\times 17.7\approx{\red\mbox{antilog}}(3.7945)
    =10^{3.7945}=10^3\times 10^{0.7945}\approx6230$
    \pause

  \item[\red(5)] Check: {\blue Is this reasonable?}\  Should be about $300\times 20=6000$

  \end{itemize}
  Did you get close?
  \begin{center}
    A $=$ Yes 
    \quad 
    B $=$ No
    \quad 
    C $=$ 
    Didn't finish
  \end{center}

}

\section*{Division}

\frame{
  \frametitle{\S7.5: Using logs to divide}

  Remember Log Rule (5): \fbox{$\log(a{\blue\div} b) = \log(a){\blue -}\log(b)$}
  \gap

  \alert{Example:}\ Use this rule to find $38.2/1.77$

  {\red{}Hint:}\ $\log(3.82)\approx 0.58$ and $\log(1.77)\approx 0.25$
  \gap
  
  {\blue Method}\\
  {\red(i)}\ Look up $\log(3.82)$ and $\log(1.77)$, find $\log(38.2)$\\
  {\red(ii)}\ {\red{}Subtract!}\\
  {\red(iii)}\ Take the ${\red\mbox{antilog}}$ of result from (ii)\\
  {\red(iv)}\ Think: Is the answer {\red reasonable} or did I goof up?\\
  \begin{center}
    A$= \text{done}$
    \qquad
    B$=\text{confused}$
  \end{center}
  \vspace*{2in}

}

\frame{
  \frametitle{\S7.5: Using logs to divide}

  Remember Log Rule (5): \fbox{$\log(a{\blue\div} b) = \log(a){\blue -}\log(b)$}
  \gap

  \alert{Example:}\ Use this rule to find $38.2/1.77$

  {\red{}Hint:}\ $\log(3.82)\approx 0.58$ and $\log(1.77)\approx 0.25$
  \gap
  
  {\blue Look how I write the answer.}
  \smallskip

  \begin{itemize}
  \item $\log(38.2{\blue \div}1.77) = \log(38.2){\blue -}\log(1.77)$\quad {\blue using $\log(a/b)=\log(a)-\log(b)$}

  \item $\log(38.2)=1+\log(3.82)\approx 1.58$\quad{\blue  from graph and move decimal point}

  \item $\log(1.77)\approx 0.25$ from graph

  \item $\log(38.2){\blue -}\log(1.77)\approx 1.58-0.25=1.33$

  \item Therefore\ $38.2{\blue \div}1.77 \approx{\red\mbox{antilog}}(1.33)=10^{1.33}$

  \item From graph $10^{0.33}\approx 2.1$ so $10^{1.33}\approx 21$.

  \item Check: Is the answer \fbox{21} reasonable? Yes, about $40\div 2=20$.

  \end{itemize}


}

\frame{
  \frametitle{Examples:}

  \alert{Example:}\ Find $352/17.7$ using logs and tables.
  \smallskip

  {\red{}Hint:}\ $\log(3.52)\approx 0.5465$ and $\log(1.77)\approx 0.2480$
  \begin{center}
    A$=\text{done}$
    \qquad
    B$=\text{I'm working!}$
    \qquad
    C$=\text{confused}$
  \end{center}
  \pause
  \gap

  {\large\blue{}My Steps:}
  \begin{itemize}
  \item[\red(1)] $\log(352)=2+\log(3.52)\approx 2.5465$\quad (move the
    decimal point)

  \item[\red(2)] $\log(17.7)=1+\log(1.77)\approx 1.2480$\quad (move
    the decimal point)
    \pause
    
  \item[\red(3)] Subtract: $\log(352{\red\div}17.7) = \log(352){\red -}\log(17.7)\approx 2.5465-1.2480= 1.2985$
    \pause

  \item[\red(4)] So: \qquad $352\div 17.7\approx{\red\mbox{antilog}}(1.2985)
    =10^{1.2985}=10^1\times 10^{0.2985}\approx19.9$
    \pause

  \item[\red(5)] Check: {\blue Is this reasonable?}\  Should be about $350\div 20\approx20$

  \end{itemize}
  Did you get close?
  \begin{center}
    A $=$ Yes 
    \quad 
    B $=$ No
    \quad 
    C $=$ 
    Didn't finish
  \end{center}

}

\section*{Further Applications}

\frame{
  \frametitle{Powers Using Logs}

  Or, exploting Log Rule (4):
  \begin{empheq}[box=\othermathbox]{align*}
    \log(a^{\red p}) = {\red p}\log(a)
  \end{empheq}
  \gap
  Use this and the graph of $y=10^x$ to find $\sqrt{70}$.

  {\large\blue{}One Approach:}
  \begin{itemize}
  \item[\red(i)] Use graph and move decimal point trick to find
    $\log(70)$

  \item[\red(ii)] $\log(\sqrt{70})=\log(70^{\red 1/2}) =({\red 1/2})\log(70)$


  \item[\red(iii)] Take the {\red$\antilog$}\ of result from
    {\red(ii)}

  \item[\red(iv)] Think: Is the answer {\red reasonable} or did I goof up?

  \end{itemize}
  \textbf{Hint:}\ $\log(7)\approx 0.84$

  \begin{center}
    A$=\text{done}$
    \quad
    B$=\text{working}$
    \quad
    C$=\text{confused}$
  \end{center}
  \pause

  Answer: $\sqrt{70} \approx 8.3$.  Is that reasonable?
  \vspace*{2in}

}

\frame{
  \frametitle{Computer Applications}

  One kilobyte ($1$\ {\red{}KB})\ is $2^{10}$.  
  \halfgap

  \alert{Problem:}\ Calculate $2^{10}$ using logs.
  \qquad 
  \textbf{Hint:}\ $\log(2)\approx 0.3$

  \begin{center}
    A$\approx 3$
    \quad 
    B$\approx 10.3$
    \quad 
    C$\approx 30$
    \quad 
    D$ \approx 1000$
    \quad 
    E$\approx 1100$
    \pause
    \qquad 
    \fbox{D}    
  \end{center}

  So: $2^{10} \approx 10^3 = 1000$ (really $2^{10} = 1024$).
  \vspace*{-1em}

  \begin{align*}
    \text{{\red 1KB} is really}\ 2^{10}
    & =1024 \approx 10^{\red 3}
    && \text{({\red K} is {\red K}ilo = thousand)}\\
    \text{{\red 1MB}  is really}\ 2^{20}
    & = \left(2^{10}\right)^2\approx(10^3)^2=10^{\red 6}
    && \text{({\red M} is {\red M}ega = million)}\\
    \text{{\red 1GB} is really}\ 2^{30}
    & =\left(2^{10}\right)^3\approx(10^3)^3=10^{\red 9}
    && \text{({\red G} is {\red G}iga = billion)}\\
    \text{{\red 1TB} is really}\ 2^{40}
    & =\left(2^{10}\right)^4\approx(10^3)^4=10^{\red 12}
    && \text{({\red T} is {\red T}era = trillion)}
  \end{align*}
  \vspace*{-1em}
  \pause

  Example: suppose on a certain island the population of rabbits
  doubles every generation.  After $20$ generations it
  multiplies by\ldots\pause\ $2^{20}\approx $ 1 million.
  \pause
  \gap 
  
  Powers of $2$ are easy to do, even in your head. To work out
  $2^{\red n}$ the {\blue log} of the answer is approximately
  $0.3{\red n},$ so $2^{\red n}$ is $1$ followed by $0.3{\red n}$
  zeroes.


}



\frame{
  \frametitle{Summary of calculations with logs} 
  \alert{[Courtesy of Daryl Cooper]}
  \smallskip
  
  Calculate the {\blue log} of the thing you want then take {\red$\antilog$}\  of the result.
  % \smallskip

  Example: To calculate ${\purple puppy}=\purple 17^{3.1}$\\ \pause
  (i)  {\orange doggy} $={\blue \log}({\purple puppy})$\\
  (ii) rules of {\blue logs} to expand {\orange doggy}\\ \pause
  (iii) look up {\blue logs} of individual terms in {\orange doggy}. Move decimal point trick.\\ \pause
  (iv) Now have numerical value for  {\orange doggy}.\\ \pause %This is $\log({\purple puppy})$\\ \pause
  (v) so ${\purple puppy}={\red\antilog}({\orange doggy})$ is the answer. \pause

\gap Make sure you never jot down a number on its own. It should always be part of an equation
like ${\blue \log}(945\times 32) \approx 4.48$ This way one can read and understand what is written.
Otherwise you get {\blue gibberish}
\smallskip

{\blue Write math the way I do.} With {\blue words} and {\blue equations}. One should be able to
 {\redgreen read and understand} what is on the paper {\red without being telepathic}.\\
  Imagine it is a {\blue report} for your employer.  In reality you are {\redgreen explaining it to yourself}.


}









\end{document}

