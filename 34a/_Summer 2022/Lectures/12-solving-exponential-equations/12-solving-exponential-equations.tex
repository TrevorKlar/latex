% \documentclass[handout]{beamer}
\documentclass{beamer}

%%
%%
%%
% From http://tex.stackexchange.com/questions/2072/beamer-navigation-circles-without-subsections
% Solution #2 or 3:
% \usepackage{etoolbox}
% \makeatletter
% % replace the subsection number test with a test that always returns true
% \patchcmd{\slideentry}{\ifnum#2>0}{\ifnum2>0}{}{\@error{unable to patch}}%
% \makeatother
% Solution #1:
\usepackage{remreset}% tiny package containing just the \@removefromreset command
\makeatletter
%\@removefromreset{subsection}{section}
%\makeatother
%\setcounter{subsection}{1}




\usepackage{etex}
\usepackage{pgf}
\usepackage{tikz}
\usepackage{url}
\usepackage{amsmath}
\usepackage{color}
% \definecolor{red}{rgb}{1,0,0}
\usepackage{ulem}
% \usepackage{booktabs}
\usepackage{colortbl,booktabs}
\renewcommand*{\thefootnote}{\fnsymbol{footnote}}
\usepackage{fancybox}
\usepackage[framemethod=TikZ]{mdframed}
\mdfdefinestyle{FactStyle}{%
  outerlinewidth=0.5,
  roundcorner=1pt,
  leftmargin=1cm,
  linecolor=blue,
  outerlinecolor=blue!70!black,
  backgroundcolor=yellow!40
}
\usepackage{cancel}

  \newcommand\Warning{%
    \makebox[2.4em][c]{%
      \makebox[0pt][c]{\raisebox{.2em}{\Large!}}%
      \makebox[0pt][c]{\color{red}\Huge$\bigtriangleup$}}}%

\usepackage{stackengine}
\usepackage{scalerel}
\usepackage{xcolor}
  \newcommand\dangersign[1][2ex]{%
    \renewcommand\stacktype{L}%
    \scaleto{\stackon[1.3pt]{\color{red}$\triangle$}{\tiny !}}{#1}%
  }



\usepackage{dcolumn}
\newcolumntype{d}[1]{D{.}{.}{#1}}

% From
% http://tex.stackexchange.com/questions/109900/how-can-i-box-multiple-aligned-equations
\usepackage{empheq}
\usepackage{tcolorbox}  \newtcbox{\othermathbox}[1][]{%
  nobeforeafter, tcbox raise base, 
  colback=black!10, colframe=red!30, 
  left=1em, top=0.5em, right=1em, bottom=0.5em}

\newcommand\blue{\color{blue}}
\newcommand\red{\color{red}}
\newcommand\green{\color{green!75!black}}
\newcommand\purple{\color{purple}}
\newcommand\bluegreen{\color{blue!75!green}}
\newcommand\orange{\color{orange}}
\newcommand\redgreen{\color{red!50!green}}
\newcommand\grey{\color{black}}
\newcommand\gap{\vspace{.1in}}
\newcommand\nb{${\red\bullet}\ $}
\newcommand\halfgap{\vspace{.05in}}
\newcommand\divideline{\line(1,0){352}}
\usepackage{marvosym} % for \Smiley

\newcommand{\bluealert}[1]{{\blue\textbf{#1}}}

% \usepackage{beamerthemesplit} %Key package for beamer
\usetheme{Singapore}
% \usetheme{Szeged}
% \usetheme{Garfield}
% \usetheme{CambridgeUS}
% \usenavigationsymbolstemplate{} %Gets rid of slide navigation symbols


\setbeamercolor{separation line}{use=structure,bg=structure.fg!50!bg}
% \begin{beamercolorbox}[colsep=0.5pt]
%   {upper separation line foot}
% \end{beamercolorbox}



\makeatletter
\setbeamertemplate{footline}
{
  \leavevmode%
  \hbox{%
% \begin{beamercolorbox}[colsep=0.5pt]
%   {upper separation line foot}
% \end{beamercolorbox}


  \begin{beamercolorbox}[wd=.5\paperwidth,ht=2.25ex,dp=2ex,colsep=0.5pt]%
    {upper separation line foot}
    \usebeamerfont{author in head/foot}%
    \hspace*{2ex}\insertshortdate:\ \insertshorttitle
  \end{beamercolorbox}%
  \begin{beamercolorbox}[wd=.5\paperwidth,ht=2.25ex,dp=2ex,right]{title in head/foot}%
    \usebeamerfont{title in head/foot}
    {\insertshortauthor}\hspace*{2ex}
  \end{beamercolorbox}}%
  % \begin{beamercolorbox}[wd=.333333\paperwidth,ht=2.25ex,dp=2ex,right]{date in head/foot}%
  %   \usebeamerfont{date in head/foot}\insertshortdate{}\hspace*{2em}
  %   \insertframenumber{} / \inserttotalframenumber\hspace*{2ex} 
  % \end{beamercolorbox}%
  \vskip0pt%
}
\makeatother

\usetikzlibrary{decorations.markings}
\usetikzlibrary{arrows}


\title{Final Exam Review}
\author{Schley, UCSB Mathematics}
\date{March 15, 2017}
%\institute{}


\useinnertheme{default}

\usefonttheme{serif}
% \usecolortheme{rose}
% \usecolortheme{whale}
% \usecolortheme{orchid}
\usecolortheme{crane}
% \usecolortheme{dolphin}


%TEMPLATE
\setbeamertemplate{navigation symbols}{}

\setbeamertemplate{note page}[compress]

\setbeamertemplate{frametitle}{
  \vspace{0.5em}
  % \begin{centering}
  {\huge\blue\textbf{\textmd{\insertframetitle}}}
  \par
  % \end{centering}
}

% From http://tex.stackexchange.com/questions/7032/good-way-to-make-textcircled-numbers:
\newcommand*\circled[1]{\tikz[baseline=(char.base)]{\node[shape=circle,draw,fill=orange,inner sep=1pt] (char) {#1};}} 
% \renewcommand{\labelenumi}{\circled{\textbf{\arabic{enumi}}}}

\let\olddescription\description
\let\oldenddescription\enddescription
\usepackage{enumitem}
\let\description\olddescription
\let\enddescription\oldenddescription

% \usepackage[loadonly]{enumitem}
\setlist[enumerate,1]{label=\colorbox{orange}{\arabic*.},font=\bfseries}
%\setlist[enumerate,2]{label=\colorbox{blue!25}{(\alph*)},font=\bfseries}
% \setlist[enumerate,1]{label=\arabic*.,font=\bfseries}
\setlist[itemize,1]{label=\red$\bullet$}
\setlist[itemize,2]{label=\blue$\bullet$}

\newcommand\answer[1]{\fbox{#1}}
% \renewcommand\answer[1]{}

\newcommand{\antilog}{\operatorname{antilog}}

\newcommand{\instructor}{Nathan Schley ({\it Sh}+{\it lye})}
\newcommand{\officehours}{T R 11-11:50, T 3:45-4:35 Details on Gauchospace.}
\newcommand{\email}{schley@math.ucsb.edu}
\newcommand{\officeloc}{South Hall 6701}
\newcommand{\copyrightinfo}{2022\ Daryl Cooper, Peter M.\ Garfield, Ebrahim Ebrahim \& Nathan Schley}
    













\title{}
\title{Solving Exponential Equations}
\date{May 1, 2017}


\begin{document}
\small

\section*{Administration}

\frame{
  \frametitle{Office Hours!}
  % \ \vspace*{0.25in}

  {\Large{}Instructor:}\\
  \ \hspace*{0.2in} Peter M.\ Garfield, \url{garfield@math.ucsb.edu}\\[0.25em]

  {\Large{}Office Hours:}\\
  \ \hspace*{0.2in} Mondays 2--3\textsc{pm}\\
  \ \hspace*{0.2in} Tuesdays 10:30--11:30\textsc{am}\\
  \ \hspace*{0.2in} Thursdays 1--2\textsc{pm}\\
  \ \hspace*{0.2in} or by appointment \\[0.25em]

  {\Large{}Office:}\\
  \ \hspace*{0.2in} South Hall 6510\\[0.5em]

  \copyright\ 2017\ Daryl Cooper, Peter M.\ Garfield

  % \vspace*{2in}
}

\section*{Review}

\frame{
  \frametitle{Summary of calculations with logs} 
  \alert{[Courtesy of Daryl Cooper]}
  \smallskip

  Calculate the {\blue log} of the thing you want then take {\red$\antilog$}\  of the result.


Example: To calculate ${\purple puppy}=\purple 17^{3.1}$\\ \pause
(i)  {\orange doggy} $={\blue \log}({\purple puppy})$\\
(ii) rules of {\blue logs} to expand {\orange doggy}\\ \pause
(iii) look up {\blue logs} of individual terms in {\orange doggy}. Move decimal point trick.\\ \pause
(iv) Now have numerical value for  {\orange doggy}.\\ \pause %This is $\log({\purple puppy})$\\ \pause
(v) so ${\purple puppy}={\red\antilog}({\orange doggy})$ is the answer. \pause

\gap Make sure you never jot down a number on its own. It should always be part of an equation
like ${\blue \log}(945\times 32) \approx 4.48$ This way one can read and understand what is written.
Otherwise you get {\blue gibberish}

\gap
{\blue Write math the way I do.} With {\blue words} and {\blue equations}. One should be able to
 \colorbox{yellow}{read and understand} what is on the paper {\red without being telepathic}.
  Imagine it is a {\blue report} for your employer.  In reality you are \colorbox{yellow}{explaining it to yourself}.


}

\section*{Solving Equations}

\frame{
  \frametitle{\S7.7: Solving Exponential Eq'ns}

  \begin{enumerate}
  \item   Find $x$ by solving $10^{x} = 5$.
    \begin{center}
      A$=5$
      \quad 
      B$=0.5$
      \quad
      C$=\log(5)$
      \quad
      D$=\log(0.5)$
      \quad
      E$=\log(5)-\log(10)$
      \pause
      \qquad
      \fbox{C}
    \end{center}
  \end{enumerate}
  \pause

  {\Large\blue{}Look how I write the answer!}
  \begin{align*}
    \log(10^{x})
    & =\log(5)
    & & \text{\blue Take logs of both sides}\\
    x = \log(10^{x})
    & =\log(5)
    & & \text{\blue Using $\log(a^{\red p})={\red p}\log(a)$ and $\log(10)=1$}
  \end{align*}
  \pause
  % \gap

  {\Large\blue{}Why didn't I use antilog?}
  \pause

  Answer: I found $x$, so why would I?

  I have \alert{written}\ equations so I can \alert{see}\ what each
  thing I write \alert{means}. I can \alert{see}\ that I've found $x$
  and so don't need to take antilog.
  \pause\gap

  {\Large\blue{}How do I \emph{know}\ when to take antilog?}
  \pause
  \qquad
  How do {\red{you}}\ know?
  \pause
  My answer: If you write the problem the way I do, so it makes sense,
  you can \alert{see} what to do. 


}

\frame{
  \frametitle{Examples:}

  Use the Fourth Law:
  \begin{empheq}[box=\othermathbox]{align*}
    \log(a^{\red{}x})={\red{}x} \log(a)
  \end{empheq}
  Slogan: Logs bring exponents down to ground level. 
  \gap 

  % If ${\red x}$ is the unknown you can't find it until it is at {\blue ground level}\\
  % So with these equations the first step is always to write on the paper \\ {\blue Take logs of both sides}.
  % \gap  

  \begin{enumerate}
    \setcounter{enumi}{1}
  \item Solve $3^{\red x} = 7$
    \begin{center}
      A$=\log(7/3)$
      \quad 
      B$ = \log(7)-\log(3)$
      \quad 
      C$=\log(7)+\log(3)$
      \\
      D$ =\log(3)/\log(7)$
      \quad 
      E$=\log(7)/\log(3)$
      \pause
      \quad 
      \fbox{E}
    \end{center}
  \end{enumerate}
  \gap

  {\large\blue Look how I write the answer:}
  \begin{align*}
    \log(3^{x})
    & =\log(7)
    && \text{\blue Take logs of both sides}\\
    x\log(3)
    =\log(3^{x})
    & = \log(7)
    && \text{\blue Using $\log(a^{\red{}p})={\red{}p}\log(a)$} \\
    \text{So:}\qquad 
    x & = \log(7)/\log(3)
  \end{align*}
  \vspace*{1in}

}

\frame{
  \frametitle{Examples:}

  Use the Fourth Law:
  \begin{empheq}[box=\othermathbox]{align*}
    \log(a^{\red{}x})={\red{}x} \log(a)
  \end{empheq}
  Slogan: Logs bring exponents down to ground level. 
  \gap 

  % If ${\red x}$ is the unknown you can't find it until it is at {\blue ground level}\\
  % So with these equations the first step is always to write on the paper \\ {\blue Take logs of both sides}.
  % \gap  

  \begin{enumerate}
    \setcounter{enumi}{2}
  \item Solve $7^{{\red x}+2} =30.$\\[1em]
    \begin{center}
      A$\displaystyle=\frac{\log(30)-2\log(7)}{\log(7)}$
      \quad
      B$\displaystyle= \frac{\log(30)}{\log(7)} - 2$ 
      \quad
      C$\displaystyle= \frac{\log(30)-\log(49)}{\log(7)}$
      \\[1em]
      D$\displaystyle = \frac{\log(30/49)}{\log(7)}$
      \quad\qquad 
      E$\approx -0.25213$
    \end{center}
    \pause 
    \fbox{\blue All are correct!}
    \vspace*{2in}
  \end{enumerate}
}

\frame{
  \frametitle{Examples:}

  Use the Fourth Law:
  \begin{empheq}[box=\othermathbox]{align*}
    \log(a^{\red{}x})={\red{}x} \log(a)
  \end{empheq}
  Slogan: Logs bring exponents down to ground level. 
  \gap 

  % If ${\red x}$ is the unknown you can't find it until it is at {\blue ground level}\\
  % So with these equations the first step is always to write on the paper \\ {\blue Take logs of both sides}.
  % \gap  

  \begin{enumerate}
    \setcounter{enumi}{3}
  \item Solve $7\times 3^{y} =  2^{4y+3}$ \\[1em]
    \begin{center}
      A$\displaystyle=\frac{3\log(2)-\log(7)}{\log(3)-4\log(2)}$
      \quad
      B$\displaystyle=\frac{3\log(2)}{7\log(3)}$
      \quad
      C$\displaystyle=\frac{3\log(2)}{7\log(3)-4\log(2)}$
      \\[2em]
      D$\displaystyle=\frac{7\log(3)-4\log(2)}{3\log(2)}$
      \qquad\qquad
      E=$\text{none of the above}$
    \end{center}
    \pause\gap 
    \fbox{A}
    \vspace*{2in}
   \end{enumerate}

}


\section*{Interest}

\frame{
  \frametitle{Compound Interest}

  At end of each year a bank pays $7\%$ interest into your
  account. Initially have \$10,000 in account. How much after 10 years?
  \gap 

  {\red Think} $10\times 7\%=70\%$ in 10 years, so 
  \uncover<2->{have {\red \$17,000} but that is {\red wrong}.}
  \vspace*{-1em}

  \begin{align*}
    \uncover<3->{%
    \text{After $1$ year:}
    && \$10,000\times 1.07 & =\$10,700\\[.25em]
    }
    \uncover<4->{%
    \text{After 2 years:}
    && \$10,700\times 1.07
       = \$10,000 \times 1.07 \times 1.07
       & = \$11,4{\red 49}\\[.25em]
    }
    \uncover<5->{%
    \text{After 3 years:}
    && \$11,449\times 1.07
       = \$10,000 \times (1.07)^3
       & = \$12,{\red{}250.40}
         }
  \end{align*}
  %
  \uncover<5->{%
    Each year {\blue what you had before} is {\red{}multiplied} by
    $1.07$. Thus {\blue compound} interest.
  }

  \uncover<6->{%
    So after 10 years have
    \begin{equation*}
      \$10,000\times 
      \underbrace{1.07\times 1.07\times\cdots\times 1.07}_{\text{$10$ times}}
      = 10,000\times(1.07)^{10}
      \approx \fbox{{\blue \$20,000}}
    \end{equation*}
  }
  \vspace*{-1em}

  \uncover<7->{%
    {\blue Conclusion:}\  
    Money approximately doubles in 10 years!\\ 
    So in 20 years multiplies by 4, in 30 years by 8,$\ldots$
  }
}


\frame{
  \frametitle{General Compound Interest}

  If the interest rate is ${\red r}\%$, then each year money {\blue
    m}ultiplies by
  \begin{equation*}
    {\blue m}=1+\frac{{\red r}}{100}.
  \end{equation*}
  If you start with an initial amount ${\orange A}$ of money then
  after ${\redgreen t}$ years you have

  \begin{equation*}
    {\orange A}\times {\blue m}^{\redgreen t}
    = {\orange A}\times \left( 1 + \frac{{\red r}}{100} \right)^{\redgreen t}
  \end{equation*}

  \begin{enumerate}
    \setcounter{enumi}{4}
  \item If you invest \$1000 at $14$\% interest, how much
    will you have $5$ years later? ({\blue{}Guess!})
    \begin{center}
      A$\approx \$700$
      \quad 
      B$\approx \$1400$
      \quad 
      C$\approx \$1500$
      \quad 
      D$\approx\$1700$
      \quad 
      E$\approx \$2000$
      \pause
      % \quad
      \fbox{E}
    \end{center}
  \end{enumerate}

  After 5 years, you have
  \begin{equation*}
    \$1,000\times \left(1 + \frac{14}{100} \right)^5 
    = \$1,000 \times (1.14)^5.
  \end{equation*}
  How much is this? 
  \pause
  % \gap 
  {\red Smart way: 14\% in 1 year $\approx$ 7\% per year
    for 2.} 
  \vspace*{1in}
}


\frame{
  \frametitle{General Compound Interest}

  If the interest rate is ${\red r}\%$, then each year money {\blue
    m}ultiplies by
  \begin{equation*}
    {\blue m}=1+\frac{{\red r}}{100}.
  \end{equation*}
  If you start with an initial amount ${\orange A}$ of money then
  after ${\redgreen t}$ years you have

  \begin{equation*}
    {\orange A}\times {\blue m}^{\redgreen t}
    = {\orange A}\times \left( 1 + \frac{{\red r}}{100} \right)^{\redgreen t}
  \end{equation*}

  \begin{enumerate}
    \setcounter{enumi}{5}
  \item If you invest $\$1,000$  at $14$\% interest, how
    many years until you have \$$7,000$? 
    \begin{center}
      A$ = \log(7/1.14)$
      \quad 
      B$= \log(7)/\log(1.14)$
      \quad 
      C$= \log(1.14)/\log(7)$
      \\[0.5em]
      D$ = \log(7)-\log(1.14)$
      \quad 
      E$ =\text{other}$
      \pause
      \quad
      \fbox{B}
    \end{center}
  \end{enumerate}
  \vspace*{2in}
}

\section*{Population}

\frame{
  \frametitle{\S7.9: Population Growth}
  
  Assume each generation of bunnies has $3$ times as many bunnies as
  previous one.  Initially have $100$ bunnies. How many bunnies after
  $n$ generations?
  \begin{center}
    A$= 100\times 3n$
    \quad 
    B$ = 100+3n$
    \quad 
    C$ = 100(1+3n)$
    \\
    D$= 100^{3n}$
    \quad 
    E$= 100\times 3^n$
  \end{center}
  \alert{Answer:}\ \answer{E}
  \halfgap
 
  Start with $100$\\
  After 1 generation have $100\times 3$ bunnies\\ \pause
  After 2 generations have $100\times 3\times 3$ bunnies \\ \pause
  After 3 generations have $100\times 3\times 3\times 3$ bunnies \\ \pause
  So$\ldots$after $n$ generations have 
  \begin{equation*}
    100\times \underbrace{{\blue3\times3\times\cdots\times 3}}_{\text{$n$\ times}} 
    = 100\times 3^n\ \text{bunnies}.
  \end{equation*}

}

\frame{
  \frametitle{More Bunnies}

  We saw that:
  \begin{itemize}
  \item if we start with $100$\ bunnies, and
  \item the bunny population triples every generation,
  \end{itemize}
  then we have $100 \times 3^n$ bunnies after $n$ generations.
  \bigskip

  \begin{enumerate}
    \setcounter{enumi}{6}
  \item How many generations until there are $10^7 = 10,000,000$ bunnies?
    \begin{center}
      A$ = \log(5/3)$
      \qquad 
      B$ = 5-\log(3)$
      \qquad 
      C$= 5/\log(3)$ 
      \\
      D$ = 5/3$
      \qquad 
      E$ = 10^5/3$
      \pause
      % \quad
      % \fbox{C}
    \end{center}
    \begin{center}
      A$\approx 0.22$
      \qquad 
      B$\approx 4.52$
      \qquad 
      C$\approx 10.48$
      \\ 
      \ 
      \qquad
      D$\approx 1.67$
      \qquad 
      E$\approx 3,333$
      \pause
      \qquad
      \answer{C}
    \end{center}

  \end{enumerate}
}


\frame{
  \frametitle{Flu Outbreak}

  At the start of an outbreak of H1N1 flu in a large herd of cattle,
  there were $5$ infected individuals. The numbers doubles every $3$
  days. How many days until there are $80$ infected cows?
  \medskip

  \begin{center}
    A$ = \log(16)/\log(2)$
    \quad 
    B$=\log(16/2)$
    \quad 
    C$=16/\log(2)$
    \\[1em]
    D$=3\log(16)/\log(2)$
    \quad 
    E$ = \log(48/2)$
  \end{center}
  \pause
  \vspace*{0.5in}

  At the start of an outbreak of H1N1 flu in a large class of
  students, there were $5$ infected individuals. The numbers doubles
  every $3$ days. How many days until there are $80$ infected
  students?
  \smallskip

  \begin{center}
    A$ = \log(16)/\log(2)$
    \quad 
    B$=\log(16/2)$
    \quad 
    C$=16/\log(2)$
    \\[1em]
    D$=3\log(16)/\log(2)$
    \quad 
    E$ = \log(48/2)$
    \pause
    \qquad
    \fbox{D}
  \end{center}
  \vspace*{1in}
}



\end{document}

