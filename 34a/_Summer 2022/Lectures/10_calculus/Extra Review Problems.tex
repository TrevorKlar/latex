\documentclass{article}
\usepackage{graphicx}
\usepackage{amsmath}
\usepackage{amssymb}
\usepackage{tabto}
\usepackage{amsfonts}
%\usepackage{MnSymbol}
\usepackage{wasysym}
\usepackage{amsthm}
\usepackage{indentfirst}
\usepackage[utf8x]{inputenc}
\usepackage{caption}
\usepackage{subcaption}
\usepackage{adjustbox}
\usepackage{verbatim}
\usepackage{tikz, pgfplots}
\usepackage{tkz-euclide}
\usepgfplotslibrary{fillbetween}
\usepackage{multicol}
\usepackage{enumitem}
\usepackage{relsize}
\usepackage{xfrac}
\usepackage{array}
\usepackage{enumitem}
\usepackage{graphicx}
\usepackage[english]{babel}
\usepackage{fancyhdr}

\newtheorem{theorem}{Theorem}[section]
\newtheorem{corollary}{Corollary}[theorem]
\newtheorem{lemma}[theorem]{Lemma}

\graphicspath{ {images/} }


\renewcommand{\v}{\vec{v}}
\renewcommand{\u}{\vec{u}}
\renewcommand{\w}{\vec{w}}


\setlength\parindent{0pt}
\addtolength{\oddsidemargin}{-.875in}
\addtolength{\evensidemargin}{-.875in}
\addtolength{\textwidth}{1.75in}
\DeclareMathOperator{\im}{im}
\DeclareMathOperator{\Aut}{Aut} 
\DeclareMathOperator{\Span}{span}
\DeclareMathOperator{\End}{End}
\DeclareMathOperator{\lcm}{lcm}
\DeclareMathOperator{\Int}{int}
\DeclareMathOperator{\If}{if}
\DeclareMathOperator{\Or}{or}
\DeclareMathOperator{\Nd}{and}
\DeclareMathOperator{\st}{such\ that}
\DeclareMathOperator{\writhe}{writhe}
\DeclareMathOperator{\R}{\mathbb{R}}
\DeclareMathOperator{\PP}{\mathbb{P}}
\font\msbmx=msbm10 at 10pt
\textfont15=\msbmx
\mathchardef\subsetneqq="3F24
\mathchardef\subsetneq="3F28
\mathchardef\supsetneq="3F29

\newcommand\inv[1]{#1\raisebox{1.15ex}{$\scriptscriptstyle-\!1$}}
\newcommand{\pn}[1]{\left( #1 \right)}
\newcommand{\bk}[1]{\left[ #1 \right]}
\newcommand{\set}[1]{\left\{ #1 \right\}}
\newcommand{\vc}[1]{\left\langle #1 \right\rangle}
\newcommand{\abs}[1]{\left\lvert #1 \right\rvert}
\newcommand{\norm}[1]{\left\lVert #1 \right\rVert}
\newcommand{\mat}[1]{\ensuremath{ \begin{bmatrix} #1 \end{bmatrix} }}

\newcommand{\ansbox}[2]{\raisebox{-.5\height}{\framebox(#1,#2){}}}




\addtolength{\topmargin}{-.875in}
\addtolength{\textheight}{1.75in}
\pgfplotsset{soldot/.style={color=black,only marks,mark=*},
	holdot/.style={color=black,fill=white,only marks,mark=*},
	compat=1.12}
\pagestyle{fancy}
\lhead{Math 34A, UCSB}
\rhead{Spring 2022}

\pagenumbering{gobble}
\begin{document}
\newtheorem*{theorem*}{Theorem}
	
	
	\begin{enumerate} 
	\centerline{\Large{ Sums}}\vspace{12 pt}
	\item Write out the following sum:
    $$\sum_{n=1}^6 (n+1)(n+2)$$
    $$\phantom{\fbox{2\cdot3+3\cdot4+4\cdot5+5\cdot6+6\cdot7+7\cdot8}}$$
    \item Write out the following sum:
    $$ \sum_{m=2}^4 \frac{m^2}{1-m}. $$ 
    $$\phantom{\frac{2^2}{-1}+\frac{3^2}{-2}+\frac{4^2}{-3}}$$\vspace{50pt} \\ 

	
	\centerline{\Large{ Limits}}\vspace{12 pt}
	\item $\underset{x \rightarrow \infty}{\lim} \ 4 - \frac{1}{x}$ $\phantom{=\fbox{4}}$ 
	\item $\underset{h \rightarrow 0}{\lim} \ \frac{4h - 4h^2}{h}$ $\phantom{=\fbox{4}}$
	\item $\underset{h \rightarrow 0}{\lim} \ \frac{147h + 21h^2+h^3}{h}$ $\phantom{=\fbox{147}}$
	\vspace{50pt} \\ 

	
	\centerline{\Large{Average Speed}}\vspace{12 pt}
	\item Find the average speed of a race car over the time period from 2 seconds to 3 seconds if $f(t) = t^3$ is the distance from the starting line $t$ seconds after the start. \\ \\ \\ 
	\phantom{$\frac{27-8}{1}$=\fbox{$19$}}
	
	\end{enumerate}
    

	
\end{document}
	