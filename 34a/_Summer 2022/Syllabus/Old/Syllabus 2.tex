\documentclass[a4paper,12pt]{article}
\usepackage{amssymb,amsmath,amsthm}
\usepackage[tmargin=1.5in,bmargin=1.5in]{geometry}
\usepackage{graphicx}

\parindent=0in
\setlength{\parskip}{1em}

\geometry{
  body={7in, 10in},
  left=0.5in,
  top=0.75in
}
\pagestyle{empty}

\begin{document}

\begin{center} {\Large \scshape Math 34A Syllabus}
\end{center}

\vspace{0.75cm}

\begin{itemize}
\item {\bf Course:} Math 34B, Summer 2022
\item {\bf Instructor:} Trevor Klar, \texttt{trevorklar@math.ucsb.edu}
\item {\bf Teaching Assistant:} Alfredo Ramirez, \texttt{a\_ramirez730@ucsb.edu}
\item {\bf Lecture:} M T W R, 12:30-1:35 Ellison Hall 2626
\item {\bf Text:} UCSB's custom-printed book for this course: Calculus and Mathematical Reasoning for Social and Life Sciences (Cooper), now available free online!
\item {\bf Office Location:} South Hall 6431X (Grad Tower, 6th floor, blue side, first door on the right)
\item {\bf Office Hours:} After each class M T W R, 2:00-3:00 and by appointment (Really! Just send me an email and I'm happy to make an appointment).
\end{itemize}

\paragraph{Course Description}
The first of a two-course sequence, Math 34A is an introduction to differential and integral calculus with applications to modeling in the biological sciences.

\paragraph{About Me}
I’ve been teaching and tutoring math for 8 years. Before I came to grad school at UCSB I was a high school math teacher for 5 years in Lancaster, California. I earned my Master’s in Mathematics at UCSB in 2021, I have 2 cute kids, and I like 90s video games.

Every individual student comes to my classroom with a different set of circumstances, beliefs, abilities, cultures, challenges, talents, and other factors that will affect their learning and success. I believe that everybody can learn math, and I do my best to tailor my teaching practices to meet the needs of my diverse student population. Mistakes are a part of learning, and everyone's opinions, questions, and way of being is accepted. My classroom is a safe place to explore and learn. 

\paragraph{Grading}
\begin{center}
\begin{tabular}{lll}
Homework       & 20\% & due after every class before 11:00PM \\
Quizzes        & 10\% & weekly in section  \\ 
Midterm Exams & 40\% & 1st Midterm is \_\_\_day, June \_\_ in class \\
Final Exam  & 30\% & Tuesday, July \_ at 4:00PM
\end{tabular}
\end{center}
Letter grades will be assigned in the standard way:
\begin{center}
\begin{tabular}{llllllll}
A+       & 97\% \qquad \phantom{.}	& B+       & 87\% \qquad \phantom{.}	& C+      & 77\% \qquad \phantom{.} 	& D+       & 67\% \\
A        & 93\%  	& B        & 83\% 	& C        & 73\% 	& D       & 63\%\\ 
A-       & 90\%  	& B-       & 80\% 	& C-       & 70\% 	& D-       & 60\% \\
\end{tabular}
\end{center}
There will not be a curve.


\paragraph{Homework}
 The lectures are meant to be a supplement to the book, not a replacement. Please read the book as we go along. Homework will be due on class days before the end of the day. 
 
 \paragraph{Quizzes} 
 There will be a quiz in every section meeting (10 in total) based on the material learned so far (including review topics). The quizzes are designed to be a short checkup of how you're doing so far; they'll give you early feedback about how effective your studying methods are so far. If you're struggling, \textit{get help!}
 
 \paragraph{Exams} 
 Late homework will not be accepted except in the case of a medical emergency when a doctor's note can be provided. 




\paragraph{Resources}
\begin{itemize}
\item Instructor office hours (posted on Gauchospace)
\item TA office hours (posted on Gauchospace)
\item Math Lab: M-F, 12-5PM in South Hall
\item CLAS : Registration and schedules available at http://clas.sa.ucsb.edu/
\end{itemize} 



\end{document}
