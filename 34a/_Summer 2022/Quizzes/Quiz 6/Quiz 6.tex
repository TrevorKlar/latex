\documentclass{article}
\usepackage{graphicx}
\usepackage{amsmath}
\usepackage{amssymb}
\usepackage{tabto}
\usepackage{amsfonts}
%\usepackage{MnSymbol}
\usepackage{wasysym}
\usepackage{amsthm}
\usepackage{indentfirst}
\usepackage[utf8x]{inputenc}
\usepackage{caption}
\usepackage{subcaption}
\usepackage{adjustbox}
\usepackage{verbatim}
\usepackage{tikz, pgfplots}
\usepackage{tkz-euclide}
\usepgfplotslibrary{fillbetween}
\usepackage{multicol}
\usepackage{enumitem}
\usepackage{relsize}
\usepackage{xfrac}
\usepackage{array}
\usepackage{enumitem}
\usepackage{graphicx}
\usepackage[english]{babel}
\usepackage{fancyhdr}

\newtheorem{theorem}{Theorem}[section]
\newtheorem{corollary}{Corollary}[theorem]
\newtheorem{lemma}[theorem]{Lemma}

\graphicspath{ {images/} }


%\renewcommand{\v}{\vec{v}}
%\renewcommand{\u}{\vec{u}}
%\renewcommand{\w}{\vec{w}}


\setlength\parindent{0pt}
\addtolength{\oddsidemargin}{-.875in}
\addtolength{\evensidemargin}{-.875in}
\addtolength{\textwidth}{1.75in}
\DeclareMathOperator{\im}{im}
\DeclareMathOperator{\Aut}{Aut} 
\DeclareMathOperator{\Span}{span}
\DeclareMathOperator{\End}{End}
\DeclareMathOperator{\lcm}{lcm}
\DeclareMathOperator{\Int}{int}
\DeclareMathOperator{\If}{if}
\DeclareMathOperator{\Or}{or}
\DeclareMathOperator{\Nd}{and}
\DeclareMathOperator{\st}{such\ that}
\DeclareMathOperator{\writhe}{writhe}
\DeclareMathOperator{\R}{\mathbb{R}}
\DeclareMathOperator{\PP}{\mathbb{P}}
\font\msbmx=msbm10 at 10pt
\textfont15=\msbmx
\mathchardef\subsetneqq="3F24
\mathchardef\subsetneq="3F28
\mathchardef\supsetneq="3F29

\newcommand\inv[1]{#1\raisebox{1.15ex}{$\scriptscriptstyle-\!1$}}
\newcommand{\pn}[1]{\left( #1 \right)}
\newcommand{\bk}[1]{\left[ #1 \right]}
\newcommand{\set}[1]{\left\{ #1 \right\}}
\newcommand{\vc}[1]{\left\langle #1 \right\rangle}
\newcommand{\abs}[1]{\left\lvert #1 \right\rvert}
\newcommand{\norm}[1]{\left\lVert #1 \right\rVert}
\newcommand{\mat}[1]{\ensuremath{ \begin{bmatrix} #1 \end{bmatrix} }}

\newcommand{\ansbox}[2]{\raisebox{-.5\height}{\framebox(#1,#2){}}}




\addtolength{\topmargin}{-.875in}
\addtolength{\textheight}{1.75in}
\pgfplotsset{soldot/.style={color=black,only marks,mark=*},
	holdot/.style={color=black,fill=white,only marks,mark=*},
	compat=1.12}
\pagestyle{fancy}
\lhead{Math 34A, UCSB}
\rhead{Spring 2022}

\pagenumbering{gobble}
\begin{document}
\newtheorem*{theorem*}{Theorem}


	\centerline{\Large{ Quiz 6}}\vspace{12 pt}
	\begin{tabular}{ll}
    {\bf Name}: \ansbox{230}{35} %\hspace{1.8in} 
    & {\bf Perm Number}: \ansbox{120}{35} 
    \end{tabular} \\ \\
	
	
	

{\Large 1)}	According to some reports, in some parts of the world the number of reported cases of a disease grows by approximately $26\%$ every 2 days during its initial days of outbreak. This is to say, the number of people reported to have the disease is multiplied by about $1.26$ every two days. If the growth continues this way, how many days would it take for the number of reported cases to quadruple?
	\vfill
	\hfill \ansbox{160}{35}
	\vspace*{.25 in}

{\Large 2)} Solve the below equation for $x$. 
    $$3\cdot 4^{2x-3}=9$$

    \vfill 
    \hfill $y=\quad$\ansbox{160}{35}
	\vspace*{.25 in}
	
	\pagebreak
{\Large 3)} I have $x$ dollars. Veggie burgers cost $v$ dollars each and sodas costs $s$ dollars each. If I buy $y$ veggie burgers, how many sodas can I buy?

    \vfill 
    \hfill $y=\quad$\ansbox{160}{35}
	\vspace*{.25 in}
	
{\Large 4)} Your friend Brian is taking a business class this summer and wants to use his skills to start a hot dog stand. He tells you that if he charges $p$ dollars for a hot dog, he can find an equation for the total number of hot dogs he would sell, and use it to compute his total revenue from selling all those hot dogs. Let $R(p)$ represent the total revenue of the hot dog stand (in dollars), if $p$ is the price of a hot dog. What is the interpretation of $R(5)=200$, $R'(5)=-10$, in words?

    %\vfill 
    \vspace*{.25 in}
    \hfill \ansbox{\textwidth}{300}
	%\vspace*{.5 in}
    

	
\end{document}
	