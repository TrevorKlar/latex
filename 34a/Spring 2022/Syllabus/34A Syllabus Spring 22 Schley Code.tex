\documentclass[a4paper,12pt]{article}
\usepackage{amssymb,amsmath,amsthm}
\usepackage[tmargin=1.5in,bmargin=1.5in]{geometry}
\usepackage{graphicx}

\parindent=0in
\setlength{\parskip}{1em}

\geometry{
  body={7in, 10in},
  left=0.5in,
  top=0.75in
}
\pagestyle{empty}

\begin{document}

\begin{center} {\Large \scshape Math 34A Course Information}
\end{center}

\vspace{0.75cm}

\begin{itemize}
\item {\bf Course:} Math 34B, Spring 2022
\item {\bf Instructor:} Nathan Schley
\item {\bf Instructor Contact:} schley@math.ucsb.edu
\item {\bf Teaching Assistants:} Jeremy Khoo, Trevor Klar, and Daniel Ralston
\item {\bf Lecture:} T R, 2:00-3:15 Chem 1179
\item {\bf Text:} UCSB's custom-printed book for this course: Calculus and Mathematical Reasoning for Social and Life Sciences (Cooper), now available free online!
\item {\bf Office Location:} South Hall 6701
\end{itemize}

\paragraph{Grading}
The grading scheme is as follows:
\begin{center}
\begin{tabular}{lll}
Homework       & 20\% & due Tuesdays and Thursdays before 11:00PM \\
Quizzes        & 10\% & weekly in section  \\ 
Midterm Exams & 40\% & 1st Midterm is Tuesday, April 19 in class \\
Final Exam  & 30\% & Tuesday, June 7 at 4:00PM
\end{tabular}
\end{center}
Letter grades are determined using a standard 10-point scale, uncurved.



\paragraph{Reading and Homework}
 The lectures are meant to be a supplement to the book, not a replacement. Please read the book as well as we go along. Homework will be due on class days before the end of the day. There will be weekly quizzes in section based on the last week's material and review topics. Late homework will not be accepted except in the case of a medical emergency when a doctor's note can be provided. 




\paragraph{Resources}
\begin{itemize}
\item Instructor office hours (posted on Gauchospace)
\item TA office hours (posted on Gauchospace)
\item Math Lab: M-F, 12-5PM in South Hall
\item CLAS : Registration and schedules available at http://clas.sa.ucsb.edu/
\end{itemize} 



\end{document}
