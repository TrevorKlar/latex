\documentclass{article}
\usepackage{graphicx}
\usepackage{amsmath}
\usepackage{amssymb}
\usepackage{tabto}
\usepackage{amsfonts}
%\usepackage{MnSymbol}
\usepackage{wasysym}
\usepackage{amsthm}
\usepackage{indentfirst}
\usepackage[utf8x]{inputenc}
\usepackage{caption}
\usepackage{subcaption}
\usepackage{adjustbox}
\usepackage{verbatim}
\usepackage{tikz, pgfplots}
\usepackage{tkz-euclide}
\usepgfplotslibrary{fillbetween}
\usepackage{multicol}
\usepackage{enumitem}
\usepackage{relsize}
\usepackage{xfrac}
\usepackage{array}
\usepackage{enumitem}
\usepackage{graphicx}
\usepackage[english]{babel}
\usepackage{fancyhdr}

\newtheorem{theorem}{Theorem}[section]
\newtheorem{corollary}{Corollary}[theorem]
\newtheorem{lemma}[theorem]{Lemma}

\graphicspath{ {images/} }


\renewcommand{\v}{\vec{v}}
\renewcommand{\u}{\vec{u}}
\renewcommand{\w}{\vec{w}}


\setlength\parindent{0pt}
\addtolength{\oddsidemargin}{-.875in}
\addtolength{\evensidemargin}{-.875in}
\addtolength{\textwidth}{1.75in}
\DeclareMathOperator{\im}{im}
\DeclareMathOperator{\Aut}{Aut} 
\DeclareMathOperator{\Span}{span}
\DeclareMathOperator{\End}{End}
\DeclareMathOperator{\lcm}{lcm}
\DeclareMathOperator{\Int}{int}
\DeclareMathOperator{\If}{if}
\DeclareMathOperator{\Or}{or}
\DeclareMathOperator{\Nd}{and}
\DeclareMathOperator{\st}{such\ that}
\DeclareMathOperator{\writhe}{writhe}
\DeclareMathOperator{\R}{\mathbb{R}}
\DeclareMathOperator{\PP}{\mathbb{P}}
\font\msbmx=msbm10 at 10pt
\textfont15=\msbmx
\mathchardef\subsetneqq="3F24
\mathchardef\subsetneq="3F28
\mathchardef\supsetneq="3F29

\newcommand\inv[1]{#1\raisebox{1.15ex}{$\scriptscriptstyle-\!1$}}
\newcommand{\pn}[1]{\left( #1 \right)}
\newcommand{\bk}[1]{\left[ #1 \right]}
\newcommand{\set}[1]{\left\{ #1 \right\}}
\newcommand{\vc}[1]{\left\langle #1 \right\rangle}
\newcommand{\abs}[1]{\left\lvert #1 \right\rvert}
\newcommand{\norm}[1]{\left\lVert #1 \right\rVert}
\newcommand{\mat}[1]{\ensuremath{ \begin{bmatrix} #1 \end{bmatrix} }}

\newcommand{\ansbox}[2]{\raisebox{-.5\height}{\framebox(#1,#2){}}}




\addtolength{\topmargin}{-.875in}
\addtolength{\textheight}{1.75in}
\pgfplotsset{soldot/.style={color=black,only marks,mark=*},
	holdot/.style={color=black,fill=white,only marks,mark=*},
	compat=1.12}
\pagestyle{fancy}
\lhead{Math 34A, UCSB}
\rhead{Spring 2022}

\pagenumbering{gobble}
\begin{document}
\newtheorem*{theorem*}{Theorem}


	\centerline{\Large{ Quiz 2}}\vspace{12 pt}
	\begin{tabular}{ll}
    {\bf Name}: \ansbox{230}{35} %\hspace{1.8in} 
    & {\bf Perm Number}: \ansbox{120}{35} 
    \end{tabular} \\ \\
	
	
	

{\Large 1)}	Simplify the expression below and write it as a percent. 
    $$\frac{1}{20}\left( 12^{-1} + 18^{-1} \right)^{-1}$$
    \vfill \\
    
\phantom{.} \hfill \ansbox{160}{35} \\ 
{\Large 2)} \ Multiply out and simplify. writing the simplified answer in the box. 
	{\Large $$(x-1)(x^3+x^2+x+1)$$}
    \\
    \vfill \\
    
\phantom{.} \hfill \ansbox{160}{35} \\ 


\newpage
{\Large 3)} \ Multiply out and simplify. writing the simplified answer in the box. 
	{\Large $$\left(\frac{5a - 3ab}{2a}\right)\left(\frac{14a-28b}{7}\right)$$}
    \\
\vfill \\
    \vfill \\
    \phantom{.} \hfill \ansbox{200}{70} \\ 
    
{\Large 4)} A bad guy leaves Isla Vista at midnight, traveling on the 101. To avoid being pulled over, he maintains a speed of 60 miles per hour. At 1:20AM, Jason Borne leaves Isla Vista in the same direction at 100 miles per hour. At what time will Jason catch up to the bad guy? \\
	
	\vfill \\
	\vfill \\
	${\phantom{..............}}$ \hfill \ansbox{160}{35} \\ 
    
    

	
\end{document}
	