\documentclass[12pt,letterpaper]{article}

% \input{worksheet-setup.tex}
\usepackage{fancyhdr,fancybox}
\usepackage{amsmath}
\usepackage{amsthm}
\usepackage{amssymb}
\usepackage{tikz}

\usepackage{enumitem}
\usepackage{multicol}
\usepackage[export]{adjustbox}

\usepackage[more]{tasks}
\NewTasks[style=enumerate,counter-format=tsk[1].,label-format=\bfseries,label-width=2.6ex,column-sep=2ex,before-skip=2pt,
      item-indent = {1em}]{problems}[\problem](2)
\ExplSyntaxOn
\newcommand{\ShowProbNumber}{
  \int_use:N \g__tasks_int
}
\newcommand{\SetProbNumber}[1]{
  \int_gset:Nn \g__tasks_int {#1 + 1}
}
\ExplSyntaxOff

%% Useful packages
\usepackage{amssymb, amsmath, amsthm} 
%\usepackage{graphicx}  %%this is currently enabled in the default document, so it is commented out here. 
\usepackage{calrsfs}
\usepackage{braket}
\usepackage{mathtools}
\usepackage{lipsum}
\usepackage{tikz}
\usetikzlibrary{cd}
\usepackage{verbatim}
%\usepackage{ntheorem}% for theorem-like environments
\usepackage{mdframed}%can make highlighted boxes of text
%Use case: https://tex.stackexchange.com/questions/46828/how-to-highlight-important-parts-with-a-gray-background
\usepackage{wrapfig}
\usepackage{centernot}
\usepackage{subcaption}%\begin{subfigure}{0.5\textwidth}
\usepackage{pgfplots}
\pgfplotsset{compat=1.13}
\usepackage[colorinlistoftodos]{todonotes}
\usepackage[colorlinks=true, allcolors=blue]{hyperref}
\usepackage{xfrac}					%to make slanted fractions \sfrac{numerator}{denominator}
\usepackage{enumitem}            
    %syntax: \begin{enumerate}[label=(\alph*)]
    %possible arguments: f \alph*, \Alph*, \arabic*, \roman* and \Roman*
\usetikzlibrary{arrows,shapes.geometric,fit}

\DeclareMathAlphabet{\pazocal}{OMS}{zplm}{m}{n}
%% Use \pazocal{letter} to typeset a letter in the other kind 
%%  of math calligraphic font. 

%% This puts the QED block at the end of each proof, the way I like it. 
\renewenvironment{proof}{{\bfseries Proof}}{\qed}
\makeatletter
\renewenvironment{proof}[1][\bfseries \proofname]{\par
  \pushQED{\qed}%
  \normalfont \topsep6\p@\@plus6\p@\relax
  \trivlist
  %\itemindent\normalparindent
  \item[\hskip\labelsep
        \scshape
    #1\@addpunct{}]\ignorespaces
}{%
  \popQED\endtrivlist\@endpefalse
}
\makeatother

%% This adds a \rewnewtheorem command, which enables me to override the settings for theorems contained in this document.
\makeatletter
\def\renewtheorem#1{%
  \expandafter\let\csname#1\endcsname\relax
  \expandafter\let\csname c@#1\endcsname\relax
  \gdef\renewtheorem@envname{#1}
  \renewtheorem@secpar
}
\def\renewtheorem@secpar{\@ifnextchar[{\renewtheorem@numberedlike}{\renewtheorem@nonumberedlike}}
\def\renewtheorem@numberedlike[#1]#2{\newtheorem{\renewtheorem@envname}[#1]{#2}}
\def\renewtheorem@nonumberedlike#1{  
\def\renewtheorem@caption{#1}
\edef\renewtheorem@nowithin{\noexpand\newtheorem{\renewtheorem@envname}{\renewtheorem@caption}}
\renewtheorem@thirdpar
}
\def\renewtheorem@thirdpar{\@ifnextchar[{\renewtheorem@within}{\renewtheorem@nowithin}}
\def\renewtheorem@within[#1]{\renewtheorem@nowithin[#1]}
\makeatother

%% This makes theorems and definitions with names show up in bold, the way I like it. 
\makeatletter
\def\th@plain{%
  \thm@notefont{}% same as heading font
  \itshape % body font
}
\def\th@definition{%
  \thm@notefont{}% same as heading font
  \normalfont % body font
}
\makeatother

%===============================================
%==============Shortcut Commands================
%===============================================
\newcommand{\ds}{\displaystyle}
\newcommand{\B}{\mathcal{B}}
\newcommand{\C}{\mathbb{C}}
\newcommand{\F}{\mathbb{F}}
\newcommand{\N}{\mathbb{N}}
\newcommand{\R}{\mathbb{R}}
\newcommand{\Q}{\mathbb{Q}}
\newcommand{\T}{\mathcal{T}}
\newcommand{\Z}{\mathbb{Z}}
\renewcommand\qedsymbol{$\blacksquare$}
\newcommand{\qedwhite}{\hfill\ensuremath{\square}}
\newcommand*\conj[1]{\overline{#1}}
\newcommand*\closure[1]{\overline{#1}}
\newcommand*\mean[1]{\overline{#1}}
%\newcommand{\inner}[1]{\left< #1 \right>}
\newcommand{\inner}[2]{\left< #1, #2 \right>}
\newcommand{\powerset}[1]{\pazocal{P}(#1)}
%% Use \pazocal{letter} to typeset a letter in the other kind 
%%  of math calligraphic font. 
\newcommand{\cardinality}[1]{\left| #1 \right|}
\newcommand{\domain}[1]{\mathcal{D}(#1)}
\newcommand{\image}{\text{Im}}
\newcommand{\inv}[1]{#1^{-1}}
\newcommand{\preimage}[2]{#1^{-1}\left(#2\right)}
\newcommand{\script}[1]{\mathcal{#1}}


\newenvironment{highlight}{\begin{mdframed}[backgroundcolor=gray!20]}{\end{mdframed}}

\DeclarePairedDelimiter\ceil{\lceil}{\rceil}
\DeclarePairedDelimiter\floor{\lfloor}{\rfloor}

%===============================================
%===============My Tikz Commands================
%===============================================
\newcommand{\drawsquiggle}[1]{\draw[shift={(#1,0)}] (.005,.05) -- (-.005,.02) -- (.005,-.02) -- (-.005,-.05);}
\newcommand{\drawpoint}[2]{\draw[*-*] (#1,0.01) node[below, shift={(0,-.2)}] {#2};}
\newcommand{\drawopoint}[2]{\draw[o-o] (#1,0.01) node[below, shift={(0,-.2)}] {#2};}
\newcommand{\drawlpoint}[2]{\draw (#1,0.02) -- (#1,-0.02) node[below] {#2};}
\newcommand{\drawlbrack}[2]{\draw (#1+.01,0.02) --(#1,0.02) -- (#1,-0.02) -- (#1+.01,-0.02) node[below, shift={(-.01,0)}] {#2};}
\newcommand{\drawrbrack}[2]{\draw (#1-.01,0.02) --(#1,0.02) -- (#1,-0.02) -- (#1-.01,-0.02) node[below, shift={(+.01,0)}] {#2};}

%***********************************************
%**************Start of Document****************
%***********************************************
 %find me at /home/trevor/texmf/tex/latex/tskpreamble_nothms.tex

%%
%% Page set-up:
%%
\pagestyle{empty}
\lhead{\textsc{34A - Calculus and More Calculus!}} %=================UPDATE THIS=================%
\rhead{\textsc{Winter 2020}}
%\chead{\Large\textbf{A New Integration Technique \\ }}
\rfoot{trevorklar@math.ucsb.edu}
\renewcommand{\headrulewidth}{1pt}

\setlength{\parindent}{0in}
\setlength{\textwidth}{7in}
\setlength{\evensidemargin}{-0.25in}
\setlength{\oddsidemargin}{-0.25in}
\setlength{\parskip}{.5\baselineskip}
\setlength{\topmargin}{-0.5in}
\setlength{\textheight}{9in}

\setlist[enumerate,1]{label=\textbf{\arabic*.}}

\begin{document}
\thispagestyle{fancy}
\begin{center}
3B - Calculus for Social and Life Sciences\\
Week 7 %=================UPDATE THIS=================%
\end{center}

\hrule

\begin{center}
\begin{tabular}{|rl|}
\hline
\multicolumn{2}{|c|}{Contact Information} \\
\hline
\bf{TA's name:} & Trevor Klar \\
\bf{Email:} & trevorklar@math.ucsb.edu \\
\bf{Office hours:} & Wednesdays 2:00-3:00 \\
\bf{Math Lab hours:} & Tuesdays 3:00-5:00 \\
\bf{Office:} & South Hall 6431x \\
\hline
\end{tabular}
\end{center}

 %=================START OF WORKSHEET=================%

Remember, derivatives are basically just slopes.
\begin{align*}
\text{slope }&= \frac{\Delta y}{\Delta x}= \frac{y_2-y_1}{x_2-x_1} \\ 
\text{derivative }&= \frac{dy}{dx} = \lim_{\Delta x\to 0} \frac{f(x+\Delta x)-f(x)}{\Delta x}
\end{align*}
You can think of a derivative as \textbf{a slope} or \textbf{a rate of change,} when interpreting problems. 

%Helpful hints and formulas go here
\bigskip

\hrule
\textbf{Exercise.} A goblin catapult sits at at the top of a 200 foot cliff, overlooking the enemy. Because the goblins forgot to build any safety features whatsoever for the catapult, one of the goblin artillerymen slips and falls off the cliff. His height in feet is given by the function $h(t)=200-16t^2$, where $t$ is the amount of time he has been falling, in seconds. 

Find the average velocity of the goblin's fall, from
	\begin{problems}
	\problem $t=1$ to $t=2$. 
	\problem $t=1$ to $t=1.1$. 
	\problem $t=1$ to $t=1.01$. 
	\problem $t=1$ to $t=1.001$. 	
	\end{problems}
	
	
	\begin{enumerate}%[label=\arabic*.]
	\setcounter{enumi}{4}
	\item What would you guess is the goblin's exact velocity at $t=1$? [If you're not sure, use a calculator to find the average for $t$ going from 1 to 1.01 and 1 to 1.001.]
	\vfill
	
	\item Find the average velocity of the goblin's fall from time $t$, to time $t+h$. [Here we are thinking of $h$ as a very small number.]
  \vfill	
	

%\begin{problems}
%\problem\SetProbNumber{1} $\log(4.56)$
%\problem $\log(999)$
%\problem $\log(\text{1 billion})$
%\problem $\log(0.000738)$ 
%\problem $\log(2.125)$ [Hint: Linear interpolation]
%\problem $\text{antilog}(0.2945)$
%\problem $\sqrt{10}$
%\problem $10^{0.49}$
%\problem $10^{\sfrac{2}{3}}$
%\problem $(2.25)(8.12)$
%\problem $\dfrac{-4.33}{2.01}$
%\problem $\dfrac{1}{5.72}$
%\problem $\dfrac{2.01}{4.33}$
%\problem $\left (\dfrac{12 \text{ in}}{1 \text{ ft}} \right )\left (\dfrac{5280 \text{ ft}}{1 \text{ mi}}\right ) \left (\dfrac{1 \text{ mi}}{1.609 \text{ km}}\right )$
%\end{problems}

\end{enumerate}









\end{document}


%%% Local Variables: 
%%% mode: latex
%%% TeX-master: t
%%% End: 
