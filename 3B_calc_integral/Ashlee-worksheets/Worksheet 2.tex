\documentclass[12pt]{report}
\makeindex

\usepackage{amsmath, amsfonts, amssymb, amstext, amscd, amsthm, makeidx, graphicx, hyperref, url, enumerate, mathabx, bbm}
\allowdisplaybreaks

\setcounter{chapter}{0}
\topmargin -.75in
\textheight 9.25in
\oddsidemargin 0.0in
\textwidth 6.5in

\newcommand{\al}{\alpha}
\newcommand{\be}{\beta}
\newcommand{\ga}{\gamma}
\newcommand{\ov}{\overline}
\newcommand{\ep}{\epsilon}
\newcommand{\N}{\mathbb N}
\newcommand{\C}{\mathbb C}
\newcommand{\R}{\mathbb R}
\newcommand{\Z}{\mathbb Z}
\newcommand{\Q}{\mathbb Q}
\newcommand{\la}{\langle}
\newcommand{\ra}{\rangle}
\newcommand{\da}{\lambda}
\newcommand{\I}{\mathbb I}
\newcommand{\sP}{\mathcal P}
\newcommand{\ds}{\displaystyle}
\newcommand{\id}{\mathbbm{1}}
\newcommand{\inc}{\hookrightarrow}
\newcommand{\homm}{\simeq}
\newcommand{\til}{\widetilde}
\newcommand{\om}{\omega}
\newcommand{\defn}{\noindent\textbf{Defn:}}
\newcommand{\sA}{\mathcal A}
\newcommand{\sE}{\mathcal E}
\newcommand{\thm}{\noindent\textbf{Theorem:}}


\begin{document}
\begin{center}
\textbf{{\LARGE MATH 3B}  \hfill} Discussion Worksheet - Thursday, April 12\\

\end{center}


\noindent\textbf{Fundamental Theorem of Calculus}

\begin{itemize}

\item Fundamental Theorem of Calculus Part 1: If $g(x) = \ds \int_{a}^x f(t) \, dt$ then \hrulefill %$g'(x) = f(x)$. 

\item BE CAREFUL: If $\ds h(x) = \int^{\sin(x)}_1 4x \, dx $ then $h'(x) = $ \underline{\hspace{3cm}}

\item Fundamental Theorem of Calculus Part 2: If $F$ is an antiderivative of $f$, then $$\hspace{-1in}\ds \int_a^b f(x) \, dx = \text{\underline{\hspace{3cm}}}$$% F(a) - F(b)

\item What's the difference between definite and indefinite integrals? \hrulefill 
\\ \mbox{} \hrulefill
\\ \mbox{} \hrulefill

\item You Try!

(1) $\ds \int_0^2 x(2 + x^2) \, dx$ \hspace{1.5in} (2) Find $h'(x)$ if $h(x) = \ds \int_0^{x^2} \sqrt{1+r^3} \, dr$

\vspace{1in}

(3) $\ds\int \sqrt[3]{x} \, dx$

\end{itemize}


\noindent\textbf{$U$-Substitutions:}

\begin{itemize}

\item Strategy: (1) Choose $u$ to be \hrulefill \\
(2) Substitute. You might need to \hrulefill \\
(3) Evaluate the integral.

\item Example: $\ds \int \sec^2(10x)\tan^7(10x) \, dx$

\bigskip\bigskip\bigskip

\item You Try!

(1) $\ds \int \dfrac{x}{x^2 + 1} \, dx$ \hspace{2in} (2) $\ds\int \tan(x) \, dx$

\bigskip\bigskip\bigskip


\end{itemize}



% Net Change Theorem: The integral of a rate of change is the net change $\ds \int_a^b F'(x) \, dx = F(b) - F(a)$








\newpage


\noindent\textbf{Definite Integrals W/ U-Substitutions:}


\begin{itemize}

\item Strategy: %choose the appropriate U using strategy above, when you do the u-sub, substitute limits into U as well for new limits

\bigskip\bigskip

\item Example: $\ds \int_{-\pi/4}^{\pi/4} \sec^2(10x)\tan^7(10x) \, dx$

\vspace{1in}

\item You Try!

(1) $\ds \int_0^{\pi} \sec^2(t/4) \, dt$ \hspace{2in} (2) $\ds \int_0^2 (x-1)^{25} \, dx$

\vspace{1.5in}

\end{itemize}


\noindent\textbf{Integrals of Piecewise Functions and the Absolute Value Function:}

\begin{itemize}

\item Absolute value: $\ds |x| = \left\{ \begin{array}{cl} -x & x< 0 \\ x & x \geq 0 \end{array} \right. \text{   so   } \int_{-5}^5 |x| \, dx = $

\bigskip

\item Piecewise Functions (example): If $\ds f(x) = \left\{ \begin{array}{cl} -x+3 & x \leq -1 \\ x^2 + 3 & x > -1 \end{array} \right. \text{ then }$ 

\medskip

$\ds \int_{-2}^2 f(x) \, dx= $

\vspace{1in}

\item You Try! $\ds \int_{-3}^4 |x^2 - 4| \, dx$

\vspace{1.25in}


\end{itemize}




%\noindent\textbf{Problem 1: }
%
%\medskip
%
%(a) Use the right endpoint Riemann sum approximation with $3$ subintervals to approximate the area under the curve $f(x) = x^2$ between $x = 1$ and $x = 3$. Do you expect your answer to be an over or underestimate of the area?
%
%\medskip
%
%(b) Find the area under the curve $f(x) = x^2$ between $x = 1$ and $x=3$ using the definition of the integral as the limit of a Riemann sum. Was your guess in (a) correct? Why?\
%
%\bigskip\bigskip
%
%\noindent\textbf{Problem 2:} Compute each of the following integrals:
%
%\medskip
%
%(a) $\ds \int \sec(x)\tan(x) + x^{\sqrt2} - \sqrt[3]{x} + \ln(2) \, dx$\\
%
%(b) $\ds \int \sec^2(x)e^{\tan(x)} \, dx$\\
%
%(c) $\ds \int_1^2 \dfrac{x}{\sqrt{x+3}} \, dx$\\
%
%(d) $\ds \int_{-6}^{6} x^4\tan(x) \, dx$\\
%
%\bigskip\bigskip
%
%\noindent\textbf{Problem 3:} The sum $\ds \sum_{i = 1}^n \left(\dfrac6n\right) \sqrt{9 + \dfrac{6i}{n}}$ approximates what area (Hint: there is more than one correct answer; be sure to give both the integral and limits of integration)?
%
%\bigskip\bigskip
%
%\noindent\textbf{Problem 4:} Find $\dfrac{d}{dx}\ds \int_{ax+b}^2 \dfrac{\sqrt{t^3}}{4\cos(t) + 3} \, dt$.
%
%\bigskip\bigskip
%
%\noindent\textbf{Problem 5:} The velocity function for a particle moving along a line is given by $v(t) = -t^2 + 5t - 6$. Find both the (i) displacement and (ii) the total distance traveled by the particle on the time interval $[-1,5]$.
%
%\bigskip\bigskip
%
%\noindent\textbf{Problem 6:} Sketch the region enclosed by the curves $x = 24-3y$ and $x = y^2 - 4$ and find the area of the region.
%




\end{document}