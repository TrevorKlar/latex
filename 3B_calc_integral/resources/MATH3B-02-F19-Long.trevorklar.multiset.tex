%%%%%%%%%%%%%%%%%%%%%%%%%%%%%%%%%%%%%%%%%%%%%%%%%%%%%%%%%%%%%%%%%%%%%%%%%%%%%%%%
% WeBWorK Online Homework Delivery System
% Copyright � 2000-2007 The WeBWorK Project, http://openwebwork.sf.net/
% $CVSHeader: webwork2/conf/snippets/hardcopyPreamble.tex,v 1.3 2005/09/17 20:12:01 gage Exp $
% 
% This program is free software; you can redistribute it and/or modify it under
% the terms of either: (a) the GNU General Public License as published by the
% Free Software Foundation; either version 2, or (at your option) any later
% version, or (b) the "Artistic License" which comes with this package.
% 
% This program is distributed in the hope that it will be useful, but WITHOUT
% ANY WARRANTY; without even the implied warranty of MERCHANTABILITY or FITNESS
% FOR A PARTICULAR PURPOSE.  See either the GNU General Public License or the
% Artistic License for more details.
%%%%%%%%%%%%%%%%%%%%%%%%%%%%%%%%%%%%%%%%%%%%%%%%%%%%%%%%%%%%%%%%%%%%%%%%%%%%%%%%

\batchmode
\documentclass[10pt,dvips]{amsart}
\usepackage{amsmath,amsfonts,amssymb,multicol}
\usepackage[pdftex]{graphicx}
\usepackage{epstopdf}  % allows use of eps files with pdftex
\usepackage{epsf}
\usepackage{epsfig}
\usepackage{pslatex}
\usepackage[utf8]{inputenc}
\pagestyle{plain}
\textheight 9in
\oddsidemargin = -0.42in
\evensidemargin = -0.42in
\textwidth= 7.28in
\columnsep = .25in
\columnseprule = .4pt
\def\endline{\bigskip\hrule width \hsize height 0.8pt }
\newcommand{\lt}{<}
\newcommand{\gt}{>}
\newcommand{\less}{<}
\newcommand{\grt}{>}

% BEGIN capa tex macros

\newcommand{\capa}{{\sl C\kern-.10em\raise-.00ex\hbox{\rm A}\kern-.22em%
{\sl P}\kern-.14em\kern-.01em{\rm A}}}
  
\newenvironment{choicelist}
{\begin{list}{}
	{\setlength{\rightmargin}{0in}\setlength{\leftmargin}{0.13in}
	\setlength{\topsep}{0.05in}\setlength{\itemsep}{0.022in}
	\setlength{\parsep}{0in}\setlength{\belowdisplayskip}{0.04in}
	\setlength{\abovedisplayskip}{0.05in}
	\setlength{\abovedisplayshortskip}{-0.04in}
	\setlength{\belowdisplayshortskip}{0.04in}}
	}
{\end{list}}

% END capa tex macros 

\begin{document}
\voffset=-0.8in
\newpage
\setcounter{page}{1}
\begin{multicols}{2}
\columnwidth=\linewidth
%% decoded old answers, saved. (keys = 
 \end{multicols}

\noindent {\large \bf Trevor Klar}
\hfill
{\large \bf {MATH3B-02-F19-Long}}
% Uncomment the line below if this course has sections. Note that this is a comment in TeX mode since this is only processed by LaTeX
%   {\large \bf { Section:  } }
\par
\noindent{\large \bf {Assignment Revision\_Chain\_Rule  due 10/09/2019 at 10:00pm PDT}}
\par\noindent \bigskip
% Uncomment and edit the line below if this course has a web page. Note that this is a comment in TeX mode.
%See the course web page for information http://yoururl/yourcourse



 \begin{multicols}{2}
\columnwidth=\linewidth
%%%%%%%%%%%%%%%%%%%%%%%%%%%%%%%%%%%%%%%%%%%%%%%%%%%%%%%%%%%%%%%%%%%%%%%%%%%%%%%%
% WeBWorK Online Homework Delivery System
% Copyright � 2000-2007 The WeBWorK Project, http://openwebwork.sf.net/
% $CVSHeader: webwork2/conf/snippets/hardcopyProblemDivider.tex,v 1.3 2004/06/24 21:10:50 dpvc Exp $
% 
% This program is free software; you can redistribute it and/or modify it under
% the terms of either: (a) the GNU General Public License as published by the
% Free Software Foundation; either version 2, or (at your option) any later
% version, or (b) the "Artistic License" which comes with this package.
% 
% This program is distributed in the hope that it will be useful, but WITHOUT
% ANY WARRANTY; without even the implied warranty of MERCHANTABILITY or FITNESS
% FOR A PARTICULAR PURPOSE.  See either the GNU General Public License or the
% Artistic License for more details.
%%%%%%%%%%%%%%%%%%%%%%%%%%%%%%%%%%%%%%%%%%%%%%%%%%%%%%%%%%%%%%%%%%%%%%%%%%%%%%%%

\medskip
\goodbreak
\hrule
\nobreak
\smallskip
%% decoded old answers, saved. (keys = 
{\bf 1. {\footnotesize (0 pts) Library\-/UVA-Stew5e\-/setUVA-Stew5e-C03S08-DerivLogs\-/3-8-27.pg}}\newline Suppose that
\[f(x) = x^2 \ln(9 - 5 x^2).\]
Find  \(f'(x)\), and use interval notation to give the domain of \(f\).
\leavevmode\\\relax 
{\bf  Note: } When entering interval notation in WeBWorK, use
{\bf  I } for \(\infty\), {\bf  -I } for \(-\infty\),
and {\bf  U } for the union symbol.  If the set is empty,
enter "{}" without the quotation marks.
\par 
\(f'(x)\) = \mbox{\parbox[t]{17.5ex}{\hrulefill}}
\par 
Domain = \mbox{\parbox[t]{17.5ex}{\hrulefill}}
\par 

\par{\small{\it Answer(s) submitted:}
\vspace{-\parskip}\begin{itemize}
\item\begin{verbatim}\end{verbatim}
\item\begin{verbatim}\end{verbatim}
\end{itemize}} (incorrect)\par
%%%%%%%%%%%%%%%%%%%%%%%%%%%%%%%%%%%%%%%%%%%%%%%%%%%%%%%%%%%%%%%%%%%%%%%%%%%%%%%%
% WeBWorK Online Homework Delivery System
% Copyright � 2000-2007 The WeBWorK Project, http://openwebwork.sf.net/
% $CVSHeader: webwork2/conf/snippets/hardcopyProblemDivider.tex,v 1.3 2004/06/24 21:10:50 dpvc Exp $
% 
% This program is free software; you can redistribute it and/or modify it under
% the terms of either: (a) the GNU General Public License as published by the
% Free Software Foundation; either version 2, or (at your option) any later
% version, or (b) the "Artistic License" which comes with this package.
% 
% This program is distributed in the hope that it will be useful, but WITHOUT
% ANY WARRANTY; without even the implied warranty of MERCHANTABILITY or FITNESS
% FOR A PARTICULAR PURPOSE.  See either the GNU General Public License or the
% Artistic License for more details.
%%%%%%%%%%%%%%%%%%%%%%%%%%%%%%%%%%%%%%%%%%%%%%%%%%%%%%%%%%%%%%%%%%%%%%%%%%%%%%%%

\medskip
\goodbreak
\hrule
\nobreak
\smallskip
%% decoded old answers, saved. (keys = 
{\bf 2. {\footnotesize (0 pts) Library\-/UVA-Stew5e\-/setUVA-Stew5e-C03S08-DerivLogs\-/3-8-18.pg}}\newline Find \(f'(x)\) if
 \[f(x) = \ln { \sqrt { \frac { 8 x + 9 } { 7 x  - 8 } } }.\]
\leavevmode\\\relax \leavevmode\\\relax 
\(f'( x ) =\) \mbox{\parbox[t]{25ex}{\hrulefill}}

\par{\small{\it Answer(s) submitted:}
\vspace{-\parskip}\begin{itemize}
\item\begin{verbatim}\end{verbatim}
\end{itemize}} (incorrect)\par
%%%%%%%%%%%%%%%%%%%%%%%%%%%%%%%%%%%%%%%%%%%%%%%%%%%%%%%%%%%%%%%%%%%%%%%%%%%%%%%%
% WeBWorK Online Homework Delivery System
% Copyright � 2000-2007 The WeBWorK Project, http://openwebwork.sf.net/
% $CVSHeader: webwork2/conf/snippets/hardcopyProblemDivider.tex,v 1.3 2004/06/24 21:10:50 dpvc Exp $
% 
% This program is free software; you can redistribute it and/or modify it under
% the terms of either: (a) the GNU General Public License as published by the
% Free Software Foundation; either version 2, or (at your option) any later
% version, or (b) the "Artistic License" which comes with this package.
% 
% This program is distributed in the hope that it will be useful, but WITHOUT
% ANY WARRANTY; without even the implied warranty of MERCHANTABILITY or FITNESS
% FOR A PARTICULAR PURPOSE.  See either the GNU General Public License or the
% Artistic License for more details.
%%%%%%%%%%%%%%%%%%%%%%%%%%%%%%%%%%%%%%%%%%%%%%%%%%%%%%%%%%%%%%%%%%%%%%%%%%%%%%%%

\medskip
\goodbreak
\hrule
\nobreak
\smallskip
%% decoded old answers, saved. (keys = 
{\bf 3. {\footnotesize (0 pts) Library\-/ASU-topics\-/setChainRuleExpLogFunctions\-/5-3-49.pg}}\newline Suppose that
\[f(x) = \frac{7}{\ln(x^2 + 4)}.\]
Find \(f'(1)\).
\leavevmode\\\relax 
\leavevmode\\\relax 
\(f'(1)\) = \mbox{\parbox[t]{10ex}{\hrulefill}}
\leavevmode\\\relax 
\par{\small{\it Answer(s) submitted:}
\vspace{-\parskip}\begin{itemize}
\item\begin{verbatim}\end{verbatim}
\end{itemize}} (incorrect)\par
%%%%%%%%%%%%%%%%%%%%%%%%%%%%%%%%%%%%%%%%%%%%%%%%%%%%%%%%%%%%%%%%%%%%%%%%%%%%%%%%
% WeBWorK Online Homework Delivery System
% Copyright � 2000-2007 The WeBWorK Project, http://openwebwork.sf.net/
% $CVSHeader: webwork2/conf/snippets/hardcopyProblemDivider.tex,v 1.3 2004/06/24 21:10:50 dpvc Exp $
% 
% This program is free software; you can redistribute it and/or modify it under
% the terms of either: (a) the GNU General Public License as published by the
% Free Software Foundation; either version 2, or (at your option) any later
% version, or (b) the "Artistic License" which comes with this package.
% 
% This program is distributed in the hope that it will be useful, but WITHOUT
% ANY WARRANTY; without even the implied warranty of MERCHANTABILITY or FITNESS
% FOR A PARTICULAR PURPOSE.  See either the GNU General Public License or the
% Artistic License for more details.
%%%%%%%%%%%%%%%%%%%%%%%%%%%%%%%%%%%%%%%%%%%%%%%%%%%%%%%%%%%%%%%%%%%%%%%%%%%%%%%%

\medskip
\goodbreak
\hrule
\nobreak
\smallskip
%% decoded old answers, saved. (keys = 
{\bf 4. {\footnotesize (0 pts) Library\-/Rochester\-/setDerivatives2\_5Implicit\-/S03.07.LogarithmicDifferentiation.PTP01.pg}}\newline 
Suppose 
\[f(x) = \ln\left( \frac{e x^{4}}{(x-5)^{5}} \right).\]
\leavevmode\\\relax 
(a) Find \(f'( x ) =\) \mbox{\parbox[t]{25ex}{\hrulefill}}.  (Hint: Apply the laws of logarithms to \(f(x)\) before taking its derivative.)
\leavevmode\\\relax 
(b) Find \(\frac{d}{dx} \left( e^{f(x)} \right) =\) \mbox{\parbox[t]{25ex}{\hrulefill}}.
\leavevmode\\\relax 
\par{\small{\it Answer(s) submitted:}
\vspace{-\parskip}\begin{itemize}
\item\begin{verbatim}\end{verbatim}
\item\begin{verbatim}\end{verbatim}
\end{itemize}} (incorrect)\par
%%%%%%%%%%%%%%%%%%%%%%%%%%%%%%%%%%%%%%%%%%%%%%%%%%%%%%%%%%%%%%%%%%%%%%%%%%%%%%%%
% WeBWorK Online Homework Delivery System
% Copyright � 2000-2007 The WeBWorK Project, http://openwebwork.sf.net/
% $CVSHeader: webwork2/conf/snippets/hardcopyProblemDivider.tex,v 1.3 2004/06/24 21:10:50 dpvc Exp $
% 
% This program is free software; you can redistribute it and/or modify it under
% the terms of either: (a) the GNU General Public License as published by the
% Free Software Foundation; either version 2, or (at your option) any later
% version, or (b) the "Artistic License" which comes with this package.
% 
% This program is distributed in the hope that it will be useful, but WITHOUT
% ANY WARRANTY; without even the implied warranty of MERCHANTABILITY or FITNESS
% FOR A PARTICULAR PURPOSE.  See either the GNU General Public License or the
% Artistic License for more details.
%%%%%%%%%%%%%%%%%%%%%%%%%%%%%%%%%%%%%%%%%%%%%%%%%%%%%%%%%%%%%%%%%%%%%%%%%%%%%%%%

\medskip
\goodbreak
\hrule
\nobreak
\smallskip
%% decoded old answers, saved. (keys = 
{\bf 5. {\footnotesize (0 pts) Library\-/OSU\-/high\_school\_apcalc\-/dchmwk4\-/prob13.pg}}\newline If \(f(x) = \sin(e^{3 x})\), find \(f'( x )\).
\leavevmode\\\relax  \leavevmode\\\relax  \mbox{\parbox[t]{25ex}{\hrulefill}}
\leavevmode\\\relax 
\par{\small{\it Answer(s) submitted:}
\vspace{-\parskip}\begin{itemize}
\item\begin{verbatim}\end{verbatim}
\end{itemize}} (incorrect)\par
%%%%%%%%%%%%%%%%%%%%%%%%%%%%%%%%%%%%%%%%%%%%%%%%%%%%%%%%%%%%%%%%%%%%%%%%%%%%%%%%
% WeBWorK Online Homework Delivery System
% Copyright � 2000-2007 The WeBWorK Project, http://openwebwork.sf.net/
% $CVSHeader: webwork2/conf/snippets/hardcopyProblemDivider.tex,v 1.3 2004/06/24 21:10:50 dpvc Exp $
% 
% This program is free software; you can redistribute it and/or modify it under
% the terms of either: (a) the GNU General Public License as published by the
% Free Software Foundation; either version 2, or (at your option) any later
% version, or (b) the "Artistic License" which comes with this package.
% 
% This program is distributed in the hope that it will be useful, but WITHOUT
% ANY WARRANTY; without even the implied warranty of MERCHANTABILITY or FITNESS
% FOR A PARTICULAR PURPOSE.  See either the GNU General Public License or the
% Artistic License for more details.
%%%%%%%%%%%%%%%%%%%%%%%%%%%%%%%%%%%%%%%%%%%%%%%%%%%%%%%%%%%%%%%%%%%%%%%%%%%%%%%%

\medskip
\goodbreak
\hrule
\nobreak
\smallskip
%% decoded old answers, saved. (keys = 
{\bf 6. {\footnotesize (0 pts) Library\-/OSU\-/high\_school\_apcalc\-/dchmwk4\-/prob10.pg}}\newline Let \[f(x) = -9 \cos ( \sin ( x ^{8} ))\]
\par 
\(f'( x ) =\) \mbox{\parbox[t]{15ex}{\hrulefill}}
\leavevmode\\\relax 
\par{\small{\it Answer(s) submitted:}
\vspace{-\parskip}\begin{itemize}
\item\begin{verbatim}\end{verbatim}
\end{itemize}} (incorrect)\par
%%%%%%%%%%%%%%%%%%%%%%%%%%%%%%%%%%%%%%%%%%%%%%%%%%%%%%%%%%%%%%%%%%%%%%%%%%%%%%%%
% WeBWorK Online Homework Delivery System
% Copyright � 2000-2007 The WeBWorK Project, http://openwebwork.sf.net/
% $CVSHeader: webwork2/conf/snippets/hardcopyProblemDivider.tex,v 1.3 2004/06/24 21:10:50 dpvc Exp $
% 
% This program is free software; you can redistribute it and/or modify it under
% the terms of either: (a) the GNU General Public License as published by the
% Free Software Foundation; either version 2, or (at your option) any later
% version, or (b) the "Artistic License" which comes with this package.
% 
% This program is distributed in the hope that it will be useful, but WITHOUT
% ANY WARRANTY; without even the implied warranty of MERCHANTABILITY or FITNESS
% FOR A PARTICULAR PURPOSE.  See either the GNU General Public License or the
% Artistic License for more details.
%%%%%%%%%%%%%%%%%%%%%%%%%%%%%%%%%%%%%%%%%%%%%%%%%%%%%%%%%%%%%%%%%%%%%%%%%%%%%%%%

\medskip
\goodbreak
\hrule
\nobreak
\smallskip
%% decoded old answers, saved. (keys = 
{\bf 7. {\footnotesize (0 pts) Library\-/Utah\-/AP\_Calculus\_I\-/set7\_Trigonometric\_Functions\-/1210s5p4.pg}}\newline Let
\[f(x) = \sin\frac{1}{x}.\]
\leavevmode\\\relax 
\(f'(x) =\) \mbox{\parbox[t]{22.5ex}{\hrulefill}}.
\leavevmode\\\relax 
Let
\[g(x) = \frac{1}{\sin x}.\]
\leavevmode\\\relax 
\(g'(x) =\) \mbox{\parbox[t]{22.5ex}{\hrulefill}}.
\par{\small{\it Answer(s) submitted:}
\vspace{-\parskip}\begin{itemize}
\item\begin{verbatim}\end{verbatim}
\item\begin{verbatim}\end{verbatim}
\end{itemize}} (incorrect)\par
%%%%%%%%%%%%%%%%%%%%%%%%%%%%%%%%%%%%%%%%%%%%%%%%%%%%%%%%%%%%%%%%%%%%%%%%%%%%%%%%
% WeBWorK Online Homework Delivery System
% Copyright � 2000-2007 The WeBWorK Project, http://openwebwork.sf.net/
% $CVSHeader: webwork2/conf/snippets/hardcopyProblemDivider.tex,v 1.3 2004/06/24 21:10:50 dpvc Exp $
% 
% This program is free software; you can redistribute it and/or modify it under
% the terms of either: (a) the GNU General Public License as published by the
% Free Software Foundation; either version 2, or (at your option) any later
% version, or (b) the "Artistic License" which comes with this package.
% 
% This program is distributed in the hope that it will be useful, but WITHOUT
% ANY WARRANTY; without even the implied warranty of MERCHANTABILITY or FITNESS
% FOR A PARTICULAR PURPOSE.  See either the GNU General Public License or the
% Artistic License for more details.
%%%%%%%%%%%%%%%%%%%%%%%%%%%%%%%%%%%%%%%%%%%%%%%%%%%%%%%%%%%%%%%%%%%%%%%%%%%%%%%%

\medskip
\goodbreak
\hrule
\nobreak
\smallskip
%% decoded old answers, saved. (keys = 
{\bf 8. {\footnotesize (0 pts) Library\-/ma122DB\-/set6\-/s3\_8\_4.pg}}\newline If \(f(x) = 5 \cos(4\ln(x))\), find \(f'( x )\).
\leavevmode\\\relax  \leavevmode\\\relax  Answer: \mbox{\parbox[t]{25ex}{\hrulefill}}
\leavevmode\\\relax 

\par{\small{\it Answer(s) submitted:}
\vspace{-\parskip}\begin{itemize}
\item\begin{verbatim}\end{verbatim}
\end{itemize}} (incorrect)\par
%% decoded old answers, saved. (keys = 
 \end{multicols}


\noindent {\tiny Generated by \copyright WeBWorK, http://webwork.maa.org, Mathematical Association of America}

 \begin{multicols}{2}
\columnwidth=\linewidth


%%%%%%%%%%%%%%%%%%%%%%%%%%%%%%%%%%%%%%%%%%%%%%%%%%%%%%%%%%%%%%%%%%%%%%%%%%%%%%%%
% WeBWorK Online Homework Delivery System
% Copyright � 2000-2007 The WeBWorK Project, http://openwebwork.sf.net/
% $CVSHeader: webwork2/conf/snippets/hardcopySetDivider.tex,v 1.4 2004/07/07 11:35:34 gage Exp $
% 
% This program is free software; you can redistribute it and/or modify it under
% the terms of either: (a) the GNU General Public License as published by the
% Free Software Foundation; either version 2, or (at your option) any later
% version, or (b) the "Artistic License" which comes with this package.
% 
% This program is distributed in the hope that it will be useful, but WITHOUT
% ANY WARRANTY; without even the implied warranty of MERCHANTABILITY or FITNESS
% FOR A PARTICULAR PURPOSE.  See either the GNU General Public License or the
% Artistic License for more details.
%%%%%%%%%%%%%%%%%%%%%%%%%%%%%%%%%%%%%%%%%%%%%%%%%%%%%%%%%%%%%%%%%%%%%%%%%%%%%%%%

\end{multicols}   % close off the columns from the set above

\newpage%
\setcounter{page}{1}% 
\begin{multicols}{2}
\columnwidth=\linewidth % reopen the columns for the following set

%% decoded old answers, saved. (keys = 
 \end{multicols}

\noindent {\large \bf Trevor Klar}
\hfill
{\large \bf {MATH3B-02-F19-Long}}
% Uncomment the line below if this course has sections. Note that this is a comment in TeX mode since this is only processed by LaTeX
%   {\large \bf { Section:  } }
\par
\noindent{\large \bf {Assignment Revision\_Function\_derivatives  due 10/09/2019 at 10:00pm PDT}}
\par\noindent \bigskip
% Uncomment and edit the line below if this course has a web page. Note that this is a comment in TeX mode.
%See the course web page for information http://yoururl/yourcourse



 \begin{multicols}{2}
\columnwidth=\linewidth
%%%%%%%%%%%%%%%%%%%%%%%%%%%%%%%%%%%%%%%%%%%%%%%%%%%%%%%%%%%%%%%%%%%%%%%%%%%%%%%%
% WeBWorK Online Homework Delivery System
% Copyright � 2000-2007 The WeBWorK Project, http://openwebwork.sf.net/
% $CVSHeader: webwork2/conf/snippets/hardcopyProblemDivider.tex,v 1.3 2004/06/24 21:10:50 dpvc Exp $
% 
% This program is free software; you can redistribute it and/or modify it under
% the terms of either: (a) the GNU General Public License as published by the
% Free Software Foundation; either version 2, or (at your option) any later
% version, or (b) the "Artistic License" which comes with this package.
% 
% This program is distributed in the hope that it will be useful, but WITHOUT
% ANY WARRANTY; without even the implied warranty of MERCHANTABILITY or FITNESS
% FOR A PARTICULAR PURPOSE.  See either the GNU General Public License or the
% Artistic License for more details.
%%%%%%%%%%%%%%%%%%%%%%%%%%%%%%%%%%%%%%%%%%%%%%%%%%%%%%%%%%%%%%%%%%%%%%%%%%%%%%%%

\medskip
\goodbreak
\hrule
\nobreak
\smallskip
%% decoded old answers, saved. (keys = 
{\bf 1. {\footnotesize (0 pts) Library\-/Utah\-/Calculus\_I\-/set4\_The\_Derivative\-/1210s4p3.pg}}\newline If \(f(x) = 7 \sin x + 4 \cos x\), then 
\leavevmode\\\relax 
\(f'(x) =\)
\mbox{\parbox[t]{17.5ex}{\hrulefill}}
\par{\small{\it Answer(s) submitted:}
\vspace{-\parskip}\begin{itemize}
\item\begin{verbatim}\end{verbatim}
\end{itemize}} (incorrect)\par
%%%%%%%%%%%%%%%%%%%%%%%%%%%%%%%%%%%%%%%%%%%%%%%%%%%%%%%%%%%%%%%%%%%%%%%%%%%%%%%%
% WeBWorK Online Homework Delivery System
% Copyright � 2000-2007 The WeBWorK Project, http://openwebwork.sf.net/
% $CVSHeader: webwork2/conf/snippets/hardcopyProblemDivider.tex,v 1.3 2004/06/24 21:10:50 dpvc Exp $
% 
% This program is free software; you can redistribute it and/or modify it under
% the terms of either: (a) the GNU General Public License as published by the
% Free Software Foundation; either version 2, or (at your option) any later
% version, or (b) the "Artistic License" which comes with this package.
% 
% This program is distributed in the hope that it will be useful, but WITHOUT
% ANY WARRANTY; without even the implied warranty of MERCHANTABILITY or FITNESS
% FOR A PARTICULAR PURPOSE.  See either the GNU General Public License or the
% Artistic License for more details.
%%%%%%%%%%%%%%%%%%%%%%%%%%%%%%%%%%%%%%%%%%%%%%%%%%%%%%%%%%%%%%%%%%%%%%%%%%%%%%%%

\medskip
\goodbreak
\hrule
\nobreak
\smallskip
%% decoded old answers, saved. (keys = 
{\bf 2. {\footnotesize (0 pts) Library\-/Rochester\-/setDerivatives4Trig\-/s2\_4\_21a.pg}}\newline Let \[f(x) = 12 \sin x + 12 \cos x\]
\par 
\(f'(x) =\)  \mbox{\parbox[t]{17.5ex}{\hrulefill}}
\par 
\(f'( - \frac {  \pi } {6} ) =\) \mbox{\parbox[t]{17.5ex}{\hrulefill}}
\par 
[Note:  When entering trigonometric functions into Webwork, you must include parentheses around the arguement.  I.e.  "sinx" would not be accepted but "sin(x)" would.]
\par{\small{\it Answer(s) submitted:}
\vspace{-\parskip}\begin{itemize}
\item\begin{verbatim}\end{verbatim}
\item\begin{verbatim}\end{verbatim}
\end{itemize}} (incorrect)\par
%%%%%%%%%%%%%%%%%%%%%%%%%%%%%%%%%%%%%%%%%%%%%%%%%%%%%%%%%%%%%%%%%%%%%%%%%%%%%%%%
% WeBWorK Online Homework Delivery System
% Copyright � 2000-2007 The WeBWorK Project, http://openwebwork.sf.net/
% $CVSHeader: webwork2/conf/snippets/hardcopyProblemDivider.tex,v 1.3 2004/06/24 21:10:50 dpvc Exp $
% 
% This program is free software; you can redistribute it and/or modify it under
% the terms of either: (a) the GNU General Public License as published by the
% Free Software Foundation; either version 2, or (at your option) any later
% version, or (b) the "Artistic License" which comes with this package.
% 
% This program is distributed in the hope that it will be useful, but WITHOUT
% ANY WARRANTY; without even the implied warranty of MERCHANTABILITY or FITNESS
% FOR A PARTICULAR PURPOSE.  See either the GNU General Public License or the
% Artistic License for more details.
%%%%%%%%%%%%%%%%%%%%%%%%%%%%%%%%%%%%%%%%%%%%%%%%%%%%%%%%%%%%%%%%%%%%%%%%%%%%%%%%

\medskip
\goodbreak
\hrule
\nobreak
\smallskip
%% decoded old answers, saved. (keys = 
{\bf 3. {\footnotesize (0 pts) Library\-/Union\-/setDervTrigonometric\-/s2\_4\_35.pg}}\newline Find the equation of the tangent line to the curve
 \(y =   6 x \cos x\)
at the point \(( \pi , -6 \pi)\).
\par 
The equation of this tangent line can be written in the form \(y = mx+b\)
where
\par 

\par\begin{tabular}{rcl}
\(m\)&=&\ \mbox{\parbox[t]{25ex}{\hrulefill}}\\ \noalign{\kern 3pt}
\(b\)&=&\ \mbox{\parbox[t]{25ex}{\hrulefill}}\\ \noalign{\kern 3pt}
\end{tabular}


\par{\small{\it Answer(s) submitted:}
\vspace{-\parskip}\begin{itemize}
\item\begin{verbatim}\end{verbatim}
\item\begin{verbatim}\end{verbatim}
\end{itemize}} (incorrect)\par
%%%%%%%%%%%%%%%%%%%%%%%%%%%%%%%%%%%%%%%%%%%%%%%%%%%%%%%%%%%%%%%%%%%%%%%%%%%%%%%%
% WeBWorK Online Homework Delivery System
% Copyright � 2000-2007 The WeBWorK Project, http://openwebwork.sf.net/
% $CVSHeader: webwork2/conf/snippets/hardcopyProblemDivider.tex,v 1.3 2004/06/24 21:10:50 dpvc Exp $
% 
% This program is free software; you can redistribute it and/or modify it under
% the terms of either: (a) the GNU General Public License as published by the
% Free Software Foundation; either version 2, or (at your option) any later
% version, or (b) the "Artistic License" which comes with this package.
% 
% This program is distributed in the hope that it will be useful, but WITHOUT
% ANY WARRANTY; without even the implied warranty of MERCHANTABILITY or FITNESS
% FOR A PARTICULAR PURPOSE.  See either the GNU General Public License or the
% Artistic License for more details.
%%%%%%%%%%%%%%%%%%%%%%%%%%%%%%%%%%%%%%%%%%%%%%%%%%%%%%%%%%%%%%%%%%%%%%%%%%%%%%%%

\medskip
\goodbreak
\hrule
\nobreak
\smallskip
%% decoded old answers, saved. (keys = 
{\bf 4. {\footnotesize (0 pts) Library\-/UVA-Stew5e\-/setUVA-Stew5e-C03S02-ProdQuotRules\-/3-2-13a.pg}}\newline 
Suppose that \(f(x) = 13 e^{x} - e x^{e}\).  Find \(f'(3)\).

\leavevmode\\\relax  \leavevmode\\\relax 

\(f'(3)\) = \mbox{\parbox[t]{15ex}{\hrulefill}}
\leavevmode\\\relax 

\par{\small{\it Answer(s) submitted:}
\vspace{-\parskip}\begin{itemize}
\item\begin{verbatim}\end{verbatim}
\end{itemize}} (incorrect)\par
%%%%%%%%%%%%%%%%%%%%%%%%%%%%%%%%%%%%%%%%%%%%%%%%%%%%%%%%%%%%%%%%%%%%%%%%%%%%%%%%
% WeBWorK Online Homework Delivery System
% Copyright � 2000-2007 The WeBWorK Project, http://openwebwork.sf.net/
% $CVSHeader: webwork2/conf/snippets/hardcopyProblemDivider.tex,v 1.3 2004/06/24 21:10:50 dpvc Exp $
% 
% This program is free software; you can redistribute it and/or modify it under
% the terms of either: (a) the GNU General Public License as published by the
% Free Software Foundation; either version 2, or (at your option) any later
% version, or (b) the "Artistic License" which comes with this package.
% 
% This program is distributed in the hope that it will be useful, but WITHOUT
% ANY WARRANTY; without even the implied warranty of MERCHANTABILITY or FITNESS
% FOR A PARTICULAR PURPOSE.  See either the GNU General Public License or the
% Artistic License for more details.
%%%%%%%%%%%%%%%%%%%%%%%%%%%%%%%%%%%%%%%%%%%%%%%%%%%%%%%%%%%%%%%%%%%%%%%%%%%%%%%%

\medskip
\goodbreak
\hrule
\nobreak
\smallskip
%% decoded old answers, saved. (keys = 
{\bf 5. {\footnotesize (0 pts) Library\-/ASU-topics\-/setProductQuotientRule\-/5-2-33.pg}}\newline 
Find the equation of the line tangent to the graph of \(y=6 e^x\)
at \(x=2\).

\leavevmode\\\relax  \leavevmode\\\relax 

Tangent Line: \(y =\) \mbox{\parbox[t]{15ex}{\hrulefill}}
\leavevmode\\\relax 

\par{\small{\it Answer(s) submitted:}
\vspace{-\parskip}\begin{itemize}
\item\begin{verbatim}\end{verbatim}
\end{itemize}} (incorrect)\par
%%%%%%%%%%%%%%%%%%%%%%%%%%%%%%%%%%%%%%%%%%%%%%%%%%%%%%%%%%%%%%%%%%%%%%%%%%%%%%%%
% WeBWorK Online Homework Delivery System
% Copyright � 2000-2007 The WeBWorK Project, http://openwebwork.sf.net/
% $CVSHeader: webwork2/conf/snippets/hardcopyProblemDivider.tex,v 1.3 2004/06/24 21:10:50 dpvc Exp $
% 
% This program is free software; you can redistribute it and/or modify it under
% the terms of either: (a) the GNU General Public License as published by the
% Free Software Foundation; either version 2, or (at your option) any later
% version, or (b) the "Artistic License" which comes with this package.
% 
% This program is distributed in the hope that it will be useful, but WITHOUT
% ANY WARRANTY; without even the implied warranty of MERCHANTABILITY or FITNESS
% FOR A PARTICULAR PURPOSE.  See either the GNU General Public License or the
% Artistic License for more details.
%%%%%%%%%%%%%%%%%%%%%%%%%%%%%%%%%%%%%%%%%%%%%%%%%%%%%%%%%%%%%%%%%%%%%%%%%%%%%%%%

\medskip
\goodbreak
\hrule
\nobreak
\smallskip
%% decoded old answers, saved. (keys = 
{\bf 6. {\footnotesize (0 pts) Library\-/Union\-/setDervLogs\-/an4\_3\_40.pg}}\newline Let \(f(x) =\displaystyle 8^{-x}\).  Find \(f'(x)\).
\par 
\(f'(x) =\) \mbox{\parbox[t]{25ex}{\hrulefill}}

\par{\small{\it Answer(s) submitted:}
\vspace{-\parskip}\begin{itemize}
\item\begin{verbatim}\end{verbatim}
\end{itemize}} (incorrect)\par
%%%%%%%%%%%%%%%%%%%%%%%%%%%%%%%%%%%%%%%%%%%%%%%%%%%%%%%%%%%%%%%%%%%%%%%%%%%%%%%%
% WeBWorK Online Homework Delivery System
% Copyright � 2000-2007 The WeBWorK Project, http://openwebwork.sf.net/
% $CVSHeader: webwork2/conf/snippets/hardcopyProblemDivider.tex,v 1.3 2004/06/24 21:10:50 dpvc Exp $
% 
% This program is free software; you can redistribute it and/or modify it under
% the terms of either: (a) the GNU General Public License as published by the
% Free Software Foundation; either version 2, or (at your option) any later
% version, or (b) the "Artistic License" which comes with this package.
% 
% This program is distributed in the hope that it will be useful, but WITHOUT
% ANY WARRANTY; without even the implied warranty of MERCHANTABILITY or FITNESS
% FOR A PARTICULAR PURPOSE.  See either the GNU General Public License or the
% Artistic License for more details.
%%%%%%%%%%%%%%%%%%%%%%%%%%%%%%%%%%%%%%%%%%%%%%%%%%%%%%%%%%%%%%%%%%%%%%%%%%%%%%%%

\medskip
\goodbreak
\hrule
\nobreak
\smallskip
%% decoded old answers, saved. (keys = 
{\bf 7. {\footnotesize (0 pts) Library\-/ASU-topics\-/setDerivativeBasicFunctions\-/5-2-02.pg}}\newline 
Suppose that \(f(x) = 10 e^{x} + 2 \ln(x)\).  Find \(f'(3)\).

\leavevmode\\\relax  \leavevmode\\\relax 

\(f'(3)\) = \mbox{\parbox[t]{15ex}{\hrulefill}}
\leavevmode\\\relax 

\par{\small{\it Answer(s) submitted:}
\vspace{-\parskip}\begin{itemize}
\item\begin{verbatim}\end{verbatim}
\end{itemize}} (incorrect)\par
%%%%%%%%%%%%%%%%%%%%%%%%%%%%%%%%%%%%%%%%%%%%%%%%%%%%%%%%%%%%%%%%%%%%%%%%%%%%%%%%
% WeBWorK Online Homework Delivery System
% Copyright � 2000-2007 The WeBWorK Project, http://openwebwork.sf.net/
% $CVSHeader: webwork2/conf/snippets/hardcopyProblemDivider.tex,v 1.3 2004/06/24 21:10:50 dpvc Exp $
% 
% This program is free software; you can redistribute it and/or modify it under
% the terms of either: (a) the GNU General Public License as published by the
% Free Software Foundation; either version 2, or (at your option) any later
% version, or (b) the "Artistic License" which comes with this package.
% 
% This program is distributed in the hope that it will be useful, but WITHOUT
% ANY WARRANTY; without even the implied warranty of MERCHANTABILITY or FITNESS
% FOR A PARTICULAR PURPOSE.  See either the GNU General Public License or the
% Artistic License for more details.
%%%%%%%%%%%%%%%%%%%%%%%%%%%%%%%%%%%%%%%%%%%%%%%%%%%%%%%%%%%%%%%%%%%%%%%%%%%%%%%%

\medskip
\goodbreak
\hrule
\nobreak
\smallskip
%% decoded old answers, saved. (keys = 
{\bf 8. {\footnotesize (0 pts) Library\-/ASU-topics\-/setProductQuotientRule\-/5-2-35.pg}}\newline 
Find the equation of the line tangent to the graph of \(y=3 \ln(x)\)
at \(x=4\).

\leavevmode\\\relax  \leavevmode\\\relax 

Tangent Line: \(y =\) \mbox{\parbox[t]{15ex}{\hrulefill}}
\leavevmode\\\relax 

\par{\small{\it Answer(s) submitted:}
\vspace{-\parskip}\begin{itemize}
\item\begin{verbatim}\end{verbatim}
\end{itemize}} (incorrect)\par
%%%%%%%%%%%%%%%%%%%%%%%%%%%%%%%%%%%%%%%%%%%%%%%%%%%%%%%%%%%%%%%%%%%%%%%%%%%%%%%%
% WeBWorK Online Homework Delivery System
% Copyright � 2000-2007 The WeBWorK Project, http://openwebwork.sf.net/
% $CVSHeader: webwork2/conf/snippets/hardcopyProblemDivider.tex,v 1.3 2004/06/24 21:10:50 dpvc Exp $
% 
% This program is free software; you can redistribute it and/or modify it under
% the terms of either: (a) the GNU General Public License as published by the
% Free Software Foundation; either version 2, or (at your option) any later
% version, or (b) the "Artistic License" which comes with this package.
% 
% This program is distributed in the hope that it will be useful, but WITHOUT
% ANY WARRANTY; without even the implied warranty of MERCHANTABILITY or FITNESS
% FOR A PARTICULAR PURPOSE.  See either the GNU General Public License or the
% Artistic License for more details.
%%%%%%%%%%%%%%%%%%%%%%%%%%%%%%%%%%%%%%%%%%%%%%%%%%%%%%%%%%%%%%%%%%%%%%%%%%%%%%%%

\medskip
\goodbreak
\hrule
\nobreak
\smallskip
%% decoded old answers, saved. (keys = 
{\bf 9. {\footnotesize (0 pts) Library\-/Utah\-/Calculus\_II\-/set2\_Transcendental\_Functions\-/set2\_pr8.pg}}\newline Let \par 
\(f(x) = \frac{x-1}{x+1}\).
\par 
(a) Evaluate \(f^{-1}(3) =\) \leavevmode\\\relax  \mbox{\parbox[t]{25ex}{\hrulefill}}

\leavevmode\\\relax  
(b) Evaluate \((f^{-1})'(3) =\)\leavevmode\\\relax  \mbox{\parbox[t]{25ex}{\hrulefill}}
\leavevmode\\\relax 
\par{\small{\it Answer(s) submitted:}
\vspace{-\parskip}\begin{itemize}
\item\begin{verbatim}\end{verbatim}
\item\begin{verbatim}\end{verbatim}
\end{itemize}} (incorrect)\par
%%%%%%%%%%%%%%%%%%%%%%%%%%%%%%%%%%%%%%%%%%%%%%%%%%%%%%%%%%%%%%%%%%%%%%%%%%%%%%%%
% WeBWorK Online Homework Delivery System
% Copyright � 2000-2007 The WeBWorK Project, http://openwebwork.sf.net/
% $CVSHeader: webwork2/conf/snippets/hardcopyProblemDivider.tex,v 1.3 2004/06/24 21:10:50 dpvc Exp $
% 
% This program is free software; you can redistribute it and/or modify it under
% the terms of either: (a) the GNU General Public License as published by the
% Free Software Foundation; either version 2, or (at your option) any later
% version, or (b) the "Artistic License" which comes with this package.
% 
% This program is distributed in the hope that it will be useful, but WITHOUT
% ANY WARRANTY; without even the implied warranty of MERCHANTABILITY or FITNESS
% FOR A PARTICULAR PURPOSE.  See either the GNU General Public License or the
% Artistic License for more details.
%%%%%%%%%%%%%%%%%%%%%%%%%%%%%%%%%%%%%%%%%%%%%%%%%%%%%%%%%%%%%%%%%%%%%%%%%%%%%%%%

\medskip
\goodbreak
\hrule
\nobreak
\smallskip
%% decoded old answers, saved. (keys = 
{\bf 10. {\footnotesize (0 pts) Library\-/Utah\-/AP\_Calculus\_I\-/set7\_Trigonometric\_Functions\-/1220s3p4.pg}}\newline 
Find a formula for \(f^{-1}(x)\)  if : 
\leavevmode\\\relax 
\(f(x) = ({\frac{x - 1}{x + 1}})^3\) \leavevmode\\\relax 

\(f^{-1}(x)\) =  \mbox{\parbox[t]{20ex}{\hrulefill}}.
\leavevmode\\\relax 

\leavevmode\\\relax 
\(\left(f^{-1}\right)'(x) =\) \mbox{\parbox[t]{20ex}{\hrulefill}}.



\leavevmode\\\relax  {\bf  Hint:} Undo the operations of \(f(x)\) from the
outside in.  Undo the cubed operation first.

\par{\small{\it Answer(s) submitted:}
\vspace{-\parskip}\begin{itemize}
\item\begin{verbatim}\end{verbatim}
\item\begin{verbatim}\end{verbatim}
\end{itemize}} (incorrect)\par
%%%%%%%%%%%%%%%%%%%%%%%%%%%%%%%%%%%%%%%%%%%%%%%%%%%%%%%%%%%%%%%%%%%%%%%%%%%%%%%%
% WeBWorK Online Homework Delivery System
% Copyright � 2000-2007 The WeBWorK Project, http://openwebwork.sf.net/
% $CVSHeader: webwork2/conf/snippets/hardcopyProblemDivider.tex,v 1.3 2004/06/24 21:10:50 dpvc Exp $
% 
% This program is free software; you can redistribute it and/or modify it under
% the terms of either: (a) the GNU General Public License as published by the
% Free Software Foundation; either version 2, or (at your option) any later
% version, or (b) the "Artistic License" which comes with this package.
% 
% This program is distributed in the hope that it will be useful, but WITHOUT
% ANY WARRANTY; without even the implied warranty of MERCHANTABILITY or FITNESS
% FOR A PARTICULAR PURPOSE.  See either the GNU General Public License or the
% Artistic License for more details.
%%%%%%%%%%%%%%%%%%%%%%%%%%%%%%%%%%%%%%%%%%%%%%%%%%%%%%%%%%%%%%%%%%%%%%%%%%%%%%%%

\medskip
\goodbreak
\hrule
\nobreak
\smallskip
%% decoded old answers, saved. (keys = 
{\bf 11. {\footnotesize (0 pts) Library\-/Utah\-/AP\_Calculus\_I\-/set7\_Trigonometric\_Functions\-/1220s3p3.pg}}\newline Find a formula for \(f^{-1}(x)\)  if : 
\leavevmode\\\relax 
\(f(x) = \sqrt{\frac{1}{x - 2}}\) \leavevmode\\\relax 

\(f^{-1}(x)\) =  \mbox{\parbox[t]{20ex}{\hrulefill}}.
\leavevmode\\\relax 

\leavevmode\\\relax 
\(\left(f^{-1}\right)'(x) =\) \mbox{\parbox[t]{20ex}{\hrulefill}}.



\par{\small{\it Answer(s) submitted:}
\vspace{-\parskip}\begin{itemize}
\item\begin{verbatim}\end{verbatim}
\item\begin{verbatim}\end{verbatim}
\end{itemize}} (incorrect)\par
%% decoded old answers, saved. (keys = 
 \end{multicols}


\noindent {\tiny Generated by \copyright WeBWorK, http://webwork.maa.org, Mathematical Association of America}

 \begin{multicols}{2}
\columnwidth=\linewidth


%%%%%%%%%%%%%%%%%%%%%%%%%%%%%%%%%%%%%%%%%%%%%%%%%%%%%%%%%%%%%%%%%%%%%%%%%%%%%%%%
% WeBWorK Online Homework Delivery System
% Copyright � 2000-2007 The WeBWorK Project, http://openwebwork.sf.net/
% $CVSHeader: webwork2/conf/snippets/hardcopySetDivider.tex,v 1.4 2004/07/07 11:35:34 gage Exp $
% 
% This program is free software; you can redistribute it and/or modify it under
% the terms of either: (a) the GNU General Public License as published by the
% Free Software Foundation; either version 2, or (at your option) any later
% version, or (b) the "Artistic License" which comes with this package.
% 
% This program is distributed in the hope that it will be useful, but WITHOUT
% ANY WARRANTY; without even the implied warranty of MERCHANTABILITY or FITNESS
% FOR A PARTICULAR PURPOSE.  See either the GNU General Public License or the
% Artistic License for more details.
%%%%%%%%%%%%%%%%%%%%%%%%%%%%%%%%%%%%%%%%%%%%%%%%%%%%%%%%%%%%%%%%%%%%%%%%%%%%%%%%

\end{multicols}   % close off the columns from the set above

\newpage%
\setcounter{page}{1}% 
\begin{multicols}{2}
\columnwidth=\linewidth % reopen the columns for the following set

%% decoded old answers, saved. (keys = 
 \end{multicols}

\noindent {\large \bf Trevor Klar}
\hfill
{\large \bf {MATH3B-02-F19-Long}}
% Uncomment the line below if this course has sections. Note that this is a comment in TeX mode since this is only processed by LaTeX
%   {\large \bf { Section:  } }
\par
\noindent{\large \bf {Assignment Revision\_Product\_Rule  due 10/10/2019 at 10:00pm PDT}}
\par\noindent \bigskip
% Uncomment and edit the line below if this course has a web page. Note that this is a comment in TeX mode.
%See the course web page for information http://yoururl/yourcourse



 \begin{multicols}{2}
\columnwidth=\linewidth
%%%%%%%%%%%%%%%%%%%%%%%%%%%%%%%%%%%%%%%%%%%%%%%%%%%%%%%%%%%%%%%%%%%%%%%%%%%%%%%%
% WeBWorK Online Homework Delivery System
% Copyright � 2000-2007 The WeBWorK Project, http://openwebwork.sf.net/
% $CVSHeader: webwork2/conf/snippets/hardcopyProblemDivider.tex,v 1.3 2004/06/24 21:10:50 dpvc Exp $
% 
% This program is free software; you can redistribute it and/or modify it under
% the terms of either: (a) the GNU General Public License as published by the
% Free Software Foundation; either version 2, or (at your option) any later
% version, or (b) the "Artistic License" which comes with this package.
% 
% This program is distributed in the hope that it will be useful, but WITHOUT
% ANY WARRANTY; without even the implied warranty of MERCHANTABILITY or FITNESS
% FOR A PARTICULAR PURPOSE.  See either the GNU General Public License or the
% Artistic License for more details.
%%%%%%%%%%%%%%%%%%%%%%%%%%%%%%%%%%%%%%%%%%%%%%%%%%%%%%%%%%%%%%%%%%%%%%%%%%%%%%%%

\medskip
\goodbreak
\hrule
\nobreak
\smallskip
%% decoded old answers, saved. (keys = 
{\bf 1. {\footnotesize (0 pts) Library\-/UVA-Stew5e\-/setUVA-Stew5e-C03S02-ProdQuotRules\-/3-2-09.pg}}\newline Let \(f(x) = 12 x^3 (x^2 - 3)\).
Evaluate \(f'(x)\) at the following points:
\leavevmode\\\relax 
(A) \(f'(1)\) = \mbox{\parbox[t]{20ex}{\hrulefill}}
\leavevmode\\\relax 
\leavevmode\\\relax 
(B) \(f'(-6)\) = \mbox{\parbox[t]{20ex}{\hrulefill}}
\leavevmode\\\relax 
\par{\small{\it Answer(s) submitted:}
\vspace{-\parskip}\begin{itemize}
\item\begin{verbatim}\end{verbatim}
\item\begin{verbatim}\end{verbatim}
\end{itemize}} (incorrect)\par
%%%%%%%%%%%%%%%%%%%%%%%%%%%%%%%%%%%%%%%%%%%%%%%%%%%%%%%%%%%%%%%%%%%%%%%%%%%%%%%%
% WeBWorK Online Homework Delivery System
% Copyright � 2000-2007 The WeBWorK Project, http://openwebwork.sf.net/
% $CVSHeader: webwork2/conf/snippets/hardcopyProblemDivider.tex,v 1.3 2004/06/24 21:10:50 dpvc Exp $
% 
% This program is free software; you can redistribute it and/or modify it under
% the terms of either: (a) the GNU General Public License as published by the
% Free Software Foundation; either version 2, or (at your option) any later
% version, or (b) the "Artistic License" which comes with this package.
% 
% This program is distributed in the hope that it will be useful, but WITHOUT
% ANY WARRANTY; without even the implied warranty of MERCHANTABILITY or FITNESS
% FOR A PARTICULAR PURPOSE.  See either the GNU General Public License or the
% Artistic License for more details.
%%%%%%%%%%%%%%%%%%%%%%%%%%%%%%%%%%%%%%%%%%%%%%%%%%%%%%%%%%%%%%%%%%%%%%%%%%%%%%%%

\medskip
\goodbreak
\hrule
\nobreak
\smallskip
%% decoded old answers, saved. (keys = 
{\bf 2. {\footnotesize (0 pts) Library\-/UVA-Stew5e\-/setUVA-Stew5e-C03S02-ProdQuotRules\-/3-2-25.pg}}\newline Find an equation for the line tangent to the graph of
\[f(x) = 8 x e^x\]
at the point \((a,f(a))\) for \(a=2\).
\par 
\par 
\(y\) = \mbox{\parbox[t]{15ex}{\hrulefill}}

\par{\small{\it Answer(s) submitted:}
\vspace{-\parskip}\begin{itemize}
\item\begin{verbatim}\end{verbatim}
\end{itemize}} (incorrect)\par
%%%%%%%%%%%%%%%%%%%%%%%%%%%%%%%%%%%%%%%%%%%%%%%%%%%%%%%%%%%%%%%%%%%%%%%%%%%%%%%%
% WeBWorK Online Homework Delivery System
% Copyright � 2000-2007 The WeBWorK Project, http://openwebwork.sf.net/
% $CVSHeader: webwork2/conf/snippets/hardcopyProblemDivider.tex,v 1.3 2004/06/24 21:10:50 dpvc Exp $
% 
% This program is free software; you can redistribute it and/or modify it under
% the terms of either: (a) the GNU General Public License as published by the
% Free Software Foundation; either version 2, or (at your option) any later
% version, or (b) the "Artistic License" which comes with this package.
% 
% This program is distributed in the hope that it will be useful, but WITHOUT
% ANY WARRANTY; without even the implied warranty of MERCHANTABILITY or FITNESS
% FOR A PARTICULAR PURPOSE.  See either the GNU General Public License or the
% Artistic License for more details.
%%%%%%%%%%%%%%%%%%%%%%%%%%%%%%%%%%%%%%%%%%%%%%%%%%%%%%%%%%%%%%%%%%%%%%%%%%%%%%%%

\medskip
\goodbreak
\hrule
\nobreak
\smallskip
%% decoded old answers, saved. (keys = 
{\bf 3. {\footnotesize (0 pts) Library\-/Utah\-/Quantitative\_Analysis\-/set5\_Derivatives\-/pr\_2.pg}}\newline Let \[f(x) = 5 x^ {5} \ln x\]
\par 
\(f'( x ) =\) \mbox{\parbox[t]{20ex}{\hrulefill}}
\par 
\(f'( e^ {3} ) =\) \mbox{\parbox[t]{20ex}{\hrulefill}}
\par{\small{\it Answer(s) submitted:}
\vspace{-\parskip}\begin{itemize}
\item\begin{verbatim}\end{verbatim}
\item\begin{verbatim}\end{verbatim}
\end{itemize}} (incorrect)\par
%%%%%%%%%%%%%%%%%%%%%%%%%%%%%%%%%%%%%%%%%%%%%%%%%%%%%%%%%%%%%%%%%%%%%%%%%%%%%%%%
% WeBWorK Online Homework Delivery System
% Copyright � 2000-2007 The WeBWorK Project, http://openwebwork.sf.net/
% $CVSHeader: webwork2/conf/snippets/hardcopyProblemDivider.tex,v 1.3 2004/06/24 21:10:50 dpvc Exp $
% 
% This program is free software; you can redistribute it and/or modify it under
% the terms of either: (a) the GNU General Public License as published by the
% Free Software Foundation; either version 2, or (at your option) any later
% version, or (b) the "Artistic License" which comes with this package.
% 
% This program is distributed in the hope that it will be useful, but WITHOUT
% ANY WARRANTY; without even the implied warranty of MERCHANTABILITY or FITNESS
% FOR A PARTICULAR PURPOSE.  See either the GNU General Public License or the
% Artistic License for more details.
%%%%%%%%%%%%%%%%%%%%%%%%%%%%%%%%%%%%%%%%%%%%%%%%%%%%%%%%%%%%%%%%%%%%%%%%%%%%%%%%

\medskip
\goodbreak
\hrule
\nobreak
\smallskip
%% decoded old answers, saved. (keys = 
{\bf 4. {\footnotesize (0 pts) Library\-/Rochester\-/setDerivatives7Log\-/mec12.pg}}\newline Let \[f(x) = 6^x \log_{4} (x)\]
\par 
\(f'( x ) =\) \mbox{\parbox[t]{20ex}{\hrulefill}}
\par{\small{\it Answer(s) submitted:}
\vspace{-\parskip}\begin{itemize}
\item\begin{verbatim}\end{verbatim}
\end{itemize}} (incorrect)\par
%%%%%%%%%%%%%%%%%%%%%%%%%%%%%%%%%%%%%%%%%%%%%%%%%%%%%%%%%%%%%%%%%%%%%%%%%%%%%%%%
% WeBWorK Online Homework Delivery System
% Copyright � 2000-2007 The WeBWorK Project, http://openwebwork.sf.net/
% $CVSHeader: webwork2/conf/snippets/hardcopyProblemDivider.tex,v 1.3 2004/06/24 21:10:50 dpvc Exp $
% 
% This program is free software; you can redistribute it and/or modify it under
% the terms of either: (a) the GNU General Public License as published by the
% Free Software Foundation; either version 2, or (at your option) any later
% version, or (b) the "Artistic License" which comes with this package.
% 
% This program is distributed in the hope that it will be useful, but WITHOUT
% ANY WARRANTY; without even the implied warranty of MERCHANTABILITY or FITNESS
% FOR A PARTICULAR PURPOSE.  See either the GNU General Public License or the
% Artistic License for more details.
%%%%%%%%%%%%%%%%%%%%%%%%%%%%%%%%%%%%%%%%%%%%%%%%%%%%%%%%%%%%%%%%%%%%%%%%%%%%%%%%

\medskip
\goodbreak
\hrule
\nobreak
\smallskip
%% decoded old answers, saved. (keys = 
{\bf 5. {\footnotesize (0 pts) Library\-/UVA-Stew5e\-/setUVA-Stew5e-C03S04-DerivsTrig\-/3-4-25.pg}}\newline \par 
Find the equation of the tangent line to the curve
 \[y =   3 x \cos x\]
at the point \(( \pi , -3 \pi)\).
\par 
The equation of this tangent line can be written in the form \(y = mx+b\) where
\par 
\(m =\) \mbox{\parbox[t]{10ex}{\hrulefill}}
\par 
and \(b =\) \mbox{\parbox[t]{10ex}{\hrulefill}}
\par{\small{\it Answer(s) submitted:}
\vspace{-\parskip}\begin{itemize}
\item\begin{verbatim}\end{verbatim}
\item\begin{verbatim}\end{verbatim}
\end{itemize}} (incorrect)\par
%%%%%%%%%%%%%%%%%%%%%%%%%%%%%%%%%%%%%%%%%%%%%%%%%%%%%%%%%%%%%%%%%%%%%%%%%%%%%%%%
% WeBWorK Online Homework Delivery System
% Copyright � 2000-2007 The WeBWorK Project, http://openwebwork.sf.net/
% $CVSHeader: webwork2/conf/snippets/hardcopyProblemDivider.tex,v 1.3 2004/06/24 21:10:50 dpvc Exp $
% 
% This program is free software; you can redistribute it and/or modify it under
% the terms of either: (a) the GNU General Public License as published by the
% Free Software Foundation; either version 2, or (at your option) any later
% version, or (b) the "Artistic License" which comes with this package.
% 
% This program is distributed in the hope that it will be useful, but WITHOUT
% ANY WARRANTY; without even the implied warranty of MERCHANTABILITY or FITNESS
% FOR A PARTICULAR PURPOSE.  See either the GNU General Public License or the
% Artistic License for more details.
%%%%%%%%%%%%%%%%%%%%%%%%%%%%%%%%%%%%%%%%%%%%%%%%%%%%%%%%%%%%%%%%%%%%%%%%%%%%%%%%

\medskip
\goodbreak
\hrule
\nobreak
\smallskip
%% decoded old answers, saved. (keys = 
{\bf 6. {\footnotesize (0 pts) Library\-/OSU\-/high\_school\_apcalc\-/dchmwk4\-/prob7.pg}}\newline Let \[f(x) = -5 x^ {8} \cos(x)\]
\par 
\(f'( x ) =\) \mbox{\parbox[t]{20ex}{\hrulefill}}
\par{\small{\it Answer(s) submitted:}
\vspace{-\parskip}\begin{itemize}
\item\begin{verbatim}\end{verbatim}
\end{itemize}} (incorrect)\par
%%%%%%%%%%%%%%%%%%%%%%%%%%%%%%%%%%%%%%%%%%%%%%%%%%%%%%%%%%%%%%%%%%%%%%%%%%%%%%%%
% WeBWorK Online Homework Delivery System
% Copyright � 2000-2007 The WeBWorK Project, http://openwebwork.sf.net/
% $CVSHeader: webwork2/conf/snippets/hardcopyProblemDivider.tex,v 1.3 2004/06/24 21:10:50 dpvc Exp $
% 
% This program is free software; you can redistribute it and/or modify it under
% the terms of either: (a) the GNU General Public License as published by the
% Free Software Foundation; either version 2, or (at your option) any later
% version, or (b) the "Artistic License" which comes with this package.
% 
% This program is distributed in the hope that it will be useful, but WITHOUT
% ANY WARRANTY; without even the implied warranty of MERCHANTABILITY or FITNESS
% FOR A PARTICULAR PURPOSE.  See either the GNU General Public License or the
% Artistic License for more details.
%%%%%%%%%%%%%%%%%%%%%%%%%%%%%%%%%%%%%%%%%%%%%%%%%%%%%%%%%%%%%%%%%%%%%%%%%%%%%%%%

\medskip
\goodbreak
\hrule
\nobreak
\smallskip
%% decoded old answers, saved. (keys = 
{\bf 7. {\footnotesize (0 pts) Library\-/Michigan\-/Chap3Sec5\-/Q23.pg}}\newline Find the derivative of 
\(f(x)=e^{-6 x}\cdot\sin x\)

\par 
\(f'(x) =\) \mbox{\parbox[t]{32.5ex}{\hrulefill}}

\par{\small{\it Answer(s) submitted:}
\vspace{-\parskip}\begin{itemize}
\item\begin{verbatim}\end{verbatim}
\end{itemize}} (incorrect)\par
%% decoded old answers, saved. (keys = 
 \end{multicols}


\noindent {\tiny Generated by \copyright WeBWorK, http://webwork.maa.org, Mathematical Association of America}

 \begin{multicols}{2}
\columnwidth=\linewidth


%%%%%%%%%%%%%%%%%%%%%%%%%%%%%%%%%%%%%%%%%%%%%%%%%%%%%%%%%%%%%%%%%%%%%%%%%%%%%%%%
% WeBWorK Online Homework Delivery System
% Copyright � 2000-2007 The WeBWorK Project, http://openwebwork.sf.net/
% $CVSHeader: webwork2/conf/snippets/hardcopyPostamble.tex,v 1.2 2003/12/09 01:12:29 sh002i Exp $
% 
% This program is free software; you can redistribute it and/or modify it under
% the terms of either: (a) the GNU General Public License as published by the
% Free Software Foundation; either version 2, or (at your option) any later
% version, or (b) the "Artistic License" which comes with this package.
% 
% This program is distributed in the hope that it will be useful, but WITHOUT
% ANY WARRANTY; without even the implied warranty of MERCHANTABILITY or FITNESS
% FOR A PARTICULAR PURPOSE.  See either the GNU General Public License or the
% Artistic License for more details.
%%%%%%%%%%%%%%%%%%%%%%%%%%%%%%%%%%%%%%%%%%%%%%%%%%%%%%%%%%%%%%%%%%%%%%%%%%%%%%%%

\end{multicols}
\vfill
\end{document}