\documentclass[12pt,letterpaper]{article}

\usepackage{fancyhdr,fancybox}

%% Useful packages
\usepackage{amssymb, amsmath, amsthm} 
%\usepackage{graphicx}  %%this is currently enabled in the default document, so it is commented out here. 
\usepackage{calrsfs}
\usepackage{braket}
\usepackage{mathtools}
\usepackage{lipsum}
\usepackage{tikz}
\usetikzlibrary{cd}
\usepackage{verbatim}
%\usepackage{ntheorem}% for theorem-like environments
\usepackage{mdframed}%can make highlighted boxes of text
%Use case: https://tex.stackexchange.com/questions/46828/how-to-highlight-important-parts-with-a-gray-background
\usepackage{wrapfig}
\usepackage{centernot}
\usepackage{subcaption}%\begin{subfigure}{0.5\textwidth}
\usepackage{pgfplots}
\pgfplotsset{compat=1.13}
\usepackage[colorinlistoftodos]{todonotes}
\usepackage[colorlinks=true, allcolors=blue]{hyperref}
\usepackage{xfrac}					%to make slanted fractions \sfrac{numerator}{denominator}
\usepackage{enumitem}            
    %syntax: \begin{enumerate}[label=(\alph*)]
    %possible arguments: f \alph*, \Alph*, \arabic*, \roman* and \Roman*
\usetikzlibrary{arrows,shapes.geometric,fit}

\DeclareMathAlphabet{\pazocal}{OMS}{zplm}{m}{n}
%% Use \pazocal{letter} to typeset a letter in the other kind 
%%  of math calligraphic font. 

%% This puts the QED block at the end of each proof, the way I like it. 
\renewenvironment{proof}{{\bfseries Proof}}{\qed}
\makeatletter
\renewenvironment{proof}[1][\bfseries \proofname]{\par
  \pushQED{\qed}%
  \normalfont \topsep6\p@\@plus6\p@\relax
  \trivlist
  %\itemindent\normalparindent
  \item[\hskip\labelsep
        \scshape
    #1\@addpunct{}]\ignorespaces
}{%
  \popQED\endtrivlist\@endpefalse
}
\makeatother

%% This adds a \rewnewtheorem command, which enables me to override the settings for theorems contained in this document.
\makeatletter
\def\renewtheorem#1{%
  \expandafter\let\csname#1\endcsname\relax
  \expandafter\let\csname c@#1\endcsname\relax
  \gdef\renewtheorem@envname{#1}
  \renewtheorem@secpar
}
\def\renewtheorem@secpar{\@ifnextchar[{\renewtheorem@numberedlike}{\renewtheorem@nonumberedlike}}
\def\renewtheorem@numberedlike[#1]#2{\newtheorem{\renewtheorem@envname}[#1]{#2}}
\def\renewtheorem@nonumberedlike#1{  
\def\renewtheorem@caption{#1}
\edef\renewtheorem@nowithin{\noexpand\newtheorem{\renewtheorem@envname}{\renewtheorem@caption}}
\renewtheorem@thirdpar
}
\def\renewtheorem@thirdpar{\@ifnextchar[{\renewtheorem@within}{\renewtheorem@nowithin}}
\def\renewtheorem@within[#1]{\renewtheorem@nowithin[#1]}
\makeatother

%% This makes theorems and definitions with names show up in bold, the way I like it. 
\makeatletter
\def\th@plain{%
  \thm@notefont{}% same as heading font
  \itshape % body font
}
\def\th@definition{%
  \thm@notefont{}% same as heading font
  \normalfont % body font
}
\makeatother

%===============================================
%==============Shortcut Commands================
%===============================================
\newcommand{\ds}{\displaystyle}
\newcommand{\B}{\mathcal{B}}
\newcommand{\C}{\mathbb{C}}
\newcommand{\F}{\mathbb{F}}
\newcommand{\N}{\mathbb{N}}
\newcommand{\R}{\mathbb{R}}
\newcommand{\Q}{\mathbb{Q}}
\newcommand{\T}{\mathcal{T}}
\newcommand{\Z}{\mathbb{Z}}
\renewcommand\qedsymbol{$\blacksquare$}
\newcommand{\qedwhite}{\hfill\ensuremath{\square}}
\newcommand*\conj[1]{\overline{#1}}
\newcommand*\closure[1]{\overline{#1}}
\newcommand*\mean[1]{\overline{#1}}
%\newcommand{\inner}[1]{\left< #1 \right>}
\newcommand{\inner}[2]{\left< #1, #2 \right>}
\newcommand{\powerset}[1]{\pazocal{P}(#1)}
%% Use \pazocal{letter} to typeset a letter in the other kind 
%%  of math calligraphic font. 
\newcommand{\cardinality}[1]{\left| #1 \right|}
\newcommand{\domain}[1]{\mathcal{D}(#1)}
\newcommand{\image}{\text{Im}}
\newcommand{\inv}[1]{#1^{-1}}
\newcommand{\preimage}[2]{#1^{-1}\left(#2\right)}
\newcommand{\script}[1]{\mathcal{#1}}


\newenvironment{highlight}{\begin{mdframed}[backgroundcolor=gray!20]}{\end{mdframed}}

\DeclarePairedDelimiter\ceil{\lceil}{\rceil}
\DeclarePairedDelimiter\floor{\lfloor}{\rfloor}

%===============================================
%===============My Tikz Commands================
%===============================================
\newcommand{\drawsquiggle}[1]{\draw[shift={(#1,0)}] (.005,.05) -- (-.005,.02) -- (.005,-.02) -- (-.005,-.05);}
\newcommand{\drawpoint}[2]{\draw[*-*] (#1,0.01) node[below, shift={(0,-.2)}] {#2};}
\newcommand{\drawopoint}[2]{\draw[o-o] (#1,0.01) node[below, shift={(0,-.2)}] {#2};}
\newcommand{\drawlpoint}[2]{\draw (#1,0.02) -- (#1,-0.02) node[below] {#2};}
\newcommand{\drawlbrack}[2]{\draw (#1+.01,0.02) --(#1,0.02) -- (#1,-0.02) -- (#1+.01,-0.02) node[below, shift={(-.01,0)}] {#2};}
\newcommand{\drawrbrack}[2]{\draw (#1-.01,0.02) --(#1,0.02) -- (#1,-0.02) -- (#1-.01,-0.02) node[below, shift={(+.01,0)}] {#2};}

%***********************************************
%**************Start of Document****************
%***********************************************


%%
%% Page set-up:
%%
\pagestyle{empty}
\lhead{\textsc{3B - FTC and $u$-sub}} %=================UPDATE THIS=================%
\rhead{\textsc{Fall 2019}}
%\chead{\Large\textbf{A New Integration Technique \\ }}
\renewcommand{\headrulewidth}{1pt}
%
\renewcommand{\footrulewidth}{1pt}
\lfoot{
%\begin{tabular}{rll}
Office: \quad \quad \, M 2-3 \, \, SH 6431x \\
Math Lab: \, W 12-2 \, SH 1607
%\end{tabular}
}
\rfoot{trevorklar@math.ucsb.edu}


\setlength{\parindent}{0in}
\setlength{\textwidth}{7in}
\setlength{\evensidemargin}{-0.25in}
\setlength{\oddsidemargin}{-0.25in}
\setlength{\parskip}{.5\baselineskip}
\setlength{\topmargin}{-0.5in}
\setlength{\textheight}{9in}

\setlist[enumerate,1]{label=\textbf{\arabic*.}}

\begin{document}
\pagestyle{fancy}
\begin{center}
3B - Integral Calculus \\
Week 2 %=================UPDATE THIS=================%
\end{center}

\hrule

\begin{comment}
\begin{center}
\begin{tabular}{|rl|}
\hline
\multicolumn{2}{|c|}{Contact Information} \\
\hline
\bf{TA's name:} & Trevor Klar \\
\bf{Email:} & trevorklar@math.ucsb.edu \\
\bf{Office hours:} & Mondays 2:00-3:00 \\
\bf{Math Lab hours:} & Wednesdays 12:00-2:00 \\
\bf{Office:} & South Hall 6431x \\
\hline
\end{tabular}
\end{center}
\end{comment}

 %=================START OF WORKSHEET=================%

\noindent\textbf{Fundamental Theorem of Calculus}

\begin{itemize}

\item Fundamental Theorem of Calculus Part 1: If $g(x) = \ds \int_{a}^x f(t) \, dt$ then \hrulefill %$g'(x) = f(x)$. 

\item BE CAREFUL: If $\ds h(x) = \int^{\sin(x)}_1 4x \, dx $ then $h'(x) = $ \underline{\hspace{3cm}}

\item Fundamental Theorem of Calculus Part 2: If $F$ is an antiderivative of $f$, then $$\hspace{-1in}\ds \int_a^b f(x) \, dx = \text{\underline{\hspace{3cm}}}$$% F(a) - F(b)

\item What's the difference between definite and indefinite integrals? \hrulefill 
\\ \mbox{} \hrulefill
%\\ \mbox{} \hrulefill

\end{itemize}

\begin{itemize}
\item You Try!

(1) $\ds \int_0^2 x(2 + x^2) \, dx$ \hspace{1.5in} (2) Find $h'(x)$ if $h(x) = \ds \int_0^{x^2} \sqrt{1+r^3} \, dr$

%\vspace{1in}
\vfill

(3) $\ds\int \sqrt[3]{x} \, dx$

\end{itemize}


\noindent\textbf{$U$-Substitutions:}

\begin{itemize}

\item Strategy: (1) Choose $u$ to be \hrulefill \\
(2) Find $\du$ and substitute. You might need to \hrulefill \\
(3) Evaluate the integral. Then \hrulefill

\item Example: $\ds \int \sec^2(10x)\tan^7(10x) \, dx$

%\bigskip\bigskip\bigskip
\vfill
\end{itemize}

\begin{itemize}
\item You Try!

(1) $\ds \int \dfrac{x}{x^2 + 1} \, dx$ \hspace{2in} (2) $\ds\int \tan(x) \, dx$

%\bigskip\bigskip\bigskip
\vfill

\end{itemize}



% Net Change Theorem: The integral of a rate of change is the net change $\ds \int_a^b F'(x) \, dx = F(b) - F(a)$








\newpage


\noindent\textbf{Definite Integrals W/ U-Substitutions:}


\begin{itemize}

\item Strategy: (1) Choose $u$. (2) Find $\du$ and substitute. (3) Change the bounds. (\hrulefill \\
 %Plug in $x=a,b$ and find $u(a) and u(b)$.
\underline{\hspace{3in}}) (4) Evaluate the definite integral.  


\item Example: $\ds \int_{-\pi/40}^{\pi/40} \sec^2(10x)\tan^7(10x) \, dx$

\vfill
\end{itemize}

\begin{itemize}
\item You Try!

(1) $\ds \int_0^{\pi} \sec^2(t/4) \, dt$ \hspace{2in} (2) $\ds \int_0^2 (x-1)^{25} \, dx$

\vfill

\end{itemize}


\noindent\textbf{Integrals of Piecewise Functions and the Absolute Value Function:}

\begin{itemize}

\item Absolute value: $\ds |x| = \left\{ \begin{array}{cl} -x & x< 0 \\ x & x \geq 0 \end{array} \right. \text{   so   } \int_{-5}^5 |x| \, dx = $ \begin{tabular}{l}
\\
\underline{\hspace{2.5in}}
\end{tabular}

\bigskip

\item Piecewise Functions (example): If $\ds f(x) = \left\{ \begin{array}{cl} -x+3 & x \leq -1 \\ x^2 + 3 & x > -1 \end{array} \right. \text{ then }$ 

\medskip

$\ds \int_{-2}^2 f(x) \, dx= $

\vfill
\end{itemize}

\begin{itemize}
\item You Try! $\ds \int_{-3}^4 |x^2 - 4| \, dx$

\vfill


\end{itemize}




\end{document}


%%% Local Variables: 
%%% mode: latex
%%% TeX-master: t
%%% End: 
