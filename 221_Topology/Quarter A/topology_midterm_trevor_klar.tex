\documentclass[12pt,letterpaper]{article}

\usepackage{fancyhdr,fancybox}

%% Useful packages
\usepackage{amssymb, amsmath, amsthm} 
%\usepackage{graphicx}  %%this is currently enabled in the default document, so it is commented out here. 
\usepackage{calrsfs}
\usepackage{braket}
\usepackage{mathtools}
\usepackage{lipsum}
\usepackage{tikz}
\usetikzlibrary{cd}
\usepackage{verbatim}
%\usepackage{ntheorem}% for theorem-like environments
\usepackage{mdframed}%can make highlighted boxes of text
%Use case: https://tex.stackexchange.com/questions/46828/how-to-highlight-important-parts-with-a-gray-background
\usepackage{wrapfig}
\usepackage{centernot}
\usepackage{subcaption}%\begin{subfigure}{0.5\textwidth}
\usepackage{pgfplots}
\pgfplotsset{compat=1.13}
\usepackage[colorinlistoftodos]{todonotes}
\usepackage[colorlinks=true, allcolors=blue]{hyperref}
\usepackage{xfrac}					%to make slanted fractions \sfrac{numerator}{denominator}
\usepackage{enumitem}            
    %syntax: \begin{enumerate}[label=(\alph*)]
    %possible arguments: f \alph*, \Alph*, \arabic*, \roman* and \Roman*
\usetikzlibrary{arrows,shapes.geometric,fit}

\DeclareMathAlphabet{\pazocal}{OMS}{zplm}{m}{n}
%% Use \pazocal{letter} to typeset a letter in the other kind 
%%  of math calligraphic font. 

%% This puts the QED block at the end of each proof, the way I like it. 
\renewenvironment{proof}{{\bfseries Proof}}{\qed}
\makeatletter
\renewenvironment{proof}[1][\bfseries \proofname]{\par
  \pushQED{\qed}%
  \normalfont \topsep6\p@\@plus6\p@\relax
  \trivlist
  %\itemindent\normalparindent
  \item[\hskip\labelsep
        \scshape
    #1\@addpunct{}]\ignorespaces
}{%
  \popQED\endtrivlist\@endpefalse
}
\makeatother

%% This adds a \rewnewtheorem command, which enables me to override the settings for theorems contained in this document.
\makeatletter
\def\renewtheorem#1{%
  \expandafter\let\csname#1\endcsname\relax
  \expandafter\let\csname c@#1\endcsname\relax
  \gdef\renewtheorem@envname{#1}
  \renewtheorem@secpar
}
\def\renewtheorem@secpar{\@ifnextchar[{\renewtheorem@numberedlike}{\renewtheorem@nonumberedlike}}
\def\renewtheorem@numberedlike[#1]#2{\newtheorem{\renewtheorem@envname}[#1]{#2}}
\def\renewtheorem@nonumberedlike#1{  
\def\renewtheorem@caption{#1}
\edef\renewtheorem@nowithin{\noexpand\newtheorem{\renewtheorem@envname}{\renewtheorem@caption}}
\renewtheorem@thirdpar
}
\def\renewtheorem@thirdpar{\@ifnextchar[{\renewtheorem@within}{\renewtheorem@nowithin}}
\def\renewtheorem@within[#1]{\renewtheorem@nowithin[#1]}
\makeatother

%% This makes theorems and definitions with names show up in bold, the way I like it. 
\makeatletter
\def\th@plain{%
  \thm@notefont{}% same as heading font
  \itshape % body font
}
\def\th@definition{%
  \thm@notefont{}% same as heading font
  \normalfont % body font
}
\makeatother

%===============================================
%==============Shortcut Commands================
%===============================================
\newcommand{\ds}{\displaystyle}
\newcommand{\B}{\mathcal{B}}
\newcommand{\C}{\mathbb{C}}
\newcommand{\F}{\mathbb{F}}
\newcommand{\N}{\mathbb{N}}
\newcommand{\R}{\mathbb{R}}
\newcommand{\Q}{\mathbb{Q}}
\newcommand{\T}{\mathcal{T}}
\newcommand{\Z}{\mathbb{Z}}
\renewcommand\qedsymbol{$\blacksquare$}
\newcommand{\qedwhite}{\hfill\ensuremath{\square}}
\newcommand*\conj[1]{\overline{#1}}
\newcommand*\closure[1]{\overline{#1}}
\newcommand*\mean[1]{\overline{#1}}
%\newcommand{\inner}[1]{\left< #1 \right>}
\newcommand{\inner}[2]{\left< #1, #2 \right>}
\newcommand{\powerset}[1]{\pazocal{P}(#1)}
%% Use \pazocal{letter} to typeset a letter in the other kind 
%%  of math calligraphic font. 
\newcommand{\cardinality}[1]{\left| #1 \right|}
\newcommand{\domain}[1]{\mathcal{D}(#1)}
\newcommand{\image}{\text{Im}}
\newcommand{\inv}[1]{#1^{-1}}
\newcommand{\preimage}[2]{#1^{-1}\left(#2\right)}
\newcommand{\script}[1]{\mathcal{#1}}


\newenvironment{highlight}{\begin{mdframed}[backgroundcolor=gray!20]}{\end{mdframed}}

\DeclarePairedDelimiter\ceil{\lceil}{\rceil}
\DeclarePairedDelimiter\floor{\lfloor}{\rfloor}

%===============================================
%===============My Tikz Commands================
%===============================================
\newcommand{\drawsquiggle}[1]{\draw[shift={(#1,0)}] (.005,.05) -- (-.005,.02) -- (.005,-.02) -- (-.005,-.05);}
\newcommand{\drawpoint}[2]{\draw[*-*] (#1,0.01) node[below, shift={(0,-.2)}] {#2};}
\newcommand{\drawopoint}[2]{\draw[o-o] (#1,0.01) node[below, shift={(0,-.2)}] {#2};}
\newcommand{\drawlpoint}[2]{\draw (#1,0.02) -- (#1,-0.02) node[below] {#2};}
\newcommand{\drawlbrack}[2]{\draw (#1+.01,0.02) --(#1,0.02) -- (#1,-0.02) -- (#1+.01,-0.02) node[below, shift={(-.01,0)}] {#2};}
\newcommand{\drawrbrack}[2]{\draw (#1-.01,0.02) --(#1,0.02) -- (#1,-0.02) -- (#1-.01,-0.02) node[below, shift={(+.01,0)}] {#2};}

%***********************************************
%**************Start of Document****************
%***********************************************
 %find me at /home/trevor/texmf/tex/latex/tskpreamble_nothms.tex
%===============================================
%===============Theorem Styles==================
%===============================================

%================Default Style==================
\theoremstyle{plain}% is the default. it sets the text in italic and adds extra space above and below the \newtheorems listed below it in the input. it is recommended for theorems, corollaries, lemmas, propositions, conjectures, criteria, and (possibly; depends on the subject area) algorithms.
\newtheorem{theorem}{Theorem}
\numberwithin{theorem}{section} %This sets the numbering system for theorems to number them down to the {argument} level. I have it set to number down to the {section} level right now.
\newtheorem*{theorem*}{Theorem} %Theorem with no numbering
\newtheorem{corollary}[theorem]{Corollary}
\newtheorem*{corollary*}{Corollary}
\newtheorem{conjecture}[theorem]{Conjecture}
\newtheorem{lemma}[theorem]{Lemma}
\newtheorem*{lemma*}{Lemma}
\newtheorem{proposition}[theorem]{Proposition}
\newtheorem*{proposition*}{Proposition}
\newtheorem{problemstatement}[theorem]{Problem Statement}


%==============Definition Style=================
\theoremstyle{definition}% adds extra space above and below, but sets the text in roman. it is recommended for definitions, conditions, problems, and examples; i've alse seen it used for exercises.
\newtheorem{definition}[theorem]{Definition}
\newtheorem*{definition*}{Definition}
\newtheorem{condition}[theorem]{Condition}
\newtheorem{problem}[theorem]{Problem}
\newtheorem{example}[theorem]{Example}
\newtheorem*{example*}{Example}
\newtheorem*{counterexample*}{Counterexample}
\newtheorem*{romantheorem*}{Theorem} %Theorem with no numbering
\newtheorem{exercise}{Exercise}
\numberwithin{exercise}{section}
\newtheorem{algorithm}[theorem]{Algorithm}

%================Remark Style===================
\theoremstyle{remark}% is set in roman, with no additional space above or below. it is recommended for remarks, notes, notation, claims, summaries, acknowledgments, cases, and conclusions.
\newtheorem{remark}[theorem]{Remark}
\newtheorem*{remark*}{Remark}
\newtheorem{notation}[theorem]{Notation}
\newtheorem*{notation*}{Notation}
%\newtheorem{claim}[theorem]{Claim}  %%use this if you ever want claims to be numbered
\newtheorem*{claim}{Claim}


\renewcommand{\T}{\pazocal{T}}

\DeclareMathOperator{\diameter}{diam}
\newcommand{\diam}[1]{\diameter\left(#1\right)}
\renewcommand{\dist}[2]{d\!\left(#1,#2\right)}

%%
%% Page set-up:
%%
\pagestyle{empty}
\lhead{\textsc{221 - Topology} \\} %=================UPDATE THIS=================%
\rhead{\textsc{Long, Fall 2019}}
%\chead{\Large\textbf{A New Integration Technique \\ }}
\renewcommand{\headrulewidth}{0pt}
%
\renewcommand{\footrulewidth}{0pt}
%\lfoot{
%Office: \quad \quad \, M 2-3 \, \, SH 6431x \\
%Math Lab: \, W 12-2 \, SH 1607
%}
%\rfoot{trevorklar@math.ucsb.edu}


\setlength{\parindent}{0in}
\setlength{\textwidth}{7in}
\setlength{\evensidemargin}{-0.25in}
\setlength{\oddsidemargin}{-0.25in}
\setlength{\parskip}{.5\baselineskip}
\setlength{\topmargin}{-0.5in}
\setlength{\textheight}{9in}

\setlist[enumerate,1]{label=\textbf{\arabic*.}}

\begin{document}
\thispagestyle{fancy}
\begin{center}
{{\Large Midterm} \\Trevor Klar }%=================UPDATE THIS=================%
\end{center}

\pagebreak
\begin{enumerate}
\item Explain how one defines the \emph{product topology} and the \emph{subspace topology}. Using the definitions you have given, show that $Y$ is homeomorphic to $\{x\}\times Y$ when equipped with the subspace topology (considered as a subset of $X\times Y$).
\begin{definition*}
Let $(X,\pazocal{T}_X), (Y,\pazocal{T}_Y)$ be topological spaces. We denote the \emph{product topology on} $X\times Y$ by $\pazocal{T}_{X\times Y}$. A set $U\subseteq X\times Y$ is open with respect to $\pazocal{T}_{X\times Y}$ if there exist open sets $U_\alpha\in \pazocal{T}_X$, $V_\alpha\in \pazocal{T}_Y$ such that 
$$U=\bigcup_\alpha U_\alpha\times V_\alpha.$$
\end{definition*}
\begin{definition*}
Let $(X,\pazocal{T}_X)$ be a topological space, and let $A\subseteq X$. We denote the \emph{subspace topology on} $A$ by $\pazocal{T}_A$. A set $U\subseteq A$ is open with respect to $\pazocal{T}_A$ if there exists a open set $\tilde{U}\in\pazocal{T}_X$ such that 
$$\tilde{U}\cap A=U.$$ 
\end{definition*}
\begin{proof}
Let $x\in X$. We will show that the inclusion map $\iota:Y\hookrightarrow\{x\}\times Y$ (whose inverse is the projection map $\pi:\{x\}\times Y\to Y$) is a homeomorphism.
	\begin{itemize}
	\item $\iota$ is 1-1 and onto, since 
	$$\iota(y_1)=\iota(y_2)\implies(x,y_1)=(x,y_2)\implies y_1=y_2,$$
	and for any $(x,y)\in \{x\}\times Y$, there exists $y\in Y$ with $\iota(y)=(x,y)$.
	
	\item $\iota$ and $\pi$ are inverses, since for any $y\in Y$,
	\begin{align*}
	&\iota\circ\pi(x,y)=\iota(y)=(x,y), \text{ and } \\
	&\pi\circ\iota(y)=\pi(x,y)=y.
	\end{align*}
	
	\item $\iota$ is an open map, since for any $V\in\pazocal{T}_Y$, 
	$$\iota(V)=\{x\}\times V,$$
	and we can choose any $U\in\pazocal{T}_X$ containing $x$ to see that 
	$$U\times V \cap \{x\}\times Y = \{x\}\times V,$$ 
	so $\{x\}\times V$ is open in the subspace topology of the product topology. 
	
	\item $\pi$ is an open map, since any open set in $\{x\}\times Y$ is of the form $U\times V \cap \{x\}\times Y = \{x\}\times V$, where $U\in\pazocal{T}_X, V\in \pazocal{T}_Y$. Then 
	$$\pi\left(\{x\}\times V\right)=V,$$
	Which is open in $Y$. 
	\end{itemize}
Thus $\iota$ is a continuous bijection with continuous inverse, and therefore a homeomorphism. 
\end{proof}

\pagebreak
\item Suppose that $(X,\pazocal{T})$ is a topological space with two properties: namely $X$ is compact and Hausdorff. 

Prove that one cannot make the topology on on $X$ either coarser (i.e. is a strict subset of $\pazocal{T}$) or finer (i.e. is a strict superset of $\pazocal{T}$) without destroying one of those properties. 
\begin{proof}We will show that (i) any topology finer than $\T$ is not compact, and (ii) any topology courser than $\T$ is not Hausdorff. If we call a set "open" without reference to $\T$ or $\T'$, then it is open in both topologies.  The same goes for "closed".
	\begin{enumerate}[label=(\roman*)]
	\item Suppose $\T'\supsetneq\T$. Then there exists $W\in\T'$ which is not open in $\T$. To see that $(X,\T')$ is not compact, we will produce a covering of $X$ consisting of sets in $\T'$ which has no finite subcovering. 
	%More specifically, we will use $W$ together with a covering of $W^\complement=X\setminus W$. 
	Since $W$ is not open in $\T$, then $W^\complement$ is not closed in $\T$, so there exists some $x\in W$ which is also in $\closure{W^\complement}$. Since $(X,\T)$ is Hausdorff, for any $y\in W^\complement$, we can find $U_y\ni x$, $V_y\ni y$ with $U_y, V_y$ open and disjoint. Then 
	$$W\cup\bigcup_{y\in W^\complement} V_y$$
	covers $X$. Suppose for contradiction that this covering has a finite subcovering, call it $W\cup\bigcup_{i=1}^NV_{y_i}.$	Then since each $V$ has a corresponding $U$, then $\bigcap_{i=1}^NU_{y_i}$ is an open set containing $x\in\closure{W^\complement}$, so 
	$$\bigcap_{i=1}^NU_{y_i}\cap W^\complement\neq\emptyset.$$
	But $\bigcap_{i=1}^NU_{y_i}\subset U_{y_i}$ for all $U_{y_i}$, so
	$$\bigcap_{i=1}^NU_{y_i}\cap V_{y_i}=\emptyset$$ 
	for all $V_{y_i}$, so $\bigcap_{i=1}^NU_{y_i}$ and $\bigcup_{i=1}^NV_{y_i}$ are disjoint. This means that $W\cup\bigcup_{i=1}^NV_{y_i}$ doesn't cover $\bigcap_{i=1}^NU_{y_i}\cap W^\complement$, contradiction. 
	
	\item Let $\T'\subseteq\T$. We will show that if $(X,\T')$ is Hausdorff, then $\T'\supseteq \T$, so $\T'= \T$. Let $W$ be open in $\T$, and let $x\in W$. Since $(X,\T')$ is Hausdorff, then for every $y\in W^\complement$, there exist sets $U_y'\ni x$, $V_y'\ni y$ which are open in $\T'$ and disjoint. Since $W^\complement$ is closed in $(X,\T)$, and $(X,\T)$ is compact, and $\{V_y'\}_{v\in W^\complement}\subset\T'\subseteq\T$, then we can produce a finite subcover $\{V_{y_i}'\}_{i=1}^N$ of $W^\complement$. Now for each $V'$ we have a corresponding $U'$, so by similar reasoning as in (i) we find that $\bigcap_{i=1}^NU_{y_i}'$ is open in $\T'$, disjoint with $\bigcup_{i=1}^NV_{y_i}'$ so a subset of $W$, and contains $x$ by construction. Thus by the openness criterion, $W$ is open in $\T'$. 
	\end{enumerate}
\end{proof}

\pagebreak 
\item Let $X$ and $Y$ be topological spaces and let $f:X\to Y$ be a function. 

\begin{enumerate}[label=(\roman*)]
\item Show that $f$ is continuous if and only if $f(\closure{A})\subset \closure{f(A)}$ for all $A\subset X$, where $\closure{A}$ denotes the closure of $A$. 
\begin{proof}\mbox{} 

($\implies$) %If $A$ is closed, then we're done since $f(\closure{A}=f(A)\subset\closure{f(A)}$. So suppose $A$ is not closed, and let $x\in\closure{A}$, but $x\not\in A$. 
Let $A\subset X$, let $x\in\closure{A}$, and let $V\in Y$ be any open set containing $f(x)$. Then since $f$ is continuous, $\preimage{f}{V}$ contains $x$ and is open. Since $x\in\closure{A}$, then $\preimage{f}{V}\cap A\neq\emptyset$, so there exists $a\in \preimage{f}{V}\cap A$ such that $f(a)\in V\cap f(A)$. Thus $f(x)\in\closure{f(A)}$, and we are done. \qedwhite

($\impliedby$) Let $D\in Y$ be closed. Then $f\left(\closure{\preimage{f}{D}}\right)\subset\closure{f(\preimage{f}{D})}=\closure{D}=D$, so $\closure{\preimage{f}{D}}\subseteq\preimage{f}{D}$ and it is always true that $\closure{\preimage{f}{D}}\supseteq\preimage{f}{D}$, therefore they are equal and $\preimage{f}{D}$ is closed. 
\end{proof}

\item Show that if $f$ is continuous and $f(\closure{A})$ is closed, then $f(\closure{A})=\closure{f(A)}$. 
\begin{proof}
We know already that $f(\closure{A})\subseteq\closure{f(A)}$ by (i), so we need to show that $\closure{f(A)}\subseteq f(\closure{A})$. 

%So let $y\in \closure{f(A)}$. For any open $V\ni y$, $V\cap f(A)\neq\emptyset$, so there exists $a\in A$ such that $f(a)\in V$. 
%
%Then $\preimage{f}{V}$ is open because $f$ is continuous, and contains $\preimage{f}{y}$ and $a$.
Let $y\in \closure{f(A)}$. %, and let $U\ni\preimage{f}{y}$ be open. If $U,A$ are not disjoint, then $\preimage{f}{y}\in\closure{A}$ and we're done. So consider the case where $U,A$ are disjoint. Then $y\not\in f(A)$. 
This means that for any open $V\ni y$, we have $V\cap f(A)\neq\emptyset$.
%This means that ($\dagger$) for any open $V\ni y$, we have $V\cap f(A)\neq\emptyset$, so there exists $a\in A$ such that $f(a)\in V$. 

Now suppose for contradiction that $y\not\in f(\closure{A}).$ Since $f(\closure{A})$ is closed, then $f(\closure{A})^\complement$ is open, and $y\in f(\closure{A})^\complement$. So there exists an open set $V'$ such that $y\in V'\subset f(\closure{A})^\complement$ and furthermore that $V'\subset f(A)^\complement$. But since $y\in \closure{f(A)}$, then $V'\cap f(A)\neq\emptyset$, contradiction. 
\end{proof}
\end{enumerate}

\item Let $(X,d)$ be a compact metric space and let $f:X\to X$ be a map with the property that 
$$\dist{f(x)}{f(y)}<\dist{x}{y}$$
for every distinct $x,y\in X$. 

Prove that (i) there is a unique point $x_0$ with $f(x_0)=x_0$, and (ii) show that this fails if the inequality is not always strict. 
	\begin{proof}
	(i) If such a point exists, it is unique. If not and there exist two invariant points $x_0\neq x_1$, then $\dist{x_0}{x_1}>\dist{f(x_0)}{f(x_1)}=\dist{x_0}{x_1}$, contradiction. 
	
	It remains to be shown that there exists an invariant point. 
	Observe that $f$ is continuous (and in fact, uniformly continuous), since for any $\epsilon>0$, if $\dist{x}{y}<\epsilon$, then $\dist{f(x)}{f(y)}<\epsilon$. %(we will need this fact later). 
	
	Let $x_1\in X$, and define $\{x_n\}_{n=1}^\infty$ by $x_{n+1}=f(x_n)$ for all $n>1$. Since $X$ is compact, then it is sequentially compact, so $\{x_n\}$ has a convergent subsequence (and \Wlog{} suppose that subsequence is $\{x_n\}$ itself), and call the limit $x_0$. Then 
	\begin{align*}
	f(x_0)&=f\left(\lim_{n\to\infty}(x_n)\right) & \text{and since $f$ is continuous,}\\
	&=\lim_{n\to\infty}f\left(x_n\right)\\
	&=\lim_{n\to\infty}(x_{n+1})\\
	&=x_0.
	\end{align*}
		
	(ii) If we modify the property to be $\dist{f(x)}{f(y)}\leq\dist{x}{y},$	then we do not have uniqueness since the identity $f(x)=x$ satisfies the property. Existence is still guaranteed.
	\end{proof}

\item Let $A$ and $B$ be disjoint compact subspaces of a Hausdorff topological space $X$. Prove that there are disjoint open sets $U$ and $V$ such that $U\supset A$ and $V\supset B$.
\begin{proof}
Let $x\in A$. Since $X$ is Hausdorff, for every $y\in B$ there exist disjoint open sets $U_{x_y}\ni x$ and $V_{x_y}\ni y$. Then $\bigcup_{y\in B}V_{x_y}$ is a covering of the compact set $B$, so it has a finite subcover $\bigcup_{i=1}^NV_{x_{y_i}}$. Using the corresponding $U$ sets, define 
\begin{align*}
U_x=\bigcap_{i=1}^NU_{x_{y_i}} \quad\text{and}\quad V_x=\bigcup_{i=1}^NV_{x_{y_i}},
\end{align*}
and observe that $U_x$ and $V_x$ are disjoint open sets such that $U_x\ni x$ and $V_x$ covers $B$.\footnote{They are disjoint because $\forall i, U_x\subset U_{x_{y_{i}}}\subset V_{x_{y_{i}}}^\complement$, open because finite unions and intersections of open sets are open, and cover $x$ and $B$ respectively by construction.}

Construct similarly $U_x$ and $V_x$ for every $x\in A$. Then $\bigcup_{x\in A}U_x$ is an open cover of the compact set $A$, so it has finite subcover $\bigcup_{i=1}^NU_{x_i}$. Using the corresponding $V$ sets, define 
\begin{align*}
U=\bigcup_{i=1}^NU_{x_i} \quad\text{and}\quad V=\bigcap_{i=1}^NV_{x_i},
\end{align*}
and observe that since every $V_{x_i}$ covers $B$ and is disjoint with $U_{x_i}$, then $U\supset A$, $V\supset B$, and $U,V$ are open and disjoint. 
\end{proof}

\end{enumerate}

\end{document}


%%% Local Variables: 
%%% mode: latex
%%% TeX-master: t
%%% End: 
