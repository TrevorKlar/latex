\documentclass[12pt,letterpaper]{article}

\usepackage{fancyhdr,fancybox}

%% Useful packages
\usepackage{amssymb, amsmath, amsthm} 
%\usepackage{graphicx}  %%this is currently enabled in the default document, so it is commented out here. 
\usepackage{calrsfs}
\usepackage{braket}
\usepackage{mathtools}
\usepackage{lipsum}
\usepackage{tikz}
\usetikzlibrary{cd}
\usepackage{verbatim}
%\usepackage{ntheorem}% for theorem-like environments
\usepackage{mdframed}%can make highlighted boxes of text
%Use case: https://tex.stackexchange.com/questions/46828/how-to-highlight-important-parts-with-a-gray-background
\usepackage{wrapfig}
\usepackage{centernot}
\usepackage{subcaption}%\begin{subfigure}{0.5\textwidth}
\usepackage{pgfplots}
\pgfplotsset{compat=1.13}
\usepackage[colorinlistoftodos]{todonotes}
\usepackage[colorlinks=true, allcolors=blue]{hyperref}
\usepackage{xfrac}					%to make slanted fractions \sfrac{numerator}{denominator}
\usepackage{enumitem}            
    %syntax: \begin{enumerate}[label=(\alph*)]
    %possible arguments: f \alph*, \Alph*, \arabic*, \roman* and \Roman*
\usetikzlibrary{arrows,shapes.geometric,fit}

\DeclareMathAlphabet{\pazocal}{OMS}{zplm}{m}{n}
%% Use \pazocal{letter} to typeset a letter in the other kind 
%%  of math calligraphic font. 

%% This puts the QED block at the end of each proof, the way I like it. 
\renewenvironment{proof}{{\bfseries Proof}}{\qed}
\makeatletter
\renewenvironment{proof}[1][\bfseries \proofname]{\par
  \pushQED{\qed}%
  \normalfont \topsep6\p@\@plus6\p@\relax
  \trivlist
  %\itemindent\normalparindent
  \item[\hskip\labelsep
        \scshape
    #1\@addpunct{}]\ignorespaces
}{%
  \popQED\endtrivlist\@endpefalse
}
\makeatother

%% This adds a \rewnewtheorem command, which enables me to override the settings for theorems contained in this document.
\makeatletter
\def\renewtheorem#1{%
  \expandafter\let\csname#1\endcsname\relax
  \expandafter\let\csname c@#1\endcsname\relax
  \gdef\renewtheorem@envname{#1}
  \renewtheorem@secpar
}
\def\renewtheorem@secpar{\@ifnextchar[{\renewtheorem@numberedlike}{\renewtheorem@nonumberedlike}}
\def\renewtheorem@numberedlike[#1]#2{\newtheorem{\renewtheorem@envname}[#1]{#2}}
\def\renewtheorem@nonumberedlike#1{  
\def\renewtheorem@caption{#1}
\edef\renewtheorem@nowithin{\noexpand\newtheorem{\renewtheorem@envname}{\renewtheorem@caption}}
\renewtheorem@thirdpar
}
\def\renewtheorem@thirdpar{\@ifnextchar[{\renewtheorem@within}{\renewtheorem@nowithin}}
\def\renewtheorem@within[#1]{\renewtheorem@nowithin[#1]}
\makeatother

%% This makes theorems and definitions with names show up in bold, the way I like it. 
\makeatletter
\def\th@plain{%
  \thm@notefont{}% same as heading font
  \itshape % body font
}
\def\th@definition{%
  \thm@notefont{}% same as heading font
  \normalfont % body font
}
\makeatother

%===============================================
%==============Shortcut Commands================
%===============================================
\newcommand{\ds}{\displaystyle}
\newcommand{\B}{\mathcal{B}}
\newcommand{\C}{\mathbb{C}}
\newcommand{\F}{\mathbb{F}}
\newcommand{\N}{\mathbb{N}}
\newcommand{\R}{\mathbb{R}}
\newcommand{\Q}{\mathbb{Q}}
\newcommand{\T}{\mathcal{T}}
\newcommand{\Z}{\mathbb{Z}}
\renewcommand\qedsymbol{$\blacksquare$}
\newcommand{\qedwhite}{\hfill\ensuremath{\square}}
\newcommand*\conj[1]{\overline{#1}}
\newcommand*\closure[1]{\overline{#1}}
\newcommand*\mean[1]{\overline{#1}}
%\newcommand{\inner}[1]{\left< #1 \right>}
\newcommand{\inner}[2]{\left< #1, #2 \right>}
\newcommand{\powerset}[1]{\pazocal{P}(#1)}
%% Use \pazocal{letter} to typeset a letter in the other kind 
%%  of math calligraphic font. 
\newcommand{\cardinality}[1]{\left| #1 \right|}
\newcommand{\domain}[1]{\mathcal{D}(#1)}
\newcommand{\image}{\text{Im}}
\newcommand{\inv}[1]{#1^{-1}}
\newcommand{\preimage}[2]{#1^{-1}\left(#2\right)}
\newcommand{\script}[1]{\mathcal{#1}}


\newenvironment{highlight}{\begin{mdframed}[backgroundcolor=gray!20]}{\end{mdframed}}

\DeclarePairedDelimiter\ceil{\lceil}{\rceil}
\DeclarePairedDelimiter\floor{\lfloor}{\rfloor}

%===============================================
%===============My Tikz Commands================
%===============================================
\newcommand{\drawsquiggle}[1]{\draw[shift={(#1,0)}] (.005,.05) -- (-.005,.02) -- (.005,-.02) -- (-.005,-.05);}
\newcommand{\drawpoint}[2]{\draw[*-*] (#1,0.01) node[below, shift={(0,-.2)}] {#2};}
\newcommand{\drawopoint}[2]{\draw[o-o] (#1,0.01) node[below, shift={(0,-.2)}] {#2};}
\newcommand{\drawlpoint}[2]{\draw (#1,0.02) -- (#1,-0.02) node[below] {#2};}
\newcommand{\drawlbrack}[2]{\draw (#1+.01,0.02) --(#1,0.02) -- (#1,-0.02) -- (#1+.01,-0.02) node[below, shift={(-.01,0)}] {#2};}
\newcommand{\drawrbrack}[2]{\draw (#1-.01,0.02) --(#1,0.02) -- (#1,-0.02) -- (#1-.01,-0.02) node[below, shift={(+.01,0)}] {#2};}

%***********************************************
%**************Start of Document****************
%***********************************************
 %find me at /home/trevor/texmf/tex/latex/tskpreamble_nothms.tex
%===============================================
%===============Theorem Styles==================
%===============================================

%================Default Style==================
\theoremstyle{plain}% is the default. it sets the text in italic and adds extra space above and below the \newtheorems listed below it in the input. it is recommended for theorems, corollaries, lemmas, propositions, conjectures, criteria, and (possibly; depends on the subject area) algorithms.
\newtheorem{theorem}{Theorem}
\numberwithin{theorem}{section} %This sets the numbering system for theorems to number them down to the {argument} level. I have it set to number down to the {section} level right now.
\newtheorem*{theorem*}{Theorem} %Theorem with no numbering
\newtheorem{corollary}[theorem]{Corollary}
\newtheorem*{corollary*}{Corollary}
\newtheorem{conjecture}[theorem]{Conjecture}
\newtheorem{lemma}[theorem]{Lemma}
\newtheorem*{lemma*}{Lemma}
\newtheorem{proposition}[theorem]{Proposition}
\newtheorem*{proposition*}{Proposition}
\newtheorem{problemstatement}[theorem]{Problem Statement}


%==============Definition Style=================
\theoremstyle{definition}% adds extra space above and below, but sets the text in roman. it is recommended for definitions, conditions, problems, and examples; i've alse seen it used for exercises.
\newtheorem{definition}[theorem]{Definition}
\newtheorem*{definition*}{Definition}
\newtheorem{condition}[theorem]{Condition}
\newtheorem{problem}[theorem]{Problem}
\newtheorem{example}[theorem]{Example}
\newtheorem*{example*}{Example}
\newtheorem*{counterexample*}{Counterexample}
\newtheorem*{romantheorem*}{Theorem} %Theorem with no numbering
\newtheorem{exercise}{Exercise}
\numberwithin{exercise}{section}
\newtheorem{algorithm}[theorem]{Algorithm}

%================Remark Style===================
\theoremstyle{remark}% is set in roman, with no additional space above or below. it is recommended for remarks, notes, notation, claims, summaries, acknowledgments, cases, and conclusions.
\newtheorem{remark}[theorem]{Remark}
\newtheorem*{remark*}{Remark}
\newtheorem{notation}[theorem]{Notation}
\newtheorem*{notation*}{Notation}
%\newtheorem{claim}[theorem]{Claim}  %%use this if you ever want claims to be numbered
\newtheorem*{claim}{Claim}


\DeclareMathOperator{\diameter}{diam}
\newcommand{\diam}[1]{\diameter\left(#1\right)}

%%
%% Page set-up:
%%
\pagestyle{empty}
\lhead{\textsc{221 - Topology} \\} %=================UPDATE THIS=================%
\rhead{\textsc{Long, Fall 2019}\\Trevor Klar}
%\chead{\Large\textbf{A New Integration Technique \\ }}
\renewcommand{\headrulewidth}{0pt}
%
\renewcommand{\footrulewidth}{0pt}
%\lfoot{
%Office: \quad \quad \, M 2-3 \, \, SH 6431x \\
%Math Lab: \, W 12-2 \, SH 1607
%}
%\rfoot{trevorklar@math.ucsb.edu}


\setlength{\parindent}{0in}
\setlength{\textwidth}{7in}
\setlength{\evensidemargin}{-0.25in}
\setlength{\oddsidemargin}{-0.25in}
\setlength{\parskip}{.5\baselineskip}
\setlength{\topmargin}{-0.5in}
\setlength{\textheight}{9in}

\setlist[enumerate,1]{label=\textbf{\arabic*.}}

\begin{document}
\thispagestyle{fancy}
\begin{center}
{\Large Homework 1 \\ }%=================UPDATE THIS=================%
\end{center}

\begin{definition*}
Let $\Sigma$ be a subset of $(X,d)$. We say $\Sigma$ is \textbf{bounded} if there exists $x_0\in X$ and $0<r<\infty$ such that $\Sigma\subset B_r(x_0)$. 
\end{definition*}

\begin{enumerate}
\item Prove that $\Sigma$ is bounded if and only if there exists $L>0$ such that
$$d(x,x')\leq L$$
for any $x, x'\in\Sigma$.
\begin{proof}($\implies$) Suppose $\Sigma$ is bounded by $B_r(x_0)$. Then for any $x,x'\in\Sigma$, we know $d(x,x_0)<r$ and $d(x',x_0)<r$ since $\Sigma\subset B_r(x_0)$. Thus, by triangle inequality, $d(x,x')<2r$, so we let $L=2r$ and we are done. \qedwhite

($\impliedby$) Suppose there exists $L>0$ such that $d(x,x')\leq L$ for any $x, x'\in\Sigma$. Then fix any $x_0\in\Sigma$, and $\Sigma$ is bounded by $B_L(x_0)$, since for any $x\in\Sigma,$ we have $d(x,x_0)\leq L$. 
\end{proof}

\item Suppose $\Sigma$ is bounded and $A\subset\Sigma$. 
\begin{enumerate}
\item Prove that $A$ is bounded. 
\begin{proof}
Obvious.\footnote{I can't write "obvious" on a homework problem? All right. Observe that $A\subset\Sigma\subset B_r(x_0)$, so $A\subset B_r(x_0)$.}
\end{proof}
\end{enumerate}
\begin{definition*}
Define $\diam{\Sigma} = \sup\limits_{\delta, \delta'\in\Sigma} d(\delta, \delta')$.
\end{definition*}
\begin{enumerate}[resume]
\item Prove that $\diam{A}\leq\diam{\Sigma}$. 
\begin{proof}
For any $x,x'\in A$, we also know $x,x'\in\Sigma$, so $d(x,x')\leq\diam{\Sigma}$. Since $\diam{\Sigma}$ is an upper bound for $d|_{A\times A}$, then $\diam{A}=\sup\limits_{\delta, \delta'\in A} d(\delta, \delta') \leq \diam{\Sigma}$. 
\end{proof}
\end{enumerate}

\item Let $(X,d)$ be a metric space, and let 
$$d_p\big((x,y),(x',y')\big)=d(x,x')+d(y,y').$$
\begin{enumerate}
\item Show that $d_p$ is a metric on $X^2$. 
\begin{proof}\mbox{}
\begin{itemize}
\item Since $(x,y)=(x',y')$ iff both $x=x'$ and $y=y'$, and since $d$ is a metric, then 
\begin{align*}
d_p\big((x,y),(x',y')\big)=0 &\iff d(x,x')=0 \text{ and } d(y,y')=0 \\
&\iff x=x' \text{ and } y=y',
\end{align*}
so $d_p$ is positive-definite.
\item To see that $d_p$ is symmetric, observe that $d$ is symmetric, so 
\begin{align*}
d_p\big((x,y),(x',y')\big)&=d(x,x')+d(y,y') \\
&=d(x',x)+d(y',y) \\
&=d_p\big((x',y'),(x,y)\big).
\end{align*}
\item Now we show that the triangle inequality holds. 
\begin{align*}
d_p\big((x_1,x_2),(y_1,y_2)\big)+d_p\big((y_1,y_2),(z_1,z_2)\big)&=d(x_1,y_1)+d(x_2,y_2)+d(y_1,z_1)+d(y_2,z_2)\\
&=d(x_1,y_1)+d(y_1,z_1)+d(x_2,y_2)+d(y_2,z_2)\\
&\geq d(x_1,z_1)+d(x_2,z_2)
\end{align*}
\end{itemize}
Thus $d_p$ is positive-definite, symmetric, and has the triangle inequality, so it is a metric on $X^2$. \qedhere
\end{proof}
\item Prove that $d:X\times X \to (\R, \text{MKM})$ is continuous. 
\begin{proof}
Let $r\in\R$, be given. Then choose $x,y \in X$ such that $d(x,y)\leq r$. Let $\epsilon>0$.

Observe that for any $(x', y')\in B_{\frac{\epsilon}{2}}\big((x,y)\big)$,  
\begin{align*}
d(x,x')+d(y,y') = d_p\big((x',y'),(x,y)\big) < \frac{\epsilon}{2},
\end{align*}
so $d(x,x')<\frac{\epsilon}{2}$ and $d(y,y')<\frac{\epsilon}{2}$. Now by the triangle inequality, 
$$d(x',y')\leq d(x', x) + d(x,y)+ d(y,y') = r+\epsilon\phantom{,}$$
and 
$$d(x',y') \geq d(x,y) - d(x', x) - d(y,y') = r-\epsilon,$$
\jpg{width=0.4\textwidth}{hw1-3b}
so $d(x', y')\in B_\epsilon(r)$, which means that $d$ is continuous by the $\delta$-$\epsilon$ definition.
\end{proof}
%\begin{proof}
%We have shown that $d_p$ is a metric, so it suffices to show that any metric is continuous. Let $(X,d)$ be a metric space. Let $B_r(c)$ be any ball in $(\R, \text{MKM})$\footnote{WLOG, assume that $c>0$ and $r\leq c$, since metrics can only map to balls of this form. If you chose a ball not according to these restrictions, you could intersect it with $[0,\infty)$ to fix the proof.}. Choose $x,y\in X$ such that $d(x,y)=c$. Then let 
%\begin{align*}
%U&=B_{r/2}(x)\subset(X,d)\\
%V&=B_{r/2}(y)\subset(X,d).
%\end{align*}
%Now $(U\times V)$ is open in the product topology (are we supposed to know this yet?), and $D(U\times V) \subset d^{-1}\left(B_r(c)\right)$\footnote{If this is not obvious, observe that any $\hat{y}\in V$ can be no closer that $c-r/2$ to $x$, so any $\hat{x}\in U$ can be no closer than $c-r$ to $\hat{y}$. Similarly, any $\tilde{x}$ and $\tilde{y}$ in $U,V$ respectively can be no further than $c+r$.}, so $d^{-1}\left(B_r(c)\right)$ is open and $d$ is continuous. 
%\footnotetext{WLOG, assume that $c>0$ and $r\leq c$, since metrics can only map to balls of this form. If you chose a ball not according to these restrictions, you could intersect it with $[0,\infty)$ to fix the proof.}
%\footnotetext{If this is not obvious, observe that any $\hat{y}\in V$ can be no closer that $c-r/2$ to $x$, so any $\hat{x}\in U$ can be no closer than $c-r$ to $\hat{y}$. Similarly, any $\tilde{x}$ and $\tilde{y}$ in $U,V$ respectively can be no further than $c+r$.}
%\end{proof}
\end{enumerate}

\item Give examples to show that if $B_r(x)=B_s(y)$, it need not be true that $r=s$ or $x=y$. 
\begin{proof}
Consider $(\R, d_\epsilon)$, where 
$d_\epsilon(x,y)=\min(|x-y|, 1)$. Then $B_0(100)=B_1(50)=\R$. 
\end{proof}
\end{enumerate}

\end{document}


%%% Local Variables: 
%%% mode: latex
%%% TeX-master: t
%%% End: 
