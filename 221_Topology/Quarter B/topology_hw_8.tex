\documentclass[12pt,letterpaper]{article}

\usepackage{fancyhdr,fancybox}
\usepackage{wrapfig}

%% Useful packages
\usepackage{amssymb, amsmath, amsthm} 
%\usepackage{graphicx}  %%this is currently enabled in the default document, so it is commented out here. 
\usepackage{calrsfs}
\usepackage{braket}
\usepackage{mathtools}
\usepackage{lipsum}
\usepackage{tikz}
\usetikzlibrary{cd}
\usepackage{verbatim}
%\usepackage{ntheorem}% for theorem-like environments
\usepackage{mdframed}%can make highlighted boxes of text
%Use case: https://tex.stackexchange.com/questions/46828/how-to-highlight-important-parts-with-a-gray-background
\usepackage{wrapfig}
\usepackage{centernot}
\usepackage{subcaption}%\begin{subfigure}{0.5\textwidth}
\usepackage{pgfplots}
\pgfplotsset{compat=1.13}
\usepackage[colorinlistoftodos]{todonotes}
\usepackage[colorlinks=true, allcolors=blue]{hyperref}
\usepackage{xfrac}					%to make slanted fractions \sfrac{numerator}{denominator}
\usepackage{enumitem}            
    %syntax: \begin{enumerate}[label=(\alph*)]
    %possible arguments: f \alph*, \Alph*, \arabic*, \roman* and \Roman*
\usetikzlibrary{arrows,shapes.geometric,fit}

\DeclareMathAlphabet{\pazocal}{OMS}{zplm}{m}{n}
%% Use \pazocal{letter} to typeset a letter in the other kind 
%%  of math calligraphic font. 

%% This puts the QED block at the end of each proof, the way I like it. 
\renewenvironment{proof}{{\bfseries Proof}}{\qed}
\makeatletter
\renewenvironment{proof}[1][\bfseries \proofname]{\par
  \pushQED{\qed}%
  \normalfont \topsep6\p@\@plus6\p@\relax
  \trivlist
  %\itemindent\normalparindent
  \item[\hskip\labelsep
        \scshape
    #1\@addpunct{}]\ignorespaces
}{%
  \popQED\endtrivlist\@endpefalse
}
\makeatother

%% This adds a \rewnewtheorem command, which enables me to override the settings for theorems contained in this document.
\makeatletter
\def\renewtheorem#1{%
  \expandafter\let\csname#1\endcsname\relax
  \expandafter\let\csname c@#1\endcsname\relax
  \gdef\renewtheorem@envname{#1}
  \renewtheorem@secpar
}
\def\renewtheorem@secpar{\@ifnextchar[{\renewtheorem@numberedlike}{\renewtheorem@nonumberedlike}}
\def\renewtheorem@numberedlike[#1]#2{\newtheorem{\renewtheorem@envname}[#1]{#2}}
\def\renewtheorem@nonumberedlike#1{  
\def\renewtheorem@caption{#1}
\edef\renewtheorem@nowithin{\noexpand\newtheorem{\renewtheorem@envname}{\renewtheorem@caption}}
\renewtheorem@thirdpar
}
\def\renewtheorem@thirdpar{\@ifnextchar[{\renewtheorem@within}{\renewtheorem@nowithin}}
\def\renewtheorem@within[#1]{\renewtheorem@nowithin[#1]}
\makeatother

%% This makes theorems and definitions with names show up in bold, the way I like it. 
\makeatletter
\def\th@plain{%
  \thm@notefont{}% same as heading font
  \itshape % body font
}
\def\th@definition{%
  \thm@notefont{}% same as heading font
  \normalfont % body font
}
\makeatother

%===============================================
%==============Shortcut Commands================
%===============================================
\newcommand{\ds}{\displaystyle}
\newcommand{\B}{\mathcal{B}}
\newcommand{\C}{\mathbb{C}}
\newcommand{\F}{\mathbb{F}}
\newcommand{\N}{\mathbb{N}}
\newcommand{\R}{\mathbb{R}}
\newcommand{\Q}{\mathbb{Q}}
\newcommand{\T}{\mathcal{T}}
\newcommand{\Z}{\mathbb{Z}}
\renewcommand\qedsymbol{$\blacksquare$}
\newcommand{\qedwhite}{\hfill\ensuremath{\square}}
\newcommand*\conj[1]{\overline{#1}}
\newcommand*\closure[1]{\overline{#1}}
\newcommand*\mean[1]{\overline{#1}}
%\newcommand{\inner}[1]{\left< #1 \right>}
\newcommand{\inner}[2]{\left< #1, #2 \right>}
\newcommand{\powerset}[1]{\pazocal{P}(#1)}
%% Use \pazocal{letter} to typeset a letter in the other kind 
%%  of math calligraphic font. 
\newcommand{\cardinality}[1]{\left| #1 \right|}
\newcommand{\domain}[1]{\mathcal{D}(#1)}
\newcommand{\image}{\text{Im}}
\newcommand{\inv}[1]{#1^{-1}}
\newcommand{\preimage}[2]{#1^{-1}\left(#2\right)}
\newcommand{\script}[1]{\mathcal{#1}}


\newenvironment{highlight}{\begin{mdframed}[backgroundcolor=gray!20]}{\end{mdframed}}

\DeclarePairedDelimiter\ceil{\lceil}{\rceil}
\DeclarePairedDelimiter\floor{\lfloor}{\rfloor}

%===============================================
%===============My Tikz Commands================
%===============================================
\newcommand{\drawsquiggle}[1]{\draw[shift={(#1,0)}] (.005,.05) -- (-.005,.02) -- (.005,-.02) -- (-.005,-.05);}
\newcommand{\drawpoint}[2]{\draw[*-*] (#1,0.01) node[below, shift={(0,-.2)}] {#2};}
\newcommand{\drawopoint}[2]{\draw[o-o] (#1,0.01) node[below, shift={(0,-.2)}] {#2};}
\newcommand{\drawlpoint}[2]{\draw (#1,0.02) -- (#1,-0.02) node[below] {#2};}
\newcommand{\drawlbrack}[2]{\draw (#1+.01,0.02) --(#1,0.02) -- (#1,-0.02) -- (#1+.01,-0.02) node[below, shift={(-.01,0)}] {#2};}
\newcommand{\drawrbrack}[2]{\draw (#1-.01,0.02) --(#1,0.02) -- (#1,-0.02) -- (#1-.01,-0.02) node[below, shift={(+.01,0)}] {#2};}

%***********************************************
%**************Start of Document****************
%***********************************************
 %find me at /home/trevor/texmf/tex/latex/tskpreamble_nothms.tex
%===============================================
%===============Theorem Styles==================
%===============================================

%================Default Style==================
\theoremstyle{plain}% is the default. it sets the text in italic and adds extra space above and below the \newtheorems listed below it in the input. it is recommended for theorems, corollaries, lemmas, propositions, conjectures, criteria, and (possibly; depends on the subject area) algorithms.
\newtheorem{theorem}{Theorem}
\numberwithin{theorem}{section} %This sets the numbering system for theorems to number them down to the {argument} level. I have it set to number down to the {section} level right now.
\newtheorem*{theorem*}{Theorem} %Theorem with no numbering
\newtheorem{corollary}[theorem]{Corollary}
\newtheorem*{corollary*}{Corollary}
\newtheorem{conjecture}[theorem]{Conjecture}
\newtheorem{lemma}[theorem]{Lemma}
\newtheorem*{lemma*}{Lemma}
\newtheorem{proposition}[theorem]{Proposition}
\newtheorem*{proposition*}{Proposition}
\newtheorem{problemstatement}[theorem]{Problem Statement}


%==============Definition Style=================
\theoremstyle{definition}% adds extra space above and below, but sets the text in roman. it is recommended for definitions, conditions, problems, and examples; i've alse seen it used for exercises.
\newtheorem{definition}[theorem]{Definition}
\newtheorem*{definition*}{Definition}
\newtheorem{condition}[theorem]{Condition}
\newtheorem{problem}[theorem]{Problem}
\newtheorem{example}[theorem]{Example}
\newtheorem*{example*}{Example}
\newtheorem*{counterexample*}{Counterexample}
\newtheorem*{romantheorem*}{Theorem} %Theorem with no numbering
\newtheorem{exercise}{Exercise}
\numberwithin{exercise}{section}
\newtheorem{algorithm}[theorem]{Algorithm}

%================Remark Style===================
\theoremstyle{remark}% is set in roman, with no additional space above or below. it is recommended for remarks, notes, notation, claims, summaries, acknowledgments, cases, and conclusions.
\newtheorem{remark}[theorem]{Remark}
\newtheorem*{remark*}{Remark}
\newtheorem{notation}[theorem]{Notation}
\newtheorem*{notation*}{Notation}
%\newtheorem{claim}[theorem]{Claim}  %%use this if you ever want claims to be numbered
\newtheorem*{claim}{Claim}


%%
%% Page set-up:
%%
\pagestyle{empty}
\lhead{\textsc{221B - Topology} \\ }
\rhead{\textsc{McCammond, Winter 2019} \\ Trevor Klar}
%\chead{\Large\textbf{A New Integration Technique \\ }}
\renewcommand{\headrulewidth}{0pt}
%
\renewcommand{\footrulewidth}{0pt}

\setlength{\parindent}{0in}
\setlength{\textwidth}{7in}
\setlength{\evensidemargin}{-0.25in}
\setlength{\oddsidemargin}{-0.25in}
\setlength{\parskip}{.5\baselineskip}
\setlength{\topmargin}{-0.5in}
\setlength{\textheight}{9in}

\setlist[enumerate,1]{label=\textbf{\arabic*.}}

\renewcommand{\inv}[1]{\overline{#1}}

\begin{document}
\pagestyle{fancy}
\begin{center}
{\Large Homework 8}%===============UPDATE THIS===============%
\end{center}

\begin{enumerate}

\item For a covering space $p:\tilde{X}\to X$ and a subspace $A\subset X$, let $\tilde{A}=\preimage{p}{A}$. Show that the restriction $p:\tilde{A}\to A$ is a covering space. 

\begin{proof}
Let $a\in A$. Since $a\in X$, there exists $U\ni a$ open in $X$ which is evenly covered by $p$. So $\preimage{p}{U}=\coprod_\alpha \tilde{U}_\alpha$, with each $\tilde{U}_\alpha$ open in $\tilde{X}$ and homeomorphic to $U$. Intersecting with $\tilde{A}$, we get $\tilde{V}_\alpha=\tilde{U}_\alpha\cap \tilde{A}$  with each $\tilde{V}_\alpha$ open in the subspace topology on $\tilde{A}$ and homeomorphic to $\tilde{A}$, and the collection of sets is disjoint. 
\end{proof}


\item Show that if $p:\tilde{X}\to X$ and $\rho:\tilde{Y}\to Y$ are covering spaces, so is their product $p\times \rho:\tilde{X}\times \tilde{Y}\to X\times Y$. 

\begin{proof} Note that I have renamed the spaces in the statement of the problem to simplify notation. Let $(x, y)\in X\times Y$. Using $p$ and $\rho$, there exist neighborhoods $U\ni x$, $V \ni y$ which are evenly covered by their respective spaces. Observe that the preimage 
\begin{align*}
\preimage{(p\times \rho)}{U\times V}&=\preimage{p}{U}\times \preimage{\rho}{V} \\
&=\coprod_\alpha \tilde{U}_\alpha \times \coprod_\beta \tilde{V}_\beta \\
&= \coprod_{\alpha, \beta} (\tilde{U}_\alpha \times\tilde{V}_\beta)
\end{align*}
is a collection of open rectangles. They are disjoint, since any distinct rectangles differ in either their first coordinate or their second, and $\{\tilde{U}_\alpha\}$ and $\{\tilde{V}_\beta\}$ are disjoint collections. For any particular values of $\alpha, \beta$, $(\tilde{U}_\alpha \times\tilde{V}_\beta)$ is homeomorphic to $(U\times V)$ since $\tilde{U}_\alpha\ \cong U$ and $\tilde{V}_\beta \cong V$. Thus $(U\times V)$ is an open neighborhood of $(x,y)$ which is evenly covered by $(p\times \rho)$. 
\end{proof}


\item Let $p:\tilde{X}\to X$ be a covering space with $\preimage{p}{x}$ finite and nonempty for all $x\in X$. Show that $\tilde{X}$ is compact Hausdorff iff $X$ is compact Hausdorff.

\begin{proof}(Compact $\implies$)
Suppose $\tilde{X}$ is compact, and let $\{U_\alpha\}$ be any open cover of $X$. Taking preimages we obtain $\{\tilde{U}_\alpha\}$ which is an open cover of $\tilde{X}$, since each $\tilde{U}_\alpha=\preimage{p}{U_\alpha}$ is a disjoint union of open sets in $\tilde{X}$ and is thus itself an open set. Since $\tilde{X}$ is compact, there exists a finite subcover $\{\tilde{U}_i\}$. Since $\preimage{p}{x}$ is nonempty for all $x\in X$, then $p$ is onto, so taking images in our cover yields $\{U_i\}$ which covers $X$. Since each $U_i$ is the image of the preimage of an open set in $\{U_\alpha\}$ that has been reindexed, then each $U_i$ is open. Thus $\{U_i\}$ is a finite subcover of $\{U_\alpha\}$, so $X$ is compact. \qedwhitehere
\end{proof}

\begin{proof}(Compact $\impliedby$)
Suppose $X$ is compact Hausdorff, and let $\{\tilde{U}_\alpha\}_{\alpha\in\Gamma}$ be any open cover of $\tilde{X}$. We need evenly-coveredness, so for each $x\in X$, let $V_x$ be a neighborhood of $x$ which is evenly covered. Since $X$ is compact, we can take a finite subcover $\{V_i\}$ of $\{V_x\}$. Since $p$ is finite-sheeted, then for each $i$ then $\preimage{p}{V_i}=\coprod\limits_{j} \tilde{V}_{i,j}$. %Since $\tilde{V}_{i,j}\cong V_i$ for each $j$, then $\tilde{V}_{i,j}$ is compact. This gives us a finite 

Since there are finitely many $\tilde{V}_{i,j}$, if we can show that each a $\tilde{V}_{i,j}$ is covered by a finite subcollection of $\{\tilde{U}_\alpha\}$, then we are done.

Let $\Delta=\{\alpha\in \Gamma \mid \tilde{U}_\alpha\cap\tilde{V}_{i,j}\neq\emptyset\}$, and observe that $\{\tilde{U}_{\delta}\}_{\delta\in\Delta}$ covers the closure $\text{cl}(\tilde{V}_{i,j})$.\footnote{To see this, apply the definition of boundary points and see that for any $b\in\del(\tilde{V}_{i,j})$ any open $\tilde{U}_{\delta}\ni b$ intersects $\tilde{V}_{i,j}$, so $\delta\in \Delta$.} This means that taking images in $p$ yields an open\footnote{Each $U_\delta$ is open because every covering space is an open map, a fact which I choose not to prove here since this problem is insanely long already.} cover $\{U_\delta\}$ of $\closure{V_i}=\text{cl}(p(\tilde{V}_i))$. 
%Now the boundary $\del(V_i)$ is closed and $X$ is compact, so there exists a finite subcover $\{U_i^1\}$ of $\{U_\delta\}$ which covers $\del(V_i)$. 
%
%Next, since $\bigcup_i U_i^1$ is open, $(\bigcup_i U_i^1)^\complement$ is closed. Since $\del V_i \subset \bigcup_i U_i^1$, then $\del V_i$ misses $(\bigcup_i U_i^1)^\complement$. This means that
%\begin{align*}
%%\text{int}(V_i) & \subset 
%V_i \cap \left(\bigcup_i U_i^1\right)^\complement %\\
%&= (V_i\cup \del V_i) \cap \left(\bigcup_i U_i^1\right)^\complement \\
%&= \closure{V_i} \cap \left(\bigcup_i U_i^1\right)^\complement 
%\end{align*}
%which is closed in $X$, so it also has a finite subcover $\{U_i^2\}$. 
Since $\closure{V_i}$ is closed and $X$ is compact, then there exists a finite subcover $\{U_i\}$ of $\{U_\delta\}$ which covers $\closure{V_i}$. Thus the corresponding sets $\{\tilde U_i\}$ cover $\tilde{V}_{i,j}$, and we are done. \qedwhitehere
\end{proof}

\begin{proof}(Hausdorff $\impliedby$) Suppose $X$ is Hausdorff, and let $\tilde{x}\neq \tilde{y} \in \tilde{X}$. Consider $p(\tilde{x})=x$ and $p(\tilde{y})=y$. Since $X$ is Hausdorff, there exist disjoint sets $U\ni x, V\ni y$ which are open in $X$. Then the corresponding preimages $\preimage{p}{U}\ni\tilde{x}$ and $\preimage{p}{V}\ni \tilde{y}$ are disjoint, and they are open because $p$ is a covering space, so $\tilde{X}$ is Hausdorff. \qedwhitehere
\end{proof}

%\textsc{Lemma.} Every covering space is an open map. 
%
%\textsc{Proof of Lemma.} 
%\qedwhite

\begin{proof}(Hausdorff $\implies$) Suppose $\tilde{X}$ is Hausdorff. Let $x\neq y\in X$, and denote $\preimage{p}{x}=\{\tilde{x}_i\}_{i=1}^N$ and $\preimage{p}{y}=\{\tilde{y}_i\}_{i=1}^N$. Since we can separate any two points in $\tilde{X}$ using the Hausdorff property, we can do it with finitely many points. So let 
$$\{\tilde{U}_i', \tilde{V}_i'\}_{i=1}^N$$ 
be a collection of sets open in $\tilde{X}$ such that $\tilde{U}_i'\ni \tilde{x}_i$ and $\tilde{V}_i'\ni \tilde{y}_i$ for all $i=1\dots N$ and every pair of sets in the collection is disjoint. Next let $E$ be an evenly covered neighborhood of $x$, and $F$ an evenly covered neighborhood of $y$. Then 
$$\{\tilde{E}_i, \tilde{F}_i\}_{i=1}^N$$ 
is a collection of sets open in $\tilde{X}$ such that $\tilde{E}_i\ni \tilde{x}_i$ and $\tilde{F}_i\ni \tilde{y}_i$ for all $i=1\dots N$ and every $\tilde{E}_i, \tilde{F}_i$ is homeomorphic to $E,F$, respectively. Taking $\tilde{U}_i=\tilde{U}_i'\cap \tilde{E}_i$ and $\tilde{V}_i=\tilde{V}_i'\cap \tilde{F}_i$, we obtain 
$$\{\tilde{U}_i, \tilde{V}_i\}_{i=1}^N$$ 
which are open, disjoint, contain $\tilde{x}_i, \tilde{y}_i$ as appropriate, and are homeomorphic to $U_i,V_i$ as appropriate, where we denote 
\begin{align*}
U_i&=p(\tilde{U}_i)\cap E\\
V_i&=p(\tilde{V}_i)\cap F.\\
\end{align*}
To finish the proof, we let
\begin{align*}
U&=\bigcap_{i=1}^N U_i \\ 
V&=\bigcap_{i=1}^N V_i
\end{align*}
and observe that $x\in U$, $y\in V$ since every $\tilde{U}_i, \tilde{V}_i$ contains a point which maps to $x,y$ respectively. To see that $U,V$ are disjoint, suppose $d\in U\cap V$. Observe that for all $i$, $d\in U \subset U_i \subset p(\tilde{U}_i)$ so the preimage 
$$p|^{-1}(d)
%=\{\tilde{d}_i\}_{i=1}^N
\subset \bigcup_{i=1}^N \tilde{U}_i,$$
and similarly $d\in V \subset V_i \subset p(\tilde{V}_i)$ so 
$$p|^{-1}(d)
%=\{\tilde{d}_i\}_{i=1}^N
\subset \bigcup_{i=1}^N \tilde{V}_i,$$
but $\bigcup_{i=1}^N\tilde{U}_i$ and $\bigcup_{i=1}^N\tilde{V}_i$ are disjoint.
\end{proof} 


\item Construct a a simply-connected covering space of the space $X\subset\R^3$ that is the union of a sphere and a diameter. Do the same when $X$ is the union of a sphere and a circle intersecting it in two points.





\pagebreak


\setcounter{enumi}{7}
\item Let $\tilde{X}$ and $\tilde{Y}$ be simply-connected covering spaces of the path-connected, locally path-connected spaces $X$ and $Y$. Show that if $X \simeq Y$ then $\tilde{X} \simeq \tilde{Y}$. [Exercise 11 in Chapter 0 may be helpful.]


\item Show that if a path-connected, locally path-connected space $X$ has $\pi_1(X)$ finite, then every map $X\to S^1$ is nullhomotopic.  [Use the covering space $R\to S^1$.]


\item Find all the connected 2-sheeted and 3-sheeted covering spaces of $S^1\wedgeprod S^1$, up to isomorphism of covering spaces without basepoints.


\end{enumerate}
\vfill

Collaborators:
\begin{enumerate}
\item[] None for this homework.
\end{enumerate}
\end{document}



