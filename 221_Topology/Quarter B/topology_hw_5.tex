\documentclass[12pt,letterpaper]{article}

\usepackage{fancyhdr,fancybox}
\usepackage{wrapfig}

%% Useful packages
\usepackage{amssymb, amsmath, amsthm} 
%\usepackage{graphicx}  %%this is currently enabled in the default document, so it is commented out here. 
\usepackage{calrsfs}
\usepackage{braket}
\usepackage{mathtools}
\usepackage{lipsum}
\usepackage{tikz}
\usetikzlibrary{cd}
\usepackage{verbatim}
%\usepackage{ntheorem}% for theorem-like environments
\usepackage{mdframed}%can make highlighted boxes of text
%Use case: https://tex.stackexchange.com/questions/46828/how-to-highlight-important-parts-with-a-gray-background
\usepackage{wrapfig}
\usepackage{centernot}
\usepackage{subcaption}%\begin{subfigure}{0.5\textwidth}
\usepackage{pgfplots}
\pgfplotsset{compat=1.13}
\usepackage[colorinlistoftodos]{todonotes}
\usepackage[colorlinks=true, allcolors=blue]{hyperref}
\usepackage{xfrac}					%to make slanted fractions \sfrac{numerator}{denominator}
\usepackage{enumitem}            
    %syntax: \begin{enumerate}[label=(\alph*)]
    %possible arguments: f \alph*, \Alph*, \arabic*, \roman* and \Roman*
\usetikzlibrary{arrows,shapes.geometric,fit}

\DeclareMathAlphabet{\pazocal}{OMS}{zplm}{m}{n}
%% Use \pazocal{letter} to typeset a letter in the other kind 
%%  of math calligraphic font. 

%% This puts the QED block at the end of each proof, the way I like it. 
\renewenvironment{proof}{{\bfseries Proof}}{\qed}
\makeatletter
\renewenvironment{proof}[1][\bfseries \proofname]{\par
  \pushQED{\qed}%
  \normalfont \topsep6\p@\@plus6\p@\relax
  \trivlist
  %\itemindent\normalparindent
  \item[\hskip\labelsep
        \scshape
    #1\@addpunct{}]\ignorespaces
}{%
  \popQED\endtrivlist\@endpefalse
}
\makeatother

%% This adds a \rewnewtheorem command, which enables me to override the settings for theorems contained in this document.
\makeatletter
\def\renewtheorem#1{%
  \expandafter\let\csname#1\endcsname\relax
  \expandafter\let\csname c@#1\endcsname\relax
  \gdef\renewtheorem@envname{#1}
  \renewtheorem@secpar
}
\def\renewtheorem@secpar{\@ifnextchar[{\renewtheorem@numberedlike}{\renewtheorem@nonumberedlike}}
\def\renewtheorem@numberedlike[#1]#2{\newtheorem{\renewtheorem@envname}[#1]{#2}}
\def\renewtheorem@nonumberedlike#1{  
\def\renewtheorem@caption{#1}
\edef\renewtheorem@nowithin{\noexpand\newtheorem{\renewtheorem@envname}{\renewtheorem@caption}}
\renewtheorem@thirdpar
}
\def\renewtheorem@thirdpar{\@ifnextchar[{\renewtheorem@within}{\renewtheorem@nowithin}}
\def\renewtheorem@within[#1]{\renewtheorem@nowithin[#1]}
\makeatother

%% This makes theorems and definitions with names show up in bold, the way I like it. 
\makeatletter
\def\th@plain{%
  \thm@notefont{}% same as heading font
  \itshape % body font
}
\def\th@definition{%
  \thm@notefont{}% same as heading font
  \normalfont % body font
}
\makeatother

%===============================================
%==============Shortcut Commands================
%===============================================
\newcommand{\ds}{\displaystyle}
\newcommand{\B}{\mathcal{B}}
\newcommand{\C}{\mathbb{C}}
\newcommand{\F}{\mathbb{F}}
\newcommand{\N}{\mathbb{N}}
\newcommand{\R}{\mathbb{R}}
\newcommand{\Q}{\mathbb{Q}}
\newcommand{\T}{\mathcal{T}}
\newcommand{\Z}{\mathbb{Z}}
\renewcommand\qedsymbol{$\blacksquare$}
\newcommand{\qedwhite}{\hfill\ensuremath{\square}}
\newcommand*\conj[1]{\overline{#1}}
\newcommand*\closure[1]{\overline{#1}}
\newcommand*\mean[1]{\overline{#1}}
%\newcommand{\inner}[1]{\left< #1 \right>}
\newcommand{\inner}[2]{\left< #1, #2 \right>}
\newcommand{\powerset}[1]{\pazocal{P}(#1)}
%% Use \pazocal{letter} to typeset a letter in the other kind 
%%  of math calligraphic font. 
\newcommand{\cardinality}[1]{\left| #1 \right|}
\newcommand{\domain}[1]{\mathcal{D}(#1)}
\newcommand{\image}{\text{Im}}
\newcommand{\inv}[1]{#1^{-1}}
\newcommand{\preimage}[2]{#1^{-1}\left(#2\right)}
\newcommand{\script}[1]{\mathcal{#1}}


\newenvironment{highlight}{\begin{mdframed}[backgroundcolor=gray!20]}{\end{mdframed}}

\DeclarePairedDelimiter\ceil{\lceil}{\rceil}
\DeclarePairedDelimiter\floor{\lfloor}{\rfloor}

%===============================================
%===============My Tikz Commands================
%===============================================
\newcommand{\drawsquiggle}[1]{\draw[shift={(#1,0)}] (.005,.05) -- (-.005,.02) -- (.005,-.02) -- (-.005,-.05);}
\newcommand{\drawpoint}[2]{\draw[*-*] (#1,0.01) node[below, shift={(0,-.2)}] {#2};}
\newcommand{\drawopoint}[2]{\draw[o-o] (#1,0.01) node[below, shift={(0,-.2)}] {#2};}
\newcommand{\drawlpoint}[2]{\draw (#1,0.02) -- (#1,-0.02) node[below] {#2};}
\newcommand{\drawlbrack}[2]{\draw (#1+.01,0.02) --(#1,0.02) -- (#1,-0.02) -- (#1+.01,-0.02) node[below, shift={(-.01,0)}] {#2};}
\newcommand{\drawrbrack}[2]{\draw (#1-.01,0.02) --(#1,0.02) -- (#1,-0.02) -- (#1-.01,-0.02) node[below, shift={(+.01,0)}] {#2};}

%***********************************************
%**************Start of Document****************
%***********************************************
 %find me at /home/trevor/texmf/tex/latex/tskpreamble_nothms.tex
%===============================================
%===============Theorem Styles==================
%===============================================

%================Default Style==================
\theoremstyle{plain}% is the default. it sets the text in italic and adds extra space above and below the \newtheorems listed below it in the input. it is recommended for theorems, corollaries, lemmas, propositions, conjectures, criteria, and (possibly; depends on the subject area) algorithms.
\newtheorem{theorem}{Theorem}
\numberwithin{theorem}{section} %This sets the numbering system for theorems to number them down to the {argument} level. I have it set to number down to the {section} level right now.
\newtheorem*{theorem*}{Theorem} %Theorem with no numbering
\newtheorem{corollary}[theorem]{Corollary}
\newtheorem*{corollary*}{Corollary}
\newtheorem{conjecture}[theorem]{Conjecture}
\newtheorem{lemma}[theorem]{Lemma}
\newtheorem*{lemma*}{Lemma}
\newtheorem{proposition}[theorem]{Proposition}
\newtheorem*{proposition*}{Proposition}
\newtheorem{problemstatement}[theorem]{Problem Statement}


%==============Definition Style=================
\theoremstyle{definition}% adds extra space above and below, but sets the text in roman. it is recommended for definitions, conditions, problems, and examples; i've alse seen it used for exercises.
\newtheorem{definition}[theorem]{Definition}
\newtheorem*{definition*}{Definition}
\newtheorem{condition}[theorem]{Condition}
\newtheorem{problem}[theorem]{Problem}
\newtheorem{example}[theorem]{Example}
\newtheorem*{example*}{Example}
\newtheorem*{counterexample*}{Counterexample}
\newtheorem*{romantheorem*}{Theorem} %Theorem with no numbering
\newtheorem{exercise}{Exercise}
\numberwithin{exercise}{section}
\newtheorem{algorithm}[theorem]{Algorithm}

%================Remark Style===================
\theoremstyle{remark}% is set in roman, with no additional space above or below. it is recommended for remarks, notes, notation, claims, summaries, acknowledgments, cases, and conclusions.
\newtheorem{remark}[theorem]{Remark}
\newtheorem*{remark*}{Remark}
\newtheorem{notation}[theorem]{Notation}
\newtheorem*{notation*}{Notation}
%\newtheorem{claim}[theorem]{Claim}  %%use this if you ever want claims to be numbered
\newtheorem*{claim}{Claim}


%%
%% Page set-up:
%%
\pagestyle{empty}
\lhead{\textsc{221 - Topology} \\ }
\rhead{\textsc{McCammond, Winter 2019} \\ Trevor Klar}
%\chead{\Large\textbf{A New Integration Technique \\ }}
\renewcommand{\headrulewidth}{0pt}
%
\renewcommand{\footrulewidth}{0pt}

\setlength{\parindent}{0in}
\setlength{\textwidth}{7in}
\setlength{\evensidemargin}{-0.25in}
\setlength{\oddsidemargin}{-0.25in}
\setlength{\parskip}{.5\baselineskip}
\setlength{\topmargin}{-0.5in}
\setlength{\textheight}{9in}

\setlist[enumerate,1]{label=\textbf{\arabic*.}}

\begin{document}
\pagestyle{fancy}
\begin{center}
{\Large Homework 5}%===============UPDATE THIS===============%
\end{center}

\begin{enumerate}
\item Show that composition of paths satisfies the following cancellation property: If $f_0\sqcdot g_0\simeq f_1\sqcdot g_1$ and $g_{0}\simeq g_1$, then $f_0\simeq f_1$. 
\begin{definition*}(Concatenation of Path Homotopies $F\sqcdot G$)
Given homotopic paths $f_0\overtext{\simeq}{$F$} f_1$ and $g_0\overtext{\simeq}{$G$} g_1$ such that $f_s\sqcdot g_s$ is defined, then 
$$F\sqcdot G:=\begin{cases}
F(2s,t) & s\in [0,\frac{1}{2}] \\
G(2s-1,t) & s\in [\frac{1}{2},1] \\
\end{cases}$$
or in words, apply $F$ in the first region, and $G$ in the second. 
\end{definition*}
\begin{proof}
Since $f_0\sqcdot g_0\simeq f_1\sqcdot g_1$, then call the path homotopy relating them $\Phi$, and the homotopy relating $g_0$ and $g_1$ $G$. We can (making a minor abuse of notation) consider $G$ to be a homotopy between the inverses $\bar{g}_0$ and $\bar{g}_1$ as well. Using the waiting homotopy we have discussed in class, we know that $f\simeq f\sqcdot (g\sqcdot\bar{g})$ whenever the concatenation is defined. Then 
\begin{align*}
f_0 &\simeq f_0\sqcdot (g_0 \sqcdot \bar{g}_0) &\text{by the waiting homotopy}\\
&\simeq (f_0\sqcdot g_0) \sqcdot \bar{g}_0 &\text{reparametrizing}\\
&\simeq (f_1\sqcdot g_1) \sqcdot \bar{g}_1 &\text{by $\Phi \sqcdot G$}  \\
&\simeq f_1\sqcdot (g_1 \sqcdot \bar{g}_1) &\text{reparametrizing}\\
&\simeq f_1 &\text{by the waiting homotopy}
\end{align*}
and we're done.
\end{proof}

\item Show that the change-of-basepoint homomorphism $\beta_h$ depends only on the homotopy class of $h$. 
\begin{proof}
Suppose $h_0\overtext{\simeq}{$H$} h_1$. Then for any loop $f$, 
\begin{align*}
h_0\sqcdot(f\sqcdot\bar{h}_0) &\simeq h_1\sqcdot(f\sqcdot\bar{h}_0) &\text{by } H\sqcdot\id \\
&\simeq (h_1\sqcdot f) \sqcdot\bar{h}_0 &\text{reparametrizing}\\
&\simeq (h_1\sqcdot f) \sqcdot\bar{h}_1 &\text{by } \id\sqcdot H	
\end{align*}
Thus $\beta_{h_0}[f]=[h_0\sqcdot f\sqcdot\bar{h}_0]=[h_1\sqcdot f\sqcdot\bar{h}_1]=\beta_{h_1}[f]$ and we're done.
\end{proof}

\vfill
\pagebreak
\item For a path-connected space $X$, show that $\pi_1(X)$ is abelian if and only if all basepoint-change homomorphisms $\beta_h$ depend only on the endpoints of the path $h$. 
\begin{proof}
For $\pi_1(X)$ to be abelian means that for all loops $f,g$ with the same basepoint, $f\sqcdot g\simeq g\sqcdot f$. For the change-of-basepoint homomorphism $\beta_h$ to depend only on the endpoints of the path $h$ means that if $h, h'$ have the same endpoints, then $h \sqcdot f \sqcdot \bar{h} \simeq h' \sqcdot f \sqcdot \bar{h'}$. We will show that these are equivalent. 

$(\implies)$ Suppose $\pi_1(X)$ is abelian, let $f$ be a loop, and let $h, h'$ have the same endpoints. Then 
\begin{align*}
h\sqcdot f \sqcdot \bar{h} & \simeq (h'\sqcdot\bar{h}') \sqcdot h\sqcdot f \sqcdot \bar{h} \sqcdot (h'\sqcdot\bar{h}') & \text{by the waiting homotopy}\\
& \simeq h'\sqcdot (\bar{h}' \sqcdot h)\sqcdot f \sqcdot (\bar{h} \sqcdot h')\sqcdot\bar{h}' & \text{reparametrizing}\\
& \simeq h'\sqcdot f \sqcdot (\bar{h}' \sqcdot h)\sqcdot (\bar{h} \sqcdot h')\sqcdot\bar{h}' & \pi_1(X) \text{ is abelian}\\
&\simeq h'\sqcdot f \sqcdot (\bar{h}'\sqcdot (h\sqcdot\bar{h}) \sqcdot h') \sqcdot h' & \text{reparametrizing}\\
&\simeq h'\sqcdot f \sqcdot (\bar{h}'\sqcdot h') \sqcdot h' & \text{by the waiting homotopy}\\
&\simeq h'\sqcdot f \sqcdot h' & \text{by the waiting homotopy}
\end{align*}

$(\impliedby)$ Let $f,g$ be loops with the same basepoint, and suppose $\beta_h$ depends only on the endpoints of $h$. Then 
\begin{align*}
f\sqcdot g&\simeq	f\sqcdot g\sqcdot\bar{f}\sqcdot f & \text{by the waiting homotopy}\\
&\simeq g\sqcdot g\sqcdot \bar{g}\sqcdot f & f,g \text{ have the same endpoints}\\
&\simeq g\sqcdot f& \text{by the waiting homotopy} 
\end{align*}\mbox{}\vspace*{-5ex}\qedhere
\end{proof}
\setcounter{enumi}{4}
\item Show that for a space $X$, the following conditions are equivalent: 
\begin{enumerate}[label=(\alph*)]
\item Every map $S^1\to X$ is homotopic to a constant map.
\item Evey map  $S^1\to X$ extends to a map $D^2\to X$. 
\item $\pi_1(X)=0$ for all $x_0\in X$. 
\end{enumerate}
\begin{proof}$(a\implies b)$
Let $f:S^1\to X$ be a loop, and let $F:S^1\times I\to X$ be a homotopy with $f_1=f$ and $f_0\equiv x_0$ for some $x_0\in X$. For each point $(\theta, r)\in D^2$, we can consider $\theta$ to be in $S^1$ and $r$ to be in $I$. Then $F$ is a map $D^2\to X$, and it is well defined since if $r=0$ then $(\theta_1,r)\sim(\theta_2,r)$, and $F$ is constant when $r=0$. 
\end{proof}
\begin{proof}$(b\implies c)$
Let $x_0\in X$, and let $f$ be a loop with basepoint $x_0$. Then extend $f$ to $F:D^2\to X$. We know that $D^2$ deformation retracts to any point in $D^2$, so let $G:D^2\times I \to D^2$  be a deformation retraction to $\preimage{f_0}{x_0}$. If we restrict $G$ to $S^1\times I$, then $G$ is a homotopy of loops, and 
$$F(G|_{S^1\times I}(\theta, t)) \text{ is a path homotopy from $f$ to $x_0$,}$$
since $g_0=\id$ and $g_1\equiv \preimage{F}{x_0}$. 
\end{proof}
\begin{proof}$(c\implies a)$ By definition of trivial fundamental group, any two loops in $X$ are homotopic, including any loop based at $x_0$ and the constant loop $x_0$. 
\end{proof}

\item We can regard $\pi_1(X,x_0)$ as the set of basepoint-preserving homotopy classes of maps $(S^1,s_0)\to(X,x_0)$. Let $[S^1,X]$ be the set of homotopy classes of maps $S^1\to X$ with no conditions on basepoints. Thus there is a natural map $\Phi:\pi_1(X,x_0)\to [S^1,X]$ obtained by ignoring basepoints. 

Show that 
\begin{enumerate}
\item $\Phi$ is onto if $X$ is path-connected, and 
\item $\Phi([f])=\Phi([g])$ iff $[f]$ and $[g]$ are conjugate in $\pi_1(X,x_0)$.
\item Hence $\Phi$ induces a one-to-one correspondence between $[S^1,X]$ and the set of conjugacy classes in $\pi_1(X)$, when $X$ is path-connected. 
\end{enumerate}
\begin{proof}(i)
Let $x_0$ be the basepoint of $X$, and let $g$ be a loop with arbitrary start and end point $x_1$. Since $X$ is path-connected, there exists a path $p$ from $x_0$ to $x_1$. Now observe that $p\sqcdot g \sqcdot \bar{p}$ is a loop with basepoint $x_0$, and $\Phi([p\sqcdot g \sqcdot \bar{p}])=\Phi([g])$ since we can produce a homotopy of maps (not of loops) that shows $p\sqcdot g\sqcdot \bar{p} \simeq g\sqcdot \bar{p} \sqcdot p$ as follows: 
\begin{align*}
h_0(s)&= p\sqcdot g\sqcdot \bar{p}, &\text{parametrized in 3 equal parts} \\
h_t(s)&=h_0\left(s+\tfrac{t}{3}\right) &\text{where we identify $s\sim (s-1)$ if $s>1$.}
\end{align*}
and of course $g\sqcdot \bar{p} \sqcdot p \simeq g$ by the waiting homotopy. 
\end{proof}
\begin{proof}(ii)$(\impliedby)$ Since $[f]$ and $[g]$ are conjugate in $\pi_1(X,x_0)$, then there exists a loop $p$ such that $f\simeq p\sqcdot g\sqcdot \bar{p}$, and $p\sqcdot g\sqcdot \bar{p} \simeq g$ by the same reasoning as in (i). 
\end{proof}
\begin{proof}(ii)$(\implies)$
Since $\Phi([f])=\Phi([g])$, then 
$$f(s)\overtext{\simeq}{$H(t,s)$}g(s)$$ 
where $H$ is a homotopy of maps. Thus the common basepoint of $f$ and $g$ (denoted $x_0$), is not fixed over time $t$. Define a path $p$ by $p(t)=H(t,0)$, the image of the basepoint in the homotopy (we will hereafter write $p$ as a function of $s$). Since $f$ and $g$ have the same basepoint, $p$ is a loop with endpoint $x_0$. Define a homotopy of maps $p_t(s):=p(ts)$ so that 
\begin{align*}
p_0&\equiv x_0 \\
p_1&=p \\
p_t(1)&=H(t,0),
\end{align*}
and also define a homotopy $\bar{p}_t$ that gives the inverse path of $p_t$ for each time $t$. Now observe that for all $t$, $p_t(0)=x_0, p_t(1)=H(t,0), H(t,1)=\bar{p}_t(0), \bar{p}_t(1)=x_0$, so we can concatenate 
$$P\sqcdot H \sqcdot \bar{P}$$
to obtain an actual homotopy of loops showing $f\simeq p\sqcdot g\sqcdot \bar{p}$. To see that we have done this, observe that for all time the endpoints are fixed at $x_0$, at $t=0$ we have $\{x_0\}\sqcdot f \sqcdot \{x_0\} \simeq f$, and at $t=1$ we have $p\sqcdot g \sqcdot \bar{p}$.
\end{proof}
\begin{proof}(iii)
Consider the map induced by $\Phi$, 
$$\tilde{\Phi}:\{\text{conjugacy classes of }\pi_1(X)\}\to [S^1,X].$$
\begin{itemize}
\item Since $\Phi$ is onto, then $\tilde{\Phi}$ is onto.
\item Since $\Phi([f])=\Phi([g])$ implies $[f]$ and $[g]$ are conjugate, then $\tilde{\Phi}$ is one-to-one.
\item Since $[f]$ and $[g]$ are conjugate implies $\Phi([f])=\Phi([g])$, then $\tilde{\Phi}$ is well-defined. \qedhere
\end{itemize}
\end{proof}

\setcounter{enumi}{10}
\item If $X_0$ is the path-component of a space $X$ containing the basepoint $x_0$, show that the inclusion $X_0\hookrightarrow X$ induces an isomorphism $\pi_1(X_0,x_0)\to \pi_1(X,x_0)$. 
\begin{proof}\mbox{}
\begin{itemize}
\item Any loop in $X$ with basepoint $x_0$ must be in the path-component of $x_0$, since the loop itself connects every point in it to $x_0$. Thus the induced map $\pi_1(X_0,x_0)\to \pi_1(X,x_0)$ is onto. 
\item It is obviously true that $f\simeq g$ iff $f\simeq g$, so the induced map $\pi_1(X_0,x_0)\to \pi_1(X,x_0)$ is one-to-one and well-defined. \qedhere
\end{itemize}
\end{proof}

\setcounter{enumi}{12}
\item Given a space $X$ and a path-connected subspace $A$ containing the basepoint $x_0$, show that the map $\pi_1(A,x_0)\to \pi_1(X,x_0)$ induced by the inclusion $A\hookrightarrow X$ is surjective iff every path in $X$ with endpoints in $A$ is homotopic to a path in $A$. 
\begin{proof}$(\impliedby)$
Every loop $f$ in $X$ with basepoint $x_0$ is a path with basepoints in $A$, so it is homotopic to a loop $a$ in $A$. Thus $[a]=[f]$, and the map is surjective. 
\end{proof}
\begin{proof}$(\implies)$
 Suppose the map is onto. Let $f_0$ be a path with endpoints $a_0,a_1\in A$. Since $A$ is path-connected, let $g_0, g_1$ be paths $x_0\to a_0$ and $a_1\to x_0$ respectively. Then $[g_0\sqcdot f_0 \sqcdot g_1]\in \pi_1(X,x_0)$, so there exists 
 $g_0\sqcdot f_0 \sqcdot g_1 \,\overtext{\simeq}{$H(s,t)$}\,\, \tilde{f}$
 such that $[\tilde{f}]\in \pi_1(A,x_0)$. Here we are parameterizing $g_0\sqcdot f_0 \sqcdot g_1$ in three equal parts, so $H(\frac{1}{3},0)=a_0$ and $H(\frac{2}{3},0)=a_1$.  Call $a_2=H(\frac{1}{3},1)$ and $a_3=H(\frac{2}{3},1)$, and note that $H(\frac{1}{3},t)$ and $H(\frac{2}{3},t)$ are paths connecting $a_0$ to $a_2$ and $a_1$ to $a_3$ respectively. 
 \jpg{width=0.6\textwidth}{hw5-p13-1}
 To finish up, we concatenate three homotopies of maps to form a homotopy of paths, similar to how we did in Problem 6(ii). 
\begin{align*}
 \text{Let }P^0(s,t)&=H\left(\tfrac{1}{3},st\right), \\
 \text{let }P^1(s,t)&=H\left(\tfrac{2}{3},st\right),\text{ and }\\
 \text{let }\tilde{H}(s,t)&=H\left|_{\left[\frac{1}{3},\frac{2}{3}\right]\times I}\right., \text{ rescaled so that }\tilde{H}:I\times I\to X. 
\end{align*}
Then for all $t$,  
\begin{itemize}
\item $P^0(0,t)=a_0,$
\item $P^0(1,t)=H\left(\tfrac{1}{3},t\right)=\tilde{H}(0,t) $
\item $P^1(1,t)=H\left(\tfrac{2}{3},t\right)=\tilde{H}(1,t) $
\item $P^1(0,t)=a_1,$
\end{itemize}
so we can concatenate $P^0\sqcdot \tilde{H} \sqcdot \bar{P}^1$ to produce a path homotopy between $f_0$ and $p^0_1\sqcdot \tilde{h}_1 \sqcdot \bar{p}^1_1$, the latter which path is completely in $A$. 
\end{proof}



\end{enumerate}
\vfill

Collaborators:
\begin{enumerate}
\item Zach Wagner

\item 

\item 

\setcounter{enumi}{4}
\item Zach Wagner

\item Sandy Schoettler

\setcounter{enumi}{10}
\item 

\setcounter{enumi}{12}
\item
\end{enumerate}
\end{document}



