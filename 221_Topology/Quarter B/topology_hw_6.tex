\documentclass[12pt,letterpaper]{article}

\usepackage{fancyhdr,fancybox}
\usepackage{wrapfig}

%% Useful packages
\usepackage{amssymb, amsmath, amsthm} 
%\usepackage{graphicx}  %%this is currently enabled in the default document, so it is commented out here. 
\usepackage{calrsfs}
\usepackage{braket}
\usepackage{mathtools}
\usepackage{lipsum}
\usepackage{tikz}
\usetikzlibrary{cd}
\usepackage{verbatim}
%\usepackage{ntheorem}% for theorem-like environments
\usepackage{mdframed}%can make highlighted boxes of text
%Use case: https://tex.stackexchange.com/questions/46828/how-to-highlight-important-parts-with-a-gray-background
\usepackage{wrapfig}
\usepackage{centernot}
\usepackage{subcaption}%\begin{subfigure}{0.5\textwidth}
\usepackage{pgfplots}
\pgfplotsset{compat=1.13}
\usepackage[colorinlistoftodos]{todonotes}
\usepackage[colorlinks=true, allcolors=blue]{hyperref}
\usepackage{xfrac}					%to make slanted fractions \sfrac{numerator}{denominator}
\usepackage{enumitem}            
    %syntax: \begin{enumerate}[label=(\alph*)]
    %possible arguments: f \alph*, \Alph*, \arabic*, \roman* and \Roman*
\usetikzlibrary{arrows,shapes.geometric,fit}

\DeclareMathAlphabet{\pazocal}{OMS}{zplm}{m}{n}
%% Use \pazocal{letter} to typeset a letter in the other kind 
%%  of math calligraphic font. 

%% This puts the QED block at the end of each proof, the way I like it. 
\renewenvironment{proof}{{\bfseries Proof}}{\qed}
\makeatletter
\renewenvironment{proof}[1][\bfseries \proofname]{\par
  \pushQED{\qed}%
  \normalfont \topsep6\p@\@plus6\p@\relax
  \trivlist
  %\itemindent\normalparindent
  \item[\hskip\labelsep
        \scshape
    #1\@addpunct{}]\ignorespaces
}{%
  \popQED\endtrivlist\@endpefalse
}
\makeatother

%% This adds a \rewnewtheorem command, which enables me to override the settings for theorems contained in this document.
\makeatletter
\def\renewtheorem#1{%
  \expandafter\let\csname#1\endcsname\relax
  \expandafter\let\csname c@#1\endcsname\relax
  \gdef\renewtheorem@envname{#1}
  \renewtheorem@secpar
}
\def\renewtheorem@secpar{\@ifnextchar[{\renewtheorem@numberedlike}{\renewtheorem@nonumberedlike}}
\def\renewtheorem@numberedlike[#1]#2{\newtheorem{\renewtheorem@envname}[#1]{#2}}
\def\renewtheorem@nonumberedlike#1{  
\def\renewtheorem@caption{#1}
\edef\renewtheorem@nowithin{\noexpand\newtheorem{\renewtheorem@envname}{\renewtheorem@caption}}
\renewtheorem@thirdpar
}
\def\renewtheorem@thirdpar{\@ifnextchar[{\renewtheorem@within}{\renewtheorem@nowithin}}
\def\renewtheorem@within[#1]{\renewtheorem@nowithin[#1]}
\makeatother

%% This makes theorems and definitions with names show up in bold, the way I like it. 
\makeatletter
\def\th@plain{%
  \thm@notefont{}% same as heading font
  \itshape % body font
}
\def\th@definition{%
  \thm@notefont{}% same as heading font
  \normalfont % body font
}
\makeatother

%===============================================
%==============Shortcut Commands================
%===============================================
\newcommand{\ds}{\displaystyle}
\newcommand{\B}{\mathcal{B}}
\newcommand{\C}{\mathbb{C}}
\newcommand{\F}{\mathbb{F}}
\newcommand{\N}{\mathbb{N}}
\newcommand{\R}{\mathbb{R}}
\newcommand{\Q}{\mathbb{Q}}
\newcommand{\T}{\mathcal{T}}
\newcommand{\Z}{\mathbb{Z}}
\renewcommand\qedsymbol{$\blacksquare$}
\newcommand{\qedwhite}{\hfill\ensuremath{\square}}
\newcommand*\conj[1]{\overline{#1}}
\newcommand*\closure[1]{\overline{#1}}
\newcommand*\mean[1]{\overline{#1}}
%\newcommand{\inner}[1]{\left< #1 \right>}
\newcommand{\inner}[2]{\left< #1, #2 \right>}
\newcommand{\powerset}[1]{\pazocal{P}(#1)}
%% Use \pazocal{letter} to typeset a letter in the other kind 
%%  of math calligraphic font. 
\newcommand{\cardinality}[1]{\left| #1 \right|}
\newcommand{\domain}[1]{\mathcal{D}(#1)}
\newcommand{\image}{\text{Im}}
\newcommand{\inv}[1]{#1^{-1}}
\newcommand{\preimage}[2]{#1^{-1}\left(#2\right)}
\newcommand{\script}[1]{\mathcal{#1}}


\newenvironment{highlight}{\begin{mdframed}[backgroundcolor=gray!20]}{\end{mdframed}}

\DeclarePairedDelimiter\ceil{\lceil}{\rceil}
\DeclarePairedDelimiter\floor{\lfloor}{\rfloor}

%===============================================
%===============My Tikz Commands================
%===============================================
\newcommand{\drawsquiggle}[1]{\draw[shift={(#1,0)}] (.005,.05) -- (-.005,.02) -- (.005,-.02) -- (-.005,-.05);}
\newcommand{\drawpoint}[2]{\draw[*-*] (#1,0.01) node[below, shift={(0,-.2)}] {#2};}
\newcommand{\drawopoint}[2]{\draw[o-o] (#1,0.01) node[below, shift={(0,-.2)}] {#2};}
\newcommand{\drawlpoint}[2]{\draw (#1,0.02) -- (#1,-0.02) node[below] {#2};}
\newcommand{\drawlbrack}[2]{\draw (#1+.01,0.02) --(#1,0.02) -- (#1,-0.02) -- (#1+.01,-0.02) node[below, shift={(-.01,0)}] {#2};}
\newcommand{\drawrbrack}[2]{\draw (#1-.01,0.02) --(#1,0.02) -- (#1,-0.02) -- (#1-.01,-0.02) node[below, shift={(+.01,0)}] {#2};}

%***********************************************
%**************Start of Document****************
%***********************************************
 %find me at /home/trevor/texmf/tex/latex/tskpreamble_nothms.tex
%===============================================
%===============Theorem Styles==================
%===============================================

%================Default Style==================
\theoremstyle{plain}% is the default. it sets the text in italic and adds extra space above and below the \newtheorems listed below it in the input. it is recommended for theorems, corollaries, lemmas, propositions, conjectures, criteria, and (possibly; depends on the subject area) algorithms.
\newtheorem{theorem}{Theorem}
\numberwithin{theorem}{section} %This sets the numbering system for theorems to number them down to the {argument} level. I have it set to number down to the {section} level right now.
\newtheorem*{theorem*}{Theorem} %Theorem with no numbering
\newtheorem{corollary}[theorem]{Corollary}
\newtheorem*{corollary*}{Corollary}
\newtheorem{conjecture}[theorem]{Conjecture}
\newtheorem{lemma}[theorem]{Lemma}
\newtheorem*{lemma*}{Lemma}
\newtheorem{proposition}[theorem]{Proposition}
\newtheorem*{proposition*}{Proposition}
\newtheorem{problemstatement}[theorem]{Problem Statement}


%==============Definition Style=================
\theoremstyle{definition}% adds extra space above and below, but sets the text in roman. it is recommended for definitions, conditions, problems, and examples; i've alse seen it used for exercises.
\newtheorem{definition}[theorem]{Definition}
\newtheorem*{definition*}{Definition}
\newtheorem{condition}[theorem]{Condition}
\newtheorem{problem}[theorem]{Problem}
\newtheorem{example}[theorem]{Example}
\newtheorem*{example*}{Example}
\newtheorem*{counterexample*}{Counterexample}
\newtheorem*{romantheorem*}{Theorem} %Theorem with no numbering
\newtheorem{exercise}{Exercise}
\numberwithin{exercise}{section}
\newtheorem{algorithm}[theorem]{Algorithm}

%================Remark Style===================
\theoremstyle{remark}% is set in roman, with no additional space above or below. it is recommended for remarks, notes, notation, claims, summaries, acknowledgments, cases, and conclusions.
\newtheorem{remark}[theorem]{Remark}
\newtheorem*{remark*}{Remark}
\newtheorem{notation}[theorem]{Notation}
\newtheorem*{notation*}{Notation}
%\newtheorem{claim}[theorem]{Claim}  %%use this if you ever want claims to be numbered
\newtheorem*{claim}{Claim}


%%
%% Page set-up:
%%
\pagestyle{empty}
\lhead{\textsc{221 - Topology} \\ }
\rhead{\textsc{McCammond, Winter 2019} \\ Trevor Klar}
%\chead{\Large\textbf{A New Integration Technique \\ }}
\renewcommand{\headrulewidth}{0pt}
%
\renewcommand{\footrulewidth}{0pt}

\setlength{\parindent}{0in}
\setlength{\textwidth}{7in}
\setlength{\evensidemargin}{-0.25in}
\setlength{\oddsidemargin}{-0.25in}
\setlength{\parskip}{.5\baselineskip}
\setlength{\topmargin}{-0.5in}
\setlength{\textheight}{9in}

\setlist[enumerate,1]{label=\textbf{\arabic*.}}

\renewcommand{\inv}[1]{\overline{#1}}

\begin{document}
\pagestyle{fancy}
\begin{center}
{\Large Homework 6}%===============UPDATE THIS===============%
\end{center}

\begin{enumerate}

\setcounter{enumi}{7}
\item Does the Borsuk-Ulam Theorem hold for the torus? In other words, for every map $f:S^1\times S^1\to \R^2$ must there exist $(x,y)\in S^1\times S^1$ such that $f(x,y)=f(-x,-y)$?

\answer No. Consider $f(x,y)=(\cos x, \sin x)$. This is clearly a map since it is constant with respect to $y$ and it is the inclusion map with respect to $x$, but $f$ never vanishes and is odd, so $f(x,y)\neq f(-x,-y)$ for all $(x,y)$. \qed

\setcounter{enumi}{9}
\item From the isomorphism $\pi_1\big((X\times Y),(x_0,y_0)\big)\approx \pi_1(X,x_0)\times \pi_1(Y,y_0)$ it follows that loops in $X\times	\{y_0\}$ and $\{x_0\times Y\}$ represent commuting elements of $\pi_1\big((X\times Y),(x_0,y_0)\big)$. Construct an explicit homotopy demonstrating this. 

\answer Let 
\begin{itemize}
\item $\gamma=\big(\tilde{\gamma}(s),y_0\big)$, where $\tilde{\gamma}$ is a loop in $(X,x_0)$. 
\item $\eta=\big(x_0,\tilde{\eta}(s)\big)$ where $\tilde{\eta}$ is a loop in $(Y,y_0)$. 
\item For any loop $\alpha$ in any space, let $\omega_t\sqcdot\alpha$ denote a reparametrization of $\alpha$ which is constant for $s\in [0,t]$, then does $\alpha$ over the remaining $s$-values in $[0,1]$. Define $\alpha\sqcdot\omega_t$ similarly except do $\alpha$ first, then wait. 
\end{itemize}
Observe that 
\begin{align*}
\gamma\sqcdot\eta&=(\tilde{\gamma}\sqcdot\omega_{\sfrac{1}{2}}\,\,,\, \omega_{\sfrac{1}{2}}\sqcdot\tilde{\eta})\\
\eta\sqcdot\gamma&=(\omega_{\sfrac{1}{2}}\sqcdot\tilde{\gamma}\,\,,\, \tilde{\eta}\sqcdot\omega_{\sfrac{1}{2}}),
\end{align*}
so 
$$f_t=(\omega_{\sfrac{t}{2}}\sqcdot\tilde{\gamma}\sqcdot\omega_{1-\sfrac{t}{2}}\,\,,\, \omega_{1-\sfrac{t}{2}}\sqcdot\tilde{\eta}\sqcdot\omega_{\sfrac{t}{2}})$$
is the desired homotopy. \qed

\end{enumerate}
\jpg{width=\textwidth}{hw6-p15-1}
\begin{enumerate}
\item[]
\begin{proof}
Let $\gamma$ be a loop based at $x_1$. Then 
$$\gamma\xmapsto{\quad\beta_h\quad}h\sqcdot\gamma\sqcdot\inv{h}\xmapsto{\quad f_*\quad}f(h\sqcdot \gamma\sqcdot \inv{h})=fh\sqcdot f\gamma\sqcdot f\inv{h}$$
and 
$$\gamma\xmapsto{\quad f_*\quad} f\gamma \xmapsto{\quad\beta_{fh}\quad}fh\sqcdot f\gamma\sqcdot \overline{fh}=fh\sqcdot f\gamma\sqcdot f\inv{h},$$
so the diagram commutes.
\end{proof}

\pagebreak
\setcounter{enumi}{15}
\item Show that there are no retractions $r:X\to A$ in the following cases:
	\begin{enumerate}[label=(\alph*)]
	\item $X=\R^3$ with $A$ any subspace homeomorphic to $S^1$. 
	\begin{proof}
	Suppose for contradiction there is a retraction $r:X\to A$, and let $\gamma$ be any loop in $A$, where $x_0$ denotes the basepoint. We know $\R^3$ is contractible, so by composing such a homotopy with $\gamma$ we can produce $f_t$, a straight-line homotopy in $\R^3$ from $\gamma$ to $x_0$. Then $r\circ f_t$ is a homotopy in $A$ from $\gamma$ to $x_0$, contradicting that $\pi_1(A)\neq0$. \qedwhitehere
	\end{proof}
	\item $X=S^1\times D^2$ with $A$ its boundary torus $S^1\times S^1$.
	\begin{proof}
	We know that $\pi_1(S^1\times S^1)=\Z\times\Z$, so any loop in $A$ is homotopic to a loop $(\omega_n, \gamma_m)$ for $n,m\in \Z$ which goes around the torus $n$ times in one direction and $m$ times in the the other.  Consider $(\omega_n, \gamma_m)\in X$, and note that $D^2$ is contractible, so there is a homotopy $f_t$ in $X$ between $(\omega_n, \gamma_m)$ and $(\omega_n, 0)$. Assuming a retraction $r:X\to A$ exists, we can compose $r\circ	f$ to find that $(\omega_n, \gamma_m)\simeq(\omega_n, x_0)$ in $A$, contradicting that $\pi_1(S^1\times S^1)\neq \Z\times 0$. \qedwhitehere
	\end{proof}
	\item $X=S^1\times D^2$ with $A$ the circle shown in the figure. 
	\begin{proof}
	We will use the fact that if $X$ retracts to $A$, then the inclusion map $A\hookrightarrow X$ induces an injection $\pi_1(A)\inj\pi_1(X)$ (and in fact, would have saved time by using it on parts (a) and (b) as well). Observe that the loop which traverses $A$ once is homotopic to $0$ in $X$ by the following homotopy: 
	\jpg{width=0.8\textwidth}{hw6-p16c-1}
	Thus $1\in\pi_1(A)\mapsto 0\in\pi_1(X)$, so the kernel is nontrivial, and the homomorphism is not injective. 
	\qedwhitehere
	\end{proof}
	\item $X=D^2\wedgeprod D^2$ with $A$ its boundary $S^1\wedgeprod S^1$.
	\begin{proof}
	We know that $\pi_1(X)=0$ since $X$ is contractible and $\pi_1(A)$ is the free group on two generators, so any homomorphism from $\pi_1(A)\to\pi_1(X)$ fails to be injective since everything maps to zero. 
	\end{proof}
	\item $X$ a disk with two points on its boundary identified and $A$ its boundary $S^1\wedgeprod S^1$.
	\begin{proof}
	If $g,f$ are the generators of $\pi_1(A)$, then $g\inv{f}$ is a nonzero element in $\pi_1(A)$ which maps to $0\in \pi_1(X)$ under the homomorphism induced by the inclusion:
	\jpg{width=0.5\textwidth}{hw6-p16e-1}
	so the homomorphism is not injective. \qedwhitehere
	\end{proof}
	\item $X$ the M\"obius band and $A$ its boundary circle. 
	\begin{proof}
	I can't figure this out. $X$ deformation retracts to its central circle, so $\pi_1(X)=\Z$, but also $A$ is a circle, so $\pi_1(A)=\Z$ as well. I can't come up with any reason why $\iota^*$ would fail to be injective either, since we can't homotope a nontrivial loop in the boundary to a constant loop in $X$ in any obvious way. In fact, it seems to me that $\iota^*(n)=2n$, which \emph{is }injective. 
	\end{proof}
	\end{enumerate}

\item Construct infinitely many nonhomotopic retractions $S^1\wedgeprod S^1\to S^1$. 

\answer Denote points in $S^1\wedgeprod S^1$ by $\theta\in(0,4\pi)$, so that $(0,2\pi]$ represents a point in the first circle, $[2\pi,4\pi)$ represents a point in the second circle, and $2\pi$ is in both circles. Then define 
\[
r_n(\theta)=\begin{cases}
\theta, &\theta \in (0,2\pi] \\
n\theta, &\theta \in [2\pi,4\pi)
\end{cases}
\]
where we make the usual identification in the codomain that $\theta\sim 2n\pi \,\,\forall n\in \N$. It is clear that these are nonhomotopic maps since they are loops in distinct equivalence classes of $\pi_1(S^1)$. 
\item 
Using Lemma 1.15, show that if a space $X$ is obtained from a path-connected subspace $A$ by attaching a cell $e^n$ with $n\geq 2$, then the inclusion $A\hookrightarrow X$ induces a surjection on $\pi_1$. Use this to show:
	\begin{enumerate}
	\item The wedge sum $S^1\wedgeprod S^2$ has fundamental group $\Z$. 
	\item For a path-connected CW complex $X$ the inclusion map $X^1\hookrightarrow X$ of its 1-skeleton induces a surjection $\pi_1(X^1)\to \pi_1(X)$. [For the case that $X$ has infinitely many cels, see Proposition A.1 in the appendix.]
	\end{enumerate}
\begin{proof}Consider $X$ as the union of $A_\epsilon$ and $e^n_\epsilon$, which are $\epsilon$-neighborhoods of $A$ and $e^n$ respectively in $X$. Since $A,e^n$ are deformation retractions of $A_\epsilon, e^n_\epsilon$, it suffices to show that the property holds for those subspaces. 
\jpg{width=0.8\textwidth}{hw6-p18-1}
For path-connected spaces, we know that $\pi_1$ is independent of base point, and $A$ and $e^n$ are both path-connected, which means $X$ is as well. Thus \Wlog{} we can suppose the basepoint $x_0\in(A_\epsilon\cap e^n_\epsilon)$. Given any loop $f$ in $X$, we can use Lemma 1.15 to write it as $f=\left(\bigsqcdot_{i=1}^n \gamma_i\right)$, where each $\gamma_i$ is in $A_\epsilon$ or $e^n_\epsilon$. Let 
$$E=\{i:\gamma_i\in e^n_\epsilon\}.$$
Since $e^n_\epsilon$ is contractible, then for each ${\gamma_i}$ such that ${i\in E}$, $\gamma_i\simeq\id$, so 
$$f=\left(
%\overset{n}{\underset{i=1}{\bigsqcdot}}
\bigsqcdot_{i=1}^n
\gamma_i\right)\simeq \left(\bigsqcdot_{i\in E^\complement} \gamma_i\right)$$
which is a loop in $A_\epsilon$. 

Thus for every loop in $X$, there is a homotopy equivalent loop in $A$, which means $\pi_1(X)$ is a subgroup of $\pi_1(A)$ (up to isomorphism). 
\qedwhitehere 
\end{proof}
\begin{proof}(i)
It suffices to show that the inclusion $S^1\hookrightarrow S^1\wedgeprod S^2$ induces an injective homomorphism $\pi_1(S^1)\to\pi_1(S^1\wedgeprod S^2)$, since it is already surjective by above. For any loop $\gamma\in S^1$ such that $\gamma\not\simeq c$ where $c$ is the constant loop, consider $\gamma\in (S^1\wedgeprod S^2)$. Call the intersection point of the wedge $x_0$, and \Wlog{} suppose $x_0$ is the basepoint. Suppose there is a homotopy $f_t$ of loops in $(S^1\wedgeprod S^2)$ such that $f_0=\gamma$ and $f_1=c$.  Since any loop with points in $S^1-\{x_0\}$ and $S^2-\{x_0\}$ must pass through $\{x_0\}$, then at every time $t$, $f_t$ is a concatenation of loops in $S^1$ with loops in $S^2$. Then we can write $F$ as a concatenation of homotopies 
$$f_t=\left(\bigsqcdot_{i=1}^n f_{i,t}\right). $$
Since $f_1=c$, then $f_{i,1}=c$ for all $i$, which means that if $f_{j,0}$ is any of the loops in $S^1$, then $f_{j,t}$ is a homotopy between that loop and $c$ ion $S^1$, contradicting that $\gamma\not\simeq c$. \qedwhitehere
\end{proof}
\begin{proof}(ii)
Inductively, we can see that 
$$\pi_1(X^1)\surj \pi_1(X^2)$$ 
since $X^2$ is exactly $X^1$ with 2-cells attached, and 
$$\pi_1(X^1)\surj \pi_1(X^n)\implies \pi_1(X^1)\surj \pi_1(X^{n+1})$$ 
since $X^n$ is exactly $X^{n+1}$ with $(n+1)$-cells attached. Thus the map is surjective for finite-dimensional CW complexes. 

Proposition A.1 shows that a compact subspace of a CW complex is contained in a finite subcomplex. This means that for an arbitrary curve $\gamma\in X^\infty$, the image of the curve is compact since $I$ is compact, so it is contained in $X^n$ for some $n$. Thus there is some $\gamma'\in X^n$ with $\gamma'\simeq\gamma	$ and we are done. 
\end{proof}
\end{enumerate}
\vfill

Collaborators:
\begin{enumerate}
\item[] None for this homework.
\end{enumerate}
\end{document}



