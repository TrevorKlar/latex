\documentclass[12pt,letterpaper]{article}

\usepackage{fancyhdr,fancybox}
\usepackage{wrapfig}

%% Useful packages
\usepackage{amssymb, amsmath, amsthm} 
%\usepackage{graphicx}  %%this is currently enabled in the default document, so it is commented out here. 
\usepackage{calrsfs}
\usepackage{braket}
\usepackage{mathtools}
\usepackage{lipsum}
\usepackage{tikz}
\usetikzlibrary{cd}
\usepackage{verbatim}
%\usepackage{ntheorem}% for theorem-like environments
\usepackage{mdframed}%can make highlighted boxes of text
%Use case: https://tex.stackexchange.com/questions/46828/how-to-highlight-important-parts-with-a-gray-background
\usepackage{wrapfig}
\usepackage{centernot}
\usepackage{subcaption}%\begin{subfigure}{0.5\textwidth}
\usepackage{pgfplots}
\pgfplotsset{compat=1.13}
\usepackage[colorinlistoftodos]{todonotes}
\usepackage[colorlinks=true, allcolors=blue]{hyperref}
\usepackage{xfrac}					%to make slanted fractions \sfrac{numerator}{denominator}
\usepackage{enumitem}            
    %syntax: \begin{enumerate}[label=(\alph*)]
    %possible arguments: f \alph*, \Alph*, \arabic*, \roman* and \Roman*
\usetikzlibrary{arrows,shapes.geometric,fit}

\DeclareMathAlphabet{\pazocal}{OMS}{zplm}{m}{n}
%% Use \pazocal{letter} to typeset a letter in the other kind 
%%  of math calligraphic font. 

%% This puts the QED block at the end of each proof, the way I like it. 
\renewenvironment{proof}{{\bfseries Proof}}{\qed}
\makeatletter
\renewenvironment{proof}[1][\bfseries \proofname]{\par
  \pushQED{\qed}%
  \normalfont \topsep6\p@\@plus6\p@\relax
  \trivlist
  %\itemindent\normalparindent
  \item[\hskip\labelsep
        \scshape
    #1\@addpunct{}]\ignorespaces
}{%
  \popQED\endtrivlist\@endpefalse
}
\makeatother

%% This adds a \rewnewtheorem command, which enables me to override the settings for theorems contained in this document.
\makeatletter
\def\renewtheorem#1{%
  \expandafter\let\csname#1\endcsname\relax
  \expandafter\let\csname c@#1\endcsname\relax
  \gdef\renewtheorem@envname{#1}
  \renewtheorem@secpar
}
\def\renewtheorem@secpar{\@ifnextchar[{\renewtheorem@numberedlike}{\renewtheorem@nonumberedlike}}
\def\renewtheorem@numberedlike[#1]#2{\newtheorem{\renewtheorem@envname}[#1]{#2}}
\def\renewtheorem@nonumberedlike#1{  
\def\renewtheorem@caption{#1}
\edef\renewtheorem@nowithin{\noexpand\newtheorem{\renewtheorem@envname}{\renewtheorem@caption}}
\renewtheorem@thirdpar
}
\def\renewtheorem@thirdpar{\@ifnextchar[{\renewtheorem@within}{\renewtheorem@nowithin}}
\def\renewtheorem@within[#1]{\renewtheorem@nowithin[#1]}
\makeatother

%% This makes theorems and definitions with names show up in bold, the way I like it. 
\makeatletter
\def\th@plain{%
  \thm@notefont{}% same as heading font
  \itshape % body font
}
\def\th@definition{%
  \thm@notefont{}% same as heading font
  \normalfont % body font
}
\makeatother

%===============================================
%==============Shortcut Commands================
%===============================================
\newcommand{\ds}{\displaystyle}
\newcommand{\B}{\mathcal{B}}
\newcommand{\C}{\mathbb{C}}
\newcommand{\F}{\mathbb{F}}
\newcommand{\N}{\mathbb{N}}
\newcommand{\R}{\mathbb{R}}
\newcommand{\Q}{\mathbb{Q}}
\newcommand{\T}{\mathcal{T}}
\newcommand{\Z}{\mathbb{Z}}
\renewcommand\qedsymbol{$\blacksquare$}
\newcommand{\qedwhite}{\hfill\ensuremath{\square}}
\newcommand*\conj[1]{\overline{#1}}
\newcommand*\closure[1]{\overline{#1}}
\newcommand*\mean[1]{\overline{#1}}
%\newcommand{\inner}[1]{\left< #1 \right>}
\newcommand{\inner}[2]{\left< #1, #2 \right>}
\newcommand{\powerset}[1]{\pazocal{P}(#1)}
%% Use \pazocal{letter} to typeset a letter in the other kind 
%%  of math calligraphic font. 
\newcommand{\cardinality}[1]{\left| #1 \right|}
\newcommand{\domain}[1]{\mathcal{D}(#1)}
\newcommand{\image}{\text{Im}}
\newcommand{\inv}[1]{#1^{-1}}
\newcommand{\preimage}[2]{#1^{-1}\left(#2\right)}
\newcommand{\script}[1]{\mathcal{#1}}


\newenvironment{highlight}{\begin{mdframed}[backgroundcolor=gray!20]}{\end{mdframed}}

\DeclarePairedDelimiter\ceil{\lceil}{\rceil}
\DeclarePairedDelimiter\floor{\lfloor}{\rfloor}

%===============================================
%===============My Tikz Commands================
%===============================================
\newcommand{\drawsquiggle}[1]{\draw[shift={(#1,0)}] (.005,.05) -- (-.005,.02) -- (.005,-.02) -- (-.005,-.05);}
\newcommand{\drawpoint}[2]{\draw[*-*] (#1,0.01) node[below, shift={(0,-.2)}] {#2};}
\newcommand{\drawopoint}[2]{\draw[o-o] (#1,0.01) node[below, shift={(0,-.2)}] {#2};}
\newcommand{\drawlpoint}[2]{\draw (#1,0.02) -- (#1,-0.02) node[below] {#2};}
\newcommand{\drawlbrack}[2]{\draw (#1+.01,0.02) --(#1,0.02) -- (#1,-0.02) -- (#1+.01,-0.02) node[below, shift={(-.01,0)}] {#2};}
\newcommand{\drawrbrack}[2]{\draw (#1-.01,0.02) --(#1,0.02) -- (#1,-0.02) -- (#1-.01,-0.02) node[below, shift={(+.01,0)}] {#2};}

%***********************************************
%**************Start of Document****************
%***********************************************
 %find me at /home/trevor/texmf/tex/latex/tskpreamble_nothms.tex
%===============================================
%===============Theorem Styles==================
%===============================================

%================Default Style==================
\theoremstyle{plain}% is the default. it sets the text in italic and adds extra space above and below the \newtheorems listed below it in the input. it is recommended for theorems, corollaries, lemmas, propositions, conjectures, criteria, and (possibly; depends on the subject area) algorithms.
\newtheorem{theorem}{Theorem}
\numberwithin{theorem}{section} %This sets the numbering system for theorems to number them down to the {argument} level. I have it set to number down to the {section} level right now.
\newtheorem*{theorem*}{Theorem} %Theorem with no numbering
\newtheorem{corollary}[theorem]{Corollary}
\newtheorem*{corollary*}{Corollary}
\newtheorem{conjecture}[theorem]{Conjecture}
\newtheorem{lemma}[theorem]{Lemma}
\newtheorem*{lemma*}{Lemma}
\newtheorem{proposition}[theorem]{Proposition}
\newtheorem*{proposition*}{Proposition}
\newtheorem{problemstatement}[theorem]{Problem Statement}


%==============Definition Style=================
\theoremstyle{definition}% adds extra space above and below, but sets the text in roman. it is recommended for definitions, conditions, problems, and examples; i've alse seen it used for exercises.
\newtheorem{definition}[theorem]{Definition}
\newtheorem*{definition*}{Definition}
\newtheorem{condition}[theorem]{Condition}
\newtheorem{problem}[theorem]{Problem}
\newtheorem{example}[theorem]{Example}
\newtheorem*{example*}{Example}
\newtheorem*{counterexample*}{Counterexample}
\newtheorem*{romantheorem*}{Theorem} %Theorem with no numbering
\newtheorem{exercise}{Exercise}
\numberwithin{exercise}{section}
\newtheorem{algorithm}[theorem]{Algorithm}

%================Remark Style===================
\theoremstyle{remark}% is set in roman, with no additional space above or below. it is recommended for remarks, notes, notation, claims, summaries, acknowledgments, cases, and conclusions.
\newtheorem{remark}[theorem]{Remark}
\newtheorem*{remark*}{Remark}
\newtheorem{notation}[theorem]{Notation}
\newtheorem*{notation*}{Notation}
%\newtheorem{claim}[theorem]{Claim}  %%use this if you ever want claims to be numbered
\newtheorem*{claim}{Claim}


%%
%% Page set-up:
%%
\pagestyle{empty}
\lhead{\textsc{221 - Topology} \\ }
\rhead{\textsc{McCammond, Winter 2019} \\ Trevor Klar}
%\chead{\Large\textbf{A New Integration Technique \\ }}
\renewcommand{\headrulewidth}{0pt}
%
\renewcommand{\footrulewidth}{0pt}

\setlength{\parindent}{0in}
\setlength{\textwidth}{7in}
\setlength{\evensidemargin}{-0.25in}
\setlength{\oddsidemargin}{-0.25in}
\setlength{\parskip}{.5\baselineskip}
\setlength{\topmargin}{-0.5in}
\setlength{\textheight}{9in}

\setlist[enumerate,1]{label=\textbf{\arabic*.}}

\renewcommand{\inv}[1]{\overline{#1}}

\begin{document}
\pagestyle{fancy}
\begin{center}
{\Large Homework 7}%===============UPDATE THIS===============%
\end{center}

\begin{enumerate}


\item Show that (i) the free product $G*H$ of nontrivial groups $G$ and $H$ has trivial center, and (ii) the only elements of $G*H$ of finite order are conjugates of finite-order elements of $G$ and $H$. 
\begin{proof}(i)
Let $g\in G$, $h\in H$. If $z$ is in the center of $G*H$, then $zg=gz$ which means that $z$ is either $g, g^{-1},$ or the empty word $[e]$. Similarly, $zh=hz$ which means that $z$ is either $h, h^{-1},$ or $[e]$. Thus if $G,H$ are nontrivial then for nontrivial $g,h$ we have $g\neq g^{-1}\neq h\neq h^{-1}$, so $z$ must be $[e]$. 
\end{proof}
\begin{proof}(ii)
Let $x\in G*H$ with $x^n=[e]$. By simplifying, we can write $x$ as a word of alternating letters in $G$ and $H$, so $x=(g_1h_1g_2\dots g_{k-1}h_kg_k)$. Then for as many letters as possible (potentially none), we group the letters on the ends of the word which are inverses; i.e. 
\begin{align*}
x&=(g_1h_1)(g_2\dots g_{k-1})(h_kg_k)\\
x&=(g_1h_1)(g_2\dots g_{k-1})(h_1^{-1}g_1^{-1}).\\
x&=(g_1h_1)(z)(h_1^{-1}g_1^{-1}),\\
\end{align*}
where we let $z$ be the alternating word $(g_2\dots g_{k-1})$. Now we consider $x^n$. We know that everything must cancel, and the ends of course cancel, which means 
\begin{align*}
[e]&=x^n \\
&=(g_1h_1)(z)(h_1^{-1}g_1^{-1}) \, \, 
(g_1h_1)(z)(h_1^{-1}g_1^{-1}) \cdots
(g_1h_1)(z)(h_1^{-1}g_1^{-1})\\
&=(g_1h_1)(z)^n(h_1^{-1}g_1^{-1})\\
\end{align*}
which means that $z$ must be of order $n$. However, no  alternating word of $g$'s and $h$'s with length at least 2 is of finite order, since self-multiplying would just concatenate the same word with itself. Therefore $z$ must be a single letter of order $n$ in either $G$ or $H$, and $x$ is a conjugate of $z$. 
\end{proof}

\setcounter{enumi}{2}
\item Show that the complement of a finite set of points in $\R^n$ is simply connected if $n\geq3$. 
\begin{proof}
\WLOG{} Let $X=\R^n - \{\vec{x}_i\}_{i=1}^{k}$, with basepoint the origin. 

(i) To see that $X$ is path-connected, let $x\in X$. If $x $ is not a multiple of $ x_i$ for any $i$, then there is a straight-line path from $x$ to the origin. Otherwise suppose $x \parallel x_i$ for some $i$. Then choose a point $x'$ near $x$ such that $x $ is not a multiple of $ x_i$ for any $i$ (we know we can do this since there are infinitely many directions orthogonal to $x$ in which to find $x'$, and only infinitely many $x_i$), and there is a straight-line path from $x$ to $x$ to $0$. Thus for every $x\in X$ we have a path $\rho_x:I\to X$ connecting $x$ to $0$, so $X$ is path-connected.

(ii) To see that $\pi_1(X)=0$, observe that for any loop $\gamma$ in $X$, we can find a homotopy from $\gamma$ to the constant loop $0$ by using $\rho_x$. Let $R:X\times I \to X$ be given by $R(x,t)=\rho_x(t)$. Then $R(\gamma(s),t)$ is the desired homotopy. 
\end{proof}

\item Let $X\subset \R^3$ be the union of $n$ lines through the origin. Compute $\pi_1(\R^3 - X)$. We can deformation retract $X$ to a ball with $n$ lines missing, and since the origin is missing, we can deformation retract to a sphere with $2n$ antipodal points missing. 
\jpg{width=0.8\textwidth}{hw7-p4-1}
This sphere is diffeomorphic via spherical projection to a disk with $2n-1$ points missing, which deformation retracts to a wedge of $2n-1$ circles. Thus $\pi_1(\R^3-X)$ is the free group on $2n-1$ generators. \qed 

\setcounter{enumi}{6}
\item Let $X$ be the quotient space of $S^2$ obtained by identifying the north and south poles to a single point. Put a cell complex structure on $X$ and use it to compute $\pi_1(X)$. 

\answer First observe that $X$ the following shapes have the same homotopy type:
\jpg{width=0.8\textwidth}{hw7-p7-1}
The last figure is the same because the center disc is contractible, and modding it out yields the middle figure. It is on the last figure that we put a cell structure. $X$ has one 0-cell, two 1-cells and a 2-cell to form a torus, with one additional 2-cell attached spanning the center. We know that a torus has fundamental group $\Z\times	\Z$, and if we attach the additional 2-cell along the loop $(1,0)$, then 
$$\pi_1(X)=\quotient{\Z\times\Z}{\langle(1,0)\rangle}=\quotient{\Z\times\Z}{\Z}=\Z.$$
\qed 

\pagebreak
\item Compute the fundamental group of the space obtained from two tori $S^1\times S^1$ by identifying a circle $S^1\times \{x_0\}$ with the corresponding circle in the other torus. 

\answer We can put a cell structure on this space as a wedge of 3 circles $a,b,c$, with 2-cells attached along $ab\inv{a}\inv{b}$ and $ac\inv{a}\inv{c}$, respectively. 
\jpg{width=0.8\textwidth}{hw7-p8-1}
Thus $\pi_1(X)=\quotient{\bigop{*}\limits_{i=1}^3\Z_i}{\begin{array}{l}
aba^{-1}b^{-1} \\ aca^{-1}c^{-1}
\end{array} }$.
\qed 

Remark: Is the above notation equivalent to $\pi_1(X)=\langle a,b,c \mid ab=ba, ac=ca \rangle$? I'm not totally clear on how that notation works.

\setcounter{enumi}{16}
\item Show that $\pi_1(\R^2-\Q^2)$ is uncountable. 
\begin{proof}
Let $(\pi,0)$ be the basepoint. For every irrational number $\alpha$, let $\gamma_\alpha$ be the straight line path from $(\pi,0)$ to $(\pi,\pi)$ to $(\alpha,\pi)$ to $(\alpha, -\pi)$ to  $(\pi, -\pi)$ to $(\pi,0)$. 
\jpg{width=0.6\textwidth}{hw7-p17-1}
Clearly $\gamma_\alpha\not\simeq\gamma_\beta$ for $\alpha\neq\beta$, since any homotopy between the corresponding third segments of the paths would have to pass through points in $\Q^2$ as the $x$-coordinates moved from $\alpha$ to $\beta$. Thus we have an injective function from irrational numbers to loops in $\pi_1(X)$, so we're done. 
\end{proof}



\end{enumerate}
\vfill

Collaborators:
\begin{enumerate}
\item[] None for this homework.
\end{enumerate}
\end{document}



