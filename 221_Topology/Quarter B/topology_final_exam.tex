\documentclass[12pt,letterpaper]{article}

\usepackage{fancyhdr,fancybox}
\usepackage{wrapfig}

%% Useful packages
\usepackage{amssymb, amsmath, amsthm} 
%\usepackage{graphicx}  %%this is currently enabled in the default document, so it is commented out here. 
\usepackage{calrsfs}
\usepackage{braket}
\usepackage{mathtools}
\usepackage{lipsum}
\usepackage{tikz}
\usetikzlibrary{cd}
\usepackage{verbatim}
%\usepackage{ntheorem}% for theorem-like environments
\usepackage{mdframed}%can make highlighted boxes of text
%Use case: https://tex.stackexchange.com/questions/46828/how-to-highlight-important-parts-with-a-gray-background
\usepackage{wrapfig}
\usepackage{centernot}
\usepackage{subcaption}%\begin{subfigure}{0.5\textwidth}
\usepackage{pgfplots}
\pgfplotsset{compat=1.13}
\usepackage[colorinlistoftodos]{todonotes}
\usepackage[colorlinks=true, allcolors=blue]{hyperref}
\usepackage{xfrac}					%to make slanted fractions \sfrac{numerator}{denominator}
\usepackage{enumitem}            
    %syntax: \begin{enumerate}[label=(\alph*)]
    %possible arguments: f \alph*, \Alph*, \arabic*, \roman* and \Roman*
\usetikzlibrary{arrows,shapes.geometric,fit}

\DeclareMathAlphabet{\pazocal}{OMS}{zplm}{m}{n}
%% Use \pazocal{letter} to typeset a letter in the other kind 
%%  of math calligraphic font. 

%% This puts the QED block at the end of each proof, the way I like it. 
\renewenvironment{proof}{{\bfseries Proof}}{\qed}
\makeatletter
\renewenvironment{proof}[1][\bfseries \proofname]{\par
  \pushQED{\qed}%
  \normalfont \topsep6\p@\@plus6\p@\relax
  \trivlist
  %\itemindent\normalparindent
  \item[\hskip\labelsep
        \scshape
    #1\@addpunct{}]\ignorespaces
}{%
  \popQED\endtrivlist\@endpefalse
}
\makeatother

%% This adds a \rewnewtheorem command, which enables me to override the settings for theorems contained in this document.
\makeatletter
\def\renewtheorem#1{%
  \expandafter\let\csname#1\endcsname\relax
  \expandafter\let\csname c@#1\endcsname\relax
  \gdef\renewtheorem@envname{#1}
  \renewtheorem@secpar
}
\def\renewtheorem@secpar{\@ifnextchar[{\renewtheorem@numberedlike}{\renewtheorem@nonumberedlike}}
\def\renewtheorem@numberedlike[#1]#2{\newtheorem{\renewtheorem@envname}[#1]{#2}}
\def\renewtheorem@nonumberedlike#1{  
\def\renewtheorem@caption{#1}
\edef\renewtheorem@nowithin{\noexpand\newtheorem{\renewtheorem@envname}{\renewtheorem@caption}}
\renewtheorem@thirdpar
}
\def\renewtheorem@thirdpar{\@ifnextchar[{\renewtheorem@within}{\renewtheorem@nowithin}}
\def\renewtheorem@within[#1]{\renewtheorem@nowithin[#1]}
\makeatother

%% This makes theorems and definitions with names show up in bold, the way I like it. 
\makeatletter
\def\th@plain{%
  \thm@notefont{}% same as heading font
  \itshape % body font
}
\def\th@definition{%
  \thm@notefont{}% same as heading font
  \normalfont % body font
}
\makeatother

%===============================================
%==============Shortcut Commands================
%===============================================
\newcommand{\ds}{\displaystyle}
\newcommand{\B}{\mathcal{B}}
\newcommand{\C}{\mathbb{C}}
\newcommand{\F}{\mathbb{F}}
\newcommand{\N}{\mathbb{N}}
\newcommand{\R}{\mathbb{R}}
\newcommand{\Q}{\mathbb{Q}}
\newcommand{\T}{\mathcal{T}}
\newcommand{\Z}{\mathbb{Z}}
\renewcommand\qedsymbol{$\blacksquare$}
\newcommand{\qedwhite}{\hfill\ensuremath{\square}}
\newcommand*\conj[1]{\overline{#1}}
\newcommand*\closure[1]{\overline{#1}}
\newcommand*\mean[1]{\overline{#1}}
%\newcommand{\inner}[1]{\left< #1 \right>}
\newcommand{\inner}[2]{\left< #1, #2 \right>}
\newcommand{\powerset}[1]{\pazocal{P}(#1)}
%% Use \pazocal{letter} to typeset a letter in the other kind 
%%  of math calligraphic font. 
\newcommand{\cardinality}[1]{\left| #1 \right|}
\newcommand{\domain}[1]{\mathcal{D}(#1)}
\newcommand{\image}{\text{Im}}
\newcommand{\inv}[1]{#1^{-1}}
\newcommand{\preimage}[2]{#1^{-1}\left(#2\right)}
\newcommand{\script}[1]{\mathcal{#1}}


\newenvironment{highlight}{\begin{mdframed}[backgroundcolor=gray!20]}{\end{mdframed}}

\DeclarePairedDelimiter\ceil{\lceil}{\rceil}
\DeclarePairedDelimiter\floor{\lfloor}{\rfloor}

%===============================================
%===============My Tikz Commands================
%===============================================
\newcommand{\drawsquiggle}[1]{\draw[shift={(#1,0)}] (.005,.05) -- (-.005,.02) -- (.005,-.02) -- (-.005,-.05);}
\newcommand{\drawpoint}[2]{\draw[*-*] (#1,0.01) node[below, shift={(0,-.2)}] {#2};}
\newcommand{\drawopoint}[2]{\draw[o-o] (#1,0.01) node[below, shift={(0,-.2)}] {#2};}
\newcommand{\drawlpoint}[2]{\draw (#1,0.02) -- (#1,-0.02) node[below] {#2};}
\newcommand{\drawlbrack}[2]{\draw (#1+.01,0.02) --(#1,0.02) -- (#1,-0.02) -- (#1+.01,-0.02) node[below, shift={(-.01,0)}] {#2};}
\newcommand{\drawrbrack}[2]{\draw (#1-.01,0.02) --(#1,0.02) -- (#1,-0.02) -- (#1-.01,-0.02) node[below, shift={(+.01,0)}] {#2};}

%***********************************************
%**************Start of Document****************
%***********************************************
 %find me at /home/trevor/texmf/tex/latex/tskpreamble_nothms.tex
%===============================================
%===============Theorem Styles==================
%===============================================

%================Default Style==================
\theoremstyle{plain}% is the default. it sets the text in italic and adds extra space above and below the \newtheorems listed below it in the input. it is recommended for theorems, corollaries, lemmas, propositions, conjectures, criteria, and (possibly; depends on the subject area) algorithms.
\newtheorem{theorem}{Theorem}
\numberwithin{theorem}{section} %This sets the numbering system for theorems to number them down to the {argument} level. I have it set to number down to the {section} level right now.
\newtheorem*{theorem*}{Theorem} %Theorem with no numbering
\newtheorem{corollary}[theorem]{Corollary}
\newtheorem*{corollary*}{Corollary}
\newtheorem{conjecture}[theorem]{Conjecture}
\newtheorem{lemma}[theorem]{Lemma}
\newtheorem*{lemma*}{Lemma}
\newtheorem{proposition}[theorem]{Proposition}
\newtheorem*{proposition*}{Proposition}
\newtheorem{problemstatement}[theorem]{Problem Statement}


%==============Definition Style=================
\theoremstyle{definition}% adds extra space above and below, but sets the text in roman. it is recommended for definitions, conditions, problems, and examples; i've alse seen it used for exercises.
\newtheorem{definition}[theorem]{Definition}
\newtheorem*{definition*}{Definition}
\newtheorem{condition}[theorem]{Condition}
\newtheorem{problem}[theorem]{Problem}
\newtheorem{example}[theorem]{Example}
\newtheorem*{example*}{Example}
\newtheorem*{counterexample*}{Counterexample}
\newtheorem*{romantheorem*}{Theorem} %Theorem with no numbering
\newtheorem{exercise}{Exercise}
\numberwithin{exercise}{section}
\newtheorem{algorithm}[theorem]{Algorithm}

%================Remark Style===================
\theoremstyle{remark}% is set in roman, with no additional space above or below. it is recommended for remarks, notes, notation, claims, summaries, acknowledgments, cases, and conclusions.
\newtheorem{remark}[theorem]{Remark}
\newtheorem*{remark*}{Remark}
\newtheorem{notation}[theorem]{Notation}
\newtheorem*{notation*}{Notation}
%\newtheorem{claim}[theorem]{Claim}  %%use this if you ever want claims to be numbered
\newtheorem*{claim}{Claim}


%%
%% Page set-up:
%%
\pagestyle{empty}
\lhead{\textsc{221B - Topology} \\ }
\rhead{\textsc{McCammond, Winter 2019} \\ Trevor Klar}
%\chead{\Large\textbf{A New Integration Technique \\ }}
\renewcommand{\headrulewidth}{0pt}
%
\renewcommand{\footrulewidth}{0pt}
\newcommand{\RP}{\mathbb{RP}}

\setlength{\parindent}{0in}
\setlength{\textwidth}{7in}
\setlength{\evensidemargin}{-0.25in}
\setlength{\oddsidemargin}{-0.25in}
\setlength{\parskip}{.5\baselineskip}
\setlength{\topmargin}{-0.5in}
\setlength{\textheight}{9in}

\setlist[enumerate,1]{label=\textbf{\arabic*.}}

\renewcommand{\inv}[1]{\overline{#1}}
\let\oldphi\phi
\renewcommand{\phi}{\varphi}
\renewcommand{\image}[1]{\operatorname{im}\left(#1\right)}
\renewcommand{\S}{\mathbb{S}}

\begin{document}
\pagestyle{fancy}
\begin{center}
{\Large Final Exam}%===============UPDATE THIS===============%
\end{center}

\begin{enumerate}

\item \mbox{}\vspace*{-24pt}
\begin{definition*}
Let $X$ be a topological space, with $x\in X$. A \emph{loop} $f$ in $X$ is a map $f:I\to X$ with $f(0)=f(1)=x$. We call $x$ the \emph{basepoint} of $f$. 
\end{definition*}

\begin{definition*}
Let $f$ and $g$ be two loops in $X$, both with basepoint $x_0$. A \emph{loop homotopy} between $f$ and $g$ is a homotopy $\phi_t$ such that $\phi_0=f, \phi_1=g,$ and $\phi_t(0)=\phi_t(1)=x_0$ for all $t$. 
\end{definition*}

\begin{definition*}
Let $(X,x_0)$ be a based topological space. The \textbf{fundamental group} $\pi_1(X,x_0)$\footnote{When the basepoint is irrelevant or it is clear what the basepoint is, we will suppress notation and write $\pi_1(X)$. If not specified, one can assume $x_0$ is the basepoint of $X$, $a_0$ is the basepoint of $A$, etc.} is the set of all loop homotopy classes $[f]$ of loops in $X$ based at $x_0$, equipped with the product $[f][g]:=[f\sqcdot g]$.\footnote{Here $\sqcdot$ denotes concatenation, and sometimes we will denote concatenation by juxtaposition when the meaning is clear.}
\end{definition*}

\begin{definition*}
Let $p:C\to X$ be a map, and let $U$ be open in $X$. We say $U$ is \emph{evenly covered} by $p$ if 
	\begin{itemize}
	\item $\preimage{p}{U}=\coprod_\alpha \tilde{U}_\alpha$, where each $\tilde{U}_\alpha$ open in $C$, and 
	\item each $\tilde{U}_\alpha$ is homeomorphic to $U$, denoted $\tilde{U}_\alpha \cong U$. 
	\end{itemize}
\end{definition*}

\begin{definition*}
Let $(X,x_0)$ be a based topological space. A \textbf{covering space} of $X$ is a based topological space $(C,c_0)$ together with a basepoint preserving map $p:C\to X$ such that 
\begin{itemize}
\item for every $x\in X$, there exists a neighborhood $U\ni x$ which is evenly covered by $p$. \footnote{We often suppress notation by calling $p:C\to X$ a covering space, inferring that $C$ is the accompanying topological space, and assuming the basepoints are named whatever makes sense.}
\end{itemize} 
\end{definition*}

\begin{remark*}
 We show in Hatcher that a covering map $p:C\to X$ induces an injective homomorphism $p_*:\pi_1(C)\to\pi_1(X)$, and that the image $\image{p_*}$ is a subgroup of $\pi_1(X)$. 
\end{remark*}

\begin{definition*}
Let $p:(C,c_0)\to(X,x_0)$ and $r:(D,d_0)\to(X,x_0)$ be covering spaces. We say that \textbf{$p$ and $r$ are isomorphic} if 
$$\image{p_*}\cong\image{r_*}$$
\end{definition*}

\item \mbox{}\vspace*{-24pt}
\begin{definition*}
Let $p:\tilde{X}\to X$ be a covering space, and let $f:Y\to X$ be a map. A \emph{lift} of $f$ is a map $\tilde{f}:Y \to \tilde{X}$ such that $p\circ \tilde{f}=f$. 
\end{definition*}

\begin{theorem*}(Homotopy Lifting Property)\\
Given a homotopy $f_t:Y\to X$ and a map $\tilde{f}_0:Y\to X$ lifting $f_0$, there is a unique homotopy $\tilde{f}_t$ lifting $f_t$ which extends $\tilde{f}_0$. 
\end{theorem*}
\begin{remark*}
We proved this in Hatcher. Also note that we can apply this to map homotopies, path homotopies, loop homotopies, and even paths themselves which can be thought of as point homotopies. 
\end{remark*}

\pagebreak
\begin{theorem*}
$\pi_1(\mathbb{S}^1)=\Z$. 
\end{theorem*}
\begin{proof}
Let $p:\R\to\mathbb{S}^1$ be 
$$p(s)=\big(\cos(2\pi s), \sin(2\pi s)\big).$$
\textsc{Claim.} $p:(\R, 0)\to \big(\mathbb{S}^1, (1,0)\big)$ is a covering space. 

\textsc{Proof of claim.} Observe that $p$ is a basepoint preserving map since it is continuous and $p(0)=(1,0)$. To see that every neighborhood of a point in $S$ is evenly covered, observe that it can be factored as $p=\text{proj}_{xy}\circ h$, where $h:\R\to\R^3$ and $\text{proj}_{xy}:\image{h}\to\R^2$ are given by 
\begin{align*}
h(s)&=\big(\cos(2\pi s), \sin(2\pi s), s\big) \\
\text{proj}_{xy} (x,y,z) &= (x,y).
\end{align*}
Geometrically it is clear that for every $s\in\mathbb{S}^1$ with sufficiently small $U\ni s$ open in $\mathbb{S}^1$, the preimage $\preimage{\text{proj}_{xy}}{U}$ is a collection of disjoint stacked arcs in the helix, and $\preimage{h}{\preimage{\text{proj}_{xy}}{U}}$ is a collection of disjoint open intervals, each of length less than $1$. 

To see that each of these disjoint open intervals is homeomorphic to its image in $p$, observe that $h$ is a homeomorphism since it is continuous with continuous inverse $h^{-1}(x,y,z)=z$, and $\text{proj}_{xy}$ is continuous because it is a projection. To see that $\text{proj}_{xy}$ has continuous inverse on one of these small helix arcs, let $(x_1,y_1), (x_2,y_2)$ be two points in $\mathbb{S}^1$ such that the angle between them $\theta$ is less than $2\pi$. For any $(x,y)$ in the arc between $(x_1,y_1)$ and $(x_2,y_2)$ in $\mathbb S^1$, let $\theta(x,y)$ be the angle between $(x_1,y_1)$ and $(x,y)$. Denote $s_0$ the lower bound of the small interval whose image in $h$ gives us the helix arc from $(x_1,y_1)$ to $(x_2,y_2)$. Then
$$\preimage{\text{proj}_{xy}}{(x,y)}=\left(x,y,s_0+\tfrac{1}{2\pi} \theta(x,y)\right)$$
is the inverse of $\text{proj}_{xy}$ and it is continuous. Thus the claim is proved. \qedwhite

Now that we have established that $p$ is a covering map of $\mathbb{S}^1$, we classify the loops in $\pi_1(\mathbb{S}^1)$. 
Let $f$ be a loop in $\mathbb{S}^1$ based at $(1,0)$. If we choose $\tilde{f}(0)=0$, then by the Homotopy Lifting Property,  $f$ lifts uniquely to $\tilde{f}$ with $\tilde{f}(1)\in\preimage{p}{1,0}=\Z$. 

\textsc{Claim. } $\Theta:\pi_1(\mathbb{S}^1)\to\Z$ is a group isomorphism given by 
$$\Theta[f]=\tilde{f}(1) \text{ where } \tilde{f}(0)=0.$$

\textsc{Proof of claim. }
	\begin{itemize}
	\item To see that $\Theta$ is well-defined, let $f\overset{\phi}{\simeq} g$, with $\tilde{f}$ such that $\tilde{f}(0)=0$. Then applying the Homotopy Lifting Property to $\phi$ we obtain $\tilde{\phi}$ with $\tilde{\phi}_0=\tilde{f}$, and since $\phi_t(1)$ is fixed for all time then so is $\tilde{\phi}_t(1)$, so $\tilde{\phi}_1(1)=\tilde{g}(1)=\tilde{f}(1)$. Thus $\Theta[f]=\Theta[g]$. 
	
	\item To see that $\Theta$ is onto, for all $n\in \Z$ let 
	$$\omega_n(s)=(\cos 2\pi ns, \sin 2\pi ns).$$
	If we choose $\tilde{\omega}_n(0)=0$, then by the Homotopy Lifting Property $\tilde{\omega}_n(s)=ns$, so $\tilde{\omega}_n(1)=n$ and $\Theta[\omega_n]=n$. 
	
	\item To see that $\Theta$ is a homomorphism, note that since 
	$\omega_1\sqcdot\omega_1$ reparametrizes to $\omega_2$, then $[\omega]^n=[\omega_n]$. Thus 
	$$\Theta\big([\omega_n][\omega_m]\big)=\Theta[\omega_{n+m}]=n+m.$$
	Also, if $f$ is a loop with $\tilde{f}(1)=n$, then $\tilde{f}$ and $\tilde{\omega}_n$ are both paths from 0 to $n$, and they are homotopic by the straight-line homotopy $\Phi(t,s)=(1-t)\tilde{f}-t\tilde{\omega}_n$. Thus $p\circ\Phi:I\times I \to \mathbb{S}^1$ is a map with $\phi_0=f$, $\phi_1=\omega_n$. This is a homotopy exhibiting $f\simeq \omega_n$. 
	
	\item To see that $\Theta$ is injective, observe that $[f]\in\ker(\Theta)$ means that $f\simeq\omega_0$ which is a constant loop, so $\Theta$ has trivial kernel. 
	\end{itemize}
Thus we have shown that $\pi_1(\mathbb{S}^1)$ is exactly $\Z$, and we are done.
\end{proof}

\item \mbox{}\vspace*{-25pt}
%\begin{lemma*}
%Let $(X,x_0)$ be a based topological space. If $\phi_t:X\to Y$ is a homotopy and we denote $h(t)=\phi_t(x_0)$ as $t$ varies, then $\phi_{0*}=\beta_h\circ	\phi_{1*}$ and the following diagram commutes:
%\jpg{width=0.33\textwidth}{final_w20_p3-1}
%\end{lemma*}
%\vspace*{-18pt}
%\begin{proof}
%Let $h_t$ be the homotopy which traverses $h$ up to point $t$. Explicitly, $h_t(s)=h(ts)$. Let $f$ be a loop in $X$ based at $x_0$ and consider the composition $\phi_t\circ f(s)$. Since for all $t$  
%$$h_t(1)=h(t)=\phi_t(x_0)=\phi_t\circ f(0) = \phi_t\circ f(1),$$
%we can concatenate to form a homotopy
%$$h_t\sqcdot\phi_t\circ f \sqcdot \inv{h}_t$$
%which is actually a loop homotopy exhibiting $\phi_0\circ f \simeq h\sqcdot\phi_1\circ f \sqcdot \inv{h}$. By applying the definitions, we find that $\phi_{0*}[f]=\beta_h\circ\phi_{1*}[f]$ which means $\phi_{0*}=\beta_h\circ	\phi_{1*}$.
%\qedwhitehere
%\end{proof}
%
%\begin{corollary*}
%For any two homotopic maps $f,g:X\to Y$, the induced maps satisfy 
%$$f_*=\beta_h \circ g_*,$$
%where $\beta_h$ is a change-of-basepoint homomorphism.
%\end{corollary*}

\begin{definition*}
Let $(X,x_0)$ and $(Y,y_0)$ be based topological spaces. We say that these spaces are \emph{homotopic} and write $(X,x_0) \simeq  (Y,y_0)$ if there are basepoint-preserving maps 
$$f:(X,x_0)\to (Y,y_0)\text{ and }g:(Y,y_0)\to(X,x_0)$$ 
such that 
$$f\circ g\simeq \id\text{ and }g\circ f\simeq 1.$$
Such $f,g$ are called \emph{homotopy equivalences}. 
\end{definition*}

\begin{theorem*}
If $f:(X,x_0)\to (Y,y_0)$ is a homotopy equivalence, then $f_*:\pi_1(X,x_0)\to\pi_1(Y,y_0)$ is a group isomorphism. 
\end{theorem*}
\begin{proof}
Let $g$ be the homotopy inverse for $f$, so $f\circ g\simeq	\id$ and $g\circ f\simeq \id$. We will show that the induced maps are inverse homomorphisms. Observe that $f_*g_*=(fg)_*$ since for any loop $\gamma$ in $(Y,y_0)$
$$[\gamma]\xmapsto{\quad\;\,\,(fg)_*\quad\;\,\,} [fg(\gamma)]$$
$$[\gamma]\overset{g_*}{\mapsto}[g(\gamma)]\overset{f_*}{\mapsto}[fg(\gamma)]$$
and since both $f,g$ preserve basepoints, then $fg$ and $\id$ both fix their basepoint. Since $fg\simeq\id$, so there is a homotopy $\phi_t$ exhibiting $fg\simeq\id$ which fixes the basepoint for all time. Thus composing with $\gamma$ yields $\phi_t(\gamma)$ which is a loop homotopy exhibiting $fg(\gamma)\simeq \gamma$. Thus
$$[fg(\gamma)]=[\gamma],$$
and so $f_*g_*[\gamma]=[\gamma]$, so $f_*g_*=\id$. 

The same argument shows that $g_*f_*=\id$, and so $f_*$ and $g_*$ are group isomorphisms. 
\end{proof}

\pagebreak
\item Let $X=\RP^2\wedgeprod\RP^2$.
	\begin{enumerate}[label=(\alph*)]
	\item Find $\pi_1(X)$. 
	
	\answer	We know already that $\pi_1(\RP^2)=\Z_2$, so applying Van Kampen's Theorem, $\pi_1(\RP^2\wedgeprod\RP^2)=\Z_2*\Z_2$. 
	
	\item Find the universal cover of $X$. 
	
	\answer The following cover, where each $S^2$ maps to the same color $\RP^2$ by the real projection map. 
	
	\jpg{width=0.8\textwidth}{final_w20_p4-1}
	
	\item Find all of its connected 2-sheeted covers.
	
	\answer The following 4 covers with points at the end identified (thus all points in pink are identified). Each part maps to the same color part in $X$ with either the real projection map or the identity map, as appropriate. 
	
	\jpg{width=0.8\textwidth}{final-w20-p4-2}
	\end{enumerate}

\pagebreak
\item Let $X=\RP^2\times\RP^2$.
	\begin{enumerate}[label=(\alph*)]
	\item Find $\pi_1(X)$. 
	
	\answer	$\Z_2\times\Z_2$. 
	
	\item Find the universal cover of $X$. 
	
	\answer $\S^2\times \S^2$, with the product of real projection maps. 
	
	\jpg{width=0.33\textwidth}{final-w20-p5-1}
	
	\item Find all of its connected 2-sheeted covers.
	
	\answer A multi-sheeted cover of $\RP^2\times\RP^2$ with $n$ sheets in the first coordinate and $m$ sheets in the second coordinate will necessarily have $nm$ sheets, so any two-sheeted cover with have 2 sheets in one coordinate and 1 sheet in the other. This leaves two choices for product space covers: 
	\jpg{width=0.8\textwidth}{final-w20-p5-2}
	%However, there is one more subgroup of $\Z_2\times\Z_2$, namely $\{(0,0), (1,1)\}$ which we can present as $\angles{ab \mid aba^{-1}b^{-1}}$. 
	\end{enumerate}

\pagebreak 

\begin{theorem*}(Van Kampen's Theorem [two-set version])

If 
	\begin{itemize}
	\item $X=A\cup B$, 
	\item each of $A,B$ is open, path-connected, and contains the basepoint, 
	\item $A\cap B$ is path-connected,
	\end{itemize}
then $\pi_1(X)$ is given by  $$\quotient{\pi_1(A)*\pi_1(B)}{N},$$
where $N$ is the normal subgroup generated by all elements of the form 
$$i_{A*}[\gamma]i_{B*}[\gamma]^{-1}$$ 
for all $[\gamma]\in\pi_1(A\cap B)$. 
\end{theorem*}

\item Calculate the fundamental group of $S(K^2\wedgeprod T^2)$  where $S$ denotes suspension, $K^2$ is the Klein bottle, and $T^2$ is the torus. 

\answer	Let $X=K^2\wedgeprod T^2$. We can write $SX$ as $CX\cup CX$, where the two cones overlap a bit, say $t\in [0,0.6)$ in the first cone and $t\in(0.4,1]$ in the second cone. Applying Van Kampen's theorem, we find that $\pi_1(SX)$ is 
$$\quotient{\pi_1(CX)*\pi_1(CX)}{N}$$
but every cone is contractible, so $\pi_1(SX)$ is trivial. \qed

\newcommand{\D}{\mathbb{D}}
\item Let $X_n$ be the space obtained by attaching $\D^2$ to $\S^1$ along $\del\D^2$ by wrapping the boundary $n$ times around $\S^1$. 
	\begin{enumerate}[label=(\alph*)]
	\item Use the Van Kampen theorem to calculate $\pi_1(X_n)$. 
	
	\answer Write $X_n=A\cup B$, where $B$ is the interior of $\D^2$, and $A$ is constructed as follows. Take $\left(\bigwedgeprod_{k=1}^n [0,0.1)\right) \times I$ to yield an $n$-pointed asterisk noodle, then twist $\frac{1}{n}$th of a turn and identify the ends as shown below for $n=3$:
	\jpg{width=0.33\textwidth}{final-w20-p7-1}
	$A$ deformation retracts to $\S^1$ and $B$ is contractible, so $\pi_1(X_n)$ is 
	$$\quotient{\Z(a)*0}{N},\footnote{Here $\Z(a)$ denotes the free group generated by $a$.}$$
	so all that remains is to determine $N$. Since $A\cap B$ deformation retracts to the boundary circle of $A$, then $\pi_1(A\cap B)$ has one generator, call it $[\gamma]$. Then $i_{B*}[\gamma]$ is trivial, and $i_{A*}[\gamma]=[a^n]$, so $N=\langle a^n \rangle$ and 
	$$\pi_1(X_n)=\langle a \mid a^n \rangle = \Z_n.$$
	
	\item Find all subgroups of $\pi_1(X_{18})$ and all of the corresponding connected based covering spaces. The covers will be hard to draw but you should be able to describe them fairly
precisely.

	\answer To construct the universal cover $\tilde X$ of $X_{18}$, we build the Cayley complex: $\pi_1(X_{18})=\angles{a\mid a^{18}}=\Z_{18}$, so we begin with 18 discrete points $0, a, a^2, \dots, a^17$. Then we add an edge connecting each point to its image under the action of the generator $a$, forming an 18-sided polygon. Finally there is a loop $a^{18}$ starting at each point, so we add 18 2-cells attached along the polygon. Thus we obtain the an 18-gon with 18 2-cells attached inside. 
	
	Motivated by this picture, we can reimagine $X_{18}$ as the same edge $a$ repeated 18 times in a loop, with one $e^2$ attached along $a^{18}$. That is, an 18-gon with all edges identified. (In the figures, pretend that each polygon has 18 sides.)
	\vspace*{-6pt}
	\jpg{width=0.3\textwidth}{final-w20-p7-2}
	\vspace*{-12pt}
	Note that $p:\tilde{X}\to X$ is the map which is suggested by the images; a point in any of the edges of $\tilde X$ is mapped to the corresponding point in the only edge of $X_{18}$, and a point in any of the $e^2$ cells of $\tilde{X}$ maps to the corresponding point in the only $e^2$ of $X_{18}$. 
		
	The subgroups of $\angles{a\mid a^{18}}$ are
	
	$\hfill \angles{a\mid a^{18}} \hfill \angles{a^2\mid a^{18}} \hfill \angles{a^3\mid a^{18}} \hfill \angles{a^6\mid a^{18}} \hfill \angles{a^9\mid a^{18}} \hfill \angles{a^{18}\mid a^{18}} \hfill$
	
	and for each $\angles{a^n\mid a^{18}}$ we will construct a cover $C_n$ of $X_{18}$. 
	
	Clearly $C_1=X_{18}$ and $C_{18}=\tilde{X}$. For all other $C_n$, we start with $\tilde{X}$ and identify every pair of points $\tilde{x}, \tilde{y}\in \tilde{X}$ such that for any path $\tilde{\gamma}$ between them, $\tilde{\gamma}\xmapsto{p_*}\gamma\in \angles{a^n\mid a^{18}}$. For the edges, only powers of $a^n$ are loops, so we identify corresponding points in $\frac{18}{n}$ pieces of the perimeter. 
	\vspace*{-6pt}
	\jpg{width=0.15\textwidth}{final-w20-p7-3}
	\vspace*{-6pt}
	For the faces, let $\tilde{x}_n$ be 18 corresponding points on each side of $\tilde{X}$, and let $\tilde{y}_n$ be 18 corresponding points on each $e^2$, and connect them all with paths. Concatenating any pair of them give a path which maps to  a loop in $X_18$, but there are $n$ distinct groups such that any pair in the group maps to a loop in $\angles{a^n\mid a^{18}}$. 
	\jpg{width=0.48\textwidth}{final-w20-p7-4}
	Thus we find that $C_n$ is an $\frac{18}{n}$-gon with edges identified, and $n$ $e^2$ cells attached inside. \qed
	
	\item Draw the subgroup lattice (i.e. illustrate all of the inclusion relations between subgroups in a diagram).
	\jpg{width=0.5\textwidth}{final-w20-p7-5}
	\vspace*{-34pt} \qed
	\end{enumerate}

\item 
%\mbox{}\vspace*{-24pt}
%\begin{definition*}
%A \emph{symmetry} of a cover of a 1-dimensional cell complex is an automorphism which relabels the vertices of the graph and after relabeling, the two graphs remain the same.
%\end{definition*}
%
%\begin{definition*}
%We say a graph has \emph{maximal symmetry} if for every pair of vertices $a,b$ in the graph, there exists a symmetry which maps $a$ to $b$. 
%\end{definition*}
%
%\begin{proposition*}
%Let $X$ be a 1-dimensional cell complex. A subgroup $G$ of $\pi_1(X)$ is normal iff the graph of the cover generated by $G$ has maximal symmetry. 
%\end{proposition*}

Consider the cover of $\S^1\wedgeprod\S^1$ drawn below based at its bottom vertex. Is the image of its fundamental group in $ \pi_1(\S^1\wedgeprod\S^1)$ a normal subgroup? Why or why not?
\jpg{width=0.15\textwidth}{final-w20-p8-1}


\answer 
No, because the graph does not have maximal symmetry. Observe that if we label the vertices $1, 2, 3, 4$ from top to bottom, then no deck transformation maps $1\mapsto4$, since 1 has a $b$ arrow from itself to itself, and 4 does not. 
\qed

\item (Extra credit) Use the 2-dimensional Brouwer fixed point theorem to prove that every $3\times3$ matrix with positive real entries
has a positive real eigenvalue. [Hint: focus on how the matrix
acts on rays through the origin in $\R^3$, particularly those that point into the first octant.]

\begin{proof}
Let $A$ be a $3\times3$ matrix, $\vec{x}\in\R^3$, and define $\phi(\vec{x})$ to be the linear transformation $A\vec{x}$ restricted to $\S^2\cap O_1$, where $O_1$ is the first octant. Composing with projection back to $\S^2$ we obtain $p\circ \phi(\vec{x})$, a continuous function. Since $A\vec{x}$ gives a vector with all positive coordinates, $p\circ \phi:\S^2\cap O_1\to\S^2\cap O_1$. Observe that $\S^2\cap O_1\cong \D^2$, so by the Brouwer fixed point theorem, $p\circ \phi$ has a fixed point $\vec{x}_0$. Since $\phi(\vec{x})=A\vec{x}$ maps to a point in $O_1$ which projects to itself, then $\phi(\vec{x})=\lambda\vec{x}$ with $\lambda>0$ and $\vec{x}$ is an eigenvector of $A$. 
\end{proof}





\end{enumerate}
\end{document}



