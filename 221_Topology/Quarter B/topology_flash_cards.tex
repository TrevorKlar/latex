\documentclass{article}
\usepackage[paperwidth=.5\paperwidth,paperheight=.25\paperheight]{geometry}
\usepackage{pgfpages}
\pagestyle{empty}
\thispagestyle{empty}
\pgfpagesuselayout{8 on 1}[a4paper]
\makeatletter
\@tempcnta=1\relax
\loop\ifnum\@tempcnta<9\relax
\pgf@pset{\the\@tempcnta}{bordercode}{\pgfusepath{stroke}}
\advance\@tempcnta by 1\relax
\repeat
\makeatother

%% Useful packages
\usepackage{amssymb, amsmath, amsthm} 
%\usepackage{graphicx}  %%this is currently enabled in the default document, so it is commented out here. 
\usepackage{calrsfs}
\usepackage{braket}
\usepackage{mathtools}
\usepackage{lipsum}
\usepackage{tikz}
\usetikzlibrary{cd}
\usepackage{verbatim}
%\usepackage{ntheorem}% for theorem-like environments
\usepackage{mdframed}%can make highlighted boxes of text
%Use case: https://tex.stackexchange.com/questions/46828/how-to-highlight-important-parts-with-a-gray-background
\usepackage{wrapfig}
\usepackage{centernot}
\usepackage{subcaption}%\begin{subfigure}{0.5\textwidth}
\usepackage{pgfplots}
\pgfplotsset{compat=1.13}
\usepackage[colorinlistoftodos]{todonotes}
\usepackage[colorlinks=true, allcolors=blue]{hyperref}
\usepackage{xfrac}					%to make slanted fractions \sfrac{numerator}{denominator}
\usepackage{enumitem}            
    %syntax: \begin{enumerate}[label=(\alph*)]
    %possible arguments: f \alph*, \Alph*, \arabic*, \roman* and \Roman*
\usetikzlibrary{arrows,shapes.geometric,fit}

\DeclareMathAlphabet{\pazocal}{OMS}{zplm}{m}{n}
%% Use \pazocal{letter} to typeset a letter in the other kind 
%%  of math calligraphic font. 

%% This puts the QED block at the end of each proof, the way I like it. 
\renewenvironment{proof}{{\bfseries Proof}}{\qed}
\makeatletter
\renewenvironment{proof}[1][\bfseries \proofname]{\par
  \pushQED{\qed}%
  \normalfont \topsep6\p@\@plus6\p@\relax
  \trivlist
  %\itemindent\normalparindent
  \item[\hskip\labelsep
        \scshape
    #1\@addpunct{}]\ignorespaces
}{%
  \popQED\endtrivlist\@endpefalse
}
\makeatother

%% This adds a \rewnewtheorem command, which enables me to override the settings for theorems contained in this document.
\makeatletter
\def\renewtheorem#1{%
  \expandafter\let\csname#1\endcsname\relax
  \expandafter\let\csname c@#1\endcsname\relax
  \gdef\renewtheorem@envname{#1}
  \renewtheorem@secpar
}
\def\renewtheorem@secpar{\@ifnextchar[{\renewtheorem@numberedlike}{\renewtheorem@nonumberedlike}}
\def\renewtheorem@numberedlike[#1]#2{\newtheorem{\renewtheorem@envname}[#1]{#2}}
\def\renewtheorem@nonumberedlike#1{  
\def\renewtheorem@caption{#1}
\edef\renewtheorem@nowithin{\noexpand\newtheorem{\renewtheorem@envname}{\renewtheorem@caption}}
\renewtheorem@thirdpar
}
\def\renewtheorem@thirdpar{\@ifnextchar[{\renewtheorem@within}{\renewtheorem@nowithin}}
\def\renewtheorem@within[#1]{\renewtheorem@nowithin[#1]}
\makeatother

%% This makes theorems and definitions with names show up in bold, the way I like it. 
\makeatletter
\def\th@plain{%
  \thm@notefont{}% same as heading font
  \itshape % body font
}
\def\th@definition{%
  \thm@notefont{}% same as heading font
  \normalfont % body font
}
\makeatother

%===============================================
%==============Shortcut Commands================
%===============================================
\newcommand{\ds}{\displaystyle}
\newcommand{\B}{\mathcal{B}}
\newcommand{\C}{\mathbb{C}}
\newcommand{\F}{\mathbb{F}}
\newcommand{\N}{\mathbb{N}}
\newcommand{\R}{\mathbb{R}}
\newcommand{\Q}{\mathbb{Q}}
\newcommand{\T}{\mathcal{T}}
\newcommand{\Z}{\mathbb{Z}}
\renewcommand\qedsymbol{$\blacksquare$}
\newcommand{\qedwhite}{\hfill\ensuremath{\square}}
\newcommand*\conj[1]{\overline{#1}}
\newcommand*\closure[1]{\overline{#1}}
\newcommand*\mean[1]{\overline{#1}}
%\newcommand{\inner}[1]{\left< #1 \right>}
\newcommand{\inner}[2]{\left< #1, #2 \right>}
\newcommand{\powerset}[1]{\pazocal{P}(#1)}
%% Use \pazocal{letter} to typeset a letter in the other kind 
%%  of math calligraphic font. 
\newcommand{\cardinality}[1]{\left| #1 \right|}
\newcommand{\domain}[1]{\mathcal{D}(#1)}
\newcommand{\image}{\text{Im}}
\newcommand{\inv}[1]{#1^{-1}}
\newcommand{\preimage}[2]{#1^{-1}\left(#2\right)}
\newcommand{\script}[1]{\mathcal{#1}}


\newenvironment{highlight}{\begin{mdframed}[backgroundcolor=gray!20]}{\end{mdframed}}

\DeclarePairedDelimiter\ceil{\lceil}{\rceil}
\DeclarePairedDelimiter\floor{\lfloor}{\rfloor}

%===============================================
%===============My Tikz Commands================
%===============================================
\newcommand{\drawsquiggle}[1]{\draw[shift={(#1,0)}] (.005,.05) -- (-.005,.02) -- (.005,-.02) -- (-.005,-.05);}
\newcommand{\drawpoint}[2]{\draw[*-*] (#1,0.01) node[below, shift={(0,-.2)}] {#2};}
\newcommand{\drawopoint}[2]{\draw[o-o] (#1,0.01) node[below, shift={(0,-.2)}] {#2};}
\newcommand{\drawlpoint}[2]{\draw (#1,0.02) -- (#1,-0.02) node[below] {#2};}
\newcommand{\drawlbrack}[2]{\draw (#1+.01,0.02) --(#1,0.02) -- (#1,-0.02) -- (#1+.01,-0.02) node[below, shift={(-.01,0)}] {#2};}
\newcommand{\drawrbrack}[2]{\draw (#1-.01,0.02) --(#1,0.02) -- (#1,-0.02) -- (#1-.01,-0.02) node[below, shift={(+.01,0)}] {#2};}

%***********************************************
%**************Start of Document****************
%***********************************************
 %find me at /home/trevor/texmf/tex/latex/tskpreamble_nothms.tex

\newenvironment{flashcard}[2][]{%
\noindent  \textsc{#1}

\vfill 
\centerline{{\Large{#2}}}
\vfill
\newpage \vspace*{\stretch{1}} \noindent
}
{\vspace*{\stretch{1}}\newpage}

\usepackage[latin1]{inputenc}
\usepackage{amsfonts}
\usepackage{amsmath}

\let\oldphi\phi
\renewcommand{\phi}{\varphi}

\begin{document}


\begin{flashcard}[Definition]{CW-complex}
Anything that can be constructed with the following type of construction: 
\begin{itemize}
\item Start with any set of points $X^0$ with the discrete topology. 
\item Form $X^n= D^n_\alpha \sqcup_{\varphi_\alpha} X^{n-1}$ by attaching $n$-cells to the $(n-1)$-skeleton. 
\item If you go infinitely, use the weak topology; where $A{\undertext{\subset}{open}}{X}$ if $A{\undertext{\subset}{open}}{X^n}$ for all $n$. 
\end{itemize}
\end{flashcard}

\begin{flashcard}[Definition]{$g$ Homotopic to $h$ rel $A$}
$g\simeq h \rel A$ if $\exists$ a homotopy $F$ s.t. 
\begin{itemize}
\item $f_0 = g$
\item $f_1 = h$ 
\item $f_{t_1}(a)=f_{t_2}(a) \quad \forall a\in A$
\end{itemize}
\end{flashcard}

\begin{flashcard}[Definition]{Homotopy Equivalent $\rel A$}
$\exists f:X\to Y, g:Y\to X$ such that 
\begin{itemize}
\item $gf\simeq\id \rel A$
\item $fg\simeq\id \rel A$
\end{itemize}
\end{flashcard}

\begin{flashcard}[Definition]{Homotopy Extension Property}
The following are equivalent:
\begin{itemize}
\item \mbox{} \vspace*{-5ex}\begin{align*}
\forall F&:A\times I \to Y \text{ and }\\
f&: X \to Y \text{ s.t. } f \text{ extends } F_0,\\
\exists \bar{F}&: X\times I \text{ which extends } F \text{ and }f.
\end{align*}

\item $X\times \{0\} \cup A\times I$ is a retract of $X\times I$. 
\end{itemize}
\end{flashcard}

\begin{flashcard}[Definition]{Smash Product}
$$X \smashprod Y = \quotient{X\times Y}{X\wedgeprod Y},$$
where we wedge $X$ and $Y$ at their respective base points $x_0$, $y_0$, usually given. 
\end{flashcard}

\begin{flashcard}[Definition]{
\begin{tabular}{c}
Concatenation of Paths $f\sqcdot g$ \\ {\small (Product Path)}
\end{tabular}
}
Given two paths $f,g:I\to X$ such that $f(1)=g(0)$, then 
$$f\sqcdot g(s)=\begin{cases}
f(2s) & s\in [0,\frac{1}{2}] \\
g(2s-1) & s\in [\frac{1}{2},1] \\
\end{cases}$$
or in words, do $f$ then do $g$, but go twice as fast.
\end{flashcard}

\begin{flashcard}[Definition]{
\begin{tabular}{c}
Concatenation of Path Homotopies $F\sqcdot G$ \\ {\small (Not defined in Hatcher)}
\end{tabular}
}
Given homotopic paths $f_0\overtext{\simeq}{$F$} f_1$ and $g_0\overtext{\simeq}{$G$} g_1$ such that $f_s\sqcdot g_s$ is defined, then 
$$F\sqcdot G:=\begin{cases}
F(2s,t) & s\in [0,\frac{1}{2}] \\
G(2s-1,t) & s\in [\frac{1}{2},1] \\
\end{cases}$$
or in words, apply $F$ in the first region, and $G$ in the second. 
\end{flashcard}

\begin{flashcard}[Definition]{
\begin{tabular}{c}
Composition of Path Homotopies $F\sqcdotwhite F'$ \\ {\small (Not defined in Hatcher)}
\end{tabular}
}
Given homotopic paths $f_0\overtext{\simeq}{$F$} f_1 \overtext{\simeq}{$F'$} f_2$, we can compose the homotopies by 
$$F\sqcdot F'(s,t)=\begin{cases}
F(s,2t) & t\in [0,\frac{1}{2}] \\
F'(s,2t-1) & t\in [\frac{1}{2},1] \\
\end{cases}$$
That is, smoothly change $f_1$ into $f_2$ via $F'$, then change $f_2$ into $f_3$ via $F'$. 
\end{flashcard}

\begin{flashcard}[Definition]{Simply Connected Space}
\begin{itemize}
\item $X$ is path-connected
\item $\pi_1(X)=0$, that is, the fundamental group is the trivial group. 
\end{itemize}
\end{flashcard}

\begin{flashcard}[Theorem]{If $X$ is path-connected, then $\pi_1(X)$...}
is independent of basepoint, since the change-of-basepoint homomorphism is an isomorphism. 
\end{flashcard}

\begin{flashcard}[Theorem]{
\begin{tabular}{c}
If $\phi:X\to Y$ is a homotopy equivalence map, \\
what can we say about $\pi_1$?
\end{tabular}
}
$\phi_*$ is an isomorphism, so $\pi_1(X)\cong\pi_1(Y)$. 
\end{flashcard}

\begin{flashcard}[Theorem]{
\begin{tabular}{c}
If $X$ retracts to $A$, 
what can we say about $\pi_1$? \\
What if it deformation retracts?
\end{tabular}
}
$\iota_*$ is injective, so $\pi_1(A)\undertext{\subset}{group}\pi_1(X)$ up to isomorphism.\\
$X\simeq A$, so  $\pi_1(X)\cong\pi_1(A)$. 
\end{flashcard}

\begin{flashcard}[Definition]{Covering Space}

\end{flashcard}

\begin{flashcard}[Definition]{Evenly Covered}

\end{flashcard}

\end{document}