 \documentclass[12pt,letterpaper]{article}

\usepackage{fancyhdr,fancybox,tensor}

%% Useful packages
\usepackage{amssymb, amsmath, amsthm} 
%\usepackage{graphicx}  %%this is currently enabled in the default document, so it is commented out here. 
\usepackage{calrsfs}
\usepackage{braket}
\usepackage{mathtools}
\usepackage{lipsum}
\usepackage{tikz}
\usetikzlibrary{cd}
\usepackage{verbatim}
%\usepackage{ntheorem}% for theorem-like environments
\usepackage{mdframed}%can make highlighted boxes of text
%Use case: https://tex.stackexchange.com/questions/46828/how-to-highlight-important-parts-with-a-gray-background
\usepackage{wrapfig}
\usepackage{centernot}
\usepackage{subcaption}%\begin{subfigure}{0.5\textwidth}
\usepackage{pgfplots}
\pgfplotsset{compat=1.13}
\usepackage[colorinlistoftodos]{todonotes}
\usepackage[colorlinks=true, allcolors=blue]{hyperref}
\usepackage{xfrac}					%to make slanted fractions \sfrac{numerator}{denominator}
\usepackage{enumitem}            
    %syntax: \begin{enumerate}[label=(\alph*)]
    %possible arguments: f \alph*, \Alph*, \arabic*, \roman* and \Roman*
\usetikzlibrary{arrows,shapes.geometric,fit}

\DeclareMathAlphabet{\pazocal}{OMS}{zplm}{m}{n}
%% Use \pazocal{letter} to typeset a letter in the other kind 
%%  of math calligraphic font. 

%% This puts the QED block at the end of each proof, the way I like it. 
\renewenvironment{proof}{{\bfseries Proof}}{\qed}
\makeatletter
\renewenvironment{proof}[1][\bfseries \proofname]{\par
  \pushQED{\qed}%
  \normalfont \topsep6\p@\@plus6\p@\relax
  \trivlist
  %\itemindent\normalparindent
  \item[\hskip\labelsep
        \scshape
    #1\@addpunct{}]\ignorespaces
}{%
  \popQED\endtrivlist\@endpefalse
}
\makeatother

%% This adds a \rewnewtheorem command, which enables me to override the settings for theorems contained in this document.
\makeatletter
\def\renewtheorem#1{%
  \expandafter\let\csname#1\endcsname\relax
  \expandafter\let\csname c@#1\endcsname\relax
  \gdef\renewtheorem@envname{#1}
  \renewtheorem@secpar
}
\def\renewtheorem@secpar{\@ifnextchar[{\renewtheorem@numberedlike}{\renewtheorem@nonumberedlike}}
\def\renewtheorem@numberedlike[#1]#2{\newtheorem{\renewtheorem@envname}[#1]{#2}}
\def\renewtheorem@nonumberedlike#1{  
\def\renewtheorem@caption{#1}
\edef\renewtheorem@nowithin{\noexpand\newtheorem{\renewtheorem@envname}{\renewtheorem@caption}}
\renewtheorem@thirdpar
}
\def\renewtheorem@thirdpar{\@ifnextchar[{\renewtheorem@within}{\renewtheorem@nowithin}}
\def\renewtheorem@within[#1]{\renewtheorem@nowithin[#1]}
\makeatother

%% This makes theorems and definitions with names show up in bold, the way I like it. 
\makeatletter
\def\th@plain{%
  \thm@notefont{}% same as heading font
  \itshape % body font
}
\def\th@definition{%
  \thm@notefont{}% same as heading font
  \normalfont % body font
}
\makeatother

%===============================================
%==============Shortcut Commands================
%===============================================
\newcommand{\ds}{\displaystyle}
\newcommand{\B}{\mathcal{B}}
\newcommand{\C}{\mathbb{C}}
\newcommand{\F}{\mathbb{F}}
\newcommand{\N}{\mathbb{N}}
\newcommand{\R}{\mathbb{R}}
\newcommand{\Q}{\mathbb{Q}}
\newcommand{\T}{\mathcal{T}}
\newcommand{\Z}{\mathbb{Z}}
\renewcommand\qedsymbol{$\blacksquare$}
\newcommand{\qedwhite}{\hfill\ensuremath{\square}}
\newcommand*\conj[1]{\overline{#1}}
\newcommand*\closure[1]{\overline{#1}}
\newcommand*\mean[1]{\overline{#1}}
%\newcommand{\inner}[1]{\left< #1 \right>}
\newcommand{\inner}[2]{\left< #1, #2 \right>}
\newcommand{\powerset}[1]{\pazocal{P}(#1)}
%% Use \pazocal{letter} to typeset a letter in the other kind 
%%  of math calligraphic font. 
\newcommand{\cardinality}[1]{\left| #1 \right|}
\newcommand{\domain}[1]{\mathcal{D}(#1)}
\newcommand{\image}{\text{Im}}
\newcommand{\inv}[1]{#1^{-1}}
\newcommand{\preimage}[2]{#1^{-1}\left(#2\right)}
\newcommand{\script}[1]{\mathcal{#1}}


\newenvironment{highlight}{\begin{mdframed}[backgroundcolor=gray!20]}{\end{mdframed}}

\DeclarePairedDelimiter\ceil{\lceil}{\rceil}
\DeclarePairedDelimiter\floor{\lfloor}{\rfloor}

%===============================================
%===============My Tikz Commands================
%===============================================
\newcommand{\drawsquiggle}[1]{\draw[shift={(#1,0)}] (.005,.05) -- (-.005,.02) -- (.005,-.02) -- (-.005,-.05);}
\newcommand{\drawpoint}[2]{\draw[*-*] (#1,0.01) node[below, shift={(0,-.2)}] {#2};}
\newcommand{\drawopoint}[2]{\draw[o-o] (#1,0.01) node[below, shift={(0,-.2)}] {#2};}
\newcommand{\drawlpoint}[2]{\draw (#1,0.02) -- (#1,-0.02) node[below] {#2};}
\newcommand{\drawlbrack}[2]{\draw (#1+.01,0.02) --(#1,0.02) -- (#1,-0.02) -- (#1+.01,-0.02) node[below, shift={(-.01,0)}] {#2};}
\newcommand{\drawrbrack}[2]{\draw (#1-.01,0.02) --(#1,0.02) -- (#1,-0.02) -- (#1-.01,-0.02) node[below, shift={(+.01,0)}] {#2};}

%***********************************************
%**************Start of Document****************
%***********************************************
 %find me at /home/trevor/texmf/tex/latex/tskpreamble_nothms.tex
%===============================================
%===============Theorem Styles==================
%===============================================

%================Default Style==================
\theoremstyle{plain}% is the default. it sets the text in italic and adds extra space above and below the \newtheorems listed below it in the input. it is recommended for theorems, corollaries, lemmas, propositions, conjectures, criteria, and (possibly; depends on the subject area) algorithms.
\newtheorem{theorem}{Theorem}
\numberwithin{theorem}{section} %This sets the numbering system for theorems to number them down to the {argument} level. I have it set to number down to the {section} level right now.
\newtheorem*{theorem*}{Theorem} %Theorem with no numbering
\newtheorem{corollary}[theorem]{Corollary}
\newtheorem*{corollary*}{Corollary}
\newtheorem{conjecture}[theorem]{Conjecture}
\newtheorem{lemma}[theorem]{Lemma}
\newtheorem*{lemma*}{Lemma}
\newtheorem{proposition}[theorem]{Proposition}
\newtheorem*{proposition*}{Proposition}
\newtheorem{problemstatement}[theorem]{Problem Statement}


%==============Definition Style=================
\theoremstyle{definition}% adds extra space above and below, but sets the text in roman. it is recommended for definitions, conditions, problems, and examples; i've alse seen it used for exercises.
\newtheorem{definition}[theorem]{Definition}
\newtheorem*{definition*}{Definition}
\newtheorem{condition}[theorem]{Condition}
\newtheorem{problem}[theorem]{Problem}
\newtheorem{example}[theorem]{Example}
\newtheorem*{example*}{Example}
\newtheorem*{counterexample*}{Counterexample}
\newtheorem*{romantheorem*}{Theorem} %Theorem with no numbering
\newtheorem{exercise}{Exercise}
\numberwithin{exercise}{section}
\newtheorem{algorithm}[theorem]{Algorithm}

%================Remark Style===================
\theoremstyle{remark}% is set in roman, with no additional space above or below. it is recommended for remarks, notes, notation, claims, summaries, acknowledgments, cases, and conclusions.
\newtheorem{remark}[theorem]{Remark}
\newtheorem*{remark*}{Remark}
\newtheorem{notation}[theorem]{Notation}
\newtheorem*{notation*}{Notation}
%\newtheorem{claim}[theorem]{Claim}  %%use this if you ever want claims to be numbered
\newtheorem*{claim}{Claim}


%%
%% Page set-up:
%%
\pagestyle{empty}
\lhead{\textsc{221c - Differential Topology} \\Quarter of COVID-19} 
\rhead{\textsc{Long, Spring 2020} \\ Trevor Klar}
%\chead{\Large\textbf{A New Integration Technique \\ }}
\renewcommand{\headrulewidth}{0pt}
%
\renewcommand{\footrulewidth}{0pt}
%\lfoot{
%Office: \quad \quad \, M 2-3 \, \, SH 6431x \\
%Math Lab: \, W 12-2 \, SH 1607
%}
%\rfoot{trevorklar@math.ucsb.edu}

\setlength{\parindent}{0in}
\setlength{\textwidth}{7in}
\setlength{\evensidemargin}{-0.25in}
\setlength{\oddsidemargin}{-0.25in}
\setlength{\parskip}{.5\baselineskip}
\setlength{\topmargin}{-0.5in}
\setlength{\textheight}{9in}

\setlist[enumerate,1]{label=\textbf{\arabic*.}}

\let\oldphi\phi
\renewcommand{\phi}{\oldphi}
\renewcommand{\epsilon}{\varepsilon}
\renewcommand{\H}{\mathbb{H}}

\begin{document}
\pagestyle{fancy}
\begin{center}
{\Large Midterm Exam}%=================UPDATE THIS=================%
\end{center}


\begin{enumerate}


\setcounter{enumi}{-1}
\item 
	\begin{enumerate}[label=(\alph*)]
	\item \mbox{}	\vspace*{-1.33\baselineskip}
	\begin{definition*}
	Let $S\subset \R^k$. We says that a map $\phi:S\to \R^q$ is \emph{smooth} if all partial derivatives (of all orders) of $\phi$ exist.
	\end{definition*}
	
	\item \mbox{}	\vspace*{-1.33\baselineskip}
	\begin{definition*}
	Let $X\subset\R^n$, and $x\in X$. A \emph{chart of $X$ near $x$} is a diffeomorphism $\phi$ between open sets $U\ni \preimage{\phi}{x}$ and $V\ni x$ where $U \subset \R^k$ (or $	\H^k$ in the case of manifolds with boundary), and $V\subset X$. 
	\end{definition*}		
	\begin{remark*}
	We generally assume $\phi(0)=x$, unless we have reason to do otherwise. 
	\end{remark*}
	\begin{definition*}
	Let $X\subset\R^n$. We say that $X$ is a \emph{smooth $k$-manifold with boundary} if every $x\in X$ has a chart $\phi:U\subset \H^k \to V\subset X$. 
	\end{definition*}
	
	\begin{remark*}
	Since any point $x$ with a chart from $\R^k$ also has a chart from the interior of $\H^k$ (just shift the domain up enough), then if we just say \emph{smooth manifold}, we mean a smooth manifold with boundary (whose boundary may or may not empty).
	\end{remark*}
	
	\item \mbox{}	\vspace*{-1.33\baselineskip}
	\begin{definition*}
	From calculus, the \emph{derivative of $f$ at $x$ in the direction of $v$} is 
	$$\lim_{t\to0}\frac{f(\vec x+t\vec v)-f(\vec x)}{t}$$
	\end{definition*}
	\begin{definition*}
	Let $X$ be a smooth $k$-manifold with $x\in X$, and assume that a chart $\phi$ has $\phi(0)=x$. We define \emph{the tangent space of $X$ at $x$} as 
	$$T_x(X)=\der\phi_0(\R^k),$$
	that is, the tangent space is the image of the derivative of the chart.
	\end{definition*}
	
	\item \mbox{}	\vspace*{-1.5\baselineskip}
	\begin{definition*}
	Let $f:X\xto{smooth}Y$ and let $y\in Y$. We say $y$ is a \emph{regular value} if, for every $x\in \preimage{f}{y}$, we have that $\der f_x$ is surjective. \qed
	\end{definition*}
	
	\end{enumerate}

\pagebreak
\item Suppose that $M^m\subset\R^n$ is a smooth manifold without boundary and that $h :
M \to \R$ is a smooth map for which $0$ is a regular value. Prove that
$\preimage{h}{[0,\infty)}$ is a manifold with boundary.
\begin{proof}
Let $y\geq 0$ and let $x=\preimage{h}{y}$. 

\jpg{width=0.5\linewidth}{221-midterm-p1-1}

\textsc{Case I:} If $y$ is strictly positive, then $h\in \preimage{h}{(0,\infty)}$, which is open in $M$, and $M$ is a manifold so it is locally diffeomorphic to $\R^m$.

This means there exists open sets $U,U'$ such that $x\in U\subset \preimage{h}{(0,\infty)}$ and $U'\subset \R^m$, and a chart $\phi:U'\to U$. As long as we choose $U$ so that $\diam{U'}<\infty$,\footnote{We can always do this, just restrict $\phi$ to the unit ball centered at $\preimage{\phi}{x}$.} we can choose $k$ large enough that $\phi'(\vec x)=\phi(\vec x-k \vec{e}_m)$ is a chart from $\tilde U\subset \H^k\to U'$. \qedwhite

\textsc{Case II:} If $y=0$, then $y$ is a regular value, so $\der h_x$ has rank 1, and $\ker \der h_x$ has dimension $(m-1)$. Let $T$ be an invertible linear transformation from $\ker \der h_x\to \R^{m-1}$, and extend it to one on all $\R^n$.\footnote{Recall that $M\subset \R^n$.} Then define 
\begin{align*}
H:M&\to \R^{m-1}\times \R \\
H(\xi) &= (T\xi, h(\xi)). 
\end{align*}

Now we can see that 
\begin{align*}
\der H_x(v) &= (Tv, \der h_x(v))
\end{align*}
which has rank $(m-1)+1=m$. Thus $\der h_x$ is an isomorphism, so by the Inverse Function Theorem there exist neighborhoods 
$$U\ni x, \quad V\ni (Tx, 0)$$
where $h$ is a diffeomorphism. 
\jpg{width=0.8\linewidth}{221-midterm-p1-2}
By intersecting $U\cap \preimage{f}{[0,\infty)}$ and $V\cap \H^m$ and observing that the two sets correspond under $H$, we obtain an open neighborhood of $x$ (with the subspace topology) which is diffeomorphic via $H$ to an open neighborhood in $\H^m$. \qedwhite

Thus in either case, we can produce a neighborhood of $x$ in $\preimage{f}{[0,\infty)}$ diffeomorphic to an open set in $\H^k$, so $\preimage{f}{[0,\infty)}$ is a $k$-manifold with boundary. 
\end{proof}



\pagebreak
\item Suppose that $f : X \to Y$ is a smooth map between compact manifolds
without boundary of the same dimension. Suppose that $y\in Y$ is a regular
value.

Show that $\preimage{f}{y}$ is a finite set $\braces{x_1, \dots x_n}$. Show further that there is an open neighborhood $V$ of $y$ so that $\preimage{f}{V}$ is a finite disjoint of open sets $\braces{U_1, \dots, U_n}$, so that each $U_i$ is a neighborhood of $x_i$ and each $U_i$ is mapped
diffeomorphically onto $V$ by $f$.
\begin{proof}

[Note that I have reversed the notation for $U$ and $V$.] Suppose that $\preimage{f}{y}$ is infinite, then there exists a sequence $\braces{x_n}_{n=1}^\infty\subset \preimage{f}{y}$. Since $X$ is compact, then it is sequentially compact so $x_n$ has a convergent subsequence $x_{n_k}\to \tilde{x}$. Since $f$ is continuous, then $\preimage{f}{\braces{y}}$ is closed, so $\tilde{x}\in \preimage{f}{y}$. 
\vspace*{-\baselineskip}
\jpg{width=0.5\linewidth}{221-midterm-p2-1}
\vspace*{-\baselineskip}
Now since $y$ is a regular point of $f$, then for all $x\in \preimage{f}{y}$, we have $\der f_x$ is surjective, and since 
\begin{align*}
\dim \ker \der f_x &= \dim Y - \dim \image{\der f_x}\\
&= 0,
\end{align*}
then $\der f_x$ is an isomorphism so by the Inverse Function Theorem, $f$ is a local diffeomorphism at each $x\in \preimage{f}{y}$. In particular $f$ is injective on some neighborhood $W$ of $\tilde{x}$, but since $x_{n_k}\to \tilde{x}$ then every neighborhood of $\tilde{x}$ contains some $x_{n_k}$ and $f(x_{n_k})=f(\tilde{x}) = y$, which contradicts that $f$ is injective on $W$. Thus $\preimage{f}{y}$ is finite, and from now on denote $\preimage{f}{y}=\braces{x_i}_{i=1}^n$. 

Next, since $f$ is a local diffeomorphism at each $x\in \preimage{f}{y}$, 
\vspace*{-0.5\baselineskip}
\jpg{width=0.5\linewidth}{221-midterm-p2-2}
\vspace*{-\baselineskip}
there exist $U'_i \ni x_i$ such that $f$ is a diffeomorphism on $U'_i$. Since every manifold is Hausdorff\footnote{Since it is locally diffeomorphic to $\R^n$ or $\H^n$}, then we can separate the finite set of points $\braces{x_i}$ by disjoint open sets $U''_i$, and $\braces{U'_i\cap U''_i}_{i=1}^n$ are disjoint open sets where $f$ is a diffeomorphism onto its image, but they may not all have the same image. So let 
$$V=\bigcap_{i=1}^n f\left(U'_i\cap U''_i\right)$$
and then $\preimage{f}{V}=\coprod_{i=1}^n U_i$ is a disjoint collection of open neighborhoods, one for each $x_i$, where each $U_i$ is diffeomorphic to $V$, as desired. 
\end{proof}

\pagebreak
\item Prove (a) that $O(n)=\braces{A\in M(n,\R) \mid A^\top\!A = I}$ is a manifold. (b) Compute its dimension and identify $T_I(O(n))$. 

	\begin{proof}\textbf{(a)} Let $f:M_n(\R)\to \Sigma_n(\R)$ be the map given by 
$$f(A)=A^\top A.$$
	We will show that (i) $M_n(\R)$\footnote{Where $M_n(\R)$ denotes the set of all real $n\times n$ matrices.} and $\Sigma_n(\R)$\footnote{Where $\Sigma_n(\R)$ denotes the set of symmetric $n\times n$ real matrices.} are both smooth manifolds, (ii) $f$ is a smooth map, and (iii) $O(n)=f^{-1}(I)$ with $I$ a regular value of $f$, and then we're done since the preimage of a regular point is a smooth manifold.

(i) As vector spaces, $M_n(\R)\cong \R^{n^2}$ and $\Sigma_n(\R)\cong\R^{t(n)}$ where $t(n)$ is the $n$th triangle number, so they are definitely smooth manifolds. 

(ii) Since the computations of $A^\top A$ just consist of multiplying and adding different elements of $A$, then the function $f$ is just a polynomial in $n^2$ variables, so it is smooth. 

(iii) We can see by inspection that $O(n)=f^{-1}(I)$, so let us show that $I$ is a regular value of $f$. Fix $A\in M_n(\R)$, and let's compute the derivative $\der f_A:T_A(M_n)\to T_{A^\top A}(\Sigma_n)$, and check that it is surjective whenever $A\in f^{-1}(I)$. 
\begin{align*}
\der f_A(B)&= \lim_{t\to0}\frac{f(A+tB)-f(A)}{t} \\
&= \lim_{t\to0}\frac{(A+tB)^\top (A+tB)-A^\top A}{t}\\
&= \lim_{t\to0}\tfrac{1}{t}\left(tB^\top A+tA^\top B+t^2B^\top B\right)\\
&= B^\top A+A^\top B
\end{align*}
For any $C\in \Sigma_n(\R)$\footnote{The tangent space to a vector space at any point is itself.} and $A\in\preimage{f}{I}$, if we can find $B$ such that $B^\top A+A^\top B=C$, then we're done. Observe that, since $A^\top =A^{-1}$, then if $B=\frac{1}{2}AC$, then 
$$B^\top A+A^\top B=C.$$
Thus $\der f_A$ is surjective for all $A\in\preimage{f}{I}$, so (a) is proved. 
	\qedwhitehere
	\end{proof}
	\newcommand{\antiSigma}{\reflectbox{$\Sigma$}}
	\begin{proof}\textbf{(b)}
	The kernel of $\der f_I$ gives us the desired information here, so let's compute it. For any $B\in M_n(\R)$, 
	$$\der f_I(B)=B^\top + B,$$
	so the kernel is the set of all antisymmetric matrices, the matrices such that 
	$$B=-B^\top$$
	which I denote $\antiSigma_n$. Observe that this is a vector space since $\antiSigma_n \subset M_n$ and for all $\lambda\in \R$, and $B,C\in \antiSigma_n$, 
	$$(B+C)^\top + (B+C) =  B^\top + B + C^\top + C = 0, \text{ and }$$
	$$(\lambda B)^\top	+ (\lambda B) = \lambda (B^\top	 + B) = \lambda (0) = 0.$$
	Since the diagonal entries are all zero, and each entry $b_{ij}$ for $i>j$ is determined by $b_{ji}$, then $\antiSigma_n$ is isomorphic to $\R^{t(n-1)}$, so $\dim\left(\antiSigma_n\right)=t(n-1)$ and $T_I(O(n))=\ker \der f_I=\antiSigma_n$.
	\end{proof}
\end{enumerate}



\end{document}
