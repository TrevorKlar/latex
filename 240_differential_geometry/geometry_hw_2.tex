\documentclass[12pt,letterpaper]{article}

\usepackage{fancyhdr,fancybox}

%% Useful packages
\usepackage{amssymb, amsmath, amsthm} 
%\usepackage{graphicx}  %%this is currently enabled in the default document, so it is commented out here. 
\usepackage{calrsfs}
\usepackage{braket}
\usepackage{mathtools}
\usepackage{lipsum}
\usepackage{tikz}
\usetikzlibrary{cd}
\usepackage{verbatim}
%\usepackage{ntheorem}% for theorem-like environments
\usepackage{mdframed}%can make highlighted boxes of text
%Use case: https://tex.stackexchange.com/questions/46828/how-to-highlight-important-parts-with-a-gray-background
\usepackage{wrapfig}
\usepackage{centernot}
\usepackage{subcaption}%\begin{subfigure}{0.5\textwidth}
\usepackage{pgfplots}
\pgfplotsset{compat=1.13}
\usepackage[colorinlistoftodos]{todonotes}
\usepackage[colorlinks=true, allcolors=blue]{hyperref}
\usepackage{xfrac}					%to make slanted fractions \sfrac{numerator}{denominator}
\usepackage{enumitem}            
    %syntax: \begin{enumerate}[label=(\alph*)]
    %possible arguments: f \alph*, \Alph*, \arabic*, \roman* and \Roman*
\usetikzlibrary{arrows,shapes.geometric,fit}

\DeclareMathAlphabet{\pazocal}{OMS}{zplm}{m}{n}
%% Use \pazocal{letter} to typeset a letter in the other kind 
%%  of math calligraphic font. 

%% This puts the QED block at the end of each proof, the way I like it. 
\renewenvironment{proof}{{\bfseries Proof}}{\qed}
\makeatletter
\renewenvironment{proof}[1][\bfseries \proofname]{\par
  \pushQED{\qed}%
  \normalfont \topsep6\p@\@plus6\p@\relax
  \trivlist
  %\itemindent\normalparindent
  \item[\hskip\labelsep
        \scshape
    #1\@addpunct{}]\ignorespaces
}{%
  \popQED\endtrivlist\@endpefalse
}
\makeatother

%% This adds a \rewnewtheorem command, which enables me to override the settings for theorems contained in this document.
\makeatletter
\def\renewtheorem#1{%
  \expandafter\let\csname#1\endcsname\relax
  \expandafter\let\csname c@#1\endcsname\relax
  \gdef\renewtheorem@envname{#1}
  \renewtheorem@secpar
}
\def\renewtheorem@secpar{\@ifnextchar[{\renewtheorem@numberedlike}{\renewtheorem@nonumberedlike}}
\def\renewtheorem@numberedlike[#1]#2{\newtheorem{\renewtheorem@envname}[#1]{#2}}
\def\renewtheorem@nonumberedlike#1{  
\def\renewtheorem@caption{#1}
\edef\renewtheorem@nowithin{\noexpand\newtheorem{\renewtheorem@envname}{\renewtheorem@caption}}
\renewtheorem@thirdpar
}
\def\renewtheorem@thirdpar{\@ifnextchar[{\renewtheorem@within}{\renewtheorem@nowithin}}
\def\renewtheorem@within[#1]{\renewtheorem@nowithin[#1]}
\makeatother

%% This makes theorems and definitions with names show up in bold, the way I like it. 
\makeatletter
\def\th@plain{%
  \thm@notefont{}% same as heading font
  \itshape % body font
}
\def\th@definition{%
  \thm@notefont{}% same as heading font
  \normalfont % body font
}
\makeatother

%===============================================
%==============Shortcut Commands================
%===============================================
\newcommand{\ds}{\displaystyle}
\newcommand{\B}{\mathcal{B}}
\newcommand{\C}{\mathbb{C}}
\newcommand{\F}{\mathbb{F}}
\newcommand{\N}{\mathbb{N}}
\newcommand{\R}{\mathbb{R}}
\newcommand{\Q}{\mathbb{Q}}
\newcommand{\T}{\mathcal{T}}
\newcommand{\Z}{\mathbb{Z}}
\renewcommand\qedsymbol{$\blacksquare$}
\newcommand{\qedwhite}{\hfill\ensuremath{\square}}
\newcommand*\conj[1]{\overline{#1}}
\newcommand*\closure[1]{\overline{#1}}
\newcommand*\mean[1]{\overline{#1}}
%\newcommand{\inner}[1]{\left< #1 \right>}
\newcommand{\inner}[2]{\left< #1, #2 \right>}
\newcommand{\powerset}[1]{\pazocal{P}(#1)}
%% Use \pazocal{letter} to typeset a letter in the other kind 
%%  of math calligraphic font. 
\newcommand{\cardinality}[1]{\left| #1 \right|}
\newcommand{\domain}[1]{\mathcal{D}(#1)}
\newcommand{\image}{\text{Im}}
\newcommand{\inv}[1]{#1^{-1}}
\newcommand{\preimage}[2]{#1^{-1}\left(#2\right)}
\newcommand{\script}[1]{\mathcal{#1}}


\newenvironment{highlight}{\begin{mdframed}[backgroundcolor=gray!20]}{\end{mdframed}}

\DeclarePairedDelimiter\ceil{\lceil}{\rceil}
\DeclarePairedDelimiter\floor{\lfloor}{\rfloor}

%===============================================
%===============My Tikz Commands================
%===============================================
\newcommand{\drawsquiggle}[1]{\draw[shift={(#1,0)}] (.005,.05) -- (-.005,.02) -- (.005,-.02) -- (-.005,-.05);}
\newcommand{\drawpoint}[2]{\draw[*-*] (#1,0.01) node[below, shift={(0,-.2)}] {#2};}
\newcommand{\drawopoint}[2]{\draw[o-o] (#1,0.01) node[below, shift={(0,-.2)}] {#2};}
\newcommand{\drawlpoint}[2]{\draw (#1,0.02) -- (#1,-0.02) node[below] {#2};}
\newcommand{\drawlbrack}[2]{\draw (#1+.01,0.02) --(#1,0.02) -- (#1,-0.02) -- (#1+.01,-0.02) node[below, shift={(-.01,0)}] {#2};}
\newcommand{\drawrbrack}[2]{\draw (#1-.01,0.02) --(#1,0.02) -- (#1,-0.02) -- (#1-.01,-0.02) node[below, shift={(+.01,0)}] {#2};}

%***********************************************
%**************Start of Document****************
%***********************************************
 %find me at /home/trevor/texmf/tex/latex/tskpreamble_nothms.tex
%===============================================
%===============Theorem Styles==================
%===============================================

%================Default Style==================
\theoremstyle{plain}% is the default. it sets the text in italic and adds extra space above and below the \newtheorems listed below it in the input. it is recommended for theorems, corollaries, lemmas, propositions, conjectures, criteria, and (possibly; depends on the subject area) algorithms.
\newtheorem{theorem}{Theorem}
\numberwithin{theorem}{section} %This sets the numbering system for theorems to number them down to the {argument} level. I have it set to number down to the {section} level right now.
\newtheorem*{theorem*}{Theorem} %Theorem with no numbering
\newtheorem{corollary}[theorem]{Corollary}
\newtheorem*{corollary*}{Corollary}
\newtheorem{conjecture}[theorem]{Conjecture}
\newtheorem{lemma}[theorem]{Lemma}
\newtheorem*{lemma*}{Lemma}
\newtheorem{proposition}[theorem]{Proposition}
\newtheorem*{proposition*}{Proposition}
\newtheorem{problemstatement}[theorem]{Problem Statement}


%==============Definition Style=================
\theoremstyle{definition}% adds extra space above and below, but sets the text in roman. it is recommended for definitions, conditions, problems, and examples; i've alse seen it used for exercises.
\newtheorem{definition}[theorem]{Definition}
\newtheorem*{definition*}{Definition}
\newtheorem{condition}[theorem]{Condition}
\newtheorem{problem}[theorem]{Problem}
\newtheorem{example}[theorem]{Example}
\newtheorem*{example*}{Example}
\newtheorem*{counterexample*}{Counterexample}
\newtheorem*{romantheorem*}{Theorem} %Theorem with no numbering
\newtheorem{exercise}{Exercise}
\numberwithin{exercise}{section}
\newtheorem{algorithm}[theorem]{Algorithm}

%================Remark Style===================
\theoremstyle{remark}% is set in roman, with no additional space above or below. it is recommended for remarks, notes, notation, claims, summaries, acknowledgments, cases, and conclusions.
\newtheorem{remark}[theorem]{Remark}
\newtheorem*{remark*}{Remark}
\newtheorem{notation}[theorem]{Notation}
\newtheorem*{notation*}{Notation}
%\newtheorem{claim}[theorem]{Claim}  %%use this if you ever want claims to be numbered
\newtheorem*{claim}{Claim}


\DeclareMathOperator{\diameter}{diam}
\newcommand{\diam}[1]{\diameter\left(#1\right)}
\newcommand{\CP}{\C\mathbb{P}}
\let\temp\phi
\let\phi\varphi
\let\varphi\temp
\newcommand{\id}{\text{id}}

%%
%% Page set-up:
%%
\pagestyle{empty}
\lhead{\textsc{240 - Differential Geometry \\}} %=================UPDATE THIS=================%
\rhead{\textsc{Dai, Fall 2019 \\ Trevor Klar}}
%\chead{\Large\textbf{A New Integration Technique \\ }}
\renewcommand{\headrulewidth}{0pt}
%
\renewcommand{\footrulewidth}{0pt}
%\lfoot{
%Office: \quad \quad \, M 2-3 \, \, SH 6431x \\
%Math Lab: \, W 12-2 \, SH 1607
%}
%\rfoot{trevorklar@math.ucsb.edu}


\setlength{\parindent}{0in}
\setlength{\textwidth}{7in}
\setlength{\evensidemargin}{-0.25in}
\setlength{\oddsidemargin}{-0.25in}
\setlength{\parskip}{.5\baselineskip}
\setlength{\topmargin}{-0.5in}
\setlength{\textheight}{9in}

\setlist[enumerate,1]{label=\textbf{\arabic*.}}

\begin{document}
\pagestyle{fancy}
\begin{center}
{\Large Homework 2}%=================UPDATE THIS=================%
\end{center}

\begin{enumerate}
\item Let $\pi;\C^{n+1}-\{0\}\to\C\mathbb{P}^n$ be the quotient map defined by 
$$\pi(\vec{z})=\{w(z^1, \dots, z^n, z^{n+1}) | w\in\C\} = [z^1, \dots, z^{n+1}]=[\vec{z}].$$
We will show that $\CP^n$ is a smooth compact topological manifold. First we will show that $\CP^n$ is locally Euclidean by producing a smooth atlas on it. Let $\tilde{U}_i=\{\vec{z}\in\C^{n+1}|z^i\neq0\}$, and let $U_i=\pi\left(\tilde{U}_i\right)$. So an element of $U_i\subset \CP^n$ is of the form $[z^1, \dots, z^i\neq0, \dots, z^{n+1}]$. Then we define a map $\tilde{\phi}_i:U_i\to\C^n$ by 
$$\tilde{\phi}_i[\vec{z}]=\left(\frac{z^1}{z^i}, \dots, \hat{z}^i, \dots, \frac{z^{n+1}}{z^i} \right)_,$$
where hat denotes a removed quantity. 
We can see that $\tilde{\phi}_i$ is well-defined because taking $\tilde{\phi}(w\vec{z})$ just gives a multiple of $w\inv{w}$ in each coordinate. Finally we push forward the map to $\R^{2n}$ in the obvious way: let 
$$\phi_i=\rho\circ\tilde{\phi}_i,$$ 
where $\rho(x^1+iy^1, \dots, x^n+iy^n)=(x^1, y^1, \dots, x^n, y^n)$. 
Now $\{\left(U_i,\phi_i\right)\}$ is an atlas on $\CP^n$. 
To see this, observe that $\{U_i\}$ covers $C^{n+1}$ and each $\phi_i$ is continuous (since $z_i\neq0$) with continuous inverse $\inv{\tilde{\phi}}\circ\inv{\rho}$, where $\inv{\tilde{\phi}}_i:\C^n\to U_i$ is given by 
$$\inv{\tilde{\phi}}_i(\vec{w})=\left[w^1, \dots, w^{i-1}, 1, w^{i+1},\dots, w^n\right],$$
so $\CP^n$ is locally Euclidean. \qedwhite

If $\pi$ is open, then we can show that $\CP^n$ is second countable and Hausdorff. Let $U\subset\CP^n$ be open, then $\pi(U)=\{\xi U | \xi \in \C\}$, so $\inv{\pi}(\pi(U))=\{\xi U | \xi \in \C\}$, which is open. Thus $\pi$ is open. Since $\R^{2n+2}\cong C^{n+1} \supset \left(\C^{n+1}-\{\vec{0}\}\right)$ and $\R^{2n+2}$ is second countable, then so is $\pi\left(\C^{n+1}-\{\vec{0}\}\right)=\CP^n$. 

Now we show that $\CP^n$ is Hausdorff. To do this, let 
$$R=\left\lbrace(\vec{z},\vec{w})  \vert \pi(\vec{z})=\pi(\vec{w})\right\rbrace,$$
where $\vec{z},\vec{w} \in\left(\C^{n+1}-\{\vec{0}\}\right)$, and if $R$ is closed, then $\pi\left(\C^{n+1}-\{\vec{0}\}\right)=\CP^n$ is Hausdorff. Consider $(\vec{z},\vec{w})\in R$. Since $[\vec{z}]=[\vec{w}]$, then there exists some $\xi\in\C$ such that $\xi\vec{z}=\vec{w}$, that is, $z^iw^j=z^jw^i$ for each $i,j=1,\dots n+1$. Now let 
$$f(\vec{z},\vec{w})=\sum_{1\leq i,j \leq n+1}\norm{w^iz^j-z^iw^j}^2$$
and we find that $f$ is a continuous function which vanishes precisely on $R$. Thus since $\{0\}$ is closed, then so is $\inv{f}(\{0\})=R$ and $\CP^n$ is Hausdorff. \qedwhite

\pagebreak
The fact that $\pi$ is open also gives us that $\CP^n$ is compact. To see this, note that it is also continuous since $\inv{\pi}[\vec{z}]=[\vec{z}]$, where we think of $[\vec{z}]$ as an equivalence class on the left hand side and a set on the right hand side. This means that for an open subset of $\CP^n$, call it $U=\{[\vec{z}]| \vec{z}\in\Gamma\}$ for $\Gamma$ an indexing set, $\inv{\pi}(U)=\bigcup_\Gamma[\vec{z}]$. For simplicity, we can simply write $\inv{\pi}(U)=U$, with the understanding that $U\in\CP^n$ is a collection of equivalence classes and $U\in C^{n+1}-\{\vec{0}\}$ consists of all the elements of those equivalence classes. Now denote the box 
$$Q_\C^{n+1}=\{(w^1, \dots, w^{n+1}) \quad|\quad \real{w^i}, \imaginary{w^i} \in [-1,1] \quad \forall i\in1, \dots n+1\},$$
and observe that $Q_\C^{n+1}\subset \C^{n+1}-\{0\}$ is compact. 
For any open cover $\arbcoll{U}$ of $\CP^n$, 
$$\preimage{\pi}{\arbcoll{U}}=\arbcoll{U},$$
which covers $Q_\C^{n+1}$. Since $Q_\C^{n+1}$ is compact, we can produce a finite subcover $\{U_{\alpha_i}\}_{i=1}^n$, and $\pi\left(\{U_{\alpha_i}\}_{i=1}^n\right)=\{U_{\alpha_i}\}_{i=1}^n$ which is open and covers $\CP^n$. To see this, let $[\vec{z}]\in\CP^n$ and observe that $[\vec{z}]=\frac{1}{\delta}[\vec{z}]\in\Q_\C^{n+1}$, where $\delta=\max_i(\abs{\real{z^i}}, \abs{\imaginary{z^i}})$. Thus $\{U_{\alpha_i}\}_{i=1}^n$ covers $\CP^n$, so it is compact. \qedwhite

Thus we have shown that $\CP^n$ is a compact topological manifold. It only remains to be shown that it is a smooth manifold, that is, that the atlas we constructed is a smooth atlas. Let's check that two arbitrary charts $(U_i, \phi_i)$ and $(U_j, \phi_j)$ are compatible. Clearly $\phi_j\circ\inv{\phi_i}=\rho\circ\tilde{\phi}_j\circ\inv{\tilde{\phi}_i}\circ\inv{\rho}$ is smooth if and only if $\tilde{\phi}_j\circ\inv{\tilde{\phi}_i}$ is, so we will check the latter. 
$$\tilde{\phi}_j\circ\inv{\tilde{\phi}_i} (z^1, \dots, z^n)= \left(\frac{z^1}{z^j},\dots,\widehat{z^j},\dots,\frac{z^{i-1}}{z^j}, \frac{1}{z^j}, \frac{z^{i+1}}{z^j} \dots, \frac{z^n}{z^j}\right),$$
which is smooth since $z^j\neq0$ on $U_j$, and we are done. \qed


\item Let $f:\R\to\R$ be defined by 
$$f(x)=
\begin{cases}
1, &x\geq 0,\\
0, &x<0.
\end{cases}$$
Consider the charts $(\R,\id)$ on the domain and $(B_\frac{1}{2}(1), \id), (B_\frac{1}{2}(0), \id)$ on the range. 
\jpg{width=0.7\textwidth}{geo_hw2-2}
For any $x\in\R$ in the domain of $f$, $f(x)=0,1$, so suppose $f(x)=1$. Then $(\R,\id)$ contains $x$ and $(B_\frac{1}{2}(1), \id)$ contains $f(x)$, and 
\begin{align*}
\id\left(\R\cap\preimage{f}{B_\frac{1}{2}(1)}\right) &= \left(\R\cap[0,\infty)\right)\\
&= [0,\infty),
\end{align*}
and $\id\circ f\circ \id =f$ which is constantly $1$ on $[0,\infty)$, so it is smooth. Similarly, for $x$ such that $f(x)=0$, we can use the other chart which contains $0$ and we find that $\id\circ f\circ \id =f$ is constant in that case as well. \qedwhite

However, $f$ is not a smooth function from $\R\to\R$, because for $x=0$ in the domain, there is no chart $(U,\phi)\ni x$ such that $f\circ\inv{\phi}$ is smooth because $f$ is discontinuous at 0 (the choice of chart for the range doesn't help with this). \qed

\end{enumerate}

\end{document}


%%% Local Variables: 
%%% mode: latex
%%% TeX-master: t
%%% End: 
