\documentclass[12pt,letterpaper]{article}

\usepackage{fancyhdr,fancybox}

%% Useful packages
\usepackage{amssymb, amsmath, amsthm} 
%\usepackage{graphicx}  %%this is currently enabled in the default document, so it is commented out here. 
\usepackage{calrsfs}
\usepackage{braket}
\usepackage{mathtools}
\usepackage{lipsum}
\usepackage{tikz}
\usetikzlibrary{cd}
\usepackage{verbatim}
%\usepackage{ntheorem}% for theorem-like environments
\usepackage{mdframed}%can make highlighted boxes of text
%Use case: https://tex.stackexchange.com/questions/46828/how-to-highlight-important-parts-with-a-gray-background
\usepackage{wrapfig}
\usepackage{centernot}
\usepackage{subcaption}%\begin{subfigure}{0.5\textwidth}
\usepackage{pgfplots}
\pgfplotsset{compat=1.13}
\usepackage[colorinlistoftodos]{todonotes}
\usepackage[colorlinks=true, allcolors=blue]{hyperref}
\usepackage{xfrac}					%to make slanted fractions \sfrac{numerator}{denominator}
\usepackage{enumitem}            
    %syntax: \begin{enumerate}[label=(\alph*)]
    %possible arguments: f \alph*, \Alph*, \arabic*, \roman* and \Roman*
\usetikzlibrary{arrows,shapes.geometric,fit}

\DeclareMathAlphabet{\pazocal}{OMS}{zplm}{m}{n}
%% Use \pazocal{letter} to typeset a letter in the other kind 
%%  of math calligraphic font. 

%% This puts the QED block at the end of each proof, the way I like it. 
\renewenvironment{proof}{{\bfseries Proof}}{\qed}
\makeatletter
\renewenvironment{proof}[1][\bfseries \proofname]{\par
  \pushQED{\qed}%
  \normalfont \topsep6\p@\@plus6\p@\relax
  \trivlist
  %\itemindent\normalparindent
  \item[\hskip\labelsep
        \scshape
    #1\@addpunct{}]\ignorespaces
}{%
  \popQED\endtrivlist\@endpefalse
}
\makeatother

%% This adds a \rewnewtheorem command, which enables me to override the settings for theorems contained in this document.
\makeatletter
\def\renewtheorem#1{%
  \expandafter\let\csname#1\endcsname\relax
  \expandafter\let\csname c@#1\endcsname\relax
  \gdef\renewtheorem@envname{#1}
  \renewtheorem@secpar
}
\def\renewtheorem@secpar{\@ifnextchar[{\renewtheorem@numberedlike}{\renewtheorem@nonumberedlike}}
\def\renewtheorem@numberedlike[#1]#2{\newtheorem{\renewtheorem@envname}[#1]{#2}}
\def\renewtheorem@nonumberedlike#1{  
\def\renewtheorem@caption{#1}
\edef\renewtheorem@nowithin{\noexpand\newtheorem{\renewtheorem@envname}{\renewtheorem@caption}}
\renewtheorem@thirdpar
}
\def\renewtheorem@thirdpar{\@ifnextchar[{\renewtheorem@within}{\renewtheorem@nowithin}}
\def\renewtheorem@within[#1]{\renewtheorem@nowithin[#1]}
\makeatother

%% This makes theorems and definitions with names show up in bold, the way I like it. 
\makeatletter
\def\th@plain{%
  \thm@notefont{}% same as heading font
  \itshape % body font
}
\def\th@definition{%
  \thm@notefont{}% same as heading font
  \normalfont % body font
}
\makeatother

%===============================================
%==============Shortcut Commands================
%===============================================
\newcommand{\ds}{\displaystyle}
\newcommand{\B}{\mathcal{B}}
\newcommand{\C}{\mathbb{C}}
\newcommand{\F}{\mathbb{F}}
\newcommand{\N}{\mathbb{N}}
\newcommand{\R}{\mathbb{R}}
\newcommand{\Q}{\mathbb{Q}}
\newcommand{\T}{\mathcal{T}}
\newcommand{\Z}{\mathbb{Z}}
\renewcommand\qedsymbol{$\blacksquare$}
\newcommand{\qedwhite}{\hfill\ensuremath{\square}}
\newcommand*\conj[1]{\overline{#1}}
\newcommand*\closure[1]{\overline{#1}}
\newcommand*\mean[1]{\overline{#1}}
%\newcommand{\inner}[1]{\left< #1 \right>}
\newcommand{\inner}[2]{\left< #1, #2 \right>}
\newcommand{\powerset}[1]{\pazocal{P}(#1)}
%% Use \pazocal{letter} to typeset a letter in the other kind 
%%  of math calligraphic font. 
\newcommand{\cardinality}[1]{\left| #1 \right|}
\newcommand{\domain}[1]{\mathcal{D}(#1)}
\newcommand{\image}{\text{Im}}
\newcommand{\inv}[1]{#1^{-1}}
\newcommand{\preimage}[2]{#1^{-1}\left(#2\right)}
\newcommand{\script}[1]{\mathcal{#1}}


\newenvironment{highlight}{\begin{mdframed}[backgroundcolor=gray!20]}{\end{mdframed}}

\DeclarePairedDelimiter\ceil{\lceil}{\rceil}
\DeclarePairedDelimiter\floor{\lfloor}{\rfloor}

%===============================================
%===============My Tikz Commands================
%===============================================
\newcommand{\drawsquiggle}[1]{\draw[shift={(#1,0)}] (.005,.05) -- (-.005,.02) -- (.005,-.02) -- (-.005,-.05);}
\newcommand{\drawpoint}[2]{\draw[*-*] (#1,0.01) node[below, shift={(0,-.2)}] {#2};}
\newcommand{\drawopoint}[2]{\draw[o-o] (#1,0.01) node[below, shift={(0,-.2)}] {#2};}
\newcommand{\drawlpoint}[2]{\draw (#1,0.02) -- (#1,-0.02) node[below] {#2};}
\newcommand{\drawlbrack}[2]{\draw (#1+.01,0.02) --(#1,0.02) -- (#1,-0.02) -- (#1+.01,-0.02) node[below, shift={(-.01,0)}] {#2};}
\newcommand{\drawrbrack}[2]{\draw (#1-.01,0.02) --(#1,0.02) -- (#1,-0.02) -- (#1-.01,-0.02) node[below, shift={(+.01,0)}] {#2};}

%***********************************************
%**************Start of Document****************
%***********************************************
 %find me at /home/trevor/texmf/tex/latex/tskpreamble_nothms.tex
%===============================================
%===============Theorem Styles==================
%===============================================

%================Default Style==================
\theoremstyle{plain}% is the default. it sets the text in italic and adds extra space above and below the \newtheorems listed below it in the input. it is recommended for theorems, corollaries, lemmas, propositions, conjectures, criteria, and (possibly; depends on the subject area) algorithms.
\newtheorem{theorem}{Theorem}
\numberwithin{theorem}{section} %This sets the numbering system for theorems to number them down to the {argument} level. I have it set to number down to the {section} level right now.
\newtheorem*{theorem*}{Theorem} %Theorem with no numbering
\newtheorem{corollary}[theorem]{Corollary}
\newtheorem*{corollary*}{Corollary}
\newtheorem{conjecture}[theorem]{Conjecture}
\newtheorem{lemma}[theorem]{Lemma}
\newtheorem*{lemma*}{Lemma}
\newtheorem{proposition}[theorem]{Proposition}
\newtheorem*{proposition*}{Proposition}
\newtheorem{problemstatement}[theorem]{Problem Statement}


%==============Definition Style=================
\theoremstyle{definition}% adds extra space above and below, but sets the text in roman. it is recommended for definitions, conditions, problems, and examples; i've alse seen it used for exercises.
\newtheorem{definition}[theorem]{Definition}
\newtheorem*{definition*}{Definition}
\newtheorem{condition}[theorem]{Condition}
\newtheorem{problem}[theorem]{Problem}
\newtheorem{example}[theorem]{Example}
\newtheorem*{example*}{Example}
\newtheorem*{counterexample*}{Counterexample}
\newtheorem*{romantheorem*}{Theorem} %Theorem with no numbering
\newtheorem{exercise}{Exercise}
\numberwithin{exercise}{section}
\newtheorem{algorithm}[theorem]{Algorithm}

%================Remark Style===================
\theoremstyle{remark}% is set in roman, with no additional space above or below. it is recommended for remarks, notes, notation, claims, summaries, acknowledgments, cases, and conclusions.
\newtheorem{remark}[theorem]{Remark}
\newtheorem*{remark*}{Remark}
\newtheorem{notation}[theorem]{Notation}
\newtheorem*{notation*}{Notation}
%\newtheorem{claim}[theorem]{Claim}  %%use this if you ever want claims to be numbered
\newtheorem*{claim}{Claim}


%\DeclareMathOperator{\diameter}{diam}
%\newcommand{\diam}[1]{\diameter\left(#1\right)}
\newcommand{\CP}{\C\mathbb{P}}
\let\temp\phi
\let\phi\varphi
\let\varphi\temp
\newcommand{\id}{\text{id}}

%%
%% Page set-up:
%%
\pagestyle{empty}
\lhead{\textsc{240 - Differential Geometry \\}} 
\rhead{\textsc{Dai, Fall 2019 \\ Trevor Klar}}
%\chead{\Large\textbf{A New Integration Technique \\ }}
\renewcommand{\headrulewidth}{0pt}
%
\renewcommand{\footrulewidth}{0pt}
%\lfoot{
%Office: \quad \quad \, M 2-3 \, \, SH 6431x \\
%Math Lab: \, W 12-2 \, SH 1607
%}
%\rfoot{trevorklar@math.ucsb.edu}


\setlength{\parindent}{0in}
\setlength{\textwidth}{7in}
\setlength{\evensidemargin}{-0.25in}
\setlength{\oddsidemargin}{-0.25in}
\setlength{\parskip}{.5\baselineskip}
\setlength{\topmargin}{-0.5in}
\setlength{\textheight}{9in}

\setlist[enumerate,1]{label=\textbf{\arabic*.}}

\begin{document}
\pagestyle{fancy}
\begin{center}
{\LARGE Final Exam}%=================UPDATE THIS=================%
\end{center}

\begin{enumerate}

\item Let $M^n$ be an embedded submanifold of $N^m$. Show that $TM$ is an embedded submanifold of $TN$.
\begin{proof}
Since $M$ is an embedded submanifold of $N$, then the inclusion map $\iota:M\hookrightarrow N$ is an embedding; that is, a smooth map of rank $n$ which is a homeomorphism onto its image with the subspace topology. 

If we consider the global differential $d\iota:TM\to TN$, we know that $d\iota$ is smooth since $\iota$ is smooth (Proposition 3.21). Now observe that 
\begin{align*}
d\iota(x^1, \dots, x^n, v^1, \dots, v^n)&= \left(\iota^1(x),\dots \iota^n(x), \frac{\del\iota^1}{\del x^i} (x)v^i, \dots, \frac{\del\iota^n}{\del x^i} (x)v^i\right) \\
&= (x^1, \dots, x^n,v^1, \dots, v^n),\\
\end{align*}
where the last line comes from the fact that $\iota^i(n)$ just gives the $i$-th coordinate of $x$ and $\frac{\del\iota^j}{\del x^i} (x)$ is constantly 1 if $i=j$ and constantly 0 if $i\neq j$. This means that the rank of $d\iota$ is $2n$, and $d\iota$ is the inclusion map $TM\hookrightarrow TN$. Since $\iota$ is a homeomorphism onto its image and $T_pM$ is a linear subspace of $T_pN$ for every $p\in M$, then $d\iota$ is also a homeomorphism onto its image, so $d\iota$ is an embedding, and we are done. 
\end{proof}

\item Let $X,Y,Z$ be vector fields on $\R^3$ given by 
$$X=y\frac{\del}{\del z}-z\frac{\del}{\del y}, \quad 
  Y=z\frac{\del}{\del x}-x\frac{\del}{\del z}, \quad 
  Z=x\frac{\del}{\del y}-y\frac{\del}{\del x}.$$
Let $A$ be the linear space spanned by $X,Y,Z$. Show that $A$ is a 3-dimensional Lie algebra with the Lie bracket of $\mathfrak{X}(\R^3)$. 
\begin{proof}
Since $X,Y,Z$ are clearly linearly independent, then their span over coefficients in $\R$ is a 3-dimensional vector space. Now we show that the bracket $[X,Y]=XY-YX$ on $A$ satisfies the desired properties. By the symmetry in their definitions, $[X,Y], [Y,Z],$ and $[Z,X]$ will all have the same properties, and while an arbitrary vector field in $A$ will be of the form $(aX+bY+cZ)$, it suffices to show that the desired properties hold for the basis vectors. 
	\begin{enumerate}
	\item \textsc{Bilinearity}: 
	\begin{align*}
	[aX+bY,Z]f&=(aX+bY)Zf-Z(aX+bY)f\\
	\\
	(aX+bY)Zf&=\left(ay\frac{\del}{\del z}-az\frac{\del}{\del y}+bz\frac{\del}{\del x}-bx\frac{\del}{\del z} \right) \circ
	  \left(x\frac{\del}{\del y}-y\frac{\del}{\del x}\right)f \\
	&=\left(bz\frac{\del}{\del x} -az\frac{\del}{\del y}+ (ay-bx)\frac{\del}{\del z} \right) \circ	 
	  \left(x\frac{\del f}{\del y}-y\frac{\del f}{\del x}\right)  \\
 	&=bz\frac{\del f}{\del y}+bxz\frac{\del^2f}{\del y\del x}-bzy\frac{\del^2f}{\del x^2}
 	  -axz\frac{\del^2f}{\del y^2}+az\frac{\del f}{\del x}+ayz\frac{\del^2f}{\del x\del y}\\
 	  &\quad +(ay-bx)\left(x\frac{\del^2f}{\del y\del z}-y\frac{\del^2f}{\del x\del z}\right) \\
 	\\ 
 	-Z(aX+bY)f&=
 	  -\left(x\frac{\del }{\del y}-y\frac{\del }{\del x}\right)
 	  \circ \left(bz\frac{\del}{\del x} -az\frac{\del}{\del y}+ (ay-bx)\frac{\del}{\del z} \right)f \\
 	&=\left(-x\frac{\del }{\del y}+y\frac{\del }{\del x}\right)
 	  \circ \left(bz\frac{\del f}{\del x} -az\frac{\del f}{\del y}+ ay\frac{\del f}{\del z}-bx\frac{\del f}{\del z} \right) \\
 	&=-bxz\frac{\del^2f}{\del x\del y}+axz\frac{\del^2f}{\del y^2}-ax\frac{\del f}{\del z}-axy\frac{\del^2f}{\del z\del y}+bx^2\frac{\del^2f}{\del z\del y} \\
 	  &\quad + byz\frac{\del^2f}{\del x^2}-ayz\frac{\del^2f}{\del y\del x}+ay^2\frac{\del^2f}{\del z\del x}-by\frac{\del f}{\del z}-bxy\frac{\del^2f}{\del z\del x}
	\end{align*}
	and all of the second-order derivatives cancel\footnote{Since $f\in C^\infty(\R)$, mixed partials are equal.}, the sum is given by
	\[[aX+bY,Z]f = az\frac{\del f}{\del x}+bz\frac{\del f }{\del y}-(ax+by)\frac{\del f}{\del z}.\]
	Now we check that this is equal to $a[X,Z]f+b[Y,Z]f:$
	\begin{align*}
	a[X,Z]f &= a(XZf-ZXf) \\
%##1##	
	&= a\left[\left(y\frac{\del}{\del z}-z\frac{\del}{\del y}\right)\circ\left(x\frac{\del}{\del y}-y\frac{\del}{\del x}\right)f-\left(x\frac{\del}{\del y}-y\frac{\del}{\del x}\right)\circ \left(y\frac{\del}{\del z}-z\frac{\del}{\del y}\right)f\right]\\
%##2##
  &= a\left[\left(y\frac{\del}{\del z}-z\frac{\del}{\del y}\right)\circ\left(x\frac{\del f}{\del y}-y\frac{\del f}{\del x}\right)
  -\left(x\frac{\del}{\del y}-y\frac{\del}{\del x}\right)\circ \left(y\frac{\del f}{\del z}-z\frac{\del f}{\del y}\right)\right]\\
%##3##
  &=a\left[-x\frac{\del f}{\del z}+z\frac{\del f}{\del x}\right] \\
    &=-ax\frac{\del f}{\del z}+az\frac{\del f}{\del x}
\end{align*}
\begin{align*}
%##1##
  b[Y,Z]f&=b(YZf-ZYf)\\
%##2##
  &=b\left[\left(z\frac{\del}{\del x}-x\frac{\del}{\del z} \right) \circ \left(x\frac{\del}{\del y}-y\frac{\del}{\del x}\right)f
  -\left(x\frac{\del}{\del y}-y\frac{\del}{\del x}\right) \circ \left(z\frac{\del}{\del x}-x\frac{\del}{\del z} \right)f\right] \\
%##3##
	&= b\left[z\frac{\del f}{\del y}-y\frac{\del f}{\del z}\right] \\
	&= bz\frac{\del f}{\del y}-by\frac{\del f}{\del z} 
	\end{align*}
	So, 
	\begin{align*}
	[aX+bY,Z]f &= az\frac{\del f}{\del x}+bz\frac{\del f }{\del y}-(ax+by)\frac{\del f}{\del z}\\
	&=-ax\frac{\del f}{\del z}+az\frac{\del f}{\del x} + bz\frac{\del f}{\del y}-by\frac{\del f}{\del z} \\
	&= a[X,Z]f + b[Y,Z]f
	\end{align*}
and	a similar proof will show that $[Z,aX+bY]=a[Z,X]+b[Z,Y]$. Thus, the bilinearity property is shown. 
	
	\item \textsc{Antisymmetry}:
	\begin{align*}
	[X,Y]f&=XYf-YXf\\
	&=-(-XYf+YXf)\\
	&=-[Y,X]f
	\end{align*}
	
	\item \textsc{Jacobi Identity}:
	
	By examining some of the computations in the bilinearity part of this proof, we can see that 
	\begin{align*}
	[Y,X]&=Z,\\
	[X,Z]&=Y, \text{ and}\\
	[Z,Y]&=X.
	\end{align*}
	Thus 
	\begin{align*}
	[X,[Y,Z]]+[Y,[Z,X]]+[Z,[X,Y]]&=[X,-X]+[Y,-Y]+[Z,-Z]\\
	&=0+0+0
	\end{align*}
	\end{enumerate}
Therefore, $A$ is a 3-dimensional vector space with a bracket operation having the bilinearity, antisymmetry, and Jacobi identity properties, so $A$ is a 3-dimensional Lie algebra.
\end{proof}

\pagebreak
\item %\mbox{}
	\begin{enumerate} [label=\alph*)]
	\item Compute the flow of the vector field $X$ on $\R^2$: 
$$X=y\frac{\del}{\del x}-\frac{\del}{\del y}.$$
	\answer The vector field $X$ corresponds to the ODE system 
	\begin{align*}
	\dot{y}&=-1\\
	\dot{x}&=y
	\end{align*}	
which can be solved one at a time to find the solutions $y(t)=-t+y_0$ and $x(t)=-\frac{t^2}{2}+y_0t+x_0$. This corresponds to the flow 
$$\theta_t(x,y)=(-\tfrac{t^2}{2}+ty+x, \,\, -t+y),$$
where integral curves are parabolas. \qed

	\item Let $M = M_n(\R)$ be the space of all $n \times n$ matrices. For $A \in M$ let $V_A$ be the vector field on $M$ so that $V_A(X) = AX, $ where $X \in M$ (we have used the identification $T_XM = \R^{n^2} = M$). Compute the
flow $\theta_t$ generated by $V_A$. 
	\answer	The vector field $AX$ corresponds to the linear system of ODEs 
	$X'=AX,$ and if we write $X$ as a concatenation of column vectors, we obtain 
	\[
	\left[ \begin{array}{ccc}
	\mid & & \mid \\
	(\mathbf{x}^1)' & \cdots & (\mathbf{x}^n)' \\
	\mid & & \mid
	\end{array}	\right]
	=	
	\mathbf{A}
	\left[ \begin{array}{ccc}
	\mid & & \mid \\
	\mathbf{x}^1 & \cdots & \mathbf{x}^n \\
	\mid & & \mid
	\end{array}	\right],	
	\]
	and this system can be solved column by column (they will all have the same family of solutions, differing only in their initial values) as 
	\[ (\mathbf{x}^i)'=\mathbf{A}\mathbf{x}^i \quad, \text{for } i\in1, \dots, n\]
	according to the usual method of finding eigenvalues and eigenvectors, finding coefficients, and determining constants assuming that $\mathbf{x}^i(0)=\mathbf{x}^i_0$ for each $i$. For example, 
	$$\mathbf{x}^1(t)=\phi_1(t,\mathbf{x}^1_0)+\dots+ \phi_n(t,\mathbf{x}^1_0),$$
	Where each $\phi_j$ is $\R^n$-valued. Now each $\mathbf{x}^i(t)$ is given by the same set of functions with different initial points, so 
	$$\mathbf{x}^i(t)=\phi_1(t,\mathbf{x}^i_0)+\dots+ \phi_n(t,\mathbf{x}^i_0) \quad \text{for each i.}$$
	To write the flow, let $\tau_t(\mathbf{x})=\phi_1(t,\mathbf{x})+\dots+ \phi_n(t,\mathbf{x})$, then 
	\[ \theta_t(\mathbf{X})=
	\left[\begin{array}{ccc}
	\mid & & \mid \\
	\tau_t(\mathbf{x}^1) & \cdots & \tau_t(\mathbf{x}^n) \\
	\mid & & \mid \\
	\end{array} \right]
	\]\qed
	
	\end{enumerate}

\pagebreak
\item 
	\begin{enumerate}[label=\alph*)]
	\item Give an example of complete vector field and an example of incomplete vector field. Explain
why it is complete or incomplete.

	\answer We will discuss examples from the text, as they are readily available. Let $M=\R^2$, $V=\frac{\del}{\del x}$. Then the flow generated by $V$ is 
	$$\tau_t(x,y)=(x+t,y),$$
	and this is a global flow, since given any $(x,y)\in M$, we can see that $\tau_t(x,y)$ determines an integral curve which is defined for all $t\in\R$. Thus $V$ is complete. 
	
	For an incomplete example, let $M=\R^2$, $V=x^2\frac{\del }{\del x}$. This corresponds to the ODE system 
	\begin{align*}
	\frac{\der x}{\der t}&=x^2 &
	\frac{\der y}{\der t}&\equiv 0
	\end{align*}
	which has solution\footnote{Assuming $x_0\neq0$. Otherwise if $x_0=0$ then $x=0$ for all time.}
	\begin{align*}
	x(t)&=\frac{1}{\frac{1}{x_0}-t} & y(t)=y_0
	\end{align*}
	so the flow is given by 
	$$\theta_t(x,y)=\left(\frac{1}{\frac{1}{x}-t}, y \right).$$
	Consider $\theta^{(1,0)}(t)$. Since 
	$$\theta^{(1,0)}(t)=\left(\frac{1}{1-t}, 0 \right),$$
	then the integral curve cannot be continuously extended past $t=1$, since $x\to\infty$ as $t\increasesto1$. \qed
	
	\item Let $X$ be a vector field on a manifold $M$, and $\gamma(t)$ an integral curve of $X$ starting at $p \in  M$. If $f \in C^\infty(M)$, $f > 0$, find the integral curve of $fX$ starting at $p$.
	
	\answer Assume that $fX$ denotes multiplication, so $(f\cdot X)|_p=f|_p\cdot X|_p$\footnote{This is one of those times when using juxtaposition for multiplication, composition, and evaluation makes me a go a liiiiiiiitle bit crazy.} for any point $p\in M$.	Since $\gamma'=X|_{\gamma}$, then 
	$$(f\cdot X)|_\gamma=f|_\gamma\cdot X|_\gamma=f(\gamma)\cdot \gamma'.$$	
	We seek a function $\Gamma:\R\to M$ such that $\Gamma'= (f\cdot X)|_\Gamma$, and $f(\gamma)\cdot \gamma'$ looks like a chain rule derivative:
	\begin{align*}
	(f\circ\gamma)(t)\cdot \gamma'(t) &= f\big(\gamma^1(t), \dots, \gamma^n(t)\big)\cdot \big(\tfrac{\del \gamma^1}{\del t}+ \dots + \tfrac{\del \gamma^n}{\del t}\big) \\
	&= \sum_{i=1}^n \frac{\del \gamma^i}{\del t}\cdot f\big(\gamma^1(t), \dots, \gamma^n(t)\big)
	\end{align*}
	Thus if there exists a function $F$ such that\footnote{Here we are using $x^i\dots x^n$ to denote the coordinate functions on $M$.} 
	$$\frac{\del F}{\del x^i}=\frac{\del F}{\del x^j}=f \quad\text{for all } 1\leq i,j\leq n,$$
	then we can let $\Gamma=F\circ\gamma$ and we're done. Fortunately, since $f$ is smooth, we can just take indefinite integrals with respect to each $x^i$ and combine the antiderivatives to obtain $F$:
	\begin{align*}
	F(x^1, \dots, x^n)&=\int f(x^1, \dots, x^n) \der x^1 + g(x^2, \dots, x^n) + C_1\\
	&\,\,\,\vdots\\
	&=\int f(x^1, \dots, x^n) \der x^i + g(x^1, \dots, \hat{x}^i, \dots, x^n) + C_i\\
	&\,\,\,\vdots\\
	&=\int f(x^1, \dots, x^n) \der x^n + g(x^1, \dots, x^{n-1}) + C_{n-1}
	\end{align*}
	Combining these together yields a function $F\in C^\infty(M)$ such that 
	$$(F(\gamma	)'=f(\gamma)\cdot\gamma'=fX|_\gamma,$$
	So $\Gamma=F\circ\gamma$ is the desired integral curve. \qed
	
	
	\item Give an explanation why the following statement could be true: for any manifold $M$ and any vector field $X$ on $M$, there is a $f\in C^\infty(M)$, $f > 0$, such that $fX$ is complete. 
	
	\answer Anywhere that an integral curve shoots off to infinity in finite time, you can multiply the vector field by a function which goes to zero faster than the integral curve would go to infinity. Anywhere an integral curve goes to a point not in the manifold, you can similarly scale down the vector field so that it takes infinite time to get there. 
	\end{enumerate}

\item Let $M^n$ be a compact manifold which carries $n$ vector fields $X_1, \dots , X_n$ such that $[X_i, X_j] = 0$ for all $i, j = 1, \dots , n$ and $X_1, \dots , X_n$ are pointwise linearly independent. Let $\theta^1_t, \dots , \theta^n_t$ be the flow generated by $X_1, \dots , X_n$ respectively.
	\begin{enumerate}
	\item Show that $F : \R^n \to M$ defined by $F(x_1, \dots , x_n) = \theta^1_{x_1} \circ \dots \circ \theta^n_{x_n}(p)$ for some fixed $p \in M$ is
well-defined and a submersion. Conclude that $F$ is a local diffeomorphism.
	
	\begin{proof}
	Since $[X_i, X_j] = 0$ for all $i, j = 1, \dots , n$, then the vector fields all commute, so they are invariant under each other's flows. This means that regardless of which vector field we call $X_1, X_2,$ etc, we are always referring to the same function when we say $F$ is the composition of all $n$ flows starting at $p$. 
	
	$F$ is a submersion because $[X_i, X_j] = 0$ for all $i, j = 1, \dots , n$ and $X_1, \dots , X_n$ are pointwise linearly independent, so each flow is locally moving in a linearly independent direction, which means $F$ has rank $n$. Since $F$ is a map of constant rank between two $n$-dimensional manifolds, then it is a local diffeomorphism. 
	\end{proof}
	\end{enumerate}


\end{enumerate}
\end{document}