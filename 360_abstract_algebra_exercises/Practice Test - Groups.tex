\documentclass[letterpaper]{article}
%\documentclass[a5paper]{article}

%% Language and font encodings
\usepackage[english]{babel}
\usepackage[utf8x]{inputenc}
\usepackage[T1]{fontenc}
\usepackage{xfrac}

%% Sets page size and margins
\usepackage[letterpaper,top=.75in,bottom=1in,left=1in,right=1in,marginparwidth=1.75cm]{geometry}
%\usepackage[a5paper,top=1cm,bottom=1cm,left=1cm,right=1.5cm,marginparwidth=1.75cm]{geometry}

\usepackage{graphicx}
%\graphicspath{../images}	  %%where to look for images

%% Useful packages
\usepackage{amssymb, amsmath, amsthm} 
%\usepackage{graphicx}  %%this is currently enabled in the default document, so it is commented out here. 
\usepackage{calrsfs}
\usepackage{braket}
\usepackage{mathtools}
\usepackage{lipsum}
\usepackage{tikz}
\usetikzlibrary{cd}
\usepackage{verbatim}
%\usepackage{ntheorem}% for theorem-like environments
\usepackage{mdframed}%can make highlighted boxes of text
%Use case: https://tex.stackexchange.com/questions/46828/how-to-highlight-important-parts-with-a-gray-background
\usepackage{wrapfig}
\usepackage{centernot}
\usepackage{subcaption}%\begin{subfigure}{0.5\textwidth}
\usepackage{pgfplots}
\pgfplotsset{compat=1.13}
\usepackage[colorinlistoftodos]{todonotes}
\usepackage[colorlinks=true, allcolors=blue]{hyperref}
\usepackage{xfrac}					%to make slanted fractions \sfrac{numerator}{denominator}
\usepackage{enumitem}            
    %syntax: \begin{enumerate}[label=(\alph*)]
    %possible arguments: f \alph*, \Alph*, \arabic*, \roman* and \Roman*
\usetikzlibrary{arrows,shapes.geometric,fit}

\DeclareMathAlphabet{\pazocal}{OMS}{zplm}{m}{n}
%% Use \pazocal{letter} to typeset a letter in the other kind 
%%  of math calligraphic font. 

%% This puts the QED block at the end of each proof, the way I like it. 
\renewenvironment{proof}{{\bfseries Proof}}{\qed}
\makeatletter
\renewenvironment{proof}[1][\bfseries \proofname]{\par
  \pushQED{\qed}%
  \normalfont \topsep6\p@\@plus6\p@\relax
  \trivlist
  %\itemindent\normalparindent
  \item[\hskip\labelsep
        \scshape
    #1\@addpunct{}]\ignorespaces
}{%
  \popQED\endtrivlist\@endpefalse
}
\makeatother

%% This adds a \rewnewtheorem command, which enables me to override the settings for theorems contained in this document.
\makeatletter
\def\renewtheorem#1{%
  \expandafter\let\csname#1\endcsname\relax
  \expandafter\let\csname c@#1\endcsname\relax
  \gdef\renewtheorem@envname{#1}
  \renewtheorem@secpar
}
\def\renewtheorem@secpar{\@ifnextchar[{\renewtheorem@numberedlike}{\renewtheorem@nonumberedlike}}
\def\renewtheorem@numberedlike[#1]#2{\newtheorem{\renewtheorem@envname}[#1]{#2}}
\def\renewtheorem@nonumberedlike#1{  
\def\renewtheorem@caption{#1}
\edef\renewtheorem@nowithin{\noexpand\newtheorem{\renewtheorem@envname}{\renewtheorem@caption}}
\renewtheorem@thirdpar
}
\def\renewtheorem@thirdpar{\@ifnextchar[{\renewtheorem@within}{\renewtheorem@nowithin}}
\def\renewtheorem@within[#1]{\renewtheorem@nowithin[#1]}
\makeatother

%% This makes theorems and definitions with names show up in bold, the way I like it. 
\makeatletter
\def\th@plain{%
  \thm@notefont{}% same as heading font
  \itshape % body font
}
\def\th@definition{%
  \thm@notefont{}% same as heading font
  \normalfont % body font
}
\makeatother

%===============================================
%==============Shortcut Commands================
%===============================================
\newcommand{\ds}{\displaystyle}
\newcommand{\B}{\mathcal{B}}
\newcommand{\C}{\mathbb{C}}
\newcommand{\F}{\mathbb{F}}
\newcommand{\N}{\mathbb{N}}
\newcommand{\R}{\mathbb{R}}
\newcommand{\Q}{\mathbb{Q}}
\newcommand{\T}{\mathcal{T}}
\newcommand{\Z}{\mathbb{Z}}
\renewcommand\qedsymbol{$\blacksquare$}
\newcommand{\qedwhite}{\hfill\ensuremath{\square}}
\newcommand*\conj[1]{\overline{#1}}
\newcommand*\closure[1]{\overline{#1}}
\newcommand*\mean[1]{\overline{#1}}
%\newcommand{\inner}[1]{\left< #1 \right>}
\newcommand{\inner}[2]{\left< #1, #2 \right>}
\newcommand{\powerset}[1]{\pazocal{P}(#1)}
%% Use \pazocal{letter} to typeset a letter in the other kind 
%%  of math calligraphic font. 
\newcommand{\cardinality}[1]{\left| #1 \right|}
\newcommand{\domain}[1]{\mathcal{D}(#1)}
\newcommand{\image}{\text{Im}}
\newcommand{\inv}[1]{#1^{-1}}
\newcommand{\preimage}[2]{#1^{-1}\left(#2\right)}
\newcommand{\script}[1]{\mathcal{#1}}


\newenvironment{highlight}{\begin{mdframed}[backgroundcolor=gray!20]}{\end{mdframed}}

\DeclarePairedDelimiter\ceil{\lceil}{\rceil}
\DeclarePairedDelimiter\floor{\lfloor}{\rfloor}

%===============================================
%===============My Tikz Commands================
%===============================================
\newcommand{\drawsquiggle}[1]{\draw[shift={(#1,0)}] (.005,.05) -- (-.005,.02) -- (.005,-.02) -- (-.005,-.05);}
\newcommand{\drawpoint}[2]{\draw[*-*] (#1,0.01) node[below, shift={(0,-.2)}] {#2};}
\newcommand{\drawopoint}[2]{\draw[o-o] (#1,0.01) node[below, shift={(0,-.2)}] {#2};}
\newcommand{\drawlpoint}[2]{\draw (#1,0.02) -- (#1,-0.02) node[below] {#2};}
\newcommand{\drawlbrack}[2]{\draw (#1+.01,0.02) --(#1,0.02) -- (#1,-0.02) -- (#1+.01,-0.02) node[below, shift={(-.01,0)}] {#2};}
\newcommand{\drawrbrack}[2]{\draw (#1-.01,0.02) --(#1,0.02) -- (#1,-0.02) -- (#1-.01,-0.02) node[below, shift={(+.01,0)}] {#2};}

%***********************************************
%**************Start of Document****************
%***********************************************

%===============================================
%===============Theorem Styles==================
%===============================================

%================Default Style==================
\theoremstyle{plain}% is the default. it sets the text in italic and adds extra space above and below the \newtheorems listed below it in the input. it is recommended for theorems, corollaries, lemmas, propositions, conjectures, criteria, and (possibly; depends on the subject area) algorithms.
\newtheorem{theorem}{Theorem}
\numberwithin{theorem}{section} %This sets the numbering system for theorems to number them down to the {argument} level. I have it set to number down to the {section} level right now.
\newtheorem*{theorem*}{Theorem} %Theorem with no numbering
\newtheorem{corollary}[theorem]{Corollary}
\newtheorem*{corollary*}{Corollary}
\newtheorem{conjecture}[theorem]{Conjecture}
\newtheorem{lemma}[theorem]{Lemma}
\newtheorem*{lemma*}{Lemma}
\newtheorem{proposition}[theorem]{Proposition}
\newtheorem*{proposition*}{Proposition}
\newtheorem{problemstatement}[theorem]{Problem Statement}


%==============Definition Style=================
\theoremstyle{definition}% adds extra space above and below, but sets the text in roman. it is recommended for definitions, conditions, problems, and examples; i've alse seen it used for exercises.
\newtheorem{definition}[theorem]{Definition}
\newtheorem*{definition*}{Definition}
\newtheorem{condition}[theorem]{Condition}
\newtheorem{problem}[theorem]{Problem}
\newtheorem{example}[theorem]{Example}
\newtheorem*{example*}{Example}
\newtheorem*{counterexample*}{Counterexample}
\newtheorem*{romantheorem*}{Theorem} %Theorem with no numbering
\newtheorem{exercise}{Exercise}
\numberwithin{exercise}{section}
\newtheorem{algorithm}[theorem]{Algorithm}

%================Remark Style===================
\theoremstyle{remark}% is set in roman, with no additional space above or below. it is recommended for remarks, notes, notation, claims, summaries, acknowledgments, cases, and conclusions.
\newtheorem{remark}[theorem]{Remark}
\newtheorem*{remark*}{Remark}
\newtheorem{notation}[theorem]{Notation}
\newtheorem*{notation*}{Notation}
%\newtheorem{claim}[theorem]{Claim}  %%use this if you ever want claims to be numbered
\newtheorem*{claim}{Claim}



\pgfplotsset{compat=1.13}

%%%%%%%%%%%%%%%%%%%%%%%%%% HEADER %%%%%%%%%%%%%%%%%%%%%%%%%%%%%%%
\title{Math 360 \linebreak
Section 2.4 Exercises}
\author{Trevor Klar}
%%%%%%%%%%%%%%%%%%%%%%%%%%%%%%%%%%%%%%%%%%%%%%%%%%%%%%%%%%%%%%%%%

\newcommand{\instructions}[1]{\hspace*{-0.6cm}\textbf{#1}}

\begin{document}

%\maketitle

\begin{enumerate}

\item[3.] Recall that $GL_2(\R)$ is the group of invertible $2\times 2$ matrices, $SL_2(\R)$ is its subgroup of all invertible $2\times 2$ matrices with determinant equal to one.
	\begin{enumerate}
	\item Let $H=\left\lbrace
	\arraycolsep=2.4pt\def\arraystretch{1}
	\left[\begin{array}{lr}
	a & b \\
	0 & c \\
	\end{array} \right]
	: a,b,c \in \R, a,c,\neq 0
	\right\rbrace.$
	\textbf{Prove} $H$ is a subgroup of $GL_2(\R)$. 
	\begin{proof}
	We must show that (1) every element of $H$ is in $GL_2(\R)$, and (2) $H$ is closed under matrix multiplication. To see that (1) holds, observe that for any 
	$h=\arraycolsep=2.4pt\def\arraystretch{1}
	\left[\begin{array}{lr}
	a & b \\
	0 & c \\
	\end{array} \right]
	\in H$, we have that $\det(h)=ac$, and since $h\in H$, then $a,c\neq 0$, so $ac\neq 0$. Thus $h$ is invertible, and $h\in GL_2(\R)$. Now we will prove (2). Let $h_1, h_2\in H$, with 
	\[h_1=\arraycolsep=2.4pt\def\arraystretch{1}
	\left[\begin{array}{lr}
	a_1 & b_1 \\
	0 & c_1 \\
	\end{array} \right]
	, \text{ and }
	h_2=\arraycolsep=2.4pt\def\arraystretch{1}
	\left[\begin{array}{lr}
	a_2 & b_2 \\
	0 & c_2 \\
	\end{array} \right].\]
	Then,
	\[\def\arraystretch{2}\begin{array}{rcl}
	h_1h_2 &=& 
	\arraycolsep=2.4pt\def\arraystretch{1}
	\left[\begin{array}{lr}
	a_1 & b_1 \\
	0 & c_1 \\
	\end{array} \right]
	\left[\begin{array}{lr}
	a_2 & b_2 \\
	0 & c_2 \\
	\end{array} \right]
	\\
	%%%%%%%%%%%%%%%%%%%%%%%%%%%%%%
	&=&
	\arraycolsep=3.4pt\def\arraystretch{1}
	\left[\begin{array}{cc}
	a_1a_2 & (a_1b_2+b_1c_2) \\
	0 & c_1c_2 \\
	\end{array} \right]\\
	%	
	\end{array} \]
	and, since we know that $a_1,a_2,c_1,c_2\neq0$, then $a_1a_2,c_1c_2\neq0$ as well. Thus, $h_1h_2\in H$, and $H$ is closed. 
	\end{proof}		
	
	\item Consider the following two matrices in $GL_2(\R)$: 
	$x=
	\arraycolsep=2.4pt\def\arraystretch{1}
	\left[\begin{array}{lr}
	4 & 0 \\
	0 & 2 \\
	\end{array} \right],
	y=
	\arraycolsep=2.4pt\def\arraystretch{1}
	\left[\begin{array}{lr}
	4 & 2 \\
	1 & 1 \\
	\end{array} \right].$
	
	Explain why $SL_2(\R)x=SL_2(\R)y$ (that is, why are these two right cosets
equal? Think of the equivalence relation).
	
	\begin{proof}
	\textbf{(Or how I expected the proof to go)} 
	
	It suffices to show that $x\sim_{{}_R} y$, because since equivalence relations are transitive and symmetric, any matrix which is in one coset will also be in the other. To see that $x\sim_{{}_R} y$, observe that 
	\[\def\arraystretch{2}\begin{array}{rcl}
	x\inv{y}&=&
	\arraycolsep=2.4pt\def\arraystretch{1}
	\left[\begin{array}{lr}
	4 & 0 \\
	0 & 2 \\
	\end{array} \right]
	\frac{1}{4-2}
	\arraycolsep=2.4pt\def\arraystretch{1}
	\left[\begin{array}{lr}
	1 & -2 \\
	-1 & 4 \\
	\end{array} \right]
	\\
	&=&
	\arraycolsep=2.4pt\def\arraystretch{1}
	\left[\begin{array}{lr}
	4 & 0 \\
	0 & 2 \\
	\end{array} \right]
	\arraycolsep=2.4pt\def\arraystretch{1}
	\left[\begin{array}{lr}
	\sfrac{1}{2} & -1 \\
	-\sfrac{1}{2} & 2 \\
	\end{array} \right]
	\\
	&=&
	\arraycolsep=2.4pt\def\arraystretch{1}
	\left[\begin{array}{lr}
	2 & -4 \\
	-1 & 4 \\
	\end{array} \right]
	\end{array}\]
	and $\det(x\inv{y})=1$, so $x\inv{y}\in SL_2(\R)$. \textbf{EXCEPT:} $\det(x\inv{y})$ is not 1, it's 4. So is this my mistake? Or should the numbers in the problem be slightly different?
	
	\end{proof}
	\end{enumerate}

%(b)  ��! R : � = 4 0

%0 2 , � = 4 2
%1 1 .
%Explain why ��! R � = ��! R � (that is, why are these two right cosets
%equal? Think of the equivalence relation)
%(I don’t mind if you have to look up the formula for the inverse of a two by two
%invertible matrix – but it may be on the final)

\end{enumerate}

\end{document}
