\documentclass[letterpaper]{article}
%\documentclass[a5paper]{article}

%% Language and font encodings
\usepackage[english]{babel}
\usepackage[utf8x]{inputenc}
\usepackage[T1]{fontenc}


%% Sets page size and margins
\usepackage[letterpaper,top=.75in,bottom=1in,left=1in,right=1in,marginparwidth=1.75cm]{geometry}
%\usepackage[a5paper,top=1cm,bottom=1cm,left=1cm,right=1.5cm,marginparwidth=1.75cm]{geometry}

\usepackage{graphicx}
%\graphicspath{../images}	  %%where to look for images

%% Useful packages
\usepackage{amssymb, amsmath, amsthm} 
%\usepackage{graphicx}  %%this is currently enabled in the default document, so it is commented out here. 
\usepackage{calrsfs}
\usepackage{braket}
\usepackage{mathtools}
\usepackage{lipsum}
\usepackage{tikz}
\usetikzlibrary{cd}
\usepackage{verbatim}
%\usepackage{ntheorem}% for theorem-like environments
\usepackage{mdframed}%can make highlighted boxes of text
%Use case: https://tex.stackexchange.com/questions/46828/how-to-highlight-important-parts-with-a-gray-background
\usepackage{wrapfig}
\usepackage{centernot}
\usepackage{subcaption}%\begin{subfigure}{0.5\textwidth}
\usepackage{pgfplots}
\pgfplotsset{compat=1.13}
\usepackage[colorinlistoftodos]{todonotes}
\usepackage[colorlinks=true, allcolors=blue]{hyperref}
\usepackage{xfrac}					%to make slanted fractions \sfrac{numerator}{denominator}
\usepackage{enumitem}            
    %syntax: \begin{enumerate}[label=(\alph*)]
    %possible arguments: f \alph*, \Alph*, \arabic*, \roman* and \Roman*
\usetikzlibrary{arrows,shapes.geometric,fit}

\DeclareMathAlphabet{\pazocal}{OMS}{zplm}{m}{n}
%% Use \pazocal{letter} to typeset a letter in the other kind 
%%  of math calligraphic font. 

%% This puts the QED block at the end of each proof, the way I like it. 
\renewenvironment{proof}{{\bfseries Proof}}{\qed}
\makeatletter
\renewenvironment{proof}[1][\bfseries \proofname]{\par
  \pushQED{\qed}%
  \normalfont \topsep6\p@\@plus6\p@\relax
  \trivlist
  %\itemindent\normalparindent
  \item[\hskip\labelsep
        \scshape
    #1\@addpunct{}]\ignorespaces
}{%
  \popQED\endtrivlist\@endpefalse
}
\makeatother

%% This adds a \rewnewtheorem command, which enables me to override the settings for theorems contained in this document.
\makeatletter
\def\renewtheorem#1{%
  \expandafter\let\csname#1\endcsname\relax
  \expandafter\let\csname c@#1\endcsname\relax
  \gdef\renewtheorem@envname{#1}
  \renewtheorem@secpar
}
\def\renewtheorem@secpar{\@ifnextchar[{\renewtheorem@numberedlike}{\renewtheorem@nonumberedlike}}
\def\renewtheorem@numberedlike[#1]#2{\newtheorem{\renewtheorem@envname}[#1]{#2}}
\def\renewtheorem@nonumberedlike#1{  
\def\renewtheorem@caption{#1}
\edef\renewtheorem@nowithin{\noexpand\newtheorem{\renewtheorem@envname}{\renewtheorem@caption}}
\renewtheorem@thirdpar
}
\def\renewtheorem@thirdpar{\@ifnextchar[{\renewtheorem@within}{\renewtheorem@nowithin}}
\def\renewtheorem@within[#1]{\renewtheorem@nowithin[#1]}
\makeatother

%% This makes theorems and definitions with names show up in bold, the way I like it. 
\makeatletter
\def\th@plain{%
  \thm@notefont{}% same as heading font
  \itshape % body font
}
\def\th@definition{%
  \thm@notefont{}% same as heading font
  \normalfont % body font
}
\makeatother

%===============================================
%==============Shortcut Commands================
%===============================================
\newcommand{\ds}{\displaystyle}
\newcommand{\B}{\mathcal{B}}
\newcommand{\C}{\mathbb{C}}
\newcommand{\F}{\mathbb{F}}
\newcommand{\N}{\mathbb{N}}
\newcommand{\R}{\mathbb{R}}
\newcommand{\Q}{\mathbb{Q}}
\newcommand{\T}{\mathcal{T}}
\newcommand{\Z}{\mathbb{Z}}
\renewcommand\qedsymbol{$\blacksquare$}
\newcommand{\qedwhite}{\hfill\ensuremath{\square}}
\newcommand*\conj[1]{\overline{#1}}
\newcommand*\closure[1]{\overline{#1}}
\newcommand*\mean[1]{\overline{#1}}
%\newcommand{\inner}[1]{\left< #1 \right>}
\newcommand{\inner}[2]{\left< #1, #2 \right>}
\newcommand{\powerset}[1]{\pazocal{P}(#1)}
%% Use \pazocal{letter} to typeset a letter in the other kind 
%%  of math calligraphic font. 
\newcommand{\cardinality}[1]{\left| #1 \right|}
\newcommand{\domain}[1]{\mathcal{D}(#1)}
\newcommand{\image}{\text{Im}}
\newcommand{\inv}[1]{#1^{-1}}
\newcommand{\preimage}[2]{#1^{-1}\left(#2\right)}
\newcommand{\script}[1]{\mathcal{#1}}


\newenvironment{highlight}{\begin{mdframed}[backgroundcolor=gray!20]}{\end{mdframed}}

\DeclarePairedDelimiter\ceil{\lceil}{\rceil}
\DeclarePairedDelimiter\floor{\lfloor}{\rfloor}

%===============================================
%===============My Tikz Commands================
%===============================================
\newcommand{\drawsquiggle}[1]{\draw[shift={(#1,0)}] (.005,.05) -- (-.005,.02) -- (.005,-.02) -- (-.005,-.05);}
\newcommand{\drawpoint}[2]{\draw[*-*] (#1,0.01) node[below, shift={(0,-.2)}] {#2};}
\newcommand{\drawopoint}[2]{\draw[o-o] (#1,0.01) node[below, shift={(0,-.2)}] {#2};}
\newcommand{\drawlpoint}[2]{\draw (#1,0.02) -- (#1,-0.02) node[below] {#2};}
\newcommand{\drawlbrack}[2]{\draw (#1+.01,0.02) --(#1,0.02) -- (#1,-0.02) -- (#1+.01,-0.02) node[below, shift={(-.01,0)}] {#2};}
\newcommand{\drawrbrack}[2]{\draw (#1-.01,0.02) --(#1,0.02) -- (#1,-0.02) -- (#1-.01,-0.02) node[below, shift={(+.01,0)}] {#2};}

%***********************************************
%**************Start of Document****************
%***********************************************

%===============================================
%===============Theorem Styles==================
%===============================================

%================Default Style==================
\theoremstyle{plain}% is the default. it sets the text in italic and adds extra space above and below the \newtheorems listed below it in the input. it is recommended for theorems, corollaries, lemmas, propositions, conjectures, criteria, and (possibly; depends on the subject area) algorithms.
\newtheorem{theorem}{Theorem}
\numberwithin{theorem}{section} %This sets the numbering system for theorems to number them down to the {argument} level. I have it set to number down to the {section} level right now.
\newtheorem*{theorem*}{Theorem} %Theorem with no numbering
\newtheorem{corollary}[theorem]{Corollary}
\newtheorem*{corollary*}{Corollary}
\newtheorem{conjecture}[theorem]{Conjecture}
\newtheorem{lemma}[theorem]{Lemma}
\newtheorem*{lemma*}{Lemma}
\newtheorem{proposition}[theorem]{Proposition}
\newtheorem*{proposition*}{Proposition}
\newtheorem{problemstatement}[theorem]{Problem Statement}


%==============Definition Style=================
\theoremstyle{definition}% adds extra space above and below, but sets the text in roman. it is recommended for definitions, conditions, problems, and examples; i've alse seen it used for exercises.
\newtheorem{definition}[theorem]{Definition}
\newtheorem*{definition*}{Definition}
\newtheorem{condition}[theorem]{Condition}
\newtheorem{problem}[theorem]{Problem}
\newtheorem{example}[theorem]{Example}
\newtheorem*{example*}{Example}
\newtheorem*{counterexample*}{Counterexample}
\newtheorem*{romantheorem*}{Theorem} %Theorem with no numbering
\newtheorem{exercise}{Exercise}
\numberwithin{exercise}{section}
\newtheorem{algorithm}[theorem]{Algorithm}

%================Remark Style===================
\theoremstyle{remark}% is set in roman, with no additional space above or below. it is recommended for remarks, notes, notation, claims, summaries, acknowledgments, cases, and conclusions.
\newtheorem{remark}[theorem]{Remark}
\newtheorem*{remark*}{Remark}
\newtheorem{notation}[theorem]{Notation}
\newtheorem*{notation*}{Notation}
%\newtheorem{claim}[theorem]{Claim}  %%use this if you ever want claims to be numbered
\newtheorem*{claim}{Claim}



\pgfplotsset{compat=1.13}

%\newcommand{\T}{\mathcal{T}}
%\newcommand{\B}{\mathcal{B}}

%These commands are now in tskpreamble_nothms.tex, but are left as a comment here for reference. 
%\newcommand{\arbcup}[1]{\bigcup\limits_{\alpha\in\Gamma}#1_\alpha}
%\newcommand{\arbcap}[1]{\bigcap\limits_{\alpha\in\Gamma}#1_\alpha}
%\newcommand{\arbcoll}[1]{\{#1_\alpha\}_{\alpha\in\Gamma}}
%\newcommand{\arbprod}[1]{\prod\limits_{\alpha\in\Gamma}#1_\alpha}
%\newcommand{\finitecoll}[1]{#1_1, \ldots, #1_n}
%\newcommand{\finitefuncts}[2]{#1(#2_1), \ldots, #1(#2_n)}
%\newcommand{\abs}[1]{\left|#1\right|}
%\newcommand{\norm}[1]{\left|\left|#1\right|\right|}

\title{Math 360 \linebreak
Section 1.2 Exercises}
\author{Trevor Klar}

\begin{document}

\maketitle

\begin{enumerate}
\item The three things we must check in order to confirm that a function $\phi:{S}\to{S'}$ is an isomorphism are the following:
	\begin{itemize}
	\item $\phi$ is one-to-one
	\item $\phi$ is onto
	\item For all $a,b\in S$, we have $\phi(a*b)=\phi(a)*'\phi(b)$. 
	\end{itemize}

\hspace*{-0.74cm}\textbf{Determine whether the given map $\phi$ is an isomorphism of the first binary structure with the second. If not, why not?}

\item[2.] $\inner{\Z}{+}$ with $\inner{\Z}{+}$ where $\phi(n)=-n$ for $n\in\Z$

\textbf{Answer: }Yes, because $\phi$ is a bijection, and for all $a,b\in\Z$, $-(a+b)=-(a)+-(b)$. 

\item[3.] $\inner{\Z}{+}$ with $\inner{\Z}{+}$ where $\phi(n)=2n$ for $n\in\Z$

\answer No, because $\phi$ is not onto; there is no $n\in\Z$ such that $2n=3$, for example. 

\item[5.] $\inner{\Q}{\cdot}$ with $\inner{\Q}{\cdot}$ where $\phi(x)=\frac{x}{2}$ for $x\in\Q$

\answer No, because for any nonzero $a,b\in\Q$, we have $\frac{a\cdot b}{2}\neq\frac{a}{2}\cdot\frac{b}{2}$

\item[7.] $\inner{\R}{\cdot}$ with $\inner{\R}{\cdot}$ where $\phi(x)=x^3$ for $x\in\R$. 

\answer Yes. It is clear from its graph that $\phi$ is a bijection. Also, for all $a,b\in\R$, we have $(a\cdot b)^3=a^3\cdot b^3$. 

\item[8.] $\inner{M_2(\R)}{\cdot}$ with $\inner{\R}{\cdot}$ where $\phi(A)$ is the determinant of matrix $A$. 

\answer No, since $\phi$ is not one-to-one. To see this, observe that for 
$A = 
\arraycolsep=1.4pt\def\arraystretch{1}
\left[\begin{array}{lr}
1 & 0 \\
1 & 0 \\
\end{array} \right]$ and 
$B = 
\arraycolsep=1.4pt\def\arraystretch{1}
\left[\begin{array}{lr}
2 & 0 \\
2 & 0 \\
\end{array} \right]$, we have $A\neq B$ but $\phi(A)=0=\phi(B)$.

\item[9.] $\inner{M_1(\R)}{\cdot}$ with $\inner{\R}{\cdot}$ where $\phi(A)$ is the determinant of matrix $A$. 

\answer Yes. Since any $A\in M_1(\R)$ is a matrix of the form $A=\left[a\right]$ where $a\in\R$, then $\det(A)=a$. So clearly, $\phi$ is a bijection. Now we apply the definition of matrix multipliaction. For any $A,B\in M_1(\R)$, we have $$\det(AB)=\det([a][b])=\det[ab]=a\cdot b=\det([a])\cdot\det([b])=\det(A)\cdot\det(B)$$ and we are done. 

\hspace*{-0.74cm}
\textbf{Same instructions. Let $F=\{f:\R\to\R \, \, | \, \, f(0)=0, f\in C^\infty\}$.}

\item[11.] $\inner{F}{+}$ with $\inner{F}{+}$ where $\phi(f)=f'$. 

\answer Yes, though verifying this does require a  bit of thought. For any $f, g\in F$, we have that $(f+g)'=f'+g'$ by linearity. We can also see that $\phi$ is onto, since every element of $F$ is smooth and therefore integrable. Thus, for any $g\in F$, we have that $f(x)=\int_0^x g(t)dt$ is an element of $F$ such that $\phi(f)=g$. Now to see that $\phi$ is one-to-one, we point out that $f(x)=\int_0^x g(t)dt$ is the \emph{only} element of $F$ which maps to $g$ under $\phi$. Let $h:\R\to\R$ be a function such that $\phi(h)=g$. Then, since $f$ and $h$ have the same derivative, they differ only by a constant. Thus, if $h(0)=0$, then $h=f$; and otherwise, $h\not\in F$; and we are done. 

\item[12.] $\inner{F}{+}$ with $\inner{\R}{+}$ where $\phi(f)=f'(0)$. 

\answer Let $f,g\in F$. We can see that $\phi$ commutes with the operations, since $$\phi(f+g)=(f+g)'(0)=f'(0)+g'(0)=\phi(f)+\phi(g).$$ However, $\phi$ is not one-to-one, since given some $x\in\R$, there are many functions in $F$ whose derivative at zero is $x$. For example, consider $x=0$. The identically zero function and $x^2$ are both in $F$, and $\phi(0)=\phi(x^2)=0$. 

\item[16.] The map $\phi:\Z\to\Z$ defined by $\phi(n)=n+1$ is a bijection. Give the definition of a binary operation $*$ on $\Z$ such that $\phi$ is an isomorphism of 
	\begin{enumerate}[label=\alph*.]
	\item $\inner{\Z}{+}$ with $\inner{\Z}{*}$
	\item $\inner{\Z}{*}$ with $\inner{\Z}{+}$
	\end{enumerate}
In each case, give the identity for $*$ on $\Z$. 

\answer \textbf{(a)} Let $*:(\Z\times\Z)\to\Z$ be defined as $$a*b=a+b-1$$ for all $a,b\in\Z$. To see that $\phi$ commutes with $+$ and $*$, let $n,m\in\Z.$ Now, 
$$\phi(n+m)=n+m+1=(n+1)+(m+1)-1=\phi(n)*\phi(m).$$
Note that the identity of $*$ is 1, since $n*1=1*n=n+1-1=n$. 

\answer \textbf{(b)} Let $*:(\Z\times\Z)\to\Z$ be defined as $$a*b=a+b+1$$ for all $a,b\in\Z$. To see that $\phi$ commutes with $*$ and $+$, let $n,m\in\Z.$ Now, 
$$\phi(n*m)=(n*m)+1=(n+m+1)+1=(n+1)+(m+1)=\phi(n)+\phi(m).$$
Note that the identity of $*$ is $-1$, since $n*(-1)=(-1)*n=n+(-1)+1=n$. 

\item[20.] The displayed condition for an isomorphism $\phi$ in Definition 1.2.7 is sometimes summarized by saying "$\phi$ must commute with the binary operation(s)".\footnote{I have been saying this in answers to previous problems, having already noticed this question.} Explain how that condition can be viewed in this manner. 

\answer If we think of $*$ and $*'$ as functions and use function notation (as opposed to binary operation notation), this commutative relationship is more clear. Suppose $\inner{A}{*}$ and $\inner{B}{*'}$ are two isomorphic binary structures, and $\phi:A\to B$ is an isomorphism relating them. We would usually write that for all $a_1, a_2 \in A$, 
$\phi(a_1*a_2)=\phi(a_1)*'\phi(a_2).$ However using function notation, we see the two functions commuting:
$$\phi(*(a_1,a_2))=*'(\phi(a_1),\phi(a_2))$$

\item[23.] An identity for a binary operation $*$ as described by Definition 1.2.12 is sometimes referred to as "a two-sided identity." Give analogous definitions for 

\textbf{a.} a \emph{left identity} $e_L$ for $*$, \quad\quad and \quad\quad \textbf{b.} a \emph{right identity} $e_R$ for $*$. 

\answer Let $*:(S\times S)\to S$ be binary operation on a set $S$. A \emph{left identity} for $*$ is some $e_L\in S$ such that for any $x\in S$, we have $e_L*x=x$. Similarly, a \emph{right identity} for $*$ is some $e_R\in S$ such that for any $x\in S$, we have $x*e_R=x$. \qedwhite

(Problem continued) Theorem 1.2.13 shows that if a two-sided identity for $*$ exists, then it is unique. Is the same true for a one-sided identity you just defined? Prove or give a counterexample $\inner{S}{*}$ for a finite set $S$ and find the first place where the proof of Theorem 1.2.13 breaks down. 

\answer The proof of Theorem 1.2.13 uses one identity as a left identity, and the other as a right identity. We can't do this here, and given two (WLOG) left identities $e$ and $e'$, we have no reason to believe that $e*e'=e'*e$, so while $e*e'=e'$ and $e'*e=e$, we cannot conclude that $e=e'$. Following is a table that give a counterexample: 
\[\begin{array}{c!c|c|c|c}
* & e & e'& a & b \\ \thickhline
e & e & e'& a & b \\ \hline
e'& e & e'& a & b \\ \hline
a & e'& a & b & e \\ \hline
b & a & b & e & e'\\ 
\end{array}\]
Note that $a*e=e'$ while $a*e'=a$, so since we cannot violate the substitution property of equality, we can see that $e\neq e'$. \qed

\item[24.] Can a binary structure have a left identity and a right identity which are distinct from each other?

\answer This is impossible. If there exist left and right identities $e$ and $e'$ respectively, we can apply the proof of Theorem 1.2.13: $e*e'=e$ (right identity), and $e*e'=e'$ (left identity). Thus $e=e'$ and there is one identity which is two-sided. \qed

\item[25.] Prove that if $\phi:S\to S'$ is an isomorphism of $\inner{S}{*}$ with $\inner{S'}{*'}$, then the inverse $\inv{\phi}$ is an isomorphism of $\inner{S'}{*'}$ with $\inner{S}{*}$. 

\begin{proof}
Since $\phi$ is a bijection, then so is $\inv{\phi}$, so all that remains is to show that $\inv{\phi}$ commutes with $*'$ and $*$. For any $a',b'\in S'$, there exists $a,b\in S$ such that $\phi(a)=a'$ and $\phi(b)=b'$. Since $\phi$ is an isomorphism, we know that 
$$\phi(a*b)=\phi(a)*'\phi(b)=a'*'b'.$$
Now, 
\[\begin{array}{rcl}
\inv{\phi}(a'*'b')&=& \inv{\phi}(\phi(a*b))\\

&=&a*b\\
&=&\inv{\phi}(\phi(a))*\inv{\phi}(\phi(b))\\
&=&\inv{\phi}(a')*\inv{\phi}(b')\\
\end{array}\]
and we are done.
\end{proof}

\hspace*{-0.74cm}
\textbf{For 28 through 31, prove that the indicated property of the binary structure $\inner{S}{*}$ is indeed a structural property.}

\item[28.] The operation $*$ is commutative. 
\begin{proof}
To prove that commutativity is a structural property of $\inner{S}{*}$, we will show that any binary structure which is isometric to $\inner{S}{*}$ must also have that property. Let $\phi$ be an isomorphism of $\inner{S}{*}$ with $\inner{S'}{*'}$, where $\inner{S'}{*'}$ is another structure. Since $*$ is commutative, we know that for all $a,b\in S$, $a*b=b*a$. Now we show that $*'$ is commutative as well. Let $c', d'\in S'$ be given. Then there exist unique $c,d\in S$ such that $\phi(c)=c', \phi(d)=d'$. Now, 
\[\begin{array}{rcl}
c'*'d'&=&\phi(c)*'\phi(d)\\
&=&\phi(c*d)\\
&=&\phi(d*c)\\
&=&\phi(d)*'\phi(c)\\
&=&d'*'c'\\
\end{array}\]
and we are done.
\end{proof}

\pagebreak
\item[29.] The operation $*$ is associative. 
\begin{proof}
Suppose $*$ is asociative, and define $a,b,c\in S$ and $*$ and $a',b',c'\in S'$ and $\phi$ similarly as above. Then, 
\[\begin{array}{rcl}
(a'*'b')*'c'&=&(\phi(a)*'\phi(b))*'c'\\
&=&\phi(a*b)*'\phi(c)\\
&=&\phi([a*b]*c)\\
&=&\phi(a*[b*c])\\
&=&\phi(a)*'\phi(b*c)\\
&=&a'*'(\phi(b)*'\phi(c))\\
&=&a'*'(b'*'c')\\
\end{array}\]
\end{proof}

\item[31.] There exists an element $b\in S$ such that $b*b=b$. 

\begin{proof}
Using notation as expected; if $b*b=b$, then $\phi(b*b)=\phi(b)$, so $\phi(b)*'\phi(b)=\phi(b)$, thus $b'*'b'=b'$. Therefore we have found an element $b'\in S'$ with the desired property, so we are done. 
\end{proof}

\item[32.] Let $H$ be the subset of $M_2(\R)$ consisting of all matrices of the form 
$\arraycolsep=1.4pt\def\arraystretch{1}
\left[\begin{array}{lr}
a & -b \\
b & a \\
\end{array} \right]$
for $a,b\in\R$. 
	\begin{enumerate}[label=\alph*.]
	\item Show that $\inner{\C}{+}$ is isomorphic to $\inner{H}{+}.$
	\item Show that $\inner{\C}{\cdot}$ is isomorphic to $\inner{H}{\cdot}.$
	\end{enumerate}
(We say that $H$ is a \emph{matrix representation} of the complex numbers $\C$.) 

\begin{proof} 
Let $\phi:\C\to H$ be defined as 
$\phi(a+bi)=
\arraycolsep=1.4pt\def\arraystretch{1}
\left[\begin{array}{lr}
a & -b \\
b & a \\
\end{array} \right].$ It should be clear that $\phi$ is one-to-one and onto, since this definition holds for all real numbers $a$ and $b$. Now we show that $\phi$ commutes with the operations in parts (a) and (b). 
	\begin{enumerate}[label=\alph*.]
	\item \[\def\arraystretch{2}
	\begin{array}{rcl}
	\phi\big((a+bi)+(c+di)\big)&=&\phi\big((a+c)+(b+d)i\big)\\
	&=&
		\arraycolsep=3pt\def\arraystretch{1}
		\left[\begin{array}{lr}
		(a+c) & -(b+d) \\
		(b+d) & (a+c) \\
		\end{array} \right]
		\\
	&=&
		\arraycolsep=3pt\def\arraystretch{1}
		\left[\begin{array}{lr}
		a & -b \\
		b & a  \\
		\end{array} \right]
		+
		\arraycolsep=3pt\def\arraystretch{1}
		\left[\begin{array}{lr}
		c & -d \\
		d & c  \\
		\end{array} \right]
		\\
	&=&\phi(a+bi)+\phi(c+di)\\
	\end{array}\]\qedwhite
	
	\item \[\def\arraystretch{2}
	\begin{array}{rcl}
	\phi\big((a+bi)(c+di)\big)&=&\phi\big((ac-bd)+(ad+bc)i\big)\\
	&=&
		\arraycolsep=3pt\def\arraystretch{1}
		\left[\begin{array}{lr}
		(ac-bd) & -(ad+bc) \\
		(ad+bc) & (ac-bd) \\
		\end{array} \right]
		\\
	&=&
		\arraycolsep=3pt\def\arraystretch{1}
		\left[\begin{array}{lr}
		a & -b \\
		b & a  \\
		\end{array} \right]
		%
		\arraycolsep=3pt\def\arraystretch{1}
		\left[\begin{array}{lr}
		c & -d \\
		d & c  \\
		\end{array} \right]
		\\
	&=&\phi(a+bi)\cdot\phi(c+di)\\
	\end{array}\]
	
	\end{enumerate}
\end{proof}

\end{enumerate}
\end{document}
