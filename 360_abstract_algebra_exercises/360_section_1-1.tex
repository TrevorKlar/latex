\documentclass[letterpaper]{article}
%\documentclass[a5paper]{article}

%% Language and font encodings
\usepackage[english]{babel}
\usepackage[utf8x]{inputenc}
\usepackage[T1]{fontenc}


%% Sets page size and margins
\usepackage[letterpaper,top=.75in,bottom=1in,left=1in,right=1in,marginparwidth=1.75cm]{geometry}
%\usepackage[a5paper,top=1cm,bottom=1cm,left=1cm,right=1.5cm,marginparwidth=1.75cm]{geometry}

\usepackage{graphicx}
%\graphicspath{../images}	  %%where to look for images

%% Useful packages
\usepackage{amssymb, amsmath, amsthm} 
%\usepackage{graphicx}  %%this is currently enabled in the default document, so it is commented out here. 
\usepackage{calrsfs}
\usepackage{braket}
\usepackage{mathtools}
\usepackage{lipsum}
\usepackage{tikz}
\usetikzlibrary{cd}
\usepackage{verbatim}
%\usepackage{ntheorem}% for theorem-like environments
\usepackage{mdframed}%can make highlighted boxes of text
%Use case: https://tex.stackexchange.com/questions/46828/how-to-highlight-important-parts-with-a-gray-background
\usepackage{wrapfig}
\usepackage{centernot}
\usepackage{subcaption}%\begin{subfigure}{0.5\textwidth}
\usepackage{pgfplots}
\pgfplotsset{compat=1.13}
\usepackage[colorinlistoftodos]{todonotes}
\usepackage[colorlinks=true, allcolors=blue]{hyperref}
\usepackage{xfrac}					%to make slanted fractions \sfrac{numerator}{denominator}
\usepackage{enumitem}            
    %syntax: \begin{enumerate}[label=(\alph*)]
    %possible arguments: f \alph*, \Alph*, \arabic*, \roman* and \Roman*
\usetikzlibrary{arrows,shapes.geometric,fit}

\DeclareMathAlphabet{\pazocal}{OMS}{zplm}{m}{n}
%% Use \pazocal{letter} to typeset a letter in the other kind 
%%  of math calligraphic font. 

%% This puts the QED block at the end of each proof, the way I like it. 
\renewenvironment{proof}{{\bfseries Proof}}{\qed}
\makeatletter
\renewenvironment{proof}[1][\bfseries \proofname]{\par
  \pushQED{\qed}%
  \normalfont \topsep6\p@\@plus6\p@\relax
  \trivlist
  %\itemindent\normalparindent
  \item[\hskip\labelsep
        \scshape
    #1\@addpunct{}]\ignorespaces
}{%
  \popQED\endtrivlist\@endpefalse
}
\makeatother

%% This adds a \rewnewtheorem command, which enables me to override the settings for theorems contained in this document.
\makeatletter
\def\renewtheorem#1{%
  \expandafter\let\csname#1\endcsname\relax
  \expandafter\let\csname c@#1\endcsname\relax
  \gdef\renewtheorem@envname{#1}
  \renewtheorem@secpar
}
\def\renewtheorem@secpar{\@ifnextchar[{\renewtheorem@numberedlike}{\renewtheorem@nonumberedlike}}
\def\renewtheorem@numberedlike[#1]#2{\newtheorem{\renewtheorem@envname}[#1]{#2}}
\def\renewtheorem@nonumberedlike#1{  
\def\renewtheorem@caption{#1}
\edef\renewtheorem@nowithin{\noexpand\newtheorem{\renewtheorem@envname}{\renewtheorem@caption}}
\renewtheorem@thirdpar
}
\def\renewtheorem@thirdpar{\@ifnextchar[{\renewtheorem@within}{\renewtheorem@nowithin}}
\def\renewtheorem@within[#1]{\renewtheorem@nowithin[#1]}
\makeatother

%% This makes theorems and definitions with names show up in bold, the way I like it. 
\makeatletter
\def\th@plain{%
  \thm@notefont{}% same as heading font
  \itshape % body font
}
\def\th@definition{%
  \thm@notefont{}% same as heading font
  \normalfont % body font
}
\makeatother

%===============================================
%==============Shortcut Commands================
%===============================================
\newcommand{\ds}{\displaystyle}
\newcommand{\B}{\mathcal{B}}
\newcommand{\C}{\mathbb{C}}
\newcommand{\F}{\mathbb{F}}
\newcommand{\N}{\mathbb{N}}
\newcommand{\R}{\mathbb{R}}
\newcommand{\Q}{\mathbb{Q}}
\newcommand{\T}{\mathcal{T}}
\newcommand{\Z}{\mathbb{Z}}
\renewcommand\qedsymbol{$\blacksquare$}
\newcommand{\qedwhite}{\hfill\ensuremath{\square}}
\newcommand*\conj[1]{\overline{#1}}
\newcommand*\closure[1]{\overline{#1}}
\newcommand*\mean[1]{\overline{#1}}
%\newcommand{\inner}[1]{\left< #1 \right>}
\newcommand{\inner}[2]{\left< #1, #2 \right>}
\newcommand{\powerset}[1]{\pazocal{P}(#1)}
%% Use \pazocal{letter} to typeset a letter in the other kind 
%%  of math calligraphic font. 
\newcommand{\cardinality}[1]{\left| #1 \right|}
\newcommand{\domain}[1]{\mathcal{D}(#1)}
\newcommand{\image}{\text{Im}}
\newcommand{\inv}[1]{#1^{-1}}
\newcommand{\preimage}[2]{#1^{-1}\left(#2\right)}
\newcommand{\script}[1]{\mathcal{#1}}


\newenvironment{highlight}{\begin{mdframed}[backgroundcolor=gray!20]}{\end{mdframed}}

\DeclarePairedDelimiter\ceil{\lceil}{\rceil}
\DeclarePairedDelimiter\floor{\lfloor}{\rfloor}

%===============================================
%===============My Tikz Commands================
%===============================================
\newcommand{\drawsquiggle}[1]{\draw[shift={(#1,0)}] (.005,.05) -- (-.005,.02) -- (.005,-.02) -- (-.005,-.05);}
\newcommand{\drawpoint}[2]{\draw[*-*] (#1,0.01) node[below, shift={(0,-.2)}] {#2};}
\newcommand{\drawopoint}[2]{\draw[o-o] (#1,0.01) node[below, shift={(0,-.2)}] {#2};}
\newcommand{\drawlpoint}[2]{\draw (#1,0.02) -- (#1,-0.02) node[below] {#2};}
\newcommand{\drawlbrack}[2]{\draw (#1+.01,0.02) --(#1,0.02) -- (#1,-0.02) -- (#1+.01,-0.02) node[below, shift={(-.01,0)}] {#2};}
\newcommand{\drawrbrack}[2]{\draw (#1-.01,0.02) --(#1,0.02) -- (#1,-0.02) -- (#1-.01,-0.02) node[below, shift={(+.01,0)}] {#2};}

%***********************************************
%**************Start of Document****************
%***********************************************

%===============================================
%===============Theorem Styles==================
%===============================================

%================Default Style==================
\theoremstyle{plain}% is the default. it sets the text in italic and adds extra space above and below the \newtheorems listed below it in the input. it is recommended for theorems, corollaries, lemmas, propositions, conjectures, criteria, and (possibly; depends on the subject area) algorithms.
\newtheorem{theorem}{Theorem}
\numberwithin{theorem}{section} %This sets the numbering system for theorems to number them down to the {argument} level. I have it set to number down to the {section} level right now.
\newtheorem*{theorem*}{Theorem} %Theorem with no numbering
\newtheorem{corollary}[theorem]{Corollary}
\newtheorem*{corollary*}{Corollary}
\newtheorem{conjecture}[theorem]{Conjecture}
\newtheorem{lemma}[theorem]{Lemma}
\newtheorem*{lemma*}{Lemma}
\newtheorem{proposition}[theorem]{Proposition}
\newtheorem*{proposition*}{Proposition}
\newtheorem{problemstatement}[theorem]{Problem Statement}


%==============Definition Style=================
\theoremstyle{definition}% adds extra space above and below, but sets the text in roman. it is recommended for definitions, conditions, problems, and examples; i've alse seen it used for exercises.
\newtheorem{definition}[theorem]{Definition}
\newtheorem*{definition*}{Definition}
\newtheorem{condition}[theorem]{Condition}
\newtheorem{problem}[theorem]{Problem}
\newtheorem{example}[theorem]{Example}
\newtheorem*{example*}{Example}
\newtheorem*{counterexample*}{Counterexample}
\newtheorem*{romantheorem*}{Theorem} %Theorem with no numbering
\newtheorem{exercise}{Exercise}
\numberwithin{exercise}{section}
\newtheorem{algorithm}[theorem]{Algorithm}

%================Remark Style===================
\theoremstyle{remark}% is set in roman, with no additional space above or below. it is recommended for remarks, notes, notation, claims, summaries, acknowledgments, cases, and conclusions.
\newtheorem{remark}[theorem]{Remark}
\newtheorem*{remark*}{Remark}
\newtheorem{notation}[theorem]{Notation}
\newtheorem*{notation*}{Notation}
%\newtheorem{claim}[theorem]{Claim}  %%use this if you ever want claims to be numbered
\newtheorem*{claim}{Claim}



\pgfplotsset{compat=1.13}

%\newcommand{\T}{\mathcal{T}}
%\newcommand{\B}{\mathcal{B}}

%These commands are now in tskpreamble_nothms.tex, but are left as a comment here for reference. 
%\newcommand{\arbcup}[1]{\bigcup\limits_{\alpha\in\Gamma}#1_\alpha}
%\newcommand{\arbcap}[1]{\bigcap\limits_{\alpha\in\Gamma}#1_\alpha}
%\newcommand{\arbcoll}[1]{\{#1_\alpha\}_{\alpha\in\Gamma}}
%\newcommand{\arbprod}[1]{\prod\limits_{\alpha\in\Gamma}#1_\alpha}
%\newcommand{\finitecoll}[1]{#1_1, \ldots, #1_n}
%\newcommand{\finitefuncts}[2]{#1(#2_1), \ldots, #1(#2_n)}
%\newcommand{\abs}[1]{\left|#1\right|}
%\newcommand{\norm}[1]{\left|\left|#1\right|\right|}

\title{Math 360 \linebreak
Section 1.1 Exercises}
\author{Trevor Klar}

\begin{document}

\maketitle

\begin{enumerate}
\item 
	\begin{enumerate}
	\item $b*d=e$
	\item $c*c=b$
	\item $[(a*c)*e]*a = [c*e]*a = a*a = a$
	\end{enumerate}
\item[7.] $*$ is clearly not commutative, since $1*2\neq 2*1$. Observe the following to see that $*$ is not associative either: $(1*2)*3=-1*3=-4$, whereas $1*(2*3)=1*-1=2$.
\item[9.] $*$ is commutative, since $\frac{ab}{2}=\frac{ba}{2}$. Also, $*$ is associative, since $(a*b)*c = \frac{ab}{2}*c = \frac{abc}{4} = a*\frac{bc}{2} = a*(b*c)$.
\item[11.] $*$ is commutative, since ${2}^{ab}={2}^{ba}$. However, $*$ is not associative, since $(3*5)*7=2^{15}*7=2^{(7)(2^{15})}$, but $3*(5*7)=3*2^{35}=2^{(3)(2^{35})}$. 
\item[13.] For a set of 2 elements, consider the table: 
\[\begin{array}{c!c|c}
* & a & b \\
\thickhline
a & x_1 & x_2\\
\hline
b & x_2 & x_3\\
\end{array}\]
There are three unique outputs, $x_1, x_2, x_3$, and each one can map to one of two possible values, so there are $2^3=8$ possible commutative binary operations on a set of 2 elements. 

For 3 elements, the table gives 
\[\begin{array}{c!c|c|c}
* & a & b & c\\ \thickhline
a & x_1 & x_2 & x_4 \\ \hline
b & x_2 & x_3 & x_5 \\ \hline
c & x_4 & x_5 & x_6
\end{array}\]
Thus, there are 6 unique outputs, which can each take 3 possible values. Therefore there are $3^6$ possible commutative binary operations on a set of 3 elements. 

As we increase the number of elements from $n-1$ to $n$, we see that the number of unique outputs increases by $n$. Thus, the number of unique outputs for a set of $n$ elements is given by the $n$-th triangle number. Thus, the number of possible commutative binary operations on a set of $n$ elements is given by 
$$ n^{T_n}, $$
where $T_n=\frac{(n)(n+1)}{2}$ is the $n$-th triangle number.
\item[14.] A binary operation $*$ on a set $S$ is \emph{commutative} if and only if, for all $a,b\in S$, we have $a*b=b*a$. 
\item[15.] A binary operation $*$ on a set $S$ is \emph{associative} if and only if, for all $a,b,c\in S$, we have $(a*b)*c=a*(b*c)$. 
\item[16.] A subset $H$ of a set $S$ is \emph{closed} under $*$ if and only if, for all $a,b\in H$, we have $(a*b)\in H$. 

\textbf{For Exercises 17-22, determine if $*$ is a binary operation on $S$. If not, state whether Condition 1, Condition 2, or both are violated. }
\item[17.] On $\Z^+$, define $*$ by $a*b=a-b$. 
\textbf{Answer: } No. Condition 2 is violated since $1,2\in \Z^+$, but $1*2=-1\not\in\Z^+$. 
\item[19.] On $\R$, define $*$ by $a*b=a-b$. This is of course a binary operation. For every subtraction problem, there is exactly one answer, and every difference of real numbers is a real number. 
\item[21.] On $\Z^+$, define $a*b=c$, where $c$ is at least 5 more than $a+b$. \textbf{Answer: } No. Condition 1 is violated, since there are infinitely many real numbers $c$ such that $c\geq a+b+5$. 
\item[23.] Let $H$ be the subset of $M_2(\R)$ consisting of all matrices of the form 
$\arraycolsep=1.4pt\def\arraystretch{1}
\left[\begin{array}{lr}
a & -b \\
b & a \\
\end{array} \right]$ 
for $a,b\in \R$. Is $H$ closed under
\begin{enumerate}
	\item[a.] matrix addition?
	\item[b.] matrix multiplication?
\end{enumerate}

\textbf{Answer to a:} Yes. Adding two arbitrary elements of $H$, we find  
$\arraycolsep=1.4pt\def\arraystretch{1}
\left[\begin{array}{lr}
a & -b \\
b & a \\
\end{array} \right]
+
\arraycolsep=1.4pt\def\arraystretch{1}
\left[\begin{array}{lr}
c & -d \\
d & c \\
\end{array} \right] 
=
\arraycolsep=1.4pt\def\arraystretch{1}
\left[\begin{array}{lr}
a+c & \, \, -(b+d) \\
b+d & a+c\phantom{)} \\
\end{array} \right],$
which is an element of $H$.  

\textbf{Answer to b:} Yes. Multiplying two arbitrary elements of $H$, we find  
$\arraycolsep=1.4pt\def\arraystretch{1}
\left[\begin{array}{lr}
a & -b \\
b & a \\
\end{array} \right]
\arraycolsep=1.4pt\def\arraystretch{1}
\left[\begin{array}{lr}
c & -d \\
d & c \\
\end{array} \right] 
=
\arraycolsep=1.4pt\def\arraystretch{1}
\left[\begin{array}{lr}
ac-bd & \, \, \, -(ad+bc) \\
ad+bc & ac-bd\phantom{)} \\
\end{array} \right],$
which is an element of $H$.  

\item[24.] 
	\begin{enumerate}[label=\alph*.]
	\item F
	\item T
	\item F
	\item F
	\item F
	\item T
	\item F
	\item F
	\item T
	\item F
	\end{enumerate}

\item[26.] Prove that if $*$ is and associative and commutative binary operation on a set $S$, then 
$$(a*b)*(c*d)=[(d*c)*a]*b$$
for all $a,b,c,d\in S$. 
\begin{proof}
\[\begin{array}{rcll}
(a*b)*(c*d) &=& (c*d)*(a*b) & \text{commutative property}\\
&=& (d*c)*(a*b) & \text{commutative property}\\
&=& [(d*c)*a]*b & \text{associative property}\\
\end{array}\]
\end{proof}

\textbf{In 27 and 28, prove or give a counterexample.}
\item[27.] Every binary operation on a set consisting of a single element is both commutative and associative.
\begin{proof}
Denote our singleton set as $S=\{e\}$, and let $*$ be a binary operation on $S$. Applying the definition of commutativity, we can see that $*$ is commutative: for all $a,b\in S$, we have that $a*b=b*a$. To see this, observe that since $a\in S, a=e$. Also, since $b\in S, b=e$. Thus, $a*b=e*e=b*a$. 

We can apply the definition of associativity similarly if we observe that $e*e$ can only have one possible result: $e*e=e$. Thus, for all $a,b,c\in S$, 
\[\begin{array}{rcl}
(a*b)*c&=&(e*e)*e\\
&=&e*e\\
&=&e*(e*e)\\
&=&a*(b*c)\\
\end{array}\]
and we are done. \end{proof}

\item[28.] Every commutative binary operation on a set having just two elements is associative. 

\textbf{Counterexample:} Consider the set $S=\{\blacksquare, \square\}$, with $*$ defined by the following table:
\[\begin{array}{c!c|c}
* & \blacksquare & \square \\
\thickhline
\blacksquare & \square & \blacksquare\\
\hline
\square & \blacksquare & \blacksquare\\
\end{array}\]
We will show that $(\square*\blacksquare)*\blacksquare \neq \square*(\blacksquare*\blacksquare)$, and thus the associative property does not hold. 
$$(\square*\blacksquare)*\blacksquare = \blacksquare*\blacksquare = \square,$$
however, 
$$\square*(\blacksquare*\blacksquare)=\square*\square=\blacksquare.$$
Thus $*$ on $S$ is not commutative (observe that the table is symmetric), but not associative. \qed

\item[41.] This is the state diagram for a machine which determines if the input string has at least two $c$'s. If the final state is $s_2$, then the input string has at least two $c$'s.

\jpg{scale=.5}{360_hw1-1_prob_41}

\item[42.] This is the state diagram for a machine which determines if the input string has exactly 3 $c$'s. If the final state is $s_3$, then the input string has exactly 3 $c$'s. Note that if the final state is $s_f$, then the input string has greater than 3 $c$'s, and if the final state is $s_0, s_1,$ or $s_2$, then the input string has less than 3 $c$'s. 

\jpg{scale=.5}{360_hw1-1_prob_42}

\item[43.] This is the state diagram for a machine which determines if the number of 1's in the input string is congruent to 0, 1, or 2 modulo 3. $s_0$ is the initial state, and corresponds to $0\equiv 0$ mod 3. The subscript of the final state gives the result. 

\jpg{scale=.5}{360_hw1-1_prob_43}



\end{enumerate}
\end{document}
