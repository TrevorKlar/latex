\documentclass{article}
\usepackage[paperwidth=.5\paperwidth,paperheight=.25\paperheight]{geometry}
\usepackage{pgfpages}
\pagestyle{empty}
\thispagestyle{empty}
\pgfpagesuselayout{8 on 1}[a4paper]
\makeatletter
\@tempcnta=1\relax
\loop\ifnum\@tempcnta<9\relax
\pgf@pset{\the\@tempcnta}{bordercode}{\pgfusepath{stroke}}
\advance\@tempcnta by 1\relax
\repeat
\makeatother

%% Useful packages
\usepackage{amssymb, amsmath, amsthm} 
%\usepackage{graphicx}  %%this is currently enabled in the default document, so it is commented out here. 
\usepackage{calrsfs}
\usepackage{braket}
\usepackage{mathtools}
\usepackage{lipsum}
\usepackage{tikz}
\usetikzlibrary{cd}
\usepackage{verbatim}
%\usepackage{ntheorem}% for theorem-like environments
\usepackage{mdframed}%can make highlighted boxes of text
%Use case: https://tex.stackexchange.com/questions/46828/how-to-highlight-important-parts-with-a-gray-background
\usepackage{wrapfig}
\usepackage{centernot}
\usepackage{subcaption}%\begin{subfigure}{0.5\textwidth}
\usepackage{pgfplots}
\pgfplotsset{compat=1.13}
\usepackage[colorinlistoftodos]{todonotes}
\usepackage[colorlinks=true, allcolors=blue]{hyperref}
\usepackage{xfrac}					%to make slanted fractions \sfrac{numerator}{denominator}
\usepackage{enumitem}            
    %syntax: \begin{enumerate}[label=(\alph*)]
    %possible arguments: f \alph*, \Alph*, \arabic*, \roman* and \Roman*
\usetikzlibrary{arrows,shapes.geometric,fit}

\DeclareMathAlphabet{\pazocal}{OMS}{zplm}{m}{n}
%% Use \pazocal{letter} to typeset a letter in the other kind 
%%  of math calligraphic font. 

%% This puts the QED block at the end of each proof, the way I like it. 
\renewenvironment{proof}{{\bfseries Proof}}{\qed}
\makeatletter
\renewenvironment{proof}[1][\bfseries \proofname]{\par
  \pushQED{\qed}%
  \normalfont \topsep6\p@\@plus6\p@\relax
  \trivlist
  %\itemindent\normalparindent
  \item[\hskip\labelsep
        \scshape
    #1\@addpunct{}]\ignorespaces
}{%
  \popQED\endtrivlist\@endpefalse
}
\makeatother

%% This adds a \rewnewtheorem command, which enables me to override the settings for theorems contained in this document.
\makeatletter
\def\renewtheorem#1{%
  \expandafter\let\csname#1\endcsname\relax
  \expandafter\let\csname c@#1\endcsname\relax
  \gdef\renewtheorem@envname{#1}
  \renewtheorem@secpar
}
\def\renewtheorem@secpar{\@ifnextchar[{\renewtheorem@numberedlike}{\renewtheorem@nonumberedlike}}
\def\renewtheorem@numberedlike[#1]#2{\newtheorem{\renewtheorem@envname}[#1]{#2}}
\def\renewtheorem@nonumberedlike#1{  
\def\renewtheorem@caption{#1}
\edef\renewtheorem@nowithin{\noexpand\newtheorem{\renewtheorem@envname}{\renewtheorem@caption}}
\renewtheorem@thirdpar
}
\def\renewtheorem@thirdpar{\@ifnextchar[{\renewtheorem@within}{\renewtheorem@nowithin}}
\def\renewtheorem@within[#1]{\renewtheorem@nowithin[#1]}
\makeatother

%% This makes theorems and definitions with names show up in bold, the way I like it. 
\makeatletter
\def\th@plain{%
  \thm@notefont{}% same as heading font
  \itshape % body font
}
\def\th@definition{%
  \thm@notefont{}% same as heading font
  \normalfont % body font
}
\makeatother

%===============================================
%==============Shortcut Commands================
%===============================================
\newcommand{\ds}{\displaystyle}
\newcommand{\B}{\mathcal{B}}
\newcommand{\C}{\mathbb{C}}
\newcommand{\F}{\mathbb{F}}
\newcommand{\N}{\mathbb{N}}
\newcommand{\R}{\mathbb{R}}
\newcommand{\Q}{\mathbb{Q}}
\newcommand{\T}{\mathcal{T}}
\newcommand{\Z}{\mathbb{Z}}
\renewcommand\qedsymbol{$\blacksquare$}
\newcommand{\qedwhite}{\hfill\ensuremath{\square}}
\newcommand*\conj[1]{\overline{#1}}
\newcommand*\closure[1]{\overline{#1}}
\newcommand*\mean[1]{\overline{#1}}
%\newcommand{\inner}[1]{\left< #1 \right>}
\newcommand{\inner}[2]{\left< #1, #2 \right>}
\newcommand{\powerset}[1]{\pazocal{P}(#1)}
%% Use \pazocal{letter} to typeset a letter in the other kind 
%%  of math calligraphic font. 
\newcommand{\cardinality}[1]{\left| #1 \right|}
\newcommand{\domain}[1]{\mathcal{D}(#1)}
\newcommand{\image}{\text{Im}}
\newcommand{\inv}[1]{#1^{-1}}
\newcommand{\preimage}[2]{#1^{-1}\left(#2\right)}
\newcommand{\script}[1]{\mathcal{#1}}


\newenvironment{highlight}{\begin{mdframed}[backgroundcolor=gray!20]}{\end{mdframed}}

\DeclarePairedDelimiter\ceil{\lceil}{\rceil}
\DeclarePairedDelimiter\floor{\lfloor}{\rfloor}

%===============================================
%===============My Tikz Commands================
%===============================================
\newcommand{\drawsquiggle}[1]{\draw[shift={(#1,0)}] (.005,.05) -- (-.005,.02) -- (.005,-.02) -- (-.005,-.05);}
\newcommand{\drawpoint}[2]{\draw[*-*] (#1,0.01) node[below, shift={(0,-.2)}] {#2};}
\newcommand{\drawopoint}[2]{\draw[o-o] (#1,0.01) node[below, shift={(0,-.2)}] {#2};}
\newcommand{\drawlpoint}[2]{\draw (#1,0.02) -- (#1,-0.02) node[below] {#2};}
\newcommand{\drawlbrack}[2]{\draw (#1+.01,0.02) --(#1,0.02) -- (#1,-0.02) -- (#1+.01,-0.02) node[below, shift={(-.01,0)}] {#2};}
\newcommand{\drawrbrack}[2]{\draw (#1-.01,0.02) --(#1,0.02) -- (#1,-0.02) -- (#1-.01,-0.02) node[below, shift={(+.01,0)}] {#2};}

%***********************************************
%**************Start of Document****************
%***********************************************
 %find me at /home/trevor/texmf/tex/latex/tskpreamble_nothms.tex

\newenvironment{flashcard}[2][]{%
\noindent  \textsc{#1}

\vfill 
\centerline{{\Large{#2}}}
\vfill
\newpage \vspace*{\stretch{1}} \noindent
}
{\vspace*{\stretch{1}}\newpage}

\usepackage[latin1]{inputenc}
\usepackage{amsfonts}
\usepackage{amsmath}

\begin{document}


\begin{flashcard}[Definition]{Center of a ring $R$}
\begin{itemize}
\item $\{z\in R | zr=rz \text{ for all } r\in R\},$
\item "The set of all elements which commute with $R$." 
\end{itemize}
\end{flashcard}

\begin{flashcard}[Proof Technique]{Subring Criterion for $S\subset R$}
\begin{itemize}
\item $S\neq\emptyset$
\item $x-y\in S$ (closed under subtraction)
\item $xy\in S$ (closed under multiplication)
\end{itemize}
\end{flashcard}

\begin{flashcard}[Definition]{Characteristic of a ring $R$}
The characteristic $\text{char}(R)$ is the smallest positive number $n$ such that
$${\displaystyle \underbrace {1+\cdots +1} _{n{\text{ summands}}}=0.}$$
This also means that any element vanishes when added to itself this many times. 
\end{flashcard}

\begin{flashcard}[Definition]{Ring}
\begin{itemize}
\item $(R,+)$ is an abelian group (associative, identity, inverse, commutative) 
\item $(R,\times)$ is a monoid (associative, identity) 
\item $\times$ distributes over $+$ from either side. (distributive)
\end{itemize}
\end{flashcard}

\begin{flashcard}[Definition]{Unique Factorization Domain}
An integral domain $R$ in which every non-zero element $x\in R$ can be written as a product (an empty product if $x$ is a unit) of irreducible elements $p_i$ of $R$ and a unit $u$:
$$x = u \, p_1 \, p_2 \dots p_n \quad \text{with } n \geq 0$$
\end{flashcard}

\begin{flashcard}[Definition]{Principal Ideal Domain (PID)}
An integral domain in which every ideal is a principal ideal. 
\end{flashcard}

\begin{flashcard}[Definition]{Principal Ideal}
An ideal $I\ideal R$ generated by a single element. That is if $\langle a \rangle = I$, start with $a\in R$, and make all the elements possible by multiplying something in $R$ by $a$, and then make all elements possible by finite sums of those elements. 
\end{flashcard}

\begin{flashcard}[Definition]{Discrete Valuation}
$$v:R^\times\rightarrowtail \Z \text{ such that}$$
\begin{itemize}
\item $v$ is surjective
\item $v(ab)=v(a)+v(b)$
\item $v(x+y)\geq\min\{v(x),v(y)\} \quad \forall x+y\neq0$
\end{itemize}
\end{flashcard}

\begin{flashcard}[Definition]{Ideal $S\ideal R$}
\begin{itemize}
\item $S\neq\emptyset$
\item $S$ closed under subtraction
\item $rs, sr \in S \quad \forall s\in S, r\in R$. ($S$ absorbs multiplicands in $R$.)
\end{itemize}
\end{flashcard}

\begin{flashcard}[Proposition]{
\begin{tabular}{c}
Let $\varphi:R\to S$ be a homomorphism. \\ \\
What do we know about $\image{\varphi}$ and $\ker{\varphi}$?
\end{tabular}
}

\begin{itemize}
\item $\image{\varphi} \undertext{\subset}{ring} S$
\item $\ker\varphi \ideal R$
\end{itemize}
\end{flashcard}

\begin{flashcard}[Definition]{Augmentation ideal of $RG$}
An element in the augmentation ideal of a group ring is of the form $\sum r_ig_i$, where $\sum r_i=0$. 
\end{flashcard}

\begin{flashcard}[Definition]{Nilradical}
The nilradical of a ring is an ideal consisting of all the nilpotent elements, that is, 

$$\{r\in R : r^k=0 \text{ for some }k\}$$
\end{flashcard}

\begin{flashcard}[Definition]{Radical of ideal $I$}
The radical of a ring ideal $I$ is itself an ideal consisting of all the $I$-potent elements, that is, 

$$\{r\in R : r^k\in I \text{ for some }k\}$$
\end{flashcard}

\begin{flashcard}[Definition]{Group Ring}
Let $R$ be a commutative ring with $1\neq0$ and $G$ a \textbf{finite }multiplicative group. Then $RG$ is %the set of all formal sums 
$$a_1g_1 + \dots + a_n g_n \quad a_i\in R.$$
with addition defined "componentwise":

$\sum_{i=1}^n a_ig_i + \sum_{i=1}^n b_ig_i = \sum_{i=1}^n (a_i+b_i)g_i$

\noindent and multiplication defined by 

$(ag_i)(bg_j)=(ab)(g_ig_j)=cg_k$

\noindent and extending via the distributive property (taking care if $R$ is not commutative). 
\end{flashcard}

\begin{flashcard}[Theorem]{The First Isomorphism Theorem for Rings}
Let $\phi:R\to S$ be a ring homomorphism. Then 
\begin{itemize}
\item $\ker(\phi)\ideal R$,
\item $\phi(R)\subring S$, and 
\item $\quotient{R}{\ker(\phi)} \cong \phi(R)$. 
\end{itemize}
\end{flashcard}

\begin{flashcard}[Theorem]{The Second Isomorphism Theorem for Rings}
Let $A\subring	R$, $I\ideal R$. Then 
\begin{itemize}
\item $A+I=\{a+i:a\in A, i\in I\}\subring R$,
\item $A\cap I\ideal A$, and 
\item $\quotient{(A+I)}{I}\cong \quotient{A}{(A\cap I)}$.
\end{itemize}
\end{flashcard}

\begin{flashcard}[Theorem]{The Third Isomorphism Theorem for Rings}
Let $I,J\ideal R$ with $I\subseteq J$. Then 
\begin{itemize}
\item $\quotient{J}{I}\ideal\quotient{R}{I}$
\item $\quotient{\left(\quotient{R}{I}\right )}{\left (\quotient{J}{I}\right )}\cong	\quotient{R}{J}$. 
\end{itemize}
\end{flashcard}



\end{document}