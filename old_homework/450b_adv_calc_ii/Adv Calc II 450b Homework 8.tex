\documentclass[letterpaper]{article}
%\documentclass[a5paper]{article}

%% Language and font encodings
\usepackage[english]{babel}
\usepackage[utf8x]{inputenc}
\usepackage[T1]{fontenc}


%% Sets page size and margins
\usepackage[letterpaper,top=.75in,bottom=1in,left=1in,right=1in,marginparwidth=1.75cm]{geometry}
%\usepackage[a5paper,top=1cm,bottom=1cm,left=1cm,right=1.5cm,marginparwidth=1.75cm]{geometry}

\usepackage{graphicx}
%\graphicspath{../images}	  %%where to look for images

%% Useful packages
\usepackage{amssymb, amsmath, amsthm} 
%\usepackage{graphicx}  %%this is currently enabled in the default document, so it is commented out here. 
\usepackage{calrsfs}
\usepackage{braket}
\usepackage{mathtools}
\usepackage{lipsum}
\usepackage{tikz}
\usetikzlibrary{cd}
\usepackage{verbatim}
%\usepackage{ntheorem}% for theorem-like environments
\usepackage{mdframed}%can make highlighted boxes of text
%Use case: https://tex.stackexchange.com/questions/46828/how-to-highlight-important-parts-with-a-gray-background
\usepackage{wrapfig}
\usepackage{centernot}
\usepackage{subcaption}%\begin{subfigure}{0.5\textwidth}
\usepackage{pgfplots}
\pgfplotsset{compat=1.13}
\usepackage[colorinlistoftodos]{todonotes}
\usepackage[colorlinks=true, allcolors=blue]{hyperref}
\usepackage{xfrac}					%to make slanted fractions \sfrac{numerator}{denominator}
\usepackage{enumitem}            
    %syntax: \begin{enumerate}[label=(\alph*)]
    %possible arguments: f \alph*, \Alph*, \arabic*, \roman* and \Roman*
\usetikzlibrary{arrows,shapes.geometric,fit}

\DeclareMathAlphabet{\pazocal}{OMS}{zplm}{m}{n}
%% Use \pazocal{letter} to typeset a letter in the other kind 
%%  of math calligraphic font. 

%% This puts the QED block at the end of each proof, the way I like it. 
\renewenvironment{proof}{{\bfseries Proof}}{\qed}
\makeatletter
\renewenvironment{proof}[1][\bfseries \proofname]{\par
  \pushQED{\qed}%
  \normalfont \topsep6\p@\@plus6\p@\relax
  \trivlist
  %\itemindent\normalparindent
  \item[\hskip\labelsep
        \scshape
    #1\@addpunct{}]\ignorespaces
}{%
  \popQED\endtrivlist\@endpefalse
}
\makeatother

%% This adds a \rewnewtheorem command, which enables me to override the settings for theorems contained in this document.
\makeatletter
\def\renewtheorem#1{%
  \expandafter\let\csname#1\endcsname\relax
  \expandafter\let\csname c@#1\endcsname\relax
  \gdef\renewtheorem@envname{#1}
  \renewtheorem@secpar
}
\def\renewtheorem@secpar{\@ifnextchar[{\renewtheorem@numberedlike}{\renewtheorem@nonumberedlike}}
\def\renewtheorem@numberedlike[#1]#2{\newtheorem{\renewtheorem@envname}[#1]{#2}}
\def\renewtheorem@nonumberedlike#1{  
\def\renewtheorem@caption{#1}
\edef\renewtheorem@nowithin{\noexpand\newtheorem{\renewtheorem@envname}{\renewtheorem@caption}}
\renewtheorem@thirdpar
}
\def\renewtheorem@thirdpar{\@ifnextchar[{\renewtheorem@within}{\renewtheorem@nowithin}}
\def\renewtheorem@within[#1]{\renewtheorem@nowithin[#1]}
\makeatother

%% This makes theorems and definitions with names show up in bold, the way I like it. 
\makeatletter
\def\th@plain{%
  \thm@notefont{}% same as heading font
  \itshape % body font
}
\def\th@definition{%
  \thm@notefont{}% same as heading font
  \normalfont % body font
}
\makeatother

%===============================================
%==============Shortcut Commands================
%===============================================
\newcommand{\ds}{\displaystyle}
\newcommand{\B}{\mathcal{B}}
\newcommand{\C}{\mathbb{C}}
\newcommand{\F}{\mathbb{F}}
\newcommand{\N}{\mathbb{N}}
\newcommand{\R}{\mathbb{R}}
\newcommand{\Q}{\mathbb{Q}}
\newcommand{\T}{\mathcal{T}}
\newcommand{\Z}{\mathbb{Z}}
\renewcommand\qedsymbol{$\blacksquare$}
\newcommand{\qedwhite}{\hfill\ensuremath{\square}}
\newcommand*\conj[1]{\overline{#1}}
\newcommand*\closure[1]{\overline{#1}}
\newcommand*\mean[1]{\overline{#1}}
%\newcommand{\inner}[1]{\left< #1 \right>}
\newcommand{\inner}[2]{\left< #1, #2 \right>}
\newcommand{\powerset}[1]{\pazocal{P}(#1)}
%% Use \pazocal{letter} to typeset a letter in the other kind 
%%  of math calligraphic font. 
\newcommand{\cardinality}[1]{\left| #1 \right|}
\newcommand{\domain}[1]{\mathcal{D}(#1)}
\newcommand{\image}{\text{Im}}
\newcommand{\inv}[1]{#1^{-1}}
\newcommand{\preimage}[2]{#1^{-1}\left(#2\right)}
\newcommand{\script}[1]{\mathcal{#1}}


\newenvironment{highlight}{\begin{mdframed}[backgroundcolor=gray!20]}{\end{mdframed}}

\DeclarePairedDelimiter\ceil{\lceil}{\rceil}
\DeclarePairedDelimiter\floor{\lfloor}{\rfloor}

%===============================================
%===============My Tikz Commands================
%===============================================
\newcommand{\drawsquiggle}[1]{\draw[shift={(#1,0)}] (.005,.05) -- (-.005,.02) -- (.005,-.02) -- (-.005,-.05);}
\newcommand{\drawpoint}[2]{\draw[*-*] (#1,0.01) node[below, shift={(0,-.2)}] {#2};}
\newcommand{\drawopoint}[2]{\draw[o-o] (#1,0.01) node[below, shift={(0,-.2)}] {#2};}
\newcommand{\drawlpoint}[2]{\draw (#1,0.02) -- (#1,-0.02) node[below] {#2};}
\newcommand{\drawlbrack}[2]{\draw (#1+.01,0.02) --(#1,0.02) -- (#1,-0.02) -- (#1+.01,-0.02) node[below, shift={(-.01,0)}] {#2};}
\newcommand{\drawrbrack}[2]{\draw (#1-.01,0.02) --(#1,0.02) -- (#1,-0.02) -- (#1-.01,-0.02) node[below, shift={(+.01,0)}] {#2};}

%***********************************************
%**************Start of Document****************
%***********************************************

%===============================================
%===============Theorem Styles==================
%===============================================

%================Default Style==================
\theoremstyle{plain}% is the default. it sets the text in italic and adds extra space above and below the \newtheorems listed below it in the input. it is recommended for theorems, corollaries, lemmas, propositions, conjectures, criteria, and (possibly; depends on the subject area) algorithms.
\newtheorem{theorem}{Theorem}
\numberwithin{theorem}{section} %This sets the numbering system for theorems to number them down to the {argument} level. I have it set to number down to the {section} level right now.
\newtheorem*{theorem*}{Theorem} %Theorem with no numbering
\newtheorem{corollary}[theorem]{Corollary}
\newtheorem*{corollary*}{Corollary}
\newtheorem{conjecture}[theorem]{Conjecture}
\newtheorem{lemma}[theorem]{Lemma}
\newtheorem*{lemma*}{Lemma}
\newtheorem{proposition}[theorem]{Proposition}
\newtheorem*{proposition*}{Proposition}
\newtheorem{problemstatement}[theorem]{Problem Statement}


%==============Definition Style=================
\theoremstyle{definition}% adds extra space above and below, but sets the text in roman. it is recommended for definitions, conditions, problems, and examples; i've alse seen it used for exercises.
\newtheorem{definition}[theorem]{Definition}
\newtheorem*{definition*}{Definition}
\newtheorem{condition}[theorem]{Condition}
\newtheorem{problem}[theorem]{Problem}
\newtheorem{example}[theorem]{Example}
\newtheorem*{example*}{Example}
\newtheorem*{counterexample*}{Counterexample}
\newtheorem*{romantheorem*}{Theorem} %Theorem with no numbering
\newtheorem{exercise}{Exercise}
\numberwithin{exercise}{section}
\newtheorem{algorithm}[theorem]{Algorithm}

%================Remark Style===================
\theoremstyle{remark}% is set in roman, with no additional space above or below. it is recommended for remarks, notes, notation, claims, summaries, acknowledgments, cases, and conclusions.
\newtheorem{remark}[theorem]{Remark}
\newtheorem*{remark*}{Remark}
\newtheorem{notation}[theorem]{Notation}
\newtheorem*{notation*}{Notation}
%\newtheorem{claim}[theorem]{Claim}  %%use this if you ever want claims to be numbered
\newtheorem*{claim}{Claim}



\pgfplotsset{compat=1.13}

%\newcommand{\T}{\mathcal{T}}
%\newcommand{\B}{\mathcal{B}}

%These commands are now in tskpreamble_nothms.tex, but are left as a comment here for reference. 
%\newcommand{\arbcup}[1]{\bigcup\limits_{\alpha\in\Gamma}#1_\alpha}
%\newcommand{\arbcap}[1]{\bigcap\limits_{\alpha\in\Gamma}#1_\alpha}
%\newcommand{\arbcoll}[1]{\{#1_\alpha\}_{\alpha\in\Gamma}}
%\newcommand{\arbprod}[1]{\prod\limits_{\alpha\in\Gamma}#1_\alpha}
%\newcommand{\finitecoll}[1]{#1_1, \ldots, #1_n}
%\newcommand{\finitefuncts}[2]{#1(#2_1), \ldots, #1(#2_n)}
%\newcommand{\abs}[1]{\left|#1\right|}
%\newcommand{\norm}[1]{\left|\left|#1\right|\right|}

\title{Math 450b \linebreak
Homework 8}
\author{Trevor Klar}

\begin{document}

\maketitle

\begin{enumerate}
\item Let $f:\R^n\to \R^n$ and suppose there is a constant $M$ such that $\norm{f(\vecb{x})}\leq M\norm{\vecb{x}}^2$ for all $\vecb{x}\in \R^n$. Let $g(\vecb{x})=T(\vecb{x})+f(\vecb{x})$, where $T:\R^n\to \R^n$ is an invertible linear transformation. Prove that $g$ is locally invertible near $\vecb{0} $. 
\begin{proof}
To prove that $g$ is locally invertible near $\vecb{0} $, we will show that $\det \left(Dg(\vecb{0})\right)\neq0$. Since
$$g(\vecb{x})=T(\vecb{x})+f(\vecb{x}),$$
Then by linearity of derivatives, 
\[\begin{array}{rcl}
Dg(\vecb{0})(\vecb{x})&=&DT(\vecb{0})(\vecb{x})+Df(\vecb{0})(\vecb{x})\\
&=&T(\vecb{x})+Df(\vecb{0})(\vecb{x})
\end{array}\]

By an earlier homework problem, since $\norm{f(\vecb{x})}\leq M\norm{\vecb{x}}^2$ for all $\vecb{x}\in \R^n$, then $Df(\vecb{0})\equiv\vecb{0}$. Thus, 
$$Dg(\vecb{0})(\vecb{x})=T(\vecb{x}),$$
ans since $T$ is invertible, $\det(Dg(\vecb{0}))=\det(T)\neq 0$. 
\end{proof}

\item Determine whether the system 
\[\begin{array}{rcl}
u &=& x+xyz\\
v &=& y+xy\\
w &=& z+2x+3z^2\\
\end{array}\]
can be solved for $x,y,z$ in terms of $u,v,w$ near $(0,0,0)$. 
\begin{proof}
Let $f:\R^3\to \R^3$ be defined as $f(x,y,z)=(u,v,w)$. Note that $f(\vecb{0})=\vecb{0}$. We seek some $\inv{f}(u,v,w)=(x,y,z)$ near $\vecb{0}$. If $\det(Df(\vecb{0}))\neq0$ (We can already can see that $f$ is continuous), then the Inverse Function Theorem guarantees the desired function. 
\[Df=\left[
\begin{array}{ccc}
1+yz & xz & xy\\
y & 1+x & 0\\
z & 0 & 1+6z\\
\end{array}\right]
\quad
Df(\vecb{0})=\left[
\begin{array}{ccc}
1&0&0\\
0&1&0\\
2&0&1\\
\end{array}\right]
\]
$Df(\vecb{0})$ is lower-triangular, so $\det Df(\vecb{0})=(1)(1)(1)\neq0$. Thus, $f$ is locally invertible near $\vecb{0}$ and we are done. 
\end{proof}

\item Suppose $f:U\subset \R^n\to \R^n$ is $C^1$ and one-to-one, with $Df(\vecb{a})\neq0$ for all $\vecb{a}\in U$. Prove that $f(U)$ is an open set. 
\begin{proof}
Let $\vecb{y}\in f(U)$ with $f(\vecb{x})=\vecb{y}$ for some $\vecb{x}\in U$. Since $\det(Df(\vecb{x}))\neq0$ by assumption, the Inverse Function Thm gives open sets $V, W$ such that 
$$\vecb{x}\in V \subset U, \quad \vecb{y}\in W \subset f(U),$$
thus $f(U)$ is open by the openness criterion. 
\end{proof}

\item Determine whether the system 
\[\begin{array}{rcl}
3x+2y+z^2+u+v^2 &=& 0\\
4x+3y+z+u^2+v+w+2 &=& 0\\
x+z+w+u^2+2 &=& 0\\
\end{array}\]
can be solved for $u,v,w$ in terms of $x,y,z$ near $x=y=z=u=v=0, w=-2$.
\begin{proof}
Denote the point $$x=y=z=u=v=0, w=-2$$ as $-2\vecb{e}_w$. Let $F:\R^6\to\R^3$ be defined as 
\[
F(x,y,z,u,v,w)=\left[
\begin{array}{c}
3x+2y+z^2+u+v^2\\
4x+3y+z+u^2+v+w+2\\
x+z+w+u^2+2\\
\end{array}\right]
\]
Observe that $F(-2\vecb{e}_w)=\vecb{0}$. If 
$$\det\left(\left[\frac{\del F_i}{\del j}(-2\vecb{e}_w)\right]_{\substack{i\in\{1,2,3\}\\j\in\{u,v,w\}}}\right)\neq0,$$
then the Implicit Function Theorem gives open sets $V_1\in \R^3, V_2\in \R^3$ with $\vecb{0}\in V_1$ and $(0,0,-2)\in V_2$, and some $f:V_1\to V_2$ such that $F(x,y,z,f(x,y,z))=\vecb{0}$ for all $(x,y)\in V_1$. Thus, we calculate the above determinant. 
\[
\det\left(\left[\frac{\del F_i}{\del j}(-2\vecb{e}_w)\right]_{\substack{i\in\{1,2,3\}\\j\in\{u,v,w\}}}\right)
=
\det\left(\left.\left[
\begin{array}{ccc}
1 & 2v & 0 \\
2u & 1 & 1 \\
2u & 0 & 1 \\
\end{array}
\right]\right|_{-2\vecb{e}_w}\right)
=
\left|\begin{array}{ccc}
1 & 0 & 0 \\
0 & 1 & 1 \\
0 & 0 & 1 \\
\end{array}\right|
=
1\neq0
\]
as desired, and we are done.
\end{proof}

\item Show that the equations 
\[\begin{array}{rcl}
x^2-y^2-u^3+v^2+4 &=& 0\\
2xy+y^2-2u^2+3v^4+8 &=& 0\\
\end{array}\]
determine functions $u(x,y), v(x,y)$ near $x=2, y=-1$ such that $u(2,-1)=2$, $v(2,-1)=1$. Compute $\frac{\del u}{\del x}$. 

\begin{proof}
Let $F:\R^4\to\R^2$ be defined as 
\[
F(x,y,u,v)=\left[
\begin{array}{c}
x^2-y^2-u^3+v^2+4\\
2xy+y^2-2u^2+3v^4+8\\
\end{array}\right]_.
\]
Observe that $F(2,-1,2,1)=\vecb{0}$. If 
$$\det\left(\left[\frac{\del F_i}{\del j}(2,-1,2,1)\right]_{\substack{i\in\{1,2\}\\j\in\{u,v\}}}\right)\neq0,$$
then the Implicit Function Theorem gives open sets $V_1\in \R^2, V_2\in \R^2$ with $(2,-1)\in V_1$ and $(2,1)\in V_2$, and some $f:V_1\to V_2$ such that $F(x,y,f(x,y))=F(x,y,u(x,y),v(x,y))=\vecb{0}$ for all $(x,y)\in V_1$. Thus, we calculate the above determinant. 
\[
\det\left(\left[\frac{\del F_i}{\del j}(2,-1,2,1)\right]_{\substack{i\in\{1,2\}\\j\in\{u,v\}}}\right)
=
\det\left(\left.\left[
\begin{array}{ccc}
-3u^2 & 2v \\
-4u & 12v^3 \\
\end{array}
\right]\right|_{(2,-1,2,1)}\right)
=
\left|\begin{array}{ccc}
-12 & 2 \\
-8 & 12 \\
\end{array}\right|
=
-144+16\neq0
\]
as desired, and we are done.
\end{proof}
\textbf{Solution} Now we calculate $\frac{\del u}{\del x}$ by differentiating $F(x,y,u,v)=\vecb{0}$ implicitly. 
\[
\frac{\del}{\del x}
\left[\begin{array}{c}
x^2-y^2-u^3+v^2+4\\
2xy+y^2-2u^2+3v^4+8\\
\end{array}\right]
=
\frac{\del}{\del x}
\left[\begin{array}{c}
0\\
0\\
\end{array}\right]
\]
This gives the following system of equations, if we consider $u$ and $v$ to be functions of $x$ but $y$ to be constant with respect to $x$:
\[
\begin{array}{rrrcl}
2x&-3u^2\frac{\del u}{\del x}&+2v\frac{\del v}{\del x} &=& 0\\
2y&-4u\frac{\del u}{\del x}&+12v^3\frac{\del v}{\del x} &=& 0\\
\end{array}
\]
Solving this first equation for $\frac{\del v}{\del x}$ gives 
$$\frac{\del v}{\del x} = \frac{3u^2}{2v}\frac{\del u}{\del x}-\frac{x}{v},$$
and we can substitute this into the second equation to find 
$$\frac{\del u}{\del x}=\frac{6v^2x-y}{9v^2u^2-2u}$$
and we are done. \qed

\end{enumerate}
\end{document}
