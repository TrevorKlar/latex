\documentclass[letterpaper]{article}
%\documentclass[a5paper]{article}

%% Language and font encodings
\usepackage[english]{babel}
\usepackage[utf8x]{inputenc}
\usepackage[T1]{fontenc}


%% Sets page size and margins
\usepackage[letterpaper,top=.75in,bottom=1in,left=1in,right=1in,marginparwidth=1.75cm]{geometry}
%\usepackage[a5paper,top=1cm,bottom=1cm,left=1cm,right=1.5cm,marginparwidth=1.75cm]{geometry}

\usepackage{graphicx}
%\graphicspath{../images}	  %%where to look for images

%% Useful packages
\usepackage{amssymb, amsmath, amsthm} 
%\usepackage{graphicx}  %%this is currently enabled in the default document, so it is commented out here. 
\usepackage{calrsfs}
\usepackage{braket}
\usepackage{mathtools}
\usepackage{lipsum}
\usepackage{tikz}
\usetikzlibrary{cd}
\usepackage{verbatim}
%\usepackage{ntheorem}% for theorem-like environments
\usepackage{mdframed}%can make highlighted boxes of text
%Use case: https://tex.stackexchange.com/questions/46828/how-to-highlight-important-parts-with-a-gray-background
\usepackage{wrapfig}
\usepackage{centernot}
\usepackage{subcaption}%\begin{subfigure}{0.5\textwidth}
\usepackage{pgfplots}
\pgfplotsset{compat=1.13}
\usepackage[colorinlistoftodos]{todonotes}
\usepackage[colorlinks=true, allcolors=blue]{hyperref}
\usepackage{xfrac}					%to make slanted fractions \sfrac{numerator}{denominator}
\usepackage{enumitem}            
    %syntax: \begin{enumerate}[label=(\alph*)]
    %possible arguments: f \alph*, \Alph*, \arabic*, \roman* and \Roman*
\usetikzlibrary{arrows,shapes.geometric,fit}

\DeclareMathAlphabet{\pazocal}{OMS}{zplm}{m}{n}
%% Use \pazocal{letter} to typeset a letter in the other kind 
%%  of math calligraphic font. 

%% This puts the QED block at the end of each proof, the way I like it. 
\renewenvironment{proof}{{\bfseries Proof}}{\qed}
\makeatletter
\renewenvironment{proof}[1][\bfseries \proofname]{\par
  \pushQED{\qed}%
  \normalfont \topsep6\p@\@plus6\p@\relax
  \trivlist
  %\itemindent\normalparindent
  \item[\hskip\labelsep
        \scshape
    #1\@addpunct{}]\ignorespaces
}{%
  \popQED\endtrivlist\@endpefalse
}
\makeatother

%% This adds a \rewnewtheorem command, which enables me to override the settings for theorems contained in this document.
\makeatletter
\def\renewtheorem#1{%
  \expandafter\let\csname#1\endcsname\relax
  \expandafter\let\csname c@#1\endcsname\relax
  \gdef\renewtheorem@envname{#1}
  \renewtheorem@secpar
}
\def\renewtheorem@secpar{\@ifnextchar[{\renewtheorem@numberedlike}{\renewtheorem@nonumberedlike}}
\def\renewtheorem@numberedlike[#1]#2{\newtheorem{\renewtheorem@envname}[#1]{#2}}
\def\renewtheorem@nonumberedlike#1{  
\def\renewtheorem@caption{#1}
\edef\renewtheorem@nowithin{\noexpand\newtheorem{\renewtheorem@envname}{\renewtheorem@caption}}
\renewtheorem@thirdpar
}
\def\renewtheorem@thirdpar{\@ifnextchar[{\renewtheorem@within}{\renewtheorem@nowithin}}
\def\renewtheorem@within[#1]{\renewtheorem@nowithin[#1]}
\makeatother

%% This makes theorems and definitions with names show up in bold, the way I like it. 
\makeatletter
\def\th@plain{%
  \thm@notefont{}% same as heading font
  \itshape % body font
}
\def\th@definition{%
  \thm@notefont{}% same as heading font
  \normalfont % body font
}
\makeatother

%===============================================
%==============Shortcut Commands================
%===============================================
\newcommand{\ds}{\displaystyle}
\newcommand{\B}{\mathcal{B}}
\newcommand{\C}{\mathbb{C}}
\newcommand{\F}{\mathbb{F}}
\newcommand{\N}{\mathbb{N}}
\newcommand{\R}{\mathbb{R}}
\newcommand{\Q}{\mathbb{Q}}
\newcommand{\T}{\mathcal{T}}
\newcommand{\Z}{\mathbb{Z}}
\renewcommand\qedsymbol{$\blacksquare$}
\newcommand{\qedwhite}{\hfill\ensuremath{\square}}
\newcommand*\conj[1]{\overline{#1}}
\newcommand*\closure[1]{\overline{#1}}
\newcommand*\mean[1]{\overline{#1}}
%\newcommand{\inner}[1]{\left< #1 \right>}
\newcommand{\inner}[2]{\left< #1, #2 \right>}
\newcommand{\powerset}[1]{\pazocal{P}(#1)}
%% Use \pazocal{letter} to typeset a letter in the other kind 
%%  of math calligraphic font. 
\newcommand{\cardinality}[1]{\left| #1 \right|}
\newcommand{\domain}[1]{\mathcal{D}(#1)}
\newcommand{\image}{\text{Im}}
\newcommand{\inv}[1]{#1^{-1}}
\newcommand{\preimage}[2]{#1^{-1}\left(#2\right)}
\newcommand{\script}[1]{\mathcal{#1}}


\newenvironment{highlight}{\begin{mdframed}[backgroundcolor=gray!20]}{\end{mdframed}}

\DeclarePairedDelimiter\ceil{\lceil}{\rceil}
\DeclarePairedDelimiter\floor{\lfloor}{\rfloor}

%===============================================
%===============My Tikz Commands================
%===============================================
\newcommand{\drawsquiggle}[1]{\draw[shift={(#1,0)}] (.005,.05) -- (-.005,.02) -- (.005,-.02) -- (-.005,-.05);}
\newcommand{\drawpoint}[2]{\draw[*-*] (#1,0.01) node[below, shift={(0,-.2)}] {#2};}
\newcommand{\drawopoint}[2]{\draw[o-o] (#1,0.01) node[below, shift={(0,-.2)}] {#2};}
\newcommand{\drawlpoint}[2]{\draw (#1,0.02) -- (#1,-0.02) node[below] {#2};}
\newcommand{\drawlbrack}[2]{\draw (#1+.01,0.02) --(#1,0.02) -- (#1,-0.02) -- (#1+.01,-0.02) node[below, shift={(-.01,0)}] {#2};}
\newcommand{\drawrbrack}[2]{\draw (#1-.01,0.02) --(#1,0.02) -- (#1,-0.02) -- (#1-.01,-0.02) node[below, shift={(+.01,0)}] {#2};}

%***********************************************
%**************Start of Document****************
%***********************************************

%===============================================
%===============Theorem Styles==================
%===============================================

%================Default Style==================
\theoremstyle{plain}% is the default. it sets the text in italic and adds extra space above and below the \newtheorems listed below it in the input. it is recommended for theorems, corollaries, lemmas, propositions, conjectures, criteria, and (possibly; depends on the subject area) algorithms.
\newtheorem{theorem}{Theorem}
\numberwithin{theorem}{section} %This sets the numbering system for theorems to number them down to the {argument} level. I have it set to number down to the {section} level right now.
\newtheorem*{theorem*}{Theorem} %Theorem with no numbering
\newtheorem{corollary}[theorem]{Corollary}
\newtheorem*{corollary*}{Corollary}
\newtheorem{conjecture}[theorem]{Conjecture}
\newtheorem{lemma}[theorem]{Lemma}
\newtheorem*{lemma*}{Lemma}
\newtheorem{proposition}[theorem]{Proposition}
\newtheorem*{proposition*}{Proposition}
\newtheorem{problemstatement}[theorem]{Problem Statement}


%==============Definition Style=================
\theoremstyle{definition}% adds extra space above and below, but sets the text in roman. it is recommended for definitions, conditions, problems, and examples; i've alse seen it used for exercises.
\newtheorem{definition}[theorem]{Definition}
\newtheorem*{definition*}{Definition}
\newtheorem{condition}[theorem]{Condition}
\newtheorem{problem}[theorem]{Problem}
\newtheorem{example}[theorem]{Example}
\newtheorem*{example*}{Example}
\newtheorem*{counterexample*}{Counterexample}
\newtheorem*{romantheorem*}{Theorem} %Theorem with no numbering
\newtheorem{exercise}{Exercise}
\numberwithin{exercise}{section}
\newtheorem{algorithm}[theorem]{Algorithm}

%================Remark Style===================
\theoremstyle{remark}% is set in roman, with no additional space above or below. it is recommended for remarks, notes, notation, claims, summaries, acknowledgments, cases, and conclusions.
\newtheorem{remark}[theorem]{Remark}
\newtheorem*{remark*}{Remark}
\newtheorem{notation}[theorem]{Notation}
\newtheorem*{notation*}{Notation}
%\newtheorem{claim}[theorem]{Claim}  %%use this if you ever want claims to be numbered
\newtheorem*{claim}{Claim}



\pgfplotsset{compat=1.13}

%\newcommand{\T}{\mathcal{T}}
%\newcommand{\B}{\mathcal{B}}

%These commands are now in tskpreamble_nothms.tex, but are left as a comment here for reference. 
%\newcommand{\arbcup}[1]{\bigcup\limits_{\alpha\in\Gamma}#1_\alpha}
%\newcommand{\arbcap}[1]{\bigcap\limits_{\alpha\in\Gamma}#1_\alpha}
%\newcommand{\arbcoll}[1]{\{#1_\alpha\}_{\alpha\in\Gamma}}
%\newcommand{\arbprod}[1]{\prod\limits_{\alpha\in\Gamma}#1_\alpha}
%\newcommand{\finitecoll}[1]{#1_1, \ldots, #1_n}
%\newcommand{\finitefuncts}[2]{#1(#2_1), \ldots, #1(#2_n)}
%\newcommand{\abs}[1]{\left|#1\right|}
%\newcommand{\norm}[1]{\left|\left|#1\right|\right|}

\title{Math 450b \linebreak
Homework 11}
\author{Trevor Klar}

\begin{document}

\maketitle

\begin{enumerate}
\item Show that the volume of a parallelepiped spanned by the vectors $v_1, \dots, v_n$ in $\R^n$ is given by $\abs{\det M}^{\frac{1}{2}}$, where $M= [\inner{v_i}{v_j}]$.
\begin{proof}
Let $P$ denote the parallelepiped in question. Observe that 
\[M= [\inner{v_i}{v_j}] = 
\left[
\begin{array}{c}
v_1\\
v_2\\
v_3\\
\vdots\\
v_n\\
\end{array}
\right]
\left[v_1\,\, v_2\,\, v_3\,\, \cdots\,\, v_n\right]
=T^T T,\]
Where $T$ is the matrix representing the linear transformation which maps the unit cube to $P$. So,
$$\vol(P)^2=|\det T|^2=|\det T^T||\det T|=\abs{\det M},$$
and taking square roots, we find that $\vol(P)=\abs{\det M}^\frac{1}{2}$, as desired. 
\end{proof} 

\item Use a change of variables to calculate $\int_A f$, where 
$$f(x,y,z) = (x^2+y^2)z^2,$$
$$A=\{(x,y,z):x^2+y^2<1, \,\, \abs{z}<1\}.$$

\textbf{Answer:} Let $g:B\subset\R^3\to\R^3$ be a function, and $B$ be a set such that 
$$g(r,\theta, z)=(r\cos\theta, r\sin\theta, z),$$
$$B=\{(r,\theta, z):r<1, \,\,0<\theta<2\pi, \,\,\abs{z}<1\}.$$ Observe that $g(B)=A$ with $g$ being one-to-one and $C^1$ with $\det Dg\neq0$ for all $(r,\theta, z)$ in $B$ (we claim these facts without proof since this is a common change of variables). Then by the Change of Variables Thm, 
$$\int_A f=\int_B f\circ g \abs{\det Dg}
=\int_{-1}^1 \int_0^{2\pi} \int_0^1 r^2 z^2 \abs{r} dr \, d\theta \, dz =\int_{-1}^1 \int_0^{2\pi} \tfrac{z^2}{4} d\theta \, dz=2\pi(2)\frac{1}{12}=\frac{\pi}{3}.$$\qed

\pagebreak
\item Use a change of variables to calculate $\int_A f$, where 
$$f(x,y)=xy\sin (x^2-y^2),$$ 
$$A=\{(x,y):0<y<1, y<x, x^2-y^2<1\}.$$

\textbf{Answer:} Let $g:B\subset\R^2\to\R^2$ be a function, and $B$ be a set such that 
$$g(u,v)=(\sqrt{u+v^2},v),$$
$$B=\{(u,v):0<u<1, 0<v<1\}.$$ 
Observe that $x^2-y^2=u$, so $u<1$, and $v=y<x=\sqrt{u+v^2}$, so $u>0$. Thus $g(B)=A$ with $g$ being one-to-one and $C^1$ for all $(r,\theta, z)$ in $B$. Now we compute $\abs{\det Dg}$. 
\[\abs{\det Dg}=
\left|\det\left[\def\arraystretch{1.618}
\begin{array}{cc}
\frac{1}{2\sqrt{u+v^2}} & \frac{v}{\sqrt{u+v^2}}\\
0&1\\
\end{array}
\right]\right|
= \abs{\frac{1}{2\sqrt{u+v^2}}} = \frac{1}{2\sqrt{u+v^2}}
\]
Then by the Change of Variables Thm, 
$$\int_A f=\int_B f\circ g \abs{\det Dg}=\int_0^1 \int_0^1 \frac{v\sqrt{u+v^2}\sin{u}}{2\sqrt{u+v^2}} du\, dv=\int_0^1 \int_0^1 \frac{v}{2}\sin u \, du \, dv = \frac{1-\cos(1)}{4}.$$\qed


\item Give a counterexample to show that the change of variable formula does not hold if $g$ is not one-to-one, even if $\det Dg\neq0$ for all $x\in\Omega$. (Hint: Take $f=1$ and $g(x,y) = (e^x\cos y, e^x\sin y)$ for a suitable region $\Omega$.)

\textbf{Answer:} Let $f:\R^2\to\R$ and $g:\Omega\subset\R^2\to\R^2$ be defined by 
\[\begin{array}{rcl}
f&\equiv&1\\
g(x,y) &=& (e^x\cos y, e^x\sin y)\\
\end{array}\]
and consider the regions 
\[\begin{array}{rcl}
A&=&B(\vec{0},1)-B(\vec{0},\frac{1}{e})\\
\Omega&=&\{(x,y):-1<x<0, \, 0<y<4\pi\}.\\
\end{array}\]
Observe that $g(\Omega)=A$, although $g$ is not one-to-one. Now we compute $\abs{\det Dg}$. 
\[\abs{\det Dg}=
\left|\det\left[
\begin{array}{cc}
e^x\cos y &  -e^x\sin y \\
e^x\sin y &  e^x\cos y \\
\end{array}
\right]\right|
= \abs{e^{2x}} = e^{2x}
\]
Note that $\det Dg\neq0$ for all $x\in\Omega$. Now we compare the two halves of the change of variables formula: 
$$\int_A f\stackrel{?}{=}\int_B f\circ g \abs{\det Dg}.$$
$$\int_A f = \vol(A) = \pi(1)^2-\pi(\tfrac{1}{e})^2 = \pi-\frac{\pi}{e^2}$$
$$\int_B f\circ g \abs{\det Dg} = \int_{-1}^0 \int_0^{4\pi} (1)e^{2x}\, dy\, dx = 2\left(\pi-\frac{\pi}{e^2}\right)$$
Thus, the RHS$\neq$LHS, so the formula does not hold. \qed

\pagebreak
\item \begin{enumerate}
\item Calculate $\int_{B_r} e^{-x^2-y^2} dx \, dy$, where $B_r=\{(x,y): x^2+y^2\leq r\}$. \\
\textbf{Answer:} Using a change of variables to polar coordinates, we find that 
$$\int_{B_r} e^{-x^2-y^2} dx \, dy = \int_0^{2\pi}\int_0^{\sqrt{r}}u e^{-u^2}\, du\, d\theta = -\frac{1}{2} \int_0^{2\pi}\int_0^{\sqrt{r}}-2u e^{-u^2}\, du\, d\theta= -\frac{1}{2} \int_0^{2\pi} e^{-r}-1\, d\theta$$
$$=\pi-\pi e^{-r}$$
\item Show that $\int_{C_r} e^{-x^2-y^2} dx \, dy = (\int_{-r}^r e^{-x^2} dx)^2$, where $C_r=[-r,r]\times[-r,r]$. 
\begin{proof}
$$\int_{C_r} e^{-x^2-y^2} dx \, dy = \int_{-r}^r\int_{-r}^r e^{-x^2}e^{-y^2} dx \, dy= \int_{-r}^r e^{-x^2} dx\, \int_{-r}^r e^{-y^2} dy=\left(\int_{-r}^r e^{-x^2} dx\right)^2$$
\end{proof}
\item Show that 
$$\lim_{r\to\infty} \int_{B_r} e^{-x^2-y^2} dx \, dy = \lim_{r\to\infty} \int_{C_r} e^{-x^2-y^2} dx \, dy$$
\begin{proof} First, observe that the LHS converges to $\pi$:
$$\lim_{r\to\infty} \int_{B_r} e^{-x^2-y^2} dx \, dy = \lim_{r\to\infty} \left(\pi-\pi e^{-r}\right) = \pi.$$
Now, since $B_r\subset C_r\subset B_{r\sqrt{2}}$ for any $r>0$, and $e^{-x^2-y^2}>0$ for all $(x,y)\in\R^2$, then 
$$\int_{B_r} e^{-x^2-y^2} dx \, dy \leq \int_{C_r} e^{-x^2-y^2} dx \, dy \leq \int_{B_{r\sqrt{2}}} e^{-x^2-y^2} dx \, dy.$$
Thus, by the squeeze theorem, since 
$$\pi = \lim_{r\to\infty} \int_{B_r} e^{-x^2-y^2} dx \, dy =\lim_{r\to\infty} \int_{B_{r\sqrt{2}}} e^{-x^2-y^2} dx \, dy,$$
Then the RHS also converges to $\pi$. 
\end{proof}
\item Show that $\int_{-\infty}^\infty e^{-x^2} dx = \sqrt{\pi}$. 
\begin{proof} Using parts (a) thorugh (c):
$$\int_{-\infty}^\infty e^{-x^2} dx = \sqrt{\left(\int_{-\infty}^\infty e^{-x^2} dx\right)^2}=\sqrt{\left(\lim_{r\to\infty}\int_{-r}^r e^{-x^2} dx\right)^2}=$$
$$\sqrt{\lim_{r\to\infty}\left(\int_{-r}^r e^{-x^2} dx\right)^2}=\sqrt{\lim_{r\to\infty}\int_{C_r} e^{-x^2-y^2} dx \, dy}= \sqrt{\pi}$$
\end{proof}
\end{enumerate}

\item Let $E$  be the ellipsoid $\{(x,y,z)\in\R^3: (x^2/a^2)+(y^2/b^2)+(z^2/c^2)\leq 1\}$, where $a,b,$ and $c$ are positive constants. Compute the volume of $E$ using a change of variables. 

\textbf{Answer:} Perform a change of variables using $T(u,v,w)=(au, bv, cw)$. Thus, $T(B(\vec{0},1))=E$, so 
$$\vol(E)=\int_E 1 = \int\limits_{B(\vec{0},1)} \abs{\det T} = \vol(B(\vec{0},1))\abs{\det T}=\tfrac{4}{3}\pi abc.$$
To see that $\det T=abc$, observe that $a, b,$ and $c$ are the eigenvalues of $T$, so the determinant is equal to their product. \qed


\item Let $\langle e_1, \dots, e_n \rangle$ denote the standard basis for $\R^n$, and let $T$ denote the linear operator on $\R^n$ defined by $T(e_1)=(1,1,1,1,\dots,1), T(e_2)=(1,2,1,1,\dots,1), T(e_3)=(1,2,3,1,\dots,1), \dots, T(e_n)=(1,2,3,4,\dots,n).$ Suppose that $f:\Omega\to \R$ is integrable, and $\int_\Omega f=1$. Compute $\int_{\preimage{T}{\Omega}}f\circ T$. 

\textbf{Answer:} First, observe that $\int_\Omega f=\int_{\preimage{T}{\Omega}}f\circ T\abs{\det T}=1$. Thus, $\int_{\preimage{T}{\Omega}}f\circ T=\frac{1}{\abs{\det T}}$. So, we need to compute $\abs{\det T}$. Let $A$ denote the matrix representation of $T$ with respect to the standard basis. Thus:
\[|\det A|=
\abs{\det \left[
\begin{array}{ccccc}
1 & 1 & 1 & \cdots & 1\\
1 & 2 & 2 & \cdots & 2\\
1 & 1 & 3 & \cdots & 3\\
&\vdots&&\ddots&\vdots\\
1 & 1 & 1 & \cdots & n\\
\end{array}
\right]}
\]
To compute the determinant, we apply row operations to reduce $A$ to triangular form, adding $-R1+Ri$ for every row except the first. This will affect the determinant by a sign if $n-1$ is odd, but we are taking absolute value, so it doesn't matter. 
\[|\det A|=
\abs{\det \left[
\begin{array}{ccccc}
1 & 1 & 1 & \cdots & 1\\
0 & 1 & 1 & \cdots & 1\\
0 & 0 & 2 & \cdots & 2\\
&\vdots&&\ddots&\vdots\\
0 & 0 & 0 & \cdots & (n-1)\\
\end{array}
\right]}
=(n-1)!
\]
Therefore, $$\int_{\preimage{T}{\Omega}}f\circ T=\frac{1}{(n-1)!}$$ and we are done. \qed
\end{enumerate}
\end{document}
