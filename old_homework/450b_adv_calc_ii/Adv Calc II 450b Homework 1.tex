\documentclass[letterpaper]{article}
%\documentclass[a5paper]{article}

%% Language and font encodings
\usepackage[english]{babel}
\usepackage[utf8x]{inputenc}
\usepackage[T1]{fontenc}


%% Sets page size and margins
\usepackage[letterpaper,top=.75in,bottom=1in,left=1in,right=1in,marginparwidth=1.75cm]{geometry}
%\usepackage[a5paper,top=1cm,bottom=1cm,left=1cm,right=1.5cm,marginparwidth=1.75cm]{geometry}

\usepackage{graphicx}
%\graphicspath{../images}	  %%where to look for images

%% Useful packages
\usepackage{amssymb, amsmath, amsthm} 
%\usepackage{graphicx}  %%this is currently enabled in the default document, so it is commented out here. 
\usepackage{calrsfs}
\usepackage{braket}
\usepackage{mathtools}
\usepackage{lipsum}
\usepackage{tikz}
\usetikzlibrary{cd}
\usepackage{verbatim}
%\usepackage{ntheorem}% for theorem-like environments
\usepackage{mdframed}%can make highlighted boxes of text
%Use case: https://tex.stackexchange.com/questions/46828/how-to-highlight-important-parts-with-a-gray-background
\usepackage{wrapfig}
\usepackage{centernot}
\usepackage{subcaption}%\begin{subfigure}{0.5\textwidth}
\usepackage{pgfplots}
\pgfplotsset{compat=1.13}
\usepackage[colorinlistoftodos]{todonotes}
\usepackage[colorlinks=true, allcolors=blue]{hyperref}
\usepackage{xfrac}					%to make slanted fractions \sfrac{numerator}{denominator}
\usepackage{enumitem}            
    %syntax: \begin{enumerate}[label=(\alph*)]
    %possible arguments: f \alph*, \Alph*, \arabic*, \roman* and \Roman*
\usetikzlibrary{arrows,shapes.geometric,fit}

\DeclareMathAlphabet{\pazocal}{OMS}{zplm}{m}{n}
%% Use \pazocal{letter} to typeset a letter in the other kind 
%%  of math calligraphic font. 

%% This puts the QED block at the end of each proof, the way I like it. 
\renewenvironment{proof}{{\bfseries Proof}}{\qed}
\makeatletter
\renewenvironment{proof}[1][\bfseries \proofname]{\par
  \pushQED{\qed}%
  \normalfont \topsep6\p@\@plus6\p@\relax
  \trivlist
  %\itemindent\normalparindent
  \item[\hskip\labelsep
        \scshape
    #1\@addpunct{}]\ignorespaces
}{%
  \popQED\endtrivlist\@endpefalse
}
\makeatother

%% This adds a \rewnewtheorem command, which enables me to override the settings for theorems contained in this document.
\makeatletter
\def\renewtheorem#1{%
  \expandafter\let\csname#1\endcsname\relax
  \expandafter\let\csname c@#1\endcsname\relax
  \gdef\renewtheorem@envname{#1}
  \renewtheorem@secpar
}
\def\renewtheorem@secpar{\@ifnextchar[{\renewtheorem@numberedlike}{\renewtheorem@nonumberedlike}}
\def\renewtheorem@numberedlike[#1]#2{\newtheorem{\renewtheorem@envname}[#1]{#2}}
\def\renewtheorem@nonumberedlike#1{  
\def\renewtheorem@caption{#1}
\edef\renewtheorem@nowithin{\noexpand\newtheorem{\renewtheorem@envname}{\renewtheorem@caption}}
\renewtheorem@thirdpar
}
\def\renewtheorem@thirdpar{\@ifnextchar[{\renewtheorem@within}{\renewtheorem@nowithin}}
\def\renewtheorem@within[#1]{\renewtheorem@nowithin[#1]}
\makeatother

%% This makes theorems and definitions with names show up in bold, the way I like it. 
\makeatletter
\def\th@plain{%
  \thm@notefont{}% same as heading font
  \itshape % body font
}
\def\th@definition{%
  \thm@notefont{}% same as heading font
  \normalfont % body font
}
\makeatother

%===============================================
%==============Shortcut Commands================
%===============================================
\newcommand{\ds}{\displaystyle}
\newcommand{\B}{\mathcal{B}}
\newcommand{\C}{\mathbb{C}}
\newcommand{\F}{\mathbb{F}}
\newcommand{\N}{\mathbb{N}}
\newcommand{\R}{\mathbb{R}}
\newcommand{\Q}{\mathbb{Q}}
\newcommand{\T}{\mathcal{T}}
\newcommand{\Z}{\mathbb{Z}}
\renewcommand\qedsymbol{$\blacksquare$}
\newcommand{\qedwhite}{\hfill\ensuremath{\square}}
\newcommand*\conj[1]{\overline{#1}}
\newcommand*\closure[1]{\overline{#1}}
\newcommand*\mean[1]{\overline{#1}}
%\newcommand{\inner}[1]{\left< #1 \right>}
\newcommand{\inner}[2]{\left< #1, #2 \right>}
\newcommand{\powerset}[1]{\pazocal{P}(#1)}
%% Use \pazocal{letter} to typeset a letter in the other kind 
%%  of math calligraphic font. 
\newcommand{\cardinality}[1]{\left| #1 \right|}
\newcommand{\domain}[1]{\mathcal{D}(#1)}
\newcommand{\image}{\text{Im}}
\newcommand{\inv}[1]{#1^{-1}}
\newcommand{\preimage}[2]{#1^{-1}\left(#2\right)}
\newcommand{\script}[1]{\mathcal{#1}}


\newenvironment{highlight}{\begin{mdframed}[backgroundcolor=gray!20]}{\end{mdframed}}

\DeclarePairedDelimiter\ceil{\lceil}{\rceil}
\DeclarePairedDelimiter\floor{\lfloor}{\rfloor}

%===============================================
%===============My Tikz Commands================
%===============================================
\newcommand{\drawsquiggle}[1]{\draw[shift={(#1,0)}] (.005,.05) -- (-.005,.02) -- (.005,-.02) -- (-.005,-.05);}
\newcommand{\drawpoint}[2]{\draw[*-*] (#1,0.01) node[below, shift={(0,-.2)}] {#2};}
\newcommand{\drawopoint}[2]{\draw[o-o] (#1,0.01) node[below, shift={(0,-.2)}] {#2};}
\newcommand{\drawlpoint}[2]{\draw (#1,0.02) -- (#1,-0.02) node[below] {#2};}
\newcommand{\drawlbrack}[2]{\draw (#1+.01,0.02) --(#1,0.02) -- (#1,-0.02) -- (#1+.01,-0.02) node[below, shift={(-.01,0)}] {#2};}
\newcommand{\drawrbrack}[2]{\draw (#1-.01,0.02) --(#1,0.02) -- (#1,-0.02) -- (#1-.01,-0.02) node[below, shift={(+.01,0)}] {#2};}

%***********************************************
%**************Start of Document****************
%***********************************************

%===============================================
%===============Theorem Styles==================
%===============================================

%================Default Style==================
\theoremstyle{plain}% is the default. it sets the text in italic and adds extra space above and below the \newtheorems listed below it in the input. it is recommended for theorems, corollaries, lemmas, propositions, conjectures, criteria, and (possibly; depends on the subject area) algorithms.
\newtheorem{theorem}{Theorem}
\numberwithin{theorem}{section} %This sets the numbering system for theorems to number them down to the {argument} level. I have it set to number down to the {section} level right now.
\newtheorem*{theorem*}{Theorem} %Theorem with no numbering
\newtheorem{corollary}[theorem]{Corollary}
\newtheorem*{corollary*}{Corollary}
\newtheorem{conjecture}[theorem]{Conjecture}
\newtheorem{lemma}[theorem]{Lemma}
\newtheorem*{lemma*}{Lemma}
\newtheorem{proposition}[theorem]{Proposition}
\newtheorem*{proposition*}{Proposition}
\newtheorem{problemstatement}[theorem]{Problem Statement}


%==============Definition Style=================
\theoremstyle{definition}% adds extra space above and below, but sets the text in roman. it is recommended for definitions, conditions, problems, and examples; i've alse seen it used for exercises.
\newtheorem{definition}[theorem]{Definition}
\newtheorem*{definition*}{Definition}
\newtheorem{condition}[theorem]{Condition}
\newtheorem{problem}[theorem]{Problem}
\newtheorem{example}[theorem]{Example}
\newtheorem*{example*}{Example}
\newtheorem*{counterexample*}{Counterexample}
\newtheorem*{romantheorem*}{Theorem} %Theorem with no numbering
\newtheorem{exercise}{Exercise}
\numberwithin{exercise}{section}
\newtheorem{algorithm}[theorem]{Algorithm}

%================Remark Style===================
\theoremstyle{remark}% is set in roman, with no additional space above or below. it is recommended for remarks, notes, notation, claims, summaries, acknowledgments, cases, and conclusions.
\newtheorem{remark}[theorem]{Remark}
\newtheorem*{remark*}{Remark}
\newtheorem{notation}[theorem]{Notation}
\newtheorem*{notation*}{Notation}
%\newtheorem{claim}[theorem]{Claim}  %%use this if you ever want claims to be numbered
\newtheorem*{claim}{Claim}



\pgfplotsset{compat=1.13}

%\newcommand{\T}{\mathcal{T}}
%\newcommand{\B}{\mathcal{B}}

%These commands are now in tskpreamble_nothms.tex, but are left as a comment here for reference. 
%\newcommand{\arbcup}[1]{\bigcup\limits_{\alpha\in\Gamma}#1_\alpha}
%\newcommand{\arbcap}[1]{\bigcap\limits_{\alpha\in\Gamma}#1_\alpha}
%\newcommand{\arbcoll}[1]{\{#1_\alpha\}_{\alpha\in\Gamma}}
%\newcommand{\arbprod}[1]{\prod\limits_{\alpha\in\Gamma}#1_\alpha}
%\newcommand{\finitecoll}[1]{#1_1, \ldots, #1_n}
%\newcommand{\finitefuncts}[2]{#1(#2_1), \ldots, #1(#2_n)}
%\newcommand{\abs}[1]{\left|#1\right|}
%\newcommand{\norm}[1]{\left|\left|#1\right|\right|}

\title{Math 450b \linebreak
Homework 1}
\author{Trevor Klar}

\begin{document}

\maketitle

\begin{enumerate}
%1
\item Let $\vecb{x}, \vecb{y} \in \R^n$. Prove that $\abs{\inner{\vecb{x}}{\vecb{y}}}=\norm{\vecb{x}}\norm{\vecb{y}}$ if and only if $\vecb{y}=r\vecb{x}$ for some $r\in\R$. 
\begin{proof}
Both directions of this proof will rely on the fact that $\vecb{x}\neq\vec{0}$, so before we begin we will address that possibility. Suppose $\vecb{x} =\vec{0}$. Then, $\abs{\inner{\vecb{x}}{\vecb{y}}} = \abs{\inner{\vecb{0}}{\vecb{y}}} = \abs{\sum_{i=1}^n 0y_i} = 0$ and $\norm{x}\norm{y} = 0\norm{y} = 0$. Thus, $\abs{\inner{\vecb{x}}{\vecb{y}}}=0=\norm{\vecb{x}}\norm{\vecb{y}}$, so the converse direction holds (since the conclusion is always true). However, if $\vecb{x}=\vec{0}$ and $\vecb{y}\neq\vec{0}$, then there is no such $r\in\R$ such that $\vecb{y}=r\vecb{x}$, so the forward direction actually does not hold in this case (the hypothesis is always true, but the conclusion is always false). 

Since the theorem does not always hold when $\vecb{x}=\vec{0}$, we will assume that $\vecb{x}\neq\vec{0}$ in the rest of this proof. 
\end{proof}

\begin{proof}($\impliedby$) Suppose that $\vecb{y}=r\vecb{x}$ for some $r\in\R$. Then we have the following:
\[\begin{array}{rcl}
0&=&\norm{\vecb{y}-r\vecb{x}}^2\\
0&=&\inner{\vecb{y}-r\vecb{x}}{\vecb{y}-r\vecb{x}}\\
0&=&\norm{\vecb{y}}^2-2r\inner{\vecb{x}}{\vecb{y}}+r^2\norm{\vecb{x}}^2\\
\end{array}\]
Before we proceed further, we can use the fact that $\vecb{y}=r\vecb{x}$ to obtain a value for $r$:
\[\begin{array}{rcl}
\inner{\vecb{x}}{r\vecb{x}}&=&\inner{\vecb{x}}{r\vecb{x}}\\
r\inner{\vecb{x}}{\vecb{x}}&=&\inner{\vecb{x}}{\vecb{y}}\\
r\norm{\vecb{x}}^2&=&\inner{\vecb{x}}{\vecb{y}}\\
r&=&\frac{\inner{\vecb{x}}{\vecb{y}}}{\norm{\vecb{x}}^2}\\
\end{array}\]
Now we plug this in for $r$ in our previous equation and simplify:
\[\begin{array}{rcl}
0&=&\norm{\vecb{y}}^2
-2\left(\frac{\inner{\vecb{x}}{\vecb{y}}}{\norm{\vecb{x}}^2}\right)\inner{\vecb{x}}{\vecb{y}}
+\left(\frac{\inner{\vecb{x}}{\vecb{y}}}{\norm{\vecb{x}}^2}\right)^2\norm{\vecb{x}}^2\\

0&=&\norm{\vecb{y}}^2
-2\frac{\inner{\vecb{x}}{\vecb{y}}^2}{\norm{\vecb{x}}^2}
+\frac{\inner{\vecb{x}}{\vecb{y}}^2}{\norm{\vecb{x}}^2}\\

0&=&\norm{\vecb{y}}^2-\frac{\inner{\vecb{x}}{\vecb{y}}^2}{\norm{\vecb{x}}^2}\\
\end{array}\]

From this, we can rearrange to find that $\inner{\vecb{x}}{\vecb{y}}^2=\norm{\vecb{x}}^2\norm{\vecb{y}}^2$ and take square roots, yielding $\abs{\inner{\vecb{x}}{\vecb{y}}}=\norm{\vecb{x}}\norm{\vecb{y}}$ and we are done. 
\end{proof}

\begin{proof}($\implies$) Suppose that $\abs{\inner{\vecb{x}}{\vecb{y}}}=\norm{\vecb{x}}\norm{\vecb{y}}$. 

As in the converse direction (with steps reversed), we can square both sides and rearrange to find that 
$$0=\norm{\vecb{y}}^2
-2\left(\frac{\inner{\vecb{x}}{\vecb{y}}}{\norm{\vecb{x}}^2}\right)\inner{\vecb{x}}{\vecb{y}}
+\left(\frac{\inner{\vecb{x}}{\vecb{y}}}{\norm{\vecb{x}}^2}\right)^2\norm{\vecb{x}}^2.$$
Now since we have assumed that $\vecb{x}\neq\vec{0}$, we know that $\frac{\inner{\vecb{x}}{\vecb{y}}}{\norm{\vecb{x}}^2}$ is a real number. So let $r=\frac{\inner{\vecb{x}}{\vecb{y}}}{\norm{\vecb{x}}^2}$ and substitute to obtain 
$$0=\norm{\vecb{y}}^2-2r\inner{\vecb{x}}{\vecb{y}}+r^2\norm{\vecb{x}}^2.$$
Again as we did in the converse direction, we can rearrange to find that $0=\norm{\vecb{y}-r\vecb{x}}^2$. This means that $\vecb{y}=r\vecb{x}$, and we are done. 
\end{proof} 

%2
\item Let $\vecb{x}, \vecb{y} \in \R$ be nonzero. Prove that $\norm{\vecb{x}+\vecb{y}}^2=\norm{\vecb{x}}^2+\norm{\vecb{y}}^2$ if and only if $\vecb{x}$ and $\vecb{y}$ are orthogonal. 

\begin{proof}($\impliedby$) Suppose $\vecb{x}$ and $\vecb{y}$ are orthogonal. Then 
\[\begin{array}{rcl}
\norm{\vecb{x}+\vecb{y}}^2&=&\inner{\vecb{x}+\vecb{y}}{\vecb{x}+\vecb{y}}\\
&=&\norm{\vecb{x}}^2+2\inner{\vecb{x}}{\vecb{y}}+\norm{\vecb{y}}^2\\
\end{array}\]
and, since $\vecb{x}$ and $\vecb{y}$ are orthogonal, $\inner{\vecb{x}}{\vecb{y}}=0$, so 
\[\begin{array}{rcl}
\quad \quad \quad \quad &&\norm{\vecb{x}}^2+2\inner{\vecb{x}}{\vecb{y}}+\norm{\vecb{y}}^2 \\
&=& \norm{\vecb{x}}^2+\norm{\vecb{y}}^2\\
\end{array}\]
and we are done. 
\end{proof}

\begin{proof}($\implies$)
Suppose that $\norm{\vecb{x}+\vecb{y}}^2=\norm{\vecb{x}}^2+\norm{\vecb{y}}^2$. Then, 
\[\begin{array}{rcl}
\norm{\vecb{x}+\vecb{y}}^2&=&\norm{\vecb{x}}^2+\norm{\vecb{y}}^2\\
\norm{\vecb{x}}^2+2\inner{\vecb{x}}{\vecb{y}}+\norm{\vecb{y}}^2&=&\norm{\vecb{x}}^2+\norm{\vecb{y}}^2\\
2\inner{\vecb{x}}{\vecb{y}}&=&0\\
\end{array}\]
Thus, $\vecb{x}$ and $\vecb{y}$ are orthogonal by definition. 
\end{proof}

%3
\item Let $\vecb{x}=(1,1,\ldots, 1)$ and $\vecb{y}=(1,2,\ldots, n)$ in $\R^n$. Let $\theta_n$ be the angle between $\vecb{x}$ and $\vecb{y}$ in $\R^n$. Find $\lim\limits_{n\to\infty}\theta_n$. 

We know that 
$$\cos\theta_n=\frac{\inner{\vecb{x}}{\vecb{y}}}{\norm{\vecb{x}}\norm{\vecb{y}}}_,$$
So we will compute each of the parts. 
$$\inner{\vecb{x}}{\vecb{y}}=\sum_{i=1}^ni=\frac{n(n+1)}{2}$$
$$\norm{\vecb{x}}=\sqrt{n}$$
$$\norm{\vecb{y}}=\sqrt{\sum_{i=1}^n i^2}=\sqrt{\frac{n(n+1)(2n+1)}{6}}$$
Plugging these terms in and canceling, we find that 
$$\cos\theta_n=\sqrt{\frac{3n+3}{4n+2}}$$
So, to find $\lim\limits_{n\to\infty}\theta_n$, we find 
$$\lim\limits_{n\to\infty}\left(\cos^{-1}\sqrt{\frac{3n+3}{4n+2}}\right)=\inv{\cos}\left(\frac{\sqrt{3}}{2}\right)=\frac{\pi}{6}$$
and we are done. \qed

\pagebreak
%4
\item ($\square$) Decide if the following subsets of $R^n$ are open and/or closed. (Draw pictures, and give answers. No proofs necessary.)

	\begin{multicols}{2}	
	\begin{enumerate}[label=(\alph*)]
	%a
	\item $\{(x,y):xy=0\}\subset\R^2$\\
	\textbf{Answer:} Closed and not open. 
	\jpg{scale=.07}{450b_hw1_prob4a}
	\qedwhite
	
	%b	
	\item $\{(x,y):xy\neq0\}\subset\R^2$\\
	\textbf{Answer:} Open and not closed. 
	\jpg{scale=.07}{450b_hw1_prob4b}
	\qedwhite
	
	%c
	\item $\{(x,y,z):x^2+y^2<1 \text{ and } z=0\}\subset\R^3$\\
	\textbf{Answer:} Not open and not closed. 
	\jpg{scale=.07}{450b_hw1_prob4c}
	\qedwhite
	
	%d
	\item $\{(x,y,z):x^2+y^2<1\}\subset\R^3$\\
	\textbf{Answer:} Open and not closed. 
	\jpg{scale=.07}{450b_hw1_prob4d}
	\qedwhite
	\end{enumerate}
	\end{multicols}
	
	\begin{enumerate}
	\setcounter{enumi}{4}	
	%e
	\item $\{(x_1, \ldots, x_n): \text{each } x_i\in\Q\}\subset \R^n$\\
	\textbf{Answer:} Not open and not closed. \\
	This set is impossible to draw. I imagine it something like a dense infinite point grid, like a field of stars in space. Each element has infinitely many other elements surrounding it in every direction, as well as elements not in the set surrounding it in a similar way. \qed
	\end{enumerate}


%5
\item ($\square$) Let $S$ be an $(n-1)$-dimensional vector subspace of $\R^n$. Prove that $S$ is not an open set. 
\begin{proof}
Since every vector space has a basis, let $B=\{\mathbf{v_1}, \mathbf{v_2}, \ldots, \mathbf{v_{n-1}}\}$ be a basis for $S$. Now, since $B$ has only $(n-1)$ elements, it cannot span $\R^n$, and thus can be extended to a spanning set by including another vector, 
$\vecb{u}$. Now, to see that $S$ is not open, observe that for every $\vecb{x}\in S$, and every $B(\vecb{x},r)$ where $r\in\R^+$, the point $\vecb{x}+\frac{r\vecb{u}}{2\norm{\vecb{u}}}$ is an element of $B(\vecb{x},r)$, but not an element of $S$. The following image illustrates this for $\R^3$ and $S=\{(x,y,z):z=2\}$:
\jpg{scale=.05}{450b_hw1_prob5}
\end{proof}

%6
\item ($\square$) Let $\vecb{x}\in\R^n, r\geq0$, and define $\closure{B}(\vecb{x},r)=\{\vecb{y}\in\R:\norm{\vecb{x}-\vecb{y}}\leq r\}.$ Prove that $\closure{B}(\vecb{x},r)$ is closed. 
\begin{proof}
To show that $\closure{B}(\vecb{x},r)$ is closed, we will show that its complement is open. Let $\vecb{p}$ be in $\R^n$ such that $\vecb{p}\not\in\closure{B}(\vecb{x},r)$. Let $\tilde{r}=\frac{\norm{\vecb{p}-\vecb{x}}-r}{2}$. \\
\textbf{Claim:} $B(\vecb{p},\tilde{r}) \subset (\R^n-\closure{B}(\vecb{x},r))$.
\jpg{scale=.12}{450b_hw1_prob6}
To show this, we will prove that $\norm{\vecb{y}-\vecb{x}}>r$.\\
Let $\vecb{y}\in B(\vecb{p},\tilde{r})$. Then by the triangle inequality, 
$$\norm{\vecb{p}-\vecb{x}}\leq\norm{\vecb{p}-\vecb{y}}+\norm{\vecb{y}-\vecb{x}}$$
and subtracting $\norm{\vecb{p}-\vecb{y}}$, we find that 
$$\norm{\vecb{p}-\vecb{x}}-\norm{\vecb{p}-\vecb{y}}\leq\norm{\vecb{y}-\vecb{x}}.$$
Now, $\norm{\vecb{p}-\vecb{x}}=r+2\tilde{r}$ by definition, and $-\tilde{r}<-\norm{\vecb{p}-\vecb{y}}$ as well, so 
$$r+\tilde{r}=(r+2\tilde{r})-\tilde{r}<\norm{\vecb{p}-\vecb{x}}-\norm{\vecb{p}-\vecb{y}}\leq\norm{\vecb{y}-\vecb{x}},$$
Thus $r<\norm{\vecb{y}-\vecb{x}}$ and we are done. 
\end{proof}

%7
\item \mbox{}
	\begin{enumerate}[label=(\alph*)]
	%a
	\item Prove that $\R^n$ is an open set. 
	\begin{proof}
	Let $\vecb{x}\in\R^n$, and let $r>0$. Observe that $B(\vecb{x},r)\subset\R^n$, so $R^n$ is open. 
	\end{proof}
	
	%b
	\item Let $\arbcoll{U}$ be a collection of an arbitrary number of open sets in $\R^n$. Prove that $\arbcup{U}$ is an open set. 
	\begin{proof}
	Let $\vecb{x}\in\arbcup{U}$. By definition, $\vecb{x}\in U_\beta$ for some $\beta\in\Gamma$. Since $U_\beta$ is open, there exists some $r>0$ such that $B(\vecb{x},r)\subset U_\beta$. Thus, $B(\vecb{x},r)\subset \arbcup{U}$, so it is open.
	\end{proof}
	
	%c
	\item Let $U_1$ and $U_2$ be open sets in $\R^n$. Prove that $U_1\cap U_2$ is an open set. 
	\begin{proof}
	Let $\vecb{x}\in U_1\cap U_2$. Since $U_1$ and $U_2$ are open sets, there exist $r_1, r_2 >0$ such that $B(\vecb{x},r_1)\subset U_1$ and $B(\vecb{x},r_2)\subset U_2$. Let $r=\min(r_1, r_2)$. Then, $B(\vecb{x},r)\subset B(\vecb{x},r_1)\subset U_1$ and $B(\vecb{x},r)\subset B(\vecb{x},r_2)\subset U_2$; so 
	$$B(\vecb{x},r)\subset U_1\cap U_2$$
	and we are done. 
	\end{proof}
	\end{enumerate}

\pagebreak
%8
\item Let $\arbcoll{C}$ be an arbitrary collection of closed sets in $\R^n$. 
	\begin{enumerate}
	%a
	\item Prove that $\arbcap{C}$ is a closed set.
	\begin{proof} To prove that $\arbcap{C}$ is closed, we will prove that its complement is open; that is, $\arbcup{C^\complement}$ is open. Since each $C$ is closed, then each $C^\complement$ is open. Then, by problem 7(b), $\arbcup{C^\complement}$ is also open, and we are done. 
	
	\end{proof}
	
	%b
	\item Professor Doofus writes that in addition $\arbcup{C}$ is a closed set. Give an example which shows that Doofus is wrong. \\
	\textbf{Answer:} Let $\{C_n\}_{n=1}^\infty$ be the collection of all $C_n=\closure{B}(\vecb{0},1-\sfrac{1}{n})$. So since $sup\left\lbrace\left(1-\frac{1}{n}\right):n\in\N\right\rbrace=1$, then $\bigcup\limits_{n=1}^\infty C_n=B(\vecb{0},1)$. And we already know that open balls are not closed. 
	\end{enumerate}
\end{enumerate}

\end{document}
