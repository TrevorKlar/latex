\documentclass[letterpaper]{article}
%\documentclass[a5paper]{article}

%% Language and font encodings
\usepackage[english]{babel}
\usepackage[utf8x]{inputenc}
\usepackage[T1]{fontenc}

%% Sets page size and margins
\usepackage[letterpaper,top=.75in,bottom=1in,left=1in,right=1in,marginparwidth=1.75cm]{geometry}
%\usepackage[a5paper,top=1cm,bottom=1cm,left=1cm,right=1.5cm,marginparwidth=1.75cm]{geometry}

\usepackage{graphicx}
%\graphicspath{../images}	  %%where to look for images

%% Useful packages
\usepackage{amssymb, amsmath, amsthm} 
%\usepackage{graphicx}  %%this is currently enabled in the default document, so it is commented out here. 
\usepackage{calrsfs}
\usepackage{braket}
\usepackage{mathtools}
\usepackage{lipsum}
\usepackage{tikz}
\usetikzlibrary{cd}
\usepackage{verbatim}
%\usepackage{ntheorem}% for theorem-like environments
\usepackage{mdframed}%can make highlighted boxes of text
%Use case: https://tex.stackexchange.com/questions/46828/how-to-highlight-important-parts-with-a-gray-background
\usepackage{wrapfig}
\usepackage{centernot}
\usepackage{subcaption}%\begin{subfigure}{0.5\textwidth}
\usepackage{pgfplots}
\pgfplotsset{compat=1.13}
\usepackage[colorinlistoftodos]{todonotes}
\usepackage[colorlinks=true, allcolors=blue]{hyperref}
\usepackage{xfrac}					%to make slanted fractions \sfrac{numerator}{denominator}
\usepackage{enumitem}            
    %syntax: \begin{enumerate}[label=(\alph*)]
    %possible arguments: f \alph*, \Alph*, \arabic*, \roman* and \Roman*
\usetikzlibrary{arrows,shapes.geometric,fit}

\DeclareMathAlphabet{\pazocal}{OMS}{zplm}{m}{n}
%% Use \pazocal{letter} to typeset a letter in the other kind 
%%  of math calligraphic font. 

%% This puts the QED block at the end of each proof, the way I like it. 
\renewenvironment{proof}{{\bfseries Proof}}{\qed}
\makeatletter
\renewenvironment{proof}[1][\bfseries \proofname]{\par
  \pushQED{\qed}%
  \normalfont \topsep6\p@\@plus6\p@\relax
  \trivlist
  %\itemindent\normalparindent
  \item[\hskip\labelsep
        \scshape
    #1\@addpunct{}]\ignorespaces
}{%
  \popQED\endtrivlist\@endpefalse
}
\makeatother

%% This adds a \rewnewtheorem command, which enables me to override the settings for theorems contained in this document.
\makeatletter
\def\renewtheorem#1{%
  \expandafter\let\csname#1\endcsname\relax
  \expandafter\let\csname c@#1\endcsname\relax
  \gdef\renewtheorem@envname{#1}
  \renewtheorem@secpar
}
\def\renewtheorem@secpar{\@ifnextchar[{\renewtheorem@numberedlike}{\renewtheorem@nonumberedlike}}
\def\renewtheorem@numberedlike[#1]#2{\newtheorem{\renewtheorem@envname}[#1]{#2}}
\def\renewtheorem@nonumberedlike#1{  
\def\renewtheorem@caption{#1}
\edef\renewtheorem@nowithin{\noexpand\newtheorem{\renewtheorem@envname}{\renewtheorem@caption}}
\renewtheorem@thirdpar
}
\def\renewtheorem@thirdpar{\@ifnextchar[{\renewtheorem@within}{\renewtheorem@nowithin}}
\def\renewtheorem@within[#1]{\renewtheorem@nowithin[#1]}
\makeatother

%% This makes theorems and definitions with names show up in bold, the way I like it. 
\makeatletter
\def\th@plain{%
  \thm@notefont{}% same as heading font
  \itshape % body font
}
\def\th@definition{%
  \thm@notefont{}% same as heading font
  \normalfont % body font
}
\makeatother

%===============================================
%==============Shortcut Commands================
%===============================================
\newcommand{\ds}{\displaystyle}
\newcommand{\B}{\mathcal{B}}
\newcommand{\C}{\mathbb{C}}
\newcommand{\F}{\mathbb{F}}
\newcommand{\N}{\mathbb{N}}
\newcommand{\R}{\mathbb{R}}
\newcommand{\Q}{\mathbb{Q}}
\newcommand{\T}{\mathcal{T}}
\newcommand{\Z}{\mathbb{Z}}
\renewcommand\qedsymbol{$\blacksquare$}
\newcommand{\qedwhite}{\hfill\ensuremath{\square}}
\newcommand*\conj[1]{\overline{#1}}
\newcommand*\closure[1]{\overline{#1}}
\newcommand*\mean[1]{\overline{#1}}
%\newcommand{\inner}[1]{\left< #1 \right>}
\newcommand{\inner}[2]{\left< #1, #2 \right>}
\newcommand{\powerset}[1]{\pazocal{P}(#1)}
%% Use \pazocal{letter} to typeset a letter in the other kind 
%%  of math calligraphic font. 
\newcommand{\cardinality}[1]{\left| #1 \right|}
\newcommand{\domain}[1]{\mathcal{D}(#1)}
\newcommand{\image}{\text{Im}}
\newcommand{\inv}[1]{#1^{-1}}
\newcommand{\preimage}[2]{#1^{-1}\left(#2\right)}
\newcommand{\script}[1]{\mathcal{#1}}


\newenvironment{highlight}{\begin{mdframed}[backgroundcolor=gray!20]}{\end{mdframed}}

\DeclarePairedDelimiter\ceil{\lceil}{\rceil}
\DeclarePairedDelimiter\floor{\lfloor}{\rfloor}

%===============================================
%===============My Tikz Commands================
%===============================================
\newcommand{\drawsquiggle}[1]{\draw[shift={(#1,0)}] (.005,.05) -- (-.005,.02) -- (.005,-.02) -- (-.005,-.05);}
\newcommand{\drawpoint}[2]{\draw[*-*] (#1,0.01) node[below, shift={(0,-.2)}] {#2};}
\newcommand{\drawopoint}[2]{\draw[o-o] (#1,0.01) node[below, shift={(0,-.2)}] {#2};}
\newcommand{\drawlpoint}[2]{\draw (#1,0.02) -- (#1,-0.02) node[below] {#2};}
\newcommand{\drawlbrack}[2]{\draw (#1+.01,0.02) --(#1,0.02) -- (#1,-0.02) -- (#1+.01,-0.02) node[below, shift={(-.01,0)}] {#2};}
\newcommand{\drawrbrack}[2]{\draw (#1-.01,0.02) --(#1,0.02) -- (#1,-0.02) -- (#1-.01,-0.02) node[below, shift={(+.01,0)}] {#2};}

%***********************************************
%**************Start of Document****************
%***********************************************

%===============================================
%===============Theorem Styles==================
%===============================================

%================Default Style==================
\theoremstyle{plain}% is the default. it sets the text in italic and adds extra space above and below the \newtheorems listed below it in the input. it is recommended for theorems, corollaries, lemmas, propositions, conjectures, criteria, and (possibly; depends on the subject area) algorithms.
\newtheorem{theorem}{Theorem}
\numberwithin{theorem}{section} %This sets the numbering system for theorems to number them down to the {argument} level. I have it set to number down to the {section} level right now.
\newtheorem*{theorem*}{Theorem} %Theorem with no numbering
\newtheorem{corollary}[theorem]{Corollary}
\newtheorem*{corollary*}{Corollary}
\newtheorem{conjecture}[theorem]{Conjecture}
\newtheorem{lemma}[theorem]{Lemma}
\newtheorem*{lemma*}{Lemma}
\newtheorem{proposition}[theorem]{Proposition}
\newtheorem*{proposition*}{Proposition}
\newtheorem{problemstatement}[theorem]{Problem Statement}


%==============Definition Style=================
\theoremstyle{definition}% adds extra space above and below, but sets the text in roman. it is recommended for definitions, conditions, problems, and examples; i've alse seen it used for exercises.
\newtheorem{definition}[theorem]{Definition}
\newtheorem*{definition*}{Definition}
\newtheorem{condition}[theorem]{Condition}
\newtheorem{problem}[theorem]{Problem}
\newtheorem{example}[theorem]{Example}
\newtheorem*{example*}{Example}
\newtheorem*{counterexample*}{Counterexample}
\newtheorem*{romantheorem*}{Theorem} %Theorem with no numbering
\newtheorem{exercise}{Exercise}
\numberwithin{exercise}{section}
\newtheorem{algorithm}[theorem]{Algorithm}

%================Remark Style===================
\theoremstyle{remark}% is set in roman, with no additional space above or below. it is recommended for remarks, notes, notation, claims, summaries, acknowledgments, cases, and conclusions.
\newtheorem{remark}[theorem]{Remark}
\newtheorem*{remark*}{Remark}
\newtheorem{notation}[theorem]{Notation}
\newtheorem*{notation*}{Notation}
%\newtheorem{claim}[theorem]{Claim}  %%use this if you ever want claims to be numbered
\newtheorem*{claim}{Claim}



\pgfplotsset{compat=1.13}

%\newcommand{\T}{\mathcal{T}}
%\newcommand{\B}{\mathcal{B}}
\newcommand{\arbcup}[1]{\bigcup\limits_{\alpha\in\Gamma}#1_\alpha}
\newcommand{\arbcap}[1]{\bigcap\limits_{\alpha\in\Gamma}#1_\alpha}
\newcommand{\Rbad}{\R_\text{bad}}

\title{Math 501 \linebreak
Homework 7}
\author{Trevor Klar}

\begin{document}

\maketitle

\begin{enumerate}
\item Recall for a set $A$ in a space $X$, we define $\partial A = \closure{A}-\text{int}(A)$. 
	\begin{enumerate}
	\item Prove that $x \in \partial A$ if and only if for every open set $U$ containing $x$, we have $U\cap A \neq \emptyset$ and $U \cap (X-A)\neq \emptyset$. 
	\begin{proof}\mbox{}\\
	Suppose $x \in \partial A$. By definition, $x\in\closure{A}$ and $x\not\in\text{int}(A)$. Since $x\in\closure{A}$, then by definition either $x\in A$, or $x$ is a limit point of $A$. If $x\in A$, then clearly for any open set $U$ containing $x$, $U\cap A\neq \emptyset$. Otherwise if $x$ is merely a limit point of $A$, then for any open $U$ containing $x$, 
	$$U\cap A-\{x\}\neq \emptyset \implies U\cap A\neq \emptyset.$$
	Since $x\not\in\text{int}(A)$, then by definition there is no open set $U$ containing $x$ such that $U\subset A$. That is, for every open set $U$ containing $x$, 
	$$U\cap (X-A)\neq \emptyset.$$ 
	Since this conclusion follows from definintions, and definitions are biconditional, then the converse is also true.
	\end{proof}
	
	\item Prove that $A$ is open if and only if $\partial A \cap A = \emptyset$. Prove that $A$ is closed if and only if $\partial A \subseteq A$. 
	\begin{proof}
	Suppose $A$ is open. Then, $A=\text{int}(A)$, so 
	$$\partial A \cap A = (\closure{A}-A)\cap A = \emptyset.$$
	Suppose $A$ is closed. Then, $A=\closure{A}$, so 
	$$\partial A \cap A = (A-\text{int}(A))\cap A = (A-\text{int}(A))\subseteq A.$$
	Suppose $\partial A \cap A = \emptyset$. Since $\text{int}(A)\subset A\subset \closure{A}$, 
	$$\partial A \cap A = (\closure{A}-\text{int}(A))\cap A=\emptyset \implies (A-\text{int}(A))=\emptyset \implies A\subset\text{int}(A) \implies A=\text{int}(A).$$
	Suppose $\partial A \subseteq A$. Since $\text{int}(A)\subset A\subset \closure{A}$, 
	$$\partial A = (\closure{A}-\text{int}(A))\subset A \implies \closure{A}\subset A \implies \closure{A}= A.$$	
	\end{proof}
	\end{enumerate}

\item Let $X$ be a metric space with metric $d$, and suppose that for all $x\in X$ and $r<0$, the closed ball $\closure{B}(x,r) = \{y:d(x,y)\leq r\}$ is compact. Prove that $X$ is $d$-complete. 
\begin{proof}
First note that since we have assumed that all closed balls in $X$ are compact, then they are also all complete. Let $(x_n)_{n=1}^\infty$ be any Cauchy sequence in $X$. We will show that $(x_n)_{n=1}^\infty$ converges. Since$(x_n)_{n=1}^\infty$ is Cauchy, then there exists some $N\in\N$ such that for all $m,n>N$, $d(x_m,x_n)<1$. Let $n_0=N+1$. So, for all $n>N$, $d(x_{n_0},x_n)<1$. Now, let $M=max\{d(x_{n_0},x_n) : n \leq N\}$. So, for any $x_n$ in our sequence, $d(x_{n_0},x_n)\leq \max(M,1)$, and thus $x_n\in \closure{B}(x_{n_0},\max(M,1)), \forall n\in\N$. Since all closed balls in $X$ are complete, and $(x_n)_{n=1}^\infty\in\closure{B}(x_{n_0},\max(M,1))\,  \forall n\in\N$, then $(x_n)_{n=x1}^\infty$ converges. 
\end{proof}

\item Prove Corollary 28: Let $X$ be a complete metric space. Show that $X$ is not the countable union of nowhere dense closed sets.
\begin{proof}
Let $\bigcup_{i=1}^\infty F_i$ denote any countable union of nowhere dense closed sets. Consider the complement of this union:
$$\left(\bigcup_{i=1}^\infty F_i\right)^\complement = \bigcap_{i=1}^\infty \left(F_i^\complement\right) = \bigcap_{i=1}^\infty \left(X - F_i\right)$$
Now, each $(X-F_i)$ must be a dense open set, so by the Baire Category Theorem, this countable intersection of dense open sets is dense in $X$. Thus, $\bigcap_{i=1}^\infty \left(X - F_i\right)\neq\emptyset$, so $\bigcup_{i=1}^\infty F_i\neq X$.
\end{proof}

\item[3.5] Let $X$ be a complete metric space with metric $d$, and suppose that $X$ has no isolated points. Prove that $X$ is uncountable.
\begin{proof}
Suppose that $X$ is countable. Consider the colection $\{X-\{x\}\}_{x\in X}$. Since $X$ has no isolated points, each of these sets is dense. And since $X$ is Hausdorff, each of these sets is open (because singletons are closed in a Hausdorff space). Now, 
$$\bigcap_{x\in X}\{X-\{x\}\}=\emptyset,$$
but this countable union of dense open sets should be dense in $X$ by the Baire Category Theorem, so we have a contradiction. 
\end{proof}
\end{enumerate}

\end{document}
