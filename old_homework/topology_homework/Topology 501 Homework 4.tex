\documentclass[letterpaper]{article}
%\documentclass[a5paper]{article}

%% Language and font encodings
\usepackage[english]{babel}
\usepackage[utf8x]{inputenc}
\usepackage[T1]{fontenc}

%% Sets page size and margins
\usepackage[letterpaper,top=1in,bottom=1in,left=1in,right=1in,marginparwidth=1.75cm]{geometry}
%\usepackage[a5paper,top=1cm,bottom=1cm,left=1cm,right=1.5cm,marginparwidth=1.75cm]{geometry}

\usepackage{graphicx}
\graphicspath{ {./} }	  %%where to look for images

%% Useful packages
\usepackage{amssymb, amsmath, amsthm} 
%\usepackage{graphicx}  %%this is currently enabled in the default document, so it is commented out here. 
\usepackage{calrsfs}
\usepackage{braket}
\usepackage{mathtools}
\usepackage{lipsum}
\usepackage{tikz}
\usetikzlibrary{cd}
\usepackage{verbatim}
%\usepackage{ntheorem}% for theorem-like environments
\usepackage{mdframed}%can make highlighted boxes of text
%Use case: https://tex.stackexchange.com/questions/46828/how-to-highlight-important-parts-with-a-gray-background
\usepackage{wrapfig}
\usepackage{centernot}
\usepackage{subcaption}%\begin{subfigure}{0.5\textwidth}
\usepackage{pgfplots}
\pgfplotsset{compat=1.13}
\usepackage[colorinlistoftodos]{todonotes}
\usepackage[colorlinks=true, allcolors=blue]{hyperref}
\usepackage{xfrac}					%to make slanted fractions \sfrac{numerator}{denominator}
\usepackage{enumitem}            
    %syntax: \begin{enumerate}[label=(\alph*)]
    %possible arguments: f \alph*, \Alph*, \arabic*, \roman* and \Roman*
\usetikzlibrary{arrows,shapes.geometric,fit}

\DeclareMathAlphabet{\pazocal}{OMS}{zplm}{m}{n}
%% Use \pazocal{letter} to typeset a letter in the other kind 
%%  of math calligraphic font. 

%% This puts the QED block at the end of each proof, the way I like it. 
\renewenvironment{proof}{{\bfseries Proof}}{\qed}
\makeatletter
\renewenvironment{proof}[1][\bfseries \proofname]{\par
  \pushQED{\qed}%
  \normalfont \topsep6\p@\@plus6\p@\relax
  \trivlist
  %\itemindent\normalparindent
  \item[\hskip\labelsep
        \scshape
    #1\@addpunct{}]\ignorespaces
}{%
  \popQED\endtrivlist\@endpefalse
}
\makeatother

%% This adds a \rewnewtheorem command, which enables me to override the settings for theorems contained in this document.
\makeatletter
\def\renewtheorem#1{%
  \expandafter\let\csname#1\endcsname\relax
  \expandafter\let\csname c@#1\endcsname\relax
  \gdef\renewtheorem@envname{#1}
  \renewtheorem@secpar
}
\def\renewtheorem@secpar{\@ifnextchar[{\renewtheorem@numberedlike}{\renewtheorem@nonumberedlike}}
\def\renewtheorem@numberedlike[#1]#2{\newtheorem{\renewtheorem@envname}[#1]{#2}}
\def\renewtheorem@nonumberedlike#1{  
\def\renewtheorem@caption{#1}
\edef\renewtheorem@nowithin{\noexpand\newtheorem{\renewtheorem@envname}{\renewtheorem@caption}}
\renewtheorem@thirdpar
}
\def\renewtheorem@thirdpar{\@ifnextchar[{\renewtheorem@within}{\renewtheorem@nowithin}}
\def\renewtheorem@within[#1]{\renewtheorem@nowithin[#1]}
\makeatother

%% This makes theorems and definitions with names show up in bold, the way I like it. 
\makeatletter
\def\th@plain{%
  \thm@notefont{}% same as heading font
  \itshape % body font
}
\def\th@definition{%
  \thm@notefont{}% same as heading font
  \normalfont % body font
}
\makeatother

%===============================================
%==============Shortcut Commands================
%===============================================
\newcommand{\ds}{\displaystyle}
\newcommand{\B}{\mathcal{B}}
\newcommand{\C}{\mathbb{C}}
\newcommand{\F}{\mathbb{F}}
\newcommand{\N}{\mathbb{N}}
\newcommand{\R}{\mathbb{R}}
\newcommand{\Q}{\mathbb{Q}}
\newcommand{\T}{\mathcal{T}}
\newcommand{\Z}{\mathbb{Z}}
\renewcommand\qedsymbol{$\blacksquare$}
\newcommand{\qedwhite}{\hfill\ensuremath{\square}}
\newcommand*\conj[1]{\overline{#1}}
\newcommand*\closure[1]{\overline{#1}}
\newcommand*\mean[1]{\overline{#1}}
%\newcommand{\inner}[1]{\left< #1 \right>}
\newcommand{\inner}[2]{\left< #1, #2 \right>}
\newcommand{\powerset}[1]{\pazocal{P}(#1)}
%% Use \pazocal{letter} to typeset a letter in the other kind 
%%  of math calligraphic font. 
\newcommand{\cardinality}[1]{\left| #1 \right|}
\newcommand{\domain}[1]{\mathcal{D}(#1)}
\newcommand{\image}{\text{Im}}
\newcommand{\inv}[1]{#1^{-1}}
\newcommand{\preimage}[2]{#1^{-1}\left(#2\right)}
\newcommand{\script}[1]{\mathcal{#1}}


\newenvironment{highlight}{\begin{mdframed}[backgroundcolor=gray!20]}{\end{mdframed}}

\DeclarePairedDelimiter\ceil{\lceil}{\rceil}
\DeclarePairedDelimiter\floor{\lfloor}{\rfloor}

%===============================================
%===============My Tikz Commands================
%===============================================
\newcommand{\drawsquiggle}[1]{\draw[shift={(#1,0)}] (.005,.05) -- (-.005,.02) -- (.005,-.02) -- (-.005,-.05);}
\newcommand{\drawpoint}[2]{\draw[*-*] (#1,0.01) node[below, shift={(0,-.2)}] {#2};}
\newcommand{\drawopoint}[2]{\draw[o-o] (#1,0.01) node[below, shift={(0,-.2)}] {#2};}
\newcommand{\drawlpoint}[2]{\draw (#1,0.02) -- (#1,-0.02) node[below] {#2};}
\newcommand{\drawlbrack}[2]{\draw (#1+.01,0.02) --(#1,0.02) -- (#1,-0.02) -- (#1+.01,-0.02) node[below, shift={(-.01,0)}] {#2};}
\newcommand{\drawrbrack}[2]{\draw (#1-.01,0.02) --(#1,0.02) -- (#1,-0.02) -- (#1-.01,-0.02) node[below, shift={(+.01,0)}] {#2};}

%***********************************************
%**************Start of Document****************
%***********************************************

%===============================================
%===============Theorem Styles==================
%===============================================

%================Default Style==================
\theoremstyle{plain}% is the default. it sets the text in italic and adds extra space above and below the \newtheorems listed below it in the input. it is recommended for theorems, corollaries, lemmas, propositions, conjectures, criteria, and (possibly; depends on the subject area) algorithms.
\newtheorem{theorem}{Theorem}
\numberwithin{theorem}{section} %This sets the numbering system for theorems to number them down to the {argument} level. I have it set to number down to the {section} level right now.
\newtheorem*{theorem*}{Theorem} %Theorem with no numbering
\newtheorem{corollary}[theorem]{Corollary}
\newtheorem*{corollary*}{Corollary}
\newtheorem{conjecture}[theorem]{Conjecture}
\newtheorem{lemma}[theorem]{Lemma}
\newtheorem*{lemma*}{Lemma}
\newtheorem{proposition}[theorem]{Proposition}
\newtheorem*{proposition*}{Proposition}
\newtheorem{problemstatement}[theorem]{Problem Statement}


%==============Definition Style=================
\theoremstyle{definition}% adds extra space above and below, but sets the text in roman. it is recommended for definitions, conditions, problems, and examples; i've alse seen it used for exercises.
\newtheorem{definition}[theorem]{Definition}
\newtheorem*{definition*}{Definition}
\newtheorem{condition}[theorem]{Condition}
\newtheorem{problem}[theorem]{Problem}
\newtheorem{example}[theorem]{Example}
\newtheorem*{example*}{Example}
\newtheorem*{counterexample*}{Counterexample}
\newtheorem*{romantheorem*}{Theorem} %Theorem with no numbering
\newtheorem{exercise}{Exercise}
\numberwithin{exercise}{section}
\newtheorem{algorithm}[theorem]{Algorithm}

%================Remark Style===================
\theoremstyle{remark}% is set in roman, with no additional space above or below. it is recommended for remarks, notes, notation, claims, summaries, acknowledgments, cases, and conclusions.
\newtheorem{remark}[theorem]{Remark}
\newtheorem*{remark*}{Remark}
\newtheorem{notation}[theorem]{Notation}
\newtheorem*{notation*}{Notation}
%\newtheorem{claim}[theorem]{Claim}  %%use this if you ever want claims to be numbered
\newtheorem*{claim}{Claim}



\pgfplotsset{compat=1.13}

\newcommand{\T}{\mathcal{T}}
\newcommand{\B}{\mathcal{B}}
\newcommand{\arbcup}[1]{\bigcup\limits_{\alpha\in\Gamma}#1_\alpha}
\newcommand{\arbcap}[1]{\bigcap\limits_{\alpha\in\Gamma}#1_\alpha}
\newcommand{\Rbad}{\R_\text{bad}}

\title{Math 501 \linebreak
Homework 4}
\author{Trevor Klar}

\begin{document}

\maketitle


\begin{enumerate}
\item Let $f:X\to Y$ be a function. 
	\begin{enumerate}
	\item Assume $X=\arbcup{U}$, with each $U_\alpha$ open, and each $f|_{U_\alpha}:U_\alpha\to Y$ continuous. Prove that $f$ is continuous. 
	\begin{proof}
	Let $B\in Y$ be an arbitrary open subset of $Y$. Since $B$ is open, and each $f|_{U_\alpha}:U_\alpha\to Y$ is continuous, then each $\preimage{f|_{U_\alpha}}{B}$ is open in $U_\alpha$ and, since $U_\alpha$ is open in $X$, then $\preimage{f|_{U_\alpha}}{B}$ is open in $X$. Now, 
	\[
	\begin{array}{rcl}
	\bigcup\limits_{\alpha\in\Gamma}\preimage{f|_{U_\alpha}}{B}&=&\bigcup\limits_{\alpha\in\Gamma}(\preimage{f}{B}\cap U_\alpha)\\
	&=&(\preimage{f}{B}\cap \bigcup\limits_{\alpha\in\Gamma} U_\alpha)\\
	&=&\preimage{f}{B}\cap X\\
	&=&\preimage{f}{B}\\
	
	\end{array}
	\]
	So, $\preimage{f}{B}$ is a union of open sets, which means it is open. Thus, $f$ is continuous.  
	\end{proof}
	\item Assume $X=\arbcup{A}$, with each $A_\alpha$ closed, and each $f|_{A_\alpha}:A_\alpha\to Y$ continuous. Is $f$ continuous? Prove or give a counterexample. 
	\begin{counterexample*}
	Let $f:(\R,usual)\to(\R,usual)$ be 
	$$f(x)=
	\begin{cases}
	0, & x=0\\
	\sin\left(\frac{1}{x}\right), & x\neq0\\
	\end{cases}
	$$
	and consider the collection of closed sets $\R=\bigcup\limits_{n\in\Z}A_n$ with $n\in Z$, and $A_n$ defined as follows:
	$$
	A_n=
	\begin{cases}
	\left[a,\tfrac{1}{a}\right], & n<0\\
	\{0\}, & n=0\\
	\left[\tfrac{1}{a},a\right], & n>0\\
	\end{cases}
	$$
	Now, it is a common result from calculus that $\sin(\frac{1}{x})$ is continuous at every point except $x=0$, so for all $n\neq0$, $f|_{A_n}$ is continuous (since none of these sets contain 0). Now we will show that $f|_{A_0}$ is also continuous. For any closed set $F\in \R$, $\preimage{f|_{A_0}}{F}=\{0\}$ if $0\in F$, and $\preimage{f|_{A_0}}{F}=\emptyset$ if $0\not\in F$. Since $\{0\}$ and $\emptyset$ are both closed, then $\preimage{f|_{A_0}}{F}$ is closed, so $f|_{A_0}$ is continuous.
	
	The reader will recall that $f$ can easily be shown not to be continuous by the $\delta-\epsilon$ definition, but we will make the same case using the results we have learned in topology. Consider the following preimage of a closed set: 
	$$\preimage{f}{\{-1,1\}}=\left\lbrace \frac{2}{\pi},\frac{2}{3\pi},\frac{2}{5\pi},\ldots\right\rbrace \cup \left\lbrace -\frac{2}{\pi},-\frac{2}{3\pi},-\frac{2}{5\pi},\ldots\right\rbrace$$
	Since $\preimage{f}{\{-1,1\}}$ has a 0 as limit point, but does not contain $0$, then $\preimage{f}{\{-1,1\}}$ is not closed. Therefore, $f$ is not continuous.\qed
	\end{counterexample*}
	\end{enumerate}
\item 
	\begin{enumerate}
	\item Prove that the set of intervals of the form $[a,b)$ with $a,b\in\R$ are the basis for a topology on $\R$. We will refer to $\R$ with this topology as $\R^1_\text{bad}$. Show that $\R^1_\text{bad}$ is not the usual topology on $\R$. 
	\begin{proof}\mbox{}
		\begin{itemize}
		\item Since $[1,0)=\{x\in\R:1\leq x < 0\}=\emptyset$, then $\emptyset\in\Rbad^1$. 
		\item For any $x\in\R$, $x\in [x-1,x+1)$, so $\Rbad^1$ covers $\R$. 
		\item For any $a,b,c,d\in\R$, 
		%R1bad
		\[
		\begin{array}{rcl}
		[a,b)\cap[c,d)&=&\{x\in\R:a\leq x < b\}\cap \{x\in\R:c\leq x < d\}\\
		&=&\{x\in\R: \max(a,c)\leq x < \min(b,d)\}\\
		&=&[\max(a,c),\min(b,d))\\
		&\in& \Rbad^1\\
		\end{array}
		\]
		So, as desired according to Theorem 13, for any $[a,b), [c,d) \in \Rbad^1$ which both contain $x$, there exists $[a,b)\cap[c,d) \in \Rbad^1$ such that $x\in [a,b)\cap[c,d)$. 
		\end{itemize}
	Thus, $\Rbad^1$ forms the basis for a topology on $\R$.
	\end{proof}
	\begin{proof}
	Now we will show that $\R^1_\text{bad}$ is not the usual topology on $\R$. Consider the set $[a,b)$, for some $a,b \in \R$ and $a<b$. By definition, $[a,b)$ is open in $\Rbad^1$. We will show that $[a,b)$ is not open in the usual topology, and thus $\R_{usual}$ and $\Rbad^1$ are different. It suffices to show that no open interval $(m,n)$ containing $a$ is a subset of $[a,b)$.	
	$$a \in (m,n) \implies m<a<n \implies m<\tfrac{m+a}{2}<a<n.$$
	Thus, $\tfrac{m+a}{2}\in (m,n)$ but $\tfrac{m+a}{2}\not\in[a,b)$, so $(m,n)\not\subset[a,b)$. 
	\end{proof}
	\item Prove that intervals $[a,b)$ are both open and closed in $\Rbad^1$.
	\begin{proof}
	Any interval $[a,b)$ is open in $\Rbad^1$ by definition. If $a>b$, then $[a,b)=\emptyset$ and is closed. If $a=b$, then $[a,b)=\{x:a\leq x<a\}=\emptyset$, so $[a,b)$ is closed in this case as well.
	
	Now, suppose $a<b$ and consider $[a,b)^\complement = (-\infty,a)\cup[b,\infty)$. Since $(-\infty,a)=\bigcup\limits_{n\in\N}[-n,a)$, and $[b,\infty)=\bigcup\limits_{n\in\N}[b,n)$, then $(-\infty,a)\cup[b,\infty)$ is a union of sets which are open in $\Rbad^1$. Therefore, $(-\infty,a)\cup[b,\infty)$ is also open in $\Rbad^1$, so $[a,b)$ is closed.
	\end{proof}
	\item Prove that every open interval $(a,b)$ is open in $\Rbad^1$. 
	\begin{proof}
	$(a,b)=\bigcup\limits_{n\in\N}\left[a+\tfrac{1}{n},b\right)$, so $(a,b)$ is a union of open sets, and thus is open. 
	\end{proof}
	\item Prove that the set of intervals of the form $[a,b)$ with $a,b \in \Q$ are the basis for a topology on R. Show that this topology is different from $\Rbad^1$. 
	\begin{proof} We will denote this topology as ${\Rbad^1}_\Q$. 
		\begin{itemize}
		\item Since $[1,0)=\{x\in\R:1\leq x < 0\}=\emptyset$, then $\emptyset\in{\Rbad^1}_\Q$. 
		\item For any $x\in\R$, $x\in [\floor{x},\floor{x}+1)$, so ${\Rbad^1}_\Q$ covers $\R$. 
		\item For any $a,b,c,d\in\Q$, 
		%Rbad_Q
		\[
		\begin{array}{rcl}
		[a,b)\cap[c,d)&=&\{x\in\R:a\leq x < b\}\cap \{x\in\R:c\leq x < d\}\\
		&=&\{x\in\R: \max(a,c)\leq x < \min(b,d)\}\\
		&=&[\max(a,c),\min(b,d))\\
		&\in& {\Rbad^1}_\Q\\
		\end{array}
		\]
		So, as desired according to Theorem 13, for any $[a,b), [c,d) \in {\Rbad^1}_\Q$ which both contain $x$, there exists $[a,b)\cap[c,d) \in {\Rbad^1}_\Q$ such that $x\in [a,b)\cap[c,d)$. 
		\end{itemize}
	Thus, ${\Rbad^1}_\Q$ forms the basis for a topology on $\R$.\qedwhite
	
	Now we will show that ${\Rbad^1}_\Q\neq \Rbad^1$. Consider the set $[\pi, 5)$. By definition, $[\pi, 5)$ is open in $\Rbad^1$. Now, $[\pi, 5)$ is not itself a basic open set in ${\Rbad^1}_\Q$, nor is it a union of basic sets in ${\Rbad^1}_\Q$, since any union of rational intervals $[a,b)$ must either disclude $\pi$, or include reals which are less than $\pi$. 
	\end{proof}
	\end{enumerate}

\item 
	\begin{enumerate}
	\item Show that the set of half-open rectangles of the form $\{(x, y) \in \R^2: a \leq x < b, c \leq y < d\}$ form the basis for a topology on $\R^2$. We will refer to $\R^2$ endowed with this topology as $\Rbad^2$.
	\begin{notation*}
	Let $[a,b)\times[c,d)$ denote a set $\{(x, y) \in \R^2: a \leq x < b, c \leq y < d\} \in \Rbad^2$.
	\end{notation*}
	\begin{proof}\mbox{}
	\begin{itemize}
	\item $[0,0)\times[0,0)=\emptyset$, so $\emptyset\in \Rbad^2$.
	\item Let $(x,y)\in \R^2$. Then, $(x,y)\in[x,x+1)\times[y,y+1)$, so $\Rbad^2$ covers $\R^2$. 
	\item For any $x_1, \ldots x_4, y_1, \ldots y_4 \in\R$, 
		%R2bad================================		
		\[
		\begin{array}{rcll}
		[x_1,x_2)\times[y_1,y_2)\cap[x_3,x_4)\times[y_3,y_4)&=& [\max(x_1,x_3),min(x_2,x_4))\times[\max(y_1,y_3),min(y_2,y_4))\\
		%&&\{(x,y)\in\R^2:	&x_1\leq x<x_2 \\
		%&&							&x_3\leq x<x_4 \\
		%&&					 		&y_1\leq y<y_2 \\
		%&&							&y_3\leq y<y_4\} \\
		%&=&\{x\in\R: \max(a,c)\leq x < \min(b,d)\}\\
		%&=&[\max(a,c),\min(b,d))\\
		&\in& \Rbad^2\\
		\end{array}
		\]
		So, as desired according to Theorem 13, this set of half-open rectangles is the basis for a topology on $\R^2$. \qedhere
	\end{itemize}
	\end{proof}
	
	\item Let $L_1$ denote the line $y = −x$ in $\R^2$. Show that the subspace topology on $L_1$, as a subspace of $\Rbad^2$, is the discrete topology.
	\begin{proof}
	Let $(x,-x)$ be any point on the line $y = −x$. Now, since the singleton $\{(x,-x)\} = [x,x+1)\times[-x,-x+1)\cap L_1$, then $\{(x,-x)\}$ is open in $L_1$. Thus, for any set $S\subset L_1$, the union $\bigcup\limits_{(x,-x)\in S}\{(x,-x)\}=S$ is open.
	\end{proof}
	
	\item Let $L_2$ denote the line $y = x$ in $\R^2$. Show that the subspace topology on $L_2$, as a subspace of $\Rbad^2$, is not the discrete topology.
	
	\begin{proof}
	To show that the subspace topology on $L_2$ is not the discrete topology, it suffices to produce a set which is not open. Consider the singleton $\{(0,0)\}$. If $\{(0,0)\}$ is open in $L_2$, then for any $(x,y)\in \{(0,0)\}$, there exists a basic open set $U$ containing $(x,y)$ such that $U\cap L_2 = \{(0,0)\}$. 
	
	Let $[a,b)\times[c,d)$ be an any set containing the origin which is a basic open set in $\Rbad^2$. Since $(0,0) \in [a,b)\times[c,d)$, then $b>0$ and $d>0$. Let $p=\min(b,d)$ Thus, $(\frac{p}{2},\frac{p}{2})\in[a,b)\times[c,d)\cap L_2$, but $(\frac{b}{2},\frac{d}{2})\not\in\{(0,0)\}$. Thus, there is no basic open set whose intersection with $L_2$ is $\{(0,0)\}$, so $\{(0,0)\}$ is not open. 
	\end{proof}		
	\end{enumerate}

\setcounter{enumi}{4}
\item Let $X$ be a set, and let $\{0,1\}^X$ denote the set of all functions $X \to \{0,1\}$. 

	\begin{enumerate}
	\item Prove that the collection of sets of the form $U(x, \epsilon) = \{ f \in \{0,1\}^X : f(x) = \epsilon\}$, for all $x \in X$ and $\epsilon \in \{0,1\}$ forms a subbasis for a topology on $\{0,1\}^X$. 
	\begin{proof} Let $\script{S}$ be the collection of all sets of the form $U(x, \epsilon) = \{ f \in \{0,1\}^X : f(x) = \epsilon\}$, with $x \in X$ and $\epsilon \in \{0,1\}$. Let $\B$ be the collection of all finite intersections of sets in $\script{S}$. 
		\begin{itemize}
		\item For some $x_0\in X$, consider the sets $U(x_0, 1)$ and $U(x_0, 0)$. $U(x_0, 1)\cap U(x_0, 0)=\emptyset$, so $\emptyset\in\B$. 
		\item Let $x_0$ be an arbitrary element of $X$, and let $f$ be an arbitrary function $f:X\to\{0,1\}$ where $f(x_0)=\epsilon_0$. Since $f\in U(x_0,\epsilon_0)$ by definition, then $\B$ covers $\{0,1\}^X$. 
		%\item Let $U_1, U_2$ be two sets in $\B$. since $\B$ consists of all finite intersections of sets in $\script{S}$, then there exist sets in $\script{S}$ such that 
		%$$U_1=\bigcap_{i=1}^m S_i, \quad U_2=\bigcap_{j=1}^n S_j.$$
		%So, $U_1\cap U_2 = \bigcap S_i \cap \bigcap S_j$ is a finite intersection of sets in $\script{S}$, so $U_1\cap U_2 \in \B$.
		\end{itemize}
	
	Thus, by Theorem 14, $\script{S}$ forms a subbasis for a topology on $\{0,1\}^X$. 
	\end{proof}
	\item Under what conditions are two basic open sets in this topology disjoint? \\
	\textbf{Answer: }Since every basic open set is a finite intersection of sets of the form $U(x,\epsilon)$, every basic open set $U\in\B$ has the following property: $U$ has a nonempty "characteristic set" $C\subset X$ such that for any fixed $x\in C$, $f(x)=g(x)$ for all $f,g\in U$. That is, all functions in $U$ are equal at every point in $C$.\\
	Thus, two basic open sets $U,V$ in $\B$ are disjoint if and only if their characteristic sets, $C(U), C(V)$ are equal; and for any $f\in U$ and $g\in V$, $f(x) \neq g(x)$ for all $x\in C(U)=C(V)$.
	
	\item Is this topology Hausdorff?\\
	\textbf{Answer: }Yes.
	\begin{proof}
	Let $f$ and $g$ be any two distinct functions in $\{0,1\}^X$. Since they are distinct, there exists at least one $x \in X$ such that $f(x)\neq g(x)$. Without loss of generality, suppose $f(x)=1$ and $g(x)=0$. Therefore, the basic open sets $U(x,1)$ and $U(x,0)$ contain $f$ and $g$, respectively. Since their characteristic sets are equal but they contain functions which are not equal at $x\in C$, we can conclude that $U(x,1)$ and $U(x,0)$ are disjoint. Therefore, this topology is Hausdorff. 
	\end{proof}	 
	\end{enumerate}		

\item 
	\begin{enumerate}
	\item Show that the collection consisting of $\emptyset$ and the set of all intervals $[a,b]$ with $a < b$ does not form the basis for a topology on $\R$.
	\begin{proof}
	In order for this collection of sets to be a basis for some topology on $\R$, it must be true that for any two basic sets $U,V$ with $x\in U\cap V$, there exists another basic set $W$ such that $x\in W \subset U\cap V$. However, consider the basic sets 
	$$[j,k]\text{ and }[k,m].$$
	The element $k$ is in the intersection $[j,k]\cap[k,m]=\{k\}$, but the set $\{k\}$ cannot contain any interval $[a,b]$; since $a<b$ implies that $[a,b]$ contains more than just one element. 
	\end{proof}
	
	\item Show that the collection consisting of $\emptyset$ and the set of all intervals $[a,b]$ with $a < b$ does form a subbasis for a topology on $\R$.  That topology is one we have seen before. Identify it.\\
	\\
	\textbf{Claim: }Let $\script{S}$ be the collection consisting of $\emptyset$ and the set of all intervals $[a,b]$ with $a < b$, and let $\B$ be the collection of all finite intersections of sets in $\script{S}$. Then, $\B$ is a basis for the discrete topology on $\R$.
	\begin{proof} First, since $[1,2]\cap[3,4]=\emptyset$, then $\emptyset\in\B$. Now, Let $S$ be an arbitrary subset of $\R$, and let $x$ be any real number such that $x\in S$. We can see that $\{x\}\in\B$ by observing that $[x-,x]\cap[x,x+1]=\{x\}$, so we can take the union $\bigcup\limits_{x\in S}\{x\}=S$. Thus, $S$ is a union of open sets, so $S$ is open. Therefore, since any arbitrary $S\subset\R$ is open in this topology, then $\B$ is a basis for the discrete topology on $\R$.
	\end{proof}
	
	\end{enumerate}

\end{enumerate}

\end{document}