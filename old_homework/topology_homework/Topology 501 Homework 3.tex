\documentclass[letterpaper]{article}
%\documentclass[a5paper]{article}

%% Language and font encodings
\usepackage[english]{babel}
\usepackage[utf8x]{inputenc}
\usepackage[T1]{fontenc}

%% Sets page size and margins
\usepackage[letterpaper,top=1in,bottom=1in,left=1in,right=1in,marginparwidth=1.75cm]{geometry}
%\usepackage[a5paper,top=1cm,bottom=1cm,left=1cm,right=1.5cm,marginparwidth=1.75cm]{geometry}

\usepackage{graphicx}
\graphicspath{ {./} }	  %%where to look for images

%% Useful packages
\usepackage{amssymb, amsmath, amsthm} 
%\usepackage{graphicx}  %%this is currently enabled in the default document, so it is commented out here. 
\usepackage{calrsfs}
\usepackage{braket}
\usepackage{mathtools}
\usepackage{lipsum}
\usepackage{tikz}
\usetikzlibrary{cd}
\usepackage{verbatim}
%\usepackage{ntheorem}% for theorem-like environments
\usepackage{mdframed}%can make highlighted boxes of text
%Use case: https://tex.stackexchange.com/questions/46828/how-to-highlight-important-parts-with-a-gray-background
\usepackage{wrapfig}
\usepackage{centernot}
\usepackage{subcaption}%\begin{subfigure}{0.5\textwidth}
\usepackage{pgfplots}
\pgfplotsset{compat=1.13}
\usepackage[colorinlistoftodos]{todonotes}
\usepackage[colorlinks=true, allcolors=blue]{hyperref}
\usepackage{xfrac}					%to make slanted fractions \sfrac{numerator}{denominator}
\usepackage{enumitem}            
    %syntax: \begin{enumerate}[label=(\alph*)]
    %possible arguments: f \alph*, \Alph*, \arabic*, \roman* and \Roman*
\usetikzlibrary{arrows,shapes.geometric,fit}

\DeclareMathAlphabet{\pazocal}{OMS}{zplm}{m}{n}
%% Use \pazocal{letter} to typeset a letter in the other kind 
%%  of math calligraphic font. 

%% This puts the QED block at the end of each proof, the way I like it. 
\renewenvironment{proof}{{\bfseries Proof}}{\qed}
\makeatletter
\renewenvironment{proof}[1][\bfseries \proofname]{\par
  \pushQED{\qed}%
  \normalfont \topsep6\p@\@plus6\p@\relax
  \trivlist
  %\itemindent\normalparindent
  \item[\hskip\labelsep
        \scshape
    #1\@addpunct{}]\ignorespaces
}{%
  \popQED\endtrivlist\@endpefalse
}
\makeatother

%% This adds a \rewnewtheorem command, which enables me to override the settings for theorems contained in this document.
\makeatletter
\def\renewtheorem#1{%
  \expandafter\let\csname#1\endcsname\relax
  \expandafter\let\csname c@#1\endcsname\relax
  \gdef\renewtheorem@envname{#1}
  \renewtheorem@secpar
}
\def\renewtheorem@secpar{\@ifnextchar[{\renewtheorem@numberedlike}{\renewtheorem@nonumberedlike}}
\def\renewtheorem@numberedlike[#1]#2{\newtheorem{\renewtheorem@envname}[#1]{#2}}
\def\renewtheorem@nonumberedlike#1{  
\def\renewtheorem@caption{#1}
\edef\renewtheorem@nowithin{\noexpand\newtheorem{\renewtheorem@envname}{\renewtheorem@caption}}
\renewtheorem@thirdpar
}
\def\renewtheorem@thirdpar{\@ifnextchar[{\renewtheorem@within}{\renewtheorem@nowithin}}
\def\renewtheorem@within[#1]{\renewtheorem@nowithin[#1]}
\makeatother

%% This makes theorems and definitions with names show up in bold, the way I like it. 
\makeatletter
\def\th@plain{%
  \thm@notefont{}% same as heading font
  \itshape % body font
}
\def\th@definition{%
  \thm@notefont{}% same as heading font
  \normalfont % body font
}
\makeatother

%===============================================
%==============Shortcut Commands================
%===============================================
\newcommand{\ds}{\displaystyle}
\newcommand{\B}{\mathcal{B}}
\newcommand{\C}{\mathbb{C}}
\newcommand{\F}{\mathbb{F}}
\newcommand{\N}{\mathbb{N}}
\newcommand{\R}{\mathbb{R}}
\newcommand{\Q}{\mathbb{Q}}
\newcommand{\T}{\mathcal{T}}
\newcommand{\Z}{\mathbb{Z}}
\renewcommand\qedsymbol{$\blacksquare$}
\newcommand{\qedwhite}{\hfill\ensuremath{\square}}
\newcommand*\conj[1]{\overline{#1}}
\newcommand*\closure[1]{\overline{#1}}
\newcommand*\mean[1]{\overline{#1}}
%\newcommand{\inner}[1]{\left< #1 \right>}
\newcommand{\inner}[2]{\left< #1, #2 \right>}
\newcommand{\powerset}[1]{\pazocal{P}(#1)}
%% Use \pazocal{letter} to typeset a letter in the other kind 
%%  of math calligraphic font. 
\newcommand{\cardinality}[1]{\left| #1 \right|}
\newcommand{\domain}[1]{\mathcal{D}(#1)}
\newcommand{\image}{\text{Im}}
\newcommand{\inv}[1]{#1^{-1}}
\newcommand{\preimage}[2]{#1^{-1}\left(#2\right)}
\newcommand{\script}[1]{\mathcal{#1}}


\newenvironment{highlight}{\begin{mdframed}[backgroundcolor=gray!20]}{\end{mdframed}}

\DeclarePairedDelimiter\ceil{\lceil}{\rceil}
\DeclarePairedDelimiter\floor{\lfloor}{\rfloor}

%===============================================
%===============My Tikz Commands================
%===============================================
\newcommand{\drawsquiggle}[1]{\draw[shift={(#1,0)}] (.005,.05) -- (-.005,.02) -- (.005,-.02) -- (-.005,-.05);}
\newcommand{\drawpoint}[2]{\draw[*-*] (#1,0.01) node[below, shift={(0,-.2)}] {#2};}
\newcommand{\drawopoint}[2]{\draw[o-o] (#1,0.01) node[below, shift={(0,-.2)}] {#2};}
\newcommand{\drawlpoint}[2]{\draw (#1,0.02) -- (#1,-0.02) node[below] {#2};}
\newcommand{\drawlbrack}[2]{\draw (#1+.01,0.02) --(#1,0.02) -- (#1,-0.02) -- (#1+.01,-0.02) node[below, shift={(-.01,0)}] {#2};}
\newcommand{\drawrbrack}[2]{\draw (#1-.01,0.02) --(#1,0.02) -- (#1,-0.02) -- (#1-.01,-0.02) node[below, shift={(+.01,0)}] {#2};}

%***********************************************
%**************Start of Document****************
%***********************************************

%===============================================
%===============Theorem Styles==================
%===============================================

%================Default Style==================
\theoremstyle{plain}% is the default. it sets the text in italic and adds extra space above and below the \newtheorems listed below it in the input. it is recommended for theorems, corollaries, lemmas, propositions, conjectures, criteria, and (possibly; depends on the subject area) algorithms.
\newtheorem{theorem}{Theorem}
\numberwithin{theorem}{section} %This sets the numbering system for theorems to number them down to the {argument} level. I have it set to number down to the {section} level right now.
\newtheorem*{theorem*}{Theorem} %Theorem with no numbering
\newtheorem{corollary}[theorem]{Corollary}
\newtheorem*{corollary*}{Corollary}
\newtheorem{conjecture}[theorem]{Conjecture}
\newtheorem{lemma}[theorem]{Lemma}
\newtheorem*{lemma*}{Lemma}
\newtheorem{proposition}[theorem]{Proposition}
\newtheorem*{proposition*}{Proposition}
\newtheorem{problemstatement}[theorem]{Problem Statement}


%==============Definition Style=================
\theoremstyle{definition}% adds extra space above and below, but sets the text in roman. it is recommended for definitions, conditions, problems, and examples; i've alse seen it used for exercises.
\newtheorem{definition}[theorem]{Definition}
\newtheorem*{definition*}{Definition}
\newtheorem{condition}[theorem]{Condition}
\newtheorem{problem}[theorem]{Problem}
\newtheorem{example}[theorem]{Example}
\newtheorem*{example*}{Example}
\newtheorem*{counterexample*}{Counterexample}
\newtheorem*{romantheorem*}{Theorem} %Theorem with no numbering
\newtheorem{exercise}{Exercise}
\numberwithin{exercise}{section}
\newtheorem{algorithm}[theorem]{Algorithm}

%================Remark Style===================
\theoremstyle{remark}% is set in roman, with no additional space above or below. it is recommended for remarks, notes, notation, claims, summaries, acknowledgments, cases, and conclusions.
\newtheorem{remark}[theorem]{Remark}
\newtheorem*{remark*}{Remark}
\newtheorem{notation}[theorem]{Notation}
\newtheorem*{notation*}{Notation}
%\newtheorem{claim}[theorem]{Claim}  %%use this if you ever want claims to be numbered
\newtheorem*{claim}{Claim}



\pgfplotsset{compat=1.13}

\newcommand{\T}{\mathcal{T}}

\title{Math 501 \linebreak
Homework 3}
\author{Trevor Klar}

\begin{document}

\maketitle


\begin{enumerate}
\item Prove 

\begin{theorem*}\emph{\textbf{1: (Openness Criterion)}}
Let $(X, \T)$ be a topological space. A set $S \subset X$ is open if and only if for every $x \in S$, there exists an open set $U_x \subset S$. 
\end{theorem*}

\begin{proof}\mbox{}\\
$\implies$: Suppose $S \subset X$ is open. For any $x \in S$, let $U_x = S$. $U_x$ is an open set such that $x \in U_x \subset S$, so we are done. \qedwhite \\
$\impliedby$ Suppose for every $x \in S$, there exists an open set $U_x \subset S$. Consider 
$$\bigcup_{x\in S}U_x.$$
Now, every element of $\bigcup_{x\in S} U_x$ is in $S$, and every element of $S$ is in $U_x \subset \bigcup_{x\in S} U_x$, so $\bigcup_{x\in S} U_x = S$. Since any arbitrary union of open sets is open, $S$ is open. 
\end{proof}

\item Prove that in a Hausdorff space, a set consisting of a single point is a closed set. 

\begin{proof}
Let $(X,\T)$ be a Hausdorff space, and let $x_0$ be any point in $X$. \\
\textbf{Claim:} $S = X-\{x_0\}$ is open, so $\{x_o\}$ is closed.\\
Since $(X,\T)$ is Hausdorff, for any $x \in X$ which is distinct from $x_0$, there exist open sets $U_x \in \T$ and $V_x \in \T$ such that 
$$x \in U_x, \, x_0 \in V_x, \, \text{and }U_x\cap V_x=\emptyset.$$ 
%Note that since $U_x\cap V_x=\emptyset$, $x_0 \not\in U_x$. 
Now consider $\bigcup_{x\in S}U_x \equiv \bigcup U_x$. For every $x \in S$, 
$$x \in U_x \subset \bigcup U_x, $$
and every $U_x \subset S$ since $x_0 \not\in U_x \subset X$, which means that $\bigcup U_x \subset S$. Therefore, $\bigcup U_x = S$.

Thus, we have shown that $S$ can be written as a union of open sets, and since an arbitrary union of open sets is open, $S = X-\{x_0\}$ is open in $X$, so $\{x_o\}$ is closed in $X$.
\end{proof}

\item Let $U$ be open and $C$ closed subsets of a 
space $X$, with $C \subset U$. Prove that $U-C$ is open.
\begin{proof} 
By definition of set subtraction, $U-C=U\cap (X-C)$, and since $C$ is closed in $X$, $(X-C)$ is open. Thus, $U-C$ is the intersction of two open sets, so it is open.
\end{proof} 
\pagebreak
\item Let $A$ be a subset of a space $X$. Prove that $C$ is closed in $A$ if and only if $C$ = $A\cap F$, where $F$ is closed in $X$.
\begin{proof}\mbox{}\\
$\implies$: Assume $C$ is closed in $A$. By definition of closed in $A$, there exists some $U$ which is open in $X$ such that $$A-C=A\cap U.$$ Let $F = X-U$. Now, $F$ and $U$ are complements in $X$, and $A \subset X$. This means that $A\cap U$ and $A \cap F$ are complements in $A$. Thus, taking the complement in $A$ of both sides, we find that 
$$A-C=A\cap U \implies C=A\cap F.$$ 
This completes the proof. \qed

$\impliedby$: Assume $C$ = $A\cap F$, where $F$ is closed in $X$. Let $U=X-F$. Then, using the same reasoning as above, we can take the complement in $A$ of both sides to obtain
$$C=A\cap F \implies A-C=A\cap U.$$
Thus, $C$ is closed in $A$ by definition.
\end{proof}

\item Find $\closure{A}$, int $A$, and $A^\ell$ for the following sets $A$ in $\R^2$. (Just answers, no proofs.)
	\begin{enumerate}[label=(\alph*)]
	\item $A=\{(x,0): 0 \leq x < 1\}$
		\begin{itemize}
		\item $\closure{A} = \{(x,0): 0 \leq x \leq 1\}$
		\item int$(A) = \emptyset$
		\item $A^\ell = \closure{A} = \{(x,0): 0 \leq x \leq 1\}$
		\end{itemize}
	\item $A=\{(x,y):x^2+y^2\leq10\}$
		\begin{itemize}
		\item $\closure{A} = A$
		\item int$(A) = \{(x,y):x^2+y^2<10\}$
		\item $A^\ell = A$
		\end{itemize}
	\item $A=\{(x,y):x,y\in\Q$, $x^2+y^2\leq10\}$
		\begin{itemize}
		\item $\closure{A} = \{(x,y):x^2+y^2\leq10\}$
		\item int$(A) = \emptyset$
		\item $A^\ell = \closure{A} = \{(x,y):x^2+y^2\leq10\}$
		\end{itemize}
	\item $A=\{(x,y):x,y\in\Z$, $x^2+y^2\leq10\}$
		\begin{itemize}
		\item $\closure{A} = A$
		\item int$(A) = \emptyset$
		\item $A^\ell = \emptyset$
		\end{itemize}
	\end{enumerate}

\item Let $\R_f^1$ denote the real numbers endowed with the finite complement topology. What is the set of limit points of $\Z$ in $\R_f^1$?

\textbf{Claim:} $\Z^\ell$ in $\R_f^1$ is $\R$. 
\begin{proof}
In the finite complement topology on $\R$, every open set has a finite complement; that is, there is a greatest element of $\R$ which the set does not contain. This means every open set is unbounded above. Since $\Z$ is also unbounded above, every set which is open in $\R$ contains elements of $\Z$. Thus, for any real number $x$, every open set containing $x$ also contains elements of $\Z$ distinct from $x$, so $x$ is a limit point.
\end{proof}
\pagebreak
\item Let $X$ be a space, $A$ a subset of $X$. A point $p \in A$ is called an \emph{isolated point} of $A$ if $p$ is not a limit point of $A$.
	\begin{enumerate}[label=(\alph*)]
	\item What is the set of all isolated points of $\Z$ in $\R$, where $\R$ has the usual topology?\\
	\textbf{Claim: }The set of all isolated points of $\Z$ is $\Z$.
	\begin{proof}
	For any integer $n$, the interval $B=(n-\frac{1}{2},n+\frac{1}{2})$ is an open set such that $A\cap(B-\{n\})=\emptyset$, so every integer is an isolated point. 
	\end{proof}
	\item What is the set of all isolated points of $\Z$ in $\R_f^1$, where $\R_f^1$ denotes the finite complement topology?\\
	\textbf{Answer:} The set of all isolated points of $\Z$ in $\R_f^1$ is $\emptyset$, since we proved in Exercise 6 that $\Z^\ell$ in $\R_f^1$ is $\R$, and $\Z \subset	\R$. 
	\end{enumerate}

\item Prove that in a Hausdorff space $X$ with subset $A$, $x$ is a limit point of $A$ if and only if every open set containing $x$ contains infinitely many points of $A$.

\begin{proof}\mbox{}\\
$\impliedby$: If every open set $U$ containing $x$ contains infinitely many points of $A$, then $A\cap(U-\{x\})\neq \emptyset$, so $x$ is a limit point and we are done. 

$\implies$: Assume that $x$ is a limit point of $A$, and let $U$ be an arbitrary open set which contains $x$. Suppose for contradiction that $U$ contains only finitely many elements of $A$, denoted $a_1, a_2, \ldots, a_n$. Since $X$ is Hausdorff, then $U$ with the subspace topology is also Hausdorff, so
there exist $n+1$ disjoint sets which are open in $U$ and contain $a_1, a_2, \ldots, a_n,$ and $x$, respectively. Let $U_x$ denote the last of these sets, which intersects with $A$ only at the point $x$. Now, by definition, $U_x$ is the intersection of some open set and $U$, so $U_x$ is also open in $X$. 

Since $x$ is a limit point of $A$, then $A\cap(U_x-\{x\})\neq\emptyset$, so $U_x$ contains at least one element of $A$ distinct from $x$, which contradicts our construction of $U_x$. Therefore, every open set $U$ contains infinitely many points of $A$. 
\end{proof}

\item Suppose that $A$ is a subset of a space $X$. Prove that $\closure{A} = A\cup A^\ell$. 

\item Let $A$ be a subset of a space $X$. We say $A$ is \emph{dense} in $X$ if $\closure{A} = X$. (For example, $\Q$ is dense in $\R$.)\

Prove that $A$ is dense in $X$ if and only if for every nonempty open set $U$ in $X$, we have $U\cap A \neq \emptyset$.

\begin{proof}\mbox{}\\
$\implies$: Suppose $A$ is dense in $X$. Then, by definition, 
$$\closure{A} = \bigcap\{F: F\subset X\text{ is closed and } F \supset A\} = X,$$
so the only closed set which contains $A$ is $X$. Let $U$ be any nonempty open set in $X$. Now, $(X-U)$ is closed and not equal to $X$, so it does not contain $A$. This means that $U$, the complement of $(X-U)$, must contain some elements of $A$. So, $U\cap A\neq\emptyset$ and we are done. 

$\impliedby$: Suppose that for every nonempty open set $U$ in $X$, it is true that $U\cap A \neq \emptyset$. Let $F$ be any closed set in $X$ which is not equal to $X$. Now, $(X-F)$ is a nonempty open set in $X$, so $(X-F)\cap A\neq\emptyset$. This means that $F\not\supset A$. Since it is given that $A\subset X$, $X$ can be the only closed set which contains $A$, so $\closure{A}=X$. 
\end{proof}

\end{enumerate}

\end{document}