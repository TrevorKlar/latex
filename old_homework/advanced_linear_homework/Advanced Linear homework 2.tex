\documentclass[letterpaper]{article}
%\documentclass[a5paper]{article}

%% Language and font encodings
\usepackage[english]{babel}
\usepackage[utf8x]{inputenc}
\usepackage[T1]{fontenc}

%% Sets page size and margins
\usepackage[letterpaper,top=1in,bottom=1in,left=1in,right=1in,marginparwidth=1.75cm]{geometry}
%\usepackage[a5paper,top=1cm,bottom=1cm,left=1cm,right=1.5cm,marginparwidth=1.75cm]{geometry}

%% Useful packages
\usepackage{amssymb, amsmath, amsthm} 
%\usepackage{graphicx}  %%this is currently enabled in the default document, so it is commented out here. 
\usepackage{calrsfs}
\usepackage{braket}
\usepackage{mathtools}
\usepackage{lipsum}
\usepackage{tikz}
\usetikzlibrary{cd}
\usepackage{verbatim}
%\usepackage{ntheorem}% for theorem-like environments
\usepackage{mdframed}%can make highlighted boxes of text
%Use case: https://tex.stackexchange.com/questions/46828/how-to-highlight-important-parts-with-a-gray-background
\usepackage{wrapfig}
\usepackage{centernot}
\usepackage{subcaption}%\begin{subfigure}{0.5\textwidth}
\usepackage{pgfplots}
\pgfplotsset{compat=1.13}
\usepackage[colorinlistoftodos]{todonotes}
\usepackage[colorlinks=true, allcolors=blue]{hyperref}
\usepackage{xfrac}					%to make slanted fractions \sfrac{numerator}{denominator}
\usepackage{enumitem}            
    %syntax: \begin{enumerate}[label=(\alph*)]
    %possible arguments: f \alph*, \Alph*, \arabic*, \roman* and \Roman*
\usetikzlibrary{arrows,shapes.geometric,fit}

\DeclareMathAlphabet{\pazocal}{OMS}{zplm}{m}{n}
%% Use \pazocal{letter} to typeset a letter in the other kind 
%%  of math calligraphic font. 

%% This puts the QED block at the end of each proof, the way I like it. 
\renewenvironment{proof}{{\bfseries Proof}}{\qed}
\makeatletter
\renewenvironment{proof}[1][\bfseries \proofname]{\par
  \pushQED{\qed}%
  \normalfont \topsep6\p@\@plus6\p@\relax
  \trivlist
  %\itemindent\normalparindent
  \item[\hskip\labelsep
        \scshape
    #1\@addpunct{}]\ignorespaces
}{%
  \popQED\endtrivlist\@endpefalse
}
\makeatother

%% This adds a \rewnewtheorem command, which enables me to override the settings for theorems contained in this document.
\makeatletter
\def\renewtheorem#1{%
  \expandafter\let\csname#1\endcsname\relax
  \expandafter\let\csname c@#1\endcsname\relax
  \gdef\renewtheorem@envname{#1}
  \renewtheorem@secpar
}
\def\renewtheorem@secpar{\@ifnextchar[{\renewtheorem@numberedlike}{\renewtheorem@nonumberedlike}}
\def\renewtheorem@numberedlike[#1]#2{\newtheorem{\renewtheorem@envname}[#1]{#2}}
\def\renewtheorem@nonumberedlike#1{  
\def\renewtheorem@caption{#1}
\edef\renewtheorem@nowithin{\noexpand\newtheorem{\renewtheorem@envname}{\renewtheorem@caption}}
\renewtheorem@thirdpar
}
\def\renewtheorem@thirdpar{\@ifnextchar[{\renewtheorem@within}{\renewtheorem@nowithin}}
\def\renewtheorem@within[#1]{\renewtheorem@nowithin[#1]}
\makeatother

%% This makes theorems and definitions with names show up in bold, the way I like it. 
\makeatletter
\def\th@plain{%
  \thm@notefont{}% same as heading font
  \itshape % body font
}
\def\th@definition{%
  \thm@notefont{}% same as heading font
  \normalfont % body font
}
\makeatother

%===============================================
%==============Shortcut Commands================
%===============================================
\newcommand{\ds}{\displaystyle}
\newcommand{\B}{\mathcal{B}}
\newcommand{\C}{\mathbb{C}}
\newcommand{\F}{\mathbb{F}}
\newcommand{\N}{\mathbb{N}}
\newcommand{\R}{\mathbb{R}}
\newcommand{\Q}{\mathbb{Q}}
\newcommand{\T}{\mathcal{T}}
\newcommand{\Z}{\mathbb{Z}}
\renewcommand\qedsymbol{$\blacksquare$}
\newcommand{\qedwhite}{\hfill\ensuremath{\square}}
\newcommand*\conj[1]{\overline{#1}}
\newcommand*\closure[1]{\overline{#1}}
\newcommand*\mean[1]{\overline{#1}}
%\newcommand{\inner}[1]{\left< #1 \right>}
\newcommand{\inner}[2]{\left< #1, #2 \right>}
\newcommand{\powerset}[1]{\pazocal{P}(#1)}
%% Use \pazocal{letter} to typeset a letter in the other kind 
%%  of math calligraphic font. 
\newcommand{\cardinality}[1]{\left| #1 \right|}
\newcommand{\domain}[1]{\mathcal{D}(#1)}
\newcommand{\image}{\text{Im}}
\newcommand{\inv}[1]{#1^{-1}}
\newcommand{\preimage}[2]{#1^{-1}\left(#2\right)}
\newcommand{\script}[1]{\mathcal{#1}}


\newenvironment{highlight}{\begin{mdframed}[backgroundcolor=gray!20]}{\end{mdframed}}

\DeclarePairedDelimiter\ceil{\lceil}{\rceil}
\DeclarePairedDelimiter\floor{\lfloor}{\rfloor}

%===============================================
%===============My Tikz Commands================
%===============================================
\newcommand{\drawsquiggle}[1]{\draw[shift={(#1,0)}] (.005,.05) -- (-.005,.02) -- (.005,-.02) -- (-.005,-.05);}
\newcommand{\drawpoint}[2]{\draw[*-*] (#1,0.01) node[below, shift={(0,-.2)}] {#2};}
\newcommand{\drawopoint}[2]{\draw[o-o] (#1,0.01) node[below, shift={(0,-.2)}] {#2};}
\newcommand{\drawlpoint}[2]{\draw (#1,0.02) -- (#1,-0.02) node[below] {#2};}
\newcommand{\drawlbrack}[2]{\draw (#1+.01,0.02) --(#1,0.02) -- (#1,-0.02) -- (#1+.01,-0.02) node[below, shift={(-.01,0)}] {#2};}
\newcommand{\drawrbrack}[2]{\draw (#1-.01,0.02) --(#1,0.02) -- (#1,-0.02) -- (#1-.01,-0.02) node[below, shift={(+.01,0)}] {#2};}

%***********************************************
%**************Start of Document****************
%***********************************************

%===============================================
%===============Theorem Styles==================
%===============================================

%================Default Style==================
\theoremstyle{plain}% is the default. it sets the text in italic and adds extra space above and below the \newtheorems listed below it in the input. it is recommended for theorems, corollaries, lemmas, propositions, conjectures, criteria, and (possibly; depends on the subject area) algorithms.
\newtheorem{theorem}{Theorem}
\numberwithin{theorem}{section} %This sets the numbering system for theorems to number them down to the {argument} level. I have it set to number down to the {section} level right now.
\newtheorem*{theorem*}{Theorem} %Theorem with no numbering
\newtheorem{corollary}[theorem]{Corollary}
\newtheorem*{corollary*}{Corollary}
\newtheorem{conjecture}[theorem]{Conjecture}
\newtheorem{lemma}[theorem]{Lemma}
\newtheorem*{lemma*}{Lemma}
\newtheorem{proposition}[theorem]{Proposition}
\newtheorem*{proposition*}{Proposition}
\newtheorem{problemstatement}[theorem]{Problem Statement}


%==============Definition Style=================
\theoremstyle{definition}% adds extra space above and below, but sets the text in roman. it is recommended for definitions, conditions, problems, and examples; i've alse seen it used for exercises.
\newtheorem{definition}[theorem]{Definition}
\newtheorem*{definition*}{Definition}
\newtheorem{condition}[theorem]{Condition}
\newtheorem{problem}[theorem]{Problem}
\newtheorem{example}[theorem]{Example}
\newtheorem*{example*}{Example}
\newtheorem*{counterexample*}{Counterexample}
\newtheorem*{romantheorem*}{Theorem} %Theorem with no numbering
\newtheorem{exercise}{Exercise}
\numberwithin{exercise}{section}
\newtheorem{algorithm}[theorem]{Algorithm}

%================Remark Style===================
\theoremstyle{remark}% is set in roman, with no additional space above or below. it is recommended for remarks, notes, notation, claims, summaries, acknowledgments, cases, and conclusions.
\newtheorem{remark}[theorem]{Remark}
\newtheorem*{remark*}{Remark}
\newtheorem{notation}[theorem]{Notation}
\newtheorem*{notation*}{Notation}
%\newtheorem{claim}[theorem]{Claim}  %%use this if you ever want claims to be numbered
\newtheorem*{claim}{Claim}



\pgfplotsset{compat=1.13}

\title{Math 462 - Advanced Linear Algebra \linebreak
	Homework 2}
\author{Trevor Klar}

\begin{document}

\maketitle

\noindent \textbf{Exercises:}
\begin{enumerate}
\setcounter{enumi}{4}
\item Let $G_0$ and $G_1$ be groups. Consider the set 
$$G_0\times G_1 = \{(s_0,s_1):s_0 \in G_0, s_1 \in G_1\}$$
This is just the Cartesian product of the two sets $G_0$ and $G_1$. Define an operation on $G_0\times G_1$ as follows:
$$(s_0,s_1)(t_0,t_1) = (s_0t_0,s_1t_1) \quad \forall s_0,t_0\in G_0, s_1,t_1 \in G_1$$
Show that $G_0\times G_1$ is a group with respect to this operation. 
\begin{proof}
To show that $G_0\times G_1$ is a group with respect to this operation, it suffices to show that the associative, identity, and inverse properties hold.

Associative property: Consider the product $(s_0,s_1)(t_0,t_1)(u_0,u_1)$. 
\[
\begin{array}{rcl}
((s_0,s_1)(t_0,t_1))(u_0,u_1) &=& (s_0t_0,s_1t_1)(u_0,u_1) \\
&=& ((s_0t_0)u_0,(s_1t_1)u_1)\\
&=& (s_0(t_0u_0),(s_1(t_1u_1))\\
&=& (s_0,s_1)(t_0u_0,t_1u_1)\\
&=& (s_0,s_1)((t_0,t_1)(u_0,u_1))\\
\end{array}
\]

Identity property: If $e_0$ and $e_1$ are the identities of $G_0$ and $G_1$, respectively, then $(e_0,e_1)$ is the identity of $G_0\times G_1$ as shown below:
\[
\begin{array}{rcl}
(s_0,s_1)(e_0,e_1) &=& (s_0e_0,s_1e_1)\\
&=& (s_0,s_1)\\
\end{array}
\]

Inverse property: Since $G_0$ and $G_1$ are groups, then the inverse of any $(s_0,s_1)$ can be found using the inverses of its components; that is, $(s_0^{-1},s_1^{-1})$. Proof follows:
\[
\begin{array}{rcl}
(s_0,s_1)(s_0^{-1},s_1^{-1}) &=& (s_0s_0^{-1},s_1s_1^{-1})\\
&=& (e_0,e_1)\\
\end{array}
\]
\end{proof}

\item Show that $G_0\times G_1 \cong G_1\times G_0$. (Explicitly construct the isomorphism. This is easy.)
\begin{example*}
Let $f(s_0,s_1)=(s_1,s_0)$. First we will show that $f$ is a homomorphism:
\[
\begin{array}{rcl}
f((s_0,s_1)(t_0,t_1)) &=& f(s_0t_0,s_1t_1)\\
&=& (s_1t_1,s_0t_0)\\
&=& (s_1,s_0)(t_1,t_0)\\
&=& f(s_0,s_1)f(t_0,t_1)\\
\end{array}
\]

Next, we mention that $f$ is clearly bejective. $(s_0,s_1)$ can clearly be the only element which maps to $(s_1,s_0)$, so $f$ is 1-1. Similarly, given any $(s_1,s_0)$, there is always a corresponding $(s_0,s_1)$ such that $f(s_0,s_1) = (s_1,s_0)$. \qed
\end{example*}

\item Continuing in this same context, consider the functions 
\[
\begin{array}{rcl}
\rho_0:G_0\times G_1 &\to& G_0\\
(s_0,s_1) &\mapsto& s_0\\
\\
\rho_1:G_0\times G_1 &\to& G_1\\
(s_0,s_1) &\mapsto& s_1\\
\end{array}
\]
These are called \emph{projection maps}. Show that both maps are surjective homomorphisms and compute the kernel of each. 

\begin{proof}\mbox{}\\
\begin{itemize}
\item $\rho_0$ is surjective because given any $s_0 \in G_0$, we can choose an arbitrary $s_1 \in G_1$ and find that $\rho_0(s_0,s_1)=s_0$. 

\item $\rho_0$ is a homomorphism because 
\[
\begin{array}{rcl}
\rho_0((s_0,s_1)(t_0,t_1)) &=& \rho_0(s_0t_0,s_1t_1)\\
&=& s_0t_0\\
&=& (s_0)(t_0)\\
&=& \rho_0(s_0,s_1)\rho_0(t_0,t_1)\\
\end{array}
\]

\item $\ker(\rho_0)=\{(e_0,s_1): \forall s_1 \in G_1\}$, where $e_0$ is the identity of $G_0$.

\item $\rho_1$ is surjective because given any $s_1 \in G_1$, we can choose an arbitrary $s_0 \in G_0$ and find that $\rho_1(s_0,s_1)=s_1$. 

\item $\rho_0$ is a homomorphism because 
\[
\begin{array}{rcl}
\rho_1((s_0,s_1)(t_0,t_1)) &=& \rho_1(s_0t_0,s_1t_1)\\
&=& s_1t_1\\
&=& (s_1)(t_1)\\
&=& \rho_1(s_0,s_1)\rho_1(t_0,t_1)\\
\end{array}
\]

\item $\ker(\rho_1)=\{(s_0,e_1): \forall s_0 \in G_0\}$, where $e_1$ is the identity of $G_1$.
\end{itemize}
\end{proof}

\item Consider the special case of the direct product $G\times G$ of a group $G$ with itself. Define a subset $D$ of $G\times G$ by 
$$D=\{(s,s):s\in G\}$$
That is, $D$ consists of all elements with both coordinates equal. Show that $D$ is a subgroup of $G\times G$. This is called the \emph{diagonal subgroup}. Do you see why? 

\begin{proof}
To show that $D$ is a subgoup of $G\times G$, it suffices to show that $D\subset G\times G$ and $D$ is closed under the operation. It is given that $D\subset G\times G$. 

Let $s,t \in G$ such that $st = u$. Then, $(s,s),(t,t) \in D$. 
\[
\begin{array}{rcl}
(s,s)(t,t)&=&(st,st)\\
&=&(u,u)\\
&\in&D
\end{array}
\]
\end{proof}

I can see that this is called the diagonal subgroup because a table or graph of $G\times G$ will list the elements of $D$ on its main diagonal. 

\item Consider the direct product $\R\times\R$ of the additive group of real numbers with itself and the function $j:\R\times\R\to\R$ defined by $j(x,y)=2x-y$. Show that $j$ is a homomorphism of groups; describe its kernel and image. 

\[
\begin{array}{rcl}
j:(\R^2,+) &\to& (\R,+)\\
(x,y) &\mapsto& 2x+y
\end{array}
\]

\textbf{Claim: }$j$ is a homomorphism.
\begin{proof}
\[
\begin{array}{rcl}
j:((x_1,y_1)+(x_2,y_2)) &=& j:(x_1+x_2,y_1+y_2)\\
&=& 2(x_1+x_2)-(y_1+y_2)\\
&=& 2x_1+2x_2-y_1-y_2\\
&=& (2x_1-y_1)+(2x_2-y_2)\\
&=& j:(x_1,y_1)+j(x_2,y_2)\\
\end{array}
\]
\end{proof}
\textbf{Kernel:} $\ker(j)=\{(x,y):y=2x\}$\\
\textbf{Image:} $\image(j)=\R$. 

\setcounter{enumi}{2}
\item (From the Problem Set) Given any ring $R$, the set of all two by two matrices with entries in $R$, denoted $M_2(R)$, forms a ring under matrix addition and matrix multiplication (when adding and multiplying these matrices together, you would be using the addition and multiplication in $R$). 

Suppose $R=\R[x]$, the polynomial ring in $x$ over $R$.  Write down the identity under addition in $M_2(R)$, the identity under multiplication in $M_2(R)$, and write down an element (which would be a two by two matrix) in $M_2(R)$ that has an inverse under multiplication, such that at least one of the entries is a non-constant polynomial in $x$. Such matrices are important in studying the stability of systems in control theory, a branch of systems engineering.

\begin{example*}
\[
0_{M_2(R)}= 
\left[ \begin{array}{cc}
f(x)=0 & f(x)=0 \\
f(x)=0 & f(x)=0 \\
\end{array} \right]
\]

\[
1_{M_2(R)}= 
\left[ \begin{array}{cc}
f(x)=1 & f(x)=0 \\
f(x)=0 & f(x)=1 \\
\end{array} \right]
\]

???????????????????????????????????????????

\[
A= 
\left[ \begin{array}{cc}
x & 2x \\
-x & 2x \\
\end{array} \right]
\]

\[
A^{-1}= 
\left[ \begin{array}{cc}
\frac{1}{2}x & -\frac{1}{2}	x \\
\frac{1}{4}x & \frac{1}{4}x \\
\end{array} \right]
\]
\end{example*}

\end{enumerate}

\end{document}