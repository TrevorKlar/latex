%% Useful packages
\usepackage{amssymb, amsmath, amsthm} 
%\usepackage{graphicx}  %%this is currently enabled in the default document, so it is commented out here. 
\usepackage{calrsfs}
\usepackage{suffix}%allows defining starred commands, i.e. \WithSuffix\newcommand\foo*{blahblah} NOTE: The {} around command name MUST be omitted for some reason.
\usepackage{braket}
\usepackage[normalem]{ulem}%The pack­age pro­vides an \ul (un­der­line) com­mand which will break over line ends. The [normalem] option leaves \emph and \em unchanged.
\usepackage[dvipsnames]{xcolor}%adds more color choices. See https://www.sharelatex.com/learn/File:ColoursEx6.png for a reference of colors. 
\usepackage{mathtools}
\usepackage{scalerel,stackengine}
\usepackage{array}
\usepackage{lipsum}
\usepackage{soul}% Use \st{blahblah} to strikeout text
\usepackage{tikz}
\usepackage{makeidx}
\usetikzlibrary{cd}
\usepackage{verbatim}
%\usepackage{ntheorem}% for theorem-like environments
\usepackage{mdframed}%can make highlighted boxes of text
%Use case: https://tex.stackexchange.com/questions/46828/how-to-highlight-important-parts-with-a-gray-background
\usepackage{wrapfig}
\usepackage{centernot}
\usepackage{subcaption}%\begin{subfigure}{0.5\textwidth}
\usepackage{pgfplots}
\usepackage{multicol}
\pgfplotsset{compat=1.13}
\usepackage[colorinlistoftodos]{todonotes}
\usepackage[colorlinks=true, allcolors=blue]{hyperref}
\usepackage{xfrac}					%to make slanted fractions \sfrac{numerator}{denominator}
\usepackage{enumitem}            
    %syntax: \begin{enumerate}[label=(\alph*)]
    %possible arguments: f \alph*, \Alph*, \arabic*, \roman* and \Roman*
\usetikzlibrary{arrows,shapes.geometric,fit}

\DeclareMathAlphabet{\pazocal}{OMS}{zplm}{m}{n}
%% Use \pazocal{letter} to typeset a letter in the other kind 
%%  of math calligraphic font. 

%% This puts the QED block at the end of each proof, the way I like it. 
\renewenvironment{proof}{{\bfseries Proof}}{\qed}
\makeatletter
\renewenvironment{proof}[1][\bfseries \proofname]{\par
  \pushQED{\qed}%
  \normalfont \topsep6\p@\@plus6\p@\relax
  \trivlist
  %\itemindent\normalparindent
  \item[\hskip\labelsep
        \scshape
    #1\@addpunct{}]\ignorespaces
}{%
  \popQED\endtrivlist\@endpefalse
}
\makeatother

%This makes a dynamically resizing \reallywidehat command. 
\stackMath
\newcommand\reallywidehat[1]{%
\savestack{\tmpbox}{\stretchto{%
  \scaleto{%
    \scalerel*[\widthof{\ensuremath{#1}}]{\kern.1pt\mathchar"0362\kern.1pt}%
    {\rule{0ex}{\textheight}}%WIDTH-LIMITED CIRCUMFLEX
  }{\textheight}% 
}{2.4ex}}%
\stackon[-6.9pt]{#1}{\tmpbox}%
}

%% This adds a \rewnewtheorem command, which enables me to override the settings for theorems contained in this document.
\makeatletter
\def\renewtheorem#1{%
  \expandafter\let\csname#1\endcsname\relax
  \expandafter\let\csname c@#1\endcsname\relax
  \gdef\renewtheorem@envname{#1}
  \renewtheorem@secpar
}
\def\renewtheorem@secpar{\@ifnextchar[{\renewtheorem@numberedlike}{\renewtheorem@nonumberedlike}}
\def\renewtheorem@numberedlike[#1]#2{\newtheorem{\renewtheorem@envname}[#1]{#2}}
\def\renewtheorem@nonumberedlike#1{  
\def\renewtheorem@caption{#1}
\edef\renewtheorem@nowithin{\noexpand\newtheorem{\renewtheorem@envname}{\renewtheorem@caption}}
\renewtheorem@thirdpar
}
\def\renewtheorem@thirdpar{\@ifnextchar[{\renewtheorem@within}{\renewtheorem@nowithin}}
\def\renewtheorem@within[#1]{\renewtheorem@nowithin[#1]}
\makeatother

%% This makes theorems and definitions with names show up in bold, the way I like it. 
\makeatletter
\def\th@plain{%
  \thm@notefont{}% same as heading font
  \itshape % body font
}
\def\th@definition{%
  \thm@notefont{}% same as heading font
  \normalfont % body font
}
\makeatother

%===============================================
%==============Shortcut Commands================
%===============================================
\newcommand{\ds}{\displaystyle}
\newcommand{\B}{\mathcal{B}}
\newcommand{\C}{\mathbb{C}}
\newcommand{\F}{\mathbb{F}}
\newcommand{\N}{\mathbb{N}}
\newcommand{\R}{\mathbb{R}}
\newcommand{\Q}{\mathbb{Q}}
\newcommand{\T}{\mathcal{T}}
\newcommand{\Z}{\mathbb{Z}}
\renewcommand\qedsymbol{$\blacksquare$}
\newcommand{\qedwhite}{\hfill\ensuremath{\square}}
\newcommand*\conj[1]{\overline{#1}}
\newcommand*\closure[1]{\overline{#1}}
\newcommand*\mean[1]{\overline{#1}}
%\newcommand{\inner}[1]{\left< #1 \right>}
\newcommand{\inner}[2]{\left< #1, #2 \right>}
\newcommand{\powerset}[1]{\pazocal{P}(#1)}
%% Use \pazocal{letter} to typeset a letter in the other kind 
%%  of math calligraphic font. 
\newcommand{\cardinality}[1]{\left| #1 \right|}
\newcommand{\domain}[1]{\mathcal{D}(#1)}
\newcommand{\image}{\text{Im}}
\newcommand{\inv}[1]{#1^{-1}}
\newcommand{\preimage}[2]{#1^{-1}\left(#2\right)}
\newcommand{\script}[1]{\mathcal{#1}}
\newcommand{\jpg}[2]{\begin{center}\includegraphics[#1]{#2}\end{center}}
\newcommand{\vecb}[1]{\mathbf{#1}}
\newcommand{\arbcup}[1]{\bigcup\limits_{\alpha\in\Gamma}#1_\alpha}
\newcommand{\arbcap}[1]{\bigcap\limits_{\alpha\in\Gamma}#1_\alpha}
\newcommand{\arbcoll}[1]{\{#1_\alpha\}_{\alpha\in\Gamma}}
\newcommand{\arbprod}[1]{\prod\limits_{\alpha\in\Gamma}#1_\alpha}
\newcommand{\finitecoll}[1]{#1_1, \ldots, #1_n}
\newcommand{\finitefuncts}[2]{#1(#2_1), \ldots, #1(#2_n)}
\newcommand{\abs}[1]{\left|#1\right|}
\newcommand{\norm}[1]{\left|\left|#1\right|\right|}
%\newcommand{\emphindex}{1}{\emph{#1}\index{#1}}
\newcommand{\vol}{\text{vol}}
\newcommand{\answer}{\textbf{Answer: }}
\newcommand{\proj}[2]{\text{proj}_{#1}(#2)}
\WithSuffix\newcommand\proj*[2]{\text{proj}_{#1}#2}
\newcommand{\lcm}{\text{lcm}}
\newcommand{\nlist}[1]{#1_1, \dots, #1_n}
\newcommand{\ncoll}[1]{\left\lbrace #1_1, \dots, #1_n \right\rbrace}
\newcommand{\ntuple}[1]{\left( #1_1, \dots, #1_n \right)}
\newcommand{\ideal}{\trianglelefteq}
\newcommand{\nsubgroup}{\trianglelefteq}
\newcommand{\dx}{\mathop{\kern0pt dx}\!{}}
\newcommand{\dy}{\mathop{\kern0pt dy}\!{}}
\newcommand{\dz}{\mathop{\kern0pt dz}\!{}}
\newcommand*{\der}{\mathop{\kern0pt d}\!{}}
\newcommand*{\del}{\mathop{\kern0pt \partial}\!{}}
%\newcommand*{\delf}{\mathop{\kern0pt \partial f}\!{}} 		%looks ugly with subscripts
%\newcommand*{\delg}{\mathop{\kern0pt \partial g}\!{}} 		%looks ugly with subscripts
%\newcommand*{\delh}{\mathop{\kern0pt \partial h}\!{}} 		%looks ugly with subscripts
%\newcommand{\del}{\partial}					%old implementation
%\newcommand*{\Der}{\mathop{\kern0pt D}\!{}}						%italic implementation
\makeatletter 																					%non-italic implementation
\newcommand*{\Der}{%
  \mathop{\kern\z@\operator@font D}\!{}%
}
\makeatother
\newcommand{\sign}{\text{sign}}
\newcommand{\curl}{\text{curl}\,}
\newcommand{\diverg}{\text{div}\,}
\newcommand{\grad}{\text{grad}\,}






\newenvironment{highlight}{\begin{mdframed}[backgroundcolor=gray!20]}{\end{mdframed}}

\DeclarePairedDelimiter\ceil{\lceil}{\rceil}
\DeclarePairedDelimiter\floor{\lfloor}{\rfloor}

%This adds thick lines for arrays and tabulars. use \thickhline for horizontal lines and i.e. {c!c|c|c|c} for column separators.
\makeatletter
\newcommand{\thickhline}{%
    \noalign {\ifnum 0=`}\fi \hrule height 1pt
    \futurelet \reserved@a \@xhline
}
%\newcolumntype{"}{@{\hskip\tabcolsep\vrule width 1pt\hskip\tabcolsep}}
\newcolumntype{[}{@{\vrule width 1pt\hspace{6pt}}} %for thick outside line
\newcolumntype{]}{@{\hspace{6pt}\vrule width 1pt}} %for thick outside line
\newcolumntype{!}{@{\hskip\tabcolsep\vrule width 1pt\hskip\tabcolsep}}
\makeatother

%===============================================
%===============My Tikz Commands================
%===============================================
\newcommand{\drawsquiggle}[1]{\draw[shift={(#1,0)}] (.005,.05) -- (-.005,.02) -- (.005,-.02) -- (-.005,-.05);}
\newcommand{\drawpoint}[2]{\draw[*-*] (#1,0.01) node[below, shift={(0,-.2)}] {#2};}
\newcommand{\drawopoint}[2]{\draw[o-o] (#1,0.01) node[below, shift={(0,-.2)}] {#2};}
\newcommand{\drawlpoint}[2]{\draw (#1,0.02) -- (#1,-0.02) node[below] {#2};}
\newcommand{\drawlbrack}[2]{\draw (#1+.01,0.02) --(#1,0.02) -- (#1,-0.02) -- (#1+.01,-0.02) node[below, shift={(-.01,0)}] {#2};}
\newcommand{\drawrbrack}[2]{\draw (#1-.01,0.02) --(#1,0.02) -- (#1,-0.02) -- (#1-.01,-0.02) node[below, shift={(+.01,0)}] {#2};}

%***********************************************
%**************Start of Document****************
%***********************************************
