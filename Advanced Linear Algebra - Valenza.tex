%\documentclass[letterpaper]{article}
\documentclass[a5paper]{article}

%% Language and font encodings
\usepackage[english]{babel}
\usepackage[utf8x]{inputenc}
\usepackage[T1]{fontenc}

%% Sets page size and margins
%\usepackage[letterpaper,top=1in,bottom=1in,left=1in,right=1in,marginparwidth=1.75cm]{geometry}
\usepackage[a5paper,top=1cm,bottom=1cm,left=1cm,right=1.5cm,marginparwidth=1.75cm]{geometry}

%% Useful packages
\usepackage{amssymb, amsmath, amsthm} 
%\usepackage{graphicx}  %%this is currently enabled in the default document, so it is commented out here. 
\usepackage{calrsfs}
\usepackage{braket}
\usepackage{mathtools}
\usepackage{lipsum}
\usepackage{tikz}
\usetikzlibrary{cd}
\usepackage{verbatim}
%\usepackage{ntheorem}% for theorem-like environments
\usepackage{mdframed}%can make highlighted boxes of text
%Use case: https://tex.stackexchange.com/questions/46828/how-to-highlight-important-parts-with-a-gray-background
\usepackage{wrapfig}
\usepackage{centernot}
\usepackage{subcaption}%\begin{subfigure}{0.5\textwidth}
\usepackage{pgfplots}
\pgfplotsset{compat=1.13}
\usepackage[colorinlistoftodos]{todonotes}
\usepackage[colorlinks=true, allcolors=blue]{hyperref}
\usepackage{xfrac}					%to make slanted fractions \sfrac{numerator}{denominator}
\usepackage{enumitem}            
    %syntax: \begin{enumerate}[label=(\alph*)]
    %possible arguments: f \alph*, \Alph*, \arabic*, \roman* and \Roman*
\usetikzlibrary{arrows,shapes.geometric,fit}

\DeclareMathAlphabet{\pazocal}{OMS}{zplm}{m}{n}
%% Use \pazocal{letter} to typeset a letter in the other kind 
%%  of math calligraphic font. 

%% This puts the QED block at the end of each proof, the way I like it. 
\renewenvironment{proof}{{\bfseries Proof}}{\qed}
\makeatletter
\renewenvironment{proof}[1][\bfseries \proofname]{\par
  \pushQED{\qed}%
  \normalfont \topsep6\p@\@plus6\p@\relax
  \trivlist
  %\itemindent\normalparindent
  \item[\hskip\labelsep
        \scshape
    #1\@addpunct{}]\ignorespaces
}{%
  \popQED\endtrivlist\@endpefalse
}
\makeatother

%% This adds a \rewnewtheorem command, which enables me to override the settings for theorems contained in this document.
\makeatletter
\def\renewtheorem#1{%
  \expandafter\let\csname#1\endcsname\relax
  \expandafter\let\csname c@#1\endcsname\relax
  \gdef\renewtheorem@envname{#1}
  \renewtheorem@secpar
}
\def\renewtheorem@secpar{\@ifnextchar[{\renewtheorem@numberedlike}{\renewtheorem@nonumberedlike}}
\def\renewtheorem@numberedlike[#1]#2{\newtheorem{\renewtheorem@envname}[#1]{#2}}
\def\renewtheorem@nonumberedlike#1{  
\def\renewtheorem@caption{#1}
\edef\renewtheorem@nowithin{\noexpand\newtheorem{\renewtheorem@envname}{\renewtheorem@caption}}
\renewtheorem@thirdpar
}
\def\renewtheorem@thirdpar{\@ifnextchar[{\renewtheorem@within}{\renewtheorem@nowithin}}
\def\renewtheorem@within[#1]{\renewtheorem@nowithin[#1]}
\makeatother

%% This makes theorems and definitions with names show up in bold, the way I like it. 
\makeatletter
\def\th@plain{%
  \thm@notefont{}% same as heading font
  \itshape % body font
}
\def\th@definition{%
  \thm@notefont{}% same as heading font
  \normalfont % body font
}
\makeatother

%===============================================
%==============Shortcut Commands================
%===============================================
\newcommand{\ds}{\displaystyle}
\newcommand{\B}{\mathcal{B}}
\newcommand{\C}{\mathbb{C}}
\newcommand{\F}{\mathbb{F}}
\newcommand{\N}{\mathbb{N}}
\newcommand{\R}{\mathbb{R}}
\newcommand{\Q}{\mathbb{Q}}
\newcommand{\T}{\mathcal{T}}
\newcommand{\Z}{\mathbb{Z}}
\renewcommand\qedsymbol{$\blacksquare$}
\newcommand{\qedwhite}{\hfill\ensuremath{\square}}
\newcommand*\conj[1]{\overline{#1}}
\newcommand*\closure[1]{\overline{#1}}
\newcommand*\mean[1]{\overline{#1}}
%\newcommand{\inner}[1]{\left< #1 \right>}
\newcommand{\inner}[2]{\left< #1, #2 \right>}
\newcommand{\powerset}[1]{\pazocal{P}(#1)}
%% Use \pazocal{letter} to typeset a letter in the other kind 
%%  of math calligraphic font. 
\newcommand{\cardinality}[1]{\left| #1 \right|}
\newcommand{\domain}[1]{\mathcal{D}(#1)}
\newcommand{\image}{\text{Im}}
\newcommand{\inv}[1]{#1^{-1}}
\newcommand{\preimage}[2]{#1^{-1}\left(#2\right)}
\newcommand{\script}[1]{\mathcal{#1}}


\newenvironment{highlight}{\begin{mdframed}[backgroundcolor=gray!20]}{\end{mdframed}}

\DeclarePairedDelimiter\ceil{\lceil}{\rceil}
\DeclarePairedDelimiter\floor{\lfloor}{\rfloor}

%===============================================
%===============My Tikz Commands================
%===============================================
\newcommand{\drawsquiggle}[1]{\draw[shift={(#1,0)}] (.005,.05) -- (-.005,.02) -- (.005,-.02) -- (-.005,-.05);}
\newcommand{\drawpoint}[2]{\draw[*-*] (#1,0.01) node[below, shift={(0,-.2)}] {#2};}
\newcommand{\drawopoint}[2]{\draw[o-o] (#1,0.01) node[below, shift={(0,-.2)}] {#2};}
\newcommand{\drawlpoint}[2]{\draw (#1,0.02) -- (#1,-0.02) node[below] {#2};}
\newcommand{\drawlbrack}[2]{\draw (#1+.01,0.02) --(#1,0.02) -- (#1,-0.02) -- (#1+.01,-0.02) node[below, shift={(-.01,0)}] {#2};}
\newcommand{\drawrbrack}[2]{\draw (#1-.01,0.02) --(#1,0.02) -- (#1,-0.02) -- (#1-.01,-0.02) node[below, shift={(+.01,0)}] {#2};}

%***********************************************
%**************Start of Document****************
%***********************************************

%===============================================
%===============Theorem Styles==================
%===============================================

%================Default Style==================
\theoremstyle{plain}% is the default. it sets the text in italic and adds extra space above and below the \newtheorems listed below it in the input. it is recommended for theorems, corollaries, lemmas, propositions, conjectures, criteria, and (possibly; depends on the subject area) algorithms.
\newtheorem{theorem}{Theorem}
\numberwithin{theorem}{section} %This sets the numbering system for theorems to number them down to the {argument} level. I have it set to number down to the {section} level right now.
\newtheorem*{theorem*}{Theorem} %Theorem with no numbering
\newtheorem{corollary}[theorem]{Corollary}
\newtheorem*{corollary*}{Corollary}
\newtheorem{conjecture}[theorem]{Conjecture}
\newtheorem{lemma}[theorem]{Lemma}
\newtheorem*{lemma*}{Lemma}
\newtheorem{proposition}[theorem]{Proposition}
\newtheorem*{proposition*}{Proposition}
\newtheorem{problemstatement}[theorem]{Problem Statement}


%==============Definition Style=================
\theoremstyle{definition}% adds extra space above and below, but sets the text in roman. it is recommended for definitions, conditions, problems, and examples; i've alse seen it used for exercises.
\newtheorem{definition}[theorem]{Definition}
\newtheorem*{definition*}{Definition}
\newtheorem{condition}[theorem]{Condition}
\newtheorem{problem}[theorem]{Problem}
\newtheorem{example}[theorem]{Example}
\newtheorem*{example*}{Example}
\newtheorem*{counterexample*}{Counterexample}
\newtheorem*{romantheorem*}{Theorem} %Theorem with no numbering
\newtheorem{exercise}{Exercise}
\numberwithin{exercise}{section}
\newtheorem{algorithm}[theorem]{Algorithm}

%================Remark Style===================
\theoremstyle{remark}% is set in roman, with no additional space above or below. it is recommended for remarks, notes, notation, claims, summaries, acknowledgments, cases, and conclusions.
\newtheorem{remark}[theorem]{Remark}
\newtheorem*{remark*}{Remark}
\newtheorem{notation}[theorem]{Notation}
\newtheorem*{notation*}{Notation}
%\newtheorem{claim}[theorem]{Claim}  %%use this if you ever want claims to be numbered
\newtheorem*{claim}{Claim}



\newcommand{\im}{\text{im }}

\title{Advanced Linear Algebra - Valenza, 2017}
\author{Trevor Klar}

\begin{document}
\maketitle

\section{Functions}

\begin{theorem*}\emph{\textbf{(1.4)}}
a function $f:S \to T$ is invertible iff it is bijective.
\end{theorem*}

\section{Groups and group homomorphisms}

For a  nonempty set $S$, a \emph{binary operation} on $S$ is a function
$$ S \times S \to S $$
$$ (s,t) \mapsto s\star t\text{ , where }s,t \in S$$

\noindent Basically, you take two numbers, and do something to them to get a third number, acccording to a rule. 

\begin{definition*}
We say that the binary operation $\star$ is \emph{associative} if:
$$(s\star t)\star u = s\star (t\star u)$$
For any $s,t,u \in S$.
\end{definition*}

\begin{definition*}
We say that the binary operation $\star$ is \emph{commutative} if:
$$s\star t= t\star s$$
For any $s,t \in S$.
\end{definition*}

\begin{definition*}
We say that an element $e \in S$ is an \emph{identity} for $\star$ if $e\star s = s = s\star e \quad \forall s \in S$. 
\end{definition*}

\clearpage

\begin{definition*}
A group $(G,\star)$ is a pair where $G$ is a nonempty set and $\star$ is a binary operator on $G$ such that 
\begin{enumerate}
\item $\star$ is associative (associative axiom). 
\item $\exists e \in G$ that is an identity under $\star$ (identity axiom).
\item $\forall s \in G$, $\exists t \in G$ such that $s\star t = e = t \star s$ (inverse axiom).
\end{enumerate}
\end{definition*}

\begin{definition*}
A group is called \emph{commutative} or \emph{abelian} if $$s\star t = t \star s \quad \forall s,t \in G.$$
\end{definition*}

\subsection{General Properties of Groups}

\begin{definition*}\textbf{(Cancellation Property)}

Suppose $(G,\star)$ is a group and $s,t,u \in G$. Then 
$$st = su \implies t = u$$
$$st = ut \implies s = u$$
(note: $st$ means $s\star t$.)
\end{definition*}

\begin{proposition*}
Suppose $(G,\star)$ is a group. Then,
\begin{enumerate}
\item The identity element $e$ in $G$ is unique. 
\begin{proof}$e = ee' = e'$, so $e= e'$\end{proof}
\item For any $s\in G$, the inverse of $s$ is unique. (And we denote it $s^{-1}$.)
\begin{proof}Suppose $t,u \in G$ such that $ts = e$ and $us = e$. Then $ts = us$, so $t = u$ by cancellation. \end{proof}
\item If $st = e$, then $s$ is the inverse of $t$ (and $t$ is the inverse of $s$). 
\begin{proof}
$$st=e$$
$$tst=te=t$$
$$tst=(ts)t$$
so, 
$$(ts)t=t$$
$$ts=e \text{, by cancellation.}$$
\end{proof}
\item $\forall s \in G, (s^{-1})^{-1}=s$. 
\item $\forall s,t \in G, (st)^{-1}=t^{-1}s^{-1}$
\begin{proof}
$$(st)^{-1}(st)=e$$
$$(st)^{-1}(st)t^{-1}=et^{-1}$$
$$(st)^{-1}(ss^{-1}=t^{-1}s^{-1}$$
$$(st)^{-1}=t^{-1}s^{-1}$$
\end{proof}
\item If $s \in G$, then $ss=s \iff s=e$. 
\end{enumerate}
\end{proposition*}

\begin{definition*}
Suppose $(G,\star)$ is a group, and $H$ is a subset of $G$. We say $H$ is a \emph{subgroup} of $G$ if $(H,\star)$ is a group.
\end{definition*}

This means:
\begin{itemize}
\item $\star$ is a binary operator on $H$, that is, $H$ is closed under $\star$
\item $\star$ is associative for elements in $H$. (Clearly, since this also hold for all of $G$)
\item There is an identity $e'$ in $H$ such that $e'h = h = he'$ for any $h \in H$.
\item Every element $s \in H$ has an inverse in $H$, i.e. there should be an element $t \in H$ such that $s \star t = e = t \star s$. 
\begin{remark}
$t$ is the same as the inverse of $s$ taken in $G$. (We leave the proof as an excercise.)
\end{remark}
\end{itemize}

\begin{proposition*}\textbf{(Subgroup criterion)}
Suppse $(G, \star)$ is a group, and $H$ is a \emph{nonempty} subset of $G$. Then
\begin{center}H is a subgroup of $G \iff$ for any $s,t \in H, \quad s \star t^{-1} \in H$. \end{center}
\begin{example*}
Consider the group $(\Z, +). \quad$ For any $n \in \Z^+$,

$n\Z = \{nz:z \in \Z\} = \{\text{all integer multiples of }n\}.$

$1n \in n\Z,$ so $n\Z \neq \emptyset$. 

Now, apply the subgroup criterion: 

Take any two elements $s,t \in \Z$. 

then $s = na$ and $t=nb$, where $a,b \in \Z$

so $s + (-t) = na - nb = n(a-b) \in n\Z$. 

Therefore, $n\Z$ is a subgroup of $\Z$.
\end{example*}
\end{proposition*}

\begin{exercise}
Prove that $I' := \{f\in \mathcal{C}^0(\R):f(0)=1\}$ is \emph{not} a subgroup of $\mathcal{C}^0(\R)$.
\end{exercise}

\subsection{Group homomorphisms}

\begin{definition*}
Suppose $(G_1, \star_1), (G_2, \star_2)$ are groups. A funcion $f:G_1 \to G_2$ is called a \emph{group homomorphism} if:

$$\forall s,t \in G_1, \quad f(s \star_1 t)=f(s)\star_2 f(t)$$
\end{definition*}

\begin{example*}
Consider the group $(\Z, +)$. The function $f:\Z \to \Z$ where $n \mapsto 3n$ is a group homomorphism. 
\begin{proof}
Take any $s,t \in \Z$. We want $f(s+t) = f(s)+f(t)$. 
$$f(s+t) = 3(s+t) = 3s+3t = f(s)+f(t)$$
This completes the proof. \end{proof}
\end{example*}

Properties of group homomorphisms:
\begin{proposition*}
Suppose $f:G_1 \to G_2$ is a group homomorphism. Then, 
\begin{enumerate}[label=(\roman*)]
\item $f(e_1)=e_2$
\begin{proof}
$f(e_1) = f(e_1e_1) = f(e_1)f(e_1).$\\
Then, by cancellation, $e_2=f(e_1)$. 
\end{proof}
\item For any $s \in G$, $\quad f(s^{-1})=(f(s))^{-1}$
\begin{proof}
We need to prove that $f(s^{-1})$ is the inverse of $f(s)$. It suffices to prove that $f(s^{-1})f(s)=e_2$. 

$f(s^{-1})f(s)=f(s^{-1}s)=f(e_1)=e_2$. 
\end{proof}
\end{enumerate}
\end{proposition*}

\begin{definition*}
If $\phi:H \to G$ is a bijective function from the group $H$ to the group $G$, then we say it is a \emph{group isomorphism} and write $G \cong H$. 
\end{definition*}

\begin{lemma}
If $\phi: G \to H$ is a group isomorphism, then $\phi^{-1}:H \to G$  is also a group isomorphism. 
\end{lemma}

\begin{proposition*}
Given group homomorphisms $\phi: G \to H$, $\psi:  H \to I$, the composition $\psi \phi : G \to I$ is also a group homomorphism. 
\end{proposition*}

\begin{corollary}
If $\psi, \phi$ above are both isomorphisms, then $\psi \phi$ is also a group isomorphism. 
\end{corollary}

\begin{definition*} Suppose we have a function $f:S \to T$.\\
\begin{itemize}
\item For any $t \in T$, the \emph{inverse image} (or the \emph{preimage}) of $t$, denoted $f^{-1}(t)$, is the set 
$$f^{-1}(t) \equiv	\{x \in S: f(x) = t\}$$
\item For any subset $W \subset T$, the \emph{inverse image} (or the \emph{preimage}) of $t$, denoted $f^{-1}(W)$, is the set 
$$f^{-1}(W) \equiv	\{x \in S: f(x) \in W\}$$
\end{itemize}
\end{definition*}

\begin{definition*}
Given a group homomorphism $\phi:G \to H$, 
\begin{itemize}
\item the \emph{kernel} o f$\phi$ is 
$$\ker{\phi} := \{x \in G: \phi(x) = e_H\} = \phi^{-1}(e_H)$$
\item the \emph{image} of $\phi$ is 
$$\im{\phi} := \{\phi(x):x \in G\}$$
\end{itemize}
\end{definition*}

\begin{proposition*}
For a group homomorphism $\phi:G \to H$,
\begin{center}
$\ker{\phi}$ is a subgroup of $G$, \\
$\im{\phi}$ is a subgroup of $H$.
\end{center}
\end{proposition*}

\begin{lemma}
For a group homomorphism $\phi:G \to H$, then 
$$\phi \text{ is injective} \iff \ker{\phi} = \{e_G\}$$
\end{lemma}

\begin{definition*}
Let $G_0, G_1$ be groups. The \emph{direct product} of $G_0$ and $G_1$ is the set 
$$G_0\times G_1 = \{(s_0,s_1):s_0 \in G_0, s_1 \in G_1\}$$
equipped with an operation on $G_0\times G_1$ as follows:
$$(s_0,s_1)(t_0,t_1) = (s_0t_0,s_1t_1) \quad \forall s_0,t_0\in G_0, s_1,t_1 \in G_1$$

This is just the Cartesian product of the two sets $G_0$ and $G_1$, equipped with the same operations, applied componentwise.
\end{definition*}

\begin{definition*}
Let $G_0, G_1$ be groups. A \emph{projection map} is a function 
\[
\begin{array}{rcl}
\rho_0:G_0\times G_1 &\to& G_0\\
(s_0,s_1) &\mapsto& s_0\\
\end{array}
\]
\end{definition*}

\begin{definition*}
Consider the special case of the direct product $G\times G$ of a group $G$ with itself. Define a subset $D$ of $G\times G$ by 
$$D=\{(s,s):s\in G\}$$
That is, $D$ consists of all elements with both coordinates equal. This is called the \emph{diagonal subgroup}.
\end{definition*}

\subsection{Rings and Fields}

\begin{definition*}
A \emph{ring} is a triple $(A, +_A, \bullet_A)$, where $A$ is a nonempty set, $+_A$ is some 'addition' operation, and $\bullet$ is some 'multiplication' operation such that:
\begin{itemize}
\item $(A, +_A)$ is an abelian group. (We use additive notation for the inverse and identity of this operation)
\item $(A, \bullet_A)$ is a "monoid", that is, $\bullet_A$ has the associative and identity properties, but not necessarily the inverse property or the commutative property. 
\item $\bullet_A$ distributes over $+_A$ from the right and the left (distributive property).
\end{itemize}
\end{definition*}

\begin{definition*}
If $\bullet_A$ is also commutative, then we say $A$ is a \emph{commutative ring}. We often write $ab$ to denote $a\bullet_A b$.
\end{definition*}

If $k$ is a commutative ring, $k* := k-\{0_k\}$.

\begin{definition*}
A commutative ring $k$ where $(k*, \bullet_k)$ is a group is called a field.  (That is, it is a ring where $\bullet$ has commutativity and an inverse)
\end{definition*}

\begin{proposition*}
Suppose $(A, +, \bullet)$ is a ring. Then, $\forall a,b \in A,$
\begin{enumerate}
\item $0a = 0 = a0$
\item a(-b)=-(ab)=(-a)b
\item (-a)(-b)=ab
\item (-1)a=-a
\item (-1)(-1)=1
\end{enumerate}
\end{proposition*}

\pagebreak

\section{Vector Spaces and Linear Transformations}

\subsection{Vector Spaces and Subspaces}

Fix a field $k$ (e.g. $\R, \C, \Q etc.)$

\begin{definition*}
A \emph{vector space over $k$} (or a \emph{$k$-vector space}) is a set $V$, together with a binary opeation $+$ on $V$, and a \emph{scalar multiplication}. 
\end{definition*}

Vector fields have the following properties:\\
$\forall \lambda, \mu \in k, \forall v,w \in V$,

\begin{enumerate}[label=(\roman*)]
\item $(V,+)$ is an abelian group. 
\item $(\lambda\mu)\vec{v}=\lambda(\mu)\vec{v}$\\
That is, scalar multiplication is associative. 
\item $(\lambda+\mu)\vec{v}=\lambda \vec{v}+ \mu \vec{v}$\\
That is, vectors distribute over scalars.  
\item $\lambda(\vec{v}+\vec{w})=\lambda \vec{v}+ \lambda \vec{w}$\\
That is, scalars distribute over vectors. 
\item $1_k \vec{v} = \vec{v}$\\
That is, the identity of the field is also the identity of the vector space. 
\end{enumerate}

\begin{proposition}
Let $V$ be a vector space over a field $k$. Then the following assertions hold:
\begin{enumerate}[label=(\roman*)]
\item $\lambda\vec{0}=\vec{0} \quad \forall \lambda \in k$
\item $0\vec{v}=\vec{0} \quad \forall \vec{v} \in V$
\item $(-\lambda)\vec{v}=-(\lambda\vec{v}) \quad \forall \lambda \in k, \vec{v} \in V$
\item $\lambda\vec{v}=\vec{0} \iff (\lambda=0 \text{ or } v=\vec{0}) \quad \forall \lambda \in k, \vec{v} \in V$
\end{enumerate}
\end{proposition}

\begin{definition*}
A subset $W$ of a vector space $V$ over a field $k$ is called a \emph{subspace of $V$} if it constitutes a vector space over $k$ in its own right with respect to the additive and scalar operations defined on $V$. 
\end{definition*}

\begin{proposition} \textbf{(Subspace Criterion)} Let $W$ be a \underline{nonempty} subset of the vector space $V$. Then $W$ is a subspace of $V$ if and only if it is closed under addition and scalar multiplication. 
\end{proposition}

\begin{definition*}
Let $v_1 \ldots, v_n$ be a family of vectors in the vector space $V$ defined over a field $k$. Then an expression of the form 
$$\lambda_1\vec{v_1} + \lambda_2\vec{v_2} + \ldots + \lambda_n\vec{v_n} \quad (\lambda_1, \lambda_2, \ldots \lambda_n \in k)$$
is called a \emph{linear combination} of the vectors $v_1 \ldots, v_n$. The set of all such linear
combinations is called the \emph{span} of $v_1 \ldots, v_n$ and denoted Span($v_1 \ldots, v_n$). 
\end{definition*}

\begin{proposition}
Let $v_1 \ldots, v_n$ be a family of vectors in the vector space $V$ defined over a field $k$. Then $W = $Span($v_1 \ldots, v_n$) is a subspace of $V$. 
\end{proposition}

\subsection{Linear Transformations}

\begin{definition*}
Let $V$ and $V'$ be vector space over a common field $k$. Then a function $V\to V'$ is called a \emph{linear transformation} if it satisfies the following conditions:

\begin{enumerate}[label=(\roman*)]
\item $T(v+w)=T(v)+T(w) \quad \forall v,w \in V$
\item $T(\lambda v)=\lambda T(v) \quad \forall v \in V, \lambda \in k$
\end{enumerate}

One also says that $T$ is \emph{$k$-linear} or a \emph{vector space homomorphism}. 
\end{definition*}

Note that the first condition states that T is a homomorphism of additive groups, and therefore all of our previous theory of group homomorphisms applies. In particular, we have the following derived properties: 

\begin{enumerate}[label=(\roman*)]
\setcounter{enumi}{2}
\item $T(\vec{0})=\vec{0}$
\item $T(-\vec{v})=-T(\vec{v})\quad \forall v \in V$
\end{enumerate}

\begin{proposition}
The composition of linear transformations is a linear transformation.
\end{proposition}

\begin{proposition}
The kernel and image of a linear transformation are subspaces of their ambient vector spaces. 
\end{proposition}

\begin{definition*}
A bijective linear transformation $T: V \to V'$ is called an \emph{isomorphism} of vector spaces. 
\end{definition*}

\section{•}


\begin{theorem}
In any vector space, 
\begin{itemize}
	\item Every linearly independent set of vectors can be extended to a basis. 
	\item Every spanning set can be contracted to a basis. 
	\item Every vector space has a basis
\end{itemize}
\end{theorem}

\begin{corollary}
Suppose $V$ is a finite-dimensional $k$-vector space with $\dim(V)=n$. Then, 
\begin{itemize}
	\item No subset of $V$ with more than $n$ vectors can be linearly independent. 
	\item No subset of $V$ with less than $n$ vectors can span $V$. 
	\begin{proof}(i)
	Suppose $\B$ is a collection of $\ell$ vectors in $V$, and suppose $\ell>n$. Suppose also that $\B$ is linearly independent. By part (i) of the Thm, $\B$ can be extended to a basis $\B'$ for $V$. 
	$$\ell = |\B| \leq |\B'|=n $$
	which is a contradiction. 
	\end{proof}	 
\end{itemize}
\end{corollary}

\begin{corollary}
Suppose $V$ has dimension $n$ and $S$ is a collection of $n$ vectors in $V$. The following are equivalent:
\begin{itemize}
	\item $S$ is linearly independent.
	\item $S$ spans $V$.
	\item $S$ is a basis for $V$.
\end{itemize}
\end{corollary}



\end{document}