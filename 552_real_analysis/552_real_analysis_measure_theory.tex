%\documentclass[letterpaper]{article}
\documentclass[a5paper, oneside]{book}

%% Language and font encodings
\usepackage[english]{babel}
\usepackage[utf8x]{inputenc}
\usepackage[T1]{fontenc}

%% Sets page size and margins
%\usepackage[letterpaper,top=1in,bottom=1in,left=1in,right=1in,marginparwidth=1.75cm]{geometry}
\usepackage[a5paper,top=1cm,bottom=1cm,left=1cm,right=1.5cm,marginparwidth=1.75cm]{geometry}

\usepackage{graphicx}
%\graphicspath{../images}	  %%where to look for images

%% Useful packages
\usepackage{amssymb, amsmath, amsthm} 
%\usepackage{graphicx}  %%this is currently enabled in the default document, so it is commented out here. 
\usepackage{calrsfs}
\usepackage{braket}
\usepackage{mathtools}
\usepackage{lipsum}
\usepackage{tikz}
\usetikzlibrary{cd}
\usepackage{verbatim}
%\usepackage{ntheorem}% for theorem-like environments
\usepackage{mdframed}%can make highlighted boxes of text
%Use case: https://tex.stackexchange.com/questions/46828/how-to-highlight-important-parts-with-a-gray-background
\usepackage{wrapfig}
\usepackage{centernot}
\usepackage{subcaption}%\begin{subfigure}{0.5\textwidth}
\usepackage{pgfplots}
\pgfplotsset{compat=1.13}
\usepackage[colorinlistoftodos]{todonotes}
\usepackage[colorlinks=true, allcolors=blue]{hyperref}
\usepackage{xfrac}					%to make slanted fractions \sfrac{numerator}{denominator}
\usepackage{enumitem}            
    %syntax: \begin{enumerate}[label=(\alph*)]
    %possible arguments: f \alph*, \Alph*, \arabic*, \roman* and \Roman*
\usetikzlibrary{arrows,shapes.geometric,fit}

\DeclareMathAlphabet{\pazocal}{OMS}{zplm}{m}{n}
%% Use \pazocal{letter} to typeset a letter in the other kind 
%%  of math calligraphic font. 

%% This puts the QED block at the end of each proof, the way I like it. 
\renewenvironment{proof}{{\bfseries Proof}}{\qed}
\makeatletter
\renewenvironment{proof}[1][\bfseries \proofname]{\par
  \pushQED{\qed}%
  \normalfont \topsep6\p@\@plus6\p@\relax
  \trivlist
  %\itemindent\normalparindent
  \item[\hskip\labelsep
        \scshape
    #1\@addpunct{}]\ignorespaces
}{%
  \popQED\endtrivlist\@endpefalse
}
\makeatother

%% This adds a \rewnewtheorem command, which enables me to override the settings for theorems contained in this document.
\makeatletter
\def\renewtheorem#1{%
  \expandafter\let\csname#1\endcsname\relax
  \expandafter\let\csname c@#1\endcsname\relax
  \gdef\renewtheorem@envname{#1}
  \renewtheorem@secpar
}
\def\renewtheorem@secpar{\@ifnextchar[{\renewtheorem@numberedlike}{\renewtheorem@nonumberedlike}}
\def\renewtheorem@numberedlike[#1]#2{\newtheorem{\renewtheorem@envname}[#1]{#2}}
\def\renewtheorem@nonumberedlike#1{  
\def\renewtheorem@caption{#1}
\edef\renewtheorem@nowithin{\noexpand\newtheorem{\renewtheorem@envname}{\renewtheorem@caption}}
\renewtheorem@thirdpar
}
\def\renewtheorem@thirdpar{\@ifnextchar[{\renewtheorem@within}{\renewtheorem@nowithin}}
\def\renewtheorem@within[#1]{\renewtheorem@nowithin[#1]}
\makeatother

%% This makes theorems and definitions with names show up in bold, the way I like it. 
\makeatletter
\def\th@plain{%
  \thm@notefont{}% same as heading font
  \itshape % body font
}
\def\th@definition{%
  \thm@notefont{}% same as heading font
  \normalfont % body font
}
\makeatother

%===============================================
%==============Shortcut Commands================
%===============================================
\newcommand{\ds}{\displaystyle}
\newcommand{\B}{\mathcal{B}}
\newcommand{\C}{\mathbb{C}}
\newcommand{\F}{\mathbb{F}}
\newcommand{\N}{\mathbb{N}}
\newcommand{\R}{\mathbb{R}}
\newcommand{\Q}{\mathbb{Q}}
\newcommand{\T}{\mathcal{T}}
\newcommand{\Z}{\mathbb{Z}}
\renewcommand\qedsymbol{$\blacksquare$}
\newcommand{\qedwhite}{\hfill\ensuremath{\square}}
\newcommand*\conj[1]{\overline{#1}}
\newcommand*\closure[1]{\overline{#1}}
\newcommand*\mean[1]{\overline{#1}}
%\newcommand{\inner}[1]{\left< #1 \right>}
\newcommand{\inner}[2]{\left< #1, #2 \right>}
\newcommand{\powerset}[1]{\pazocal{P}(#1)}
%% Use \pazocal{letter} to typeset a letter in the other kind 
%%  of math calligraphic font. 
\newcommand{\cardinality}[1]{\left| #1 \right|}
\newcommand{\domain}[1]{\mathcal{D}(#1)}
\newcommand{\image}{\text{Im}}
\newcommand{\inv}[1]{#1^{-1}}
\newcommand{\preimage}[2]{#1^{-1}\left(#2\right)}
\newcommand{\script}[1]{\mathcal{#1}}


\newenvironment{highlight}{\begin{mdframed}[backgroundcolor=gray!20]}{\end{mdframed}}

\DeclarePairedDelimiter\ceil{\lceil}{\rceil}
\DeclarePairedDelimiter\floor{\lfloor}{\rfloor}

%===============================================
%===============My Tikz Commands================
%===============================================
\newcommand{\drawsquiggle}[1]{\draw[shift={(#1,0)}] (.005,.05) -- (-.005,.02) -- (.005,-.02) -- (-.005,-.05);}
\newcommand{\drawpoint}[2]{\draw[*-*] (#1,0.01) node[below, shift={(0,-.2)}] {#2};}
\newcommand{\drawopoint}[2]{\draw[o-o] (#1,0.01) node[below, shift={(0,-.2)}] {#2};}
\newcommand{\drawlpoint}[2]{\draw (#1,0.02) -- (#1,-0.02) node[below] {#2};}
\newcommand{\drawlbrack}[2]{\draw (#1+.01,0.02) --(#1,0.02) -- (#1,-0.02) -- (#1+.01,-0.02) node[below, shift={(-.01,0)}] {#2};}
\newcommand{\drawrbrack}[2]{\draw (#1-.01,0.02) --(#1,0.02) -- (#1,-0.02) -- (#1-.01,-0.02) node[below, shift={(+.01,0)}] {#2};}

%***********************************************
%**************Start of Document****************
%***********************************************

%===============================================
%===============Theorem Styles==================
%===============================================

%================Default Style==================
\theoremstyle{plain}% is the default. it sets the text in italic and adds extra space above and below the \newtheorems listed below it in the input. it is recommended for theorems, corollaries, lemmas, propositions, conjectures, criteria, and (possibly; depends on the subject area) algorithms.
\newtheorem{theorem}{Theorem}
\numberwithin{theorem}{section} %This sets the numbering system for theorems to number them down to the {argument} level. I have it set to number down to the {section} level right now.
\newtheorem*{theorem*}{Theorem} %Theorem with no numbering
\newtheorem{corollary}[theorem]{Corollary}
\newtheorem*{corollary*}{Corollary}
\newtheorem{conjecture}[theorem]{Conjecture}
\newtheorem{lemma}[theorem]{Lemma}
\newtheorem*{lemma*}{Lemma}
\newtheorem{proposition}[theorem]{Proposition}
\newtheorem*{proposition*}{Proposition}
\newtheorem{problemstatement}[theorem]{Problem Statement}


%==============Definition Style=================
\theoremstyle{definition}% adds extra space above and below, but sets the text in roman. it is recommended for definitions, conditions, problems, and examples; i've alse seen it used for exercises.
\newtheorem{definition}[theorem]{Definition}
\newtheorem*{definition*}{Definition}
\newtheorem{condition}[theorem]{Condition}
\newtheorem{problem}[theorem]{Problem}
\newtheorem{example}[theorem]{Example}
\newtheorem*{example*}{Example}
\newtheorem*{counterexample*}{Counterexample}
\newtheorem*{romantheorem*}{Theorem} %Theorem with no numbering
\newtheorem{exercise}{Exercise}
\numberwithin{exercise}{section}
\newtheorem{algorithm}[theorem]{Algorithm}

%================Remark Style===================
\theoremstyle{remark}% is set in roman, with no additional space above or below. it is recommended for remarks, notes, notation, claims, summaries, acknowledgments, cases, and conclusions.
\newtheorem{remark}[theorem]{Remark}
\newtheorem*{remark*}{Remark}
\newtheorem{notation}[theorem]{Notation}
\newtheorem*{notation*}{Notation}
%\newtheorem{claim}[theorem]{Claim}  %%use this if you ever want claims to be numbered
\newtheorem*{claim}{Claim}

 %located in ~/texmf/tex/latex/

\numberwithin{theorem}{chapter}
\renewcommand{\script}[1]{\pazocal{#1}}
\newcommand{\extmeasure}[1]{m_*(#1)}
\newcommand{\measure}[1]{m(#1)}

\pgfplotsset{compat=1.13}

%title

\title{Real Analysis - Horn, 2019}
\author{Trevor Klar}
\makeindex

\begin{document}
\frontmatter
\maketitle
\tableofcontents

\mainmatter
\chapter{Measure Theory}
\section{Preliminaries}

\begin{highlight}
\begin{definition*}
We say $\alpha=\inf S$ iff:
\begin{itemize}
	\item $\alpha\leq s$ for all $s\in S$ and 
	\item for any $\epsilon>0$, there exists $ s\in S$ such that $s<\alpha+\epsilon$. 
\end{itemize}
\end{definition*}
\end{highlight}

\begin{definition*}
A (closed) \textbf{rectangle} $R$ in $\R^d$ is given by the product of $d$ one-dimensional closed and bounded intervals
$$R = [a_1 , b_1 ] \times [a_2 , b_2 ] \times \cdots \times [a_d , b_d ],$$
where $a_j\leq b_j$ are real numbers. 
\end{definition*}

\begin{highlight}
\begin{definition*}
The \textbf{measure} of a rectangle $R$ is defined to be 
$$\abs{R}=\prod_{i=1}^d(b_i-a_i)=(b_1-a_1)(b_2-a_2)\cdots(b_d-a_d).$$
\end{definition*}
\end{highlight}

\begin{definition*}
A union of rectangles is said to be \textbf{almost disjoint} if the interiors of the rectangles are disjoint. (We pretty much only use closed rectangles when we say they are almost disjoint). 
\end{definition*}

\begin{lemma}
Let $R$ be a rectangle which is the almost disjoint union of finitely many other rectangles, that is, $R=\bigcup\limits_{k=1}^N R_k$. Then, 
$$\abs{R}=\sum_{k=1}^N \abs{R_k}.$$
\end{lemma}

\begin{lemma}
Let $R, R_1, \dots , R_N$ be rectangles, with $R\subseteq	\bigcup\limits_{k=1}^N R_k$. Then,
$$\abs{R}\leq\sum_{k=1}^N \abs{R_k}.$$
\end{lemma}

\begin{theorem}
Every open subset $O\subseteq\R^1$ can be written uniquely as a countable union of disjoint open intervals.\footnote{I have deliberately changed the notation slightly here. I will continue to use script letters i.e. $\script{ABCQO}$ to denote collections of sets, and ordinary capitals i.e. $ABCQO$ to denote sets.}
\end{theorem}

\begin{theorem}
Every open subset $O\subseteq\R^d$ can be written as a countable union of almost disjoint closed cubes.
\end{theorem}
\begin{proof}
Basically, do this: 
\jpg{width=0.5\textwidth}{thm_1_4}
\end{proof}
\pagebreak
\section{The exterior measure}

\begin{highlight}
\begin{definition*}
Let $E\subseteq \R^d$. Let $\script{Q}=\{Q_j\}_1^\infty$ denote a countable collection of \emph{closed} cubes which cover $E$, and let $\Gamma$ denote the collection of all possible countable covers of $E$. That is, for all $\script{Q}\in\Gamma$, $E\subset\bigcup_1^\infty Q_j$ where each $Q_j\in \script{Q}$. \\
\\
The \textbf{exterior measure} of $E$ is defined as 
$$m_*(E)=\inf_{\script{Q}\in\Gamma} \sum_{j=1}^\infty \abs{Q_j}$$
\end{definition*}
\end{highlight}

\begin{remark*}
Since the exterior measure is defined with an infimum, then $m_*(E)\leq \sum\abs{Q_i}$ for any cover $\{Q_i\}$ of $E$.
\end{remark*}
\begin{remark*}
Note that $\abs{\,\cdot\,}$ is only defined for rectangles. For any other sets, we have only the exterior measure, $m_*$. 
\end{remark*}

\begin{highlight}
\textbf{Observations about exterior measure.} 
\begin{enumerate}[label=\textbf{\arabic*.}]
	\item \textbf{(Monotonicity)} \\
	If $E_1\subseteq E_2$, then $m_*(E_1)\subseteq m_*(E_1)$.
	\item \textbf{(Countable sub-additivity)} \\
	If $E=\bigcup\limits_{j=1}^\infty E_j$, then $\extmeasure{E}\leq\sum\limits_{j=1}^\infty \extmeasure{E_j}$. 
	\item Let $E\subseteq\R^d$, and let $\script{O}=\{\text{open sets }O:E\subseteq O\}$. \\Then $\extmeasure{E}=\inf\limits_{O\in\script{O}}\extmeasure{O}$.
	\item[] (\textbf{Corollary to 3.)} If $\extmeasure{E}<\infty$, then $\exists$ open $O\supset E$ such that $\extmeasure{O}<\extmeasure{E}+\epsilon$ for any $\epsilon>0$. 
	\item If $E=E_1\cup E_2$, and $d(E_1, E_2)>0$, \\
	then $\extmeasure{E}=\extmeasure{E_1}+\extmeasure{E_2}$. 
	\item If a set $E$ is the countable union of almost disjoint  closed cubes \\$E=\bigcup\limits_{j=1}^\infty Q_j$, then $\extmeasure{E}=\sum_{j=1}^\infty\abs{Q_j}$. 
	
\end{enumerate}
\end{highlight}

\pagebreak
\section{Measurable sets and the Lebesgue measure}

\begin{highlight}
\begin{definition*}
We say that $E \subseteq R^d$ is \textbf{measurable} if 
$$\text{for any } \epsilon > 0 \text{, there exists an open set } O \supseteq E \text{ with } m_*(O-E)<\epsilon.$$
%$$\forall \epsilon > 0, \quad \exists (\text{open } O \supset E \,|\, m_*(O-E)<\epsilon).$$
\end{definition*}
\end{highlight}
If the distinction is important, we can be more specific and say the set is \textbf{Lebesgue measurable}.
\begin{highlight}
\begin{definition*}
We define the Lebesgue \textbf{measure} of a measurable set $E$ by its exterior measure, 
$$m(E)=m_*(E).$$
\end{definition*}
\end{highlight}

Now we give some propositions about the Lebesgue Measure. 

\begin{highlight}
\textbf{Property 1.} Every open set in $R^d$ is measurable.
\end{highlight}
\begin{proof}
Let $E\subseteq\R^d$ be open. Then $E$ is an open set containing $E$ where $m_*(E-E)=0<\epsilon$ for any $\epsilon>0$, so $E$ is measurable. 
\end{proof}

\begin{highlight}
\textbf{Property 2.} If $m_{∗}(E)=0$, then $E$ is measurable. In particular, if $F\subseteq E$ and $m_∗(E) = 0$, then F is measurable.
\end{highlight}
\begin{proof}
Let $F\subset E\subseteq\R^d$ with $m_{∗}(E)=0$, and let $\epsilon>0$ be given. By Observation 3 about exterior measure, there exists on open set $O\supset E$ such that $m_*(O)<\epsilon$. Thus by monotonicity, $m_*(O-E)<m_*(O-F)<m_*(O)<\epsilon$ and we are done. 
\end{proof}

\begin{highlight}
\textbf{Property 3.} A countable union of measurable sets is measurable.
\end{highlight}
\begin{proof}
\textbf{(idea)} Choose open sets so that each one is $\frac{1}{\epsilon^j}$, and they all sum to $<\epsilon$. 
\end{proof}

\begin{highlight}
\textbf{Property 4.} Closed sets are measurable. 
\end{highlight}







\end{document}
