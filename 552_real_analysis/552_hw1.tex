\documentclass[letterpaper]{article}
%\documentclass[a5paper]{article}

%% Language and font encodings
\usepackage[english]{babel}
\usepackage[utf8x]{inputenc}
\usepackage[T1]{fontenc}

%% Sets page size and margins
\usepackage[letterpaper,top=.75in,bottom=1in,left=1in,right=1in,marginparwidth=1.75cm]{geometry}
%\usepackage[a5paper,top=1cm,bottom=1cm,left=1cm,right=1.5cm,marginparwidth=1.75cm]{geometry}

\usepackage{graphicx}
%\graphicspath{../images}	  %%where to look for images

%% Useful packages
\usepackage{amssymb, amsmath, amsthm} 
%\usepackage{graphicx}  %%this is currently enabled in the default document, so it is commented out here. 
\usepackage{calrsfs}
\usepackage{braket}
\usepackage{mathtools}
\usepackage{lipsum}
\usepackage{tikz}
\usetikzlibrary{cd}
\usepackage{verbatim}
%\usepackage{ntheorem}% for theorem-like environments
\usepackage{mdframed}%can make highlighted boxes of text
%Use case: https://tex.stackexchange.com/questions/46828/how-to-highlight-important-parts-with-a-gray-background
\usepackage{wrapfig}
\usepackage{centernot}
\usepackage{subcaption}%\begin{subfigure}{0.5\textwidth}
\usepackage{pgfplots}
\pgfplotsset{compat=1.13}
\usepackage[colorinlistoftodos]{todonotes}
\usepackage[colorlinks=true, allcolors=blue]{hyperref}
\usepackage{xfrac}					%to make slanted fractions \sfrac{numerator}{denominator}
\usepackage{enumitem}            
    %syntax: \begin{enumerate}[label=(\alph*)]
    %possible arguments: f \alph*, \Alph*, \arabic*, \roman* and \Roman*
\usetikzlibrary{arrows,shapes.geometric,fit}

\DeclareMathAlphabet{\pazocal}{OMS}{zplm}{m}{n}
%% Use \pazocal{letter} to typeset a letter in the other kind 
%%  of math calligraphic font. 

%% This puts the QED block at the end of each proof, the way I like it. 
\renewenvironment{proof}{{\bfseries Proof}}{\qed}
\makeatletter
\renewenvironment{proof}[1][\bfseries \proofname]{\par
  \pushQED{\qed}%
  \normalfont \topsep6\p@\@plus6\p@\relax
  \trivlist
  %\itemindent\normalparindent
  \item[\hskip\labelsep
        \scshape
    #1\@addpunct{}]\ignorespaces
}{%
  \popQED\endtrivlist\@endpefalse
}
\makeatother

%% This adds a \rewnewtheorem command, which enables me to override the settings for theorems contained in this document.
\makeatletter
\def\renewtheorem#1{%
  \expandafter\let\csname#1\endcsname\relax
  \expandafter\let\csname c@#1\endcsname\relax
  \gdef\renewtheorem@envname{#1}
  \renewtheorem@secpar
}
\def\renewtheorem@secpar{\@ifnextchar[{\renewtheorem@numberedlike}{\renewtheorem@nonumberedlike}}
\def\renewtheorem@numberedlike[#1]#2{\newtheorem{\renewtheorem@envname}[#1]{#2}}
\def\renewtheorem@nonumberedlike#1{  
\def\renewtheorem@caption{#1}
\edef\renewtheorem@nowithin{\noexpand\newtheorem{\renewtheorem@envname}{\renewtheorem@caption}}
\renewtheorem@thirdpar
}
\def\renewtheorem@thirdpar{\@ifnextchar[{\renewtheorem@within}{\renewtheorem@nowithin}}
\def\renewtheorem@within[#1]{\renewtheorem@nowithin[#1]}
\makeatother

%% This makes theorems and definitions with names show up in bold, the way I like it. 
\makeatletter
\def\th@plain{%
  \thm@notefont{}% same as heading font
  \itshape % body font
}
\def\th@definition{%
  \thm@notefont{}% same as heading font
  \normalfont % body font
}
\makeatother

%===============================================
%==============Shortcut Commands================
%===============================================
\newcommand{\ds}{\displaystyle}
\newcommand{\B}{\mathcal{B}}
\newcommand{\C}{\mathbb{C}}
\newcommand{\F}{\mathbb{F}}
\newcommand{\N}{\mathbb{N}}
\newcommand{\R}{\mathbb{R}}
\newcommand{\Q}{\mathbb{Q}}
\newcommand{\T}{\mathcal{T}}
\newcommand{\Z}{\mathbb{Z}}
\renewcommand\qedsymbol{$\blacksquare$}
\newcommand{\qedwhite}{\hfill\ensuremath{\square}}
\newcommand*\conj[1]{\overline{#1}}
\newcommand*\closure[1]{\overline{#1}}
\newcommand*\mean[1]{\overline{#1}}
%\newcommand{\inner}[1]{\left< #1 \right>}
\newcommand{\inner}[2]{\left< #1, #2 \right>}
\newcommand{\powerset}[1]{\pazocal{P}(#1)}
%% Use \pazocal{letter} to typeset a letter in the other kind 
%%  of math calligraphic font. 
\newcommand{\cardinality}[1]{\left| #1 \right|}
\newcommand{\domain}[1]{\mathcal{D}(#1)}
\newcommand{\image}{\text{Im}}
\newcommand{\inv}[1]{#1^{-1}}
\newcommand{\preimage}[2]{#1^{-1}\left(#2\right)}
\newcommand{\script}[1]{\mathcal{#1}}


\newenvironment{highlight}{\begin{mdframed}[backgroundcolor=gray!20]}{\end{mdframed}}

\DeclarePairedDelimiter\ceil{\lceil}{\rceil}
\DeclarePairedDelimiter\floor{\lfloor}{\rfloor}

%===============================================
%===============My Tikz Commands================
%===============================================
\newcommand{\drawsquiggle}[1]{\draw[shift={(#1,0)}] (.005,.05) -- (-.005,.02) -- (.005,-.02) -- (-.005,-.05);}
\newcommand{\drawpoint}[2]{\draw[*-*] (#1,0.01) node[below, shift={(0,-.2)}] {#2};}
\newcommand{\drawopoint}[2]{\draw[o-o] (#1,0.01) node[below, shift={(0,-.2)}] {#2};}
\newcommand{\drawlpoint}[2]{\draw (#1,0.02) -- (#1,-0.02) node[below] {#2};}
\newcommand{\drawlbrack}[2]{\draw (#1+.01,0.02) --(#1,0.02) -- (#1,-0.02) -- (#1+.01,-0.02) node[below, shift={(-.01,0)}] {#2};}
\newcommand{\drawrbrack}[2]{\draw (#1-.01,0.02) --(#1,0.02) -- (#1,-0.02) -- (#1-.01,-0.02) node[below, shift={(+.01,0)}] {#2};}

%***********************************************
%**************Start of Document****************
%***********************************************

%===============================================
%===============Theorem Styles==================
%===============================================

%================Default Style==================
\theoremstyle{plain}% is the default. it sets the text in italic and adds extra space above and below the \newtheorems listed below it in the input. it is recommended for theorems, corollaries, lemmas, propositions, conjectures, criteria, and (possibly; depends on the subject area) algorithms.
\newtheorem{theorem}{Theorem}
\numberwithin{theorem}{section} %This sets the numbering system for theorems to number them down to the {argument} level. I have it set to number down to the {section} level right now.
\newtheorem*{theorem*}{Theorem} %Theorem with no numbering
\newtheorem{corollary}[theorem]{Corollary}
\newtheorem*{corollary*}{Corollary}
\newtheorem{conjecture}[theorem]{Conjecture}
\newtheorem{lemma}[theorem]{Lemma}
\newtheorem*{lemma*}{Lemma}
\newtheorem{proposition}[theorem]{Proposition}
\newtheorem*{proposition*}{Proposition}
\newtheorem{problemstatement}[theorem]{Problem Statement}


%==============Definition Style=================
\theoremstyle{definition}% adds extra space above and below, but sets the text in roman. it is recommended for definitions, conditions, problems, and examples; i've alse seen it used for exercises.
\newtheorem{definition}[theorem]{Definition}
\newtheorem*{definition*}{Definition}
\newtheorem{condition}[theorem]{Condition}
\newtheorem{problem}[theorem]{Problem}
\newtheorem{example}[theorem]{Example}
\newtheorem*{example*}{Example}
\newtheorem*{counterexample*}{Counterexample}
\newtheorem*{romantheorem*}{Theorem} %Theorem with no numbering
\newtheorem{exercise}{Exercise}
\numberwithin{exercise}{section}
\newtheorem{algorithm}[theorem]{Algorithm}

%================Remark Style===================
\theoremstyle{remark}% is set in roman, with no additional space above or below. it is recommended for remarks, notes, notation, claims, summaries, acknowledgments, cases, and conclusions.
\newtheorem{remark}[theorem]{Remark}
\newtheorem*{remark*}{Remark}
\newtheorem{notation}[theorem]{Notation}
\newtheorem*{notation*}{Notation}
%\newtheorem{claim}[theorem]{Claim}  %%use this if you ever want claims to be numbered
\newtheorem*{claim}{Claim}

 %located in ~/texmf/tex/latex/

\pgfplotsset{compat=1.13}

%\newcommand{\T}{\mathcal{T}}
%\newcommand{\B}{\mathcal{B}}

%These commands are now in tskpreamble_nothms.tex, but are left as a comment here for reference.
%\newcommand{\arbcup}[1]{\bigcup\limits_{\alpha\in\Gamma}#1_\alpha}
%\newcommand{\arbcap}[1]{\bigcap\limits_{\alpha\in\Gamma}#1_\alpha}
%\newcommand{\arbcoll}[1]{\{#1_\alpha\}_{\alpha\in\Gamma}}
%\newcommand{\arbprod}[1]{\prod\limits_{\alpha\in\Gamma}#1_\alpha}
%\newcommand{\finitecoll}[1]{#1_1, \ldots, #1_n}
%\newcommand{\finitefuncts}[2]{#1(#2_1), \ldots, #1(#2_n)}
%\newcommand{\abs}[1]{\left|#1\right|}
%\newcommand{\norm}[1]{\left|\left|#1\right|\right|}
%\newcommand{\extmeasure}[1]{m_*(#1)}
%\newcommand{\measure}[1]{m(#1)}
\renewcommand{\mathcal}[1]{\pazocal{#1}}

\title{Math 552 \linebreak
Homework 1}
\author{Trevor Klar}

\begin{document}

\maketitle

\begin{enumerate}
	\setcounter{enumi}{1}
	\item The Cantor set $\mathcal{C}$ can also be described in terms of ternary expansions.
	\begin{enumerate}
		\item Every number in $[0, 1]$ has a ternary expansion
		$$ x= \sum_{k=1}^{\infty}a_k3^{-k}, \quad \quad \text{where }a_k = 0, 1, \text{or } 2.$$
		Note that this decomposition is not unique since, for example, $\frac{1}{3} = \sum_{k=2}^{\infty}\frac{2}{3^k}.$ Prove that $x \in  \mathcal{C}$ if and only if $x$ has a representation as above where every $a_k$ is either 0 or 2.

		\begin{proof}
		First, there is another notation for ternary expansions which the reader may find easier to internalize, namely ternary decimals. Representing a number as a decimal in base 3 is exactly equivalent to the summation notation above; i.e. $\frac{1}{2}=\sum_{k=1}^{\infty}\frac{1}{3^k}=0.111\ldots = 0.\overline{1}$\,. So the construction of the cantor set is as follows:
		\jpg{width=0.5\textwidth}{cantor_set_ternary}
		Now we begin the proof in earnest. For any $x\in \mathcal C$, one of the following must be true: Case I, there exists an expansion of $x$ with no 1s. Case II, there exists an expansion of $x$ with exactly one 1. Case III, there exists an expansion of $x$ with multiple 1s. \\
		\mbox{}\\
		Case III: Let $x\in \mathcal C$ and suppose there exists an expansion of $x$ with multiple 1s; that is, $x=0.\square_11\square_21\ldots$, where the $\square$s represent zero or more other digits of the ternary decimal form of $x$. Claim: $x$ cannot be in $\mathcal{C}$, a contradiction. The construction of $\mathcal{C}$ begins by starting with the interval $[0,1]$ and, in each step, proceeds by removing middle thirds. That is, first $(0.1, 0.2)$ is removed, then $(0.01, 0.02)$, $(0.11, 0.12)$\footnote{Note, this interval was already removed in the last step, but this method of describing the removed portions is helpful, as will be seen.}, and $(0.21, 0.22)$, and so on. Thus, every such interval $(0.\square1, \,\, 0.\square2)$ is removed from the set at some point in the construction. However, when we consider our element, $x=0.\square_11\square_21\ldots$, we find that $0.\square_11, < x < 0.\square_12$, and so $x\not\in\mathcal{C}$. Thus Case III is impossible. \\
		\mbox{}\\
		Case II: Let $x\in \mathcal C$ and suppose there exists an expansion of $x$ with exactly one 1. If 1 is the terminal digit of $x$, then $x$ is of the form $0.\square1$, and we can also write $x$ as $0.\square0\overline 2$. Otherwise if 1 is not the terminal digit of $x$ then the only way $x\in\mathcal C$ is if $x=\square1\overline 2$ (using the same argument as in Case III), so we can write $x=\square2$. Thus Case II implies Case I. \\
		\mbox{}\\
		Case I: Let $x\in \mathcal C$ and suppose there exists an expansion of $x$ with no 1s. This is that which was to be proven, so we are done.

		\end{proof}
	\end{enumerate}

\setcounter{enumi}{3}
	\item \textbf{Cantor-like sets.} Construct a closed set $\hat{\mathcal{C}}$ so that at the $k$th stage of the construction one removes $2^{k−1}$ centrally situated open intervals each of length $\ell_k$, with
	$$\ell_1 + 2\ell_2 + ... + 2^{k-1}\ell_k < 1.$$
	\begin{enumerate}[label=(\alph*)]
		\item If $\ell_j$ are chosen small enough, then $\sum\limits_{k=1}^\infty 2^{k-1}\ell_k < 1$. In this case, show that $\measure{\hat{\mathcal{C}}} = 1 - \sum\limits_{k=1}^\infty 2^{k-1}\ell_k$.
		\item Show that if $x\in\hat{\mathcal{C}}$, then there exists a sequence of points $\{x_n\}_{n=1}^\infty$ such that $x\not\in\hat{\mathcal{C}}$, yet $x_n\to x$ and $x_n\in I_n$, where $I_n$ is a sub-interval in the complement of $\hat{\mathcal{C}}$ with $\abs{I_n}\to0$.
		\item Prove as a consequence that $\hat{\mathcal{C}}$ is perfect, and contains no open interval.
		\begin{proof}
		Exercise 4(b) is poorly posed, so I don't want to rely on it. This proof will be independent. We will show that for every $x\in\hat{\mathcal{C}}$ and any $\epsilon>0$, the open ball $B_\epsilon(x)$ contains a point in $\hat{\mathcal{C}}$ and a point not in $\hat{\mathcal{C}}$. This means that every element of $\hat{\mathcal{C}}$ is a limit point (thus $\hat{\mathcal{C}}$ is perfect), and no subset of $\hat{\mathcal{C}}$ is open (thus it contains no open intervals). \\
		\mbox{} \\
		Let $x\in\hat{\mathcal{C}}$ and $\epsilon>0$ be given. In step $k$ of the construction, there are $2^{k-1}$ disjoint closed intervals in $\hat{\mathcal{C}}_k$, so their lengths can be no more than $\frac{1}{2^{k-1}}$. Thus there exists some $K$ such that $\frac{1}{2^{K-1}}<\epsilon$. Since $x\in\hat{\mathcal{C}}$, then $x$ is in some interval in $\hat{\mathcal{C}}_k$ for all $k$. Thus $B_\epsilon(x)$ contains some closed interval $I$ in $\hat{\mathcal{C}}_K$, and in step $K+1$, an open interval $\hat{I}$ in $I$ is removed from $\hat{\mathcal{C}}_{K+1}$. The endpoints of $\hat{I}$ are elements of $\hat{\mathcal{C}}$, and the midpoint of $\hat{I}$ is not. Since $\hat{I}\subset I\subseteq B_\epsilon(x)$, then the ball contains a point in $\hat{\mathcal{C}}$ and a point not in $\hat{\mathcal{C}}$, as desired.
		\end{proof}
		\item Show also that $\hat{\mathcal{C}}$ is uncountable.
		\begin{proof}
		In each step $\hat{\mathcal{C}}_k$ of the construction of $\hat{\mathcal{C}}$, the set is composed of $2^{k-1}$ closed intervals. This is also true of the usual Cantor set, $\mathcal{C}$. Since any two closed intervals are isomorphic\footnote{If this is not obvious, consider $[\alpha, \beta]$ and $[a,b]$. Map $\alpha\mapsto a$ and $\beta\mapsto b$, and extend linearly.}, then each $\hat{\mathcal{C}}_k$ is isomorphic to $\mathcal{C}_k$, and so are the limiting sets $\hat{\mathcal{C}}\cong\mathcal{C}$. We know that $\mathcal{C}$ is uncountable, so we are done.
		\end{proof}
	\end{enumerate}

	\setcounter{enumi}{11}
	\item Theorem 1.3 states that every open set in $\mathbb{R}$ is the disjoint union of open intervals. The analogue in $\mathbb{R}^d, d >= 2$, is generally false. Prove the following:
		\begin{enumerate}
		\item An open disc in $\mathbb{R}^2$ is not the disjoint union of open rectangles.
		\begin{proof}
		Let $C\subset\R^2$ be an open disc.
		% if $C$ is a disjoint union of open rectangles, then there are more than one.
		Consider a rectangle $R\subset C$.\footnote{A rectangle is not a disk, so $R\subsetneq C$.} For any $x\in\del R$, we know $x\in C$ but $x$ cannot be an element of any open rectangle which is disjoint with $R$. Thus no disjoint union of open rectangles can contain $x$, nor can it cover $C$.
		\end{proof}
		\end{enumerate}

  \item The following deals with $G_{\sigma}$ and $F_{\delta}$ sets.
  \begin{enumerate}
    \item[(b)] \item Give an example of an $F_{\sigma}$ which is not a $G_{\delta}$:
    \answer $\Q$ is $F_{\sigma}$ but not $G_{\delta}$.
    \begin{proof}
      %$\Q$ is closed, so it is $F_\sigma$. To see that $\Q$ is not $G_\delta$, suppose for contradiction that $\Q=\bigcap_{n=1}^\infty U_n.$ Then each open set $U_n$ contains $Q$. Since $U_n$ is open, then for every rational $x\in\Q$ there exists an $\epsilon_x$ such that $B_{}\epsilon_x}(x)\subset U_n$, so $\bigcup_{x\in\Q} B_{}\epsilon_x}(x)$ covers
    \end{proof}
  \end{enumerate}

	\item \mbox{}
	\jpg{width=0.9\textwidth}{prob_14}
		\begin{enumerate}
		\item
		\begin{proof}
		$(\leq)$ To see that $J_*(E)\leq J_*(\closure{E})$ for all $E\in\R$, observe that any cover $\phi=\{I_n\}_{n=1}^\infty$ which covers $\closure{E}$ also covers $E$. Thus $J_*(E)$ cannot be greater than $J_*(\closure{E})$.
		\end{proof}
		\begin{proof}
		$(\geq)$ To show that $J_*(E)\geq J_*(\closure{E})$ for all $E\in\R$, we will show that for any cover $\phi=\{I_n\}_{n=1}^\infty$ which covers $E$, $\closure{\phi}=\{\closure{I_n}\}_{n=1}^\infty$ also covers $\closure{E}$, and $\abs{\phi}=\abs{\closure{\phi}}$. \\
		\mbox{}\\
		Let $\phi$ be some cover of $E$ as described above. To see that $\closure{\phi}$ covers $\closure{E}$, let $x\in\closure{E}$. If $x$ is also in $E$, then $x\in I_n \subset \closure{I_n}$ for some $n$. Otherwise if $x\not\in E$, then $x$ must be a limit point of $E$, since $x\in\closure{E}$. Since $\closure{\phi}$ is a finite union of closed intervals, it is itself closed and thus contains all its limit points; and since $E\subset\closure{\phi}$, $\closure{\phi}$ contains all the limit points of $E$ as well. Thus $x\in\closure{\phi}$.
		\end{proof}
		\end{enumerate}

\setcounter{enumi}{15}
\item \textbf{The Borel-Cantelli lemma.} Suppose $\{E_k\}_{k=1}^\infty$ is a countable family of measurable subset of $\R^d$ and that
$$\sum_{k=1}^\infty \measure{E_k}<\infty.$$
Let
\[\begin{array}{rcl}
E &=& \{x\in\R^d: \forall n\in \N, \exists k\geq n \text{ s.t. } x \in E_k\}\\
&=& \limsup\limits_{k\to\infty}(E_k).
\end{array}\]
	\begin{enumerate}
	\item Show that $E$ is measurable.
	\item Prove $\measure{E}=0$.
	\end{enumerate}
[Hint: Write $E=\bigcap_{n=1}^\infty \bigcup_{k\geq n} E_k.$]
\begin{proof}
We will show that $E$ is a subset of a set of measure zero, and thus is measurable with $\measure{E}=0$.

Since $\sum_{k=1}^\infty \measure{E_k}$ is a convergent sum, then for any $\epsilon>0$, there exists some $N\in\N$ such that $\sum\limits_{k\geq N} \measure{E_k}<\epsilon$.

Then for any $x\in E$, $x\in\bigcup\limits_{k\geq N}E_k$, since the union contains some $E_k$ with $k>N$ for any $x$. Now since $E \subseteq \bigcup\limits_{k\geq N}E_k$, then by sub-additivity,
$$\measure{E}\leq\sum\limits_{k\geq N} \measure{E_k}<\epsilon$$
and we are done.
\end{proof}

\item Let $\{f_n\}$ be a sequence of measurable functions on $[0,1]$ with $|f_n(x)|<\infty$ for a.e. $x$. Show that there exists a sequence $\{c_n\}$ of positive real numbers such that
$$\frac{f_n(x)}{c_n}\to0 \quad \text{for a.e. }x.$$
[Hint: Pick $c_n$ such that $\measure{\left\lbrace \abs{\frac{f_n(x)}{c_n}}>\frac{1}{n}\right\rbrace}<\frac{1}{2^n}$, and apply the Borel-Cantelli Lemma.]
\begin{proof}
Consider $D_n=\{|f_n|>\frac{c_n}{n}\}$, where $c_n$ is an as-yet undetermined positive number. We know that $f_n$ is measurable, so $\measure{D_n}$ exists for all values of $c_n$. If we consider $\measure{D_n}$ as a function of $c_n$ with $n$ fixed, then $\measure{D_n}(c_n)$ is monotonically decreasing. This is because $\{|f_n|>a\}\supseteq\{|f_n|>b\}$ for all $a<b$. Now we choose $c_n$ large enough that $\measure{D_n}<\frac{1}{2^n}$.\footnote{We know we can do this because if not, then $\measure{\{|f_n(x)|=\infty\}}>0$.}

Thus we have constructed a sequence of sets $\{D_n\}_{n=1}^\infty$ such that
$$\sum_{n=1}^\infty \measure{D_n} < \sum_{n=1}^\infty \frac{1}{2^n} = 1 < \infty,$$
so by the Borel-Cantelli Lemma, $\measure{\limsup\limits_{n\to\infty} (D_n)}=0$.

Now to see that we are done, observe that for $x\not\in D_n$,
$$|f_n(x)|\leq \frac{c_n}{n} \text{, so } \abs{\frac{f_n(x)}{c_n}}\leq\frac{1}{n}.$$
This means that for $x$ in finitely many $D_n$, that is, $x\not\in\limsup\limits_{n\to\infty} (D_n)$,
$$\lim_{n\to\infty}\frac{f_n(x)}{c_n}=0.$$
We have shown that $\measure{\limsup\limits_{n\to\infty} (D_n)}=0$, so we conclude that
$$\frac{f_n(x)}{c_n}\to0 \quad \text{for a.e. }x,$$
as desired.
\end{proof}

\item[] (Reading Question). Construct a function that is not measurable but finite valued everywhere. \\
\textbf{Answer:} $\Chi_\mathcal{N}$, the characteristic function of the nonmeasurable set constructed in the text. It only takes the values $\{0,1\}$, but $\{$\raisebox{2pt}{$\chi$}$_\mathcal{N} > \frac{1}{2}\}=\mathcal{N}$ is not measurable.

\end{enumerate}

\pagebreak
\textbf{1.3}

\begin{enumerate}

  \setcounter{enumi}{22}
  \item Suppose $f(x,y)$ is a function on $\mathbb{R}^2$ that is separately continuous: for each fixed variable, $f$ is continuous in the other variable. Prove that $f$ is measurable on $\mathbb{R}^2$. [Hint: Approximate $f$ by functions $f_n$ which are piecewise-constant in the variable $x$, such that $f_n \rightarrow f$ pointwise.]
  \begin{proof}
    First we approximate $f(x,y)$ by rounding $x$ down to the nearest multiple of $\frac{1}{2^k}$. That is, given $n\in\N$ and $x\in\R$, there exists some $r\in\N$ such that $\frac{r}{2^n} \leq x < \frac{r+1}{2^n}$. So we approximate $f$ by $f_n(x,y)=f(\frac{r}{2^n}, y)$. To visualize, following is a pair of illustrations of a possible such $f_n$; having some fixed $y$ on the left, and some fixed $x$ on the right.
    \jpg{width=0.66\textwidth}{f_n_1-23}
    Now we ask; given some $\alpha\in\R$, is $\{f_n > \alpha\}$ measurable? We know that $f$ is continuous in $y$ for fixed $x$, so for some $r\in\Z$ we have that
    $$\{y\in\R | f(\tfrac{r}{2^n}, y) > \alpha\}$$
    is measurable. Also, for any $x \in \R$, we know there is some $r$ such that $x \in \left[\frac{r}{2^n}, \frac{r+1}{2^n}\right)$, so
    $$\{y\in\R | f_n(x, y) > \alpha\} = \{y\in\R | f(\tfrac{r}{2^n}, y) > \alpha\}.$$
    This means that for $x \in \left[\frac{r}{2^n}, \frac{r+1}{2^n}\right)$,
    $$\{(x,y) | f_n(x,y) > \alpha\} = \left(\left[\tfrac{r}{2^n}, \tfrac{r+1}{2^n}\right) \times \{y\in\R | f(\tfrac{r}{2^n}, y) > \alpha\}\right),$$
    which is certainly measurable. Now if we consider all possible values of $x$, we find that
    $$\{f_n > \alpha\} = \bigcup_{r=-\infty}^\infty\left(\left[\tfrac{r}{2^n}, \tfrac{r+1}{2^n}\right) \times \{y\in\R | f(\tfrac{r}{2^n}, y) > \alpha\}\right),$$
    which is a countable union of measurable sets, and thus a $G_\sigma$, therefore measurable. Thus we can conclude that $f_n$ is measurable for all $n\in\Z$. Since $f_n\to f$, we can conclude that $f$ is measurable as well, and we are done.
  \end{proof}
\pagebreak

%%%%%%%%%%%%%%%%%%%%%%%%%%%%%%%%%%%%%%%%%%%%%%%%%%%%%
    \setcounter{enumi}{24}
    \item An alternative definition of measurability is as follows: $E$ is measurable if for every $\epsilon >0$ there is a \textit{closed} set $F$ contained in $E$ with $m_*(E-F) < \epsilon.$ Show that Properties 1-4 of the Lebesgue measure still hold.
    \begin{enumerate}[label=(\alph*)]
    \item{}
    \item{} Property 2: If $m_{∗}(E)=0$, then $E$ is measurable. In particular, if $F\subseteq E$ and $m_∗(E) = 0$, then F is measurable.
    \begin{proof}
    Suppose $m_{∗}(E)=0$ and let $\epsilon>0$ be given. Then let $F=\{x\}$ for some $x\in E$, thus $F$ is closed and $F\subset E$. Then by monotonicity, $\extmeasure{E-F}<\extmeasure{E}=0<\epsilon$.
    \end{proof}
    \item{} Property 3: If $E$ is a countable collection of measurable sets $E_j$ then $\bigcap_{j=1}^\infty E_j$ is also measurable.
    \begin{proof}
    Denote $E=\bigcap_{j=1}^\infty E_j$ with each $E_j$ measurable.  Then there exists an $F_j$ such that $F_j \subset E_j$ with $m_*(E_j - F_j) < \frac{\epsilon}{2^j}$.  As $F_j \subset E_j$ then $\bigcap_{j=1}^\infty F_j \subset \bigcap_{j=1}^\infty E_j$.   We denote $\bigcap_{j=1}^\infty F_j = F$ and note $F$ is the countable intersection of closed sets thus is closed.  We also observe the following:
    $$E-F = \bigg[\bigcap_{j=1}^\infty E_j\bigg]  - \bigg[\bigcap_{j=1}^\infty F_j\bigg] = \bigg[\bigcap_{j=1}^\infty E_j\bigg] \bigcap \bigg[\bigcap_{j=1}^\infty F_j\bigg]^c =$$

    $$\bigg[\bigcap_{j=1}^\infty E_j\bigg]\bigcap \bigg[ \bigcup_{j=1}^\infty F_j^c\bigg] \subseteq \bigg[\bigcup_{j=1}^\infty E_j\bigg] \bigcap \bigg[ \bigcup_{j=1}^\infty F_j^c \bigg] = \bigcup_{j=1}^\infty \big[E_j \bigcap F_j^c\big] = \bigcup_{j=1}^\infty \big[E_j - F_j\big]
    $$

    Thus by Observation 1 (Monotonicity) and $\big[E-F\big] \subseteq \bigcup_{j=1}^\infty \big[E_j - F_j\big]$ then:

    $$m_*\big[E-F\big] \leq m_*\bigcup_{j=1}^\infty \big[E_j - F_j\big] \leq \sum^\infty_{j=1}m_*\big[E_j - F_j\big] < \sum^\infty_{j=1} \frac{\epsilon}{2^j} = \epsilon$$

    Thus showing $E$ is measurable.
    \end{proof}
    \item{} Property 4: Open sets are measurable - we were unable to prove the general case, but did prove the following special cases:
    \begin{enumerate}[label=\roman*]
        \item{}Open sets with finite exterior measure are measurable.
        \begin{proof}
            Let $O$ be an open subset of $\mathbb{R}^d$ with $m_*(O) < \infty$ and let $\epsilon >0$ be arbitrary. Then $O$ is the countable union of almost disjoint cubes: $O = \cup_{j=1}^\infty Q_j$ and $m_*(O) = \sum_{j=1}^\infty |Q_j|$. Since $m_*(O) < \infty$, this is a convergent series and so the tails must go to 0. That is, there exists some $N \in {\N}$ such that $\sum_{j=N}^\infty |Q_j| < \epsilon$. Then, letting $F = \cup_{j=1}^{N-1} Q_j$, we have that $$m_*(O-F) = m_*( \cup_{j=1}^{\infty} Q_j - \cup_{j=1}^{N-1} Q_j ) = m_*(\cup_{j=N}^\infty Q_j) =  \sum_{j=N}^\infty |Q_j| < \epsilon.$$ Since $F$ is the intersection of closed sets, it is closed and $O$ is measurable.

            % [We can't generally say that a measure of a difference is the difference of the measures, but we can with cubes, why is this? Does Observation 5 justify this?]

            \end{proof}
        \item{} Open sets with complements of finite exterior measure are measurable.
        \begin{proof}
         Let $O$ be an open subset of $\mathbb{R}^d$ with $m_*(O^c) < \infty$ and let $\epsilon >0$ be arbitrary. Define $B_R$ to be the open ball of radius $R$ about the origin; clearly $B_R$ has finite measure. Observe that we can rewrite $O$ as $$ O = O\cap B_R \cup O \cap B_R^c.$$ The goal is to show that this is in fact the union of two open, measurable sets and hence measurable by Property 3, but it would have to be Property 3 of the open sets definition of measurable, so I don't see why that would be available to us at all. \textbf{Can anyone confirm that this is what we were aiming for?} Proceeding nonetheless.

         The first term $O \cap B_R$ is open and has finite exterior measure: it is open because it is the intersection of two open sets and has finite exterior measure as a result of countable subadditivity. Then, the term is measurable by Property 4.i. Dr. Horn seemed to be concerned that this was not actually open and began instead working with $O \cap Int(B_{n+1} \cap B_n$; I don't understand that construction.

         \textbf{Remaining questions: how does that construction help us? Why would the second term be open and of finite measure?} In fact, since we're given that $O^c$ is finite, then $O \cap B_R^c = B_R^c - O^c$ would seem to be a set of infinite exterior measure less something of finite exterior measure, and thus still having infinite exterior measure.

        \end{proof}
    \end{enumerate}
    \end{enumerate}

\setcounter{enumi}{26}
\item Suppose $E_1$ and $E_2$ are a pair of compact sets in $\R^d$ with $E_1\subseteq E_2$, and let $a=\measure{E_1}$ and $b=\measure{E_2}$. Prove that for any $c$ with $a<c<b$, there is a compact set $E$ with $E_1\subset E \subset E_2$ and $\measure{E}=c$.
\begin{proof}
For this proof, we will construct a continuous function to make a correspondence between numbers in $[a,b]$ and subsets of $E_2$, and use the IVT to produce the desired set.

Let $E_1\subset E_2\subset \R^d$ with $a=\measure{E_1}$ and $b=\measure{E_2}$, and let $a<c<b$. Let $\{r_i\}_{i=0}^\infty$ be an enumeration of the rational numbers $\Q\cap[1,2]$. Since $\R^d$ is a normal topological space, we can find a compact set $E_{r_0}$ such that $$E_1\subsetneq E_{r_0}\subsetneq E_2,$$ and continue this construction for all other $r_i$. That is, for each $i\in\N$ and $p,q\in\Q\cap[1,2]$, we can find $E_{r_i}$ such that
$$\text{if }p<r_i<q\text{, then }E_{p}\subsetneq E_{r_i}\subsetneq E_{q}.$$
Now we define a continuous function $f:[1,2]\to[a,b]\text{ by }$
$$f(q)=\measure{E_q} \text{ for } q\in\Q,$$ and extending continuously\footnote{This function is unique since a continuous function is completely determined by its behavior on a dense subset of its domain.}. Since $f$ is continuous, there exists some $\gamma\in[1,2]$ such that $f(\gamma)=c$ by the Intermediate Value Theorem. We would like to take $E_\gamma$ as our desired set, but $\gamma$ is probably an irrational number, so $E_\gamma$ may not be defined. However, using the decimal expansion $\gamma=1+\sum\limits_{i=1}^\infty g_i(10)^{-i}$, we can obtain a countable sequence of sets $\{E_{\gamma_n}\}_{n=1}^\infty\text{ where }$
$$\gamma_n=\left(1+\sum\limits_{i=1}^n g_i(10)^{-i}\right)+10^{-i}.$$
Note that $\gamma<\gamma_n$ for all $n$, and $\gamma_n\to\gamma$. Finally we can define $E_\gamma$ as $\bigcap_{i=1}^\infty E_{\gamma_i}$. Since $E_{\gamma_i}\searrow E_\gamma$ and $\measure{E_{\gamma_i}}<2<\infty$, then
$$\measure{E}=\lim\limits_{i\to\infty} \measure{E_{\gamma_i}}=c.$$
We know that $E_\gamma$ is compact since it is the countable intersection of compact sets, and thus we are done.
\end{proof}


\setcounter{enumi}{37}
\item Prove that $(a+b)^\gamma\geq a^\gamma + b^\gamma$ whenever $\gamma\geq 1$ and $a, b \geq 0$. Also, show that the reverse inequality holds when $0 \leq \gamma \leq 1$. \\
$[$Hint: Integrate the inequality between $(a+t)^{\gamma-1}$ and $t^{\gamma-1}$ from 0 to $b.]$
\begin{proof}
Recall that (assuming $x > 0$) $f(x)=x^\alpha$ is an increasing function when $\alpha>0$, and decreasing when $\alpha<0$. Thus when $\gamma\geq1$, then $(a+t)^{\gamma-1}-t^{\gamma-1} \geq 0$ for all $t,a\geq0$; and when $\gamma\leq1$, the inequality is reversed. Thus if $\gamma\geq1$,
\[\begin{array}{rcl}
0&\leq& \int_0^b (a+t)^{\gamma-1}-t^{\gamma-1} \der t\\
&=&\left[\frac{1}{\gamma}(a+t)^\gamma - \frac{1}{\gamma}t^\gamma\right]_{t=0}^b\\
&=&\frac{1}{\gamma}\left[(a+b)^\gamma-b^\gamma\right]-\frac{1}{\gamma}a^\gamma\\
&=&\frac{1}{\gamma}	\left[(a+b)^\gamma-(a^\gamma+b^\gamma)\right].
\end{array}\]
Thus, $0\leq\frac{1}{\gamma}	\left[(a+b)^\gamma-(a^\gamma+b^\gamma)\right]$, and rearranging yields the desired result.
\end{proof}


\item[(reading question)] Find a continuous function $f:\R^+\to\R^+$ such that $\limsup f = \infty$, but $\int\limits_{\R^+} f < \infty$. \\
\mbox{}\\
\answer Let $f:\R^+\to\R^+$ be the function determined by countably many triangles $\{T_n\}$; each centered on an integer point $(n,0)$, with area $2^{-n}$, and height $n$.
\jpg{width=0.33\textwidth}{spiky_fn}
For a more formal definition, consider the set of all vertices of $T_n$ for all $n$, together with the origin. Define $f$ by connecting all these points with line segments, in order of their $x$-coordinates. We can confirm that this function is defined as expected by observing that none of the triangles overlap: Since they are all centered on integers, their centers have distance 1, and $T_1$ has radius $\sfrac{1}{2}$, \footnote{That is, it has vertices $(n-\sfrac{1}{2}, 0), (n,n), (n+\sfrac{1}{2}, 0)$} and all future triangles have smaller width.

To see that this function has the desired properties, observe that $\limsup f = \infty$ because for all $n\in\N, f(n)=n$. Now observe that
\[\begin{array}{rcl}
\int\limits_0^\infty f &=& \sum\limits_{n=1}^\infty \text{area}(T_n)\\
&=& \sum\limits_{n=1}^\infty \frac{1}{2^n}\\
&=& 1\\
&<& \infty.
\end{array}
\]
\qed
\end{enumerate}

\section{Chapter 2}

\begin{enumerate}
  \item[]\textbf{Reading question for 2.1 stage 1:} Let $f$ and $g$ be simple functions with $f(x)=g(x)$ for a.e. $x$. Prove that $\int_\R f(x)\dx=\int_\R g(x)\dx$.
  \begin{proof}
    Let
    $$\sum_{k=1}^N a_k \Chi_{E_k}, \; \sum_{k=1}^M b_k\Chi_{F_k}$$
    be the canonical forms of $f$ and $g$ respectively, and let $\Delta = \{x:f(x)\neq g(x)\}$. Since $f=g$ almost everywhere, then $\measure{\Delta}=0$. Thus
    $$\begin{array}{rcl}
      f(x) &=& \ds\sum_{k=1}^N a_k\Chi_{\tilde{E_k}} + f(x)\Chi_\Delta, \text{ and} \\
      g(x) &=& \ds\sum_{k=1}^N b_k\Chi_{\tilde{F_k}} + g(x)\Chi_\Delta,
    \end{array}$$
    where $\tilde{E_k}=E_k-\Delta$ and $\tilde{F_k}=F_k-\Delta$. Note that after a possible reordering, $\tilde{E_k}=\tilde{F_k}$ and $a_k=b_k$, since $f|_{\R-\Delta} = g|_{\R-\Delta}$ and these are the canonical forms of $f$ and $g$. Thus
    $$\begin{array}{>{\ds}rc>{\ds}l}
      \int_\R f(x)\dx &=& \int_\R\left(\sum_{k=1}^Na_k\Chi_{\tilde{E_k}}+f(x)\Chi_\Delta\right)\dx \\
      &=&\sum_{k=1}^Na_k\measure{\tilde{E_k}}+\int_\Delta f(x)\dx\\
    \end{array}$$
    and $\int_\R f = 0 = \int_\R g$, so
    $$\begin{array}{>{\ds}rc>{\ds}l}
      \phantom{\int_\R f(x)} &=& \sum_{k=1}^Nb_k\measure{\tilde{F_k}}+\int_\Delta g(x)\dx\\
      &=& \int_\R g(x) \dx
    \end{array}$$
    and we are done.
  \end{proof}

  \setcounter{enumi}{9}
  \item We can break exercise 10 into two parts:
  \begin{enumerate}
    \item[Part 1:] Suppose $f>0$, and let $E_{2^k}=\{f > 2^k\}$ and $F_{k}=\{2^k < f \leq 2^{k+1}\}$. If $f$ is finite almost everywhere, note that
      $$\bigcup_{k=-\infty}^\infty F_k = \{ f > 0\}$$
    and the sets $F_k$ are disjoint. Prove that the following are equivalent:
    \begin{itemize}
      \item $f$ is integrable,
      \item $\ds\sum_{k=-\infty}^\infty 2^k\measure{F_k} < \infty$, and
      \item $\ds\sum_{k=-\infty}^\infty 2^k\measure{E_{2^k}} < \infty$.
    \end{itemize}
    \begin{proof}
      YELLOW
    \end{proof}

    \item[Part 2:] Let $f,g:\R^d\to\R$ be:
      $$f(x) = \begin{cases}
        \abs{x}^{-a} & \text{if } \abs{x}\leq1, \\
        0 & \text{otherwise}
      \end{cases}$$
      $$g(x) = \begin{cases}
        \abs{x}^{-b} & \text{if } \abs{x}>1, \\
        0 & \text{otherwise}
      \end{cases}$$
    for some $a,b\in\R^+$. Prove that
    \begin{enumerate}[label=(\roman*)]
      \item $f$ is integrable on $\R^d$ iff $a<d$, and
      \item $g$ is integrable on $\R^d$ iff $b>d$.
    \end{enumerate}
    \begin{proof}Part (i):
      For the function $f$, consider $E_{2^k}$ for $k\in\Z$.
        \jpg{width=0.33\textwidth}{f_and_g_prob_2-10}
      Observe that for $k>0$,
        $$E_{2^k} = \{f > 2^k\} = \{\abs{x}^{-a} > 2^k\} = \{\abs{x} < 2^{-\frac{k}{a}}\} = B_{2^{-\frac{k}{a}}},$$
      where $B_{2^{-\frac{k}{a}}}$ is a ball of radius $2^{-\frac{k}{a}}$ centered at the origin. The volume of such a ball in $\R^d$ is given by $V_dr^d$, where $r$ is the radius and $V_d$ is some constant which depends on the dimension $d$. Also observe that for $k\leq 0$, we have
        $$E_{2^k}=B_1,$$
      so $\measure{E_{2^k}}=V_d$. Now we will use these observations to calculate one of the sums from Part 1, as it relates to $f$:
      $$\begin{array}{rcl}
        \ds\sum_{k=-\infty}^\infty 2^k\measure{E_{2^k}} &=& \ds\sum_{k=-\infty}^\infty 2^k\measure{B_{2^{-\frac{k}{a}}}} \\
        &=& \ds\sum_{k=-\infty}^\infty (2^k)(V_d)(2^{-\frac{kd}{a}}) \\
        &=& \ds\sum_{k=-\infty}^\infty (V_d)(2^{1-\frac{d}{a}})^k \\
      \end{array}$$
      which converges iff $a<d$. \phantom\qedhere \qedwhite
    \end{proof}
    \begin{proof}Part (ii):
      Now we consider $E_{2^k}$ as it applies to $g$. For $k\geq0$, observe that $E_{2^k}$ is empty, so $\measure{E_{2^k}}=0$. Now we know that $g(x)=0$ for any $x\in B_1$, so for $k<0$,
      $$\begin{array}{rcl}
        E_{2^k}&=&B_{2^{-\frac{k}{b}}}\cap B_1^\complement \\
        &\subset& B_{2^{-\frac{k}{b}}},
      \end{array}$$
      and we can calculate as in Part (i):
      $$\begin{array}{rcl}
        \ds\sum_{k=-\infty}^\infty 2^k\measure{E_{2^k}} &=& \ds\sum_{k=-\infty}^{-1} 2^k\measure{E_{2^k}} \\
        &<& \ds\sum_{k=-\infty}^{-1} 2^k\measure{B_{2^{-\frac{k}{b}}}} \\
        &=& \ds\sum_{k=-\infty}^{-1} (V_d)(2^{1-\frac{d}{b}})^k \\
      \end{array}$$
      which converges iff $b>d$. \footnote{If this is not obvious, note that $k$ is negative, so the sum converges iff $2^{1-\frac{d}{b}}>1$ iff $1-\frac{d}{b}>0$ iff $1>\frac{d}{b}$ iff $b>d$.}
    \end{proof}
  \end{enumerate}

  \item Prove that if $f:\R^d\to\R$ is integrable on $\R^d$ with $\int_E f \geq 0$ for every measurable $E$, then $f(x) \geq 0$ for a.e. $x$. Also prove that if $\int_E f = 0$ for every measurable $E$, then $f(x) = 0$ for a.e. $x$.
  \begin{proof}
    Let $E_k = \{f<-\frac{1}{k}\}$. Then $\bigcup_{k=1}^\infty E_k = \{f < 0\}$. For any $k >0$,
    $$\begin{array}{rcl}
      0 &\leq& \int_{E_k} f(x)\dx \\
      &<& \int_{E_k} -\frac{1}{k}\dx \\
      &=& -\frac{1}{k}\measure{E_k}.
    \end{array}$$
    Thus $0\leq-\frac{1}{k}\measure{E_k}$, and measures are nonnegative, so $\measure{E_k}$ must be 0. Since $\measure{E_k}=0$ for all $k$, then by subadditivity
    $$\measure{\bigcup_{k=1}^\infty E_k}\leq\sum_{k=1}^\infty \measure{E_k}=0,$$
    so $\measure{\bigcup_{k=1}^\infty E_k}=\measure{f<0}=0$, and $f\geq0$ for a.e. $x$.
  \end{proof}
  \begin{proof}[Corollary]
    If $\int_E f = 0$ for every measurable $E$, then we have that $\int_E f \geq 0$ and $\int_E -f \geq 0$, so we conclude that $f(x) \geq 0$ and $-f(x) \geq 0$ for a.e. $x$, thus $f(x)=0$ almost everywhere.
  \end{proof}

\item Show that there are $f\in L^1(\R^d)$ and a sequence $\{f_n\}$ with $f_n\in L^1(\R^d)$ such that
$$\norm{f-f_n}_{L^1}\to0,$$
but $f_n(x)\to f(x)$ for no $x$.

[Hint: In $\R$, let $f_n=\Chi_{I_n}$, where $I_n$ is an appropriately chosen sequence of intervals with $\measure{I_n}\to 0$.]
\begin{proof}
  Phase I: We will first find such a sequence of functions that has the desired properties only on $[0,1]$. By Dirichlet's Approximation Theorem, we know that for any $x\in[0,1]$, there exist infinitely many $p,q$ such that
  $$\abs{\frac{p}{q}-x}<\frac{1}{q^2}.$$
  So we construct $\{I_n\}$ using an enumeration of the rationals in $[0,1]$ where $r_n=\frac{p}{q}$, so that
  $$I_n=\left(\frac{p}{q}-\frac{1}{q^2}\,,\, \frac{p}{q}+\frac{1}{q^2}\right).$$
  Now if we consider the characterstic functions of these sets, we find that $\{\Chi_{I_n}\}$ is a sequence of functions such that
  $$\norm{0-\Chi_{I_n}}\to 0,$$
  since $\int \Chi_{I_n} = \measure{I_n} = \frac{2}{q^2} \to 0$ as $n\to\infty$. To see that this is true, recall that for any $\epsilon>0$, there are finitely many $q\in\N$ such that $\frac{2}{q^2}\geq\epsilon$, and for each $q$ there are finitely many $\frac{p}{q}\in[0,1]$.

  However, $\lim\limits_{n\to\infty}\Chi_{I_n}(x)\neq0$, since each $x\in I_n$ for infinitely many $I_n$. 
  %Any enumeration of the rationals must contain $\Z$, and if $\frac{p}{q}$ is an integer, then $\frac{2}{q^2} = 2$. So $\measure{I_n}=2$ for infinitely many $I_n$, and the measure can't converge to zero.
  %The only way I can think of to recover this property is instead considering the rationals between 0 and 1, but then $f_n(x)\to f(x)$ for $x\not\in[0,1]$.

  Phase II: Now we extend the sequence of functions to work on $\R^+$. For any $x\in\R^+$, we can write $x$ as $m+\alpha$, where $m$ is given by truncating $x$, that is, $m=\floor{x}$ and $\alpha = x-m$. Then we define 
  $$f_n(x)=\frac{1}{2^m}\Chi_{I_n}(\alpha).$$
  To see that $\int f_n\to0$, observe that 
  $$\int_\R f_n(x)\dx=\int_\R\frac{1}{2^m}\Chi_{I_n}(\alpha)\dx\leq\int_\R\frac{1}{2^m}\Chi_{[0,1]}(\alpha)\dx=\sum_{m=0}^\infty \frac{1}{2^m}=2<\infty,$$
  so then since $\int\Chi_{I_n}\to0$, then so does $\int f_n\to0$ as $n\to\infty$. Also, $f_n(x)\not\to 0$ for the same reasons as in phase 1, since we are using the same map everywhere, and the only difference is in the magnitude. 
  
  Phase III: Finally, we extend the sequence of functions to all of $\R$ merely by considering $f_n(\abs{x})$, and we are done. 
\end{proof}

\item[2.7] \mbox{}
	\begin{proof} \textbf{of (a) and (b)} (I could only think of a proof that accomplishes both parts simultaneously.) Let 
	$$E_n = \bigcup_{k\in\Z}\left(\preimage{f}{\left[\frac{k}{2^n}, \frac{k+1}{2^n}\right]}\times\left[\frac{k}{2^n}, \frac{k+1}{2^n}\right]\right).$$
	We would like to say that $E_n\to\Gamma$ and $\measure{E_n}\to0$ so we are done, but we need to be a bit more careful than that. First, we bound $E_n$ and $\Gamma$, so that 
	$$E_n^b = E_n\cap\left([-\tfrac{b}{2}, \tfrac{b}{2}]^d\times\R\right), \text{ and}$$
	$$\Gamma^b = \Gamma\cap\left([-\tfrac{b}{2}, \tfrac{b}{2}]^d\times\R\right).$$
	Then we have that $\measure{E^b_n}=\frac{b^d}{2^n}$, so as $n\to\infty$ we have $E^b_n\searrow\Gamma^b$ and $\measure{\Gamma^b}=0$. \footnote{The bound was needed here because $\measure{\Gamma^b}=\lim_{n\to\infty}\measure{E^b_n}$ only when some $E^b_n$ has finite measure.} Now this holds for any $b\in\N$, so then $\Gamma^b\nearrow\Gamma$ and $\measure{\Gamma^b}=0$ for all $b$. Therefore  
	$$\lim_{b\to\infty}\measure{\Gamma^b}=\measure{\Gamma}=0$$
	and we are done. 
	\end{proof}

\item[3.14] 
	\begin{enumerate}
		\item[(b)] Let $F:[a, b]\to\R$ be increasing and bounded, and let $J(x)=\sum_{n=1}^\infty\alpha_nj_n(x)$ be the jump function associated with $F$. Show that 
		$$\limsup_{h\to 0}\frac{J(x+h)-J(x)}{h}$$
		is measurable.
		\begin{proof}
		Let $J_N=\sum_{n=1}^N\alpha_n j_n(x)$. Now define 
		$$F^N_{k,m}(x)=\sup_{\frac{1}{k}<|h|<\frac{1}{m}}\abs{\frac{J_N(x+h)-J_N(x)}{h}}.$$
		As $k\to\infty$, 
		$$F^N_{k,m}\nearrow F^N_m=\sup_{0<|h|<\frac{1}{m}}\abs{\frac{J_N(x+h)-J_N(x)}{h}}, \text{and}$$
		
		\end{proof}				
		
		 
		\begin{proof}
		Since $F$ is bounded and increasing, we know it can have at most countably many discontinuities; so let $\{x_n\}_{n=1}^\infty$ denote the set of discontinuities of $F$. 
		
		\textbf{Claim:} If $x\in\{x_n\}_{n=1}^\infty$, then the desired lim sup is $\infty$, and it vanishes for all other $x$. 
		
		To see this, let $x\in\{x_n\}_{n=1}^\infty$. This means that since $x$ is a point of jump discontinuity, then $\abs{J(x+h)-J(x)}>0$ for any $h>0$ (or if not, then it works for any $h<0$), so 
		$$\limsup_{h\to 0}\frac{J(x+h)-J(x)}{h}=\infty.$$
		Now suppose that $x\not\in\{x_n\}_{n=1}^\infty$. Either $x$ is a limit point of $\{x_n\}_{n=1}^\infty$, or it is not. If it is not, then $J(x)$ is constant near $x$, so $J'(x)$ exists and is 0, which means the desired lim sup vanishes as well. 
		
		Otherwise suppose $x\not\in\{x_n\}_{n=1}^\infty$ but $x$ is a limit point of $\{x_n\}_{n=1}^\infty$. Since $F$ is bounded and $J\leq F$, then $\sum_{n=1}^\infty\alpha_nj_n(x)<\infty$ for any $x$, which means $\alpha_n\to0$. Thus for any subsequence $x_{n_i}\to x$, $J(x_{n_i})-J(x_{n_{i+1}})\to0$, so the lim sup vanishes in this case as well.		
		$$\text{THIS TOTALLY DOESN'T WORK!}$$
		This is a freshman calculus mistake, I should have known better. Just because the numerator goes to 0 doesn't mean the quotient goes to zero, since the denominator goes to 0 for sure, and thus we get 0/0 which is an indeterminate form. In fact, you can construct a counterexample. 
		\jpg{width=0.1\textwidth}{counterexample_prob_3-14-b.jpg}		
		Just let $\{x_n\}_{n=1}^\infty=\sfrac{1}{2^n}$ and $\alpha_n=x_n$. Then as $x\to 1$, $J(x)\to x$ and the desired lim sup at $x=1$ is 1. 
		\end{proof}
	\end{enumerate}

\end{enumerate}



\end{document}
