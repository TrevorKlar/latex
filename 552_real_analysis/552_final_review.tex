\documentclass[letterpaper]{article}
%\documentclass[a5paper]{article}

%% Language and font encodings
\usepackage[english]{babel}
\usepackage[utf8x]{inputenc}
\usepackage[T1]{fontenc}

%% Sets page size and margins
\usepackage[letterpaper,top=.75in,bottom=1in,left=1in,right=1in,marginparwidth=1.75cm]{geometry}
%\usepackage[a5paper,top=1cm,bottom=1cm,left=1cm,right=1.5cm,marginparwidth=1.75cm]{geometry}

\usepackage{graphicx}
%\graphicspath{../images}	  %%where to look for images

%% Useful packages
\usepackage{amssymb, amsmath, amsthm} 
%\usepackage{graphicx}  %%this is currently enabled in the default document, so it is commented out here. 
\usepackage{calrsfs}
\usepackage{braket}
\usepackage{mathtools}
\usepackage{lipsum}
\usepackage{tikz}
\usetikzlibrary{cd}
\usepackage{verbatim}
%\usepackage{ntheorem}% for theorem-like environments
\usepackage{mdframed}%can make highlighted boxes of text
%Use case: https://tex.stackexchange.com/questions/46828/how-to-highlight-important-parts-with-a-gray-background
\usepackage{wrapfig}
\usepackage{centernot}
\usepackage{subcaption}%\begin{subfigure}{0.5\textwidth}
\usepackage{pgfplots}
\pgfplotsset{compat=1.13}
\usepackage[colorinlistoftodos]{todonotes}
\usepackage[colorlinks=true, allcolors=blue]{hyperref}
\usepackage{xfrac}					%to make slanted fractions \sfrac{numerator}{denominator}
\usepackage{enumitem}            
    %syntax: \begin{enumerate}[label=(\alph*)]
    %possible arguments: f \alph*, \Alph*, \arabic*, \roman* and \Roman*
\usetikzlibrary{arrows,shapes.geometric,fit}

\DeclareMathAlphabet{\pazocal}{OMS}{zplm}{m}{n}
%% Use \pazocal{letter} to typeset a letter in the other kind 
%%  of math calligraphic font. 

%% This puts the QED block at the end of each proof, the way I like it. 
\renewenvironment{proof}{{\bfseries Proof}}{\qed}
\makeatletter
\renewenvironment{proof}[1][\bfseries \proofname]{\par
  \pushQED{\qed}%
  \normalfont \topsep6\p@\@plus6\p@\relax
  \trivlist
  %\itemindent\normalparindent
  \item[\hskip\labelsep
        \scshape
    #1\@addpunct{}]\ignorespaces
}{%
  \popQED\endtrivlist\@endpefalse
}
\makeatother

%% This adds a \rewnewtheorem command, which enables me to override the settings for theorems contained in this document.
\makeatletter
\def\renewtheorem#1{%
  \expandafter\let\csname#1\endcsname\relax
  \expandafter\let\csname c@#1\endcsname\relax
  \gdef\renewtheorem@envname{#1}
  \renewtheorem@secpar
}
\def\renewtheorem@secpar{\@ifnextchar[{\renewtheorem@numberedlike}{\renewtheorem@nonumberedlike}}
\def\renewtheorem@numberedlike[#1]#2{\newtheorem{\renewtheorem@envname}[#1]{#2}}
\def\renewtheorem@nonumberedlike#1{  
\def\renewtheorem@caption{#1}
\edef\renewtheorem@nowithin{\noexpand\newtheorem{\renewtheorem@envname}{\renewtheorem@caption}}
\renewtheorem@thirdpar
}
\def\renewtheorem@thirdpar{\@ifnextchar[{\renewtheorem@within}{\renewtheorem@nowithin}}
\def\renewtheorem@within[#1]{\renewtheorem@nowithin[#1]}
\makeatother

%% This makes theorems and definitions with names show up in bold, the way I like it. 
\makeatletter
\def\th@plain{%
  \thm@notefont{}% same as heading font
  \itshape % body font
}
\def\th@definition{%
  \thm@notefont{}% same as heading font
  \normalfont % body font
}
\makeatother

%===============================================
%==============Shortcut Commands================
%===============================================
\newcommand{\ds}{\displaystyle}
\newcommand{\B}{\mathcal{B}}
\newcommand{\C}{\mathbb{C}}
\newcommand{\F}{\mathbb{F}}
\newcommand{\N}{\mathbb{N}}
\newcommand{\R}{\mathbb{R}}
\newcommand{\Q}{\mathbb{Q}}
\newcommand{\T}{\mathcal{T}}
\newcommand{\Z}{\mathbb{Z}}
\renewcommand\qedsymbol{$\blacksquare$}
\newcommand{\qedwhite}{\hfill\ensuremath{\square}}
\newcommand*\conj[1]{\overline{#1}}
\newcommand*\closure[1]{\overline{#1}}
\newcommand*\mean[1]{\overline{#1}}
%\newcommand{\inner}[1]{\left< #1 \right>}
\newcommand{\inner}[2]{\left< #1, #2 \right>}
\newcommand{\powerset}[1]{\pazocal{P}(#1)}
%% Use \pazocal{letter} to typeset a letter in the other kind 
%%  of math calligraphic font. 
\newcommand{\cardinality}[1]{\left| #1 \right|}
\newcommand{\domain}[1]{\mathcal{D}(#1)}
\newcommand{\image}{\text{Im}}
\newcommand{\inv}[1]{#1^{-1}}
\newcommand{\preimage}[2]{#1^{-1}\left(#2\right)}
\newcommand{\script}[1]{\mathcal{#1}}


\newenvironment{highlight}{\begin{mdframed}[backgroundcolor=gray!20]}{\end{mdframed}}

\DeclarePairedDelimiter\ceil{\lceil}{\rceil}
\DeclarePairedDelimiter\floor{\lfloor}{\rfloor}

%===============================================
%===============My Tikz Commands================
%===============================================
\newcommand{\drawsquiggle}[1]{\draw[shift={(#1,0)}] (.005,.05) -- (-.005,.02) -- (.005,-.02) -- (-.005,-.05);}
\newcommand{\drawpoint}[2]{\draw[*-*] (#1,0.01) node[below, shift={(0,-.2)}] {#2};}
\newcommand{\drawopoint}[2]{\draw[o-o] (#1,0.01) node[below, shift={(0,-.2)}] {#2};}
\newcommand{\drawlpoint}[2]{\draw (#1,0.02) -- (#1,-0.02) node[below] {#2};}
\newcommand{\drawlbrack}[2]{\draw (#1+.01,0.02) --(#1,0.02) -- (#1,-0.02) -- (#1+.01,-0.02) node[below, shift={(-.01,0)}] {#2};}
\newcommand{\drawrbrack}[2]{\draw (#1-.01,0.02) --(#1,0.02) -- (#1,-0.02) -- (#1-.01,-0.02) node[below, shift={(+.01,0)}] {#2};}

%***********************************************
%**************Start of Document****************
%***********************************************

%===============================================
%===============Theorem Styles==================
%===============================================

%================Default Style==================
\theoremstyle{plain}% is the default. it sets the text in italic and adds extra space above and below the \newtheorems listed below it in the input. it is recommended for theorems, corollaries, lemmas, propositions, conjectures, criteria, and (possibly; depends on the subject area) algorithms.
\newtheorem{theorem}{Theorem}
\numberwithin{theorem}{section} %This sets the numbering system for theorems to number them down to the {argument} level. I have it set to number down to the {section} level right now.
\newtheorem*{theorem*}{Theorem} %Theorem with no numbering
\newtheorem{corollary}[theorem]{Corollary}
\newtheorem*{corollary*}{Corollary}
\newtheorem{conjecture}[theorem]{Conjecture}
\newtheorem{lemma}[theorem]{Lemma}
\newtheorem*{lemma*}{Lemma}
\newtheorem{proposition}[theorem]{Proposition}
\newtheorem*{proposition*}{Proposition}
\newtheorem{problemstatement}[theorem]{Problem Statement}


%==============Definition Style=================
\theoremstyle{definition}% adds extra space above and below, but sets the text in roman. it is recommended for definitions, conditions, problems, and examples; i've alse seen it used for exercises.
\newtheorem{definition}[theorem]{Definition}
\newtheorem*{definition*}{Definition}
\newtheorem{condition}[theorem]{Condition}
\newtheorem{problem}[theorem]{Problem}
\newtheorem{example}[theorem]{Example}
\newtheorem*{example*}{Example}
\newtheorem*{counterexample*}{Counterexample}
\newtheorem*{romantheorem*}{Theorem} %Theorem with no numbering
\newtheorem{exercise}{Exercise}
\numberwithin{exercise}{section}
\newtheorem{algorithm}[theorem]{Algorithm}

%================Remark Style===================
\theoremstyle{remark}% is set in roman, with no additional space above or below. it is recommended for remarks, notes, notation, claims, summaries, acknowledgments, cases, and conclusions.
\newtheorem{remark}[theorem]{Remark}
\newtheorem*{remark*}{Remark}
\newtheorem{notation}[theorem]{Notation}
\newtheorem*{notation*}{Notation}
%\newtheorem{claim}[theorem]{Claim}  %%use this if you ever want claims to be numbered
\newtheorem*{claim}{Claim}

 %located in ~/texmf/tex/latex/

\pgfplotsset{compat=1.13}

%\newcommand{\T}{\mathcal{T}}
%\newcommand{\B}{\mathcal{B}}

%These commands are now in tskpreamble_nothms.tex, but are left as a comment here for reference.
%\newcommand{\arbcup}[1]{\bigcup\limits_{\alpha\in\Gamma}#1_\alpha}
%\newcommand{\arbcap}[1]{\bigcap\limits_{\alpha\in\Gamma}#1_\alpha}
%\newcommand{\arbcoll}[1]{\{#1_\alpha\}_{\alpha\in\Gamma}}
%\newcommand{\arbprod}[1]{\prod\limits_{\alpha\in\Gamma}#1_\alpha}
%\newcommand{\finitecoll}[1]{#1_1, \ldots, #1_n}
%\newcommand{\finitefuncts}[2]{#1(#2_1), \ldots, #1(#2_n)}
%\newcommand{\abs}[1]{\left|#1\right|}
%\newcommand{\norm}[1]{\left|\left|#1\right|\right|}
%\newcommand{\extmeasure}[1]{m_*(#1)}
%\newcommand{\measure}[1]{m(#1)}
\renewcommand{\mathcal}[1]{\pazocal{#1}}

\title{Math 552 \linebreak
Final Review}
\author{Trevor Klar}

\begin{document}

\maketitle

\begin{enumerate}
	\item[Ch 2 Corr 3.8] Suppose $f(x)$ is a non-negative function on $\R^d$, and let
	$\script{A}=\{(x,y)\in\R^d\times\R:0\leq y\leq f(x)\}.$
	
Prove:
	\begin{enumerate}[label=(\roman*)]
		\item $f$ is measurable on $\R^d$ if and only if $\script{A}$ is measurable in $\R^{d+1}$.
		\item If the conditions in (i) hold, then
		$\ds\int_{\R^d}f(x)\dx=\measure{\script{A}}.$
	\end{enumerate}

	
	\pagebreak
	\item[ex 2.19] Suppose $f$ is integrable on $\mathbb{R}^d$. For each $\alpha > 0$, let $E_\alpha=\{x: \abs{f(x)} > \alpha \}$. Prove that $$\int_{\mathbb{R}^d} \abs{f(x)}\dx = \int_0^\infty m(E_\alpha)\der\alpha.$$
	
	\pagebreak
	
	\item[Def:]	Write the definition of a function of bounded variation. \vspace{1.5in}
	
	\item[Thm 3.3] Prove that a real-valued function $F$ on $[a, b]$ is of bounded variation if and only if $F$ is the difference of two increasing bounded functions.
	
	\pagebreak
	\item[Def:]	Write the definition of a point of Lebesgue density. \vspace{1.5in}
	
	\item[Cor 1.5] Suppose $E$ is a measurable subset of $\R^d$. 
	
	Prove:
	\begin{enumerate}
		\item Almost every $x \in E$ is a point of density of $E$.
		\item Almost every $x \not\in E$ is not a point of density of $E$.
	\end{enumerate}
	
	\pagebreak
	\item[Ex 3.3] Suppose $0$ is a point of (Lebesgue) density of the set $E \subset \R$.  Show that for each of the individual conditions below there is an infinite sequence of points $x_n \in E$, with $x_n \neq 0$, and $x_n \rightarrow 0$ as $n \rightarrow \infty$.
	\begin{enumerate}
		\item The sequence also satisfies $-x_n \in E$ for all $n.$
	\pagebreak
	\item[Ex 3.6] In one dimension there is a version of the basic inequality (1) for the maximal function in the form of an identity. We define the "one-sided" maximal function $$f_+^*= \sup_{h>0}\; \frac{1}{h} \int_x^{x+h}\abs{f(y)}dy.$$ Prove that if $E_{\alpha}^+=\{x \in \mathbb{R}: f_+^*(x) > \alpha\}$, then $$m(E_\alpha^+)=\frac{1}{\alpha}\int_{E_\alpha^+}\abs{f(y)}dy.$$
	\end{enumerate}
	
	\pagebreak
	\item[Ex 3.7] Use Corollary 1.5 to prove that if a measurable subset E of [0,1] satisfies $m(E\cap I)\geq \alpha m(I)$ for some $\alpha >0$ and all intervals in $[0,1]$, then $E$ has measure 1. 
	
	\pagebreak
	\item[Ex 3.14 (a)] Suppose $F$ is continuous on $[a,b].$ Show that 
        $$D^+(F)(x) = \limsup_{\substack{h\to0 \\ h>0}} \frac{F(x+h) - F(x)}{h}$$
        is measurable.

	\pagebreak
	\item[Ex 3.14 (b)] Let $F:[a, b]\to\R$ be increasing and bounded, and let $J(x)=\sum_{n=1}^\infty\alpha_nj_n(x)$ be the jump function associated with $F$. Show that 
        $$\limsup_{h\to 0}\frac{J(x+h)-J(x)}{h}$$
        is measurable. 
        
  \pagebreak
	\item[Ex 3.20] This exercise deals with functions $F$ that are absolutely continuous on $[a,b]$ and are increasing.  Let $A = F(a)$ and $B=F(b)$.
	\begin{enumerate}[label=(\alph*)]
		\item[(a)] There exists such an $F$ that is in addition strictly increasing, but such that $F'(x) = 0$ on a set of positive measure.
	\end{enumerate}
	
	\pagebreak
	\item[Ex 3.20] This exercise deals with functions $F$ that are absolutely continuous on $[a,b]$ and are increasing.  Let $A = F(a)$ and $B=F(b)$.
	\begin{enumerate}[label=(\alph*)]
		\item[(b)] Show that $F(x) =  \int^x_a \Chi _K(t)dt$ has a measurable subset $E \subset [A,B]$ with $m(E)=0$ and $F^{-1}(E)$ nonmeasurable. 
	\end{enumerate}
	
	\pagebreak
	\item[Ex 3.20] This exercise deals with functions $F$ that are absolutely continuous on $[a,b]$ and are increasing.  Let $A = F(a)$ and $B=F(b)$.
	\begin{enumerate}[label=(\alph*)]
		\item[(c)] Modified version solved in class: Prove that for any increasing absolutely continuous $F$ and $O$, an open subset of $[A,B]$, $m(O) = \int_F^{-1}(0) F'(x) dx$. 
	\end{enumerate}
	
	\pagebreak
	\item[Def:]	State the definition of Caratheodory measurability. 
	\vspace{1.5in}
	
	\item[Ex 6.3] Consider the exterior Lebesgue measure $m_*$ introduced in Chapter 1.  Prove that a set $E$ in $\R^d$ is Caratheodory measurable if and only if $E$ is Lebesgue measurable in the sense of Chapter 1. 
    
	
	
\end{enumerate}






\end{document}
