\documentclass[letterpaper]{article}
%\documentclass[a5paper]{article}

%% Language and font encodings
\usepackage[english]{babel}
\usepackage[utf8x]{inputenc}
\usepackage[T1]{fontenc}


%% Sets page size and margins
\usepackage[letterpaper,top=.75in,bottom=1in,left=1in,right=1in,marginparwidth=1.75cm]{geometry}
%\usepackage[a5paper,top=1cm,bottom=1cm,left=1cm,right=1.5cm,marginparwidth=1.75cm]{geometry}

\usepackage{graphicx}
%\graphicspath{../images}	  %%where to look for images

%% Useful packages
\usepackage{amssymb, amsmath, amsthm} 
%\usepackage{graphicx}  %%this is currently enabled in the default document, so it is commented out here. 
\usepackage{calrsfs}
\usepackage{braket}
\usepackage{mathtools}
\usepackage{lipsum}
\usepackage{tikz}
\usetikzlibrary{cd}
\usepackage{verbatim}
%\usepackage{ntheorem}% for theorem-like environments
\usepackage{mdframed}%can make highlighted boxes of text
%Use case: https://tex.stackexchange.com/questions/46828/how-to-highlight-important-parts-with-a-gray-background
\usepackage{wrapfig}
\usepackage{centernot}
\usepackage{subcaption}%\begin{subfigure}{0.5\textwidth}
\usepackage{pgfplots}
\pgfplotsset{compat=1.13}
\usepackage[colorinlistoftodos]{todonotes}
\usepackage[colorlinks=true, allcolors=blue]{hyperref}
\usepackage{xfrac}					%to make slanted fractions \sfrac{numerator}{denominator}
\usepackage{enumitem}            
    %syntax: \begin{enumerate}[label=(\alph*)]
    %possible arguments: f \alph*, \Alph*, \arabic*, \roman* and \Roman*
\usetikzlibrary{arrows,shapes.geometric,fit}

\DeclareMathAlphabet{\pazocal}{OMS}{zplm}{m}{n}
%% Use \pazocal{letter} to typeset a letter in the other kind 
%%  of math calligraphic font. 

%% This puts the QED block at the end of each proof, the way I like it. 
\renewenvironment{proof}{{\bfseries Proof}}{\qed}
\makeatletter
\renewenvironment{proof}[1][\bfseries \proofname]{\par
  \pushQED{\qed}%
  \normalfont \topsep6\p@\@plus6\p@\relax
  \trivlist
  %\itemindent\normalparindent
  \item[\hskip\labelsep
        \scshape
    #1\@addpunct{}]\ignorespaces
}{%
  \popQED\endtrivlist\@endpefalse
}
\makeatother

%% This adds a \rewnewtheorem command, which enables me to override the settings for theorems contained in this document.
\makeatletter
\def\renewtheorem#1{%
  \expandafter\let\csname#1\endcsname\relax
  \expandafter\let\csname c@#1\endcsname\relax
  \gdef\renewtheorem@envname{#1}
  \renewtheorem@secpar
}
\def\renewtheorem@secpar{\@ifnextchar[{\renewtheorem@numberedlike}{\renewtheorem@nonumberedlike}}
\def\renewtheorem@numberedlike[#1]#2{\newtheorem{\renewtheorem@envname}[#1]{#2}}
\def\renewtheorem@nonumberedlike#1{  
\def\renewtheorem@caption{#1}
\edef\renewtheorem@nowithin{\noexpand\newtheorem{\renewtheorem@envname}{\renewtheorem@caption}}
\renewtheorem@thirdpar
}
\def\renewtheorem@thirdpar{\@ifnextchar[{\renewtheorem@within}{\renewtheorem@nowithin}}
\def\renewtheorem@within[#1]{\renewtheorem@nowithin[#1]}
\makeatother

%% This makes theorems and definitions with names show up in bold, the way I like it. 
\makeatletter
\def\th@plain{%
  \thm@notefont{}% same as heading font
  \itshape % body font
}
\def\th@definition{%
  \thm@notefont{}% same as heading font
  \normalfont % body font
}
\makeatother

%===============================================
%==============Shortcut Commands================
%===============================================
\newcommand{\ds}{\displaystyle}
\newcommand{\B}{\mathcal{B}}
\newcommand{\C}{\mathbb{C}}
\newcommand{\F}{\mathbb{F}}
\newcommand{\N}{\mathbb{N}}
\newcommand{\R}{\mathbb{R}}
\newcommand{\Q}{\mathbb{Q}}
\newcommand{\T}{\mathcal{T}}
\newcommand{\Z}{\mathbb{Z}}
\renewcommand\qedsymbol{$\blacksquare$}
\newcommand{\qedwhite}{\hfill\ensuremath{\square}}
\newcommand*\conj[1]{\overline{#1}}
\newcommand*\closure[1]{\overline{#1}}
\newcommand*\mean[1]{\overline{#1}}
%\newcommand{\inner}[1]{\left< #1 \right>}
\newcommand{\inner}[2]{\left< #1, #2 \right>}
\newcommand{\powerset}[1]{\pazocal{P}(#1)}
%% Use \pazocal{letter} to typeset a letter in the other kind 
%%  of math calligraphic font. 
\newcommand{\cardinality}[1]{\left| #1 \right|}
\newcommand{\domain}[1]{\mathcal{D}(#1)}
\newcommand{\image}{\text{Im}}
\newcommand{\inv}[1]{#1^{-1}}
\newcommand{\preimage}[2]{#1^{-1}\left(#2\right)}
\newcommand{\script}[1]{\mathcal{#1}}


\newenvironment{highlight}{\begin{mdframed}[backgroundcolor=gray!20]}{\end{mdframed}}

\DeclarePairedDelimiter\ceil{\lceil}{\rceil}
\DeclarePairedDelimiter\floor{\lfloor}{\rfloor}

%===============================================
%===============My Tikz Commands================
%===============================================
\newcommand{\drawsquiggle}[1]{\draw[shift={(#1,0)}] (.005,.05) -- (-.005,.02) -- (.005,-.02) -- (-.005,-.05);}
\newcommand{\drawpoint}[2]{\draw[*-*] (#1,0.01) node[below, shift={(0,-.2)}] {#2};}
\newcommand{\drawopoint}[2]{\draw[o-o] (#1,0.01) node[below, shift={(0,-.2)}] {#2};}
\newcommand{\drawlpoint}[2]{\draw (#1,0.02) -- (#1,-0.02) node[below] {#2};}
\newcommand{\drawlbrack}[2]{\draw (#1+.01,0.02) --(#1,0.02) -- (#1,-0.02) -- (#1+.01,-0.02) node[below, shift={(-.01,0)}] {#2};}
\newcommand{\drawrbrack}[2]{\draw (#1-.01,0.02) --(#1,0.02) -- (#1,-0.02) -- (#1-.01,-0.02) node[below, shift={(+.01,0)}] {#2};}

%***********************************************
%**************Start of Document****************
%***********************************************

%===============================================
%===============Theorem Styles==================
%===============================================

%================Default Style==================
\theoremstyle{plain}% is the default. it sets the text in italic and adds extra space above and below the \newtheorems listed below it in the input. it is recommended for theorems, corollaries, lemmas, propositions, conjectures, criteria, and (possibly; depends on the subject area) algorithms.
\newtheorem{theorem}{Theorem}
\numberwithin{theorem}{section} %This sets the numbering system for theorems to number them down to the {argument} level. I have it set to number down to the {section} level right now.
\newtheorem*{theorem*}{Theorem} %Theorem with no numbering
\newtheorem{corollary}[theorem]{Corollary}
\newtheorem*{corollary*}{Corollary}
\newtheorem{conjecture}[theorem]{Conjecture}
\newtheorem{lemma}[theorem]{Lemma}
\newtheorem*{lemma*}{Lemma}
\newtheorem{proposition}[theorem]{Proposition}
\newtheorem*{proposition*}{Proposition}
\newtheorem{problemstatement}[theorem]{Problem Statement}


%==============Definition Style=================
\theoremstyle{definition}% adds extra space above and below, but sets the text in roman. it is recommended for definitions, conditions, problems, and examples; i've alse seen it used for exercises.
\newtheorem{definition}[theorem]{Definition}
\newtheorem*{definition*}{Definition}
\newtheorem{condition}[theorem]{Condition}
\newtheorem{problem}[theorem]{Problem}
\newtheorem{example}[theorem]{Example}
\newtheorem*{example*}{Example}
\newtheorem*{counterexample*}{Counterexample}
\newtheorem*{romantheorem*}{Theorem} %Theorem with no numbering
\newtheorem{exercise}{Exercise}
\numberwithin{exercise}{section}
\newtheorem{algorithm}[theorem]{Algorithm}

%================Remark Style===================
\theoremstyle{remark}% is set in roman, with no additional space above or below. it is recommended for remarks, notes, notation, claims, summaries, acknowledgments, cases, and conclusions.
\newtheorem{remark}[theorem]{Remark}
\newtheorem*{remark*}{Remark}
\newtheorem{notation}[theorem]{Notation}
\newtheorem*{notation*}{Notation}
%\newtheorem{claim}[theorem]{Claim}  %%use this if you ever want claims to be numbered
\newtheorem*{claim}{Claim}



\pgfplotsset{compat=1.13}

%\newcommand{\T}{\mathcal{T}}
%\newcommand{\B}{\mathcal{B}}

%These commands are now in tskpreamble_nothms.tex, but are left as a comment here for reference. 
%\newcommand{\arbcup}[1]{\bigcup\limits_{\alpha\in\Gamma}#1_\alpha}
%\newcommand{\arbcap}[1]{\bigcap\limits_{\alpha\in\Gamma}#1_\alpha}
%\newcommand{\arbcoll}[1]{\{#1_\alpha\}_{\alpha\in\Gamma}}
%\newcommand{\arbprod}[1]{\prod\limits_{\alpha\in\Gamma}#1_\alpha}
%\newcommand{\finitecoll}[1]{#1_1, \ldots, #1_n}
%\newcommand{\finitefuncts}[2]{#1(#2_1), \ldots, #1(#2_n)}
%\newcommand{\abs}[1]{\left|#1\right|}
%\newcommand{\norm}[1]{\left|\left|#1\right|\right|}

\title{Math 550 \linebreak
Homework 10}
\author{Trevor Klar}

\begin{document}

\maketitle

\begin{enumerate}
\item For a vector field $X = (f_x, f_y)$ on $\R^2$, we may define as associated 1-form, different from the one in class, by 
$$\star\omega^1_X=-f_y\dx+f_x\dy.$$
We may also define 
$$\text{div} X = \frac{\del f_x}{\del x}+\frac{\del f_y}{\del y}.$$
	\begin{enumerate}
	\item Let $M$ be a compact 2-dimensional manifold with boundary in $\R^3$. Show that for all points $p\in \del M$, the equation $\star\omega^1_X = X \cdot n \der s$ holds. 
	\begin{proof}
	Let $p\in \del M$, and let $v\in \left(\del M\right)_p$. Then 
		\[\def\arraystretch{1.05}
		\begin{array}{rcl}
		\star\omega^1_X(p)(v)&=&\big(-f_y(p)\dx+f_x(p)\dy\big)(v)\\
		&=&\det 
			\left[\begin{array}{cc}
			|&|\\
			X_p & v\\
			|&|\\
			\end{array}\right]
		\end{array}\]
	Now $X_p$ can be written as $w+\inner{X_p}{N_p}N_p$, where $w\in(\del M)_p$ and $N_p$ is the unit outward normal vector of $M$ at $p$. Thus we have 
	\[\def\arraystretch{1.05}
	\begin{array}{rcl}
	\det 
			\left[\begin{array}{cc}
			|&|\\
			X_p & v\\
			|&|\\
			\end{array}\right]
	&=&
	\det 
			\left[\begin{array}{cc}
			|&|\\
			w+\inner{X_p}{N_p}N_p & v\\
			|&|\\
			\end{array}\right]
	\\
  &=&
	\det 
			\left[\begin{array}{cc}
			|&|\\
			w & v\\
			|&|\\
			\end{array}\right]
	+
	\det 
			\left[\begin{array}{cc}
			|&|\\
			\inner{X_p}{N_p}N_p & v\\
			|&|\\
			\end{array}\right]
	\\
	&=& 0 + \inner{X_p}{N_p}
	\det 
			\left[\begin{array}{cc}
			|&|\\
			N_p & v\\
			|&|\\
			\end{array}\right]
	\\
	&=&\inner{X_p}{N_p}\der s
	\end{array}\]
	and we are done. 
	\end{proof}
	
	\item Prove the following \emph{Divergence form of Green's Theorem:}	Let $M$ be a 2-dimensional manifold with boundary in $\R^2$, and let $X$ be a vector field on $M$. Then 
	$$\int_{M}\text{div}\,X\der A = \int_{\del M}\inner{X}{n}\der s.$$
	\begin{proof}
	Since the differential of the RHS is
	\[\begin{array}{rcl}
	\der\left(\inner{X}{n}\der s\right) &=& \der\big(-f_y(p)\dx+f_x(p)\dy\big)\\
	&=&-\dfrac{\del f_y}{\del y}\dy\wedge\dx+\dfrac{\del f_x}{\del x}\dx\wedge\dy\\
	&=&\left(\dfrac{\del f_y}{\del y}+\dfrac{\del f_x}{\del x}\right)\dx\wedge\dy\\
	&=&\text{div}\,X\der A,
	\end{array}\]
	then by Stokes' Theorem, 
	$$\int_{\del M}\inner{X}{n}\der s = \int_{M}\der\left(\inner{X}{n}\der s\right) = \int_{M}\text{div}\,X\der A,$$
	and we are done. 
	\end{proof}
	\end{enumerate}

\item Let $M$ be a compact 3-dimensional manifold with boundary in $\R^3$, with $\vec{0}\in M-\del M$. Consider the vector field $X(p)=\frac{p}{\norm{p}^3}$ defined on $\R^3-\vec{0}$. Prove that 
$$\int_{\del M}\inner{X}{N}\der A = 4\pi.$$
\begin{proof}
Define a manifold $M'=M-B_\epsilon(\vec{0})$. We will integrate over $M'$ to find the integral over $M$. Note that $\del M' = \del M \cup S^2_\epsilon$, where $S^2_\epsilon$ is a sphere of radius $\epsilon$. This means that 
$$\int_{\del M'}\inner{X}{N}\der A = \int_{\del M}\inner{X}{N}\der A - \int_{S^2_\epsilon}\inner{X}{N}\der A.$$
By the Divergence form of Green's Theorem, the LHS is $\int_{M'}\text{div}\,X\der A$, and a straightforward calculation will show that $\text{div}\,X=0$. Thus we find that 
\[\begin{array}{rcl}
0 &=& \displaystyle\int_{\del M}\inner{X}{N}\der A - \displaystyle\int_{S^2_\epsilon}\inner{X}{N}\der A\\
&=& \displaystyle\int_{\del M}\inner{X}{N}\der A - 4\pi
\end{array}\]
And we are done. 

\begin{comment}
\[\begin{array}{rcl}
\displaystyle\int_{\del M}\inner{X}{N}\der A &=&\displaystyle\int_{\del M}\omega^2_X\\[2ex]
&=&\displaystyle\int_{\del M}\dfrac{x\dy\wedge\dz-y\dx\wedge\dz+z\dx\wedge\dy}{(x^2+y^2+z^2)^{3/2}}
\end{array}\]
?????????
\end{comment}

\end{proof}

\item 
	\begin{enumerate}
	\item Show that if $X$ is a vector field on $\R^3$ with $\curl X=0$, then $X =\grad F$ for some function $F:\R^3\to\R$. 
	

	\begin{proof} Let $X=(f_x, f_y, f_z)$. Then $\omega^1_X=f_x\dx + f_y\dy + f_z\dz$. So, 
	\[
	\der(\omega^1_X)=\omega^2_{\curl X}
	=0, \]
	since $\curl X=0$. Thus, $\omega^1_X$ is exact by Poinacar\'e's Lemma. Therefore there is some function $F$ such that $\der F=\omega^1_X$, and since $\der F = \omega^1_{\grad F}$, then $\omega^1_X = \omega^1_{\grad F}$. So 
	\[f_x\dx+f_y\dy+f_z\dz=\frac{\del F}{\del x}\dx + \frac{\del F}{\del y}\dy + \frac{\del F}{\del z}\dz\]
	and since $\dx, \dy, \dz$ are linearly independent, then we can equate the coefficients, and we are done. 
	

		\begin{comment}		
	Notate $X=(f_1, f_2, f_3)$. Suppose that $\curl X =\vec{0}$, that is, 
	\[\left|\begin{array}{ccc}
	\vec{i} & \vec{j} & \vec{k}\\
	\frac{\del}{\del x} & \frac{\del}{\del y} & \frac{\del}{\del z} \\
	f_1&f_2&f_3	
	\end{array}\right|
  =0	
	\]
	This gives the following system:
	\[\begin{array}{c}
	\frac{\del f_3}{\del y}=\frac{\del f_2}{\del z}\\[.75ex]
	\frac{\del f_1}{\del z}=\frac{\del f_3}{\del x}\\[.75ex]
	\frac{\del f_2}{\del x}=\frac{\del f_1}{\del y}\\
	\end{array}\]
  Now we take indefinite integrals, 
  \[\begin{array}{rcl}
  \int f_1 \dx &=& F_1+C_1\\
  \int f_2 \dy &=& F_2+C_2\\
  \int f_3 \dz &=& F_3+C_3\\
  \end{array}\]\footnote{We will take these constants to be 0, since we seek a function which we will differentiate, and the constants will map to zero.}
  Each $F_i$ is a well-defined function on $\R^3$, so we write the terms of each $F_i$ as
  \[\begin{array}{rcl}
  F_1&=&F_1(x)+F_1(x,y)+F_1(x,z)+F_1(x,y,z)\\
  F_2&=&F_2(y)+F_2(x,y)+F_2(y,z)+F_2(x,y,z)\\
  F_3&=&F_3(z)+F_3(x,z)+F_3(y,z)+F_3(x,y,z)\\
  \end{array}\]
  Where $F_1(x)$ is only a function of $x$, $F_1(x,y)$ is only a function of $x$ and $y$, and so on.\footnote{This idea can be made more rigorous with kernels of partial derivatives; $F_1(x)$ is the part of $F_1$ that is in both the kernel of $\del y$ and $\del z$, etc.}  
  
  
  Now we observe that $F_1(x,y)=F_2(x,y)$ and so on, with each  pair of components having matching variables being equal. 
  \[\begin{array}{rcl}
  \dfrac{\del f_1}{\del y}&=&\dfrac{\del^2 F_1 }{\del y\del x}\\
  &=&\frac{\del^2}{\del y\del x}[F_1(x)+F_1(x,y)+F_1(x,z)+F_1(x,y,z)]\\
  &=&\frac{\del}{\del y}[f_1(x)+f_1(x,y)+f_1(x,z)+f_1(x,y,z)]\\
  &=&\frac{\del f_1(x,y)}{\del y}+\frac{\del f_1(x,y,z)}{\del y}\\
  &\text{and}\\
  \dfrac{\del f_2}{\del x}&=&\dfrac{\del^2 F_2 }{\del x\del y}\\
  &=&\frac{\del^2}{\del x\del y}[F_2(y)+F_2(x,y)+F_2(y,z)+F_2(x,y,z)]\\
  &=&\frac{\del}{\del x}[f_2(y)+f_2(x,y)+f_2(y,z)+f_2(x,y,z)]\\
  &=&\frac{\del f_2(x,y)}{\del x}+\frac{\del f_2(x,y,z)}{\del x}\\
  \end{array}\]
  
  Umm.. Crud this didn't exactly work the way I first thought it would. I wanted this to show that $F_1(x,y)+F_1(x,y,z)=F_2(x,y)+F_2(x,y,z)$. Anyways suppose I had shown this. 
  
  Then we define 
  $$F=F_1(x)+F_2(y)+F_3(z)+F_1(x,y)+F_1(x,z)+F_2(y,z)+F_1(x,y,z),$$
  that is, we "union" the terms of $F_1$, $F_2$, and $F_3$. Now let's confirm that this is the desired function:
  \[\begin{array}{rcl}
  \dfrac{\del F}{\del y}&=&\dfrac{\del}{\del y}[F_1(x)+F_2(y)+F_3(z)+F_1(x,y)+F_1(x,z)+F_2(y,z)+F_1(x,y,z)]\\
  &=&0+f_2(y)+0+f_1(x,y)+0+f_2(y,z)+f_1(x,y,z)\\
  &=&0+f_2(y)+0+f_2(x,y)+0+f_2(y,z)+f_2(x,y,z)\\
  &=&f_2\\
  \end{array}\]
  and a similar computation works for $f_1$ and $f_3$ as well. Thus $\nabla F=(f_1, f_2, f_3)$ and we are done. 
\end{comment}
  
  \begin{comment}
  For $f_1, f_2$, we can write $f_2=\phi_2+g_2(y,z)$, $f_1=\phi_1+g_1(x,z)$, where each $g_i$ is only a function of the stated variables. Then we take double integrals:
	\[\begin{array}{rcl}
		\iint\frac{\del f_2}{\del x}\dx\dy&=&\iint\frac{\del f_1}{\del y}\dx\dy\\
		\iint\frac{\del (\phi_2+g_2)}{\del x}\dx\dy&=&\iint\frac{\del (\phi_1+g_1)}{\del y}\dx\dy\\
		\iint\frac{\del (\phi_2)}{\del x}\dx\dy&=&\iint\frac{\del (\phi_1)}{\del y}\dx\dy\\
		\int\phi_2\dy&=&\int\phi_1\dx\\
		\\
		\iint\frac{\del f_2}{\del x}\dx\dy&=&\iint\frac{\del f_1}{\del y}\dy\dx\\
		\int f_2+g_2(y,z)\dy&=&\int f_1 + g_1(x,z)\dx \\
		F_2+G_2(y)&=&F_1+G_1(x)\\
	
		
	
	\end{array}\]
	
  \end{comment}
	
	\end{proof}
	\item Show that if $X$ is a vector field on $\R^3$ with $\diverg X=0$, then $X =\curl Y$ for some vector field $Y$ on $\R^3$. 
	\begin{proof}
	Since $\diverg X=0$, then $(\diverg X)\dx\wedge\dy\wedge\dz=\der(\omega^2_X)=0$, so $\omega^2_X$ is exact. Then there is a one form $\eta$ such that $\der\eta=\omega^2_X$. Now $\eta=n_1\dx+n_2\dy+n_3\dz$ can be written as $\eta=\omega^1_Y$, where $Y=(n_1, n_2, n_3)$. Thus, 
	\[\omega^2_X=\der \eta=\der(\omega^1_Y)=\omega^2_{\curl Y},\]
	so $X=\curl Y$, by the linear independence argument of problem 3a. 
	\end{proof}
	
	\end{enumerate}
	
\item Let $\omega=\frac{-y}{x^2+y^2}	\dx + \frac{x}{x^2+y^2}	\dy $ be a 1-form on $\R^2-\vec{0}$. Prove that $\omega$ does not extend to a 1-form on $\R^n$. 
\begin{proof}
Recall that we showed on a previous homework that $\omega$ is closed but not exact. If $\omega$ did extend to a 1-form on $\R^n$, then the extension would have to be given by the same formula, defined on all $\R^n$. So, it would be exact by Poincar\'e's Lemma, but it is not. 
\end{proof} 
	
\end{enumerate}
\end{document}
