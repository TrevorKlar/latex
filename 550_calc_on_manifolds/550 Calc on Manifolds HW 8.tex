\documentclass[letterpaper]{article}
%\documentclass[a5paper]{article}

%% Language and font encodings
\usepackage[english]{babel}
\usepackage[utf8x]{inputenc}
\usepackage[T1]{fontenc}


%% Sets page size and margins
\usepackage[letterpaper,top=.75in,bottom=1in,left=1in,right=1in,marginparwidth=1.75cm]{geometry}
%\usepackage[a5paper,top=1cm,bottom=1cm,left=1cm,right=1.5cm,marginparwidth=1.75cm]{geometry}

\usepackage{graphicx}
%\graphicspath{../images}	  %%where to look for images

%% Useful packages
\usepackage{amssymb, amsmath, amsthm} 
%\usepackage{graphicx}  %%this is currently enabled in the default document, so it is commented out here. 
\usepackage{calrsfs}
\usepackage{braket}
\usepackage{mathtools}
\usepackage{lipsum}
\usepackage{tikz}
\usetikzlibrary{cd}
\usepackage{verbatim}
%\usepackage{ntheorem}% for theorem-like environments
\usepackage{mdframed}%can make highlighted boxes of text
%Use case: https://tex.stackexchange.com/questions/46828/how-to-highlight-important-parts-with-a-gray-background
\usepackage{wrapfig}
\usepackage{centernot}
\usepackage{subcaption}%\begin{subfigure}{0.5\textwidth}
\usepackage{pgfplots}
\pgfplotsset{compat=1.13}
\usepackage[colorinlistoftodos]{todonotes}
\usepackage[colorlinks=true, allcolors=blue]{hyperref}
\usepackage{xfrac}					%to make slanted fractions \sfrac{numerator}{denominator}
\usepackage{enumitem}            
    %syntax: \begin{enumerate}[label=(\alph*)]
    %possible arguments: f \alph*, \Alph*, \arabic*, \roman* and \Roman*
\usetikzlibrary{arrows,shapes.geometric,fit}

\DeclareMathAlphabet{\pazocal}{OMS}{zplm}{m}{n}
%% Use \pazocal{letter} to typeset a letter in the other kind 
%%  of math calligraphic font. 

%% This puts the QED block at the end of each proof, the way I like it. 
\renewenvironment{proof}{{\bfseries Proof}}{\qed}
\makeatletter
\renewenvironment{proof}[1][\bfseries \proofname]{\par
  \pushQED{\qed}%
  \normalfont \topsep6\p@\@plus6\p@\relax
  \trivlist
  %\itemindent\normalparindent
  \item[\hskip\labelsep
        \scshape
    #1\@addpunct{}]\ignorespaces
}{%
  \popQED\endtrivlist\@endpefalse
}
\makeatother

%% This adds a \rewnewtheorem command, which enables me to override the settings for theorems contained in this document.
\makeatletter
\def\renewtheorem#1{%
  \expandafter\let\csname#1\endcsname\relax
  \expandafter\let\csname c@#1\endcsname\relax
  \gdef\renewtheorem@envname{#1}
  \renewtheorem@secpar
}
\def\renewtheorem@secpar{\@ifnextchar[{\renewtheorem@numberedlike}{\renewtheorem@nonumberedlike}}
\def\renewtheorem@numberedlike[#1]#2{\newtheorem{\renewtheorem@envname}[#1]{#2}}
\def\renewtheorem@nonumberedlike#1{  
\def\renewtheorem@caption{#1}
\edef\renewtheorem@nowithin{\noexpand\newtheorem{\renewtheorem@envname}{\renewtheorem@caption}}
\renewtheorem@thirdpar
}
\def\renewtheorem@thirdpar{\@ifnextchar[{\renewtheorem@within}{\renewtheorem@nowithin}}
\def\renewtheorem@within[#1]{\renewtheorem@nowithin[#1]}
\makeatother

%% This makes theorems and definitions with names show up in bold, the way I like it. 
\makeatletter
\def\th@plain{%
  \thm@notefont{}% same as heading font
  \itshape % body font
}
\def\th@definition{%
  \thm@notefont{}% same as heading font
  \normalfont % body font
}
\makeatother

%===============================================
%==============Shortcut Commands================
%===============================================
\newcommand{\ds}{\displaystyle}
\newcommand{\B}{\mathcal{B}}
\newcommand{\C}{\mathbb{C}}
\newcommand{\F}{\mathbb{F}}
\newcommand{\N}{\mathbb{N}}
\newcommand{\R}{\mathbb{R}}
\newcommand{\Q}{\mathbb{Q}}
\newcommand{\T}{\mathcal{T}}
\newcommand{\Z}{\mathbb{Z}}
\renewcommand\qedsymbol{$\blacksquare$}
\newcommand{\qedwhite}{\hfill\ensuremath{\square}}
\newcommand*\conj[1]{\overline{#1}}
\newcommand*\closure[1]{\overline{#1}}
\newcommand*\mean[1]{\overline{#1}}
%\newcommand{\inner}[1]{\left< #1 \right>}
\newcommand{\inner}[2]{\left< #1, #2 \right>}
\newcommand{\powerset}[1]{\pazocal{P}(#1)}
%% Use \pazocal{letter} to typeset a letter in the other kind 
%%  of math calligraphic font. 
\newcommand{\cardinality}[1]{\left| #1 \right|}
\newcommand{\domain}[1]{\mathcal{D}(#1)}
\newcommand{\image}{\text{Im}}
\newcommand{\inv}[1]{#1^{-1}}
\newcommand{\preimage}[2]{#1^{-1}\left(#2\right)}
\newcommand{\script}[1]{\mathcal{#1}}


\newenvironment{highlight}{\begin{mdframed}[backgroundcolor=gray!20]}{\end{mdframed}}

\DeclarePairedDelimiter\ceil{\lceil}{\rceil}
\DeclarePairedDelimiter\floor{\lfloor}{\rfloor}

%===============================================
%===============My Tikz Commands================
%===============================================
\newcommand{\drawsquiggle}[1]{\draw[shift={(#1,0)}] (.005,.05) -- (-.005,.02) -- (.005,-.02) -- (-.005,-.05);}
\newcommand{\drawpoint}[2]{\draw[*-*] (#1,0.01) node[below, shift={(0,-.2)}] {#2};}
\newcommand{\drawopoint}[2]{\draw[o-o] (#1,0.01) node[below, shift={(0,-.2)}] {#2};}
\newcommand{\drawlpoint}[2]{\draw (#1,0.02) -- (#1,-0.02) node[below] {#2};}
\newcommand{\drawlbrack}[2]{\draw (#1+.01,0.02) --(#1,0.02) -- (#1,-0.02) -- (#1+.01,-0.02) node[below, shift={(-.01,0)}] {#2};}
\newcommand{\drawrbrack}[2]{\draw (#1-.01,0.02) --(#1,0.02) -- (#1,-0.02) -- (#1-.01,-0.02) node[below, shift={(+.01,0)}] {#2};}

%***********************************************
%**************Start of Document****************
%***********************************************

%===============================================
%===============Theorem Styles==================
%===============================================

%================Default Style==================
\theoremstyle{plain}% is the default. it sets the text in italic and adds extra space above and below the \newtheorems listed below it in the input. it is recommended for theorems, corollaries, lemmas, propositions, conjectures, criteria, and (possibly; depends on the subject area) algorithms.
\newtheorem{theorem}{Theorem}
\numberwithin{theorem}{section} %This sets the numbering system for theorems to number them down to the {argument} level. I have it set to number down to the {section} level right now.
\newtheorem*{theorem*}{Theorem} %Theorem with no numbering
\newtheorem{corollary}[theorem]{Corollary}
\newtheorem*{corollary*}{Corollary}
\newtheorem{conjecture}[theorem]{Conjecture}
\newtheorem{lemma}[theorem]{Lemma}
\newtheorem*{lemma*}{Lemma}
\newtheorem{proposition}[theorem]{Proposition}
\newtheorem*{proposition*}{Proposition}
\newtheorem{problemstatement}[theorem]{Problem Statement}


%==============Definition Style=================
\theoremstyle{definition}% adds extra space above and below, but sets the text in roman. it is recommended for definitions, conditions, problems, and examples; i've alse seen it used for exercises.
\newtheorem{definition}[theorem]{Definition}
\newtheorem*{definition*}{Definition}
\newtheorem{condition}[theorem]{Condition}
\newtheorem{problem}[theorem]{Problem}
\newtheorem{example}[theorem]{Example}
\newtheorem*{example*}{Example}
\newtheorem*{counterexample*}{Counterexample}
\newtheorem*{romantheorem*}{Theorem} %Theorem with no numbering
\newtheorem{exercise}{Exercise}
\numberwithin{exercise}{section}
\newtheorem{algorithm}[theorem]{Algorithm}

%================Remark Style===================
\theoremstyle{remark}% is set in roman, with no additional space above or below. it is recommended for remarks, notes, notation, claims, summaries, acknowledgments, cases, and conclusions.
\newtheorem{remark}[theorem]{Remark}
\newtheorem*{remark*}{Remark}
\newtheorem{notation}[theorem]{Notation}
\newtheorem*{notation*}{Notation}
%\newtheorem{claim}[theorem]{Claim}  %%use this if you ever want claims to be numbered
\newtheorem*{claim}{Claim}



\pgfplotsset{compat=1.13}

%\newcommand{\T}{\mathcal{T}}
%\newcommand{\B}{\mathcal{B}}

%These commands are now in tskpreamble_nothms.tex, but are left as a comment here for reference. 
%\newcommand{\arbcup}[1]{\bigcup\limits_{\alpha\in\Gamma}#1_\alpha}
%\newcommand{\arbcap}[1]{\bigcap\limits_{\alpha\in\Gamma}#1_\alpha}
%\newcommand{\arbcoll}[1]{\{#1_\alpha\}_{\alpha\in\Gamma}}
%\newcommand{\arbprod}[1]{\prod\limits_{\alpha\in\Gamma}#1_\alpha}
%\newcommand{\finitecoll}[1]{#1_1, \ldots, #1_n}
%\newcommand{\finitefuncts}[2]{#1(#2_1), \ldots, #1(#2_n)}
%\newcommand{\abs}[1]{\left|#1\right|}
%\newcommand{\norm}[1]{\left|\left|#1\right|\right|}

\title{Math 550 \linebreak
Homework 8}
\author{Trevor Klar}

\begin{document}

\maketitle

\begin{enumerate}\setcounter{enumi}{1}
	\item Let $C$ be the intersection of the sphere $x^2+y^2+z^2=1$ and the plane $x+y+z=0$, oriented counterclockwise as viewed from above the $xy$-plane. Use Stokes' Theorem to evaluate 
	$$\int_Cz^3\dx.$$
	\textbf{Answer} Let $\tilde{C}$ be the part of the given plane which is bounded by $C$. Then since $C=\del\tilde{C}$, 
	$$\int_C z^3\dx=\int_{\tilde{C}}\der(z^3\dx)=\int_{\tilde{C}}3z^2\dz\wedge\dx=\int_{\tilde{C}}-3z^2\dx\wedge\dz$$
	%Now, if I could only find a freaking parameterization for the hemisphere or the disc, I would pull it back and integrate! But I can't think of one...
	Now we parameterize the disc $\tilde{C}$. Let 
	\[g(r,\theta)=\left(
	\begin{array}{c}
	  ar\cos\theta+abr\sin\theta,\\
	  ar\cos\theta-abr\sin\theta,\\
	  -2ar\cos\theta
	\end{array}\right), \text{ where }a=\frac{1}{\sqrt{2}}, \, \, b=\frac{1}{\sqrt{3}}\]
	To see that $g$ parameterizes $\tilde{C}$\footnote{In checking my work afterwards, I realized that my parameterization for $g_2$ has the wrong sign. This changes everything, so the rest of the work is based on a faulty parameterization. Have mercy on my soul!}, first note that the boundary $C$ is can be found by solving the system of equations, and one will find the solution set is given by $2x^2+2xy+2y^2=1$, which is an ellipse oriented diagonally. Now Observe that $g_1$ and $g_2$ are given by rotation $\rho_{\pi/4}$ composed with the parameterization in polar coordinates for an ellipse with half-width 1 and half-height $1/\sqrt{3}$:
	\[\left[\begin{array}{c}
	g_1\\g_2
	\end{array}\right]
	=
	\frac{1}{\sqrt{2}}\left[\begin{array}{cc}
	1&1\\
	-1&1
	\end{array}\right]
	\left[\begin{array}{c}
	r\cos\theta\\
	\frac{r}{\sqrt{3}}\sin\theta
  \end{array}\right]		
	\]
	and $g_3$ is given by $-g_1-g_2$. Now that we have a parameterization of $\tilde{C}$, we calculate the pullback.
	\[\begin{array}{rcl}
		\int_C z^3\dx&=&\int_{\tilde{C}}-3z^2\dx\wedge\dz\\
		&=&\int_{\preimage{g}{\tilde{C}}}g^*(-3z^2)\dx\wedge\dz\\
	\end{array}\]
	To compute the pullback, we calculate
		\[\begin{array}{rcl}
			g^*(-3z^2)&=&-6r^2\cos^2\theta\\
			g^*{\dx}&=&(a\cos\theta+ab\sin\theta)\der r+(-ar\sin\theta+abr\cos\theta)\der\theta\\
			g^*{\dz}&=&(a\cos\theta-ab\sin\theta)\der r+(-ar\sin\theta-abr\cos\theta)\der\theta\\
		\end{array}	\]
		Thus after quite some simplifying we find that 
		$$g^*(-3z^2)\dx\wedge\dz=4\sqrt{3}r^3\cos^2\theta\der r \wedge\der\theta.$$
		Then we integrate and obtain $\int_0^{2\pi}\int_0^1 4\sqrt{3}r^3\cos^2\theta\der r\der\theta=\sqrt3\pi$. \qed
	
	\item Show that 
	$$\omega = \frac{x\dy\wedge\dz-y\dx\wedge\dz+z\dx\wedge\dy}{(x^2+y^2+z^2)^{3/2}}$$
	is closed, but not exact. 
	\begin{proof}
		(Closed) To reduce notation, let $\rho^2=x^2+y^2+z^2$. Then 
		\[\begin{array}{rcl}
			\der\omega&=&\der({x}{\rho	^{-3}}\dy\wedge\dz)-\der({y}{\rho	^{-3}}\dx\wedge\dz)+\der({z}{\rho	^{-3}}\dx\wedge\dy)\\
\\			&=&\phantom{+}(-2x^2+z^2+y^2)\rho^{-5} \,\dx\wedge\dy\wedge\dz\\
			&&{+}(+2y^2-x^2-z^2)\rho^{-5} \,\dy\wedge\dx\wedge\dz\\
			&&{+}(-2z^2+y^2+x^2)\rho^{-5} \,\dz\wedge\dx\wedge\dy\\
			\\
			&=&\phantom{+}(-2x^2+z^2+y^2)\rho^{-5} \,\dx\wedge\dy\wedge\dz\\
			&&{-}(+2y^2-x^2-z^2)\rho^{-5} \,\dx\wedge\dy\wedge\dz\\
			&&{+}(-2z^2+y^2+x^2)\rho^{-5} \,\dx\wedge\dy\wedge\dz\\
\\			&=&0
		\end{array}\]
	\end{proof}
	\begin{proof}
		(Not exact) Suppose for contradiction that $\omega$ is exact, and write $\omega=\der\eta$. Let $M$ be any compact manifold with $\del M=\emptyset$. Then since Stokes' Thm gives $\int_M\der\eta=\int_{\del M}\eta$, then 
		\[
		%\begin{array}{rcl}
			\int_M\omega=0.
		%\end{array}
		\]
		Now, $S^2$ is such a manifold, but we will show that $\int_{S^2}\omega\neq0$. Observe that 
		$$g(\theta, \phi)=(\cos\theta\sin\phi, \sin\theta\sin\phi, \cos\phi)$$ is a parameterization of $S^2$, and also that on $S^2$, $\omega$ is equivalent to $(x\dy\wedge\dz-y\dx\wedge\dz+z\dx\wedge\dy)$. So, 
		\[\begin{array}{rcl}
		  \int_{S^2}\omega &=& \int\limits_{\preimage{g}{S^2}}g^*\omega\\
		  &=& \int g^*(x\dy\wedge\dz-y\dx\wedge\dz+z\dx\wedge\dy)
		\end{array}\]
		To compute this, we first calculate
		\[\begin{array}{rcl}
			g^*{\dx}&=&-\sin\theta\sin\phi\der\theta+\cos\theta\cos\phi\der\phi\\
			g^*{\dy}&=&\phantom{-}\cos\theta\sin\phi\der\theta+\sin\theta\cos\phi\der\phi\\
			g^*{\dz}&=&-\sin\phi\der\phi			
		\end{array}	\]
		Thus, 
		\[\begin{array}{rcl}
			\int_{S^2}\omega &=& \int g^*(x\dy\wedge\dz-y\dx\wedge\dz+z\dx\wedge\dy)\\
			&=&\iint\big[-\cos^2\theta\sin^3\phi-\sin^2\theta\sin^3\phi+\cos\phi(-\cos^2\theta\sin\phi\cos\phi-\sin^2\theta\sin\phi\cos\phi)\big]\der\theta\der\phi\\
			&&\text{(Pythagorean identity 3 times)}\phantom{\int\
			limits^{2}_{2}}\\
			&=&\int\limits_0^{2\pi}\der\theta\int\limits_0^{\pi}-\sin\phi\der\phi\\
			&=&4\pi
		\end{array}\]
		This contradicts our assumption that $\omega$ is exact, so we are done. 
	\end{proof}
	
	\item Show that Stokes' Theorem is false if $M$ is not compact. 
	\begin{proof}
		Let $M=\R^2$ and $\omega=x\dy$, so $\del M=\emptyset$ and $\der\omega=\dx\wedge\dy$. Then Stokes' Theorem should say that 
		$$\int\limits_{\R^2}\dx\dy=\int\limits_\emptyset x\dy,$$
		but $\int\limits_{\R^2}\dx\dy=\infty$ (that is, the integral diverges) and $\int\limits_\emptyset x\dy=0$.
	\end{proof}
	
	\item Let $M$ be a compact $k$-manifold without boundary. Show that $\int_M\der\omega=0$ for all $\omega \in \Omega^{k-1}(M)$. Give a counterexample if $M$ is not compact. 
	\begin{proof}
		Since $M$ is compact, it can be parameterized as a $k$-manifold with boundary. To see this, let $\{g_\alpha\}_{\alpha\in\Gamma}$ be a parameterization of $M$. Since $M$ is compact, there is a finite subcollection $\{g_i\}_{i\in1,\dots, N}$ which parameterizes $M$. Thus, there is a least element in the set $\{\inf \{x_k : (x_1, \dots, x_k)\in U_i\}: \forall i\}$ where each $g_i:U_i\to M$. Call this number $\beta$. Then compose each $g_i$ with the translation $\tau_\beta(x_1, \dots, x_k)=(x_1, \dots, x_k+\abs{\beta})$. Now we have a parameterization where all $U_i\subseteq H^k$, so $M$ is a manifold with boundary. 
		
		Thus $M$ and $\omega$ satisfy all the criteria for Stokes' Theorem, so 
		$$\int_M\der\omega=\int_{\del M}\omega=\int_\emptyset\omega=0.$$
		See problem 4 for the requested counterexample. 
	\end{proof}
	
	\item Suppose that $C$ is a compact 2-dimensional manifold with boundary in $\R^2$, and assume $(0,0)\not\in\del C$. Let $\omega=\frac{-y}{x^2+y^2}\dx+\frac{x}{x^2+y^2}\dy$. Prove that
	\[\int_{\del C}\omega = 
	\begin{cases}
		0 & \text{if } (0,0)\in C,\\ 
		2\pi & \text{if } (0,0)\not\in C,\\
	\end{cases}\]
	
\end{enumerate}
\end{document}
