\documentclass[letterpaper]{article}
%\documentclass[a5paper]{article}

%% Language and font encodings
\usepackage[english]{babel}
\usepackage[utf8x]{inputenc}
\usepackage[T1]{fontenc}


%% Sets page size and margins
\usepackage[letterpaper,top=.75in,bottom=1in,left=1in,right=1in,marginparwidth=1.75cm]{geometry}
%\usepackage[a5paper,top=1cm,bottom=1cm,left=1cm,right=1.5cm,marginparwidth=1.75cm]{geometry}

\usepackage{graphicx}
%\graphicspath{../images}	  %%where to look for images

%% Useful packages
\usepackage{amssymb, amsmath, amsthm} 
%\usepackage{graphicx}  %%this is currently enabled in the default document, so it is commented out here. 
\usepackage{calrsfs}
\usepackage{braket}
\usepackage{mathtools}
\usepackage{lipsum}
\usepackage{tikz}
\usetikzlibrary{cd}
\usepackage{verbatim}
%\usepackage{ntheorem}% for theorem-like environments
\usepackage{mdframed}%can make highlighted boxes of text
%Use case: https://tex.stackexchange.com/questions/46828/how-to-highlight-important-parts-with-a-gray-background
\usepackage{wrapfig}
\usepackage{centernot}
\usepackage{subcaption}%\begin{subfigure}{0.5\textwidth}
\usepackage{pgfplots}
\pgfplotsset{compat=1.13}
\usepackage[colorinlistoftodos]{todonotes}
\usepackage[colorlinks=true, allcolors=blue]{hyperref}
\usepackage{xfrac}					%to make slanted fractions \sfrac{numerator}{denominator}
\usepackage{enumitem}            
    %syntax: \begin{enumerate}[label=(\alph*)]
    %possible arguments: f \alph*, \Alph*, \arabic*, \roman* and \Roman*
\usetikzlibrary{arrows,shapes.geometric,fit}

\DeclareMathAlphabet{\pazocal}{OMS}{zplm}{m}{n}
%% Use \pazocal{letter} to typeset a letter in the other kind 
%%  of math calligraphic font. 

%% This puts the QED block at the end of each proof, the way I like it. 
\renewenvironment{proof}{{\bfseries Proof}}{\qed}
\makeatletter
\renewenvironment{proof}[1][\bfseries \proofname]{\par
  \pushQED{\qed}%
  \normalfont \topsep6\p@\@plus6\p@\relax
  \trivlist
  %\itemindent\normalparindent
  \item[\hskip\labelsep
        \scshape
    #1\@addpunct{}]\ignorespaces
}{%
  \popQED\endtrivlist\@endpefalse
}
\makeatother

%% This adds a \rewnewtheorem command, which enables me to override the settings for theorems contained in this document.
\makeatletter
\def\renewtheorem#1{%
  \expandafter\let\csname#1\endcsname\relax
  \expandafter\let\csname c@#1\endcsname\relax
  \gdef\renewtheorem@envname{#1}
  \renewtheorem@secpar
}
\def\renewtheorem@secpar{\@ifnextchar[{\renewtheorem@numberedlike}{\renewtheorem@nonumberedlike}}
\def\renewtheorem@numberedlike[#1]#2{\newtheorem{\renewtheorem@envname}[#1]{#2}}
\def\renewtheorem@nonumberedlike#1{  
\def\renewtheorem@caption{#1}
\edef\renewtheorem@nowithin{\noexpand\newtheorem{\renewtheorem@envname}{\renewtheorem@caption}}
\renewtheorem@thirdpar
}
\def\renewtheorem@thirdpar{\@ifnextchar[{\renewtheorem@within}{\renewtheorem@nowithin}}
\def\renewtheorem@within[#1]{\renewtheorem@nowithin[#1]}
\makeatother

%% This makes theorems and definitions with names show up in bold, the way I like it. 
\makeatletter
\def\th@plain{%
  \thm@notefont{}% same as heading font
  \itshape % body font
}
\def\th@definition{%
  \thm@notefont{}% same as heading font
  \normalfont % body font
}
\makeatother

%===============================================
%==============Shortcut Commands================
%===============================================
\newcommand{\ds}{\displaystyle}
\newcommand{\B}{\mathcal{B}}
\newcommand{\C}{\mathbb{C}}
\newcommand{\F}{\mathbb{F}}
\newcommand{\N}{\mathbb{N}}
\newcommand{\R}{\mathbb{R}}
\newcommand{\Q}{\mathbb{Q}}
\newcommand{\T}{\mathcal{T}}
\newcommand{\Z}{\mathbb{Z}}
\renewcommand\qedsymbol{$\blacksquare$}
\newcommand{\qedwhite}{\hfill\ensuremath{\square}}
\newcommand*\conj[1]{\overline{#1}}
\newcommand*\closure[1]{\overline{#1}}
\newcommand*\mean[1]{\overline{#1}}
%\newcommand{\inner}[1]{\left< #1 \right>}
\newcommand{\inner}[2]{\left< #1, #2 \right>}
\newcommand{\powerset}[1]{\pazocal{P}(#1)}
%% Use \pazocal{letter} to typeset a letter in the other kind 
%%  of math calligraphic font. 
\newcommand{\cardinality}[1]{\left| #1 \right|}
\newcommand{\domain}[1]{\mathcal{D}(#1)}
\newcommand{\image}{\text{Im}}
\newcommand{\inv}[1]{#1^{-1}}
\newcommand{\preimage}[2]{#1^{-1}\left(#2\right)}
\newcommand{\script}[1]{\mathcal{#1}}


\newenvironment{highlight}{\begin{mdframed}[backgroundcolor=gray!20]}{\end{mdframed}}

\DeclarePairedDelimiter\ceil{\lceil}{\rceil}
\DeclarePairedDelimiter\floor{\lfloor}{\rfloor}

%===============================================
%===============My Tikz Commands================
%===============================================
\newcommand{\drawsquiggle}[1]{\draw[shift={(#1,0)}] (.005,.05) -- (-.005,.02) -- (.005,-.02) -- (-.005,-.05);}
\newcommand{\drawpoint}[2]{\draw[*-*] (#1,0.01) node[below, shift={(0,-.2)}] {#2};}
\newcommand{\drawopoint}[2]{\draw[o-o] (#1,0.01) node[below, shift={(0,-.2)}] {#2};}
\newcommand{\drawlpoint}[2]{\draw (#1,0.02) -- (#1,-0.02) node[below] {#2};}
\newcommand{\drawlbrack}[2]{\draw (#1+.01,0.02) --(#1,0.02) -- (#1,-0.02) -- (#1+.01,-0.02) node[below, shift={(-.01,0)}] {#2};}
\newcommand{\drawrbrack}[2]{\draw (#1-.01,0.02) --(#1,0.02) -- (#1,-0.02) -- (#1-.01,-0.02) node[below, shift={(+.01,0)}] {#2};}

%***********************************************
%**************Start of Document****************
%***********************************************

%===============================================
%===============Theorem Styles==================
%===============================================

%================Default Style==================
\theoremstyle{plain}% is the default. it sets the text in italic and adds extra space above and below the \newtheorems listed below it in the input. it is recommended for theorems, corollaries, lemmas, propositions, conjectures, criteria, and (possibly; depends on the subject area) algorithms.
\newtheorem{theorem}{Theorem}
\numberwithin{theorem}{section} %This sets the numbering system for theorems to number them down to the {argument} level. I have it set to number down to the {section} level right now.
\newtheorem*{theorem*}{Theorem} %Theorem with no numbering
\newtheorem{corollary}[theorem]{Corollary}
\newtheorem*{corollary*}{Corollary}
\newtheorem{conjecture}[theorem]{Conjecture}
\newtheorem{lemma}[theorem]{Lemma}
\newtheorem*{lemma*}{Lemma}
\newtheorem{proposition}[theorem]{Proposition}
\newtheorem*{proposition*}{Proposition}
\newtheorem{problemstatement}[theorem]{Problem Statement}


%==============Definition Style=================
\theoremstyle{definition}% adds extra space above and below, but sets the text in roman. it is recommended for definitions, conditions, problems, and examples; i've alse seen it used for exercises.
\newtheorem{definition}[theorem]{Definition}
\newtheorem*{definition*}{Definition}
\newtheorem{condition}[theorem]{Condition}
\newtheorem{problem}[theorem]{Problem}
\newtheorem{example}[theorem]{Example}
\newtheorem*{example*}{Example}
\newtheorem*{counterexample*}{Counterexample}
\newtheorem*{romantheorem*}{Theorem} %Theorem with no numbering
\newtheorem{exercise}{Exercise}
\numberwithin{exercise}{section}
\newtheorem{algorithm}[theorem]{Algorithm}

%================Remark Style===================
\theoremstyle{remark}% is set in roman, with no additional space above or below. it is recommended for remarks, notes, notation, claims, summaries, acknowledgments, cases, and conclusions.
\newtheorem{remark}[theorem]{Remark}
\newtheorem*{remark*}{Remark}
\newtheorem{notation}[theorem]{Notation}
\newtheorem*{notation*}{Notation}
%\newtheorem{claim}[theorem]{Claim}  %%use this if you ever want claims to be numbered
\newtheorem*{claim}{Claim}



\pgfplotsset{compat=1.13}

%\newcommand{\T}{\mathcal{T}}
%\newcommand{\B}{\mathcal{B}}

%These commands are now in tskpreamble_nothms.tex, but are left as a comment here for reference. 
%\newcommand{\arbcup}[1]{\bigcup\limits_{\alpha\in\Gamma}#1_\alpha}
%\newcommand{\arbcap}[1]{\bigcap\limits_{\alpha\in\Gamma}#1_\alpha}
%\newcommand{\arbcoll}[1]{\{#1_\alpha\}_{\alpha\in\Gamma}}
%\newcommand{\arbprod}[1]{\prod\limits_{\alpha\in\Gamma}#1_\alpha}
%\newcommand{\finitecoll}[1]{#1_1, \ldots, #1_n}
%\newcommand{\finitefuncts}[2]{#1(#2_1), \ldots, #1(#2_n)}
%\newcommand{\abs}[1]{\left|#1\right|}
%\newcommand{\norm}[1]{\left|\left|#1\right|\right|}

\title{Math 550 \linebreak
Homework 7}
\author{Trevor Klar}

\begin{document}

\maketitle

\begin{enumerate}
	\item Let $M$ be a $k$-dimensional manifold in $\R^Nn$. Prove that if there exists a nowhere zero $k$-form on $M$, then $M$ is orientable.
		\begin{proof}
			Since $M$ is a manifold, then for every point $x\in M$ there exists a parameterization \mbox{$\varphi_x:U_x\to V_x$} with $x\in V_x$. Thus the collection $\left\{\varphi_x\right\}_{x\in M}$ parametrizes all of $M$. Let 
			\[\begin{array}{rl}
				G=&\{\varphi_x \,|\, {\varphi_x}^*\omega(\preimage{\varphi_x}{x})\left(e_1, \dots, e_k\right) > 0\},\\
			B=&\{\varphi_x \,|\, {\varphi_x}^*\omega(\preimage{\varphi_x}{x})\left(e_1, \dots, e_k\right) < 0\}, \text{ and }\\
			\tau_{12}:&\R^k\to\R^k \text{ defined by } \tau_{12}(x_1, x_2, x_3, \dots, x_k)=(x_2, x_1, x_3, \dots, x_k).
			\end{array}\]
		Consider a point $p_g\in M$ such that for some $x_1, x_2 \in M$, we have $p_g\in V_{x_1}\cap V_{x_2}$ and $\varphi_{x_1}, \varphi_{x_2}\in G$. Then by a previous homework problem, $\varphi_{x_1}$ and $\varphi_{x_2}$ induce the same orientation on $M_{p_g}$ for every $p_g\in V_{x_1}\cap V_{x_2}$. Thus all the parameterizations in $G$ have compatible orientations. The same argument shows that the parameterizations in $B$ are also compatible. Now, define 
		\[\psi_x=\begin{cases}
			\varphi_x & \text{ if } \varphi_x\in G\\
			\tau_{12}\circ\varphi_x & \text{ if } \varphi_x\in B.\\
		\end{cases}\]
		Thus, the parameterizations $\{\psi_x\}_{x\in M}$ are all compatible. To see this, observe that for any $\varphi_x\in B$, 
		\[\begin{array}{rcl}
			{\psi_x}^*\omega(e_1, \dots, e_k) &=& (\tau_{12}\circ\varphi_x)^*\omega(e_1, e_2, \dots, e_k)\\
			&=& {\varphi_x}^*{\tau_{12}}^*\omega(e_1, e_2, \dots, e_k)\\
			&=& {\varphi_x}^*\omega(e_2, e_1, \dots, e_k)\\
			&=& -{\varphi_x}^*\omega(e_1, e_2, \dots, e_k)\\
			&>&0\\
		\end{array}\]
		Thus, we have produced a collection of parameterizations $\{\psi_x\}_{x\in M}$ covering $M$ for which all $\psi_{x_1}, \psi_{x_2}$ induce the same orientation on $M_x$ whenever $x\in V_{x_1}\cup V_{x_2}$, so $M$ is orientable. 
		\end{proof}		
	\item There is a general correspondence between $k$-forms and $(n-k)$-forms on $\R^n$, for all $1\leq k\leq n$. Given $\omega	\in \Omega^k(\R^n)$, we define $\star\omega\in\Omega^{n-k}(\R^n)$ using the rule 
	$$\star(dx_{i_1}\wedge\dots\wedge dx_{i_k})=\pm dx_{j_1}\wedge\dots\wedge\dx_{j_{n-k}},$$
	and extending linearly, where $i_1<\dots<i_k, j_1<\dots<j_{n-k}$, and $\{i_1,\dots,i_k, j_1,\dots,j_{n-k}\}=\{1, \dots, n\}$. The sign is chosen so that $\omega\wedge\star\omega = \dx_1\wedge\dots\wedge\dx_n$. \\
	\textbf{Notation} Another way to notate $\star$ is: Let $\Gamma\subset\{1, \dots n\}$, and define $\star$ by 
	\[\star\left(\bigwedge_{i\in\Gamma}\dx_i\right)=\pm\bigwedge_{j\in\Gamma^\complement}\dx_j\]
	and extending linearly. The sign is chosen so that $\omega\wedge\star\omega=\bigwedge_1^n\dx_i$. \\
	\textbf{Example} In $\R^5$, $\star(\dx_1\wedge\dx_4)=\dx_2\wedge\dx_3\wedge\dx_5$ and $\star(\dx_1\wedge\dx_3)=-\dx_2\wedge\dx_4\wedge\dx_5$. \\
	\textbf{Prove} that $\star\star\omega=(-1)^{k(n-k)}\omega$. 
	\begin{proof}
		By definition of $\star$, we know that $\omega\wedge\star\omega=\bigwedge_1^n\dx_i$ and $\star\omega\wedge\star\star\omega=\bigwedge_1^n\dx_i$. So, by properties of wedge products, we can commute and write 
		\[\begin{array}{rcl}
			(-1)^{k(n-k)}\star\star\omega\wedge\star\omega&=&\bigwedge_1^n\dx_i\\
			\omega\wedge\star\omega&=&\bigwedge_1^n\dx_i\\
		\end{array}\]
		Now since $\star\star\omega$ and $\omega$ differ by at most a sign\footnote{Since $\star\star\omega$ is a wedge indexed over $\Gamma^{\complement\complement}=\Gamma$, and all that remains is to determine the sign.}, then $\sign(\star\star\omega)=(-1)^{k(n-k)}$. Thus, $\star\star\omega=(-1)^{k(n-k)}\omega$ as desired. 
	\end{proof}
	
\end{enumerate}
\end{document}
